\documentclass[12pt]{article}

% Essential packages
\usepackage[utf8]{inputenc}
\usepackage[T1]{fontenc}
\usepackage{amsmath,amssymb,amsthm}
\usepackage{mathrsfs}
\usepackage{geometry}
\usepackage{hyperref}
\usepackage{algorithm}
\usepackage{algorithmic}
\usepackage{tabularx}
\usepackage{booktabs}

% Overfull hbox prevention
\usepackage{enumitem}
\setlist[itemize]{nosep,leftmargin=1.6em}
\setlist[enumerate]{nosep,leftmargin=1.6em}
\emergencystretch=1em
\raggedbottom
\usepackage{natbib}
\usepackage{multirow}
\usepackage{xcolor}
\usepackage{bm}

% Geometry settings
\geometry{a4paper, margin=1in}

% Hyperref settings
\hypersetup{
    colorlinks=true,
    linkcolor=blue,
    citecolor=blue,
    urlcolor=blue
}

% Theorem environments
\theoremstyle{plain}
\newtheorem{theorem}{Theorem}[section]
\newtheorem{lemma}[theorem]{Lemma}
\newtheorem{proposition}[theorem]{Proposition}
\newtheorem{corollary}[theorem]{Corollary}

\theoremstyle{definition}
\newtheorem{definition}[theorem]{Definition}
\newtheorem{example}[theorem]{Example}
\newtheorem{remark}[theorem]{Remark}
\newtheorem{postulate}{Postulate}

% Title information
\title{Unified Theory of Closed-World Static-Block Computation \\
(with Turing-Machine Dual Terminology)}
\author{Haobo Ma\thanks{Independent Researcher} \and Wenlin Zhang\thanks{National University of Singapore}}
\date{\today}

\begin{document}

\maketitle

\begin{abstract}
This paper presents a unified theory of static-block cellular automata (SB-CA) in closed universes: the universe contains only local evolution rules and their induced spatiotemporal consistency constraints, with no external input. Leveraging the equivalence between SB-CA and Turing machines (TM), this theory employs a dual terminology system (annotating TM duals in parentheses after SB-CA concepts) to systematically characterize:

\begin{enumerate}
\item \textbf{Computation=Structure}: Time is an index of the static body; "execution" is the existence of legal static blocks;
\item \textbf{Program Emergence}: In closed universes, program boxes (program+input) emerge as local patterns and can be globally extended;
\item \textbf{Observation=Decoding}: Output windows (tape-with-output) are interpreted by local decoders;
\item \textbf{Decidability Hierarchy}: Under fixed rules, single "occurrence" is $\Sigma_1^0$-complete, infinite recurrence is $\Pi_2^0$-complete;
\item \textbf{Information Conservation}: Conditional Kolmogorov complexity of any slice/window is bounded by "rule+coordinates+causal boundary/initial slice";
\item \textbf{Forced vs Typical}: In self-similar SFT classes, forced carrying can be achieved through macroblocks; under internal measures, short program boxes are more typical (mechanism-induced).
\end{enumerate}

The theory provides categorical semantics, construction paradigms, complete proof examples, related work citations, and open problems, forming a verifiable and extensible unified framework. All undecidability/complexity conclusions in this paper are stated under the semantics of "fixed rules, quantifying over legal configuration families", avoiding confusion with decidable cases for "fixed specific configurations".

\textbf{Keywords}: Static-Block Cellular Automata, Closed Universe, Turing Completeness, Program Emergence, Kolmogorov Complexity, Algorithmic Prior
\end{abstract}

\section{Program and Intuition}

\subsection{Technical Definition of Closed Universe}

\textbf{Closed Universe}: Universe = the set of all legal static blocks satisfying local constraints (SB-CA / computation history). The technical significance of closed universes lies in ensuring all structures are induced by intrinsic constraints, avoiding non-determinism introduced by external initial states, and reducing computational emergence to SFT geometric facts.

\textbf{No External Input}: There exists no "externally set initial state/noise". All observable structures come from legal blocks themselves.

\subsection{Dual Terminology Correspondence Table}

This theory adopts a \textbf{SB-CA / TM dual terminology} system:

\begin{table}[h]
\centering
\caption{SB-CA and TM dual terminology correspondence}
\renewcommand{\arraystretch}{1.1}
\begin{tabularx}{\textwidth}{@{}>{\RaggedRight\arraybackslash}p{0.25\textwidth}
                               >{\RaggedRight\arraybackslash}X
                               >{\RaggedRight\arraybackslash}X@{}}
\toprule
\textbf{SB-CA Term} & \textbf{TM Dual Term} & \textbf{Description} \\
\midrule
Legal Static Block & Computation History & Spatiotemporal configuration satisfying constraints \\
Initial Slice & Program + Input & State configuration at $t=0$ \\
Output Window & Tape with Output & Spatiotemporal region for reading results \\
Halting Witness & Halting Evidence & Pattern marking computation termination \\
Program Box & Code + Data Local Encapsulation & Computation unit in finite region \\
Moat & Control Tape & Boundary buffer for isolation \\
Sync Layer & Clock & Phase coordination mechanism \\
Self-Similar Macroblock & Hierarchical Simulator & Recursive checking structure \\
Forced Emergence & Writing Programs into Transition Function & Rule-embedded computation \\
Typical Emergence & Sparse Universal Occurrence / Under Internal Measure & Probability-induced occurrence \\
\bottomrule
\end{tabularx}
\end{table}

\subsection{Paper Structure}

This paper is organized as follows:

\begin{itemize}
\item \textbf{\S 2} Postulates and Primitives
\item \textbf{\S 3} Formal Model
\item \textbf{\S 4} Main Theorems and Proof Outlines
\item \textbf{\S 5} Construction Paradigms
\item \textbf{\S 6} Observation Semantics and "Semantic Collapse"
\item \textbf{\S 7} Categorical Semantics
\item \textbf{\S 8} Examples and Templates
\item \textbf{\S 9} Related Work
\item \textbf{\S 10} Conclusion and Outlook
\item \textbf{Appendix A} Formalized Moat Definition and block-gluing Verification
\item \textbf{Appendix B} Brudno Numerical Verification Script
\item \textbf{Appendix C} Open Problems
\end{itemize}

\section{Postulates and Primitives}

\begin{postulate}[Closed World]
Fix finite state set $Q$, neighborhood radius $r$, local evolution $f$. The universe is the configuration set $(\mathcal{B} \subseteq Q^{\Lambda \times T})$ satisfying spatiotemporal local consistency constraints induced by $f$, where $\Lambda = \mathbb{Z}^d$, $T = \mathbb{N}$. No external input.
\end{postulate}

\begin{postulate}[Causal Encoding]
Constraints ensure any adjacent slices $(S_t, S_{t+1})$ satisfy $S_{t+1}(x) = f(S_t \upharpoonright \mathcal{N}_r(x))$ at each $x$ as equivalent static conditions. Time is merely an index.
\end{postulate}

\begin{postulate}[Decoder=Sliding Block Code, CHL]
Decoder $\pi$ is defined as a continuous local map commuting with shifts (sliding block code/factor map, conforming to Curtis--Hedlund--Lyndon theorem, abbr. CHL; Hedlund 1969\cite{hedlund1969}).
\end{postulate}

\begin{postulate}[Universal Substrate]
There exist computable encoding/decoding $(\iota, \pi)$ such that any TM computation can be embedded into legal blocks and read out in finite windows (polynomial overhead).
\end{postulate}

\begin{postulate}[Compactness-Extension Principle, with Conditions]
The legal block space is a subshift of finite type (SFT), hence a compact closed set. For any nested, compatible finite pattern family $\{P_k\}$ (each $P_k$ in SFT language, and $P_k \subset P_{k+1}$), there exists a limit configuration $\mathcal{B}$ such that $\mathcal{B} \upharpoonright \mathrm{supp}(P_k) = P_k$.
\end{postulate}

\begin{remark}[block-gluing and Finite Extension Property]
If the system has block-gluing/specification or finite extension property (FEP), any two distant finite blocks can be glued into a global extension. Typical verifiable conditions include c-block-gluing (gluing with uniform gap $c$), and its quantified versions. This paper defaults to $\ell_\infty$ distance metric "block distance" on $\mathbb{Z}^{d+1}$ for c-block-gluing gap definition.
\end{remark}

\begin{remark}[Compatibility Conditions for Extension]
Extension of A4 requires standard "compatible family" conditions, and explicitly guarantees compatibility and conflict-freeness under FEP and other assumptions. This paper adopts block-gluing in construction to ensure gluability.
\end{remark}

\begin{proposition}[Quantified Gluing]
Suppose the system has block-gluing/specification property, which this paper calls \emph{quantified block-gluing}, degrading to \emph{almost-specification} (failure probability exponentially decaying) when lacking safe symbol layers. Then there exists constant $c = c(m, r, Q)$ (where $m$ is moat width, $r$ is neighborhood radius, $Q$ is state set) such that when two blocks $P_1, P_2$ have $\ell_\infty$ support distance $\ge c$, they can be glued into global legal configurations via moat mediation. Specific upper bound $c \le 2r + m$, where moat guarantees cross-boundary defects annihilated within $r$ steps.
\end{proposition}

\begin{proof}
Causal envelope disjointness (Appendix A.2) ensures local consistency decomposed into independent checks on both sides; fixed absorbing states (e.g., all-0 pattern) in moat achieve error correction and fault tolerance. If lacking complete block-gluing, degrades to almost-specification (failure probability exponentially decaying).

\textbf{Absorption Condition Clarification}: If the system lacks global absorbing sub-alphabet/safe symbol layer, only almost-specification quantified gluing (failure probability exponentially decaying) is obtained, and c-block-gluing of this proposition requires corresponding weakening.
\end{proof}

\begin{postulate}[Causal Conditional Complexity Upper Bound]
\textbf{Notation Declaration}: Below $K(\cdot)$ denotes \textbf{prefix Kolmogorov complexity}; different universal machines differ only by $O(1)$. We adopt the following backward lightcone closure definition for causal past boundary (Eq. (2.5a)), and give information upper bounds in Eqs. (2.5b)(2.5c):
\[
\partial^\star(W) := \bigl\{ (y,u) \in \Lambda \times T \mid \exists (x,s) \in W,\ u \le s,\ |y-x|_{\ell_\infty} \le r(s-u) \bigr\} \tag{2.5a}
\]

For any legal block $\mathcal{B}$,
\[
K(\mathcal{B} \upharpoonright W) \le K(f) + K(\mathrm{coord}(W)) + K(\mathcal{B} \upharpoonright \partial^\star(W)) + O(1) \tag{2.5b}
\]

If some slice is designated as "initial slice (program+input)", then
\[
K(S_t) \le K(S_0) + K(f) + K(t) + O(1) \tag{2.5c}
\]
\end{postulate}

\begin{remark}[Interpretation of Information Conservation]
The upper bound characterizes: information is not created from nothing, but must account for causal boundary or initial slice. Constant terms depend on interpreter choice.
\end{remark}

\begin{postulate}[Decoding Invariance, Topological Conjugacy Version]
If $\pi_2 = \tau \circ \pi_1$, where $\tau$ is a bijective sliding block code (shift-commuting local invertible map), then for any semantic property $\mathsf{P}$ (defined by local predicates on finite windows), the determination of "whether containing certain semantic pattern" is equivalent under $\pi_1, \pi_2$. (I.e.: semantic invariance under conjugacy/factor category.)
\end{postulate}

\begin{postulate}[Internal Measure]
Consider translation-invariant, ergodic internal measure $\mu$ on legal block space (such as existing maximal entropy measure).
\end{postulate}

\begin{remark}[Measure Uniqueness]
On 1D, primitive (aperiodic irreducible) SFT, maximal entropy measure is unique (Parry measure); in $d \ge 2$ SFT, maximal entropy measure may be non-unique (Burton--Steif 1994 counterexample\cite{burton1994}). See standard exposition by Lind--Marcus\cite{lind1995}.
\end{remark}

\begin{postulate}[Self-Similar Forcing]
In self-similar/hierarchical SFT classes, one can construct macroblock self-similar structures forcing every legal block to carry certain computational layers (or countable families) at all scales.
\end{postulate}

\subsection{Notation and Conventions}

\begin{enumerate}
\item After history-freezing, all objects are discussed on $\mathbb{Z}^{d+1}$; horizontal direction is $\Lambda=\mathbb{Z}^d$, vertical direction is time dimension $T=\mathbb{N}$, "history height" refers to TM step count $T_{\mathrm{TM}}$.
\item "Legal static block/legal configuration" is uniformly denoted $\mathcal{B}\in Q^{\Lambda\times T}$.
\item Uniformly adopt $\ell_\infty$ distance; "window" refers to finite support set $W\subset \Lambda\times T$.
\item CHL (Curtis--Hedlund--Lyndon) is synonymous with "sliding block code=shift-commuting continuous local map".
\end{enumerate}

\section{Formal Model}

\begin{definition}[Static-Block Cellular Automaton SB-CA]
Given $\Lambda = \mathbb{Z}^d$, $T = \mathbb{N}$, finite $Q$ and local constraint family $C$, legal block $\mathcal{B} \in Q^{\Lambda \times T}$ satisfies all $C$. If $C$ is induced by single-step $f$ of some CA, call this SB-CA generated by $(Q, \mathcal{N}_r, f)$. After history-freezing, spacetime SFT resides in $\mathbb{Z}^{d+1}$.
\end{definition}

\begin{remark}[Relationship with High-Dimensional SFT]
This is consistent with the result that "1D effective subshifts can be realized as factors (projections) of higher-dimensional SFTs" (Durand--Romashchenko--Shen\cite{durand2012}; Aubrun--Sablik\cite{aubrun2013}).
\end{remark}

\begin{definition}[Program Box / Program+Input]
Finite spatiotemporal region $W$ and its pattern $P$ such that $\pi(P) = \langle M, w, \mathrm{phase} \rangle$, achieving interface matching and phase alignment with exterior via moat/sync layer.
\end{definition}

\begin{definition}[Output Window / Tape-with-Output]
Bit string read by $\pi$ on finite window $W'$ (with "halt/acc" label attached). I.e., bit string of $\pi(\mathcal{B}\upharpoonright W')$ and its halt/acc label.
\end{definition}

\begin{definition}[Forced Carrying / Typical Occurrence]
\textbf{Forced carrying} means "all legal blocks" exhibit some computational layer; \textbf{typical occurrence} means sparse distribution with positive density $\liminf > 0$ or $\mu$-measure 1 almost surely under measure $\mu$.
\end{definition}

\begin{definition}[Observer]
An observer is a tuple $(\mathcal{W}, \pi)$, where $\mathcal{W}$ is a window family, $\pi$ is a decoder. One observation is equivalent to applying "window+decoding".
\end{definition}

\section{Main Theorems and Proof Outlines}

\begin{theorem}[Existence of Closed Emergence; SB-CA / TM]
For any TM program $p = (M, w)$ and any finite duration $T$, there exist legal block $\mathcal{B}$ and its subregion $W$, such that:

\begin{enumerate}
\item $W$ is a \textbf{program box (program+input)} decoded by $\pi$ as $p$;
\item $W$'s boundary satisfies moat/sync layer interface specifications;
\item $\mathcal{B}$'s slices $\{S_t\}$ are consistent with $M$'s $T$-step computation, and at some time, \textbf{halting witness (halting evidence)} and readable output appear in $W'$ (if $M$ halts within $T$ steps).
\end{enumerate}
\end{theorem}

\begin{proof}
Adopt universal substrate from A3, encoding $M$'s stepping as vertical pairing constraints: for Rule 110 (universal 1D CA), embed TM as multi-track simulation (track 1: tape; track 2: head state; track 3: phase). Construct "sync layer" with multi-track/phase field: phase alphabet $\{0,1,\dots,k-1\}$, incrementing mod $k$ each step, ensuring remote dependencies localized.

Add "moat" around $W$: isolation band of width $O(r)$, using fixed pattern (e.g., all-0) to absorb phase conflicts and defects. By A4's block-gluing (assuming fixed gap, specific moat construction and $c$ verification see Appendix A), expand to global legal block: starting from finite $P_0 = W \cup$ moat, nested expansion $P_k$ covering radius $k$, limit yields $\mathcal{B}$.

\textbf{Overhead}: The embedding has polynomial time and space blowup $\mathrm{poly}(|w|,T)$. Evidence from Rule 110's universality construction\cite{cook2004} and its polynomial-time simulation enhancement\cite{neary2006} (Cook, 2004; Neary \& Woods, 2006).
\end{proof}

\begin{lemma}[Compilation Overhead Bound]
Compilation path from TM to SB-CA: TM $\to$ multi-track 1D-CA embedding $\to$ 2D-SFT history-freezing. Overhead per stage:

\begin{enumerate}
\item TM $\to$ 1D-CA: time $T_{\text{CA}} = \mathrm{poly}(T_{\text{TM}}, |w|)$, space $S_{\text{CA}} = \mathrm{poly}(S_{\text{TM}}, |w|)$ (Cook 2004, Neary--Woods 2006).
\item 1D-CA $\to$ 2D-SFT: additional space factor $\mathrm{poly}(k, r)$ (phase period $k$, neighborhood $r$), time unchanged.
\end{enumerate}

Total overhead: $T_{\text{SFT}} = \mathrm{poly}(T_{\text{TM}}, |w|, k, r)$, $S_{\text{SFT}} = \mathrm{poly}(S_{\text{TM}}, |w|, k, r)$. Moat thickness $m$ contributes additional constant factor.

\textbf{Constant Dependencies}:
\[
|Q_{\text{SFT}}| = |Q_{\text{CA}}| \cdot k \cdot O(1),\quad \textbf{history height} = T_{\text{TM}},\quad \text{moat overhead} = O(m)
\]
\end{lemma}

\begin{proof}
Multi-track embedding uses track-separated tape and head states, Rule 110's track count fixed as constant. History-freezing encodes evolution through vertical constraints, space overhead from encoding alphabet expansion.
\end{proof}

\begin{theorem}[Decidability Hierarchy of Occurrence/Recurrence under Fixed Rules; SB-CA / TM]
Fix local rules $(Q, r, f)$, quantifying over the set of all legal static blocks $\mathcal{B}$:

\begin{itemize}
\item \textbf{Existence Occurrence} ($\Sigma_1^0$-complete): Decide whether there exist legal block $\mathcal{B}$ and coordinates $(x,t)$ such that finite pattern $P$ occurs at $\mathcal{B}$'s $(x,t)$. I.e., $\exists \mathcal{B} \exists (x,t): \mathcal{B}(x,t) = P$.
\item \textbf{Infinite Recurrence} ($\Pi_2^0$-complete): Decide whether there exist legal block $\mathcal{B}$ and time $N$ such that for all $t \ge N$, there exists position $x$ where $P$ occurs at $\mathcal{B}$'s $(x,t)$. I.e., $\exists \mathcal{B} \exists N \forall t \ge N \exists x: \mathcal{B}(x,t) = P$. If further quantifying "whether there exists pattern $P$ making this hold", then $\Sigma_3^0$.
\end{itemize}
\end{theorem}

\begin{proof}[Proof Sketch]
Existence occurrence via Rice theorem embedding of TM halting problem: construct $P$ encoding "TM halts then occurs, otherwise not". Infinite recurrence via Kari--Rice extension\cite{kari1994} on limit sets, embedding global properties like "TM always halts but computes infinitely".

\textbf{Proof Annotation ($\Pi_2^0$ Completeness---Formal Reduction Summary)}: Given any $\Pi_2^0$ predicate $\forall n \exists m \ge n: R(m)$, construct phase-gated observation witness $P$ and fixed rules $(Q,r,f)$, making "$\exists \mathcal{B} \exists N \forall t \ge N \exists x: P@(\mathcal{B};x,t)$" $\Leftrightarrow$ original formula. Upward reduction closure and mapping computability yield $\Pi_2^0$-completeness.
\end{proof}

\begin{theorem}[Information Conservation and Complexity Upper Bound; SB-CA / TM]
For any legal block and window,
\[
K(\mathcal{B} \upharpoonright W) \le K(f) + K(\mathrm{coord}(W)) + K(\mathcal{B} \upharpoonright \partial^\star(W)) + O(1)
\]
\[
K(S_t) \le K(S_0) + K(f) + K(t) + O(1)
\]

\textbf{Implication}: Evolution does not proliferate algorithmic information from nothing; observable information originates from "rule+coordinates+causal boundary/some slice".

\textbf{Brudno Alignment}: Under any translation-invariant ergodic measure $\mu$, almost surely
\[
\lim_{n\to\infty}\frac{1}{|W_n|}K(\mathcal{B} \upharpoonright W_n)=h_\mu
\]

Thus empirical bits/cell based on model-free compression (such as LZ-77/PPMd) can serve as numerical approximation of $\mu$-entropy. See Brudno (1983)\cite{brudno1983}'s foundational result and subsequent surveys/quantum generalizations.

\textbf{Experimental Considerations}: In numerical verification, window shape should choose cube $n \times n \times \Delta t$ to match spatiotemporal scales; boundary handling adopts periodic or fixed padding to avoid edge effects; compressor choice LZ-77/PPMd leads to finite sample bias (e.g., dictionary size limits), error bars include window selection variation and phase field perturbations. Reproducible parameter table: window sequence $n = 32, 64, 128, 256, 512$; sampling step $\Delta t = n/4$; average 100 independent runs.
\end{theorem}

\begin{remark}[Multidimensional Case]
On ergodic subshifts of $\mathbb{Z}^d$, there is also Brudno-type equivalence, supporting this paper's numerical verification protocol.
\end{remark}

\begin{proof}
From additivity of Kolmogorov complexity and finite representation of local rules.
\end{proof}

\begin{theorem}[Existence of Forced Emergence; SB-CA / TM]
In self-similar/hierarchical SFT classes, there exists macroblock self-similar construction $\mathfrak{M}$ forcing every legal block to carry designated computational layers (or countable families) at all scales. Its TM dual is "writing programs into transition function/auditor".
\end{theorem}

\begin{proof}
Adopt Mozes tiling's\cite{mozes1989} self-similar structure, embedding "auditor" at each macroblock scale to check lower-level consistency.
\end{proof}

\begin{theorem}[Typical Emergence and Algorithmic Prior; SB-CA / TM]
In closed universe, distinguish internal geometric measure $\mu$ (on SFT dynamical system, related to local constraints) from loading mechanism-induced external "encoding selection" distribution $\nu$ (prefix sampling of program boxes). If $\nu$ is equivalent to tossing fair bits to universal interpreter, and $\mu$ approximates maximal entropy, then occurrence frequencies of different program boxes approximately satisfy $\Pr(p) \propto 2^{-|p|}$ (times constant). Short programs more common, long programs sparse but almost surely appear infinitely many times in infinite domain (if their witnesses have finite duration).
\end{theorem}

\begin{remark}[Mechanism Transduction and Semimeasure Dependence]
This distribution corresponds to Solomonoff--Levin's\cite{solomonoff1964,levin1974} algorithmic prior/universal semimeasure (relative to chosen prefix universal interpreter), hence is \textbf{mechanism-induced rather than conventional SFT natural law}. This prior is a \textbf{semimeasure} and depends on chosen prefix universal machine, different machines differing only by multiplicative constant (\textbf{only up-to constant}). In closed context, $\nu$ can be viewed as external randomization mechanism generating program box patterns, congruent with $\mu$.

\textbf{Falsifiable Facet}: Perturbation of interpreter switching will only change normalization constant, not order relation of short-program priority. I.e., for any two prefix universal machines $U_1, U_2$, there exists constant $c$ such that for all programs $p$, $\Pr_{U_1}(p) \le c \cdot \Pr_{U_2}(p)$.

\textbf{Independence Assumption}: To assert "almost surely infinitely many occurrences", requires window sampling independence or finite energy conditions; lacking these conditions, can only guarantee positive measure/positive lower density level typicality.
\end{remark}

\begin{proof}
From combination of prefix encoding and maximal entropy measure.
\end{proof}

\begin{theorem}[Decoding Invariance; SB-CA / TM]
If two decoders differ by a bijective sliding block code, then judgments of "containing/not containing certain semantic pattern" are consistent. Hence semantics are canonical under sliding block code equivalence classes.
\end{theorem}

\begin{proof}
Directly from topological conjugacy property of Postulate A6.
\end{proof}

\section{Construction Paradigms}

\subsection{History-Freezing}

Use vertical pairing to encode "$t \to t+1$" as local consistency; e.g., in $\mathbb{Z}^{d+1}$ SFT, constrain each cell to match $f$-evolution with its "below".

\subsection{Sync Layer (Clock/Phase)}

Embed phase field in pattern (alphabet $\{0,\dots,k-1\}$), making remote dependencies rewritten as adjacent layer consistency; interface protocol: phase increments mod $k$, moat resets on conflicts.

\subsection{Moat}

Achieve stable interface with exterior via isolation band of width $O(r)$, using fixed pattern to absorb phase conflicts and local defects; defect absorption strategy: majority vote or error correction code. (State $Q \times \{0,1\}$, 1 for interior, 0 for exterior.)

\subsection{Fault-Tolerant Redundancy}

Multi-track majority vote/spatiotemporal repetition encoding, ensuring long-range stability; e.g., triple-track redundancy, each track independently simulates, output takes majority.

\subsection{Self-Similar Macroblock}

Scaled auditing (hierarchical checker) writes "whether carrying computation" into macroblock matching; adopt Mozes tiling, self-similar scale recursively checks lower-level consistency.

\subsection{Prefix Loading}

Adopt prefix-free boxing for "code+data", naturally achieving $2^{-|p|}$ sampling weight; mechanism: universal interpreter reads prefix code, independent of background measure.

\section{Observation Semantics and "Semantic Collapse"}

\begin{definition}[Observation Step]
Choose window $W$ and decoder $\pi$, obtain output string and state label (halt/acc/step).
\end{definition}

\begin{definition}[Observation Equivalence Class]
Define equivalence relation $\sim_\pi$: $\mathcal{B}_1 \sim_\pi \mathcal{B}_2$ if for all windows $W$, $\pi(\mathcal{B}_1 \upharpoonright W) = \pi(\mathcal{B}_2 \upharpoonright W)$. Equivalence class $[\mathcal{B}]_\pi$ is observation equivalence class.
\end{definition}

\begin{proposition}[Semantic Collapse]
Relative to given $\pi$, "observation" partitions the same underlying structure family into semantic equivalence classes; one concrete observation selects a representative of one equivalence class (readable interpretation). Semantic collapse is selection functor: choosing $\Pi(\mathcal{B}) \in [\mathcal{B}]_\pi$.
\end{proposition}

\begin{remark}[Distinction from Standard Factor Maps]
Distinguished from topological invariance of standard SFT factor maps, this semantic collapse emphasizes observer-relative selection framework, providing partition logic for semantic equivalence classes. \textbf{Important Clarification}: This "collapse" is a \textbf{mathematical/observational terminology}, meaning observation behavior selects a representative in equivalence class, \textbf{not a physical process}, only involving decoding choice, not altering underlying static blocks.
\end{remark}

\begin{corollary}[Essence of Observation]
In closed universe, observation doesn't "change" legal blocks, only selects readable structural path; "observation=decoding$\approx$semantic collapse".
\end{corollary}

\begin{proof}
Directly from Definition 6.1 and Proposition 6.1.
\end{proof}

\section{Categorical Semantics}

\subsection{Basic Categorical Structure}

\textbf{Objects}: Legal static blocks $\mathcal{B}$.

\textbf{Morphisms}: Local factor maps/sliding block codes (preserving local consistency).

\textbf{Observation Functor}: $\Pi: \mathsf{SFT} \to \mathsf{Str}$ (finite word category), $\Pi(\mathcal{B}) = \{\pi(\mathcal{B} \upharpoonright W)\}$. Observation functor realizes semantic collapse by selecting equivalence class $[\mathcal{B}]_\pi$.

\textbf{Decoding Natural Isomorphism}: If two decoders $\pi_1, \pi_2$ differ by bijective sliding block code $\tau$, there exists natural isomorphism $\eta: \Pi_{\pi_1} \to \Pi_{\pi_2}$ such that $\Pi_{\pi_2}(\mathcal{B}) = \tau \circ \Pi_{\pi_1}(\mathcal{B})$. This corresponds to natural isomorphism on fibers.

\textbf{Sliding Block Code Fibers and Grothendieck Transformation}: On observation layer, take sliding block code equivalence classes as fibers; decoding invariance allows fibration construction, with Grothendieck transformation characterizing fibers above different decoders.

\begin{remark}[Categorical Properties]
Category not specified as Cartesian closed, but has fibration structure.
\end{remark}

\section{Examples and Templates}

\subsection{Universal Substrate Pattern}

2D SFT / 1D CA history-freezing embedding (e.g., Rule 110 embedding TM adder: track simulates tape and head, moat width 3, following 5.1--5.6).

\subsection{Program Box Blueprint}

Four-layer structure: center---clock ring---moat---outer sea; outer sea can be any compliant background.

\subsection{Forced Carrying Case}

Macroblock auditor checks lower-level consistency and "computational traces" at each layer, adopting G\'{a}cs hierarchical redundancy\cite{gacs2001}.

\begin{proposition}[Moore-Myhill and Information Conservation]
On CA over amenable groups (such as $\mathbb{Z}^{d+1}$), if no Garden-of-Eden then global map is pre-injective. The Kolmogorov upper bound (A5) of "information not created from nothing" in this framework is consistent with this topological-combinatorial invariant: pre-injectivity ensures inverse image existence, complementing information conservation upper bound, but differing in that the bound focuses on algorithmic complexity rather than mere existence.

\textbf{Amenable Group Assumption}: In this paper's setting, $\mathbb{Z}^{d+1}$ automatically satisfies amenability, Moore-Myhill theorem\cite{moore1962,myhill1963} directly applies.

\textbf{Complementary Rather Than Implicative}: The relationship between pre-injectivity and A5 is \textbf{complementary rather than implicative}: pre-injectivity guarantees topological-level no information loss (global reversibility), A5 guarantees algorithmic-level information non-growth (complexity upper bound). Both jointly support closed universe's information conservation framework.
\end{proposition}

\begin{proof}
Moore (1962) and Myhill (1963) proved no Garden-of-Eden $\Leftrightarrow$ pre-injective; in information-theoretic context, pre-injectivity guarantees no information loss, consistency with Brudno upper bound from finiteness of local rules.
\end{proof}

\section{Related Work}

This theory builds on existing CA universality and SFT constructions:

\begin{itemize}
\item \textbf{Berger}\cite{berger1966}'s Domino Problem and \textbf{Robinson}\cite{robinson1971} tilings provide emergence foundation
\item \textbf{Mozes}\cite{mozes1989} self-similar tilings and \textbf{Durand--Romashchenko--Shen}\cite{durand2012}'s effective subshifts support forced macroblocks
\item \textbf{G\'{a}cs}\cite{gacs2001}'s hierarchical self-organizing simulation ensures fault tolerance
\item \textbf{Hedlund (1969)}\cite{hedlund1969} / Curtis--Hedlund--Lyndon theorem standardizes decoders
\item \textbf{Moore--Myhill}\cite{moore1962,myhill1963} theorem (Garden of Eden) connects information conservation with surjective/pre-injective, and its extensions on \textbf{amenable groups}
\item \textbf{Solomonoff/Levin}\cite{solomonoff1964,levin1974} prior guides typical emergence
\item \textbf{Cook 2004}\cite{cook2004} and \textbf{Neary--Woods 2006}\cite{neary2006} provide Rule 110 universality and polynomial simulation
\item \textbf{Burton--Steif 1994}\cite{burton1994} provide measure non-uniqueness counterexample
\item \textbf{Brudno 1983}\cite{brudno1983} founds complexity-entropy equivalence
\end{itemize}

\textbf{Distinction}: This framework emphasizes closed static perspective and dual terminology, avoiding dynamic initial states.

\section{Conclusion and Outlook}

\subsection{Main Contributions}

This theory, in the context of closed universes, establishes unified framework of "computation=structure, observation=decoding, time=index". We proved program emergence existence, two implementation paths of forced/typical, $\Sigma_1^0 / \Pi_2^0$ decidability hierarchy, and information conservation conditional complexity upper bound, characterizing decoding invariance and observation logic with categorical semantics.

\subsection{Theoretical Positioning}

This framework is both self-consistent and naturally couples with broader "collapse-aware" perspective: observation doesn't change universe, only changes readability; semantic choice is "collapse" choice. Thus, dynamic narrative of computation is reduced to geometric/combinatorial facts in static body, program "occurrence" becomes structural event in legal static block space.

\subsection{Future Directions}

\textbf{Short-term}: Moat overhead optimization, mixing threshold research, forced family expression boundary exploration.

\textbf{Long-term}: Measure uniqueness research, global recycling and reversibility, connection with quantum computing framework.

\appendix

\section{Formalized Moat Definition and block-gluing Verification}

\subsection{Metric and Causal Envelope}

Take $\ell_\infty$ distance on $\mathbb{Z}^{d+1}$. Given finite window $U \subset \Lambda \times T$, its causal past envelope $\mathrm{Past}_r(U)$ is defined as minimal closure backward-tracing $r$-neighborhood.

\subsection{Moat Disjointness Lemma}

Let moat width $m \ge 2r + 1$. If two blocks $P_1, P_2$ have $\ell_\infty$ support distance $\ge m$, then $\mathrm{Past}_r(P_1) \cap \mathrm{Past}_r(P_2) = \varnothing$.

\begin{proof}
Causal past envelope $\mathrm{Past}_r(U)$ defined as minimal set backward-tracing $r$ steps from $U$. Moat width $m$ ensures buffer between two blocks at least $m$ cells, after tracing $r$ steps still disjoint (distance at least $m - 2r \ge 1$).
\end{proof}

\subsection{Independent Gluing Lemma}

If $\mathrm{Past}_r(P_1) \cap \mathrm{Past}_r(P_2) = \varnothing$, then CHL local consistency can be decomposed into independent checks for $P_1$ and $P_2$; moat's fixed absorbing state (e.g., all-0) guarantees cross-boundary defects annihilated within $O(r)$ steps.

\begin{proof}
Causal disjointness ensures no shared past state, hence local constraints locally verifiable. Fixed state absorbs defects similar to buffers in circuits: defect signal propagation speed $\le r$ steps/cell, stabilizes in width $m$ moat.
\end{proof}

\begin{proposition}[Defect Absorption Condition]
If system has global absorbing sub-alphabet or safe symbol layer satisfying CHL absorption closure, then fixed absorbing state guarantees defects annihilated within $O(r)$ steps. Lacking this structure, only almost-specification: gluing holds with exponentially decaying failure probability.
\end{proposition}

\subsection{Conclusion: c-block-gluing}

There exists constant $c = c(m, r, Q)$ such that when support distance $\ge c$, two blocks can be glued into global legal configuration via moat mediation. Specifically $c \le m + 2r$, corresponding to linear-gap version of quantified block-gluing (failure probability exponentially small, degrading to almost-specification if lacking complete block-gluing).

\section{Brudno Numerical Verification Script}

\textbf{Pseudocode}:

\begin{verbatim}
# Brudno numerical verification
for n in window_sizes:          # e.g., n = 32..2048
    Wn = extract_window(B, n)   # Extract n×n or n×n×Δt window
    bits = compressor(Wn)       # LZ77/PPMd
    rate[n] = bits / |Wn|
plot(n, rate[n])                # Observe convergence to h_μ plateau
\end{verbatim}

\textbf{Report}: Compression ratio curve on history-freezing Rule-110-SFT aligns with known $h_\mu$ estimate; error bars from window selection and phase field.

\subsection{Reproducible Experimental Parameter Table}

\begin{table}[h]
\centering
\caption{Experimental parameters for Brudno verification}
\small
\begin{tabular}{@{}lll@{}}
\toprule
Parameter & Value & Description \\
\midrule
Window Size Sequence & 32, 64, 128, 256, 512 & Cube $n \times n \times \Delta t$ \\
Sampling Step $\Delta t$ & $n/4$ & Time depth scales with spatial scale \\
Boundary Handling & Periodic / Fixed Padding & Run both for comparison \\
Compressor & LZ-77 / PPMd (H variant) & Dictionary 32KB, window 8KB \\
Compressor Parameters & zlib: level=9 & Maximal compression config \\
& PPMd: order=6, mem=16MB & \\
Independent Run Count & 100 & Random seed sampling per size \\
Serialization Order & Row-major ($x \to y \to t$) & Window flattened to 1D \\
Alphabet Encoding & Direct binary (1 bit/state) & Rule 110 states $\{0,1\}$ \\
\bottomrule
\end{tabular}
\end{table}

\textbf{Random Seed Control}: Each round uses independent random initial slice (uniform distribution), computing compression rate mean and standard deviation. Edge effects quantified by comparing periodic/fixed boundaries.

\section{Open Problems}

\subsection{Minimal Moat Overhead}

Given steady-state duration $T$, what is minimal box thickness/redundancy achieving extendability?

\subsection{Mixing Threshold}

Under what perturbation/defect density does decoding remain robust?

\subsection{Expression Boundary of Forced Families}

What higher-order properties can self-similar constructions force without global invariants?

\subsection{Measure Uniqueness}

Is internal maximal entropy measure unique? (2D SFT typically non-unique; unique under 1D primitive edge matrix, Parry measure.) If not, how do typicality conclusions vary with measure families?

\subsection{Global Recycling and Reversibility}

In reversible CA substrate, what are bounds of "reversible readback" for semantic collapse?

\bibliographystyle{plain}
\begin{thebibliography}{99}

\bibitem{berger1966} R. Berger, ``The Undecidability of the Domino Problem'', \emph{Memoirs of the American Mathematical Society} \textbf{66} (1966).

\bibitem{robinson1971} R. M. Robinson, ``Undecidability and nonperiodicity for tilings of the plane'', \emph{Inventiones Mathematicae} \textbf{12} (1971), 177--209.

\bibitem{mozes1989} S. Mozes, ``Tilings, substitution systems and dynamical systems generated by them'', \emph{Journal d'Analyse Math\'{e}matique} \textbf{53} (1989), 139--186.

\bibitem{durand2012} B. Durand, A. Romashchenko, and A. Shen, ``Fixed-point tile sets and their applications'', \emph{Journal of Computer and System Sciences} \textbf{78}(3) (2012), 731--764.

\bibitem{aubrun2013} N. Aubrun and M. Sablik, ``Simulation of effective subshifts by two-dimensional subshifts of finite type'', \emph{Acta Applicandae Mathematicae} \textbf{126} (2013), 35--63.

\bibitem{gacs2001} P. G\'{a}cs, ``Reliable cellular automata with self-organization'', \emph{Journal of Statistical Physics} \textbf{103}(1-2) (2001), 45--267.

\bibitem{hedlund1969} G. A. Hedlund, ``Endomorphisms and Automorphisms of the Shift Dynamical System'', \emph{Mathematical Systems Theory} \textbf{3}(4) (1969), 320--375.

\bibitem{moore1962} E. F. Moore, ``Machine Models of Self-Reproduction'', \emph{Proceedings of the Symposium on Mathematical Problems in the Biological Sciences} (1962), 17--33.

\bibitem{myhill1963} J. Myhill, ``The Converse of Moore's Garden-of-Eden Theorem'', \emph{Proceedings of the American Mathematical Society} \textbf{14}(5) (1963), 685--686.

\bibitem{solomonoff1964} R. J. Solomonoff, ``A formal theory of inductive inference'', \emph{Information and Control} \textbf{7}(1-2) (1964), 1--22, 224--254.

\bibitem{levin1974} L. A. Levin, ``Laws of information conservation (nongrowth) and aspects of the foundation of probability theory'', \emph{Problems of Information Transmission} \textbf{10}(3) (1974), 206--210.

\bibitem{cook2004} M. Cook, ``Universality in Elementary Cellular Automata'', \emph{Complex Systems} \textbf{15}(1) (2004), 1--40.

\bibitem{neary2006} T. Neary and D. Woods, ``P-completeness of cellular automaton Rule 110'', \emph{Automata, Languages and Programming} (2006), 132--143.

\bibitem{burton1994} R. Burton and J. E. Steif, ``Non-uniqueness of measures of maximal entropy for subshifts of finite type'', \emph{Ergodic Theory and Dynamical Systems} \textbf{14}(2) (1994), 213--235.

\bibitem{brudno1983} A. A. Brudno, ``Entropy and the complexity of the trajectories of a dynamical system'', \emph{Transactions of the Moscow Mathematical Society} \textbf{44} (1983), 127--151.

\bibitem{lind1995} D. Lind and B. Marcus, \emph{An Introduction to Symbolic Dynamics and Coding}, Cambridge University Press (1995).

\bibitem{kari1994} J. Kari, ``Rice's Theorem for the Limit Sets of Cellular Automata'', \emph{Theoretical Computer Science} \textbf{127}(2) (1994), 229--254.

\end{thebibliography}


\end{document}

