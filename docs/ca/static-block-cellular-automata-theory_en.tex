\documentclass[11pt]{article}
\usepackage[utf8]{inputenc}
\usepackage{amsmath,amssymb,amsthm}
\usepackage{geometry}
\geometry{margin=1in}
\usepackage{hyperref}
\usepackage{enumitem}
\setlist[itemize]{nosep,leftmargin=1.6em}
\setlist[enumerate]{nosep,leftmargin=1.6em}
\usepackage{natbib}
\usepackage{booktabs}

% Overfull hbox prevention
\emergencystretch=1em
\raggedbottom
\tolerance=2000

% Theorem environments
\newtheorem{theorem}{Theorem}[section]
\newtheorem{lemma}[theorem]{Lemma}
\newtheorem{proposition}[theorem]{Proposition}
\newtheorem{corollary}[theorem]{Corollary}
\theoremstyle{definition}
\newtheorem{definition}[theorem]{Definition}
\newtheorem{example}[theorem]{Example}
\theoremstyle{remark}
\newtheorem{remark}[theorem]{Remark}
\newtheorem{limitation}{Limitation}

\title{Static Block Cellular Automata Theory:\\ A Formal Reconstruction from Dynamic Evolution to Spacetime Graphs}

\author{Haobo Ma\thanks{Independent Researcher} \and Wenlin Zhang\thanks{National University of Singapore}}

\date{\today}

\begin{document}

\maketitle

\begin{abstract}
Static Block Cellular Automata Theory treats Cellular Automata (CA) as an eternal, immutable static block structure, where the temporal dimension is incorporated into the coordinate system, forming a high-dimensional data body. The core insight of this theory is that CA evolution is not a dynamic process, but rather a sequential reading of predetermined static mappings. From a global perspective, CA can be expressed as spacetime graphs subject to local consistency constraints. Through formal definitions, mathematical proofs, and simulation verification, we construct this theoretical framework and discuss its applications in computational theory, information theory, and philosophical eternalism. The theory is logically self-consistent: all states are uniquely determined by initial conditions and rules, forming an immutable spacetime field. This paper proves that the spacetime graph set \( Y \) is a subshift of finite type and is equivalent to traditional CA evolution, avoiding unrigorous mathematical embeddings such as forced applications of Hilbert spaces, and instead using tools more suitable for discrete structures, such as Boolean Fourier analysis and topological dynamics.

\noindent\textbf{Keywords}: Cellular automata, block universe, static spacetime, symbolic dynamical systems, subshift of finite type, Curtis-Hedlund-Lyndon theorem, Garden-of-Eden theorem
\end{abstract}

\section{Introduction}

\subsection{Background and Motivation}

Cellular automata, as discrete computational models, were proposed by von Neumann and Ulam to simulate emergent behaviors of complex systems. The traditional view treats CA as dynamic evolutionary systems: cells update their states according to local rules at discrete time steps. However, from the perspective of block universe theory and eternalism, we can reinterpret CA as static block structures.

In the block universe, spacetime is a four-dimensional eternal entity where all events exist simultaneously; the flow of time is merely an observer's illusion. When this idea is applied to CA, the temporal dimension becomes a coordinate axis, with the entire evolutionary history predetermined.

\subsection{Core Theoretical Ideas}

The Static Block CA Theory constructed in this paper emphasizes that CA is essentially a static, complete data body, consisting of spacetime graphs defined by local rules and initial conditions. This framework not only deepens our understanding of CA but also bridges computational models with physical philosophy. We avoid mismatched mathematical tools from previous versions (such as Hilbert space embeddings) and instead focus on more appropriate frameworks like topological dynamics and Boolean Fourier analysis to ensure logical self-consistency.

\subsection{Paper Structure}

This paper is organized as follows:

\begin{itemize}
\item \textbf{Section~\ref{sec:formal}} Establishes formal definitions of CA and configuration spaces
\item \textbf{Section~\ref{sec:static}} Defines static block structures and space-time graphs
\item \textbf{Section~\ref{sec:subshift}} Establishes subshift and subshift of finite type (SFT) formulations
\item \textbf{Section~\ref{sec:curtis}} States the Curtis-Hedlund-Lyndon theorem
\item \textbf{Section~\ref{sec:existence}} Proves existence and uniqueness of static blocks and local dependency cones
\item \textbf{Section~\ref{sec:linear}} Discusses group Fourier decomposition of linear CA
\item \textbf{Section~\ref{sec:walsh}} Introduces Walsh expansion for nonlinear Boolean rules
\item \textbf{Section~\ref{sec:garden}} States the Garden-of-Eden theorem and its significance for static blocks
\item \textbf{Section~\ref{sec:applications}} Applications: computational optimization, reversibility and quantum embedding, philosophical implications
\item \textbf{Section~\ref{sec:complexity}} Discusses complexity and undecidability boundaries
\item \textbf{Section~\ref{sec:conclusions}} Conclusions and prospects
\end{itemize}

\section{Formal Definition of Cellular Automata}\label{sec:formal}

\begin{definition}[Cellular Automaton]\label{def:ca}
A \( d \)-dimensional cellular automaton is defined as a quadruple \( \mathcal{C} = (\mathbb{Z}^d, \Sigma, N, f) \), where:
\begin{itemize}
\item \( \mathbb{Z}^d \) is the set of spatial lattice points
\item \( \Sigma \) is a finite state set (e.g., \( \{0,1\} \))
\item \( N \subset \mathbb{Z}^d \) is the neighborhood (e.g., von Neumann or Moore neighborhood)
\item \( f: \Sigma^{|N|} \to \Sigma \) is the local update function
\end{itemize}
\textbf{Evolution Rule}:
\[
x_{t+1}(\mathbf{r}) = f\left( \{ x_t(\mathbf{r} + \mathbf{n}) \mid \mathbf{n} \in N \} \right)
\]
where \( \mathbf{r} \in \mathbb{Z}^d \), \( t \in \mathbb{N} \).
\end{definition}

\begin{definition}[Neighborhood Radius]\label{def:radius}
Let \( r := \max_{\mathbf{n} \in N} |\mathbf{n}|_1 \) be the \( L_1 \) radius of the neighborhood (propagation radius), where \( |\mathbf{n}|_1 = \sum_{i=1}^d |n_i| \).
\end{definition}

\begin{definition}[Configuration Space]\label{def:config}
Let the configuration space \( X = \Sigma^{\mathbb{Z}^d} \) be endowed with the product discrete topology (prodiscrete topology). This is a compact space (when \( \Sigma \) is discrete and finite), guaranteed by Tychonoff's theorem.
\end{definition}

\begin{definition}[Global Mapping]\label{def:global}
Given neighborhood \( N \subset \mathbb{Z}^d \) and local rule \( f: \Sigma^{|N|} \to \Sigma \), define the global mapping
\[
G: X \to X, \qquad (G x)(\mathbf{r}) = f\left( \{ x(\mathbf{r} + \mathbf{n}) \mid \mathbf{n} \in N \} \right)
\]
This is the standard dynamic formulation of CA.
\end{definition}

\begin{definition}[Shift Mapping]\label{def:shift}
For any \( \mathbf{v} \in \mathbb{Z}^d \), define the shift action \( \sigma^\mathbf{v}: X \to X \) as
\[
(\sigma^\mathbf{v} x)(\mathbf{r}) = x(\mathbf{r} + \mathbf{v})
\]
\end{definition}

\section{Static Block: Space-Time Graphs}\label{sec:static}

\begin{definition}[Space-Time Graph]\label{def:spacetime}
For any initial state \( x_0 \in X \), define the space-time graph
\[
\mathcal{M}: \mathbb{Z}^d \times \mathbb{N} \to \Sigma, \qquad \mathcal{M}(\mathbf{r}, t) = (G^t x_0)(\mathbf{r})
\]
where \( G^t \) denotes the \( t \)-th iteration of the global mapping \( G \).
\end{definition}

\begin{definition}[Static Block]\label{def:staticblock}
We call \( \mathcal{M} \) a static block, which is a function defined on \( \mathbb{Z}^{d+1} \) (treating time as the \( (d+1) \)-th coordinate) satisfying the following \textbf{local consistency constraints}:
\[
\forall (\mathbf{r}, t) \in \mathbb{Z}^d \times \mathbb{N}, \quad \mathcal{M}(\mathbf{r}, t+1) = f\left( \{ \mathcal{M}(\mathbf{r} + \mathbf{n}, t) \mid \mathbf{n} \in N \} \right)
\]
\end{definition}

\begin{remark}
``Static block'' refers to the overall object \( \mathcal{M} \) obtained by treating time as an additional coordinate, which satisfies local consistency constraints and forms a closed invariant set under shift actions, rather than a pointwise fixed point of a single operator.
\end{remark}

\begin{definition}[Bidirectional Static Block]\label{def:bidir}
If \( G \) is invertible (bijective), then a bidirectional static block can be defined on \( t \in \mathbb{Z} \):
\[
\mathcal{M}: \mathbb{Z}^d \times \mathbb{Z} \to \Sigma
\]
If \( G \) is not invertible, the static block is naturally restricted to the forward time domain \( \mathbb{Z}^d \times \mathbb{N} \).
\end{definition}

\begin{definition}[Space-Time Shift Invariance]\label{def:stshift}
Define the space-time shift \( \sigma^{(\mathbf{v},\tau)}: \Sigma^{\mathbb{Z}^{d+1}} \to \Sigma^{\mathbb{Z}^{d+1}} \) as
\[
\big(\sigma^{(\mathbf{v},\tau)}\mathcal{M}\big)(\mathbf{r},t) = \mathcal{M}(\mathbf{r}+\mathbf{v}, t+\tau)
\]
The set of static blocks remains invariant under all spatial and temporal shifts \( (\mathbf{r},t) \mapsto (\mathbf{r} + \mathbf{v}, t + \tau) \).
\end{definition}

\begin{remark}[Observer Perspective]
The observed ``evolution'' is a projection and sequential reading of \( \mathcal{M} \) on slices \( t = \mathrm{const} \):
\[
\mathcal{O}_t: \mathcal{M} \mapsto \mathcal{M}_t = \{ (\mathbf{r}, \sigma) \mid \mathcal{M}(\mathbf{r}, t) = \sigma \}
\]
This reinterprets dynamic evolution as sequential access to a static structure.
\end{remark}

\section{Subshift and Subshift of Finite Type (SFT) Formulation}\label{sec:subshift}

\begin{definition}[Forbidden Pattern Set]\label{def:forbidden}
The local rule \( f \) introduces a finite forbidden pattern set \( \mathcal{F} \) on \( \mathbb{Z}^{d+1} \), whose members are finite patterns that violate the local consistency constraint
\[
\mathcal{M}(\mathbf{r}, t+1) = f\left( \{ \mathcal{M}(\mathbf{r} + \mathbf{n}, t) \mid \mathbf{n} \in N \} \right)
\]
\end{definition}

\begin{definition}[Subshift of Finite Type]\label{def:sft}
The set of all space-time graphs satisfying the constraints
\[
Y \subseteq \Sigma^{\mathbb{Z}^{d+1}}
\]
is a \textbf{Subshift of Finite Type (SFT)} and remains invariant under all spatial and temporal shift actions \( \sigma^{(\mathbf{v},\tau)} \).
\end{definition}

\begin{proposition}[SFT Characterization]\label{prop:sft}
The static block \( \mathcal{M} \) is an element of the SFT \( Y \), and ``evolution'' is a temporal slice reading of \( Y \).
\end{proposition}

\begin{proof}
By Definitions~\ref{def:spacetime} and~\ref{def:forbidden}, \( \mathcal{M} \) satisfies local consistency constraints, therefore \( \mathcal{M} \in Y \). The temporal slice \( \mathcal{M}_t \) corresponds to the projection at fixed \( t \).
\end{proof}

\begin{remark}[Significance of SFT]
This aligns the ``spacetime graphs subject to local consistency constraints'' with the standard framework of symbolic dynamical systems.
\end{remark}

\section{Curtis-Hedlund-Lyndon Theorem}\label{sec:curtis}

\begin{theorem}[Curtis-Hedlund-Lyndon Theorem]\label{thm:curtis}
A mapping \( G: X \to X \) commutes with all shifts \( \sigma^\mathbf{v} \) and is continuous in the product discrete topology if and only if there exists a finite neighborhood \( N \) and local rule \( f: \Sigma^{|N|} \to \Sigma \) such that
\[
(G x)(\mathbf{r}) = f\left( \{ x(\mathbf{r} + \mathbf{n}) \mid \mathbf{n} \in N \} \right)
\]
\end{theorem}

\begin{remark}
\textbf{Significance}: This theorem provides a topological characterization of CA and ensures that the locality of ``static block'' constraints is both sufficient and necessary.

\textbf{Note}: Assuming \( X \) is endowed with the product discrete topology and \( \Sigma \) is finite, otherwise the theorem does not hold.

\textbf{Reference}: Hedlund (1969), Ceccherini-Silberstein \& Coornaert (2010)
\end{remark}

\section{Existence, Uniqueness, and Local Dependence of Static Blocks}\label{sec:existence}

\begin{lemma}[Local Dependency Cone/Light Cone]\label{lem:lightcone}
Let the \( L_1 \) radius of the CA neighborhood \( N \) be \( r := \max_{\mathbf{n} \in N} |\mathbf{n}|_1 \). Then for any \( \mathbf{r}_0 \in \mathbb{Z}^d \), \( t \in \mathbb{N} \), we have
\[
\mathcal{M}(\mathbf{r}_0, t) \text{ depends only on the initial state } x_0 \restriction_{B_{L_1}(\mathbf{r}_0; rt)}
\]
where \( B_{L_1}(\mathbf{r}_0; rt) = \{ \mathbf{r} \in \mathbb{Z}^d : |\mathbf{r} - \mathbf{r}_0|_1 \le rt \} \).
\end{lemma}

\begin{proof}
By induction on \( t \).

\textbf{Base case} (\( t=0 \)): \( \mathcal{M}(\mathbf{r}_0, 0) = x_0(\mathbf{r}_0) \) depends only on \( x_0(\mathbf{r}_0) \), satisfying the condition.

\textbf{Inductive step}: Assume the statement holds for \( t \). Then
\[
\mathcal{M}(\mathbf{r}_0, t+1) = f\left( \{ \mathcal{M}(\mathbf{r}_0 + \mathbf{n}, t) \mid \mathbf{n} \in N \} \right)
\]
By the induction hypothesis, each \( \mathcal{M}(\mathbf{r}_0 + \mathbf{n}, t) \) depends only on \( B_{L_1}(\mathbf{r}_0 + \mathbf{n}; rt) \).

Since \( |\mathbf{n}|_1 \le r \), we have
\[
B_{L_1}(\mathbf{r}_0 + \mathbf{n}; rt) \subseteq B_{L_1}(\mathbf{r}_0; r(t+1))
\]
Therefore \( \mathcal{M}(\mathbf{r}_0, t+1) \) depends only on \( x_0 \restriction_{B_{L_1}(\mathbf{r}_0; r(t+1))} \).
\end{proof}

\begin{remark}[Light Cone]
In the 1D case, this is equivalent to the interval \( [\mathbf{r}_0 - rt, \mathbf{r}_0 + rt] \); in higher dimensions, it corresponds to the \( L_1 \) ball centered at \( \mathbf{r}_0 \). This is the ``light cone'' or ``dependency cone'' of CA.
\end{remark}

\begin{theorem}[Existence and Uniqueness of Space-Time Graphs]\label{thm:existence}
Given \( (N, f) \) and initial state \( x_0 \in X \), the space-time graph \( \mathcal{M} \) exists and is unique.
\end{theorem}

\begin{proof}
\textbf{Existence}: Define \( G^t x_0 \) by induction on \( t \in \mathbb{N} \). Since \( G \) is a deterministic mapping, each step \( G^{t+1} x_0 = G(G^t x_0) \) is well-defined.

\textbf{Uniqueness}: If there exist two different \( \mathcal{M}_1, \mathcal{M}_2 \) satisfying the initial state \( x_0 \) and local consistency constraints, then there must be some \( (\mathbf{r}_0, t_0) \) such that \( \mathcal{M}_1(\mathbf{r}_0, t_0) \neq \mathcal{M}_2(\mathbf{r}_0, t_0) \). Taking the smallest \( t_0 \), at time \( t_0 \) they must already differ on the neighborhood \( N \) of \( \mathbf{r}_0 \), which contradicts their agreement at time \( t_0 - 1 \).
\end{proof}

\begin{remark}[Uniqueness for Fixed Initial State]
Note that this is uniqueness for a fixed initial state \( x_0 \). From the SFT perspective, the set of admissible spacetime graphs \( Y \) is a high-dimensional shift space that may contain uncountably many elements (e.g., reversible CA).
\end{remark}

\section{Group Fourier Decomposition of Linear CA}\label{sec:linear}

\begin{definition}[Linear CA]\label{def:linear}
When the local rule \( f \) is linear over a finite field \( \mathbb{F}_q \), \( G \) is a convolution operator on the abelian group \( \mathbb{Z}^d \) (under appropriate encoding).
\end{definition}

\begin{proposition}[Group Fourier Diagonalization]\label{prop:fourier}
For linear CA, the group Fourier transform can be used to simultaneously diagonalize \( G \):
\begin{itemize}
\item \textbf{Finite rings/periodic boundaries}: Obtain discrete frequency eigenmodes
\item \textbf{Infinite lattices}: Use characters \( \chi_\xi(\mathbf{r}) = e^{2\pi i \langle \xi, \mathbf{r} \rangle} \) for spectral analysis
\end{itemize}
This provides true orthogonal decomposition and spectral radius/propagation cone conclusions.
\end{proposition}

\begin{remark}
For infinite lattice cases, the group Fourier transform exists in the \( L^2(\mathbb{Z}^d) \) square-integrable function space, requiring appropriate boundary conditions to ensure convergence of spectral analysis.
\end{remark}

\begin{example}[Rule 90]\label{ex:rule90}
In 1D periodic boundaries over \( \text{GF}(2) \), Rule 90 is equivalent to convolution action on the generating polynomial \( p(z) = z + z^{-1} \); the spectrum is given by \( p(\omega) \) at unit roots. This exhibits the light cone structure.

\textbf{Visualization Description}: Rule 90's static block exhibits the typical Sierpinski triangle pattern. Given the initial state \( x_0 = \delta_0 \) (a single pulse with 1 at the center), the static block \( \mathcal{M}(r,t) \) at position \( r \) and time \( t \) is given by the \( (t,r) \) entry of Pascal's triangle modulo 2, forming a fractal structure. This pattern clearly demonstrates ``evolution'' as temporal slice reading of a static block: the entire Sierpinski triangle pre-exists, and observers merely access its different levels in temporal order.
\end{example}

\begin{remark}[Group Ring Representation]
By embedding the state space into vector spaces over field \( \mathbb{F} \), linear rules correspond to convolution actions in the group ring \( \mathbb{F}[\mathbb{Z}^d] \).
\end{remark}

\section{Walsh Expansion for Nonlinear Boolean Rules}\label{sec:walsh}

\begin{definition}[Walsh Expansion]\label{def:walsh}
Define the standardization mapping \( \varphi: \{0,1\} \to \{-1,1\} \) as \( \varphi(0) = 1 \), \( \varphi(1) = -1 \). For a Boolean function \( f: \{0,1\}^{|N|} \to \{0,1\} \), convert it to \( \tilde{f}: \{-1,1\}^{|N|} \to \{-1,1\} \) through \( \tilde{f} = \varphi \circ f \circ \varphi^{-1} \), then perform Boolean Fourier/Walsh expansion:
\[
\tilde{f}(z) = \sum_{S \subseteq N} \widehat{\tilde{f}}(S) \, \chi_S(z), \qquad \chi_S(z) = \prod_{j \in S} z_j
\]
where \( \{\chi_S\} \) are orthogonal under the expectation inner product, and \( \widehat{\tilde{f}}(S) \) characterizes influence/higher-order interactions.
\end{definition}

\begin{proposition}[Walsh Coefficients]\label{prop:walsh}
The Walsh coefficients \( \widehat{\tilde{f}}(S) \) satisfy
\[
\widehat{\tilde{f}}(S) = \mathbb{E}_{z \sim \{-1,1\}^{|N|}} [\tilde{f}(z) \chi_S(z)]
\]
\end{proposition}

\begin{remark}[Representation Tool]
Important: This is an orthogonal expansion in function space, not implying that global dynamics can be decomposed or ``uncoupled''. We use it only as a representation tool, not as proof of dynamical independence.

\textbf{Reference}: O'Donnell (2014)
\end{remark}

\section{Garden-of-Eden Theorem}\label{sec:garden}

\begin{definition}[Pre-injective]\label{def:preinj}
A mapping \( G: X \to X \) is called \textbf{pre-injective} if for any finite set \( F \subset \mathbb{Z}^d \) and any two initial states \( x \neq y \), as long as they agree on the complement (i.e., \( x \restriction_{F^{\mathrm{c}}} = y \restriction_{F^{\mathrm{c}}} \)), we have \( G x \neq G y \).

Intuitively, it is not allowed to obtain the same image by modifying the initial state only in a finite region.
\end{definition}

\begin{theorem}[Garden-of-Eden Theorem]\label{thm:garden}
For CA with finite alphabets, the global mapping \( G \) is surjective if and only if it is pre-injective.

\textbf{Reference}: Moore (1962), Myhill (1963)
\end{theorem}

\begin{corollary}[Garden-of-Eden Patterns and Reversibility]\label{cor:garden}
\begin{itemize}
\item No ``Garden-of-Eden patterns'' (unreachable configurations) \( \Leftrightarrow \) \( G \) is surjective
\item Reversibility \( \Rightarrow \) bidirectional static block \( \mathbb{Z}^d \times \mathbb{Z} \) exists
\item If not surjective, only forward static block \( \mathbb{Z}^d \times \mathbb{N} \) can be constructed
\end{itemize}
\end{corollary}

\begin{remark}[Bidirectionality and Reversibility]
This directly connects the bidirectionality of the ``eternal block'' with reversible/surjective properties, avoiding disconnection between philosophical and mathematical levels.
\end{remark}

\section{Applications and Discussion}\label{sec:applications}

\subsection{Computational Applications}

\begin{proposition}[Parallel Optimization]\label{prop:parallel}
The static block framework can optimize simulation: large spacetime segments can be generated in parallel from the block perspective. For radius \( r \) 1D CA, the state at \( (\mathbf{r}_0, t) \) depends only on the initial state interval \( [\mathbf{r}_0 - rt, \mathbf{r}_0 + rt] \) (dependency cone/light cone); in \( d \)-dimensional cases, this corresponds to the \( L_1 \) ball \( B_{L_1}(\mathbf{r}_0; rt) \) centered at \( \mathbf{r}_0 \).
\end{proposition}

\begin{remark}[Linear CA Acceleration]
In linear or composable rule families, \( t \) steps of evaluation can be organized into \( \tilde{O}(\log t) \) rounds of parallel composition by constructing \( G^{2^k} \) (such as convolution powers and FFT/polynomial fast exponentiation); general nonlinear CA still have \( \Omega(t) \) depth lower bounds, and we do not claim stronger universal parallel upper bounds here.
\end{remark}

\subsection{Reversibility and Quantum Embedding}

\begin{proposition}[Quantum Embedding Prerequisites]\label{prop:quantum}
Only when the global mapping \( G \) is bijective can it be realized as unitary evolution in a Hilbert space of the same dimension; non-reversible CA can be converted to reversible by introducing auxiliary tapes/preserving history (Bennett trick; Toffoli-Margolus partitioned CA), then embedded into quantum frameworks.
\end{proposition}

\begin{remark}[Limiting Conditions]
Reversibility is a necessary but not sufficient condition for quantum embedding. Here we explicitly limit the scope of application.
\end{remark}

\subsection{Philosophical Implications: Isomorphism with Eternalism}

\begin{definition}[Eternalist Perspective]\label{def:eternal}
In the eternalism framework, time is the order of observation, and evolution is projection
\[
\mathcal{O}_t: \mathcal{M} \to \mathcal{M}_t = \{ (\mathbf{r}, \sigma) \mid \mathcal{M}(\mathbf{r}, t) = \sigma \}
\]
This bridges CA with the block universe: local entropy increase is compatible with global invariance.

\textbf{Reference}: Barbour (1999), Price (2011), Putnam (1967)
\end{definition}

\begin{remark}[Limitations of Analogy]
It must be emphasized that the analogy between CA static blocks and physical block universe is heuristic rather than literal. In CA, time is a discrete parameter, while in relativistic spacetime, time is part of a continuous differential manifold; CA's deterministic evolution and quantum mechanics' probabilistic interpretation also have essential differences. The value of this analogy lies in providing a computational model for thinking about ``time as coordinate''.
\end{remark}

\begin{remark}[Epistemology and Ontology]
The static perspective cannot epistemologically replace the dynamic simulation perspective, because finite observers must explore this static structure through dynamic processes. Static blocks describe the ontological level of ``what is'', while dynamic evolution describes the epistemological level of ``how to know''.
\end{remark}

\subsection{Limitations}

\begin{limitation}[Infinite Storage]
Infinite grids require infinite storage; actual simulations can only handle finite subsets.
\end{limitation}

\begin{limitation}[Computational Feasibility]
The construction of static blocks exists theoretically, but actual computation is limited by resources.
\end{limitation}

\section{Complexity and Undecidability Boundaries}\label{sec:complexity}

The static block set \( Y \subseteq \Sigma^{\mathbb{Z}^{d+1}} \) is an SFT. For dimensions \( d+1 \ge 2 \), several core decision problems of SFT have typical undecidability; in \( d = 1 \), SFT is equivalent to regular languages, and its emptiness is decidable.

\begin{proposition}[Emptiness Problem]\label{prop:emptiness}
Deciding \( Y = \varnothing \) is generally undecidable (equivalent to Berger's tiling problem).

\textbf{Reference}: Berger (1966)
\end{proposition}

\begin{proposition}[Periodicity and Entropy]\label{prop:period}
Existence of periodicity and entropy computation are also generally undecidable or highly difficult (related to high-dimensional tiling/coding constructions).
\end{proposition}

\begin{corollary}[Existence Boundary]\label{cor:boundary}
``There exists a static block satisfying given local consistency constraints'' has no uniform algorithm in high dimensions; this defines the computational boundary of the static perspective.
\end{corollary}

\begin{remark}[Decidable Subclasses]
In linear/finite ring cases, computable spectra and propagation bounds can be obtained, while decision problems for nonlinear, infinite lattice, and high-dimensional cases must be carefully limited to decidable subclasses or adopt semi-decision/approximation methods.
\end{remark}

\begin{remark}[Boundary of ``Can and Cannot'']
This section echoes the SFT/Garden-of-Eden framework above, providing readers with a clear map of the boundary between ``can and cannot''.
\end{remark}

\section{Conclusions and Prospects}\label{sec:conclusions}

\subsection{Main Contributions}

Static Block Cellular Automata Theory reconstructs CA as eternal data bodies, consisting of spacetime objects defined by local consistency constraints. This framework is logically self-consistent and provides new perspectives for understanding the nature of time and computation. We avoid mismatched mathematical frameworks (such as Hilbert space embeddings) and instead use topological dynamics and Boolean Fourier analysis to ensure rigor.

\subsection{Core Insights}

\begin{enumerate}
\item \textbf{Ontological level}: CA ``evolution'' is sequential reading of static blocks, not true dynamic change
\item \textbf{Mathematical characterization}: Through SFT and the Curtis-Hedlund-Lyndon theorem, a rigorous topological dynamics foundation is established
\item \textbf{Computational boundaries}: Through the Garden-of-Eden theorem and undecidability results, the scope of the theory is clarified
\item \textbf{Philosophical implications}: Bridges computational models with block universe theory while clarifying the limitations of analogies
\end{enumerate}

\subsection{Future Directions}

\begin{enumerate}
\item \textbf{Quantum extensions}: Extend to quantum CA or complex system simulation
\item \textbf{High-dimensional analysis}: Deepen research on decidable subclasses of high-dimensional SFT
\item \textbf{Physical applications}: Explore deep connections with quantum gravity and holographic principles
\item \textbf{Computational optimization}: Develop efficient parallel algorithms based on static block perspectives
\end{enumerate}

\begin{thebibliography}{99}

\bibitem{wolfram2002}
S. Wolfram,
\textit{A New Kind of Science},
Wolfram Media, 2002.

\bibitem{sep-ca}
Stanford Encyclopedia of Philosophy,
\textit{Cellular Automata},
2012.

\bibitem{barbour1999}
J. Barbour,
\textit{The End of Time},
Oxford University Press, 1999.

\bibitem{odonnell2014}
R. O'Donnell,
\textit{Analysis of Boolean Functions},
Cambridge University Press, 2014.

\bibitem{hedlund1969}
G. A. Hedlund,
\textit{Endomorphisms and automorphisms of the shift dynamical system},
Mathematical Systems Theory, vol.~3, pp.~320--375, 1969.

\bibitem{moore1962}
E. F. Moore,
\textit{Machine models of self-reproduction},
Proceedings of Symposia in Applied Mathematics, vol.~14, pp.~17--33, 1962.

\bibitem{myhill1963}
J. Myhill,
\textit{The converse of Moore's Garden-of-Eden theorem},
Proceedings of the American Mathematical Society, vol.~14, pp.~685--686, 1963.

\bibitem{ceccherini2010}
T. Ceccherini-Silberstein and M. Coornaert,
\textit{Cellular Automata and Groups},
Springer, 2010.

\bibitem{berger1966}
R. Berger,
\textit{The Undecidability of the Domino Problem},
Memoirs of the American Mathematical Society, no.~66, 1966.

\bibitem{price2011}
H. Price,
\textit{The Flow of Time},
In C. Callender (Ed.), The Oxford Handbook of Philosophy of Time, Oxford University Press, 2011.

\bibitem{putnam1967}
H. Putnam,
\textit{Time and Physical Geometry},
Journal of Philosophy, vol.~64, pp.~240--247, 1967.

\end{thebibliography}

\end{document}

