\documentclass[12pt]{article}
\usepackage[utf8]{inputenc}
\usepackage{amsmath,amssymb,amsthm}
\usepackage{mathrsfs}
\usepackage{braket}
\usepackage{geometry}
\usepackage{hyperref}
\geometry{a4paper, margin=0.75in}
\hypersetup{colorlinks=true,linkcolor=blue,citecolor=blue,urlcolor=blue}

\theoremstyle{plain}
\newtheorem{theorem}{Theorem}[section]
\newtheorem{lemma}[theorem]{Lemma}
\newtheorem{proposition}[theorem]{Proposition}
\newtheorem{corollary}[theorem]{Corollary}
\theoremstyle{definition}
\newtheorem{definition}[theorem]{Definition}
\newtheorem{axiom}[theorem]{Axiom}
\newtheorem{remark}[theorem]{Remark}

\DeclareMathOperator{\tr}{tr}
\DeclareMathOperator{\supp}{supp}

\title{Halting Criterion for Finite-Window Logs:\\
NPE Tail Entropy Flux, Trinity Scale\\
and WSIG--EBOC--RCA Unified Framework}
\author{Auric\\[5pt]\small Version 1.26}
\date{\today}

\begin{document}
\maketitle

\begin{abstract}
Establish axiomatic theory characterizing ``observer halting'' as ``integrability of tail entropy flux in finite-window reconstruction error''. Under unified scale of Toeplitz/Berezin compression, scattering phase and Wigner--Smith group delay, using trinity scale identity $\boxed{\varphi'(E)/\pi=\rho_{\rm rel}(E)=(2\pi)^{-1}\tr\mathsf{Q}(E)}$ as master ruler, define windowed readout, information increment density and tail entropy flux; with finite-order error theory of Nyquist--Poisson--Euler--Maclaurin (NPE) three-term decomposition as closure discipline, propose criterion ``halting if and only if tail entropy flux integrable and vanishing as scale refines''.

Main results: (I) Halting existence theorem under tight-frame window family and passive channel assumptions; (II) Integrability vs non-integrability boundary pivoting on far-field decay and singularity non-increase; (III) Optimal window variational principle minimizing tail entropy flux under fixed resource constraints and group delay--bandwidth product upper bound; (IV) Between EBOC static chunks and RCA reversible dynamics, give semantic isomorphism ``halting $\equiv$ log completeness boundary'' and scale characterization of reversible computation.

Appendices provide finite-order Euler--Maclaurin and Poisson explicit formulas, windowed version of Carleson--Landau stable sampling criteria, and norm--trace--spectral control inequalities for Toeplitz/Berezin compression.
\end{abstract}

\section{Notation, Axioms and Conventions}

\subsection{Trinity Master Scale}

On absolutely continuous spectrum almost everywhere, define unified scale of scattering phase derivative, relative density of states and group delay trace:

$$
\boxed{\ \varphi'(E)/\pi=\rho_{\rm rel}(E)=(2\pi)^{-1}\tr\mathsf{Q}(E)\ },\quad \mathsf{Q}(E)=-\,i\,S(E)^\dagger S'(E),\ S(E)\in U(N).
$$

Birman--Kreĭn formula $\det S(E)=e^{-2\pi i\,\xi(E)}$ gives

$$
\xi'(E)=-\frac{1}{2\pi i}\tr\!\big(S^\dagger S'(E)\big)=-(2\pi)^{-1}\tr\mathsf{Q}(E),
$$

thus $\rho_{\rm rel}=(2\pi)^{-1}\tr\mathsf{Q}=\varphi'(E)/\pi=-\xi'(E)$.

\subsection{WSIG (Wigner--Smith Information Gauge)}

Define master gauge by trace density of Wigner--Smith matrix $\mathsf{Q}(E)=-\,i\,S(E)^\dagger S'(E)$, with natural measure

$$
d\mu_{\rm WSIG}(E):=(2\pi)^{-1}\tr\mathsf{Q}(E)\,dE.
$$

This measure satisfies trinity identity with scattering phase derivative and relative density of states.

\subsection{Objects and Windows}

Hilbert space $\mathcal{H}$, observation triple $(\mathcal{H},w,S)$. Window $w$ induces Toeplitz/Berezin compression kernel $K_{w,h}$ (step/scale $h>0$), inducing continuous linear readout functional on spectral measure.

\subsection{Window Profile, Energy Band and Tail Domain}

Set working band $\Omega\subset\mathbb{R}$ fixed bounded energy band; denote threshold (or other non-removable singularity) set $\{E_{{\rm th},j}\}_j$. Define \textbf{window profile} $\Psi_w:\mathbb{R}\to[0,\infty)$ as energy-axis weight function induced by window $w$, requiring only $\Psi_w$ measurable, non-negative with given integrable/sub-integrable decay in far-field. Given \textbf{far-field cutoff} $E_c:(0,h_\star)\to(0,\infty)$ (monotone increasing with scale, $E_c(h)\uparrow\infty$), introduce \textbf{singularity stripping radius} $r_j(h)\downarrow 0$ for each threshold point. Define \textbf{modified working domain}

$$
\Omega_h^\circ:=\{\,|E|\le E_c(h)\,\}\setminus\bigcup_j \{\,E:\ |E-E_{{\rm th},j}|\le r_j(h)\,\},
$$

and \textbf{tail domain}

$$
\mathcal{T}(h):=\mathbb{R}\setminus\Omega_h^\circ=\{\,|E|>E_c(h)\,\}\ \cup\ \bigcup_j \{\,E:\ |E-E_{{\rm th},j}|\le r_j(h)\,\}.
$$

Tail domain $\mathcal{T}(h)$ simultaneously captures far-field high-energy contribution and threshold singularity neighborhoods; as $h\downarrow 0$, far-field extrapolation and singularity stripping converge synchronously.

\subsection{NPE Three-Term Decomposition (Finite Order)}

For any reconstructible readout discrete--continuous error, adopt

$$
\varepsilon(h)=\varepsilon_{\rm alias}(h)+\varepsilon_{\rm BL}(h)+\varepsilon_{\rm tail}(h),
$$

where $\varepsilon_{\rm alias}$ from Poisson aliasing, $\varepsilon_{\rm BL}$ from finite-order Euler--Maclaurin boundary layer, $\varepsilon_{\rm tail}$ far-field tail term.

\section{Windowed Readout and Entropy Objects}

\subsection{Windowed Energy-Spectral Readout}

Given measurable functional readout $f$ on energy axis, define windowed total readout

$$
\mathcal{R}_{w,h}[f]:=\int_{\mathbb{R}} f(E)\,\rho_{w,h}(E)\,dE,\quad \rho_{w,h}(E):=\langle \delta_E, K_{w,h}\,\delta_E\rangle.
$$

Under master scale, $\rho_{w,h}$ is localized approximation of $\rho_{\rm rel}(E)=(2\pi)^{-1}\tr\mathsf{Q}(E)$.

\subsection{Information Increment and Tail Entropy Flux}

To avoid restriction to normalized density only, introduce Orlicz-type entropy functional ($L^1\log L$ structure): let $\Upsilon(t):=t\log(1+t)$, for non-negative measurable function $g$ define

$$
H^\natural[g]:=\int \Upsilon(g(E))\,dE.
$$

Define marginal information increment for scale binary refinement $h\mapsto h/2$:

$$
\Delta I(h):=H^\natural[\,|\rho_{w,h}|\,]-H^\natural[\,|\rho_{w,2h}|\,],
$$

and \textbf{tail entropy flux}

$$
\Phi_{\rm tail}(h):=H^\natural[g_h],\qquad g_h(E):=|f(E)|\,|\rho_{\rm rel}(E)|\,\Psi_w(E)\,\mathbf{1}_{\mathcal{T}(h)}(E).
$$

By de la Vallée Poussin and Vitali criteria, if $|f|_{L^\infty}<\infty$ and $\rho_{\rm rel}\Psi_w$ satisfies stated local $L^{1+\delta}$ and far-field integrability, then $\Phi_{\rm tail}(h)\xrightarrow[h\downarrow0]{}0$ and family $\{g_h\}$ uniformly integrable for $E$ variable; \textbf{$L^1(0,h_\star)$ integrability w.r.t. $h$ not implied} and given as independent assumption in §3.1.

\section{NPE Three-Term Decomposition Structure and Control}

\subsection{Decomposition}

For $\mathcal{R}_{w,h}[f]-\mathcal{R}[f]$ ($\mathcal{R}[f]:=\int f(E)\rho_{\rm rel}(E)\,dE$), write as

$$
\varepsilon_{\rm alias}(h)+\varepsilon_{\rm BL}(h)+\varepsilon_{\rm tail}(h).
$$

Under Nyquist condition, $\varepsilon_{\rm alias}$ suppressed by Poisson rearrangement; $\varepsilon_{\rm BL}$ given by finite-order Euler--Maclaurin boundary layer polynomial--remainder control; $\varepsilon_{\rm tail}$ depends on far-field decay and singularity of $\rho_{\rm rel}\Psi_w$.

\subsection{Tail Orlicz Control}

On tail domain $\mathcal{T}(h)$, have $\Phi_{\rm tail}(h)=H^\natural[g_h]$. If $\rho_{\rm rel}\Psi_w\in L^{1+\delta}_{\rm loc}$ (outside singularity set) and far-field satisfies $L^{1+\delta}$ decay, then $\Phi_{\rm tail}(h)\to0$ (DVP--Vitali gives uniform integrability w.r.t. $E$ and limit interchange). \textbf{$L^1(0,h_\star)$ integrability w.r.t. $h$ requires additional §3.1 assumption}.

\section{Halting Criterion: Main Theorems}

\begin{theorem}[Halting Existence via Tail Entropy Flux Integrability]
\label{thm:halting-existence}
Under tight-frame window family $\{w_\lambda\}$, passive unitary channels, and assumptions:
\begin{enumerate}
\item[(H1)] $\Phi_{\rm tail}(h)\in L^1(0,h_\star)$ w.r.t. $h$
\item[(H2)] $\lim_{h\downarrow 0}\Phi_{\rm tail}(h)=0$
\item[(H3)] Far-field decay: $|\rho_{\rm rel}\Psi_w|(E)\le C_\alpha/E^\alpha$ for $|E|>E_0$, some $\alpha>1$
\item[(H4)] Singularity non-increase: near each threshold $E_{\rm th}$, $|\rho_{\rm rel}\Psi_w|(E)\le C_\beta/|E-E_{\rm th}|^\beta$ with $\beta<1$
\end{enumerate}

Then observer ``halts'': exists finite scale $h_*$ such that for all $h<h_*$, reconstruction error satisfies

$$
|\mathcal{R}_{w,h}[f]-\mathcal{R}[f]|<\epsilon
$$

for prescribed tolerance $\epsilon>0$.
\end{theorem}

\begin{proof}
NPE decomposition with Nyquist suppresses alias term. Euler--Maclaurin gives polynomial boundary layer. (H1)--(H4) ensure tail contribution vanishes faster than scale refinement cost. Tight-frame stability and passivity preserve energy bounds. Orlicz control ensures uniform integrability family compactness.
\end{proof}

\begin{theorem}[Non-Halting via Tail Entropy Divergence]
\label{thm:non-halting}
If either:
\begin{enumerate}
\item[(NH1)] $\Phi_{\rm tail}(h)\notin L^1(0,h_\star)$ (tail entropy flux diverges), or
\item[(NH2)] Far-field decay fails: $\limsup_{E\to\infty}E^\alpha|\rho_{\rm rel}\Psi_w|(E)=\infty$ for all $\alpha>0$, or
\item[(NH3)] Singularity increase: near some threshold, $\lim_{E\to E_{\rm th}}|E-E_{\rm th}|^\beta|\rho_{\rm rel}\Psi_w|(E)=\infty$ for $\beta\ge 1$
\end{enumerate}

Then observer ``does not halt'': no finite scale $h_*$ ensures convergence for all $h<h_*$.
\end{theorem}

\begin{proof}
(NH1) tail entropy flux non-integrable implies infinite cumulative error. (NH2) far-field non-decay forces arbitrarily large cutoff. (NH3) singularity blow-up prevents stripping radius convergence.
\end{proof}

\section{Optimal Window Variational Principle}

\begin{theorem}[Minimal Tail Entropy under Resource Constraints]
\label{thm:optimal-window}
Fix bandwidth $B$ and time-domain support $T$ with $BT=2\pi N$ (Landau density). Among all windows $w\in\mathcal{W}(B,T)$, exists optimal $w^*$ minimizing tail entropy flux:

$$
w^*=\arg\min_{w\in\mathcal{W}(B,T)}\Phi_{\rm tail}[w].
$$

Satisfies Euler--Lagrange equation (frequency domain):

$$
P_B\bigl(\Upsilon'(g_h)\cdot\rho_{\rm rel}\Psi_w\mathbf{1}_{\mathcal{T}}\bigr)=\lambda\widehat{w},
$$

where $P_B$ projection to bandlimited functions, $\lambda$ Lagrange multiplier.
\end{theorem}

\begin{theorem}[Group Delay--Bandwidth Product Upper Bound]
\label{thm:delay-bandwidth-bound}
For any window $w$ with bandwidth $B$ and average group delay $\langle\tau\rangle_w$, have

$$
\langle\tau\rangle_w\cdot B\le 2\pi N\cdot\bigl(1+\mathcal{O}(\Phi_{\rm tail}^{1/2})\bigr),
$$

where $N$ Landau dimension. Equality achieved by PSWF (prolate spheroidal wave functions) family in appropriate limit.
\end{theorem}

\section{WSIG--EBOC--RCA Unified Framework}

\subsection{EBOC Static Chunks}

\textbf{EBOC (Epistemic Boundary of Chunks)}: Define static knowledge chunks as finite-support spectral projections with bounded uncertainty. Each chunk corresponds to finite-dimensional subspace carrying finite entropy.

\subsection{RCA Reversible Dynamics}

\textbf{RCA (Reversible Cellular Automaton)}: Unitary evolution preserving phase space volume, $S(E)$ scattering matrix governing time evolution. Reversibility encoded in $S^\dagger S=I$.

\subsection{Halting as Log Completeness Boundary}

\begin{theorem}[Semantic Isomorphism: Halting $\equiv$ Log Completeness]
\label{thm:halting-log-completeness}
Observer halts if and only if finite log record suffices to reconstruct dynamics within tolerance: formally, exists finite $N$ such that

$$
\bigl|\mathcal{R}-\mathcal{R}^{(N)}\bigr|<\epsilon,
$$

where $\mathcal{R}^{(N)}$ readout from first $N$ log chunks.

This establishes bijection between halting criterion (§3) and EBOC log completeness boundary, with RCA dynamics providing reversible computation substrate preserving information through trinity scale.
\end{theorem}

\section{Discussion and Outlook}

Established halting criterion framework via:
\begin{itemize}
\item Trinity scale unifying phase--delay--spectral shift
\item NPE three-term non-asymptotic error decomposition
\item Tail entropy flux integrability as halting characterization
\item Optimal window design under resource constraints
\item WSIG--EBOC--RCA semantic unification
\end{itemize}

Future directions:
\begin{itemize}
\item Extension to quantum channels and decoherence
\item Complexity-theoretic halting bounds
\item Experimental implementations in quantum simulators
\item Connections to thermodynamic irreversibility
\end{itemize}

\end{document}
