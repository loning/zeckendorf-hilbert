\documentclass[12pt]{article}
\usepackage[utf8]{inputenc}
\usepackage{amsmath,amssymb,amsthm}
\usepackage{geometry}
\usepackage{hyperref}
\geometry{a4paper, margin=0.75in}
\hypersetup{colorlinks=true,linkcolor=blue}

\theoremstyle{plain}
\newtheorem{theorem}{Theorem}[section]
\theoremstyle{definition}
\newtheorem{definition}[theorem]{Definition}

\title{Mathematical Formalization of G\"odel, Escher, Bach:\\Strange Loops, Self-Reference and Tangled Hierarchies}
\author{Auric\\[5pt]\small Version 1.0}
\date{\today}

\begin{document}
\maketitle

\begin{abstract}
Provide rigorous mathematical formalization of core concepts from Hofstadter's ``G\"odel, Escher, Bach'': strange loops, self-reference, and tangled hierarchies. Establish formal framework using category theory, fixed-point theorems, and computational complexity. Core results: (I) Strange loops as categorical fixed points; (II) Self-reference via diagonal lemma and Lawvere fixed-point theorem; (III) Tangled hierarchies as non-well-founded set systems. Applications to G\"odel incompleteness, halting problem, and consciousness models.
\end{abstract}

\section{Strange Loops}

\begin{definition}[Strange Loop]
\textbf{Strange loop}: sequence of transformations $f_1,\ldots,f_n$ such that

$$
f_n\circ\cdots\circ f_1=\text{id}.
$$

Formalized as cyclic path in category returning to initial object after apparent level-crossing.
\end{definition}

\begin{theorem}[Strange Loop Fixed Point]
Strange loop corresponds to fixed point of composite functor $F=f_n\circ\cdots\circ f_1$: object $X$ such that $F(X)\cong X$.
\end{theorem}

\section{Self-Reference}

\subsection{G\"odel's Diagonal Lemma}

\begin{theorem}[Diagonal Lemma]
For any formula $\phi(x)$ in arithmetic, exists sentence $\psi$ such that

$$
\text{PA}\vdash\psi\leftrightarrow\phi(\ulcorner\psi\urcorner),
$$

where $\ulcorner\psi\urcorner$ G\"odel number of $\psi$.
\end{theorem}

\subsection{Lawvere Fixed-Point Theorem}

\begin{theorem}[Lawvere]
In category $\mathcal{C}$ with object $A$ and surjective morphism $\phi:A\to A^A$, every endomorphism $f:A\to A$ has fixed point: exists $a\in A$ with $f(a)=a$.
\end{theorem}

Unifies G\"odel incompleteness, Turing halting, and Cantor diagonal.

\section{Tangled Hierarchies}

\begin{definition}[Tangled Hierarchy]
Partially ordered set $(P,\le)$ with elements $x_1,\ldots,x_n$ such that

$$
x_1\le x_2\le\cdots\le x_n\le x_1,
$$

violating well-foundedness.
\end{definition}

Formalized via non-well-founded sets (Aczel) or circular category-theoretic structures.

\section{Applications}

\subsection{G\"odel Incompleteness}
G\"odel sentence $G\equiv\neg\text{Prov}(\ulcorner G\urcorner)$ as strange loop: statement asserts own unprovability.

\subsection{Halting Problem}
Self-referential construction: program $H$ testing if program halts on itself.

\subsection{Consciousness Models}
Strange loops as model for self-awareness: system observing itself observing itself...

\section{Discussion}

Rigorously formalized GEB core concepts using category theory and logic. Strange loops as fixed points, self-reference via diagonal constructions, tangled hierarchies as non-well-founded structures. Future work: connections to quantum self-reference and observer-dependent physics.

\end{document}
