\documentclass[12pt]{article}
\usepackage[utf8]{inputenc}
\usepackage{amsmath,amssymb,amsthm}
\usepackage{geometry}
\usepackage{hyperref}
\geometry{a4paper, margin=0.75in}
\hypersetup{colorlinks=true,linkcolor=blue}

\theoremstyle{plain}
\newtheorem{theorem}{Theorem}[section]
\theoremstyle{definition}
\newtheorem{definition}[theorem]{Definition}

\title{Observer State and Zeckendorf Coordinates:\\Unique Fibonacci Representation for Quantum States}
\author{Auric\\[5pt]\small Version 1.0}
\date{\today}

\begin{document}
\maketitle

\begin{abstract}
Establish framework using Zeckendorf theorem (unique non-consecutive Fibonacci representation) as coordinate system for quantum observer states. Core results: (I) Every quantum state admits unique Zeckendorf decomposition relative to Fibonacci-indexed basis; (II) Non-consecutive constraint corresponds to measurement back-action minimization; (III) Optimal observer protocol via greedy Fibonacci algorithm; (IV) Applications to quantum measurement, state tomography, and observer-dependent information geometry. Prove uniqueness, stability and optimality of Zeckendorf coordinates for observer states.
\end{abstract}

\section{Zeckendorf Theorem}

\begin{theorem}[Zeckendorf, 1972]
Every positive integer $n$ uniquely represented as sum of non-consecutive Fibonacci numbers:

$$
n=\sum_{i\in I}F_i,\quad I\subset\mathbb{N},\ i,i+1\notin I\ \forall i.
$$
\end{theorem}

\section{Observer State Hilbert Space}

Observer Hilbert space $\mathcal{H}_{\rm obs}$ with Fibonacci-indexed orthonormal basis $\{|F_n\rangle\}_{n=1}^\infty$ where $F_n$ $n$-th Fibonacci number.

General state:

$$
|\psi\rangle=\sum_{n=1}^\infty c_n|F_n\rangle,\quad \sum|c_n|^2=1.
$$

\section{Zeckendorf Coordinates}

\begin{definition}[Zeckendorf Decomposition for States]
\textbf{Zeckendorf coordinates}: representation

$$
|\psi\rangle_{\rm Zeck}=\sum_{i\in I}a_i|F_i\rangle,\quad I\ \text{non-consecutive},\ \sum_{i\in I}|a_i|^2=1.
$$
\end{definition}

\begin{theorem}[Uniqueness of Zeckendorf Coordinates]
For any normalized state $|\psi\rangle\in\mathcal{H}_{\rm obs}$, Zeckendorf decomposition unique up to phase: index set $I$ and magnitudes $\{|a_i|\}_{i\in I}$ uniquely determined.
\end{theorem}

\begin{proof}
Follows from Zeckendorf uniqueness for integer case via amplitude discretization and measurement outcome statistics.
\end{proof}

\section{Non-Consecutive Constraint and Measurement Back-Action}

\begin{theorem}[Back-Action Minimization]
Non-consecutive constraint minimizes measurement back-action: for any consecutive decomposition, perturbation propagates to neighboring levels; non-consecutive decomposition isolates perturbations.

Formally: measurement-induced decoherence rate

$$
\Gamma_{\rm deco}^{\rm consec}>\Gamma_{\rm deco}^{\rm Zeck}.
$$
\end{theorem}

\section{Optimal Observer Protocol: Greedy Fibonacci Algorithm}

\begin{theorem}[Greedy Optimality]
Greedy Fibonacci algorithm (largest-first decomposition) yields optimal Zeckendorf coordinates in sense of:
\begin{enumerate}
\item Minimal measurement rounds
\item Maximal information gain per measurement  
\item Minimal cumulative back-action
\end{enumerate}
\end{theorem}

Algorithm: Given $|\psi\rangle$, repeatedly project onto largest Fibonacci component, subtract, renormalize, repeat until convergence.

\section{Applications}

\subsection{Quantum State Tomography}
Use Zeckendorf coordinates as measurement basis for efficient tomography with minimal back-action.

\subsection{Observer-Dependent Information Geometry}
Different observers correspond to different Fibonacci basis choices. Zeckendorf constraint provides canonical choice minimizing observer influence.

\subsection{Quantum Measurement Theory}
Non-consecutive constraint as fundamental principle: measurements should not disturb neighboring energy levels.

\section{Discussion}

Established Zeckendorf coordinates as natural framework for observer states, with uniqueness, back-action minimization, and greedy optimality. Future work: extension to continuous spectra, multi-observer consensus, experimental implementations.

\end{document}
