\documentclass[11pt]{article}
\usepackage[utf8]{inputenc}
\usepackage[T1]{fontenc}
\usepackage{amsmath,amssymb,amsthm}
\usepackage{mathtools}
\usepackage{geometry}
\geometry{margin=1in}
\usepackage{hyperref}
\usepackage{cite}
\usepackage{braket}
\usepackage{graphicx}

\newtheorem{theorem}{Theorem}
\newtheorem{lemma}[theorem]{Lemma}
\newtheorem{proposition}[theorem]{Proposition}
\newtheorem{corollary}[theorem]{Corollary}
\theoremstyle{definition}
\newtheorem{definition}[theorem]{Definition}
\newtheorem{assumption}[theorem]{Assumption}
\theoremstyle{remark}
\newtheorem{remark}[theorem]{Remark}

\title{Unified Delay Theory of Entanglement--Consciousness--Time:\\Fourfold Bridging of Spectral--Scattering--Information--Discounting and Cross-Modal Verifiable Scales}

\author{Haobo Ma$^1$ \and Wenlin Zhang$^2$\\
\small $^1$Independent Researcher\\
\small $^2$National University of Singapore}

\date{}

\begin{document}

\maketitle

\begin{abstract}
We propose a unified delay theory spanning from quantum scattering to conscious time perception to social delay discounting. First, we establish a scale identity based on spectral shift--phase--group delay, providing regularization via Kontsevich--Vishik (KV) and relative determinants for countable channels and infinite-dimensional cases, with explicit ``almost everywhere'' differentiability and de-singularization schemes in neighborhoods of thresholds and embedded eigenstates. Second, we characterize the contraction of ``local distinguishability rate'' via monotonicity of quantum Fisher information, giving an operational definition of subjective duration and introducing Petz recovery as the necessary and sufficient condition for equality. Third, we provide unified expressions for ``effective horizon width'' via exponential, hyperbolic, and quasi-hyperbolic discounting with monotonicity criteria, establishing discount parameter mappings through the ``future-self/other-overlap'' factor. Fourth, we introduce a latent coupling strength parameter to cross-align microwave scattering group delays, behavioral thresholds, and discount curves across layers, proposing joint experiments and error budgets. These three domains close in engineering as a cross-scale verifiable framework of ``coupling enhancement--dwell increase--horizon extension.'' Appendices provide detailed proofs of threshold--pole regularization, QFI equality conditions, discount generalization, and complete error budgets.

\textbf{Keywords}: Wigner--Smith group delay; Birman--Kreĭn spectral shift; KV/relative determinant; quantum Fisher information; Petz recovery; subjective time; delay discounting; hyperbolic and quasi-hyperbolic; self--other overlap; multi-layer structural equations
\end{abstract}

\section{Introduction \& Historical Context}

``The formation of time'' manifests in three complementary languages: group delay and density of states change in quantum scattering, distinguishability rate and subjective duration dilation in conscious time perception, and discounting with effective horizons in social decision-making. On the scattering side, work by Wigner and Smith established the connection between phase derivatives and dwell times, precisely defining group delay through the ``lifetime matrix''; the Birman--Kreĭn formula links the scattering determinant to the spectral shift function, forming a closed loop of ``phase--spectral shift--DOS--group delay'' (with Yafaev's framework as rigorous background). In scattering on networks and graphs, Friedel summation can fail due to ``dark states,'' requiring corrected counting for accessible channels.

On the consciousness side, quantum Fisher information serves as the intrinsic metric for parameter estimation, satisfying monotonicity under CPTP maps; its equality and equivalence with ``recoverability/sufficient statistics'' are systematically characterized by Petz theory and subsequent work. Human time perception exhibits reversible ``fast--slow'' dilation phenomena under modulation by emotional and reward pathways, with neural mechanism evidence from classical reviews and multiple neural surveys.

On the social side, empirical facts of delay discounting are widely fitted as hyperbolic or quasi-hyperbolic forms ($\beta$--$\delta$ model), significantly correlated with ``future-self continuity/other overlap.'' This paper unifies the three domains under the common scale of ``coupling--dwell--horizon,'' providing cross-modal joint tests and error closure.

\section{Model \& Assumptions}

\textbf{Variables and measures}: Fix units $\hbar=1$, uniformly measure in frequency variable $\omega$; all derivatives, density of states, and spectral shift are taken with respect to $\omega$.

\textbf{Scattering-side assumption (H$_{\rm sca}$)}: Self-adjoint operator pair $(H,H_0)$ satisfies $H-H_0\in\mathfrak S_1$ or relative trace class; wave operators exist and are complete; energy-layer scattering matrix $S(\omega)$ is differentiable except on zero-measure sets of thresholds--poles--embedded eigenstates. In the infinite-dimensional case, define $\det_{\rm KV}S(\omega)$ and phase $\Phi(\omega)=\arg\det_{\rm KV}S(\omega)$ via KV/relative determinants; the Koplienko case corresponds to second-order spectral shift under Hilbert--Schmidt perturbations. Threshold--pole de-singularization is performed via Jost functions and resolvent expansions, explicitly specifying ``almost everywhere'' differentiability and principal value integral interpretation.

\textbf{Network and channels (H$_{\rm net}$)}: Channels decompose into accessible subspace $\mathcal H_{\rm acc}$ and dark state subspace $\mathcal H_{\rm dark}$. All counting laws are stated on $\mathcal H_{\rm acc}$, with local-form corrections to DOS when necessary.

\textbf{Consciousness side (H$_{\rm cog}$)}: Global evolution $\rho_{AB}(\theta)=e^{-i\theta H}\rho_{AB}e^{i\theta H}$, measurable only on $A$, local channel $\Lambda=\mathrm{Tr}_B$. Quantum Fisher information $F_Q^A(\theta)$ is defined by monotone metrics (Petz class), satisfying data processing inequality and necessary-sufficient equality conditions.

\textbf{Social side (H$_{\rm soc}$)}: Discount weights adopt unified weight function $V(t)$, including exponential $V(t)=\gamma^t$, hyperbolic $V(t)=(1+kt)^{-\alpha}$, and quasi-hyperbolic $V(0)=1,\,V(t\ge1)=\beta\delta^t$. Define effective horizon width $T_\ast=\sum_{t\ge0} w_t$ ($w_t$ normalized weight). Let ``future-self/other overlap'' index be $C\in[0,1]$, mapping to model parameters (e.g., $\gamma,\ k,\ \alpha,\ \beta,\ \delta$) monotonically.

\section{Main Results (Theorems and Alignments)}

\textbf{Boxed reconciliation (unified factor throughout)}: For any unitary $S(\omega)$ and its phase $\Phi(\omega)=\arg\det S(\omega)$, we have $\operatorname{tr}Q(\omega)=\partial_\omega\Phi(\omega)$, and $\varphi(\omega)=\tfrac12\Phi(\omega)$. Thus the unified scale is
$\tfrac{1}{2\pi}\operatorname{tr}Q(\omega)=\tfrac{\varphi'(\omega)}{\pi}=\rho_{\rm rel}(\omega)$ (where $\rho_{\rm rel}=-\xi'$). This equality holds in infinite dimensions and under relative determinants, interpreted via ``almost everywhere'' derivatives and de-singularization regularization.

\begin{theorem}[Theorem 1: Applicability Domain and Regularization of Spectral Shift--Phase--Group Delay]
Under (H$_{\rm sca}$), except for zero-measure sets of thresholds/resonances/embedded eigenstates, for almost all $\omega$ we have
$\tfrac{\varphi'(\omega)}{\pi}=\rho_{\rm rel}(\omega)=\tfrac{1}{2\pi}\operatorname{tr}Q(\omega)$.
If $H-H_0\in\mathfrak S_2$ rather than $\mathfrak S_1$, replace with Koplienko spectral shift and second-order determinant to obtain the corresponding second-order version. Threshold neighborhood corrections and ``dark state'' corrections are in Appendix A.
\end{theorem}

\begin{theorem}[Theorem 2: QFI Monotonicity and Equality Condition for Local Time Scale]
Let $\rho_{AB}(\theta)$ and local channel $\Lambda=\mathrm{Tr}_B$. For any $F_Q$ corresponding to a Petz monotone metric, we have $F_Q^A(\theta)\le F_Q^{AB}(\theta)$. Equality holds if and only if $\Lambda$ is sufficient for the state family, i.e., there exists a Petz recovery $\mathcal R_\sigma$ such that $\mathcal R_\sigma\circ\Lambda[\rho_{AB}(\theta)]=\rho_{AB}(\theta)$ (also holds for equivalent characterizations such as Rényi classes). This condition is the necessary and sufficient criterion for ``local non-decrease.''
\end{theorem}

\begin{proposition}[Proposition 3: Operationalization of Subjective Duration and Entanglement Monotonicity]
Define subjective duration $t_{\rm subj}(\tau)=\int_0^\tau (F_Q^A(t))^{-1/2}\,dt$. If coupling/entanglement enhancement leads to $\partial_E F_Q^A(t)\le0$ almost everywhere, then $\partial_E t_{\rm subj}(\tau)\ge0$. Behavioral agents are given by the quantum Cramér--Rao lower bound $\Delta t_{\min}\ge[mF_Q^A]^{-1/2}$, so $(F_Q^A)^{-1/2}$ can be estimated from psychophysical thresholds.
\end{proposition}

\begin{theorem}[Theorem 3: Dwell Law and Area for Single Poles and Few Channels]
The Breit--Wigner approximation gives $\tau_g(\omega)=\Gamma[(\omega-\omega_0)^2+\Gamma^2]^{-1}$, with integral area $\int_\mathbb R \tau_g(\omega)\,d\omega=\pi$; if feedback reduces effective decay $\Gamma=\Gamma(g)$ monotonically decreasing with coupling, then $\tau_g(\omega_0)=1/\Gamma(g)$ increases monotonically. In multi-channel cases, area is redistributed by coupling weights.
\end{theorem}

\begin{theorem}[Theorem 4: Horizon Monotonicity for Exponential Discounting]
Let $V(t)=\gamma^t$ and $T=(1-\gamma)^{-1}$. If $\gamma=\Gamma(C)$ is strictly increasing, then $dT/dC=\Gamma'(C)/(1-\Gamma(C))^2>0$. This monotonicity is consistent with evidence that future-self continuity/other overlap enhances discount factors.
\end{theorem}

\begin{theorem}[Theorem 5$'$: Effective Width Monotonicity for Hyperbolic Discounting]
Let $V(t)=(1+kt)^{-\alpha}$, take $w_t=V(t)/\sum_{s\ge0}V(s)$, define $T_\ast=\sum_{t\ge0}w_t$. Then $\partial_k T_\ast<0$ and $\partial_\alpha T_\ast>0$, consistent with the exponential model in the $k\to0$ limit (corresponding to $\gamma\to1^-$). Proof in Appendix D. Empirically, hyperbolic fits outperform pure exponential.
\end{theorem}

\begin{theorem}[Theorem 5$''$: Effective Width for Quasi-Hyperbolic $\beta$--$\delta$]
Let $V(0)=1,\ V(t\ge1)=\beta\delta^t$. After normalization, $T_\ast=1+\beta\delta/(1-\delta)$, so $\partial_\beta T_\ast>0$ and $\partial_\delta T_\ast>0$. Meta-analyses show $\beta$--$\delta$ is the mainstream model for characterizing ``present bias.''
\end{theorem}

\begin{theorem}[Theorem 6: Cross-Modal Identifiable Mapping via Latent Coupling Strength]
Introduce latent variable $\kappa$ to uniformly characterize coupling strength, assuming existence of monotone differentiable mappings $\Gamma_{\rm phys}:\kappa\mapsto\Gamma(g)$, $\Gamma_{\rm cog}:\kappa\mapsto F_Q^A$, $\Gamma_{\rm soc}:\kappa\mapsto(\gamma;\ k,\alpha;\ \beta,\delta)$. If joint experiments synchronously collect $\tau_g(\omega_0),\ \Delta t_{\min},\ \gamma$ satisfying a confound-free separable structural equation model, the co-directionality (sign consistency) of the three monotonic relationships can be tested and the latent scale of $\kappa$ estimated. Proof in Appendix E (identification conditions and estimation strategy).
\end{theorem}

\section{Proofs}

\textbf{Proof of Theorem 1 (key points)}: From the Birman--Kreĭn formula $\det S(\omega)=\exp\{-2\pi i\,\xi(\omega)\}$, we get $\Phi'(\omega)=-2\pi\xi'(\omega)$; and $Q=-iS^\dagger\partial_\omega S$ gives $\operatorname{tr}Q=-i\,\partial_\omega\log\det S=\partial_\omega\Phi$. Thus $(2\pi)^{-1}\operatorname{tr}Q=\rho_{\rm rel}=-\xi'=\varphi'/\pi$ ($\varphi=\tfrac12\Phi$). For KV/relative determinant cases, replace at the definition of $\det$ with KV trace and $\zeta$-regularization; threshold and embedded eigenstates are de-singularized via Jost functions and resolvent expansions, ensuring ``almost everywhere'' validity.

\textbf{Proof of Theorem 2 (key points)}: Petz-class monotone metrics satisfy data processing inequality for any CPTP map; letting $\Lambda=\mathrm{Tr}_B$ gives $F_Q^A\le F_Q^{AB}$. Equality is necessary and sufficient for recoverability: there exists a Petz recovery $\mathcal R_\sigma$ such that $\mathcal R_\sigma\circ\Lambda[\rho_{AB}(\theta)]=\rho_{AB}(\theta)$; this condition can also be stated in Rényi families and $\alpha$--$z$ generalizations.

\textbf{Proof of Proposition 3 (key points)}: By the quantum Cramér--Rao lower bound $\Delta t_{\min}\ge[mF_Q^A]^{-1/2}$, we know $(F_Q^A)^{-1/2}$ is the ``unit threshold scale thickness.'' If $\partial_EF_Q^A\le0$, then $\partial_E t_{\rm subj}(\tau)=\int_0^\tau \tfrac12(F_Q^A)^{-3/2}(-\partial_EF_Q^A)\,dt\ge0$.

\textbf{Proof of Theorem 3 (key points)}: In Breit--Wigner form, $\tau_g$ is Cauchy density with area $\pi$. If $\Gamma'(g)<0$, then $\partial_g \tau_g(\omega_0)=-\Gamma'/\Gamma^2>0$. Multi-channel decomposition by coupling, area distribution given by partial widths.

\textbf{Proofs of Theorems 4--5$'$--5$''$ (key points)}: Exponential case yields monotonicity by direct differentiation. Hyperbolic case uses integral trial $\sum_{t\ge0}(1+kt)^{-\alpha}\approx\int_0^\infty(1+kt)^{-\alpha}dt=k^{-1}(\alpha-1)^{-1}$ ($\alpha>1$); after normalization, $T_\ast$ is monotone with $k \downarrow$ and $\alpha \uparrow$. Quasi-hyperbolic yields closed form by direct summation with monotonicity. Model fit superiority in relevant reviews and meta-analyses.

\textbf{Proof of Theorem 6 (key points)}: Assume observation triplet $(\tau_g(\omega_0),\ \Delta t_{\min},\ \gamma)$ is generated by three monotone mappings of latent variable $\kappa$ with additive noise, structural equations $Y_j=h_j(\kappa)+\varepsilon_j$. If $h_j$ are monotone and noise independent, use rank correlation consistency and multi-level SEM to estimate co-directionality of $\operatorname{sign}(\partial h_j/\partial\kappa)$; cross-modal co-directionality is the verifiable criterion for unified coupling hypothesis. Identification details in Appendix E.

\section{Model Applications}

\textbf{Physical side--channel/network dwell measurement}: On a two-port vector network platform, measure $S(\omega)$ and use robust unwrapping and Cauchy smoothing difference to compute $\varphi'(\omega)$ and $\operatorname{tr}Q(\omega)$, verifying $(2\pi)^{-1}\operatorname{tr}Q=\varphi'/\pi$ and consistency with DOS counting; in feedback cavities, tune $g$ to fit monotone law of $\Gamma(g)$, reporting area conservation. For non-minimum phase loops, use Bode/Kramers--Kronig relations for phase--magnitude consistency checks.

\textbf{Consciousness side--``slow--thick'' subjective duration}: Time reproduction and minimum difference thresholds in parallel, estimating $\Delta t_{\min}$ and subjective ratings; in high-connection contexts, expect $\Delta t_{\min}\uparrow$, $t_{\rm subj}\uparrow$, aligned with internal clock--dopamine modulation evidence.

\textbf{Social side--discounting--horizon}: Use adaptive titration to obtain individual $\gamma$ or $(k,\alpha)$/($\beta,\delta$), synchronously collect IOS and future-self continuity, verifying monotone mappings of Theorems 4--5$'$--5$''$.

\section{Engineering Proposals}

\textbf{P1 | Canonical measurement of microwave network group delay}: Phase unwrapping threshold setting, frequency grid $\Delta\omega$, equivalent noise bandwidth and port mismatch error budgeting; use three-point/five-point difference and spline derivative cross-validation; for non-minimum phase, correct parasitic phase via Bode gain--phase relations and Hilbert transform.

\textbf{P2 | ``Subjective duration--QFI proxy'' dual-task paradigm}: Oddball event induction and neutral sequence control in parallel, collect $\Delta t_{\min}$, pupil/skin conductance/HRV for multi-modal fusion to isolate arousal confounds; map $\Delta t_{\min}$ to $F_Q^{\rm beh}\propto\Delta t_{\min}^{-2}$ via CRB.

\textbf{P3 | Hierarchical Bayesian fit of discount curves}: Simultaneously fit exponential/hyperbolic/quasi-hyperbolic and compare with WAIC/AIC/BIC; collect IOS and future-self continuity for mediation analysis; prosocial/trust manipulations under ethical compliance as external validation.

\section{Discussion (Risks, Boundaries, Past Work)}

\textbf{Applicability domain}: The identity holds within $\mathfrak S_1$ or relative trace class; Koplienko case gives second-order version; thresholds--embedded eigenstates require Jost/threshold expansion treatment; ``dark states'' in graphs and feedback networks need explicit exclusion or local DOS correction.

\textbf{Provable--testable boundaries}: On the consciousness side, ``subjective duration'' as behavioral proxy for $F_Q^{-1/2}$ relies on near-saturation of CRB; experimental tests for ``near-saturation'' and bias correction. Social-side model heterogeneity controlled via hierarchical Bayes and model comparison.

\textbf{Relationship with existing work}: This framework anchors in spectral--scattering scales, aligning quantification of conscious and social time on the common geometry of ``dwell--distinguishability--horizon''; it does not claim metaphysical identity of ``time equals entanglement,'' but provides cross-domain operational equivalence and joint criteria. See references for classical reviews and modern advances.

\section{Conclusion}

On the rigorous foundation of spectral--scattering regularization and QFI monotonicity, we provide unified scales and monotone laws for subjective duration and social horizons, establishing the verifiable triad of ``coupling enhancement--dwell increase--horizon extension.'' The common latent variable scale across three domains brings microwave scattering, behavioral thresholds, and discount curves into the same statistical--causal structure, constituting an engineering path for cross-scale time theory.

\section*{Acknowledgements, Code Availability}

Thanks to publicly available literature on spectral shift--group delay, quantum Fisher information, and delay discounting. Scripts for $S$-parameter fitting, phase unwrapping, CRB estimation, and hierarchical Bayesian discount fitting can be directly implemented from appendix pseudocode.

\section*{References}

Wigner (1955); Smith (1960) group delay and lifetime matrix; Yafaev (1992/2010) scattering theory monograph; Pushnitski (2010) Birman--Kreĭn; Texier (2001/2003) Friedel on graphs; Kontsevich--Vishik (1994) KV determinant; Gesztesy--Pushnitski--Simon (2007) Koplienko; Petz (1996/1988) monotone metrics and recovery; Eagleman (2008) time perception review; Ersner-Hershfield (2009/2011) future-self continuity; Mazur (1987) hyperbolic discounting; Laibson (1997) $\beta$--$\delta$.

\appendix

\section{Rigorous Identity in Infinite Dimensions and at Thresholds--Poles}

\textbf{A.1 KV/relative determinant and ``almost everywhere'' derivative}: Let $\det_{\rm KV}S(\omega)=\exp\{\mathrm{TR}_{\rm KV}\log S(\omega)\}$, phase $\Phi=\arg\det_{\rm KV}S$. In $\mathfrak S_1$ case, Birman--Kreĭn gives $\Phi'=-2\pi\xi'$; in $\mathfrak S_2$ case, adopt Koplienko spectral shift to define second-order equality. $\Phi'$ exists except on sets of thresholds--poles--embedded eigenstates.

\textbf{A.2 Jost/threshold expansion}: Threshold neighborhoods adopt Jost functions and resolvent expansions, clarifying principal value interpretation and additional terms for $\xi'$; embedded eigenstates ``leak'' via Fermi golden rule into regular patterns, see Jensen--Kato and subsequent threshold expansion work.

\textbf{A.3 Scattering on graphs and dark state correction}: For local states in $\mathcal H_{\rm dark}$, Friedel counting needs to subtract contributions from inaccessible states; local DOS and injection/emission rates give local version.

\section{QFI Monotonicity, Equality Condition, and Subjective Duration}

\textbf{B.1 Data processing and equality}: Petz results show that for monotone metric $F_Q$ and channel $\Lambda$, $F_Q(\Lambda[\rho_\theta])\le F_Q(\rho_\theta)$; equality if and only if $\Lambda$ is sufficient for the state family, recovery $\mathcal R_\sigma$ exists. Equality criteria for Rényi and $\alpha$--$z$ extensions are equivalent to Petz recovery.

\textbf{B.2 Subjective duration}: Take $t_{\rm subj}(\tau)=\int_0^\tau (F_Q^A(t))^{-1/2}\,dt$. CRB gives $\Delta t_{\min}\ge[mF_Q^A]^{-1/2}$; substituting behavioral measure of $\Delta t_{\min}$ yields empirical estimation formula; estimate is unbiased when near-saturation optimal measurement exists.

\section{Area Conservation for Single Poles and Few Channels}

Breit--Wigner $\tau_g$ is standard Cauchy form, area $\pi$ independent of $\Gamma$; in multi-channel coupling, area is distributed by partial widths; in cavity--feedback networks, test ``coupling enhancement--dwell increase'' by fitting monotone relation of $\Gamma(g)$ and $\tau_g(\omega_0)$.

\section{Discount Generalization and Effective Width}

\textbf{D.1 Hyperbolic family}: $V(t)=(1+kt)^{-\alpha}$, normalization constant $Z=\sum_{t\ge0}(1+kt)^{-\alpha} \approx k^{-1}(\alpha-1)^{-1}$; effective width $T_\ast=\sum w_t$ monotone with $k \downarrow$ and $\alpha\uparrow$.

\textbf{D.2 Quasi-hyperbolic $\beta$--$\delta$}: $T_\ast=1+\beta\delta/(1-\delta)$; when $\beta\to1$, $\delta\to\gamma$ returns to exponential. Literature comparison shows hyperbolic and quasi-hyperbolic outperform pure exponential across multiple categories.

\section{Joint Identification and Statistical Power}

\textbf{E.1 Structural equations and identification}: Let $\tau_g(\omega_0)=h_1(\kappa)+\varepsilon_1,\ \Delta t_{\min}=h_2(\kappa)+\varepsilon_2,\ \gamma=h_3(\kappa)+\varepsilon_3$, $h_j$ monotone, $\varepsilon_j$ independent. Use multi-level SEM and rank correlation to test co-directionality of $\operatorname{sign}(h_1'),\operatorname{sign}(h_2'),\operatorname{sign}(h_3')$.

\textbf{E.2 Statistical power and sample size}: Target effect size $d\in[0.3,0.5]$ detection requires tens to hundreds of samples; report multiple comparison correction and post-hoc manipulation tests. Power and effect size reporting follows standard guidelines.

\section{Implementation Details and Error Budget}

\textbf{F.1 Microwave network (P1) error closure}:
(i) Phase unwrapping: set allowable jump threshold and residual detection; (ii) difference and spline derivative cross-validation, Cauchy smoothing difference suppresses high-frequency noise; (iii) port mismatch and ENBW correction; (iv) non-minimum phase detection and Bode/Hilbert correction.

\textbf{F.2 Time perception (P2) and CRB mapping}:
Parallel collection of $\Delta t_{\min}$, subjective ratings, and physiological indicators, isolating arousal/attention confounds; map $\Delta t_{\min}$ to $F_Q^{\rm beh}$ via CRB, reporting confidence intervals.

\textbf{F.3 Discounting (P3) model comparison}:
Exponential/hyperbolic/quasi-hyperbolic simultaneously fitted with WAIC/AIC/BIC; hierarchical Bayes mitigates individual heterogeneity; collect IOS and future-self continuity as explanatory variables for mediation regression; ethical and blinding controls for potential prosocial/trust manipulations.

\end{document}

