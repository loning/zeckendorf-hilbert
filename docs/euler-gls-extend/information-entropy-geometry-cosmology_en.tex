\documentclass[11pt]{article}
\usepackage[utf8]{inputenc}
\usepackage[T1]{fontenc}
\usepackage{amsmath,amssymb,amsthm}
\usepackage{mathtools}
\usepackage{geometry}
\geometry{margin=1in}
\usepackage{hyperref}
\usepackage{cite}
\usepackage{braket}
\usepackage{graphicx}

\newtheorem{theorem}{Theorem}
\newtheorem{lemma}[theorem]{Lemma}
\newtheorem{proposition}[theorem]{Proposition}
\newtheorem{corollary}[theorem]{Corollary}
\theoremstyle{definition}
\newtheorem{definition}[theorem]{Definition}
\newtheorem{assumption}[theorem]{Assumption}
\theoremstyle{remark}
\newtheorem{remark}[theorem]{Remark}

\title{Information Entropy--Geometric Unification and Windowed Generation of Cosmological Terms:\\From Relative Entropy Hessian to Effective Action, Poisson--Euler--Maclaurin Finite-Order Discipline, and Geometric Entropy Decomposition of Friedmann Equations}

\author{Haobo Ma$^1$ \and Wenlin Zhang$^2$\\
\small $^1$Independent Researcher\\
\small $^2$National University of Singapore}

\date{Version: 1.5}

\begin{document}

\maketitle

\begin{abstract}
Within a unified ``operator--measure--function'' framework, we establish an organic assembly connecting \textbf{multi-order responses of relative entropy}, \textbf{master-scale calibrations of scattering spectra}, \textbf{windowed readout Toeplitz/Berezin compressions}, and \textbf{Nyquist--Poisson--Euler--Maclaurin (NPE) finite-order discipline} to closed derivations of \textbf{geometric effective action} and \textbf{cosmological terms}. First, under Eguchi regularized divergence and Amari $\alpha$-geometry, we prove construction of the Fisher--Rao metric and dual connections; second, under trace-class/relative trace-class perturbation and energy-differentiable scattering theory assumptions, we present a theorem-level statement of the ``master scale'' trinity

$$
\frac{\varphi'(\omega)}{\pi}=-\xi'(\omega)=\frac{1}{2\pi}\operatorname{tr}Q(\omega),\qquad Q=-iS^\dagger\partial_\omega S
$$

pointing out distributional sense corrections at thresholds/long-range potentials. Next, selecting Paley--Wiener / de Branges / Hardy environments, employing \textbf{symmetric smooth allocation} ($\widehat g=\sqrt{\widehat h}$, $h=g\ast\tilde g$), we place $\mathsf K_{w,h}=P\,M_{w^{1/2}}C_g\cdot C_{\tilde g}M_{w^{1/2}}P$ into Schatten trace class and provide \textbf{explicit upper bounds}. Subsequently, unifying Fourier conventions and distinguishing \textbf{Poisson zero-aliasing criterion} ($\Delta<2\pi/B$) from \textbf{Shannon no-aliasing reconstruction} ($\Delta<\pi/B$) in their multiplicative constant differences; under double-layer tail control of ``near band-limited'', we provide EM remainder with $\zeta(2m)$ explicit constants. Using Toeplitz--FIO diagonal-type wave-front relation, we prove \textbf{windowing--compression--convolution singularity non-increase} (holds on $T^*X\setminus 0$ away from zero cut, with band-limited/near band-limited windowing as global inclusion). In a minimal computable model of linearized gravity, we provide \textbf{explicit coefficients} from \textbf{fourth-order response} to \textbf{curvature quadratic invariants}, thereby obtaining the \textbf{scale integral law} for volume terms

$$
\Lambda_{\mathrm{eff}}(\mu)-\Lambda_{\mathrm{eff}}(\mu_0)=\int_{\mu_0}^{\mu}\Xi(\omega)\,d\ln\omega,\qquad [\Xi]=L^{-2},
$$

and provide sufficient conditions for positivity/monotonicity of $\Xi$ plus local non-monotonicity boundaries at resonances/thresholds. The action unifies as

$$
S_{\mathrm{eff}}[g]=\int d^4x\,\sqrt{-g}\Big[\frac{R-2\Lambda_{\mathrm{eff}}(\mu)}{16\pi G}+\alpha R^2+\beta R_{\mu\nu}R^{\mu\nu}+\cdots\Big],
$$

where $\alpha,\beta$ are dimensionless. Using three-dimensional $S^3$ heat kernel--counting function--curvature docking example, we close the spectral--geometric interpretation of FRW curvature term, demonstrating via one-dimensional $\delta$ potential and AB scattering the windowing mechanism of ``single-peak saturation/peak-family quasi-logarithmic accumulation''. Appendices provide complete proofs of all theorems, constant estimates and dimensional tables, plus reproducible experimental/numerical script essentials.

\textbf{MSC}: 53Bxx; 83C05; 58J35; 46E22; 47B35; 42A38; 94A17; 81U40

\textbf{Keywords}: Information geometry; Eguchi regularized divergence; Fisher--Rao metric; Amari $\alpha$-connection; Bregman/Hessian; spectral shift function; Birman--Krein; Wigner--Smith group delay; Toeplitz/Berezin compression; Schatten trace class criteria; Poisson summation; Euler--Maclaurin remainder constants; wave-front set and Toeplitz--FIO; heat kernel/Seeley--DeWitt; spectral action; running vacuum; FRW geometric entropy decomposition
\end{abstract}

\section{Introduction \& Historical Context}

Information geometry characterizes statistical manifolds via Hessian metrics and $\alpha$-connections generated by divergences; second-order response of relative entropy yields Fisher--Rao metric, third-order response corresponds to Amari--Chentsov tensor and $\alpha$-connection. Bregman divergence induces dual flat (Hessian) structure and Legendre dual coordinates in exponential families. In spectral--scattering theory, the Lifshitz--Krein trace formula and Birman--Krein identity relate spectral shift function $\xi$ with scattering determinant; Friedel--Lloyd and Wigner--Smith unify phase derivative, group delay, and density-of-states difference under the same calibration. Heat kernel/Seeley--DeWitt expansion and spectral action principle provide standard tools for bridging ``geometric invariants--windowed spectra''. This paper closes these elements under \textbf{theorem-level assumptions} into a logical chain from ``relative entropy--master scale--windowing--NPE--heat kernel--FRW''.

\section{Model \& Assumptions}

\subsection{Fourier Convention, Variables, and Window Kernel General Declaration}

Fix

$$
\widehat f(\omega)=\int_{\mathbb R} e^{-i\omega x}f(x)\,dx,\qquad
f(x)=\frac{1}{2\pi}\int_{\mathbb R} e^{i\omega x}\widehat f(\omega)\,d\omega.
$$

Throughout, we uniformly use \textbf{frequency} $\omega$ to record energy variables (readers may view $E\equiv\omega$).

Windows $w_\mu$ take smoothed logarithmic windows, satisfying $w_\mu\in C_0^\infty\cap L^\infty$ with $\operatorname{supp}w_\mu\subset[\mu_0,\mu]$, $\mu>\mu_0>0$; specifically one may take

$$
w_\mu(\omega)=\frac{\psi(\omega)}{\omega},\qquad \psi\in C_0^\infty,\ \psi\equiv1\ \text{on}\ [\mu_0,\mu]^\circ,
$$

smoothly cut off at $\omega=\mu_0,\mu$ (if covering interval around $\omega\approx 0$, first take $\mu_0>0$ then take limit).

\textbf{General declaration}: Readout kernel $h$ defaults to \textbf{Bochner positive definite} ($\widehat h\ge 0$, $\widehat h\in L^1$), thus admitting $\widehat g=\sqrt{\widehat h}\in L^2$ such that $h=g\ast\tilde g$.

\subsection{Information Divergence and Dual Flatness}

Regularized divergence $D(\theta\Vert\theta_0)$ with second/third/fourth-order responses

$$
g_{ij}=\partial_i\partial_j D|_{\theta_0},\quad
T_{ijk}=\partial_i\partial_j\partial_k D|_{\theta_0},\quad
\boxed{\ \mathcal K_{ijkl}=\partial_i\partial_j\partial_k\partial_l D|_{\theta_0}\ }.
$$

Denote $\boxed{\ \mathcal K:=\mathcal K^{ij}{}_{ij}\ }$ as full contraction of fourth-order response tensor (unrelated to $\mathsf K_{w,h}$).

Induce Fisher--Rao and $\alpha$-connection: $\Gamma^{(\alpha)}{}_{ijk}=\Gamma^{(0)}{}_{ijk}+\tfrac{\alpha}{2}T_{ijk}$. Bregman divergence $D_\psi$ makes $g=\nabla^2\psi$, $\nabla^{(\pm1)}$ flat.

\subsection{Master Scale, Scattering--Spectral Shift, and Threshold Clauses}

Self-adjoint pair $(H_0,H)$ satisfies trace-class or relative trace-class perturbation, wave operators complete; $S(\omega)$ unitary and weakly differentiable. Spectral shift $\xi(\omega)$ satisfies $\det S(\omega)=e^{-2\pi i\xi(\omega)}$.

\begin{definition}[Total Scattering Phase]
Let

$$
\varphi(\omega):=\frac{1}{2i}\mathrm{Log}\,\det S(\omega),
$$

taking the branch consistent with threshold phase renormalization and continuous as $\omega\to+\infty$. Then

$$
\varphi'(\omega)=\frac{1}{2i}\operatorname{tr}\big(S^{-1}\partial_\omega S\big)=\tfrac12\operatorname{tr}Q(\omega),\qquad Q=-iS^\dagger\partial_\omega S,
$$

hence

$$
\frac{\varphi'(\omega)}{\pi}=-\xi'(\omega)=\frac{1}{2\pi}\operatorname{tr}Q(\omega),
$$

holding in distributional sense on discrete threshold set $\Sigma$.
\end{definition}

\subsection{Toeplitz/Berezin Compression and Readout}

Take Paley--Wiener / de Branges / Hardy space $\mathcal H$, orthogonal projection $P$ (\textbf{norm $|P|=1$}). Let $w\in C_0^\infty\cap L^\infty$, $h=g\ast\tilde g$ as above.

\begin{definition}[Relative Spectral Projection Difference]
Denote $\Pi$ as the distributional kernel of \textbf{relative spectral projection difference} for self-adjoint pair $(H_0,H)$ in energy representation (equivalent to relative spectral measure), such that

$$
\operatorname{tr}(\mathsf K_{w,h}\Pi)=\int w(\omega)\,[h\!\ast!\rho_{\rm rel}](\omega)\,d\omega,
$$

where $\rho_{\rm rel}(\omega)=\frac{\varphi'(\omega)}{\pi}=\frac{1}{2\pi}\operatorname{tr}Q(\omega)$, $Q=-iS^\dagger\partial_\omega S$.
\end{definition}

Define

$$
\mathsf K_{w,h}:=P\,M_{w^{1/2}}C_g\cdot C_{\tilde g}M_{w^{1/2}}P,\qquad
\mathrm{Obs}(w,h)=\operatorname{tr}(\mathsf K_{w,h}\Pi)=\int w(\omega)\,[h\!\ast!\rho_{\rm rel}](\omega)\,d\omega.
$$

\subsection{NPE Discipline and ``Near Band-Limited''}

\textbf{Strictly band-limited}: $\operatorname{supp}\widehat f\subset[-B,B]$.

\textbf{Near band-limited}: $\int_{|\omega|>B}|\widehat f|\,d\omega\le\varepsilon$ and $\int_{|\omega|>B}|\widehat f|^2\,d\omega\le \varepsilon^2$.

\textbf{Criterion distinction} (detailed in Theorem 3): Poisson zero-aliasing term $\Delta<2\pi/B$; Shannon no-aliasing reconstruction $\Delta<\pi/B$.

\subsection{Effective Action and Dimensions}

Take $c=\hbar=1$. Action written as

$$
S_{\mathrm{eff}}=\int d^4x\,\sqrt{-g}\Big[\frac{R-2\Lambda_{\mathrm{eff}}(\mu)}{16\pi G}+\alpha R^2+\beta R_{\mu\nu}R^{\mu\nu}+\cdots\Big],
$$

in four dimensions $\alpha,\beta$ dimensionless; $[\Lambda_{\mathrm{eff}}]=L^{-2}$. Dimensional table in Appendix J.

\section{Main Results (Theorems and Alignments)}

\begin{theorem}[Relative Entropy Hessian and $\alpha$-Connection]
Second-order response of relative entropy yields Fisher--Rao metric, third-order response via $\Gamma^{(\alpha)}{}_{ijk}=\Gamma^{(0)}{}_{ijk}+\tfrac{\alpha}{2}T_{ijk}$ generates $\alpha$-connection; Bregman divergence induces dual flat (Hessian) structure.
\end{theorem}

\begin{theorem}[Master Scale Trinity: Sufficient Conditions and Threshold Corrections]
Under assumptions of §3, $\det S(\omega)=e^{-2\pi i\xi(\omega)}$ and $\boxed{\ \xi'(\omega)=-\tfrac{1}{2\pi}\operatorname{tr}Q(\omega)\ }$ holds almost everywhere; at $\omega\in\Sigma$ or long-range potentials holds in distributional sense with renormalized phase.
\end{theorem}

\begin{theorem}[Poisson Zero-Aliasing Criterion and Shannon Reconstruction Criterion]
Under the present Fourier convention, if $\operatorname{supp}\widehat f\subset[-B,B]$, then

$$
\sum_{n\in\mathbb Z} f(n\Delta)=\frac{1}{\Delta}\sum_{k\in\mathbb Z}\widehat f\!\left(\frac{2\pi k}{\Delta}\right)
$$

with $k\neq 0$ aliasing terms \textbf{strictly zero} if and only if $\Delta<2\pi/B$; \textbf{Shannon no-aliasing reconstruction} requires $\Delta<\pi/B$.
\end{theorem}

\begin{theorem}[NPE: Euler--Maclaurin Explicit Constants and Near Band-Limited Tails]
If $f\in C^{2m}[a,b]$ and is $(B,\varepsilon)$-near band-limited,

$$
|R_m|\le \frac{2\zeta(2m)}{(2\pi)^{2m}}(b-a)\sup_{[a,b]}|f^{(2m)}|+O(\varepsilon),\qquad
\sup|f^{(2m)}|\le C\,B^{2m}|f|_\infty.
$$
\end{theorem}

\begin{theorem}[Toeplitz--FIO Pseudolocality and Singularity Non-Increase]
Let $w\in C^\infty$, $h\in\mathcal S$, $P$ be Toeplitz--FIO with diagonal-type wave-front relation. For any distribution $u$ and any open cone domain $U\Subset T^*X\setminus 0$ away from zero cut,

$$
\mathrm{WF}\big(P\,M_w\,C_h\,P\,u\big)\cap U\ \subseteq\ \mathrm{WF}(u)\cap U.
$$
\end{theorem}

\begin{remark}
When energy-shell windowing (band-limited/near band-limited) excludes low-frequency neighborhood of $|\xi|\approx 0$, we obtain global inclusion

$$
\mathrm{WF}(PM_wC_hPu)\subseteq \mathrm{WF}(u),
$$

and when $\mathrm{WF}(u)\neq \varnothing$, the above inclusion is strict.
\end{remark}

\begin{theorem}[Fourth-Order Response $\to$ Curvature Quadratic Terms: Minimal Computable Model and Coefficients]
Under linearization $g_{\mu\nu}=\eta_{\mu\nu}+h_{\mu\nu}$ in harmonic gauge, decompose by scalar/transverse traceless (TT) and define windowed spectral weights

$$
\mathcal N_s=\!\int d^4k\,k^4\,W(k)\,|\mathcal A_s(k)\sigma(k)|^2,\qquad
\mathcal N_t=\!\int d^4k\,k^4\,W(k)\,|\mathcal A_t(k)h^{\rm TT}(k)|^2,
$$

where $W$ is determined by $\rho_{\rm rel},w,h$. Then

$$
\boxed{\ \int\!\sqrt{-g}\,\mathcal K
=c_1\!\int\!\sqrt{-g}\,R^2+c_2\!\int\!\sqrt{-g}\,R_{\mu\nu}R^{\mu\nu}+\text{(total derivative)}\ },
$$

with

$$
\boxed{\,c_1=\frac{\mathcal N_s}{36},\qquad c_2=\frac{\mathcal N_s}{12}+\frac{\mathcal N_t}{4}\,}.
$$
\end{theorem}

\textbf{Normalization declaration}: The definition of $\mathcal N_{s,t}$ has absorbed all $(2\pi)$ factors and measure constants in the unified Fourier convention of this section; under different conventions, rescaling is required accordingly.

\begin{theorem}[Volume Term Scale Integral Law and Positivity/Monotonicity of $\Xi$]
After information free energy windowing,

$$
\Lambda_{\mathrm{eff}}(\mu)-\Lambda_{\mathrm{eff}}(\mu_0)=\int_{\mu_0}^{\mu}\Xi(\omega)\,d\ln\omega,\qquad
\Xi(\omega)=\langle\mathcal K,\rho_{\rm rel}\rangle_{w_\omega,h},\quad [\Xi]=L^{-2}.
$$

If $\rho_{\rm rel}(\omega)\ge 0$ and induced two-point kernel with $w_\omega,h$ are non-negative/Bochner positive definite, then $\Xi(\omega)\ge 0$ and monotonically non-decreasing in $\ln\mu$; if $\rho_{\rm rel}$ sign-variable, only window-averaged sense quasi-monotonicity is obtained, or change $\Xi$ to quadratic form to obtain strict non-negativity. Thresholds/resonance clusters can cause local non-monotonicity, but when peak families are near-uniformly dense in $\ln\omega$ with slowly varying weights, $\Xi$ is nearly constant over wide intervals, exhibiting ``quasi-logarithmic'' accumulation.
\end{theorem}

\begin{theorem}[FRW Curvature Term Spectral--Geometric Docking]
Three-dimensional constant-curvature manifold heat kernel asymptotic

$$
\mathrm{Tr}\,e^{-t\Delta}\sim (4\pi t)^{-3/2}\Big[\mathrm{Vol}+\frac{t}{6}\!\int R +O(t^{2})\Big],\quad t\downarrow 0,
$$

for $S^3(L)$ has $R=6/L^2$, $\mathrm{Vol}=2\pi^2L^3$. Windowed counting function sub-leading term $\propto \int R\propto \kappa\,\mathrm{Vol}$ consistent with FRW's $-\kappa/a^2$ term; window shape only alters coefficients, does not break homogeneity and isotropy.
\end{theorem}

\section{Proofs}

\subsection{Theorem 1 (Relative Entropy Hessian and $\alpha$-Connection)}

Follows from Eguchi's contrast functional and Amari--Chentsov tensor definition. Realized via Bregman potential in exponential families for dual flatness.

\subsection{Theorem 2 (Master Scale Trinity: Sufficient Conditions and Threshold Corrections)}

Proof in three steps:
\begin{enumerate}
\item \textbf{Spectral shift function definition}: Defined by Lifshitz--Krein trace formula.
\item \textbf{Scattering determinant relation}: Birman--Krein identity yields $\det S=e^{-2\pi i\xi}$.
\item \textbf{Derivative relation}: Differentiability of stationary scattering derives $\xi'=-(2\pi)^{-1}\operatorname{tr}Q$.
\end{enumerate}

Threshold/long-range potential cases hold in distributional sense with phase renormalization correction.

\subsection{Theorem 3 (Poisson Zero-Aliasing Criterion and Shannon Reconstruction Criterion)}

By Poisson formula and present Fourier convention, $\widehat f(2\pi k/\Delta)=0$ ($k\ne0$) if and only if $\Delta<2\pi/B$.

Shannon no-aliasing reconstruction requires stricter condition: $\Delta<\pi/B$.

\subsection{Theorem 4 (NPE: Euler--Maclaurin Explicit Constants and Near Band-Limited Tails)}

Employ DLMF's EM remainder constants and Bernstein-type derivative bounds. Near band-limited tails enter $O(\varepsilon)$.

\subsection{Theorem 5 (Toeplitz--FIO Pseudolocality and Singularity Non-Increase)}

Hörmander pseudolocality yields $\mathrm{WF}(M_w u)\subseteq \mathrm{WF}(u)$, while $C_h$ is smoothing. Toeplitz--FIO diagonal-type wave-front relation implies: for any $U\Subset T^*X\setminus 0$,

$$
\mathrm{WF}(PM_wC_hPu)\cap U\subseteq \mathrm{WF}(u)\cap U.
$$

Under band-limited/near band-limited windowing, can take $U$ covering entire $T^*X\setminus 0$, thus $\mathrm{WF}(PM_wC_hPu)\subseteq \mathrm{WF}(u)$.

\subsection{Theorem 6 (Fourth-Order Response $\to$ Curvature Quadratic Terms: Minimal Computable Model and Coefficients)}

From linearized decomposition obtain $\boxed{\ \mathcal K_{ijkl}\ }$ contribution; its full contraction $\boxed{\ \mathcal K\ }$ matches $R^2,\,R_{\mu\nu}R^{\mu\nu}$ with coefficients $c_1=\mathcal N_s/36,\ c_2=\mathcal N_s/12+\mathcal N_t/4$.

\textbf{Scalar mode}: Linearized curvature $R^{(1)}=-6\Box \sigma$, thus $R^2=36\,k^4\sigma^2$, $R_{\mu\nu}^{(1)}R^{\mu\nu (1)}=12\,k^4\sigma^2$.

\textbf{TT mode}: $R^{(1)}=0$, $R_{\mu\nu}R^{\mu\nu}=\tfrac14 k^4(h^{\rm TT})^2$.

Matching windowed fourth-order kernel weights yields coefficients $c_{1,2}$.

\subsection{Theorem 7 (Volume Term Scale Integral Law and Positivity/Monotonicity of $\Xi$)}

Low-frequency cluster (Poisson's $k=0$) dominates volume term. When $\rho_{\rm rel}(\omega)\ge 0$ and kernel/window non-negative/Bochner positive definite, $\Xi\ge 0$; if $\rho_{\rm rel}$ sign-variable, requires window averaging or change to quadratic form.

Tauberian control when peak families near-uniformly dense in $\ln\omega$ ensures ``quasi-logarithmic'' intervals.

\subsection{Theorem 8 (FRW Curvature Term Spectral--Geometric Docking)}

Use $S^3$ spectrum $\lambda_n=n(n+2)/L^2$, multiplicity $(n+1)^2$ and Tauberian theorem to recover heat kernel sub-leading term and dock with FRW curvature term.

\section{Model Applications}

\subsection{One-Dimensional $\delta$ Potential: Single-Peak Saturation and Quasi-Logarithmic Accumulation}

Take $V(x)=\lambda\delta(x)$. Under the present unit convention \textbf{expressing in energy variable $E\equiv\omega$}, phase shift written as

$$
\delta(E)=\delta\big(k(E)\big)=-\arctan\frac{\lambda}{2k(E)},\quad k(E)=\sqrt{E}\ (\text{may take }2m=1),
$$

hence \textbf{relative density of states}

$$
\boxed{\ \rho_{\rm rel}(E)=\frac{1}{\pi}\frac{d\delta}{dE}
=\frac{1}{\pi}\frac{d\delta}{dk}\frac{dk}{dE}\ } \qquad\big(\text{below identify $E$ with $\omega$}\big).
$$

Subsequently employ analytic integration of logarithmic window with Lorentzian peak to demonstrate ``single-peak saturation/peak-family quasi-logarithmic accumulation'', compatible with above formula. For smooth logarithmic window

$$
I(\mu;\mu_0)=\int_{\mu_0}^{\mu}\frac{\Gamma}{(\omega-\omega_0)^2+\Gamma^2}\frac{d\omega}{\omega}
$$

has closed form

$$
\begin{aligned}
I(\mu;\mu_0)&=\frac{\Gamma}{\omega_0^2+\Gamma^2}\ln\frac{\mu}{\mu_0}
-\frac{\Gamma}{2(\omega_0^2+\Gamma^2)}\ln\frac{(\mu-\omega_0)^2+\Gamma^2}{(\mu_0-\omega_0)^2+\Gamma^2}\\
&\quad+\frac{\omega_0}{\omega_0^2+\Gamma^2}\Big[\arctan\frac{\mu-\omega_0}{\Gamma}-\arctan\frac{\mu_0-\omega_0}{\Gamma}\Big],
\end{aligned}
$$

two classes of $\ln\mu$ exactly cancel, \textbf{single-peak saturation}; when peak families near-uniformly dense in $\ln\omega$ with slowly varying weights, ``quasi-logarithmic'' accumulation emerges.

\textbf{Reproducible experimental essentials (example parameters)}: $\lambda=1$; $\mu_0=10^{-3}$, scan $\mu$ to $10^{3}$; window width smoothing parameter $\sigma=0.05$; kernel $h(\omega)=e^{-\omega^2/2\sigma_h^2}$ take $\sigma_h=0.1$.

\subsection{AB Scattering: Windowing Topology--Spectral Density Difference}

Ideal AB model phase shift energy-independent, $\operatorname{tr}Q=0$; finite-radius/screened models introduce energy dependence, windowed difference forms effective contribution to curvature/topological terms, non-analytic points correspond to steps/cusps in $\Xi$.

\section{Engineering Proposals}

\begin{enumerate}
\item \textbf{Group delay measurement chain}: Measure multi-port $S(\omega)$ and differentiate phase to obtain $Q(\omega)$, construct $\Xi(\omega)$ and $\Lambda_{\rm eff}(\mu)$ curves, NPE constants provide error bands.
\item \textbf{Toeplitz/Berezin numerical spectrology}: Implement $\mathsf K_{w,h}$ and monitor $|\mathsf K_{w,h}|_1$; semiclassical regime approximate trace by symbol integration and assess remainder by EM constants.
\item \textbf{FRW curvature windowing verification}: On $S^3/H^3/$three-torus compare heat kernel sub-leading term with windowed counting function, verify spectral--geometric docking of $-\kappa/a^2$.
\end{enumerate}

\section{Discussion}

Master scale trinity holds under trace-class/relative trace-class and differentiability assumptions; long-range potentials and thresholds corrected in distributional sense. Symmetric smooth allocation of Toeplitz/Berezin provides checkable trace-class upper bounds; NPE discipline forms finite-order error budget by $\zeta(2m)$ constants and frequency-domain tail control; windowing--compression--convolution non-increases singularity. Fourth-order response to $R^2,\,R_{\mu\nu}R^{\mu\nu}$ coefficients verifiable in minimal model; positivity--monotonicity conditions for $\Xi$ explicit, peak-family statistics support ``quasi-logarithmic'' intervals. Windowed interpretation of FRW curvature term closed via $S^3$ example. Extensions to open systems or non-unitary $S$ require dissipative scattering framework, where $\operatorname{tr}Q$ loses positivity-preservation.

\section{Conclusion}

Completing theorem-level closure from \textbf{information divergence--master scale--windowing--NPE--heat kernel--FRW}:
\begin{enumerate}
\item[(i)] Master scale trinity holds under theorem-level assumptions;
\item[(ii)] Toeplitz/Berezin compression enters trace class via symmetric smooth allocation with explicit upper bounds ($|P|=1$);
\item[(iii)] Poisson zero-aliasing and Shannon reconstruction criteria separated with consistent constants;
\item[(iv)] EM remainder has $\zeta(2m)$ constants, near band-limited tails controllable;
\item[(v)] Windowing--compression--convolution non-increases singularity (strictly non-increasing under energy-shell windowing);
\item[(vi)] Fourth-order response to curvature quadratic term coefficients \textbf{explicitly verifiable};
\item[(vii)] Volume term obeys scale integral law, positivity and ``quasi-logarithmic'' mechanism of $\Xi$ clear;
\item[(viii)] FRW curvature term spectral--geometric docking complete.
\end{enumerate}

These results provide verifiable technical foundation for unified scheme of ``information geometry $\times$ spectral--scattering $\times$ cosmology''.

\section*{Acknowledgements, Code Availability}

No proprietary code used; appendices contain reproducible experimental/numerical script essentials and parameter tables.

\section*{References}

\begin{itemize}
\item Amari, S.-i.; Nagaoka, H. \textit{Methods of Information Geometry}. AMS--OUP, 2000 (Chs. 2--4: Fisher--Rao and $\alpha$-connection; Ch. 8: dual flatness).
\item Eguchi, S. ``A differential geometric approach to statistical inference on the basis of contrast functionals.'' \textit{Hiroshima Math. J.} 15 (1985) 341--391.
\item Birman, M. S.; Krein, M. G. ``On the theory of wave and scattering operators.'' 1962 (see Yafaev, Chs. 6--8).
\item Yafaev, D. R. \textit{Mathematical Scattering Theory}. AMS, 1992/2005 (Ch. 8: scattering matrix and spectral shift; Ch. 10: thresholds).
\item Simon, B. \textit{Trace Ideals and Their Applications}. 2nd ed., AMS, 2005 (Ch. 2: Schatten ideals; HS$\times$HS $\subset$ S1).
\item Boutet de Monvel, L.; Guillemin, V. \textit{The Spectral Theory of Toeplitz Operators}. Princeton, 1981 (Chs. 1--3: Toeplitz--FIO and wave-front relations).
\item Hörmander, L. \textit{The Analysis of Linear Partial Differential Operators I}. 2nd ed., Springer, 1990 (§8: pseudolocality; wave-front set basics).
\item NIST DLMF, §24 (Euler--Maclaurin, especially §24.7 remainder constants).
\item Vassilevich, D. V. ``Heat kernel expansion: user's manual.'' \textit{Phys. Rep.} 388 (2003) 279--360 (Seeley--DeWitt coefficients).
\item Chamseddine, A.; Connes, A. ``The spectral action principle.'' \textit{Commun. Math. Phys.} 186 (1997) 731--750.
\item Chavel, I. \textit{Eigenvalues in Riemannian Geometry}. Academic Press, 1984 (Ch. III: Weyl law and heat kernel).
\item Texier, C. ``Wigner time delay and related concepts.'' \textit{Phys. Rep.} 2016 (2015 lecture version available).
\item Hagen, C. R. ``Aharonov--Bohm scattering of particles with spin.'' \textit{Phys. Rev. D} 41 (1990).
\end{itemize}

\appendix

\section{Fourier Convention and Variable Unification}

Provide the paper's fixed Fourier pair and all $(2\pi)$ factor absorption rules, declare equivalent use of $\omega\equiv E$, list dimensional consistency under transformations.

\section{Master Scale Equation KFL Chain Closure}

From Lifshitz--Krein trace formula define $\xi$; Birman--Krein identity gives $\det S=e^{-2\pi i\xi}$; differentiability of stationary scattering derives $\xi'=-(2\pi)^{-1}\operatorname{tr}Q$; distributional sense correction for thresholds/long-range potentials.

\section{Toeplitz/Berezin--Schatten Trace Class: Symmetric Smooth Allocation}

Take $\widehat g=\sqrt{\widehat h}$, $h=g\ast\tilde g$, write

$$
\mathsf K_{w,h}=(P\,M_{w^{1/2}}C_g)\,(C_{\tilde g}M_{w^{1/2}}P),
$$

both factors Hilbert--Schmidt, multiplication yields $\mathfrak S_1$. Upper bound

$$
|\mathsf K_{w,h}|_1\le |P|^2\,|w|_1\,\frac{|\widehat h|_1}{2\pi},
$$

with $|P|=1$ in three types of spaces.

\section{Poisson Zero-Aliasing and Shannon Reconstruction (Multiplicative Constant Difference)}

Under present Fourier convention prove: $\Delta<2\pi/B\Rightarrow$ Poisson zero-aliasing term; $\Delta<\pi/B\Rightarrow$ Shannon reconstruction; provide frequency-domain illustration and sufficiency-necessity proof for aliasing term vanishing.

\section{EM Remainder Constants and Near Band-Limited Tails}

Employ DLMF remainder expression and Bernoulli number constants, combine Bernstein derivative bound to yield
$|R_m|\le \frac{2\zeta(2m)}{(2\pi)^{2m}}(b-a)\sup|f^{(2m)}|+O(\varepsilon)$.

\section{Wave-Front Set Non-Increase and Away-From-Zero Cut}

Hörmander pseudolocality yields $\mathrm{WF}(M_w u)\subseteq \mathrm{WF}(u)$, while $C_h$ smoothing. Toeplitz--FIO diagonal-type wave-front relation implies: for any $U\Subset T^*X\setminus 0$,

$$
\mathrm{WF}(PM_wC_hPu)\cap U\subseteq \mathrm{WF}(u)\cap U.
$$

Under band-limited/near band-limited windowing, can take $U$ covering entire $T^*X\setminus 0$, thus $\mathrm{WF}(PM_wC_hPu)\subseteq \mathrm{WF}(u)$.

\section{Fourth-Order Response to $R^2, R_{\mu\nu}R^{\mu\nu}$ Coefficient Derivation}

From linearized decomposition obtain $\boxed{\ \mathcal K_{ijkl}\ }$ contribution; its full contraction $\boxed{\ \mathcal K\ }$ matches $R^2,\,R_{\mu\nu}R^{\mu\nu}$ with coefficients $c_1=\mathcal N_s/36,\ c_2=\mathcal N_s/12+\mathcal N_t/4$.

Scalar: $R^{(1)}=-6\Box\sigma \Rightarrow R^2=36\,k^4\sigma^2$, $R_{\mu\nu}R^{\mu\nu}=12\,k^4\sigma^2$.
TT: $R^{(1)}=0$, $R_{\mu\nu}R^{\mu\nu}=\tfrac14 k^4(h^{\rm TT})^2$.
Matching windowed fourth-order kernel weights yields
$c_1=\mathcal N_s/36,\ c_2=\mathcal N_s/12+\mathcal N_t/4$.

\section{Positivity of $\Xi(\omega)$ and Counterexample Boundaries}

When $\rho_{\rm rel}(\omega)\ge 0$ and induced two-point kernel with $w,h$ non-negative/Bochner positive definite, $\Xi\ge 0$; if $\rho_{\rm rel}$ sign-variable, $\Xi$ may be negative, requiring window averaging or change to quadratic form for non-negativity. Thresholds/resonance clusters can lead to local negativity; variation bound of window-averaged $\overline{\Xi}$ on logarithmic scale provides measurable criterion for ``quasi-logarithmic'' intervals.

\section{Logarithmic Window $\times$ Lorentzian Peak Identity and Saturation}

Complete derivation of

$$
\int_{\mu_0}^{\mu}\frac{\Gamma}{(\omega-\omega_0)^2+\Gamma^2}\frac{d\omega}{\omega}
$$

decomposition formula, prove two classes of $\ln\mu$ cancellation and single-peak saturation; Tauberian control when peak families near-uniformly dense in $\ln\omega$.

\section{Dimensional Table and Reproducible Experimental/Numerical Script Essentials}

\textbf{Dimensional table}:
$[Q]=E^{-1}$, $[\rho_{\rm rel}]=E^{-1}$, $[w_\mu]=E^{-1}$, $[h]=1$, $[\Xi]=L^{-2}$, $[\Lambda_{\rm eff}]=L^{-2}$, $[R]=L^{-2}$, $[\alpha]=[\beta]=1$, $[d\ln\omega]=1$.

\textbf{Script essentials}: Kernel width $\sigma_h$, window smoothing $\sigma$, bandwidth $B$, sampling $\Delta$, tail $\varepsilon$ recommended ranges and a set of ``$\delta$ potential/AB finite radius'' parameter tables, facilitating reproducible experiments and error budgets.

\end{document}


