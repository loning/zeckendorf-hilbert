\documentclass[11pt]{article}
\usepackage[utf8]{inputenc}
\usepackage[T1]{fontenc}
\usepackage{amsmath,amssymb,amsthm}
\usepackage{mathtools}
\usepackage{geometry}
\geometry{margin=1in}
\usepackage{hyperref}
\usepackage{cite}
\usepackage{braket}
\usepackage{graphicx}

\newtheorem{theorem}{Theorem}
\newtheorem{lemma}[theorem]{Lemma}
\newtheorem{proposition}[theorem]{Proposition}
\newtheorem{corollary}[theorem]{Corollary}
\theoremstyle{definition}
\newtheorem{definition}[theorem]{Definition}
\newtheorem{assumption}[theorem]{Assumption}
\theoremstyle{remark}
\newtheorem{remark}[theorem]{Remark}

\title{Local Quantum Sufficient Conditions for Fully Nonlinear Gravity Equations:\\Small-Diamond Generalized Entropy Extrema, Relative Entropy Foliation Independence, and QNEC Pointwise Saturation}

\author{Haobo Ma$^1$ \and Wenlin Zhang$^2$\\
\small $^1$Independent Researcher\\
\small $^2$National University of Singapore}

\date{}

\begin{document}

\maketitle

\begin{abstract}
In the pointwise small causal diamond limit, this paper proposes three \textbf{completely local} and \textbf{sufficient} quantum--geometric criteria, \textbf{rigorously} deriving fully nonlinear gravity equations with cosmological constant within semiclassical--holographic windows. The three criteria are: (A) Small-diamond \textbf{generalized entropy extremum} under fixed ``effective volume/conformal Killing energy'' constraint; (B) Boundary relative entropy \textbf{foliation independence} if and only if bulk Iyer--Wald \textbf{canonical energy conservation} (thereby yielding \textbf{quantum Bianchi identity} and its sourced form); (C) \textbf{QNEC pointwise saturation} for \textbf{all local cut surfaces} and \textbf{all null directions} through a point. We prove: Within Hadamard states, no gravitational anomaly, and non-negative canonical energy coupling windows, any one of the above criteria (supplemented by technical assumptions specified herein) \textbf{sufficiently} implies

$$
E^{\rm grav}_{ab}=8\pi G_{\rm ren}\,\langle T^{\rm tot}_{ab}\rangle+\phi\,g_{ab},\qquad \nabla_b\phi=0,
$$

thus incorporating $\phi$ into cosmological constant $\Lambda$. Core technical tools include: (i) \textbf{Volume--Hamiltonian $O(r^d)$ equivalence theorem} (Proposition K.1), i.e., ``fixed $V^{\rm eff}_\xi\Leftrightarrow$fixed $H_\xi$'' holds universally in Einstein--Hilbert and $f(R)$ prototypes; (ii) \textbf{Two-cap boundary kernel $O_{\mathcal D'}(r^{d+2})$ distributional cancellation theorem} (Theorem J.1); (iii) Existence--uniqueness--regularity of \textbf{quantum rest representative surface} and $O(r^{d+2})$ constraint on area second-order formula; (iv) \textbf{Cohomological invariance} for JKM shifts and corner corrections; (v) \textbf{Contraction mapping integrability lemma} under De Donder gauge and ``near saturation $\Rightarrow$ near equation'' stability inequality. FRW and AdS$_3$/CFT$_2$ instances plus executable specifications of ``relative entropy flux meter / QNEC saturation phase diagram'' are provided.
\end{abstract}

\section{Introduction}

Entanglement first law and ball-region family methods have established ``information to geometry'' foundations at \textbf{linear/second-order} level. To close at \textbf{pointwise} and \textbf{fully nonlinear} level requires reconciling three types of constraints: \textbf{equilibrium} (entropy extremum), \textbf{conservation} (canonical energy conservation), and \textbf{rigidity} (QNEC saturation). This paper theoremizes these three constraint types in pointwise small causal diamond $D_{p,r}$ limit, provides \textbf{complete proofs and error control}, and uniformly derives nonlinear field equations with $\Lambda$.

\section{Setting, Notation, and Assumptions}

\subsection{Small Causal Diamond and Approximate Conformal Killing Field}

Let $(M,g)$ be a smooth spacetime with $d\ge 3$, $p\in M$. Denote $D_{p,r}$ as small causal diamond of scale $r\ll \ell_{\rm curv}$, boundary composed of two $C^{1,\alpha}$ null leaves $\mathcal N_\pm$ intersecting at two corner points. Let small parameter $\varepsilon:=r/\ell_{\rm curv}\ll 1$. Take \textbf{approximate conformal Killing field} $\xi^a$ satisfying

$$
\left|\nabla_{(a}\xi_{b)}-\tfrac{1}{d}(\nabla\!\cdot\!\xi)\,g_{ab}\right|_{L^\infty(D_{p,r})}\le C_\xi\,\varepsilon^2,\qquad
\xi^a\big|_{\partial D_{p,r}}=0,
$$

normalized by ``diamond temperature'' $\kappa_\xi=\kappa_0+O(\varepsilon^2)$.

\subsection{State, Renormalization, and Total Stress}

Take Hadamard state and causal prescription. Define

$$
\langle T^{\rm tot}_{ab}\rangle:=\langle T_{ab}\rangle+\tau^{\rm ent}_{ab},\qquad
\tau^{\rm ent}_{ab}:=-\frac{2}{\sqrt{-g}}\frac{\delta W_{\rm nonloc}}{\delta g^{ab}},
$$

requiring

$$
\nabla^a\langle T^{\rm tot}_{ab}\rangle=0.
$$

Allowed local counterterms only redefine $G_{\rm ren},\Lambda$ (Appendix D).

\subsection{QES and Quantum Rest Representative Surface}

Assume existence of unique and stable quantum extremal surface $\Sigma\subset \partial D_{p,r}$, $\delta S_{\rm gen}|_\Sigma=0$. Within equivalence class construct \textbf{quantum rest representative surface} $\hat\Sigma$ with

$$
\theta\big|_p=\sigma\big|_p=0,\qquad \int_{\hat\Sigma}(\theta^2+\sigma^2)\,{\rm d}\lambda=O(r^{d+2}),
$$

whose existence--uniqueness--regularity see Appendix H.

\subsection{Covariant Phase Space and Canonical Energy}

Let $L(g,\text{curvature},\ldots)$ be smooth local, $E^{\rm grav}_{ab}:=\frac{2}{\sqrt{-g}}\frac{\delta}{\delta g^{ab}}\!\int\!\sqrt{-g}L$. By diffeomorphism invariance $\nabla^aE^{\rm grav}_{ab}\equiv 0$. Iyer--Wald structure yields symplectic potential $\boldsymbol\theta$, symplectic current $\boldsymbol\omega$, Noether charge $\mathbf Q_\xi$, and corner term $C_\xi$. Define

$$
\delta H_\xi=\int_\Sigma(\delta\mathbf Q_\xi-\xi\!\cdot\!\boldsymbol\theta-\delta C_\xi),\quad
\mathcal E_\xi(\delta_1,\delta_2)=\int_\Sigma \boldsymbol\omega(\delta_1,\mathcal L_\xi\delta_2).
$$

JKM shifts $\boldsymbol\theta\to\boldsymbol\theta+{\rm d}Y$, $\mathbf Q_\xi\to\mathbf Q_\xi+\xi\!\cdot\!Y$ and corner corrections form equivalence class (Appendix F).

\subsection{QNEC and Second-Order Deformation}

For any null direction $k^a$ and local cut surface family, QNEC reads

$$
(\sqrt h)^{-1}S''_{\rm out}\le 2\pi\,\langle T_{kk}\rangle.
$$

On representative surface, Raychaudhuri yields

$$
(\sqrt h)^{-1}\frac{{\rm d}^2A}{{\rm d}\lambda^2}=-R_{kk}-\tfrac{\theta^2}{d-2}-\sigma^2.
$$

\section{Main Results (Statements)}

\begin{theorem}[A: Generalized Entropy Extremum $\Rightarrow$ Nonlinear Tensor Equation]
Under assumptions of §2, if $\Sigma$ is QES of $D_{p,r}$ with $S_{\rm gen}$ extremal under ``fixed $V^{\rm eff}_\xi$'' (or equivalently ``fixed $H_\xi$'') constraint, then for all null directions $k^a$ through $p$,

$$
E^{\rm grav}_{kk}(p)=8\pi G_{\rm ren}\,\langle T^{\rm tot}_{kk}(p)\rangle.
$$

Furthermore, there exists distribution $\phi$ such that

$$
E^{\rm grav}_{ab}(p)=8\pi G_{\rm ren}\,\langle T^{\rm tot}_{ab}(p)\rangle+\phi(p)\,g_{ab}(p),\qquad \nabla_b\phi=0.
$$
\end{theorem}

\begin{theorem}[B: Foliation Independence $\Leftrightarrow$ Canonical Energy Conservation; Quantum Bianchi]
If boundary relative entropy $S_{\rm rel}^{\rm bdy}$ is independent of Cauchy slice $\Sigma_s$, then

$$
\nabla^a\big(E^{\rm grav}_{ab}-8\pi G_{\rm ren}\langle T^{\rm tot}_{ab}\rangle\big)=0.
$$

With external flux or corner injection, source term emerges

$$
\nabla^a\big(E^{\rm grav}_{ab}-8\pi G_{\rm ren}\langle T^{\rm tot}_{ab}\rangle\big)=J_b,\quad
J_b=\nabla^a\big[\delta Q_{\xi,ab}-(\xi\!\cdot\!\boldsymbol\theta)_{ab}-\delta C_{\xi,ab}\big],
$$

with $J_b$ invariant under JKM shifts and corner corrections.
\end{theorem}

\begin{theorem}[C: QNEC All-Direction Pointwise Saturation $\Rightarrow$ Nonlinear Closure; Near-Saturation Stability]
If in neighborhood of $p$ for all local cut surfaces and null directions $k^a$,

$$
(\sqrt h)^{-1}S''_{\rm out}(p;k)=2\pi\,\langle T_{kk}(p)\rangle,
$$

and canonical energy non-negative, De Donder gauge, and $H^s_\delta$ ($s>\tfrac d2+2$, $-1<\delta<0$) integrability lemma hold, then

$$
E^{\rm grav}_{ab}(p)=8\pi G_{\rm ren}\,\langle T^{\rm tot}_{ab}(p)\rangle+\phi(p)\,g_{ab}(p),\qquad \nabla_b\phi=0.
$$

If only $\sup_k\Delta_{\rm QNEC}(p;k)\le\varepsilon$, then there exist constant $C$ and norm $X$ such that

$$
\big|E^{\rm grav}_{ab}-8\pi G_{\rm ren}\langle T^{\rm tot}_{ab}\rangle-\phi g_{ab}\big|_X\le C\,\varepsilon.
$$
\end{theorem}

\section{Preliminaries: Small-Region Geometry, Effective Volume, and Kernel Expansion}

\subsection{Gray--Vanhecke Expansion}

RNC yields

$$
V(B_{p,r})=\Omega_{d-1}\frac{r^d}{d}\Big(1-\frac{R(p)}{6(d+2)}r^2+O(r^4)\Big),\quad
A(\partial B_{p,r})=\Omega_{d-1}r^{d-1}\Big(1-\frac{R(p)}{6d}r^2+O(r^4)\Big).
$$

Define

$$
V^{\rm eff}_\xi(D_{p,r}):=\int_{D_{p,r}}\nabla_a\xi^a\,{\rm dvol},\quad
\delta V^{\rm eff}_\xi=-\int_{D_{p,r}}\delta g^{ab}\nabla_a\xi_b\,{\rm dvol}+O(r^{d+2}).
$$

\subsection{Modular Hamiltonian Local Kernel and Two-Cap Boundary Kernel}

Small-region modular Hamiltonian written as

$$
K_{D_{p,r}}=2\pi\int_{D_{p,r}}T_{ab}\xi^a{\rm d}\Sigma^b+\sum_{\pm}\int_{\mathcal N_\pm}f_\pm\,T_{kk}\,{\rm d}\sigma+O(r^{d+2}),
$$

with $f_\pm=O(r^2)$, under mirror $\mathcal R:\mathcal N_+\to\mathcal N_-$ having $f_+=-f_-\circ\mathcal R$. Theorem J.1 proves boundary term distributionally $O(r^{d+2})$ cancels.

\subsection{Covariant Phase Space Identity and Corners}

For any variation $\delta$ and vector field $\xi$,

$$
\boldsymbol\omega(\delta,\mathcal L_\xi)=\delta \boldsymbol j_\xi-{\rm d}\big(\delta\mathbf Q_\xi-\xi\!\cdot\!\boldsymbol\theta\big),\quad
\boldsymbol j_\xi:=\boldsymbol\theta(\mathcal L_\xi)-\xi\!\cdot\!\mathbf L.
$$

Corner potential $C_\xi$ reorganizes corner total differentials (Appendix F).

\section{Proposition K.1: Fixed $V^{\rm eff}_\xi\Leftrightarrow$Fixed $H_\xi$ (to $O(r^d)$)}

\begin{proposition}[K.1, Precise Version]
If $L=L_{\rm EH}$ or $L=f(R)=R+\alpha R^2$, take §2's approximate CKV $\xi$. Then there exist constants $p=-\Lambda/(8\pi G_{\rm ren})$ and $C=C(d,|{\rm Rm}|_{C^1},C_\xi)$ such that for all solution space tangent vectors $\delta$,

$$
\Big|\delta H_\xi-\frac{\kappa_\xi}{2\pi}\delta S_{\rm grav}+p\,\delta V^{\rm eff}_\xi\Big|\le C\,r^{d+2}\Big(|\delta g|_{C^1(B_{p,2r})}+|\delta\psi|_{H^1}\Big).
$$
\end{proposition}

\begin{proof}[Proof (essentials and constant control)]
\textbf{(i) Bulk--boundary--corner decomposition.} By covariant phase space and Stokes,

$$
\delta H_\xi=\int_{D_{p,r}} \delta\boldsymbol j_\xi+\int_{\partial D_{p,r}}\!\big(\delta\mathbf Q_\xi-\xi\!\cdot\!\boldsymbol\theta-\delta C_\xi\big).
$$

Write $\boldsymbol j_\xi=\sqrt{-g}\,E^{\rm grav}_{ab}\xi^a{\rm d}\Sigma^b+\nabla\!\cdot\!(\dots)$, using $\xi|_{\partial D}=0$ and corner corrections,

$$
\delta H_\xi=\int_{D_{p,r}}\delta(\sqrt{-g}\,E^{\rm grav}_{ab}\xi^a n^b)+\frac{\kappa_\xi}{2\pi}\delta S_{\rm grav}-p\,\delta V^{\rm eff}_\xi+R_{\rm bd}.
$$

Remainder $R_{\rm bd}$ assembled by Theorem J.1 and Appendix F, $|R_{\rm bd}|\le C_1 r^{d+2}|\delta|_{C^1\oplus H^1}$.

\textbf{(ii) EH leading order.} On representative surface $\theta|_p=\sigma|_p=0$, $\int(\theta^2+\sigma^2)=O(r^{d+2})$ suppress second-order geometric terms; bulk term scale $O(r^d)$ times $\delta E^{\rm grav}=O(1)$ yields $O(r^{d+2})$.

\textbf{(iii) $f(R)$ corrections.} Wald entropy $\delta S_{\rm grav}=\tfrac{1}{4G_{\rm ren}}\int_\Sigma \delta(f'(R)\,{\rm d}A)$, RNC and Appendix B yield

$$
\Big|\!\int_{D_{p,r}}\!\delta f'(R)\Big|\le C_2 r^{d+2}|\delta g|_{C^1},\quad
\Big|\!\int_{\Sigma}\!\delta f'(R)\,{\rm d}A\Big|\le C_3 r^{d+1}|\delta g|_{C^1}.
$$

Extrinsic curvature mixed terms suppressed to $O(r^{d+2})$ by representative surface estimates. Combining yields proposition. \qed
\end{proof}

\section{Theorem J.1: Two-Cap Boundary Kernel $O(r^{d+2})$ Distributional Cancellation}

\begin{theorem}[J.1]
Let $\mathcal N_\pm$ be mirror null caps, $f_\pm\in C^{1,\alpha}$ satisfy $f_\pm=O(r^2)$, $f_+=-f_-\circ\mathcal R$. Hadamard state makes $T_{kk}\in\mathcal D'(\mathcal N_\pm)$ restrict along null surface. Then for all $\varphi\in C^1\cap H^1$ there exists constant $C$ such that

$$
\Big|\!\int_{\mathcal N_+}! f_+T_{kk}\varphi+!\int_{\mathcal N_-}! f_-T_{kk}\varphi\Big|
\le C\,r^{d+2}\,|\varphi|_{C^1\cap H^1}.
$$
\end{theorem}

\begin{proof}
Appendix E uses wave-front set and mirror map $\mathcal R$'s $C^1$ deviation $O(r)$ as core, yielding

$$
|T_{kk}-\mathcal R^*T_{kk}|_{H^{-1}}\le C r|T_{kk}|_{H^{-1}},
$$

multiplying by $|f_\pm|_{C^1}=O(r^2)$ and measure scale $O(r^{d-1})$ yields $O(r^{d+2})$ bound. Corner coordination with $C_\xi$ as total differential doesn't elevate order. \qed
\end{proof}

\section{Proof of Theorem A}

\textbf{Second-order equilibrium and null equation.} On representative surface $\hat\Sigma$ under fixed $V^{\rm eff}_\xi$ (or $H_\xi$) constraint,

$$
0=\delta^2 S_{\rm gen}=\delta^2 S_{\rm grav}+\delta^2 S_{\rm out}+\delta^2 S_{\rm ct}.
$$

Proposition K.1 rewrites constraint contribution as $\tfrac{\kappa_\xi}{2\pi}\delta^2 S_{\rm grav}-p\,\delta^2 V^{\rm eff}_\xi+O(r^{d+2})$. Raychaudhuri on $\hat\Sigma$ gives

$$
(\sqrt h)^{-1}\delta^2 A=-R_{kk}+O(r^2),
$$

while QNEC and Theorem J.1 control $\delta^2 S_{\rm out}$. Combining yields

$$
-\frac{1}{4G_{\rm ren}}R_{kk}+2\pi\langle T_{kk}\rangle+O(r^2)=0,
$$

thus

$$
E^{\rm grav}_{kk}=8\pi G_{\rm ren}\langle T^{\rm tot}_{kk}\rangle.
$$

\textbf{Tensorization and constant incorporation.} By Appendix A distributional-level tensorization lemma, there exists $\phi$ such that

$$
E^{\rm grav}_{ab}-8\pi G_{\rm ren}\langle T^{\rm tot}_{ab}\rangle=\phi g_{ab}.
$$

Using $\nabla^aE^{\rm grav}_{ab}\equiv 0$ and $\nabla^a\langle T^{\rm tot}_{ab}\rangle=0$ yields $\nabla_b\phi=0$, incorporated into $\Lambda$. \qed

\section{Proof of Theorem B}

\textbf{Sourceless version.} Covariant phase space identity yields

$$
\frac{{\rm d}}{{\rm d}s}S_{\rm rel}^{\rm bdy}(s)=\int_{\Sigma_s}\boldsymbol\omega(\delta,\mathcal L_\xi\delta)+\int_{\partial\Sigma_s}(\delta\mathbf Q_\xi-\xi\!\cdot\!\boldsymbol\theta-\delta C_\xi).
$$

If foliation-independent with no flux, right side vanishes, yielding $\int_{\Sigma_s}\boldsymbol\omega(\delta,\mathcal L_\xi\delta)=0$. Localization and using $\nabla^aE^{\rm grav}_{ab}=0$ derives

$$
\nabla^a\big(E^{\rm grav}_{ab}-8\pi G_{\rm ren}\langle T^{\rm tot}_{ab}\rangle\big)=0.
$$

\textbf{Sourced version and invariance.} With flux/corner injection define

$$
J_b:=\nabla^a\Big[\delta Q_{\xi,ab}-(\xi^c\theta_{cab})-\delta C_{\xi,ab}\Big],
$$

yielding sourced quantum Bianchi. Appendix F uses cohomology to prove $J_b$ invariant under JKM shifts and corner corrections. \qed

\section{Proof of Theorem C}

\textbf{(1) From all-direction QNEC saturation to linear kernel.} For all local cut surfaces and null directions $k$, QNEC equality and first law yield $\delta^2 S_{\rm rel}^{\rm bdy}= \mathcal E_\xi(\delta,\delta)=0$. Non-negative canonical energy implies kernel equals all physical perturbations, thus all null direction linear constraints are equalities.

\textbf{(2) Nonlinear closure.} Under De Donder gauge and $H^s_\delta$, write nonlinear equation as

$$
h=\mathcal L^{-1}\big(\mathcal S-\mathcal N(h)\big),
$$

Appendix L provides contraction mapping and unique fixed point, thus

$$
E^{\rm grav}_{ab}(g+h)=8\pi G_{\rm ren}\langle T^{\rm tot}_{ab}\rangle+\phi g_{ab}.
$$

\textbf{(3) Near-saturation stability.} If $\sup_k\Delta_{\rm QNEC}\le\varepsilon$, then

$$
\mathcal E_\xi(\delta,\delta)\le C_1\,\varepsilon,
$$

by coercivity and L.1's Lipschitz continuity, obtain

$$
\big|E^{\rm grav}_{ab}-8\pi G_{\rm ren}\langle T^{\rm tot}_{ab}\rangle-\phi g_{ab}\big|_{H^{s-2}_\delta}\le C\,\varepsilon.
$$

\qed

\section{Two-Dimensional Rewrite and Anomaly}

In $d=2$, Einstein tensor degenerates. Employ improved stress

$$
T^{\rm impr}_{\pm\pm}=T_{\pm\pm}-\frac{c}{24\pi}\{\lambda,x^\pm\},
$$

Weyl anomaly only enters trace. Distributional tensorization lemma rewrites: if $X_{++}=X_{--}=0$ and $\nabla^aX_{ab}=\hat J_b$, then $X_{+-}=\phi g_{+-}$, $\partial_\pm\phi=\hat J_\pm$. Thus two-dimensional versions A$'$/B$'$/C$'$ yield

$$
E^{\rm grav}_{ab}=8\pi G_{\rm ren}\langle T^{\rm tot,\rm impr}_{ab}\rangle+\phi g_{ab},\qquad \partial_b\phi=\hat J_b,
$$

sourceless case $\phi$ constant incorporated into $\Lambda$. Bañados/BTZ symmetric cuts achieve saturation; Vaidya scenario exhibits near-saturation satisfying stability inequality (Appendix I).

\section{Examples}

\subsection{FRW}

$$
{\rm d}s^2=-{\rm d}t^2+a^2(t)\gamma_{ij}{\rm d}x^i{\rm d}x^j.
$$

Take radial null direction $k$ through $p$, Theorem A yields $E^{\rm grav}_{kk}=8\pi G_{\rm ren}\langle T^{\rm tot}_{kk}\rangle$. Combining with quantum Bianchi time--space decomposition yields

$$
H^2+\frac{k}{a^2}=\frac{16\pi G_{\rm ren}}{(d-1)(d-2)}\,\rho+\frac{2\Lambda}{(d-1)(d-2)},\qquad
\dot H-\frac{k}{a^2}=-\frac{8\pi G_{\rm ren}}{d-2}\,(\rho+P),
$$

error controlled by $O(r^{d+2})$ (Appendix J).

\subsection{AdS$_3$/CFT$_2$}

On Bañados/BTZ background, symmetric cut families achieve QNEC saturation, C$'$ directly closes. Under AdS--Vaidya, E2 displays $\Delta\mathcal E_\xi$ foliation drift and $\int J_b$ closure; E3 saturation phase diagram shows Hausdorff distance between $\Delta_{\rm QNEC}$ zero set and $E^{\rm grav}_{kk}$ zero set converges as $O(r^2)$ with $r$ (Appendices I, K).

\section{Error Budget and Applicability Window}

\begin{itemize}
\item \textbf{Theoretical error}: Bulk term $O(r^d)$; two-cap boundary kernel and curvature--radius crossing $O(r^{d+2})$; higher-derivative extrinsic curvature corrections don't elevate leading order. Non-CFT requires $mr\ll 1$.
\item \textbf{Numerical error}: Grid scale, second-order difference step, corner discretization and denoising, log-log slope verification $d+2$.
\item \textbf{Applicability window}: Hadamard state, no gravitational anomaly; Weyl anomaly only enters trace (two-dimensional rewrite); coupling domain with non-negative canonical energy; integrability small parameter $\epsilon\sim r/\ell_{\rm curv}$ sufficiently small.
\end{itemize}

\appendix

\section{Distributional-Level Tensorization Lemma (Complete Proof)}

\begin{lemma}[A.1]
$X_{ab}\in\mathcal D'(M)$ symmetric. If for any null direction $n^a$ and $\psi\in C^\infty_0$, $\langle X_{ab}n^a n^b,\psi\rangle=0$, then there exists $\phi\in\mathcal D'(M)$ such that $X_{ab}=\phi g_{ab}$. If $\nabla^a X_{ab}=J_b=j_b{\rm dvol}$, then $\partial_b\phi=j_b$.
\end{lemma}

\begin{proof}
Take local orthonormal frame $g_{ab}={\rm diag}(-1,1,\dots)$. Any null direction $n^a=e_0^a+\hat n^i e_i^a$, $|\hat n|=1$. Pairing formula

$$
0=\langle X_{00}+2\hat n^i X_{0i}+\hat n^i\hat n^j X_{ij},\psi\rangle.
$$

Viewed as quadratic form in $\hat n$ vanishing for all unit vectors. Spherical harmonic decomposition yields linear term $\langle X_{0i},\psi\rangle=0$, quadratic term $\langle X_{ij},\psi\rangle=\lambda\,\delta_{ij}\langle\psi\rangle$, zero-order $\langle X_{00},\psi\rangle=-\lambda\langle\psi\rangle$. Thus $X_{ab}=\lambda\,{\rm diag}(-1,1,\dots)=\phi g_{ab}$. Divergence condition yields $\partial_b\phi=j_b$. \qed
\end{proof}

\section{$f(R)$ and Small-Region Expansion Order Control}

In RNC, $R(x)=R(p)+\partial_cR|_p\,x^c+O(x^2)$. For $f(R)=R+\alpha R^2$, Wald entropy

$$
S_{\rm grav}=\frac{1}{4G_{\rm ren}}\int_\Sigma f'(R)\,{\rm d}A
=\frac{1}{4G_{\rm ren}}\int_\Sigma \big(1+2\alpha R(p)\big)\,{\rm d}A+O(r^{d+1}).
$$

$\delta f'(R)=2\alpha\,\delta R$. Integral estimates

$$
\Big|\!\int_{D_{p,r}}\!\delta R\Big|\le C\,r^{d+2}|\delta g|_{C^1},\qquad
\Big|\!\int_{\Sigma}\!\delta R\,{\rm d}A\Big|\le C\,r^{d+1}|\delta g|_{C^1}.
$$

Extrinsic curvature mixed terms on representative surface suppressed by $\theta|_p=\sigma|_p=0$ and $\int(\theta^2+\sigma^2)=O(r^{d+2})$, not altering $O(r^d)$ leading order. \qed

\section{Small-Region Modular Hamiltonian Kernel and Shape Variation}

Approximate CKV $\xi$ yields shape variation kernel $f_\pm=O(r^2)$ satisfying mirror odd symmetry $f_+=-f_-\circ\mathcal R$. Hadamard condition ensures $\langle T_{kk},f_\pm\varphi\rangle$ well-defined, Theorem J.1 provides $O_{\mathcal D'}(r^{d+2})$ bound. \qed

\section{Scheme Independence and Total Stress Conservation}

Allowed local counterterms only redefine $G_{\rm ren},\Lambda$ and finite higher-derivative couplings, not altering $\nabla^a\langle T^{\rm tot}_{ab}\rangle=0$. Non-local effective action variation defines $\tau^{\rm ent}_{ab}$, whose divergence cancels bulk sources. Thus main equations invariant under equivalence class. \qed

\section{Two-Cap Cancellation Microlocal Analysis (Complete)}

Hadamard two-point function $W$'s wave-front set $WF(W)$ controls $T_{kk}$ distributional restriction along null surface. Mirror map $\mathcal R$'s $C^1$ deviation is $O(r)$, thus

$$
|T_{kk}-\mathcal R^*T_{kk}|_{H^{-1}}\le C r|T_{kk}|_{H^{-1}}.
$$

Multiplying by $|f_\pm|_{C^1}=O(r^2)$ and measure $O(r^{d-1})$, pairing with test function norm $|\varphi|_{C^1\cap H^1}$, yields $O(r^{d+2})$ bound. Corner distributional mass reorganized via $C_\xi$ as total differential, doesn't elevate order. \qed

\section{JKM Shift and Corner Correction Cohomological Invariance}

Variation change is exact form ${\rm d}\beta$. On relative homology class formed by two caps and corners, $\int_{\partial D_{p,r}}{\rm d}\beta=0$. Corner potential $C_\xi$ variation compensated by boundary total differential, preserving $\delta H_\xi$ and $J_b$ invariance. \qed

\section{Quantum Bianchi Source $J_b$ Coordinate-Free Expression and Examples}

Write

$$
J_b=\nabla^a\Big[\delta Q_{\xi,ab}-(\xi^c\theta_{cab})-\delta C_{\xi,ab}\Big],
$$

where $\theta_{cab}$ is symplectic potential double-index pullback. AdS--Vaidya small diamond discrete implementation shows $\Delta\mathcal E_\xi(\Sigma_1\to\Sigma_2)\approx\int J_b$ closes within $O(r^{d+2})$. \qed

\section{Quantum Rest Representative Surface—Existence, Uniqueness, and Construction Algorithm}

In deformation space $C^{2,\alpha}\cap H^2$, take ``quantum expansion'' as nonlinear operator $\mathcal Q$. Background QES satisfies $\mathcal Q(\Sigma)=0$, its Fréchet derivative self-adjoint positive definite. Implicit function theorem yields unique solution family making $\theta|_p=\sigma|_p=0$. Energy estimate

$$
\int_{\hat\Sigma}(\theta^2+\sigma^2)\le C\,r^{d+2}.
$$

Construction employs gradient flow/Newton iteration, Lipschitz constant $\le C\varepsilon$, converging to $\hat\Sigma$. \qed

\section{Two-Dimensional Rewrite, Improved Stress, and Phase Diagram}

In $d=2$ employ improved stress $T^{\rm impr}_{\pm\pm}$, Weyl anomaly enters trace. Two-dimensional tensorization lemma and quantum Bianchi rewrite see main text §10. Bañados/BTZ and Vaidya numerical phase diagrams display $\Delta_{\rm QNEC}$ and $E^{\rm grav}_{kk}$ zero set coincidence degree converges as $O(r^2)$ with $r$. \qed

\section{FRW Null Projection and Friedmann Combination}

Denote $H:=\dot a/a$, $\rho:=\langle T^{\rm tot}_{tt}\rangle$, $P:=a^{-2}\gamma^{ij}\langle T^{\rm tot}_{ij}\rangle/(d-1)$. Null projection

$$
E^{\rm grav}_{kk}=E^{\rm grav}_{tt}+E^{\rm grav}_{rr},
$$

combining with quantum Bianchi time--space decomposition yields main text §11.1's two scalar formulas. Error term $\le C r^{d+2}$ ($C$ depends on $|H|_{C^1}$, $|{\rm Rm}|_{C^0}$). \qed

\section{E2/E3 Minimal Implementation and Error Slopes}

\textbf{Relative entropy flux meter (E2)}
Input: $(g_{ab}),\xi^a,\Sigma_s$ grid; output: $\Delta\mathcal E_\xi$ and $\int J_b$. Steps: generate null leaf foliation; discretize $\boldsymbol\omega,\mathbf Q_\xi,\boldsymbol\theta,C_\xi$; foliation difference; report $\log$--$\log$ slope (target $d+2$).

\textbf{Saturation phase diagram (E3)}
Input: cut surface family and second-order difference step; output: $\Delta_{\rm QNEC}(x,k)$ and zero set coincidence (Hausdorff/Jaccard). Stability: vary step and filter strength, identify plateau regions, estimate constant $C$. \qed

\section{Integrability Lemma L.1 Energy Estimates and Contraction Mapping}

De Donder gauge and $H^s_\delta$, $s>\tfrac d2+2$, $-1<\delta<0$. Linearized operator $\mathcal L$ satisfies

$$
|\mathcal L^{-1}F|_{H^s_\delta}\le C|F|_{H^{s-2}_\delta}.
$$

Nonlinearity satisfies Moser-type estimate

$$
|\mathcal N(h_1)-\mathcal N(h_2)|_{H^{s-2}_\delta}\le C\big(|h_1|_{H^s_\delta}+|h_2|_{H^s_\delta}\big)|h_1-h_2|_{H^s_\delta}.
$$

When $|\mathcal S|_{H^{s-2}_\delta}\le C^{-1}\rho$, $\rho\ll 1$, $\mathcal T(h)=\mathcal L^{-1}(\mathcal S-\mathcal N(h))$ is contraction, unique fixed point exists. Coercivity constant from non-negative canonical energy, derives ``near saturation $\Rightarrow$ near equation'' Lipschitz bound. \qed

\section{Error Budget Summary Table}

\begin{itemize}
\item \textbf{Bulk term}: $O(r^d)$, constant $C_1=C_1(d,|{\rm Rm}|_{C^0})$.
\item \textbf{Two-cap kernel}: $O(r^{d+2})$, constant $C_2=C_2(d,C_\xi,\alpha)$.
\item \textbf{Curvature--radius crossing}: $O(r^{d+2})$, constant $C_3=C_3(d,|{\rm Rm}|_{C^1})$.
\item \textbf{Higher-derivative extrinsic curvature}: $O(r^{d+2})$, constant $C_4=C_4(d,|{\rm Rm}|_{C^0})$.
\item \textbf{Numerical discretization}: Grid $h$, step $\delta\lambda$ slopes $O(h^2)+O(\delta\lambda^2)$.
\end{itemize}

\end{document}

