\documentclass[11pt]{article}
\usepackage[utf8]{inputenc}
\usepackage[T1]{fontenc}
\usepackage{amsmath,amssymb,amsthm}
\usepackage{mathtools}
\usepackage{geometry}
\geometry{margin=1in}
\usepackage{hyperref}
\usepackage{cite}
\usepackage{braket}
\usepackage{graphicx}

\newtheorem{theorem}{Theorem}
\newtheorem{lemma}[theorem]{Lemma}
\newtheorem{proposition}[theorem]{Proposition}
\newtheorem{corollary}[theorem]{Corollary}
\theoremstyle{definition}
\newtheorem{definition}[theorem]{Definition}
\newtheorem{assumption}[theorem]{Assumption}
\theoremstyle{remark}
\newtheorem{remark}[theorem]{Remark}

\title{Unified Role of Relative Scattering Determinant in Quantum Gravity: Two-Domain Framework, Fixed-Energy BK ($p\in\{1,2\}$ Unified Version), Closed-Domain $\Lambda$-Slope, and Black Hole Pole Spectroscopy}

\author{Haobo Ma$^1$ \and Wenlin Zhang$^2$\\
\small $^1$Independent Researcher\\
\small $^2$National University of Singapore}

\date{}

\begin{document}

\maketitle

\begin{abstract}
Taking ``relative determinant'' as unified object, we establish rigorous and verifiable theory in two types of geometric-physical scenarios: $(\mathbf{C})$ relative $\zeta$/heat kernel determinant for Euclideanized second variation operator family in compact closed domain and its volume density response to cosmological constant term; $(\mathbf{S})$ fixed-frequency scattering matrix on stationary exterior geometry (Schwarzschild--de Sitter/Kerr--de Sitter) with relative (modified) determinant, spectral shift object, and quasinormal mode (QNM) spectroscopy. This paper provides four main theorems with complete proofs: (i) Under control of weighted limiting absorption principle (LAP) and double operator integral (DOI), prove $p\in\{1,2\}$ unified version of fixed-energy Birman--Kreĭn equality: for Lebesgue almost everywhere frequency $\omega$, $\det_p S_\Lambda(\omega)=\exp\!\big(-2\pi i\,\Xi^{(p)}_\Lambda(\omega)\big)$, where $p=1$ gives $\Xi^{(1)}=\xi$ as Lifshits--Kreĭn spectral shift function, $p=2$ yields $\Xi^{(2)}$ as cumulative antiderivative of Koplienko second-order spectral shift; (ii) In closed-domain Müller relative determinant framework, prove ``volume slope theorem'': $\lim_{\mu\to 0^+}\mathrm{Vol}_4(M)^{-1}\,\partial_\Lambda \Re\log\det_{\zeta,\mathrm{rel}}(\mathcal{K}_\Lambda+\mu^2,\mathcal{K}_0+\mu^2)=\tfrac{1}{8\pi G}$ (per signature convention fixed in text); (iii) On physical strip $\Im\omega>-\,\gamma_0$, pole set of relative scattering determinant $\tau_p(\omega)=\det_p\!\big(S(\omega)S_0(\omega)^{-1}\big)$ equivalent to QNM (counting algebraic multiplicity), independent of reference $S_0$ choice; (iv) For real frequency only ``phase'' admits equality, $\arg\det_p S=-2\pi\,\Xi^{(p)}$; while for $p=2$ Carleman determinant $|\det_2 S|=\exp\!\big(\sum_j(1-\cos\theta_j)\big)\ge 1$, generally cannot claim $|\det_2 S|=1$. Accordingly introduce ``phase-normalized determinant'' $\widehat{\det}_p S:=\det_p S/|\det_p S|$ as constrained object for frequency-domain ``globally meromorphic fitting'', providing principal angle upper bound for Fisher information. Paper concludes with parameters and acceptance standards for three reproducible experimental pipelines: closed-domain rel-zeta, exterior-domain meromorph-fit, channel--pseudo-unitary verification.
\end{abstract}

\section{Introduction}

Closed-domain relative $\zeta$/heat kernel determinant and exterior-domain relative scattering determinant share essential structure—``relative phase''. On closed-domain side, this phase recovers on-shell action's volume density response to cosmological constant $\Lambda$ via logarithmic derivative; on exterior-domain side, it's controlled by BK/LK (and second-order Koplienko version) spectral shift function to scattering matrix phase on energy fibers. This paper unifies two domains into closed loop of ``verifiable hypothesis $\Rightarrow$ theorem $\Rightarrow$ detailed proof'':

\begin{enumerate}
\item Achieve fixed-energy implementation under operator-Lipschitz and DOI techniques, with weighted LAP dominated convergence;
\item Implement regularization independence and item-wise cancellation of corner/boundary/ghost under Müller relative determinant;
\item Unify relative scattering determinant poles as QNM under analytic Fredholm framework, proving reference independence;
\item Separate ``block-level modulus conservation'' from ``global Carleman modulus non-constant identity'' under pseudo-unitary (J-unitary) framework, imposing real-axis modulus constraint via ``phase-normalized determinant''.
\end{enumerate}

All mathematical expressions in text presented inline with $\cdot$ form, avoiding ambiguity from display/environment switching.

\section{Setting, Notation, and Verifiable Hypotheses}

\subsection{Spectrum, Ideals, and Modified Determinants}

Take separable Hilbert space $\mathcal{H}$. Denote self-adjoint operator pair $(H_\Lambda,H_0)$, difference $V=H_\Lambda-H_0$. Schatten ideal $\mathfrak{S}_p$ standard definition. For $K\in\mathfrak{S}_1$ take Fredholm determinant $\det(I+K)$; for $K\in\mathfrak{S}_2$ take Carleman determinant $\det_2(I+K)=\det\!\big((I+K)\exp(-K)\big)$. If $U$ unitary with $U-I\in\mathfrak{S}_2$, spectral angles $\{\theta_j\}\in\ell^2$ satisfy $|\det_2(U)|=\exp\!\big(\sum_j(1-\cos\theta_j)\big)\ge 1$, $\arg\det_2(U)=\sum_j(\theta_j-\sin\theta_j)$.

\subsection{Spectral Shift Objects and DOI}

First-order spectral shift $\xi$ and second-order spectral shift measure $\eta$ respectively satisfy $\mathrm{Tr}(f(H_\Lambda)-f(H_0))=\int f'(E)\,\xi(E)\,dE$, $\mathrm{Tr}(f(H_\Lambda)-f(H_0)-f'(H_0)V)=\int f''(E)\,d\eta(E)$, function class taking operator-Lipschitz/appropriate Besov intersection. Cumulative antiderivative $\Xi^{(2)}(E)=\eta((-\infty,E))$, normalized $\Xi^{(2)}(-\infty)=0$. Double operator integral representation $f(H_\Lambda)-f(H_0)=\iint \Phi_f(\lambda,\mu)\,dE_\Lambda(\lambda)\,V\,dE_0(\mu)$, where $\Phi_f(\lambda,\mu)=(f(\lambda)-f(\mu))/(\lambda-\mu)$ has Schur/Haagerup bound.

\subsection{Weighted LAP and Energy Fiberization}

There exist $s>\tfrac{1}{2}$, energy window $I$, constant $C_I$ such that $|\langle x\rangle^{-s}(H_\#-\lambda\mp i0)^{-1}\langle x\rangle^{-s}|\le C_I$ holds for $\lambda\in I$, $\#\in\{\Lambda,0\}$. Stationary exterior region (SdS/KdS) stationary under time Killing field with frequency $\omega$, partial wave decomposition yields channel matrix $S_{\ell m}(\omega)$.

\subsection{Closed-Domain Relative $\zeta$-Determinant and Volume Slope}

Euclideanized second variation operator family $\mathcal{K}_\Lambda$ with reference $\mathcal{K}_0$ matching principal symbol, boundary conditions and Faddeev--Popov ghost pairing consistent, zero modes/threshold resonances removed via deprojection. Difference heat kernel $K_{\mathrm{rel}}(t)=\mathrm{Tr}\big(e^{-t(\mathcal{K}_\Lambda+\mu^2)}-e^{-t(\mathcal{K}_0+\mu^2)}\big)$ has short-time expansion, define $\log\det_{\zeta,\mathrm{rel}}(\mathcal{K}_\Lambda+\mu^2,\mathcal{K}_0+\mu^2)=-\int_0^\infty t^{-1}K_{\mathrm{rel}}(t)\,dt$. Metric signature and action convention fixed as $\partial_\Lambda S_{\mathrm{on\text{-}shell}}=(8\pi G)^{-1}\mathrm{Vol}_4(M)$.

\subsection{Exterior-Domain Reference and Pseudo-Unitary}

On strip $\Im\omega\in(-\gamma_0,0]$ choose reference scattering matrix $S_0(\omega)$, require analyticity on same sheet without zero/poles. Each channel constructs energy flux quadratic form $\eta$ via Jost--Wronskian normalization making $S_{\ell m}^\dagger\eta S_{\ell m}=\eta$.

\subsection{Verifiable Hypotheses (Assumption Box)}

$(\mathrm{H\text{-}AC})$: Wave operators exist and complete, AC part admits energy fiberization;

$(\mathrm{H\text{-}LAP})$: Weighted LAP (parameter $s>1/2$, constant $C_I$);

$(\mathrm{H\text{-}LK/DOI})$: Poisson smoothing $f_\varepsilon\in\mathrm{OL}$, $|f_\varepsilon|_{\mathrm{OL}}\le C/\varepsilon$, DOI kernel has uniform Schur/Haagerup bound;

$(\mathrm{H\text{-}S}_p)$: For a.e. $\omega\in I$, $\chi_{(-\infty,\omega]}(H_\Lambda)-\chi_{(-\infty,\omega]}(H_0)\in\mathfrak{S}_p$ and $S_\Lambda(\omega)S_0(\omega)^{-1}-I\in\mathfrak{S}_p$ (typical $p=2$);

$(\mathrm{H\text{-}relDet})$: Principal symbol consistent, boundary/ghost matching, no zero modes or dereso'd, difference heat kernel has short-time expansion;

$(\mathrm{H\text{-}Ref})$: Reference $S_0$ analytic on strip without zero/poles;

$(\mathrm{H\text{-}Can})$: Channel energy flux gauge fixed, block-level pseudo-unitary holds.

\section{Main Theorems and Conclusions}

\begin{theorem}[3.1: Fixed-Energy BK: $p\in\{1,2\}$ Unified Version]
Under $(\mathrm{H\text{-}AC})$, $(\mathrm{H\text{-}LAP})$, $(\mathrm{H\text{-}LK/DOI})$, $(\mathrm{H\text{-}S}_p)$, for Lebesgue almost everywhere $\omega\in I$: when $p=1$, $\det S_\Lambda(\omega)=\exp\big(-2\pi i\,\xi_\Lambda(\omega)\big)$; when $p=2$, $\det_2 S_\Lambda(\omega)=\exp\big(-2\pi i\,\Xi^{(2)}_\Lambda(\omega)\big)$. Thus $\arg\det_p S_\Lambda(\omega)=-2\pi\,\Xi^{(p)}_\Lambda(\omega)$.
\end{theorem}

\begin{theorem}[3.2: Closed-Domain ``Volume Slope'']
Under $(\mathrm{H\text{-}relDet})$ and deresonance projection, $\lim_{\mu\to 0^+}\mathrm{Vol}_4(M)^{-1}\,\partial_\Lambda \Re\log\det_{\zeta,\mathrm{rel}}(\mathcal{K}_\Lambda+\mu^2,\mathcal{K}_0+\mu^2)=\tfrac{1}{8\pi G}$ (per text signature convention).
\end{theorem}

\begin{theorem}[3.3: $\tau_p$ Poles = QNM, Reference Independent]
Let $\tau_p(\omega)=\det_p\big(S(\omega)S_0(\omega)^{-1}\big)$. Under $(\mathrm{H\text{-}Ref})$, on strip $\Im\omega\in(-\gamma_0,0]$, pole set of $\tau_p$ coincides with $S$ poles (QNM) counting algebraic multiplicity. If changing reference to $\widetilde{S}_0$ still satisfying $(\mathrm{H\text{-}Ref})$, then $\tau_p/\widetilde{\tau}_p$ is analytic outer function without zeros/poles on strip, leaving pole set unchanged.
\end{theorem}

\begin{theorem}[3.4: Real-Frequency Phase and Modulus; Phase-Normalized Determinant]
Block-level: If $S_{\ell m}^\dagger(\omega)\eta S_{\ell m}(\omega)=\eta$, then $|\det S_{\ell m}(\omega)|=1$. Global: generally only $\arg\det_p S(\omega)=-2\pi\,\Xi^{(p)}(\omega)$ holds. When $S(\omega)$ unitary with $S(\omega)-I\in\mathfrak{S}_2$, $|\det_2 S(\omega)|=\exp\big(\sum_j(1-\cos\theta_j(\omega))\big)\ge 1$. Define $\widehat{\det}_p S(\omega)=\det_p S(\omega)/|\det_p S(\omega)|$, $\widehat{\tau}_p(\omega)=\tau_p(\omega)/|\tau_p(\omega)|$ as real-axis ``modulus equals 1'' constrained objects.
\end{theorem}

\section{Proof of Theorem 3.1 (DOI--LAP Dominated Convergence to Fixed Energy)}

\textbf{Proof strategy overview}: Approximate step function via Poisson smoothing $f_\varepsilon$, apply DOI expression with weighted LAP establishing uniform $\mathfrak{S}_p$ domination inequality, then exchange limit $\varepsilon\downarrow 0$ at Lebesgue points of spectral shift object, finally identify scattering phase via AC fiberization and exponentiate to determinant equality. Difference between $p=1$ and $p=2$ carried by first/second-order trace formulas.

\textbf{Step 1 (Poisson smoothing and DOI kernel bound)}: Take $f_\varepsilon(\lambda)=\tfrac{1}{2}+\tfrac{1}{\pi}\arctan((\omega-\lambda)/\varepsilon)$. Then $f_\varepsilon\in\mathrm{OL}$ with $|f_\varepsilon|_{\mathrm{OL}}\le C/\varepsilon$. DOI expression $f_\varepsilon(H_\Lambda)-f_\varepsilon(H_0)=\iint \Phi_{f_\varepsilon}(\lambda,\mu)\,dE_\Lambda(\lambda)\,V\,dE_0(\mu)$, where $|\Phi_{f_\varepsilon}|_{\mathrm{Schur}}\le C/\varepsilon$.

\textbf{Step 2 (Weighted LAP and Schatten domination)}: Write weighted projection boundary value resolvent form via Stone formula, apply $(\mathrm{H\text{-}LAP})$ yielding $|\langle x\rangle^{-s}R_\#(\omega\pm i0)\langle x\rangle^{-s}|\le C_I$. By Birman--Solomyak type estimate obtain
$|f_\varepsilon(H_\Lambda)-f_\varepsilon(H_0)|_{\mathfrak{S}_p}\le C_I(C/\varepsilon)\,M_p(I)$, where $M_p(I)=\sup_{\lambda\in I}|\langle x\rangle^{-s}(R_\Lambda(\lambda\pm i0)-R_0(\lambda\pm i0))\langle x\rangle^{-s}|_{\mathfrak{S}_p}$ bounded.

\textbf{Step 3 ($p=1$: spectral shift and BK)}: First-order trace formula yields $\mathrm{Tr}(f_\varepsilon(H_\Lambda)-f_\varepsilon(H_0))=\int f_\varepsilon'(E)\,\xi(E)\,dE$. Taking $\varepsilon\downarrow 0$ with dominated convergence yields $\mathrm{Tr}(\chi_{(-\infty,\omega]}(H_\Lambda)-\chi_{(-\infty,\omega]}(H_0))=\xi(\omega)$ at Lebesgue points of $\omega$. AC fiberization with stationary scattering shows $\det S(\omega)=\exp(-2\pi i\,\xi(\omega))$.

\textbf{Step 4 ($p=2$: Koplienko phase)}: Second-order trace formula yields $\mathrm{Tr}\big(f_\varepsilon(H_\Lambda)-f_\varepsilon(H_0)-f'_\varepsilon(H_0)V\big)=\int f_\varepsilon''(E)\,d\eta(E)$. Integrate right side twice by parts, taking $\varepsilon\downarrow 0$ yields $\Xi^{(2)}(\omega)=\eta((-\infty,\omega))$. Fixed-energy implementation same as above, thus $\det_2 S(\omega)=\exp(-2\pi i\,\Xi^{(2)}(\omega))$. QED.

\section{Proof of Theorem 3.2 (Relative Heat Kernel Item-Wise Cancellation, Tauberian Exchange, and Signature Convention)}

\textbf{Step 1 (Logarithmic derivative heat kernel representation)}: $\partial_\Lambda \log\det_{\zeta,\mathrm{rel}}=-\int_0^\infty t^{-1}\,\partial_\Lambda K_{\mathrm{rel}}(t)\,dt$, where $K_{\mathrm{rel}}(t)=\mathrm{Tr}(e^{-t(\mathcal{K}_\Lambda+\mu^2)}-e^{-t(\mathcal{K}_0+\mu^2)})$.

\textbf{Step 2 (Short-time expansion and item-wise cancellation)}: Under principal symbol consistency, boundary/ghost pairing consistency, multiplicative anomaly vanishing, $K_{\mathrm{rel}}(t)\sim \sum_{k\ge 0} a_k^{\mathrm{rel}}\,t^{(k-d)/2}$ ($d=4$), local coefficients (including GHY, corners, ghost pairing) cancel item-wise except volume term $a_0^{\mathrm{rel}}$, i.e., $a_{k>0}^{\mathrm{rel}}=0$.

\textbf{Step 3 (Tauberian exchange and volume slope)}: Introduce small mass $\mu>0$ controlling large $t$ part, split $\int_0^\infty=\int_0^{t_0}+\int_{t_0}^{\infty}$. Former dominated by $a_0^{\mathrm{rel}}$, latter under deresonance projection has uniform bound. Exchanging $\mu\downarrow 0$ with volume density limit yields $\mathrm{Vol}_4^{-1}\,\partial_\Lambda \Re\log\det_{\zeta,\mathrm{rel}}=\partial_\Lambda c_0$. By text convention $\partial_\Lambda S_{\mathrm{on\text{-}shell}}=(8\pi G)^{-1}\mathrm{Vol}_4$, alignment yields $\partial_\Lambda c_0=\tfrac{1}{8\pi G}$. QED.

\section{Proof of Theorem 3.3 (Analytic Fredholm and Reference Independence)}

\textbf{Step 1 (Analytic Fredholm)}: On strip $\Im\omega\in(-\gamma_0,0]$, write $S(\omega)=I+K(\omega)$ where $K(\omega)$ is $\mathfrak{S}_p$-valued meromorphic family. Determinant $\mathcal{D}_p(\omega)=\det_p(I+K(\omega))$ meromorphic, its zero order equals kernel dimension (algebraic multiplicity) of $I+K(\omega)$.

\textbf{Step 2 (Relativization and pole counting)}: Define $\tau_p(\omega)=\det_p(S(\omega)S_0(\omega)^{-1})$. If $S_0$ analytic nonzero on strip, $\tau_p$ shares poles and orders with $S$, poles being QNM.

\textbf{Step 3 (Reference independence)}: If choosing another $\widetilde{S}_0$ also satisfying condition, then $\tau_p/\widetilde{\tau}_p=\det_p(S_0\widetilde{S}_0^{-1})$ is analytic outer function without zeros/poles, leaving pole set unchanged. QED.

\section{Proof of Theorem 3.4 (Block-Level Pseudo-Unitary and Global Carleman Modulus)}

\textbf{Block level}: By $S_{\ell m}^\dagger\eta S_{\ell m}=\eta$ and $\det(\eta^{-1}S_{\ell m}^\dagger\eta S_{\ell m})=1$ obtain $|\det S_{\ell m}|=1$.

\textbf{Global phase}: By Theorem 3.1 obtain $\arg\det_p S(\omega)=-2\pi\,\Xi^{(p)}(\omega)$.

\textbf{Global modulus ($p=2$)}: If $S(\omega)$ unitary with $S(\omega)-I\in\mathfrak{S}_2$, spectral angles $\{\theta_j(\omega)\}\in\ell^2$ yield $|\det_2 S(\omega)|=\exp\big(\sum_j(1-\cos\theta_j(\omega))\big)\ge 1$. In general J-unitary case modulus non-constant, thus phase-normalization $\widehat{\det}_p$ natural object for real-axis modulus constraint. QED.

\section{Globally Meromorphic Fitting and Fisher Projection Geometry (For Data-Side Implementation)}

On strip $\Im\omega\in[-\gamma_0,0]$ parametrize $\log\widehat{\tau}_p(\omega)=\sum_{j=1}^{J}\log\frac{\omega-\omega_j}{\omega-\overline{\omega_j}}+iQ(\omega)$, where $\omega_j$ are lower half-plane poles, $Q$ low-order entire function taking purely imaginary values on real axis. Enforce conjugate pairing and ``phase-normalized modulus equals 1'', suppress false poles via strip cross-validation.

\begin{proposition}[8.1: Fisher Principal Angle Upper Bound]
For whitened observation $y_k=\Im\log\widehat{\tau}_p(\omega_k)+\epsilon_k$, Jacobian $J$ with constraint submanifold tangent space projection $P_\mathcal{M}$ yields restricted Fisher $F_\mathcal{M}=(P_\mathcal{M} J)^\top(P_\mathcal{M} J)$. If $\vartheta$ is maximum principal angle between $\mathrm{range}(J)$ and $\mathrm{range}(P_\mathcal{M})$, then variance reduction factor $\mathcal{R}\le 1/|\sin\vartheta|$. Proof in Appendix F.
\end{proposition}

\section{Reproducible Experimental Protocols (P1--P3)}

\textbf{P1 $|$ rel-zeta (closed domain)}: Grid step $h$ (three levels), heat kernel window $t\in[t_{\min},t_{\max}]$ ($t_{\min}\sim c\,h^2$), extrapolation order $N\in\{2,3\}$, small mass $\mu$ (three to five logarithmic points). Target quantity $\mathrm{Vol}_4^{-1}\,\partial_\Lambda \Re\log\det_{\zeta,\mathrm{rel}}$. Acceptance: slope error $<1\%$; drift under different corner triangulations/gauges $<0.5\%$.

\textbf{P2 $|$ meromorph-fit (exterior domain)}: Fit $\log\widehat{\tau}_p$ recovering $\{\omega_j\}$. Priors: pairwise symmetric, strip analytic, real-axis modulus constraint (on $\widehat{\tau}_p$), and $\Re\log\det_2(\omega)\ge 0$ (if using $p=2$). Acceptance: CRLB improvement over mode-by-mode $\ge 1.3\times$; false alarm rate $\le 5\%$; cross-strip consistent.

\textbf{P3 $|$ bh-channels (pseudo-unitary and BK phase)}: Jost--Wronskian normalization constructs $\eta$, compute $|S_{\ell m}^\dagger\eta S_{\ell m}-\eta|$ and phase closure $\arg\widehat{\det}_p S +2\pi\Xi^{(p)}$. Acceptance: pseudo-unitary residual $<10^{-12}$, phase closure $<10^{-3}$ radians; converges as $a+b/\ell_{\max}$ with $\ell_{\max}$.

\section{Discussion and Outlook}

Under explicitly verifiable analytic hypotheses, this paper completes four main conclusions: $p\in\{1,2\}$ unified version of fixed-energy BK, closed-domain volume slope, reference independence of relative scattering determinant poles = QNM, and real-frequency phase--modulus decomposition, providing reproducible experimental pipelines. Limitations: LAP constant may deteriorate under strong trapping or extreme spin; non-local boundaries and singular geometry require separate verification of multiplicative anomaly; statistical side needs robust regularization against model bias. Future work includes: extending modulus--phase formula for $\det_p$ under Kreĭn spaces; seamlessly incorporating BK version for differential forms/electromagnetic fields; testing stability of ``reference independent'' poles using multi-station strip data.

\appendix

\section{Complete Derivation of DOI--LAP Dominated Convergence}

\subsection{Kernel Bound and Weight Insertion}

Take $f_\varepsilon(\lambda)=\tfrac{1}{2}+\tfrac{1}{\pi}\arctan\!\frac{\omega-\lambda}{\varepsilon}$. $f_\varepsilon\in\mathrm{OL}$, $|f_\varepsilon|_{\mathrm{OL}}\le C/\varepsilon$. DOI expression yields
$f_\varepsilon(H_\Lambda)-f_\varepsilon(H_0)=\iint \Phi_{f_\varepsilon}(\lambda,\mu)\,dE_\Lambda(\lambda)\,V\,dE_0(\mu)$, where $\Phi_{f_\varepsilon}$ satisfies $\sup_\lambda\int |\Phi_{f_\varepsilon}(\lambda,\mu)|\,d\mu\le C/\varepsilon$, $\sup_\mu\int |\Phi_{f_\varepsilon}(\lambda,\mu)|\,d\lambda\le C/\varepsilon$.

Insert $\langle x\rangle^{\pm s}$ obtaining
$f_\varepsilon(H_\Lambda)-f_\varepsilon(H_0)=\iint (\langle x\rangle^{-s}dE_\Lambda(\lambda))\,(\langle x\rangle^{s}V\langle x\rangle^{s})\, (dE_0(\mu)\langle x\rangle^{-s})\,\Phi_{f_\varepsilon}(\lambda,\mu)$.

\subsection{Schatten Domination Inequality}

By Haagerup/Schur bound with Hölder inequality (on $\mathfrak{S}_p$),
$|f_\varepsilon(H_\Lambda)-f_\varepsilon(H_0)|_{\mathfrak{S}_p}\le |\Phi_{f_\varepsilon}|_{\mathrm{Schur}}\cdot \sup_{\lambda\in I}|\langle x\rangle^{-s}E'_\Lambda(\lambda)\langle x\rangle^{-s}|\cdot |\langle x\rangle^{s}V\langle x\rangle^{s}|_{\mathfrak{S}_p}\cdot \sup_{\mu\in I}|\langle x\rangle^{-s}E'_0(\mu)\langle x\rangle^{-s}|$.

Use Stone formula $E'_\#(\lambda)=\pi^{-1}\Im R_\#(\lambda+i0)$ with $(\mathrm{H\text{-}LAP})$ obtaining
$|f_\varepsilon(H_\Lambda)-f_\varepsilon(H_0)|_{\mathfrak{S}_p}\le C_I\,\varepsilon^{-1}\,|\langle x\rangle^{s}V\langle x\rangle^{s}|_{\mathfrak{S}_p}$.

In scattering setting better use difference resolvent form
$|\langle x\rangle^{-s}(R_\Lambda(\lambda\pm i0)-R_0(\lambda\pm i0))\langle x\rangle^{-s}|_{\mathfrak{S}_p}$
controlling $|\langle x\rangle^{s}V\langle x\rangle^{s}|_{\mathfrak{S}_p}$, thus obtaining unified domination
$|f_\varepsilon(H_\Lambda)-f_\varepsilon(H_0)|_{\mathfrak{S}_p}\le C_I\,\varepsilon^{-1}\,M_p(I)$.

\subsection{Lebesgue Points and Limit Exchange}

Let $\omega$ be Lebesgue point of spectral shift object. By above obtain family of $\varepsilon$-uniformly integrable dominating function $M(\omega)=\sup_{\varepsilon<\varepsilon_0}|f_\varepsilon(H_\Lambda)-f_\varepsilon(H_0)|_{\mathfrak{S}_p} \in L^1_{\mathrm{loc}}(I)$. Thus limit of DOI-trace as $\varepsilon\downarrow 0$ commutes with local integration over $\omega$, obtaining fixed-energy version of first/second-order trace formula, completing Theorem 3.1 proof.

\section{Relative Heat Kernel Item-Wise Cancellation and $\Lambda$-Slope Refinement}

\subsection{Short-Time Expansion and Local Coefficients}

For Laplace-type operator $\mathcal{K}_\#$ (including gauge-ghost pairing) have $\mathrm{Tr}(e^{-t\mathcal{K}_\#})\sim \sum_{j\ge 0} a_j(\mathcal{K}_\#)\,t^{(j-d)/2}$. Under principal symbol consistency with boundary/ghost matching, relative difference $a_{j>0}^{\mathrm{rel}}=a_j(\mathcal{K}_\Lambda)-a_j(\mathcal{K}_0)=0$; corner and boundary term coefficients also cancel in ``relative difference'' (Wodzicki residue zero ensures multiplicative anomaly absent).

\subsection{Logarithmic Derivative and Volume Term}

Relative $\zeta$ written as $\zeta_{\mathrm{rel}}(s;\mu)=\tfrac{1}{\Gamma(s)}\int_0^\infty t^{s-1}e^{-t\mu^2}\,K_{\mathrm{rel}}(t)\,dt$. Differentiating with respect to $\Lambda$, only $a_0^{\mathrm{rel}}=\mathrm{Vol}_4(M)c_0$ contributes, yielding
$\partial_\Lambda \zeta'_{\mathrm{rel}}(0;\mu)=-\partial_\Lambda a_0^{\mathrm{rel}}\int_0^\infty t^{-1}e^{-t\mu^2}\,dt +\text{finite terms}$.
Volume density with $\mu\downarrow 0$ exchange, by text convention $\partial_\Lambda c_0=\tfrac{1}{8\pi G}$, obtaining Theorem 3.2.

\subsection{Tauberian Exchange Error Estimate}

Take $t_0=\mu^{-2\alpha}$ ($\alpha\in(0,1)$), $\int_{t_0}^\infty t^{-1}e^{-t\mu^2}\,K_{\mathrm{rel}}(t)\,dt$ controlled by spectral gap and deresonance projection as $\mathcal{O}(\mu^{2(1-\alpha)})$, while $(0,t_0)$ segment error after higher-order coefficient cancellation becomes $\mathcal{O}(t_0^{1/2})=\mathcal{O}(\mu^{-\alpha})$ coefficient nullification term, overall can take $\alpha$ making total error $o(1)$.

\section{Reference Independence of Relative Scattering Determinant and Pole Counting}

\subsection{Analytic Fredholm Tools}

Write $S(\omega)=I+K(\omega)$, $K(\omega)$ being $\mathfrak{S}_p$-valued meromorphic family. Then $\mathcal{D}_p(\omega)=\det_p(I+K(\omega))$ meromorphic with zero order equal to $\dim\ker(I+K(\omega))$.

\subsection{Relativization and Pole Transfer}

Assume $S_0$ analytic without zeros/poles on strip, define $\tau_p(\omega)=\det_p(S(\omega)S_0(\omega)^{-1})$=$\det_p(I+K(\omega))\cdot \det_p(S_0(\omega)^{-1})$. Latter factor analytic nonzero, thus $\tau_p$ poles synchronize with $\mathcal{D}_p$, order being QNM algebraic multiplicity.

\subsection{Reference Independent Outer Function Factor}

If changing reference to $\widetilde{S}_0$, then $\tau_p/\widetilde{\tau}_p=\det_p(S_0\widetilde{S}_0^{-1})$ is analytic without zeros/poles (outer function), leaving pole set and multiplicity unchanged.

\section{Pseudo-Unitary: Channel Construction and Global Carleman Modulus}

\subsection{Energy Flux Quadratic Form and J-Unitary}

Take Jost solutions $u_{\mathrm{in/out}}$ at both ends of radial equation with Wronskian normalization, making energy flux $\mathcal{F}=\Im(\overline{u}\,\partial_r u)$ consistent at both ends. Accordingly define channel quadratic form $\eta=\mathrm{diag}(1,-1)$ making $S_{\ell m}^\dagger\eta S_{\ell m}=\eta$.

\subsection{Block-Level Determinant Unit Modulus and Global Phase}

Finite-dimensional block directly yields $|\det S_{\ell m}|=1$. Global direct sum under $\mathfrak{S}_2$ setting only preserves phase equality; if globally unitary with $S-I\in\mathfrak{S}_2$, spectral angle expansion yields $|\det_2 S|=\exp(\sum(1-\cos\theta_j))\ge 1$.

\section{Koplienko Phase Fixed-Energy Construction ($p=2$)}

\subsection{Second-Order Trace Formula and DOI}

For $f_\varepsilon$ have $\mathrm{Tr}\big(f_\varepsilon(H_\Lambda)-f_\varepsilon(H_0)-f'_\varepsilon(H_0)V\big)=\int f_\varepsilon''(E)\,d\eta(E)$. Integrate right side twice by parts yielding $-\int f_\varepsilon'(E)\,d\Xi^{(2)}(E)$.

\subsection{Dominated Convergence and Lebesgue Points}

By Appendix A's $\mathfrak{S}_2$ domination with $|f'_\varepsilon|_{L^1}\le C$ obtain integrable domination, as $\varepsilon\downarrow 0$, $f'_\varepsilon$ converges to $\delta_\omega$ (weak sense) recovering $\Xi^{(2)}(\omega)$ at Lebesgue points.

\subsection{Scattering Phase and Determinant}

Fixed-energy implementation with AC fiberization identifies $\Xi^{(2)}(\omega)$ as scattering phase second-order spectral shift antiderivative, exponentiating yields $\det_2 S(\omega)=\exp(-2\pi i\,\Xi^{(2)}(\omega))$.

\section{Proof of Fisher Projection Geometry Upper Bound}

\subsection{Model and Projection}

Whitened observation $y=J\theta+\epsilon$, hard constraint $C(\theta)=0$ defining differentiable submanifold $\mathcal{M}$ with tangent space projection $P_\mathcal{M}$ satisfying $P_\mathcal{M}^2=P_\mathcal{M}$.

\subsection{Principal Angle and Spectral Bound}

Let $\vartheta$ be maximum principal angle between $\mathrm{range}(J)$ and $\mathrm{range}(P_\mathcal{M})$, have $|P_\mathcal{M} v|\ge |\sin\vartheta|\,|v|$ for all $v\in\mathrm{range}(J)$. Thus $v^\top F_\mathcal{M} v=|P_\mathcal{M} Jv|^2\ge \sin^2\vartheta\,|Jv|^2=v^\top (\sin^2\vartheta\,F) v$, yielding $F_\mathcal{M}\succeq \sin^2\vartheta\,F$. Taking maximum eigenvalue yields $\mathrm{Tr}(F_\mathcal{M}^{-1})\le \mathrm{Tr}(F^{-1})/ \sin^2\vartheta$, i.e., variance reduction factor $\mathcal{R}\le 1/|\sin\vartheta|$. QED.

\section{Reproducible Experiment Parameters and Error Budget (Brief Table)}

\textbf{G.1 P1 (closed domain)}: $h\in\{h_0,\,h_0/2,\,h_0/4\}$; $t_{\min}\sim c\,h^2$, $t_{\max}$ satisfying semiclassical window; $\mu$ taking $\{\mu_0,\,\mu_0/3,\,\mu_0/9,\,\mu_0/27\}$. Extrapolation using bilinear (for $(\mu,h)$) and Richardson (for $t$-window) hybrid. Tolerance: slope $<1\%$; drift under different corner triangulations/gauges $<0.5\%$.

\textbf{G.2 P2 (exterior domain)}: Strip $\Im\omega\in[-\gamma_0,0]$ uniform sampling; pole number $J$ jointly determined by AIC/BIC and strip cross-validation; penalty term constrains $Q(\omega)$ degree and real-axis purely imaginary condition; prior $\Re\log\det_2\ge 0$ only as soft regularization. Tolerance: CRLB improvement $\ge 1.3$, false alarm $\le 5\%$.

\textbf{G.3 P3 (channels)}: Extrapolation radius, matching radius, integration step calibrated via grid search; Wronskian normalization difference $<10^{-12}$; $|S_{\ell m}^\dagger\eta S_{\ell m}-\eta|_\infty<10^{-12}$; phase closure $<10^{-3}$ radians; converges as $a+b/\ell_{\max}$ with $\ell_{\max}$.

\section*{End of Main Text and Appendices}

\end{document}


