\documentclass[11pt]{article}
\usepackage[utf8]{inputenc}
\usepackage[T1]{fontenc}
\usepackage{amsmath,amssymb,amsthm}
\usepackage{mathtools}
\usepackage{geometry}
\geometry{margin=1in}
\usepackage{hyperref}
\usepackage{cite}
\usepackage{braket}
\usepackage{graphicx}

\newtheorem{theorem}{Theorem}
\newtheorem{lemma}[theorem]{Lemma}
\newtheorem{proposition}[theorem]{Proposition}
\newtheorem{corollary}[theorem]{Corollary}
\theoremstyle{definition}
\newtheorem{definition}[theorem]{Definition}
\newtheorem{assumption}[theorem]{Assumption}
\theoremstyle{remark}
\newtheorem{remark}[theorem]{Remark}

\title{Time Crystals--Null--Modular $\mathbb{Z}_2$ Holonomy Unification: From Floquet and Lindblad to Bulk-Integral BF Relative Cohomology Criterion}

\author{Haobo Ma$^1$ \and Wenlin Zhang$^2$\\
\small $^1$Independent Researcher\\
\small $^2$National University of Singapore}

\date{}

\begin{document}

\maketitle

\begin{abstract}
Construct theoretical chain unifying discrete/continuous time crystals with Null--Modular $\mathbb{Z}_2$ holonomy, bulk-integral $\mathbb{Z}_2$--BF choice, and relative cohomology invariant. Closed system side, provide rigidity and stability of prethermal discrete time crystals in exponentially long time windows via high-frequency Floquet--Magnus and Lieb--Robinson constraints; under strong disorder provide necessary and sufficient structure for $\pi$ spectral pairing and eigenstate time crystalline order. Open system side, establish spectral criterion for limit cycle time crystals on peripheral spectrum of single-period CPTP channel. Quasiperiodic drive side, construct "temporal quasicrystal" group representation via finite image of $\mathbb{Z}^k$ time translation group. Above four classes of phenomena interface with unified topological--algebraic skeleton: $\mathbb{Z}_2/\mathbb{Z}_m$ holonomy and relative cohomology class $[K]\in H^2(Y,\partial Y;\mathbb{Z}_2)$ of bulk-integral $\mathbb{Z}_2$--BF top term; under small causal diamond threshold, if satisfying modular--scattering mod-two alignment and parameter two-cycle detectability and generation, then "geometry--energy--topology" triplet equivalent, specifically $[K]=0 \iff$ time crystal "anomaly" vanishes on allowed loops and two-cycles. This paper simultaneously provides $\mathbb{Z}_2$ fingerprints for three solvable families ($\delta$ potential, Aharonov--Bohm, topological superconductor endpoint) and engineering schemes with error budgets for superconducting qubits, Rydberg gases, and trapped ions. Core Null--Modular double cover and BF relative cohomology criterion taken from authors' existing unified principle and restated and proved in time crystal context.
\end{abstract}

\textbf{Keywords}: Discrete time crystal; prethermalization and many-body localization; open system limit cycle; temporal quasicrystal; topological time crystal; $\pi$ spectral pairing; $\mathbb{Z}_2/\mathbb{Z}_m$ holonomy; bulk-integral $\mathbb{Z}_2$--BF; relative cohomology; small causal diamond

\section{Introduction \& Historical Context}

Spontaneous breaking of time translation symmetry rigorously negated in equilibrium systems, forcing physical carriers of time-ordered phases toward non-equilibrium drive and open dynamics. In periodically driven many-body systems, discrete time crystals characterized by subharmonic response rigidity, long-range temporal correlations, and characteristic spectral fingerprints; subsequent branches of eigenstate ordering, prethermal longevity, dissipative limit cycles, and topological (logical) time crystals constitute cross-platform experimental systems. On other hand, we developed invariants centered on square root determinant branching and mod-two holonomy along geometric--information--scattering unified thread, elevating them to necessary and sufficient criterion via bulk-integral $\mathbb{Z}_2$--BF relative cohomology language. This paper rigorously bridges these two threads, proving: time crystal "$\pi$/unit root" phenomenology at unified criterion level is $\mathbb{Z}_2/\mathbb{Z}_m$ holonomy and $[K]\in H^2(Y,\partial Y;\mathbb{Z}_2)$ non-triviality; conversely, geometry--energy criterion (first and second order layer on small causal diamond) under alignment threshold implies trivializatio

n of such invariants, thereby providing common structure penetrating closed/open/topological/multi-frequency.

\section{Model \& Assumptions}

\subsection{Closed System (Floquet--MBL/Prethermalization)}

Local many-body system on lattice $\Lambda$, periodic drive $H(t+T)=H(t)$. Floquet unitary $F=\mathcal T \exp(-\mathrm i\int_0^T H\,\mathrm dt)$ generates discrete time translation. High-frequency limit $\omega=2\pi/T\gg J$ admits quasilocal effective Hamiltonian $H_\ast$ with exponentially small truncation error; under strong disorder $H_0$ supports $l$-bit structure.

\subsection{Open System (Periodic Lindblad)}

Density matrix evolution $\dot\rho=\mathcal L_t(\rho)$ with $\mathcal L_{t+T}=\mathcal L_t$. Single-period quantum channel $\mathcal E=\mathcal T\exp(\int_0^T\mathcal L_t\,\mathrm dt)$ peripheral spectrum determines long-time limit cycle.

\subsection{Multi-Frequency Quasiperiodic}

Mutually irrational frequency family $\{\omega_i\}_{i=1}^k$ defines time translation group $\mathbb Z^k$; high-frequency prethermalization threshold $\min_i \omega_i \gg J$ ensures quasilocal $H_\star$ and finite image group representation.

\subsection{Topological Time Crystal (Logical Subspace)}

Stabilizer code (surface code) periodic engineering makes $F_{\rm logical}\simeq \overline X_L \mathrm e^{-\mathrm i H^{\rm top}_\ast T}$, non-local logical operators as natural order parameters.

\subsection{Null--Modular and Bulk-Integral BF (Relative Cohomology)}

Working space $Y=M\times X^\circ$, where $M$ small causal diamond domain or more general local spacetime patch, $X^\circ$ parameter domain removing discriminant set $D$. Define

$$
K \;=\; \pi_M^\ast w_2(TM) \;+\; \sum_j \pi_M^\ast\mu_j\smile \pi_X^\ast\mathfrak w_j \;+\; \pi_X^\ast\rho(c_1(\mathcal L_S)) \;\in H^2(Y;\mathbb Z_2),
$$

using $[K]\in H^2(Y,\partial Y;\mathbb Z_2)$ under boundary trivialization and relative cohomology lift. $\mathbb Z_2/\mathbb Z_m$ holonomy computed as mod-two (or unit root) value of square root determinant branch, stabilizing closed paths per "small semicircle/fold-back" rule.

\section{Main Results (Theorems and Alignments)}

\begin{theorem}[A: Rigidity and Exponential Lifetime of Prethermal DTC]

Under $\omega\gg J$ and piecewise near-$\pi$ "symmetric kick" conditions, quasilocal unitary $U_\ast$ and symmetry element $X^2=\mathbb 1$ exist making

$$
F \;=\; U_\ast\, \mathrm e^{-\mathrm i H_\ast T}\, X\, U_\ast^\dagger \;+\; \Delta,\qquad |\Delta|\le C\mathrm e^{-c\omega/J}.
$$

Any local observable $O$ odd under $X$ exhibits $2T$ subharmonic locking, remaining locked for $t\lesssim\tau_\ast\sim \mathrm e^{c\omega/J}$, maintaining Lipschitz stability against small perturbations.
\end{theorem}

\begin{theorem}[B: $\pi$ Spectral Pairing and Eigenstate Order in MBL--DTC]

On strongly disordered chain quasilocal unitary $U$ exists making $F\simeq \tilde X\, \mathrm e^{-\mathrm i H_{\rm MBL}T}$, spectrum exhibits $\pi$ pairing, inducing state-independent $2T$ subharmonic response.
\end{theorem}

\begin{theorem}[C: Spectral Criterion for Open System Limit Cycle]

If single-period channel $\mathcal E$ peripheral spectrum $\{ \mathrm e^{2\pi \mathrm i k/m}\}$ with spectral radius $<1$ elsewhere, then almost all initial states converge to period-$mT$ limit cycle attractor family, constituting $m$-subharmonic dissipative time crystal.
\end{theorem}

\begin{theorem}[D: Multi-Frequency "Temporal Quasicrystal" Group Representation]

Under prethermalization threshold $\min_i\omega_i\gg J$, finite image of time translation group $\mathbb Z^k$ produces multiple incommensurate subharmonic peaks, forming temporal quasicrystal.
\end{theorem}

\begin{theorem}[E: Topological Time Crystal: Non-Local Order and Topological Entanglement]

In logical subspace, non-local loop operators exhibit rigid multiple-period response, consistent with non-zero topological entanglement entropy term.
\end{theorem}

\begin{theorem}[F: Unified Topological--Cohomology Criterion]

Under boundary trivialization, relative generation and detectability threshold, following equivalent:

$$
[K]=0\ \Longleftrightarrow\ \text{for all allowed loops}\ \gamma:\ \nu_{\sqrt{\det_p S}}(\gamma)=+1\ \text{and all allowed two-cycles}\ \gamma_2:\ \langle \rho(c_1(\mathcal L_S)),[\gamma_2]\rangle=0.
$$

When $H^2(X^\circ,\partial X^\circ)=0$, above equivalent to mod-two criterion on loops only.
\end{theorem}

\begin{theorem}[G: Geometry--Energy $\Rightarrow$ Topological Triviality]

Under small causal diamond threshold, relative generation--detection, and modular--scattering mod-two alignment, if first order layer gives $G_{ab}+\Lambda g_{ab}=8\pi G T_{ab}$ and second order relative entropy $\delta^2 S_{\rm rel}=\mathcal E_{\rm can}\ge 0$, then above unified topological--cohomology criterion holds, i.e., implies $[K]=0$ and all $\mathbb Z_2$ holonomies trivial.
\end{theorem}

\section{Proofs}

\subsection{Prethermalization and $\pi$ Locking (Theorems A/B)}

Take Floquet--Magnus expansion $F=\exp\{-\mathrm iT\sum_{n\ge0}\Omega_n\}$, truncate at optimal order $n_\ast\sim \omega/J$ defining $H_\ast$. Via Lieb--Robinson and local expansion series renormalization obtain $|F-\mathrm e^{-\mathrm iH_\ast T}|\le C\mathrm e^{-c\omega/J}$. Piecewise near-$\pi$ kick $U_X$ produces $X$, obtaining structural decomposition in quasilocal unitary $U_\ast$ representation. Under strong disorder $UH_0U^\dagger=f(\{\tau_i^z\})$ nearly commutes with $\tilde X=UXU^\dagger$, spectrum exhibits $\pi$ pairing; arbitrary initial state expanded in paired subspace, obtaining state-independent $2T$ subharmonic response.

\subsection{Peripheral Spectrum and Limit Cycle (Theorem C)}

Perform Jordan--Riesz decomposition for CPTP $\mathcal E$. If peripheral spectrum $m$ unit roots with spectral gap elsewhere, then $\mathcal E^n$ exponentially converges on each residue class to period-$m$ cyclic attractor; convergence rate controlled by Liouvillian spectral gap.

\subsection{$\mathbb Z_2/\mathbb Z_m$ Holonomy and Relative Cohomology (Theorem F)}

Work with $\mathbb Z_2$ coefficients taking relative cohomology. Bulk-integral $\mathbb Z_2$--BF action

$$
I_{\rm BF}[a,b]=\mathrm i\pi\!\int_{(Y,\partial Y)} b\smile\boldsymbol\delta a\;+\;\mathrm i\pi\!\int_{(Y,\partial Y)} b\smile K\;+\;\mathrm i\pi\!\int_{\partial Y} a\smile b,
$$

after gauge transformation and boundary term cancellation, summing over $[a]\in H^1(Y,\partial Y)$, $[b]\in H^{d-2}(Y,\partial Y)$, using finite abelian group character orthogonality obtains partition function projection $Z_{\rm top}\propto \delta([K])$, i.e., $[K]=0$. By Poincaré--Lefschetz duality, $[K]=0$ if and only if Kronecker pairing vanishes for all allowed relative two-cycles $[S]$; when $H^2(X^\circ,\partial X^\circ)=0$, reduces to loop mod-two criterion only.

\subsection{Geometry--Energy Implies Topological Triviality (Theorem G)}

Under small causal diamond threshold (Hadamard, no conjugate points, corner prescription, $\nabla^aT_{ab}=0$, fixed temperature scale) and invertible--stable hypothesis for weighted null ray transformation, family constraint $\int w(\lambda)(R_{kk}-8\pi G T_{kk})\,\mathrm d\lambda=0$ with Radon-type closure implies $R_{kk}=8\pi G T_{kk}$; null cone characterization and Bianchi identity upgrade to tensor equation, obtaining first order layer $G_{ab}+\Lambda g_{ab}=8\pi G T_{ab}$. Corner prescription ensures covariant phase space symplectic flux closure, $\delta^2S_{\rm rel}=\mathcal E_{\rm can}\ge 0$. If closed path $\gamma$ exists making $\nu_{\sqrt{\det_p S}}(\gamma)=-1$, then by modular--scattering mod-two alignment, construct linear functional corresponding to holonomy in covariant phase space embedding into quadratic form kernel, obtaining $\mathcal E_{\rm can}[h,h]<0$ contradiction, thus implying all allowed loop holonomies trivial, further by relative generation--detection obtaining $[K]=0$.

\section{Model Apply}

\subsection{$\mathbb{Z}_2$ Fingerprints for Solvable Families}

(i) 1D $\delta$ potential: Select small loop around complex pole, $\oint \tfrac{1}{2\mathrm i} S^{-1}\mathrm dS=\pi\Rightarrow \nu_{\sqrt{\det_p S}}=-1$.
(ii) 2D Aharonov--Bohm: Flux crossing half-flux $\alpha=\tfrac{1}{2}$ gives $\deg(\det_p S|_\gamma)=1\Rightarrow \nu=-1$.
(iii) Topological superconductor endpoint (Class D/DIII): $\operatorname{sgn}\det_p r(0)$ or $\operatorname{sgn}\operatorname{Pf}r(0)$ flip synchronizes with $\nu_{\sqrt{\det_p r}}$. Three families after removing discriminant set have $H^2(X^\circ,\partial X^\circ)=0$, requiring loop mod-two criterion only.

\subsection{Platform Mapping and Observables}

Superconducting qubit 2D array: Logical loop operator spectrum exhibits $\omega/2$ peak only in non-local channel, accompanied by non-zero topological entanglement entropy; Rydberg gas: Unit roots in quantum channel peripheral spectrum consistent with fluorescence autocorrelation limit cycle; trapped ions: $\omega$ enhancement brings exponential lifetime growth with rigid frequency position not drifting; multi-frequency drive: Incommensurate peak positions correspond to finite image of $\mathbb Z^k$.

\section{Engineering Proposals}

\subsection{Prethermalization Window and Pulse Synthesis}

Choose $\omega$ making $\mathrm e^{-c\omega/J}\ll \varepsilon$ (gate noise amplitude), ensuring $\tau_\ast\sim \mathrm e^{c\omega/J}$ covers $10^2\!-\!10^3$ cycles; piecewise sequence implements near-$\pi$ flip at $\tau_x$ to amplify $2T$ locking.

\subsection{Open System Spectral Gap Engineering}

Construct Liouvillian spectral gap $\Delta_{\rm Liouv}$ via pumping--decoherence ratio $G/\kappa$, suppressing multistable wandering; sample peripheral eigenvalues and limit cycle period around steady-state working point.

\subsection{Multi-Frequency Temporal Quasicrystal}

Two to three mutually irrational frequencies, avoiding accidental integer period recurrence; collect spectrum using incommensurate peak positions to identify finite image; reconstruct unit root values via parameter closed paths.

\subsection{Experimental Readout of Wilson--Loop}

Amplitude--phase joint scan forming closed path; distinguish $\pm1$ via discrete Fourier peaks in Ramsey/correlation functions; for $\mathbb Z_m$ fit unit roots using phase grid.

\section{Discussion (risks, boundaries, past work)}

Boundaries and risks include: absorption-induced locking collapse outside high-frequency window; MBL stability limited in high dimensions and long-range interactions; open system non-Markovianity causes phase wandering; topological time crystal non-local readout systematic sensitivity to leakage and crosstalk. At unified criterion level, $H^2$ channel detectability requires allowed two-cycle generation of relative two-cohomology; if platform parameter domain two-skeleton insufficient, obtains only necessary non-sufficient criterion. Compared to existing work, this paper's increment: using bulk-integral $\mathbb Z_2$--BF relative cohomology class $[K]$ and mod-two holonomy as \textbf{single invariant}, uniformly characterizing closed/open/topological/multi-frequency four classes of time crystals, establishing implication chain between geometry--energy and holonomy--cohomology under small causal diamond variational threshold.

\section{Conclusion}

Time crystal "$\pi$/unit root" phenomenology uniformly characterized by $\mathbb Z_2/\mathbb Z_m$ holonomy and bulk-integral $\mathbb Z_2$--BF relative cohomology class; when relative generation--detection and modular--scattering mod-two alignment hold, geometry--energy criterion on small causal diamond implies topological--cohomology trivialization. This structure simultaneously supports prethermal DTC, MBL eigenstate order, open system limit cycle, and topological time crystal, providing group representation and experimental readout for multi-frequency temporal quasicrystals. This framework provides unified theory--engineering channel for cross-platform time-frequency devices, robust storage, and topological logical operations.

\section*{Acknowledgements, Code Availability}

Thank publicly available results and platform data establishing experimental background. This paper does not rely on proprietary code; scripts for relative cohomology pairing, $\mathbb Z_2/\mathbb Z_m$ holonomy reconstruction, and peripheral spectrum fitting can be reproduced using standard numerical tools following algorithmic steps in appendices.

\section*{References}

Select representative theoretical and experimental works: time crystal no-go theorems, Floquet--DTC definition, prethermalization upper bounds, eigenstate time crystalline order, dissipative time crystals, topological time crystals and temporal quasicrystals; as well as Null--Modular double cover and bulk-integral $\mathbb Z_2$--BF unified principle on geometric--information--scattering thread.

\appendix

\section{Rigorous Prethermalization Upper Bound and Exponential Lifetime}

Assume $|H(t)|\le J$, $\omega\gg J$. Floquet--Magnus

$$
\Omega_0=\tfrac{1}{T}\!\int_0^T\!H,\quad
\Omega_1=\tfrac{1}{2T\mathrm i}\!\iint_{0<t_1<t_2<T}[H(t_2),H(t_1)]\,\mathrm dt_1\mathrm dt_2,\ \ldots
$$

truncate at $n_\ast\sim \alpha \omega/J$, define $H_\ast=\sum_{n\le n_\ast}\Omega_n$. Via nested commutator tree renormalization prove
$|F-\mathrm e^{-\mathrm iH_\ast T}|\le C\mathrm e^{-c\omega/J}$,
$|\tfrac{\mathrm d}{\mathrm dt}\langle H_\ast\rangle|\le C'\mathrm e^{-c\omega/J}$.
Piecewise near-$\pi$ kick $U_X$ makes
$F\approx X\,\mathrm e^{-\mathrm iH_\ast T}+\mathcal O(\epsilon)+\mathcal O(\mathrm e^{-c\omega/J})$, thus for $X$-odd $O$
$\langle O(nT)\rangle\approx(-1)^n\langle O(0)\rangle+\mathcal O(\epsilon)+\mathcal O(\mathrm e^{-c\omega/J})$.
Frequency domain exhibits $\omega/2$ locking peak, peak position rigid against parameter perturbations.

\section{MBL $\pi$ Pairing and Eigenstate Order}

Unitary $U$ exists making $UH_0U^\dagger=f(\{\tau_i^z\})$, near-$\pi$ kick under $U$ transformation yields quasilocal $\tilde X$. If $F\simeq \tilde X\mathrm e^{-\mathrm iH_{\rm MBL}T}$, then for eigenstate $|\psi\rangle$
$F|\psi\rangle=\mathrm e^{-\mathrm iET}\tilde X|\psi\rangle,\quad
F(\tilde X|\psi\rangle)=\mathrm e^{-\mathrm i(ET+\pi)}|\psi\rangle$, spectrum exhibits $\pi$ pairing. Arbitrary initial state expanded in paired subspace yields state-independent $2T$ subharmonic response.

\section{CPTP Peripheral Spectrum and Limit Cycle}

Let $\sigma(\mathcal E)=\{\lambda_j\}$. If $|\lambda_j|<1$ for all non-peripheral modes, peripheral modes $\{ \mathrm e^{2\pi \mathrm i k/m}\}$, then mutually disjoint cyclic invariant subspaces exist, making $\mathcal E^n$ project arbitrary initial state to period-$m$ limit cycle; convergence rate controlled by $\Delta_{\rm Liouv}=1-\max_{|\lambda_j|<1}|\lambda_j|$.

\section{$\mathbb{Z}_2/\mathbb{Z}_m$ Holonomy: Mod-Two Robustness of Spectral Flow and Intersection Number}

Denote discriminant set $D\subset X^\circ$ as threshold/embedded eigenvalue or $-1$ eigenvalue submanifold. For closed path $\gamma$ stabilized per "small semicircle/fold-back" rule, define mod-two intersection number $I_2(\gamma,D)$. Modified determinant $\det_p$ change only alters quantized phase integer winding number, mod-two projection invariant; partial-wave truncation $N\to\infty$ and relative trace-class renormalization preserve $\nu_{\sqrt{\det_p S}}=(-1)^{I_2(\gamma,D)}$.

\section{Bulk-Integral $\mathbb{Z}_2$--BF Relative Cohomology Derivation}

On $(Y,\partial Y)$ with $\mathbb Z_2$ coefficients construct

$$
I_{\rm BF}[a,b]=\mathrm i\pi\!\int b\smile\boldsymbol\delta a+\mathrm i\pi\!\int b\smile K+\mathrm i\pi\!\int_{\partial Y} a\smile b.
$$

Under gauge transformation $a\mapsto a+\boldsymbol\delta\lambda^0$, $b\mapsto b+\boldsymbol\delta\lambda^{d-3}$ boundary term cancels, action well-defined. Summing over $[a]$ and $[b]$, using finite abelian group character orthogonality obtains $Z_{\rm top}\propto \delta([K])$; Poincaré--Lefschetz duality gives $[K]=0 \iff$ for all $[S]\in H_2(Y,\partial Y;\mathbb Z_2)$ have $\langle K,[S]\rangle=0$.

\section{Geometry--Energy $\Rightarrow$ Holonomy Triviality: Alignment and Contradiction Method}

Under corner prescription and covariant phase space framework, closed path modular holonomy defines bounded linear functional embedding into quadratic form kernel. If $\gamma$ exists making $\nu_{\sqrt{\det_p S}}(\gamma)=-1$, functional under mod-two projection provides negative direction, constructing $h$ making $\mathcal E_{\rm can}[h,h]<0$, contradicting second-order non-negativity; thus all allowed closed path holonomies $(+1)$. Combined with relative generation and detection, implies $[K]=0$.

\section{Deformation Retraction and $H^2=0$ for Three Solvable Families}

Parameter domains of $\delta$ potential, AB, and endpoint scattering after removing discriminant set deformation retract to one-dimensional skeleton, thus $H^2(X^\circ,\partial X^\circ)=0$. Therefore unified criterion reduces to mod-two condition on loops; in this case, constructing reference closed path transverse to $D$ verifies $[K]=0$ necessary and sufficient condition.

\section{Experiment--Algorithm Checklist (Pseudocode Level)}

1. Data acquisition and detrending: Time series $O(t_n)$ remove drift via polynomial regression;
2. Peak and locking metric: Discrete Fourier, fit peak width $\Gamma$ and locking ratio $\mathcal R$;
3. Unit root readout: Parameter closed path sample phase, discriminate $\pm1$ or $m$-th unit root;
4. Relative cohomology test: Select generating family loops/two-cycles, evaluate $\nu_{\sqrt{\det_p S}}$ and $\langle \rho(c_1),[\gamma_2]\rangle$ table, verify all-zero to determine $[K]=0$.

\end{document}


