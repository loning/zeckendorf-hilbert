\documentclass[11pt]{article}
\usepackage[utf8]{inputenc}
\usepackage[T1]{fontenc}
\usepackage{amsmath,amssymb,amsthm}
\usepackage{mathtools}
\usepackage{geometry}
\geometry{margin=1in}
\usepackage{hyperref}
\usepackage{cite}
\usepackage{braket}
\usepackage{graphicx}

\newtheorem{theorem}{Theorem}
\newtheorem{lemma}[theorem]{Lemma}
\newtheorem{proposition}[theorem]{Proposition}
\newtheorem{corollary}[theorem]{Corollary}
\theoremstyle{definition}
\newtheorem{definition}[theorem]{Definition}
\newtheorem{assumption}[theorem]{Assumption}
\theoremstyle{remark}
\newtheorem{remark}[theorem]{Remark}

\title{Cross-Platform Metrological Paradigm for Unified Phase--Frequency Readout: Windowed Upper Limits on FRB Vacuum Polarization, Spectral--Scattering Equivalence and Identifiability of $\delta$-Ring--AB Flux, and Critical Coupling Metrology for Scattering Topological Invariants}

\author{Haobo Ma$^1$ \and Wenlin Zhang$^2$\\
\small $^1$Independent Researcher\\
\small $^2$National University of Singapore}

\date{}

\begin{document}

\maketitle

\begin{abstract}
Propose unified metrological paradigm using "phase--frequency" as sole readout, penetrating three classes of systems with disparate scales and physical backgrounds: fast radio burst (FRB) propagation at cosmological distances, point interactions ($\delta$ potential) and Aharonov--Bohm (AB) flux coupling in cold atom/electronics one-dimensional ring systems, and scattering topological invariants for condensed matter class D endpoints. Under unified \textbf{kernel--windowing--generalized least squares/Fisher} syntax provide: First, self-consistent estimation of refractive index modification order via curved spacetime QED one-loop effective action, constructing reproducible \textbf{windowed upper limiter}; on FRW background scale upper limit phase residual scale below observational threshold, yielding only rigorous upper bounds. Second, rigorously prove equivalence between spectral quantization trigonometric equation on ring

$$
\cos\theta=\cos(kL)+\frac{\alpha_\delta}{k}\sin(kL)
$$

and "\textbf{amplitude-corrected} phase closure formula"

$$
\cos\gamma(k)=|t(k)|\,\cos\theta,\qquad
\gamma=kL-\arctan\!\frac{\alpha_\delta}{k},\quad
|t(k)|=\frac{k}{\sqrt{k^2+\alpha_\delta^2}}
$$

fully equivalent, degenerating to pure phase closure only in limit $|t|\to1$; provide \textbf{structural identifiability} theorem that $\{k_n(\theta)\}$ alone can \textbf{uniquely invert} $\alpha_\delta\equiv mg/\hbar^2$, with closed-form sensitivity, ill-conditioning avoidance criteria, and first-order equivalentization of narrow potential width. Third, on class D endpoints use $\mathcal Q=\mathrm{sgn}\det r(0)$ as topological criterion, propose under \textbf{near-unitary} premise unified finite-size (exponential) and contact/boundary (algebraic) bias extrapolation regression

$$
J_c(L)=J_c+\beta e^{-L/\xi}+\gamma/L,
$$

provide linear response mapping to cQED readout (cavity frequency shift and additional loss). Proofs and engineering protocols for above conclusions provided with paper, directly reproducible and transplantable to similar platforms.
\end{abstract}

\textbf{Keywords}: Phase--frequency metrology; fast radio burst; curved spacetime QED; windowed upper limits; $\delta$ potential; Aharonov--Bohm effect; self-adjoint extension; scattering topological invariant; Majorana; cQED

\section{Introduction \& Historical Context}

Metrological schemes using "phase--frequency" as single observable naturally span multiple frontiers: in cosmological electromagnetic propagation, phase frequency perturbations record geometry and medium; in one-dimensional quantum rings, phase shifts quantify scattering and geometric coherence; in condensed matter scattering metrology, low-energy reflection subblock phase and determinant characterize topological number. Curved spacetime QED one-loop effective action indicates vacuum polarization introduces only minute refractive index correction on weak curvature backgrounds, satisfying high-frequency causality and analyticity; this makes FRB \textbf{carrier phase} naturally a channel for setting \textbf{upper limits}, not realistic detection. $\delta$-ring problem under self-adjoint extension $U(2)$ framework yields rigorous spectral--scattering equivalence unifying with AB flux; topological endpoints implementable via single-end reflection matrix zero-energy criterion for near-field metrology of non-local topology. Above backgrounds increasingly clear in recent literature; this paper's goal is condensing into \textbf{unified metrological syntax and reproducible recipes}.

\section{Model \& Assumptions}

\subsection{Unified Observation Equation, Weights, and Windowing}

Across three platforms, observables abstractable as phase-type readout $m(\omega)$ on frequency. Unified expression

$$
m(\omega)=\int_{\mathcal{P}} \mathcal{K}(\omega,\chi)\,x(\chi)\,{\rm d}\chi+\sum_{p=0}^{P-1} a_p\,\Pi_p(\omega)+\epsilon(\omega),
\tag{2.1}
$$

where $\mathcal{K}$ is observation kernel, $\chi$ denotes propagation parameter (e.g., conformal coordinate), geometric phase, or energy scale; $x(\chi)$ is target function (or windowed parameter); $\Pi_p$ are systematic/foreground basis functions, $a_p$ their amplitudes; $\epsilon$ is noise process, covariance $C_\phi(\omega,\omega')$. On discrete frequency axis adopt weighted inner product

$$
\langle f,g\rangle \equiv \sum_{ij} f(\omega_i)\,[C_\phi^{-1}]_{ij}\,g(\omega_j).
\tag{2.2}
$$

Define weighted Gram--Schmidt orthonormalized frequency windows $W_j(\omega)$, construct design matrix with $A_{j\mu}=\langle W_j,\,\mathcal{K}(\cdot,\chi_\mu)\rangle$; generalized least squares (GLS) and Fisher information

$$
\hat{\boldsymbol{\theta}}=(A^\top A)^{-1}A^\top \mathbf{y},\qquad F=A^\top A,
\tag{2.3}
$$

where $\mathbf{y}_j=\langle W_j,m\rangle$, $\boldsymbol{\theta}$ are windowed parameters (or coefficients of $x(\chi)$ in piecewise constant basis). Classical CRB for single-frequency phase in white noise limit degenerates to $\sigma_\phi\ge (2N\rho)^{-1/2}$, signal processing reference for this paper's upper limit derivation.

\subsection{Platform-Specific Assumptions and Kernels}

\begin{itemize}
\item \textbf{FRB propagation}: Along conformal coordinate $\chi$ (${\rm d}\ell=a(\eta){\rm d}\chi$) integration, local frequency $\omega_{\rm loc}=(1+z)\omega_{\rm obs}$ cancels with scale factor, thus

$$
\boxed{\ \mathcal{K}_{\rm FRB}(\omega_{\rm obs},\chi)=\omega_{\rm obs}/c\ },
\tag{2.4}
$$

taking $x(\chi)\equiv \delta n(\chi)$ (refractive index correction) as windowed target kernel. This form guarantees high-frequency causality $n(\omega\to\infty)\to 1$ and worldline--Penrose limit analyticity constraint.

\item \textbf{$\delta$-ring + AB flux}: Observable is eigenwavevector $k(\theta)$ satisfying quantization condition, $\alpha_\delta\equiv mg/\hbar^2$ represents point interaction strength, geometric length $L$ is ring circumference, $\theta=2\pi\Phi/\Phi_0$ is flux phase. Kernel function manifests in dispersion relation and boundary conditions, see Section 4.

\item \textbf{Topological scattering}: Observable is characteristic phase of zero-energy reflection matrix $r(0)$ or sign of its determinant; kernel $\mathcal{K}$ from scattering--topology mapping and input--output linear response, see Section 6.
\end{itemize}

\section{Main Results (Theorems and Alignments)}

\begin{theorem}[A: FRB--Windowed Upper Limit and Order of Magnitude for One-Loop Vacuum Polarization]
Curved spacetime QED under weak curvature background provides one-loop correction order of magnitude for refractive index

$$
\delta n \sim \frac{\alpha_{\rm em}}{\pi}\,\lambda_e^2\,\mathcal{R},
\tag{3.1}
$$

taking FRW upper bound $|\mathcal R|\sim H_0^2/c^2$, phase accumulation upper limit for 1 GHz, 1 Gpc

$$
\Delta \phi \sim \frac{\omega_{\rm obs} L}{c}\,\delta n \approx 1.2\times 10^{-53}\ {\rm rad}.
\tag{3.2}
$$

Therefore under any realistic baseband data volume and noise, can only provide windowed upper limit for $x(\chi)=\delta n(\chi)$, not effective detection. Proof in Appendix A.
\end{theorem}

\begin{theorem}[B: $\delta$-Ring--Equivalence of Trigonometric Equation and Amplitude-Corrected Phase Formula]
For one-dimensional ring $\delta$ potential (strength $g$) and AB flux $\theta$, eigenwavevector $k$ satisfies

$$
\boxed{\ \cos\theta=\cos(kL)+\frac{\alpha_\delta}{k}\sin(kL)\ },
\tag{3.3}
$$

which is fully equivalent to

$$
\boxed{\ \cos\gamma(k)=|t(k)|\,\cos\theta,\
\gamma(k)=kL-\arctan\!\frac{\alpha_\delta}{k},\
|t(k)|=\frac{k}{\sqrt{k^2+\alpha_\delta^2}}\ }
\tag{3.4}
$$

and degenerates to pure phase closure $\cos\gamma=\cos\theta$ only in limit $|t(k)|\to 1$ (weak scattering or transmission resonance). Proof in Appendix B.
\end{theorem}

\begin{theorem}[C: $\delta$-Ring--Structural Identifiability from $\{k_n(\theta)\}$ Alone]
Under spectral observation $\{k_n(\theta)\}$ at fixed $(L,\theta)$, can uniquely invert $\alpha_\delta=mg/\hbar^2$, but cannot separate individual values of $m$ and $g$. If introducing second independent readout type (e.g., energy spectrum $E_n=\hbar^2 k_n^2/(2m)$ or absolute potential strength $g_{\rm eff}=\int V\,{\rm d}x$ calibration), then at generic points $(m,g)$ Jacobian full rank and decoupling possible. Proof in Appendix B.
\end{theorem}

\begin{theorem}[D: $\delta$-Ring--Implicit Function Sensitivity and Ill-Conditioning]
Let

$$
f(k,\alpha_\delta,\theta)=\cos(kL)+\frac{\alpha_\delta}{k}\sin(kL)-\cos\theta,
\tag{3.5}
$$

then

$$
\frac{\partial k}{\partial \alpha_\delta}
=\frac{ \dfrac{\sin(kL)}{k} }
{ L\sin(kL)-\dfrac{\alpha_\delta}{k}L\cos(kL)+\dfrac{\alpha_\delta}{k^2}\sin(kL) }.
\tag{3.6}
$$

Ill-conditioned domain given by $\partial f/\partial k=0$; when $|\alpha_\delta|/k\ll 1$ approximately $\sin(kL)\simeq (\alpha_\delta/k)\cos(kL)$. Proof in Appendix B.
\end{theorem}

\begin{theorem}[E: First-Order Equivalentization of $\delta$-Potential Finite Width]
For narrow potential $V(x)$ (width $w\ll L$) with $\int V\,{\rm d}x=g$ fixed, in $(kw)\ll1$ region, effective strength on ring

$$
\alpha_\delta^{\rm eff}(k)=\frac{m}{\hbar^2}\,g\left[1+\eta_2[V]\,(kw)^2+\mathcal O\!\big((kw)^4\big)\right],
\tag{3.7}
$$

where shape factor $\eta_2[V]$ determined by second moment of $V$; rectangular and Gaussian potentials respectively have $\eta_2=\tfrac{1}{3},\ \tfrac{1}{2}$. Proof and numerical comparison in Appendix B.
\end{theorem}

\begin{theorem}[F: Topological Scattering Invariant--Near-Unitary Extrapolation]
Under near-unitary premise ($|r^\dagger r-\mathbb 1|\le \varepsilon_\star$), finite length $L$, contact imperfection, and weak dissipation introduce bias to critical coupling $J_c$, unified by

$$
\boxed{\ J_c(L)=J_c+\beta e^{-L/\xi}+\gamma/L\ }
\tag{3.8}
$$

extrapolation regression absorption, where exponential term corresponds to endpoint state coupling overlap, algebraic term corresponds to contact/boundary effects. $\mathcal Q=\mathrm{sgn}\det r(0)$ flip provides $J_c(L)$ observation trajectory. Proof in Appendix C.
\end{theorem}

\begin{theorem}[G: cQED Redundant Readout and Scattering Invariant Consistency]
Single-port weak coupling, input--output relation gives

$$
r(\omega)=\frac{1-Z_0Y(\omega)}{1+Z_0Y(\omega)}\approx 1-2Z_0Y(\omega),
\quad
\delta\omega_c\propto g_{\rm cav}^2\,\mathrm{Re}\,\chi(\omega_c),\
\kappa_{\rm add}\propto g_{\rm cav}^2\,\mathrm{Im}\,\chi(\omega_c),
\tag{3.9}
$$

where $\chi$ is endpoint subsystem linear response. For appropriate detuning $|\omega_c-\omega_{\rm cont}|\gtrsim 3\kappa$, zero crossings of $\{\delta\omega_c,\kappa_{\rm add}\}$ consistent with $\mathcal Q$ flip. Proof in Appendix C.
\end{theorem}

\section{Proofs}

\subsection{Proof of Theorem A (FRB)}

QED effective action on curved background writable as

$$
\mathcal{L}_{\rm eff}=-\frac{1}{4} F_{\mu\nu}F^{\mu\nu}
+\frac{1}{m_e^2}\left(a R F_{\mu\nu}F^{\mu\nu}
+b R_{\mu\nu}F^{\mu\alpha}F^{\ \nu}_{\alpha}
+c R_{\mu\nu\alpha\beta}F^{\mu\nu}F^{\alpha\beta}\right)+\cdots,
\tag{4.1}
$$

linearized dispersion relation provides refractive index correction $\delta n\propto (\alpha_{\rm em}/\pi)\lambda_e^2 \mathcal R$. On FRW, taking $|\mathcal R|\sim H_0^2/c^2$, integrating along $\chi$ with kernel Eq. (2.4),

$$
\Delta\phi=\int_0^{\chi_s}\frac{\omega_{\rm obs}}{c}\,\delta n(\chi)\,{\rm d}\chi
\le \frac{\omega_{\rm obs} L}{c}\,\max_{\chi}\delta n(\chi),
\tag{4.2}
$$

substituting constants $(\alpha_{\rm em}/\pi)\simeq 2.32\times 10^{-3}$, $\lambda_e=3.8616\times 10^{-13}\ {\rm m}$, $|\mathcal R|\simeq 5.4\times10^{-53}\ {\rm m^{-2}}$ and $\omega_{\rm obs}L/c\simeq 6.46\times10^{26}$ (1 GHz/1 Gpc), obtain $\Delta\phi\approx 1.2\times 10^{-53}$ rad. Hollowood--Shore worldline--Penrose limit guarantees $n(\omega\to\infty)\to1$; this paper's constructed windowed kernel satisfies this causality/analyticity constraint.

\subsection{Proof of Theorem B ($\delta$-Ring)}

From straight-line $\delta$ potential scattering amplitude $t=\dfrac{k}{k+i\alpha_\delta}$ and $|t|=\dfrac{1}{\sqrt{1+(\alpha_\delta/k)^2}}$, expanding transfer matrix trace condition $\mathrm{tr}\,T=2\cos\theta$ yields

$$
\cos\theta=\cos(kL)+\frac{\alpha_\delta}{k}\sin(kL).
\tag{4.3}
$$

On the other hand

$$
\cos(kL)+\frac{\alpha_\delta}{k}\sin(kL)
=\Re\!\Big[(\cos kL+i\sin kL)\,\Big(1-i\frac{\alpha_\delta}{k}\Big)\Big]
=\sqrt{1+(\alpha_\delta/k)^2}\,\cos\Big(kL-\arctan\frac{\alpha_\delta}{k}\Big),
\tag{4.4}
$$

i.e., $\cos\gamma=|t|\,\cos\theta$, proved.

\subsection{Proof of Theorem C (Identifiability)}

Spectral equation (3.3) contains only combined parameter $\alpha_\delta=mg/\hbar^2$; if $(m_1,g_1)\ne (m_2,g_2)$ exists but $m_1g_1=m_2g_2$, then for any $(\theta,n)$ yields identical $\{k_n\}$. Therefore $\{k_n(\theta)\}$ alone uniquely determines $\alpha_\delta$, while $m$, $g$ non-separable. If adding energy spectrum $E_n=\hbar^2 k_n^2/(2m)$, then

$$
J=\begin{pmatrix}
\partial f/\partial m & \partial f/\partial g\\
\partial E/\partial m & \partial E/\partial g
\end{pmatrix}
=\begin{pmatrix}
\frac{\partial f}{\partial \alpha_\delta}\frac{\partial \alpha_\delta}{\partial m} & \frac{\partial f}{\partial \alpha_\delta}\frac{\partial \alpha_\delta}{\partial g}\\[4pt]
-\frac{\hbar^2 k^2}{2m^2} & 0
\end{pmatrix},
\tag{4.5}
$$

at generic points $\det J\ne 0$ (except degenerate subsets like $k=0$), thus $(m,g)$ decoupling possible, proved.

\subsection{Proof of Theorem D (Sensitivity)}

By implicit function theorem

$$
\frac{\partial k}{\partial \alpha_\delta}=-\frac{f_{\alpha_\delta}}{f_k},
\quad
f_{\alpha_\delta}=\frac{1}{k}\sin(kL),\
f_k=-L\sin(kL)+\alpha_\delta\Big(\frac{L}{k}\cos(kL)-\frac{1}{k^2}\sin(kL)\Big),
\tag{4.6}
$$

obtain Eq. (3.6). Ill-conditioned domain satisfies $f_k=0$, when $|\alpha_\delta|/k\ll 1$ approximately $\sin(kL)\simeq (\alpha_\delta/k)\cos(kL)$, proved.

\subsection{Proof of Theorem E (Finite Width)}

Let narrow potential $V(x)=V_0 u(x/w)$ ($\int u=1$) with $g=V_0 w$ fixed, transfer matrix low-energy expansion at $kw\ll 1$ yields

$$
t^{-1}(k)=1+i\frac{\alpha_\delta}{k}+\eta_2[V]\,(kw)^2+\mathcal O\big((kw)^4\big),
\tag{4.7}
$$

equivalentizing to $\alpha_\delta\to \alpha_\delta^{\rm eff}(k)$ replacement. Calculate second moment for rectangular/Gaussian $u$ comparing with solvable models, obtain $\eta_2=\tfrac{1}{3},\ \tfrac{1}{2}$. Numerical verification in Appendix B Figure B2.

\subsection{Proof of Theorems F and G (Topological Scattering and cQED)}

Class D system topological number via single-end reflection matrix zero-energy criterion $\mathcal Q=\mathrm{sgn}\det r(0)$. Finite length coupling leads to zero mode energy splitting $\propto e^{-L/\xi}$, contact/boundary introduces $1/L$ scaling; under near-unitary assumption, $\det r(0)$ sign flip trajectory $J_c(L)$ thus exhibits Eq. (3.8) shape. On the other hand, input--output gives $r\approx 1-2Z_0Y$, while $Y\propto g_{\rm cav}^2 \chi$; when cavity mode sufficiently detuned from continuum, $\chi(0)$ sign change produces zero crossing in $\{\delta\omega_c,\kappa_{\rm add}\}$, consistent with $\mathcal Q$ flip, proved.

\section{Model Apply}

\subsection{FRB "Windowed Upper Limiter"}

\begin{itemize}
\item \textbf{Kernel and weights}: Adopt $\mathcal K_{\rm FRB}=\omega_{\rm obs}/c$, $\Pi_p(\omega)\in\{1,\ \omega,\ \omega^{-1},\ \log\omega\}$. $C_\phi$ by three-channel bootstrap estimation: off-source, off-band, sidelobe.
\item \textbf{Orthogonal windows}: On discrete frequency axis $\{\log\omega_i\}$, use $C_\phi^{-1}$ weighted Gram--Schmidt obtaining $W_j$, whitening test $\langle W_j,W_\ell\rangle\simeq \delta_{j\ell}$.
\item \textbf{Upper limits}: For windowed parameters $\boldsymbol{\theta}$ adopt GLS and profile-likelihood parallel computing 95\% upper limit band, providing scaling curve with bandwidth/event number and systematic basis envelope; use "whether includes $\omega^{-1}$" as robustness switch taking envelope.
\end{itemize}

\subsection{$\delta$-Ring Inversion and Ill-Conditioning Avoidance}

\begin{itemize}
\item \textbf{Inversion}: With Eq. (3.3) $f=0$ as target, use Eq. (3.6) as Newton step iteration inverting $\hat\alpha_\delta$.
\item \textbf{Ill-conditioned domain}: Plot gray region $|f_k|<\tau$ in $(kL,\alpha_\delta/k)$ plane ($\tau\sim 10^{-2}$), experimental point selection avoids domain.
\item \textbf{Finite width}: For given $w$ and potential shape, use Eq. (3.7) correcting $\hat\alpha_\delta$, propagate $\eta_2$ uncertainty to covariance.
\end{itemize}

\subsection{Topological Extrapolation and cQED Redundancy}

\begin{itemize}
\item \textbf{Extrapolation regression}: Measure $\mathcal Q$ flip $J_c(L_i)$ on $\{L_i\}$, regress $(J_c,\beta,\gamma,\xi)$ via Eq. (3.8), compare "exponential/algebraic/composite" three models using AIC/BIC.
\item \textbf{Near-unitary quality control}: Weight by $\varepsilon_i=|r_i^\dagger r_i-\mathbb 1|$ or enter regression as covariate, report $J_c$ sensitivity to $\varepsilon$.
\item \textbf{cQED consistency}: Under detuning $|\omega_c-\omega_{\rm cont}|\gtrsim 3\kappa$, use $\{\delta\omega_c,\kappa_{\rm add}\}$ zero crossings as redundant evidence, cross-validate with $\mathcal Q$ flip.
\end{itemize}

\section{Engineering Proposals}

\begin{itemize}
\item \textbf{FRB}: Release minimized script loading public baseband examples, complete coherent dedispersion, structure-maximized DM, three-channel bootstrap covariance, weighted window orthonormalization and 95\% upper limit plotting, attach "whether includes $\omega^{-1}$ basis" robustness comparison.
\item \textbf{$\delta$-ring}: Provide $\{k_n(\theta)\}$ inversion $\alpha_\delta$ executable script, plot ill-conditioned domain gray map and $\partial k/\partial\alpha_\delta$ contours; provide finite width correction and error propagation example.
\item \textbf{Topological + cQED}: Release scripts for extracting $\mathcal Q$ from $r(\omega)$, regressing $J_c(L)$ with model comparison, and $\{\delta\omega_c,\kappa_{\rm add}\}$ consistency test with $\mathcal Q$.
\end{itemize}

\section{Discussion (Risks, Boundaries, Past Work)}

\textbf{FRB}: At FRW background order of magnitude, one-loop vacuum polarization phase effect unmeasurable; this paper positions as "windowed upper limiter", emphasizing kernel--metric selection consistency and systematic envelope, avoiding bias setting.

\textbf{$\delta$-ring}: Strictly limit "pure phase closure" to limit $|t|\to 1$, avoiding misuse at generic $\theta$; finite width equivalentization requires use within $kw\ll1$ region.

\textbf{Topological}: Near-unitary threshold and model selection key to extrapolation reliability, recommend using residual structure and cross-validation suppressing overfitting.

\section{Conclusion}

Using unified phase--frequency readout and "kernel--windowing--GLS/Fisher" syntax, established reproducible metrological recipes for three platforms: FRB side provides rigorous windowed upper limiter; $\delta$-ring side proves spectral--scattering amplitude-corrected phase equivalence, structural identifiability and closed-form sensitivity with finite width correction; topological scattering side implements critical coupling metrology via near-unitary extrapolation and cQED redundancy. Supporting engineering protocols and minimal scripts make paradigm easily transferable to similar systems.

\section*{Acknowledgements, Code Availability}

Thank related public data and literature providing reproducible benchmarks. Supporting code and minimal data examples released under general license, including environment lockfiles, unit checks, and injection--recovery tests. FRB scripts with $\delta$-ring, topological extrapolation scripts adopt unified I/O conventions and random seeds ensuring reproducibility. Data usage and citation follow respective dataset policies.

\section*{References}

1. Drummond, I. T., Hathrell, S. J., "QED vacuum polarization in a background gravitational field," \emph{Phys. Rev. D} \textbf{22}, 343 (1980).

2. Hollowood, T. J., Shore, G. M., "The causal structure of QED in curved spacetime: Analyticity and the refractive index," \emph{JHEP} \textbf{12}, 091 (2008); related works (2008--2009).

3. CHIME/FRB Collaboration, "Updating the First CHIME/FRB Catalog of Fast Radio Bursts with Baseband Data," \emph{ApJ} \textbf{969}, 145 (2024).

4. Hessels, J. W. T., \emph{et al.}, "FRB 121102 Bursts Show Complex Time--Frequency Structure," \emph{ApJL} \textbf{876}, L23 (2019).

5. Castillo-Sánchez, M., Gutiérrez-Vega, J. C., "Quantum solutions for the delta ring and delta shell," \emph{Am. J. Phys.} \textbf{93}, 557--565 (2025).

6. Fülöp, T., Tsutsui, I., "A Free Particle on a Circle with Point Interaction," \emph{Phys. Lett. A} \textbf{264}, 366--374 (2000).

7. Fulga, I., Hassler, F., Akhmerov, A. R., Beenakker, C. W. J., "Scattering formula for the topological quantum number of a disordered multi-mode wire," \emph{Phys. Rev. B} \textbf{83}, 155429 (2011).

8. Araya Day, J., \emph{et al.}, "Identifying biases of the Majorana scattering invariant," \emph{SciPost Phys. Core} \textbf{8}, 047 (2025).

9. Dmytruk, O., Trif, M., "Microwave detection of gliding Majorana zero modes," \emph{Phys. Rev. B} \textbf{107}, 115418 (2023).

10. Rife, D. C., Boorstyn, R. R., "Single-tone parameter estimation from discrete-time observations," \emph{IEEE Trans. Inf. Theory} \textbf{20}, 591--598 (1974).

\appendix

\section{FRB: One-Loop Vacuum Polarization Scale, Kernel--Metric Consistency, and Windowed Upper Limit}

\subsection{One-Loop Effective Action and Scale}

From Eq. (4.1) linearize Maxwell equations, under weak curvature and geometric optics approximation, dispersion relation $k^2=n^2\omega^2/c^2$ yields

$$
n(\omega,\chi)=1+\delta n(\chi)+\mathcal O(\mathcal R^2),\qquad
\delta n(\chi)\approx\frac{\alpha_{\rm em}}{\pi}\lambda_e^2\,\mathcal R(\chi).
$$

FRW background $R_{\mu\nu\alpha\beta}$ low-order contractions provide $|\mathcal R|\sim H_0^2/c^2$.

\subsection{Kernel--Metric Consistency (Redshift Cancellation)}

Phase increment along geodesic

$$
{\rm d}\phi=\frac{\omega_{\rm loc}}{c}\,\delta n(\chi)\,{\rm d}\ell
=\frac{(1+z)\omega_{\rm obs}}{c}\,\delta n(\chi)\,a(\eta)\,{\rm d}\chi
=\frac{\omega_{\rm obs}}{c}\,\delta n(\chi)\,{\rm d}\chi,
$$

i.e., when integrating in conformal coordinates $(1+z)$ and $a$ exactly cancel, obtaining kernel Eq. (2.4). This form automatically satisfies causality requirement $n(\omega\to\infty)\to 1$.

\subsection{Fisher/GLS Windowed Upper Limit}

Expand $\delta n(\chi)$ on piecewise constant basis, define $A_{j\mu}=\langle W_j, \omega_{\rm obs}/c\rangle \Delta\chi_\mu$, then

$$
F=A^\top A,\qquad
\boldsymbol{\sigma}_{\rm UL}=1.96\,\sqrt{{\rm diag}(F^{-1})},
$$

take upper limit envelope over different systematic basis choices (whether includes $\omega^{-1}$, $\log\omega$).

\section{$\delta$-Ring: Boundary Conditions, Equivalence, Identifiability, Sensitivity, and Finite Width}

\subsection{Boundary Condition and Trigonometric Equation}

Straight-line $\delta$ potential boundary condition

$$
\psi'(0^+)-\psi'(0^-)=2\alpha_\delta\,\psi(0),\qquad \alpha_\delta=mg/\hbar^2,
$$

combined with free propagation segment transfer matrix yields single-lap trace condition $\mathrm{tr}\,T=2\cos\theta$, simplifying to Eq. (3.3).

\subsection{Equivalence Three-Line Derivation}

Using

$$
\Re\Big[e^{ikL}\Big(1-i\frac{\alpha_\delta}{k}\Big)\Big]
=\cos(kL)+\frac{\alpha_\delta}{k}\sin(kL)
=\sqrt{1+(\alpha_\delta/k)^2}\,\cos\Big(kL-\arctan\frac{\alpha_\delta}{k}\Big),
$$

obtain Eq. (3.4), where $|t|=(1+(\alpha_\delta/k)^2)^{-1/2}$.

\subsection{Identifiability (Rank)}

Construct observation equations $\{f=0,E=\hbar^2k^2/(2m)\}$ Jacobian for $(m,g)$, avoiding degenerate strips like $k=0$ and strong scattering ensures full rank (Eq. 4.5).

\subsection{Sensitivity and Ill-Conditioned Domain}

From Eq. (4.6) obtain implicit function derivative and ill-conditioning condition $f_k=0$. Experimental design uses criterion $|f_k|>\tau$ for point selection, $\tau\sim 10^{-2}$--$10^{-3}$.

\subsection{Finite Width Equivalentization and Numerical Comparison}

For $V(x)=V_0 u(x/w)$, under low-energy expansion at $kw\ll 1$,

$$
t^{-1}(k)=1+i\frac{\alpha_\delta}{k}+\eta_2[V]\,(kw)^2+\cdots,
$$

where

$$
\eta_2[V]=\frac{1}{w^2}\,\frac{\int x^2 V(x)\,{\rm d}x}{\int V(x)\,{\rm d}x}\ \ \text{(after normalization and comparison calibration)}.
$$

Numerical scan verification for rectangular ($\eta_2=1/3$) and Gaussian ($\eta_2=1/2$) potentials, relative error $\le 1\%$ in $kw\le 0.2$ region. Solvable model results consistent with expansion in overlap region.

\section{Topological Scattering Invariant and cQED}

\subsection{$\mathcal Q=\mathrm{sgn}\det r(0)$ Scattering Criterion}

Under class D symmetry constraint, $\mathcal Q$ via single-end reflection matrix zero-energy determinant sign; boundary/contact and finite size introduced bias under near-unitarity viewable as translation of $J_c$.

\subsection{Extrapolation Regression and Model Comparison}

Fit $J_c(L)$ using Eq. (3.8), compare "exponential/algebraic/composite" three models via AIC/BIC; when $\varepsilon=|r^\dagger r-\mathbb 1|>\varepsilon_\star$, incorporate into regression as weight or covariate relaxing uncertainty.

\subsection{cQED Linear Response Mapping}

Single-port $r\simeq 1-2Z_0Y$, while $Y\propto g_{\rm cav}^2\chi$; when $|\omega_c-\omega_{\rm cont}|\gtrsim 3\kappa$, $\chi$ sign change at zero energy produces zero crossing in $\delta\omega_c$ and $\kappa_{\rm add}$, consistent with $\mathcal Q$ flip. Instantiated microwave--Majorana scheme serves as style reference.

\end{document}


