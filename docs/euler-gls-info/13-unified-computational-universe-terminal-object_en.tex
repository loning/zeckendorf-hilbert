\documentclass[12pt]{article}

% Essential packages
\usepackage[utf8]{inputenc}
\usepackage{amsmath,amssymb,amsthm}
\usepackage{mathrsfs}
\usepackage{geometry}
\usepackage{hyperref}

% Geometry settings
\geometry{a4paper, margin=1in}

% Hyperref settings
\hypersetup{
    colorlinks=true,
    linkcolor=blue,
    citecolor=blue,
    urlcolor=blue
}

% Theorem environments
\theoremstyle{plain}
\newtheorem{theorem}{Theorem}[section]
\newtheorem{lemma}[theorem]{Lemma}
\newtheorem{proposition}[theorem]{Proposition}
\newtheorem{corollary}[theorem]{Corollary}

\theoremstyle{definition}
\newtheorem{definition}[theorem]{Definition}
\newtheorem{example}[theorem]{Example}
\newtheorem{remark}[theorem]{Remark}

% Math operators
\DeclareMathOperator{\id}{id}
\DeclareMathOperator{\Id}{Id}

% Title information
\title{Unified Computational Universe Terminal Object:\\
Discrete Complexity Geometry, Information Geometry,\\
Multi-Observer Causal Network, and Capability--Risk Structure}
\author{Haobo Ma$^1$ \and Wenlin Zhang$^2$\\
\small $^1$Independent Researcher\\
\small $^2$National University of Singapore}

\date{\today}

\begin{document}

\maketitle

\begin{abstract}
This paper constructs on foundation of previous ``computational universe'' series works a \textbf{unified computational universe terminal object} with explicit categorical meaning. Previous works have axiomatized computational universe as four-tuple

$$
U_{\mathrm{comp}}
=
(X,\mathsf{T},\mathsf{C},\mathsf{I}),
$$

and established on it: discrete complexity geometry (complexity distance, volume growth, and discrete Ricci curvature), discrete information geometry (task information manifold $(\mathcal{S}_Q,g_Q)$ and embedding $\Phi_Q$), control manifold $(\mathcal{M},G)$ induced by unified time scale and time--information--complexity joint variational principle, multi-observer consensus geometry and causal network, topological complexity and undecidability, as well as theory of universal catastrophic safety and capability--risk frontier.

Goal of this paper is to unify these discrete and continuous, geometric and logical, single-observer and multi-observer, capability and risk structures into single categorical object

$$
\mathfrak{U}_{\mathrm{comp}}^{\mathrm{term}}
$$

---called \textbf{unified computational universe terminal object}. Specifically, we perform following steps:

\begin{enumerate}
\item Define computational universe category with unified time scale $\mathbf{CompUniv}_\kappa$: objects are computational universes $U_{\mathrm{comp}}$ satisfying axioms, morphisms are ``safe simulation maps'' simultaneously preserving complexity geometry, information geometry, and catastrophe specifications.

\item Construct 2-layer structure on this category: one layer is discrete configuration--event--causal diamond layer; one layer is continuous control--information geometry layer, with multi-observer network, knowledge graph families, and capability--risk frontier placed on top.

\item Prove there exists object

$$
\mathfrak{U}_{\mathrm{comp}}^{\mathrm{term}}
=
\big(
\mathfrak{X},\,
\mathfrak{G}_{\mathrm{comp}},\,
\mathfrak{G}_{\mathrm{info}},\,
\mathfrak{E}_{\mathrm{obs}},\,
\mathfrak{S}_{\mathrm{cat}},\,
\mathfrak{F}_{\mathrm{CR}}
\big),
$$

and for each $U_{\mathrm{comp}} \in \mathbf{CompUniv}_\kappa$ ``contraction'' morphism

$$
F_{U} : U_{\mathrm{comp}} \to \mathfrak{U}_{\mathrm{comp}}^{\mathrm{term}},
$$

such that:

\begin{itemize}
\item $F_U$ at discrete level is embedding of configuration--event--diamond, at continuous level is embedding of control--information--observer states;
\item $F_U$ preserves complexity distance and unified time scale (at most linear rescaling);
\item $F_U$ makes all task information geometry and multi-observer consensus geometry become some class of ``submanifold--subnetwork'' on $\mathfrak{U}_{\mathrm{comp}}^{\mathrm{term}}$;
\item $F_U$ maps catastrophe specifications and capability--risk frontier to substructures of $\mathfrak{S}_{\mathrm{cat}}$ and $\mathfrak{F}_{\mathrm{CR}}$.
\end{itemize}

\item Prove in natural 2-category sense (allowing natural transformations between morphisms), $\mathfrak{U}_{\mathrm{comp}}^{\mathrm{term}}$ satisfies ``terminal object'' property: for any two such unified objects morphisms between them have unique (in natural isomorphism sense) factorization.

\item Finally using previously established physical universe--computational universe categorical equivalence, construct correspondence between unified physical universe terminal object $\mathfrak{U}_{\mathrm{phys}}^{\mathrm{term}}$ and $\mathfrak{U}_{\mathrm{comp}}^{\mathrm{term}}$, and explain both are equivalent terminal objects under unified time scale, boundary diamonds, observer network, and capability--risk structure.
\end{enumerate}

This paper thus gives ultimate unified description of ``universe as computation'' under purely discrete, axiomatic framework: all concrete finite or local computational universes contract through safe--geometric--information compatible morphisms to same unified computational universe terminal object, which simultaneously plays terminal object role at categorical, geometric, and logical levels.
\end{abstract}

\noindent\textbf{Keywords:} Computational universe; Terminal object; Category theory; Unified time scale; Complexity geometry; Information geometry; Multi-observer consensus; Capability--risk frontier

\section{Introduction}

In previous series works, we have successively constructed axiomatization, discrete complexity geometry, discrete information geometry, unified time scale framework, time--information--complexity joint variational principle, multi-observer consensus geometry, topological complexity and undecidability, universal catastrophic safety, and capability--risk frontier theory for ``computational universe''.

This paper aims to unify all these structures into single categorical object---\textbf{unified computational universe terminal object} $\mathfrak{U}_{\mathrm{comp}}^{\mathrm{term}}$. This object plays ``terminal object'' role in computational universe category: every specific computational universe can be embedded into it through unique (up to natural isomorphism) structure-preserving map, and all previous geometric, information, multi-observer, and safety structures can be viewed as substructures of unified object.

Terminal object construction not only provides ultimate unified description for computational universe theory, but also establishes formal correspondence with physical universe category through categorical equivalence: unified physical universe terminal object $\mathfrak{U}_{\mathrm{phys}}^{\mathrm{term}}$ and unified computational universe terminal object $\mathfrak{U}_{\mathrm{comp}}^{\mathrm{term}}$ are equivalent presentations of same mathematical object in two equivalent categories.

Paper structure: Section 2 organizes previous scattered structures into 2-category framework with 2-morphisms. Section 3 gives structural data of unified computational universe terminal object: from discrete configuration--event--diamond to continuous control--information--observer--capability--risk. Section 4 defines unified computational universe terminal object and proves its terminal object property in 2-category sense. Section 5 constructs equivalence with unified physical universe terminal object. Appendices give technical construction details and formal proof outlines.

\section{Computational Universe 2-Category Under Unified Time Scale}

This section organizes previous scattered structures into 2-category framework with 2-morphisms.

\subsection{Computational Universe Objects with Unified Time Scale}

\begin{definition}[Computational Universe with Unified Time Scale]
A computational universe object with unified time scale is seven-tuple

$$
\widehat{U}_{\mathrm{comp}}
=
(X,\mathsf{T},\mathsf{C},\mathsf{I};\,
\mathcal{M},G;\,
\mathcal{S}_Q,g_Q),
$$

where:

\begin{enumerate}
\item $(X,\mathsf{T},\mathsf{C},\mathsf{I})$ is computational universe satisfying previous axioms:

\begin{itemize}
\item $X$ countable;
\item $\mathsf{T} \subset X\times X$ local finite degree;
\item $\mathsf{C}$ single-step cost positive and path-additive;
\item $\mathsf{I}$ task information quality baseline.
\end{itemize}

\item $(\mathcal{M},G)$ is control manifold and complexity metric constructed from unified time scale scattering master scale, satisfying Riemannian limit property of discrete complexity distance $d_{\mathrm{comp}}$: for each local reachable region there exists refinement family $\{X^{(h)}\}$ and map $\Phi_h:X^{(h)}\to\mathcal{M}$, such that

$$
d_{\mathrm{comp}}^{(h)}(x,y)
\to
d_G\big(\Phi_h(x),\Phi_h(y)\big).
$$

\item $(\mathcal{S}_Q,g_Q)$ is information manifold and Fisher information metric for task $Q$, with embedding

$$
\Phi_Q:X\to\mathcal{S}_Q,
$$

such that discrete Jensen--Shannon information distance $d_{\mathrm{JS},Q}(x,y)$ locally consistent with $d_{\mathcal{S}_Q}\big(\Phi_Q(x),\Phi_Q(y)\big)$.
\end{enumerate}
\end{definition}

We call such objects constitute ``0-layer objects'' of category $\mathbf{CompUniv}_\kappa$.

\subsection{Safe--Geometric--Information Compatible 1-Morphisms}

\begin{definition}[Safe--Geometric--Information Compatible Simulation Morphism]
Given two computational universes with unified time scale

$$
\widehat{U}_{\mathrm{comp}}
=
(X,\mathsf{T},\mathsf{C},\mathsf{I};\mathcal{M},G;\mathcal{S}_Q,g_Q),
$$

$$
\widehat{U}_{\mathrm{comp}}'
=
(X',\mathsf{T}',\mathsf{C}',\mathsf{I}';\mathcal{M}',G';\mathcal{S}'_Q,g'_Q),
$$

a 1-morphism

$$
F:\widehat{U}_{\mathrm{comp}} \to \widehat{U}_{\mathrm{comp}}'
$$

consists of following data:

\begin{enumerate}
\item Configuration map $f_X:X\to X'$, being previously defined simulation map (preserving step structure, cost control, and information quality monotonicity), with constants $\alpha_X,\beta_X>0$ and monotone function $\Psi$ such that
$$
(x,y)\in\mathsf{T} \Rightarrow (f_X(x),f_X(y))\in\mathsf{T}',
$$
$$
d_{\mathrm{comp}}'(f_X(x),f_X(y)) \le \alpha_X d_{\mathrm{comp}}(x,y)+\beta_X,
$$
$$
\mathsf{I}(x) \le \Psi\big(\mathsf{I}'(f_X(x))\big).
$$

\item Control manifold map $f_{\mathcal{M}}:\mathcal{M}\to\mathcal{M}'$, for metrics $G,G'$ satisfying Lipschitz--bilateral control: there exist $\alpha_{\mathcal{M}},\beta_{\mathcal{M}}>0$ such that
$$
\alpha_{\mathcal{M}} G_\theta(v,v)
\le
G'_{f_{\mathcal{M}}(\theta)}\big( \mathrm{d} f_{\mathcal{M}} v,\mathrm{d} f_{\mathcal{M}} v \big)
\le
\beta_{\mathcal{M}} G_\theta(v,v).
$$

\item Information manifold map $f_{\mathcal{S}}:\mathcal{S}_Q\to\mathcal{S}'_Q$, for Fisher metric satisfying similar Lipschitz--bilateral control, and compatible with $f_X$, i.e., there exists natural transformation $\eta_X$ such that
$$
f_{\mathcal{S}}\circ \Phi_Q
\simeq
\Phi'_Q\circ f_X.
$$

\item Safety specification compatibility: if there exists catastrophe set $C_{\mathrm{cat}} \subset X$ on $\widehat{U}_{\mathrm{comp}}$, then its image $C_{\mathrm{cat}}' = f_X(C_{\mathrm{cat}}) \subset X'$ remains catastrophe set, and $F$ does not map safe points into catastrophe points, i.e., visible safety--catastrophe partition under map preserves or ``blunts toward safety side''.
\end{enumerate}
\end{definition}

Such 1-morphisms simultaneously preserve discrete--continuous geometry and catastrophe specifications. All objects and 1-morphisms constitute category $\mathbf{CompUniv}_\kappa$.

\subsection{2-Morphisms and Natural Transformations}

At control and information manifold level, two 1-morphisms may have ``continuous deformations'', corresponding in category theory to 2-morphisms---natural transformations.

\begin{definition}[Natural Transformation as 2-Morphism]
Given two 1-morphisms

$$
F,G : \widehat{U}_{\mathrm{comp}}\to \widehat{U}_{\mathrm{comp}}',
$$

a 2-morphism

$$
\Xi: F \Rightarrow G
$$

includes:

\begin{enumerate}
\item Configuration side natural transformation $\Xi_X$, usually family of local invertible maps on $X'$, such that $G_X \simeq \Xi_X \circ F_X$;
\item Control and information side natural transformations $\Xi_{\mathcal{M}},\Xi_{\mathcal{S}}$, giving in metric compatible sense $G_{\mathcal{M}} \simeq \Xi_{\mathcal{M}}\circ F_{\mathcal{M}}$, $G_{\mathcal{S}} \simeq \Xi_{\mathcal{S}}\circ F_{\mathcal{S}}$;
\item On safety structure not breaking coarse structure of catastrophe set and capability--risk frontier.
\end{enumerate}
\end{definition}

Thus, $\mathbf{CompUniv}_\kappa$ has 2-category structure.

\section{Structural Data of Unified Computational Universe Terminal Object}

This section gives specific composition of unified computational universe terminal object

$$
\mathfrak{U}_{\mathrm{comp}}^{\mathrm{term}}
$$

from discrete configuration--event--diamond to continuous control--information--observer--capability--risk.

\subsection{Discrete Layer: Maximal Configuration Universe and Event--Diamond Structure}

\begin{definition}[Maximal Configuration Universe]
Let $\mathfrak{X}$ be ``amalgamation of all countable configuration sets and their finite representations'' under some large cardinal control, formally constructible through Grothendieck universe $\mathcal{U}$ and set theory as

$$
\mathfrak{X}
=
\bigcup_{U_{\mathrm{comp}}\in\mathbf{CompUniv}_\kappa}
\iota_U(X),
$$

where $\iota_U$ is embedding of each $X$ into common superset. Define on $\mathfrak{X}$ unified transition relation

$$
\mathsf{T}_\infty
=
\bigcup_{U_{\mathrm{comp}}} \iota_U(\mathsf{T}_U),
$$

cost function

$$
\mathsf{C}_\infty
=
\bigcup_{U_{\mathrm{comp}}} \iota_U(\mathsf{C}_U),
$$

at conflicts through equivalence class identification (i.e., merging geometrically equivalent update rules from different universes into single object). Thus obtain ``maximal global complexity graph'' containing all local computation structures

$$
\mathfrak{G}_{\mathrm{comp}}^{(1)}
=
(\mathfrak{X},\mathsf{T}_\infty,\mathsf{C}_\infty).
$$
\end{definition}

At event layer

$$
\mathfrak{E}
=
\mathfrak{X}\times\mathbb{N},
$$

can define unified causal partial order and complexity light cone; thus any finite budget causal diamond $\Diamond$ can be viewed as subgraph of $\mathfrak{G}_{\mathrm{comp}}^{(1)}$.

\subsection{Continuous Layer: Terminal Objects of Unified Control Manifold and Information Manifold}

At control and information geometry level, previous works already constructed for each $\widehat{U}_{\mathrm{comp}}$ control manifold $(\mathcal{M}_U,G_U)$ and information manifold $(\mathcal{S}_{Q,U},g_{Q,U})$.

\begin{definition}[Unified Control Manifold Terminal Object]
Let

$$
\mathfrak{M}
=
\coprod_{U_{\mathrm{comp}}}
\mathcal{M}_U \Big/ \sim,
$$

where $\sim$ is ``time scale--complexity isometry equivalence'': if there exists isometric embedding of control--scattering realization, identify corresponding points. Through this amalgamation obtain large manifold or stacky object $\mathfrak{M}$, with unified metric $\mathfrak{G}$, locally consistent with each $G_U$.
\end{definition}

\begin{definition}[Unified Information Manifold Terminal Object]
Similarly for all tasks $Q$ and universe $U_{\mathrm{comp}}$, construct information manifold family $\mathcal{S}_{Q,U}$ and perform similar amalgamation, obtaining unified information manifold

$$
\mathfrak{S}
=
\coprod_{Q,U}
\mathcal{S}_{Q,U} \Big/ \sim,
$$

carrying piecewise Fisher metric $\mathfrak{g}$.
\end{definition}

Thus, any concrete computational universe's control--information geometry can be embedded into $(\mathfrak{M},\mathfrak{G})$ and $(\mathfrak{S},\mathfrak{g})$ as submanifolds; under unified time scale master scale and scattering structure, these embeddings preserve geodesic structure and information structure.

\subsection{Multi-Observer Network Layer}

Taking single observer object

$$
O = (M_{\mathrm{int}},\Sigma_{\mathrm{obs}},\Sigma_{\mathrm{act}},\mathcal{P},\mathcal{U})
$$

as basic unit, view all countable observer families in unified computational universe as collection of points $(\theta,\phi,m,\mathcal{G},A)$.

\begin{definition}[Unified Observer State Space]
Define

$$
\mathfrak{E}_{\mathrm{obs}}
=
\bigcup_{U_{\mathrm{comp}},\mathcal{O}}
\Big(
\prod_{i\in I} \mathcal{E}_Q^{(i)}
\times
M_{\mathrm{int}}^{(i)}
\times
\mathfrak{G}^{(i)}
\times
\mathfrak{A}^{(i)}
\Big) \Big/ \sim,
$$

where $\mathcal{E}_Q^{(i)} = \mathcal{M}_U^{(i)}\times\mathcal{S}_{Q,U}^{(i)}$, $\mathfrak{G}^{(i)}$ is knowledge graph space, $\mathfrak{A}^{(i)}$ is attention configuration space, $\sim$ identifies geometrically equivalent and strategy equivalent states.

Joint metric composed of sum of each $G_U,g_{Q,U}$ and knowledge graph spectral distance.
\end{definition}

\subsection{Catastrophe Specification and Capability--Risk Frontier Layer}

Perform similar amalgamation for catastrophe specifications and capability--risk structures defined in all computational universes.

\begin{definition}[Unified Catastrophe Specification Layer]
Define catastrophe specification family

$$
\mathfrak{S}_{\mathrm{cat}}
=
\bigcup_{U_{\mathrm{comp}}}
\{(X_{0,U},C_{\mathrm{cat},U})\}
\Big/ \sim,
$$

where equivalence relation identifies equivalent initial state sets and catastrophe sets under unified embedding.
\end{definition}

\begin{definition}[Unified Capability--Risk Frontier Layer]
For each $(U_{\mathrm{comp}},Q)$, capability--risk frontier is $\mathcal{F}_{\mathrm{CR}}^{(U,Q)}\subset\mathbb{R}\times[0,1]$. Embedding all these frontiers as parametrized family into unified strategy space, obtain overall structure

$$
\mathfrak{F}_{\mathrm{CR}}
=
\bigcup_{U,Q} \mathcal{F}_{\mathrm{CR}}^{(U,Q)} \times \{(U,Q)\} \Big/ \sim.
$$

On unified control--observer strategy space, $\mathfrak{F}_{\mathrm{CR}}$ is piecewise Pareto boundary set, describing capability--risk limit curves in all possible computational universes.
\end{definition}

\section{Unified Computational Universe Terminal Object and Terminal Object Property}

This section merges all above layers, gives definition of unified computational universe terminal object, and proves its terminal object property in 2-category sense.

\subsection{Definition of Terminal Object}

\begin{definition}[Unified Computational Universe Terminal Object]
Unified computational universe terminal object defined as ten-tuple

$$
\mathfrak{U}_{\mathrm{comp}}^{\mathrm{term}}
=
\big(
\mathfrak{X},\,
\mathsf{T}_\infty,\,
\mathsf{C}_\infty,\,
\mathsf{I}_\infty;\,
\mathfrak{M},\mathfrak{G};\,
\mathfrak{S},\mathfrak{g};\,
\mathfrak{E}_{\mathrm{obs}};\,
\mathfrak{S}_{\mathrm{cat}};\,
\mathfrak{F}_{\mathrm{CR}}
\big),
$$

where each part defined as in 3.1--3.6.
\end{definition}

Intuitively, $\mathfrak{U}_{\mathrm{comp}}^{\mathrm{term}}$ simultaneously contains:

\begin{itemize}
\item All locally realizable configurations and updates;
\item All control path complexity geometry limits under unified time scale;
\item All task information geometries and observer networks;
\item All catastrophe specifications and capability--risk frontiers.
\end{itemize}

\subsection{Unified Contraction Functor}

\begin{definition}[Contraction 1-Morphism]
For each $\widehat{U}_{\mathrm{comp}} \in \mathbf{CompUniv}_\kappa$, define

$$
F_U : \widehat{U}_{\mathrm{comp}} \to \mathfrak{U}_{\mathrm{comp}}^{\mathrm{term}}
$$

as follows:

\begin{enumerate}
\item Configuration embedding
$$
f_X = \iota_U : X \hookrightarrow \mathfrak{X}.
$$

\item Control manifold embedding
$$
f_{\mathcal{M}} : \mathcal{M}_U \hookrightarrow \mathfrak{M},
$$
preserving metric structure (at most constant rescaling).

\item Information manifold embedding
$$
f_{\mathcal{S}} : \mathcal{S}_{Q,U} \hookrightarrow \mathfrak{S},
$$
preserving Fisher information metric.

\item Observer and knowledge graph embedding
$$
f_{\mathrm{obs}} : \text{ObsStates}(U_{\mathrm{comp}}) \hookrightarrow \mathfrak{E}_{\mathrm{obs}}.
$$

\item Catastrophe specification and capability--risk frontier embedding
$$
f_{\mathrm{cat}} : (X_0,C_{\mathrm{cat}}) \mapsto \mathfrak{S}_{\mathrm{cat}},
\quad
f_{\mathrm{CR}} : \mathcal{F}_{\mathrm{CR}}^{(U,Q)} \mapsto \mathfrak{F}_{\mathrm{CR}}.
$$
\end{enumerate}

Combining these $F_U$ is safe--geometric--information compatible 1-morphism.
\end{definition}

\subsection{Terminal Object Property}

\begin{theorem}[2-Terminal Object Property of Unified Computational Universe Terminal Object]
\label{thm:terminal-property}
In 2-category $\mathbf{CompUniv}_\kappa$, $\mathfrak{U}_{\mathrm{comp}}^{\mathrm{term}}$ is 2-terminal object:

For each object $\widehat{U}_{\mathrm{comp}}$, there exists 1-morphism

$$
F_U : \widehat{U}_{\mathrm{comp}} \to \mathfrak{U}_{\mathrm{comp}}^{\mathrm{term}},
$$

and for any other 1-morphism

$$
G_U : \widehat{U}_{\mathrm{comp}} \to \mathfrak{U}_{\mathrm{comp}}^{\mathrm{term}},
$$

there exists unique (in 2-morphism sense) natural transformation

$$
\Xi_U : G_U \Rightarrow F_U.
$$
\end{theorem}

\begin{proof}[Proof (Outline)]

\begin{enumerate}
\item \textbf{Existence}: By construction in 4.2, for each $\widehat{U}_{\mathrm{comp}}$ can explicitly define $F_U$ as embedding.

\item \textbf{Uniqueness (up to natural transformation)}: Any other 1-morphism $G_U$ must map $X$ to some subset of $\mathfrak{X}$, and preserve structure at control--information--observer--capability--risk layers. Since $\mathfrak{U}_{\mathrm{comp}}^{\mathrm{term}}$ constructed by ``maximal global equivalence class'', for any such $G_U$ there exists ``isometry--equivalence class'' natural transformation $\Xi_U$ pulling it back to canonical embedding $F_U$: at each layer, can eliminate redundant degrees of freedom through internal isometry or local unitary transformation, making map isomorphic to $F_U$.

\item \textbf{Naturality}: For morphism $H:\widehat{U}_{\mathrm{comp}}\to\widehat{V}_{\mathrm{comp}}$, natural transformation between two-side composition $F_V\circ H$ and $F_U$ given by unified construction---this is standard feature of ``inclusion--isometry'' structure in Grothendieck universe.
\end{enumerate}

Rigorous 2-category proof requires checking compatibility at each layer, see Appendix C.
\end{proof}

\section{Equivalence with Unified Physical Universe Terminal Object}

Previous works already constructed physical universe category $\mathbf{PhysUniv}^{\mathrm{QCA}}$ and computational universe category $\mathbf{CompUniv}^{\mathrm{phys}}$, and through functors

$$
F : \mathbf{PhysUniv}^{\mathrm{QCA}} \to \mathbf{CompUniv}^{\mathrm{phys}},
\quad
G : \mathbf{CompUniv}^{\mathrm{phys}} \to \mathbf{PhysUniv}^{\mathrm{QCA}}
$$

proved both are categorically equivalent. At physical level, unified physical universe terminal object $\mathfrak{U}_{\mathrm{phys}}^{\mathrm{term}}$ already constructed through scattering time scale, boundary time geometry, and Dirac--QCA continuous limit.

In current scenario, from inclusion $\mathbf{CompUniv}^{\mathrm{phys}} \hookrightarrow \mathbf{CompUniv}_\kappa$ and categorical equivalence, can lift terminal object relationship:

\begin{proposition}[Equivalence Between Terminal Objects]
\label{prop:terminal-equivalence}
Under constraints of unified time scale, unified physical universe terminal object $\mathfrak{U}_{\mathrm{phys}}^{\mathrm{term}}$ and unified computational universe terminal object $\mathfrak{U}_{\mathrm{comp}}^{\mathrm{term}}$ are equivalent in 2-category sense.

That is, there exists pair of 1-morphisms

$$
\mathcal{F} : \mathfrak{U}_{\mathrm{phys}}^{\mathrm{term}}
\to
\mathfrak{U}_{\mathrm{comp}}^{\mathrm{term}},
\quad
\mathcal{G} : \mathfrak{U}_{\mathrm{comp}}^{\mathrm{term}}
\to
\mathfrak{U}_{\mathrm{phys}}^{\mathrm{term}},
$$

and natural isomorphism 2-morphisms

$$
\mathcal{G}\circ\mathcal{F} \simeq \Id_{\mathfrak{U}_{\mathrm{phys}}^{\mathrm{term}}},
\quad
\mathcal{F}\circ\mathcal{G} \simeq \Id_{\mathfrak{U}_{\mathrm{comp}}^{\mathrm{term}}}.
$$
\end{proposition}

Proof relies on previous categorical equivalence theorem and terminal object property, see Appendix D.

This means: ``universe as computation'' unified terminal object and ``universe as unified time--boundary geometry'' terminal object are same 2-terminal object in different categorical presentations: one with discrete computation structure as primary, one with continuous geometric structure as primary.

\appendix

\section{Technical Details of Unified Configuration--Event--Diamond Layer}

\subsection{Set-Theoretic Construction of Maximal Configuration Universe}

To avoid size paradoxes, we operate within some Grothendieck universe $\mathcal{U}$, let

$$
\mathcal{X}_0
=
\{X_U : U\in\mathbf{CompUniv}_\kappa\},
\quad
\mathcal{T}_0
=
\{\mathsf{T}_U,\mathsf{C}_U,\mathsf{I}_U\}.
$$

Define

$$
\mathfrak{X}
=
\bigsqcup_{U} (X_U\times\{U\}),
$$

then through isometry--equivalence relation $\sim$ identify ``geometrically completely identical'' point pairs, obtaining quotient set

$$
\mathfrak{X}
=
\Big(
\bigsqcup_{U} X_U
\Big) \Big/ \sim.
$$

Unified transition relation $\mathsf{T}_\infty$ and cost $\mathsf{C}_\infty$ obtained through similar quotient construction. This process ensures:

\begin{itemize}
\item Each concrete $X_U$ embedded into $\mathfrak{X}$;
\item In isometry equivalence classes, do not repeatedly count same update structures;
\item Unified complexity distance $d_{\mathrm{comp},\infty}$ restricts to original distance on each subset.
\end{itemize}

\subsection{Unified Event Layer and Diamonds}

On event layer

$$
\mathfrak{E} = \mathfrak{X}\times\mathbb{N}
$$

unified update relation

$$
\mathsf{T}_{\mathfrak{E}}
=
\{((x,k),(y,k+1)) : (x,y)\in\mathsf{T}_\infty\}
$$

and cost function induce complexity light cone and diamond definitions completely analogous to single universe case.

\section{Construction of Unified Control and Information Manifold Terminal Objects}

\subsection{Local Construction of Control Manifold Terminal Object}

For each $U_{\mathrm{comp}}$, control manifold $\mathcal{M}_U$ locally isomorphic to some Riemannian manifold $(\mathbb{R}^{d_U},G_U)$. We perform disjoint union over all such local patches, then amalgamate by ``time scale--scattering--control structure isometry'' relation, obtaining unified control manifold $\mathfrak{M}$.

Technically, can define equivalence relation:

$$
(\theta,U) \sim (\theta',U')
\iff
\exists\ \text{isometric isomorphism}\ \varphi:
\mathcal{M}_U\supset U \to U'\subset\mathcal{M}_{U'}
\ \text{with}\ \varphi(\theta)=\theta'.
$$

Metric $\mathfrak{G}$ defined on equivalence class representatives by $G_U$, isometry equivalence ensures consistency.

\subsection{Construction of Information Manifold Terminal Object}

For all $Q,U$, information manifold $(\mathcal{S}_{Q,U},g_{Q,U})$ can be determined through relative entropy second-order structure. Same as control manifold, through disjoint union and Fisher--isometry equivalence relation amalgamation obtain $(\mathfrak{S},\mathfrak{g})$.

\section{Formal Proof Outline of Terminal Object 2-Property}

\subsection{Uniqueness of 1-Morphism (Up to Natural Transformation)}

Given $\widehat{U}_{\mathrm{comp}}$ and two 1-morphisms

$$
F_U,G_U : \widehat{U}_{\mathrm{comp}} \to \mathfrak{U}_{\mathrm{comp}}^{\mathrm{term}},
$$

compare their maps at each layer:

\begin{itemize}
\item At configuration layer, from isometry--equivalence construction of $\mathfrak{X}$, images of $F_U$ and $G_U$ must fall in same equivalence class, thus there exists some local bijection $\Xi_X$ such that $G_X = \Xi_X\circ F_X$;
\item At control and information layers, from isometry--equivalence construction of $\mathfrak{M},\mathfrak{S}$, there exist isometric natural transformations $\Xi_{\mathcal{M}},\Xi_{\mathcal{S}}$ between $F_{\mathcal{M}},G_{\mathcal{M}}$ and $F_{\mathcal{S}},G_{\mathcal{S}}$;
\item At observer and capability--risk layers, similar amalgamation ensures existence of corresponding natural transformations.
\end{itemize}

Combining these natural transformations into 2-morphism $\Xi_U$ completes.

\subsection{2-Terminal Object Property}

If there exists another candidate terminal object $\mathfrak{U}'_{\mathrm{comp}}$, then 1-morphism $H:\mathfrak{U}_{\mathrm{comp}}^{\mathrm{term}}\to\mathfrak{U}'_{\mathrm{comp}}$ starting from $\mathfrak{U}_{\mathrm{comp}}^{\mathrm{term}}$ and 1-morphism $K$ from $\mathfrak{U}'_{\mathrm{comp}}$ to $\mathfrak{U}_{\mathrm{comp}}^{\mathrm{term}}$ must compose to identity in 2-morphism sense, because both have ``inclusion--amalgamation--isometry'' structure. This is standard 2-terminal object argument.

\section{Equivalence of Unified Physical Universe Terminal Object and Unified Computational Universe Terminal Object}

At physical level, unified physical universe terminal object

$$
\mathfrak{U}_{\mathrm{phys}}^{\mathrm{term}}
=
(M_\infty,g_\infty;\,
\mathcal{F}_\infty;\,
\kappa_\infty;\,
\mathsf{S}_\infty;\,
\cdots)
$$

constructed through similar Grothendieck amalgamation in physical universe category $\mathbf{PhysUniv}^{\mathrm{QCA}}$. Previously given categorical equivalence

$$
F : \mathbf{PhysUniv}^{\mathrm{QCA}} \to \mathbf{CompUniv}^{\mathrm{phys}},
\quad
G : \mathbf{CompUniv}^{\mathrm{phys}} \to \mathbf{PhysUniv}^{\mathrm{QCA}}.
$$

By extending these functors to terminal object level, can construct

$$
\mathcal{F} : \mathfrak{U}_{\mathrm{phys}}^{\mathrm{term}}
\to
\mathfrak{U}_{\mathrm{comp}}^{\mathrm{term}},
\quad
\mathcal{G} : \mathfrak{U}_{\mathrm{comp}}^{\mathrm{term}}
\to
\mathfrak{U}_{\mathrm{phys}}^{\mathrm{term}},
$$

and using terminal object property and equivalence, obtain natural isomorphisms

$$
\mathcal{G}\circ\mathcal{F} \simeq \Id_{\mathfrak{U}_{\mathrm{phys}}^{\mathrm{term}}},
\quad
\mathcal{F}\circ\mathcal{G} \simeq \Id_{\mathfrak{U}_{\mathrm{comp}}^{\mathrm{term}}}.
$$

This fundamentally shows:

\begin{itemize}
\item Viewing universe as ``scattering--boundary geometry terminal object under unified time scale'' and
\item Viewing universe as ``computational universe terminal object under unified time scale''
\end{itemize}

are same mathematical object in dual presentation in two equivalent categories.

\end{document}
