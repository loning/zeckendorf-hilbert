\documentclass[12pt]{article}

% Essential packages
\usepackage[utf8]{inputenc}
\usepackage{amsmath,amssymb,amsthm}
\usepackage{mathrsfs}
\usepackage{geometry}
\usepackage{hyperref}

% Geometry settings
\geometry{a4paper, margin=1in}

% Hyperref settings
\hypersetup{
    colorlinks=true,
    linkcolor=blue,
    citecolor=blue,
    urlcolor=blue
}

% Theorem environments
\theoremstyle{plain}
\newtheorem{theorem}{Theorem}[section]
\newtheorem{lemma}[theorem]{Lemma}
\newtheorem{proposition}[theorem]{Proposition}
\newtheorem{corollary}[theorem]{Corollary}

\theoremstyle{definition}
\newtheorem{definition}[theorem]{Definition}
\newtheorem{example}[theorem]{Example}
\newtheorem{remark}[theorem]{Remark}

% Math operators
\DeclareMathOperator{\tr}{tr}
\DeclareMathOperator{\sinc}{sinc}
\DeclareMathOperator{\Span}{span}
\DeclareMathOperator{\ONB}{ONB}

% Title information
\title{Phase--Frequency Unified Metrology and Experimental Testbeds\\
in Computational Universe:\\
Unified Time Scale Implementation\\
from FRB Vacuum Windowing Upper Limit\\
to $\delta$-Ring Scattering Identifiability}
\author{Haobo Ma$^1$ \and Wenlin Zhang$^2$\\
\small $^1$Independent Researcher\\
\small $^2$National University of Singapore}

\date{\today}

\begin{document}

\maketitle

\begin{abstract}
In previous ``computational universe'' framework, universe axiomatized as discrete object $U_{\mathrm{comp}} = (X,\mathsf{T},\mathsf{C},\mathsf{I})$, upon which constructed discrete complexity geometry, discrete information geometry, control manifold $(\mathcal{M},G)$ induced by unified time scale, task information manifold $(\mathcal{S}_Q,g_Q)$, and time--information--complexity joint variational principle. Unified time scale given by scattering master scale

$$
\kappa(\omega) = \varphi'(\omega)/\pi = \rho_{\mathrm{rel}}(\omega) = (2\pi)^{-1}\tr Q(\omega)
$$

unifying phase derivative, spectral shift density, and Wigner--Smith group delay trace as single scale. However, this framework still remains mainly at ``theoretical geometry'' level, has not systematically given how to metrologically measure and calibrate unified time scale and computational universe structure in actual experiments.

This paper, on basis of computational universe--unified time scale--spectral windowing readout, constructs cross-platform metrology paradigm using ``phase--frequency'' as sole observable, implementing it on two representative testbeds: cosmological-distance Fast Radio Burst propagation (FRB) and laboratory-scale $\delta$-ring--Aharonov--Bohm (AB) flux scattering. Core idea: from computational universe perspective, all observables realized through phase--frequency readout under unified time scale; FRB and $\delta$-ring scattering respectively provide cosmic-scale and laboratory-scale ``homologous readouts'', viewable as implementations of same metrology paradigm at different scales under unified time scale and complexity geometry.

Main results of this paper:

\begin{enumerate}
\item Under framework of categorical equivalence between computational--physical universes, introduce ``phase--frequency readout functor'' $\mathsf{PhFr}$, sending any physically realizable computational universe object to metrology object containing only phase--frequency data. Prove $\mathsf{PhFr}$ compatible with unified time scale master scale: under traceable perturbation and wave operator completeness, $\mathsf{PhFr}$ output completely determined by $\kappa(\omega)$ and finite spectral--scattering invariants.

\item For FRB, construct ``vacuum polarization windowing upper limit'' model: window FRB frequency-domain phase using PSWF/DPSS type window functions, prove under fixed complexity budget and cosmological distance constraints, any unified time scale variation $\delta \kappa(\omega)$ contribution to FRB phase residual can be bounded by strict upper bound; if observed residual below this bound, obtain unified time scale type upper limit on vacuum polarization or other new physics.

\item For $\delta$-ring--AB flux scattering, restate equivalence between spectral quantization equation

$$
f(k,\alpha_\delta,\theta) = \cos(kL) + (\alpha_\delta/k)\sin(kL) - \cos\theta = 0
$$

and ``amplitude-corrected phase closure''

$$
\cos\gamma(k) = |t(k)|\cos\theta
$$

and prove under computational universe--control manifold perspective: under spectral observation $\{k_n(\theta)\}$ at fixed $(L,\theta)$, $\delta$--coupling strength $\alpha_\delta$ and AB flux $\theta$ are identifiable in non-pathological domain (Jacobian full rank), usable as ``laboratory ruler'' for unified time scale--phase metrology.

\item Under unified time scale--spectral windowing readout framework, embed FRB and $\delta$-ring scattering in same ``phase--frequency metrology universe'', prove existence of ``cross-platform scale unification condition'': when FRB phase residual and $\delta$-ring scattering spectral shift both explained by same $\kappa(\omega)$ model, their windowed readouts belong to same equivalence class on appropriate PSWF/DPSS space, thus can calibrate and consistency-test unified time scale through joint fitting.

\item Embed above phase--frequency metrology structure into time--information--complexity variational principle, formalize ``choosing FRB/$\delta$-ring window functions and control parameters'' as variational problem on joint manifold, give variational conditions for ``simultaneously using cosmic-scale and laboratory-scale phase--frequency readouts to maximize unified time scale identifiability under finite complexity budget''.
\end{enumerate}

This paper thus completes experimental implementation design of ``phase--frequency unified metrology'' within computational universe framework: FRB and $\delta$-ring scattering become two-end testbeds of unified time scale and complexity geometry, PSWF/DPSS window functions become natural tools for error control, both jointly constructing cross-scale, cross-platform, yet completely unified phase--frequency metrology system under computational universe perspective.
\end{abstract}

\noindent\textbf{Keywords:} Computational universe; Phase-frequency metrology; FRB; $\delta$-ring scattering; Unified time scale; PSWF/DPSS; Experimental testbed

\section{Introduction}

In previous series works, we have completed constructions at following levels:

\begin{enumerate}
\item At discrete level, abstract universe as axiomatic computational universe object $U_{\mathrm{comp}} = (X,\mathsf{T},\mathsf{C},\mathsf{I})$, upon which construct complexity graph $G_{\mathrm{comp}} = (X,E,\mathsf{C})$, complexity distance $d_{\mathrm{comp}}$, complexity dimension and discrete Ricci curvature.

\item At unified time scale--scattering theory level, introduce

$$
\kappa(\omega) = \varphi'(\omega)/\pi = \rho_{\mathrm{rel}}(\omega) = (2\pi)^{-1}\tr Q(\omega),
$$

as ``trinity master scale'', unifying scattering phase derivative, spectral shift density, and group delay trace.

\item Through unified time scale and complexity geometry construct control manifold $(\mathcal{M},G)$, prove discrete complexity distance converges to geodesic distance $d_G$ in refinement limit.

\item Through observer family and relative entropy second-order structure construct task information manifold $(\mathcal{S}_Q,g_Q)$, and give joint variational principle of time--information--complexity on joint manifold $\mathcal{E}_Q = \mathcal{M}\times\mathcal{S}_Q$.

\item In spectral windowing error control work, introduce PSWF/DPSS window functions, show in unified time scale--frequency domain, they are optimal readout windows under finite time--bandwidth--complexity budget.
\end{enumerate}

These results lay foundation for constructing ``unified time scale--computational universe'' theoretical system, but do not directly answer key question: how to metrologically **measure and calibrate** this unified time scale master scale through concrete physical experiments? Mathematical existence of unified time scale insufficient to demonstrate its physical measurability; we need to connect scattering master scale with actually observable phase--frequency data, and design cross-platform metrology strategy so phase--frequency readouts from cosmic scale and laboratory scale can be jointly used to test and calibrate unified time scale.

In this context, Fast Radio Bursts (FRB) and $\delta$-ring--AB flux scattering become two very natural testbeds:

\begin{itemize}
\item FRB are short-duration broadband radio signals traversing cosmological distances, whose propagation phase, group delay and dispersion structure contain integrated information about cosmological medium and vacuum properties; under unified time scale--scattering perspective, FRB essentially ``cosmic-level scattering experiment''.

\item $\delta$-ring--AB flux scattering is precise measurement of spectral--scattering structure of one-dimensional ring geometry, point potential and AB flux at laboratory scale; its spectral quantization equation and phase closure provide highly controllable phase--frequency testbed, usable for ``reverse calibration'' of unified time scale model under known geometric parameters and coupling constants.
\end{itemize}

Goal of this paper: unify embedding of FRB and $\delta$-ring scattering into computational universe--unified time scale framework, establish metrology paradigm using phase--frequency as sole readout, so two types of experiments can mutually calibrate and consistency-test on same unified time scale master scale.

\section{Phase--Frequency Readout Functor in Computational Universe}

This section, under background of computational--physical universe categorical equivalence, introduces ``phase--frequency readout functor'' $\mathsf{PhFr}$, whose output contains only phase--frequency data, directly connected with unified time scale master scale.

\subsection{Review of Physical--Computational Universe Equivalence}

In previous categorical equivalence work, we constructed physical universe category $\mathbf{PhysUniv}^{\mathrm{QCA}}$ and computational universe category $\mathbf{CompUniv}^{\mathrm{phys}}$, giving mutually inverse functors

$$
F:\mathbf{PhysUniv}^{\mathrm{QCA}}\to\mathbf{CompUniv}^{\mathrm{phys}},
$$

$$
G:\mathbf{CompUniv}^{\mathrm{phys}}\to\mathbf{PhysUniv}^{\mathrm{QCA}}.
$$

Physical universe object abstractable as

$$
U_{\mathrm{phys}} = (M,g,\mathcal{F},\kappa,\mathsf{S}),
$$

computational universe object as

$$
U_{\mathrm{comp}} = (X,\mathsf{T},\mathsf{C},\mathsf{I}).
$$

Functors $F,G$ preserve structure of unified time scale density $\kappa(\omega)$ and scattering data $\mathsf{S}(\omega)$: from physical side to computational side, unified time scale discretized as single-step cost; from computational side to physical side, complexity geometry continuized as control--scattering manifold.

\subsection{Phase--Frequency Data Objects}

Define ``phase--frequency data object'' as

$$
U_{\mathrm{PhFr}} = (\Omega,\Theta(\omega),\kappa(\omega)),
$$

where $\Omega \subset\mathbb{R}$ is effective frequency band, $\Theta(\omega)$ is total scattering phase (or its normalization), $\kappa(\omega)$ is unified time scale density. According to unified time scale master scale, under traceable perturbation condition

$$
\kappa(\omega) = \Theta'(\omega)/\pi
$$

holds up to additive constant.

\subsection{Phase--Frequency Readout Functor}

\begin{definition}[Phase--Frequency Readout Functor]
Define functor

$$
\mathsf{PhFr}:\mathbf{PhysUniv}^{\mathrm{QCA}}\to\mathbf{PhFrUniv},
$$

where $\mathbf{PhFrUniv}$ objects are $U_{\mathrm{PhFr}}$, morphisms are frequency-domain transformations preserving $\omega$ and $\Theta(\omega),\kappa(\omega)$ structure.

For physical universe object $U_{\mathrm{phys}}$, $\mathsf{PhFr}(U_{\mathrm{phys}}) = (\Omega,\Theta(\omega),\kappa(\omega))$ determined by its scattering data and unified time scale master scale.
\end{definition}

Through categorical equivalence, given computational universe object $U_{\mathrm{comp}}$, first use $G(U_{\mathrm{comp}}) = U_{\mathrm{phys}}$ to obtain physical universe, then apply $\mathsf{PhFr}$ to obtain phase--frequency data object. Thus obtain composite functor

$$
\mathsf{PhFr}\circ G:\mathbf{CompUniv}^{\mathrm{phys}}\to\mathbf{PhFrUniv}.
$$

\begin{proposition}[Consistency of Unified Time Scale and Phase--Frequency Readout]
\label{prop:PhFr-consistency}
Under traceable perturbation and wave operator completeness conditions, phase--frequency readout object $U_{\mathrm{PhFr}} = (\Omega,\Theta(\omega),\kappa(\omega))$ completely determined by unified time scale density $\kappa(\omega)$ and constant phase shift, i.e.,

$$
\Theta(\omega) = \pi\int^\omega \kappa(\omega')\,\mathrm{d}\omega' + \Theta_0.
$$

In particular, any two pairs $(\Theta,\kappa)$, $(\Theta',\kappa')$ with $\kappa\equiv\kappa'$ and $\Theta-\Theta'$ constant correspond to same unified time scale structure.
\end{proposition}

Proof omitted.

\section{FRB Vacuum Polarization Windowing Upper Limit: Computational Universe Perspective}

This section views FRB propagation as cosmic-scale scattering--propagation process, constructs ``vacuum polarization windowing upper limit'' under unified time scale--spectral windowing.

\subsection{Scattering--Propagation Model of FRB Propagation}

For simplicity, consider FRB signal complex amplitude $A(\omega)$ in frequency domain, whose phase part writable as

$$
A(\omega) = |A(\omega)|\exp(i\Phi_{\mathrm{FRB}}(\omega)).
$$

If propagation includes only known dispersion and reionized medium contributions, then

$$
\Phi_{\mathrm{FRB}}(\omega) = \Phi_{\mathrm{known}}(\omega) + \Phi_{\mathrm{new}}(\omega),
$$

where $\Phi_{\mathrm{known}}$ from conventional dispersion measure and medium model, $\Phi_{\mathrm{new}}$ represents possible contributions from vacuum polarization, new particles, or unified time scale perturbations.

Under unified time scale--scattering perspective, $\Phi_{\mathrm{FRB}}(\omega)$ understandable as effective scattering phase $\Theta_{\mathrm{FRB}}(\omega)$, whose derivative gives effective time scale density perturbation

$$
\delta\kappa_{\mathrm{FRB}}(\omega) = \frac{1}{\pi}\partial_\omega \Phi_{\mathrm{new}}(\omega).
$$

\subsection{Windowed FRB Phase and Error Upper Bound}

Observationally, we can only measure phase--frequency data in finite frequency band $\Omega_{\mathrm{FRB}} = [\omega_{\min},\omega_{\max}]$ with finite resolution. Introduce window function $W_{\mathrm{FRB}}(\omega)$ (e.g., generated by PSWF/DPSS spectrum) and define windowed residual

$$
R_{\mathrm{FRB}} = \int_{\Omega_{\mathrm{FRB}}} W_{\mathrm{FRB}}(\omega)\big(\Phi_{\mathrm{FRB}}(\omega) - \Phi_{\mathrm{known}}(\omega)\big)\,\mathrm{d}\omega.
$$

If unified time scale perturbation $\delta\kappa_{\mathrm{FRB}}(\omega)$ of FRB signal has constraint $|\delta\kappa_{\mathrm{FRB}}| \le \Lambda$ under some spectral norm, then through integration and Cauchy--Schwarz inequality obtain

$$
|R_{\mathrm{FRB}}| \le \Lambda |W_{\mathrm{FRB}}|_{L^2(\Omega_{\mathrm{FRB}})} C_{\mathrm{FRB}},
$$

where $C_{\mathrm{FRB}}$ determined by cosmological propagation kernel and geometric factors.

Conversely, if observationally residual $|R_{\mathrm{FRB}}| \le \varepsilon_{\mathrm{obs}}$, obtain upper bound on unified time scale perturbation

$$
\Lambda \ge \Lambda_{\min} \ge \frac{|R_{\mathrm{FRB}}|}{|W_{\mathrm{FRB}}| C_{\mathrm{FRB}}}.
$$

Writing optimal window function choice problem as constrained minimization of $|W_{\mathrm{FRB}}|$, classical results show PSWF/DPSS type windows minimize error upper bound under given time--frequency--complexity budget, thus giving ``FRB vacuum polarization windowing limiter''.

\section{$\delta$-Ring-AB Flux Scattering Spectral-Phase-Scattering Equivalence and Identifiability}

This section, from computational universe--control manifold perspective, restates spectral--phase--scattering structure of $\delta$-ring--AB flux scattering, gives identifiability theorem.

\subsection{$\delta$-Ring-AB Flux Model}

Consider one-dimensional ring, circumference $L$, coordinate $x\in[0,L)$ with periodicity $x\sim x+L$. Introduce point $\delta$--potential on ring, strength $\alpha_\delta$, and AB flux $\theta \in [0,2\pi)$. Corresponding Hamiltonian (unit mass, ignoring constants) writable as

$$
H = -\partial_x^2 + \alpha_\delta\delta(x),
$$

boundary condition includes AB phase:

$$
\psi(L^-) = \mathrm{e}^{i\theta}\psi(0^+), \quad \psi'(L^-) = \mathrm{e}^{i\theta}\psi'(0^+).
$$

Solving eigenequation $H\psi = k^2\psi$ yields spectral quantization condition

$$
f(k,\alpha_\delta,\theta) = \cos(kL) + (\alpha_\delta/k)\sin(kL) - \cos\theta = 0.
$$

Corresponding scattering amplitude $t(k)$ and phase $\gamma(k)$ satisfy certain phase closure, typical form

$$
\cos\gamma(k) = |t(k)|\cos\theta,
$$

where functional relationship between $\gamma(k)$ and $kL$, $\alpha_\delta$ given by scattering theory.

\subsection{$\delta$-Ring Control Manifold in Computational Universe}

View $\delta$-ring scattering as computational universe on low-dimensional control manifold: control parameter space

$$
\mathcal{M}_{\delta\text{-}\mathrm{ring}} = \{(L,\alpha_\delta,\theta)\},
$$

equipped with metric $G$, e.g.,

$$
G = g_{LL}\mathrm{d}L^2 + g_{\alpha\alpha}\mathrm{d}\alpha_\delta^2 + g_{\theta\theta}\mathrm{d}\theta^2.
$$

Spectral observation $\{k_n(\theta)\}$ corresponds to data points on information manifold, while unified time scale density given by scattering phase derivative and group delay. $\delta$-ring thus becomes highly controllable ``computational sub-universe'' on three-dimensional control manifold, usable for unified time scale and phase--frequency metrology.

\subsection{Spectral--Phase--Scattering Equivalence Theorem}

\begin{theorem}[Spectral Quantization and Phase Closure Equivalence]
\label{thm:spectral-phase-equiv}
In $\delta$-ring--AB flux model, spectral quantization condition

$$
f(k,\alpha_\delta,\theta) = 0
$$

and phase--amplitude closure

$$
\cos\gamma(k) = |t(k)|\cos\theta
$$

equivalent under usual scattering regularity conditions; in particular, when $|t(k)|\to 1$ (weak scattering or transmission resonance), reduces to pure phase closure $\cos\gamma(k) = \cos\theta$.
\end{theorem}

\begin{proof}[Proof sketch]
From boundary conditions and $\delta$--potential jump conditions derive transfer matrix, require wave function to match itself after one loop around ring, obtain spectral quantization equation; on other hand, compute scattering matrix elements $t(k),r(k)$ and phase shift $\gamma(k)$, rewrite spectral condition as phase--amplitude closure. Algebraic equivalence between them verified through direct substitution and simplification. See Appendix B.1 for details.
\end{proof}

\subsection{Parameter Identifiability Theorem}

\begin{theorem}[Local Identifiability of $\delta$-Ring]
\label{thm:delta-ring-identif}
Given $L$ and several AB flux values $\theta_j$, if observe sufficiently many eigenwavenumbers $\{k_n(\theta_j)\}$, and Jacobian matrix at these points

$$
J = \left( \partial f/\partial k, \partial f/\partial \alpha_\delta, \partial f/\partial \theta \right)
$$

has full rank at $(k,\alpha_\delta,\theta)$, then $(\alpha_\delta,\theta)$ are locally identifiable parameters of spectral data near this point, i.e., local inverse function exists writing $(\alpha_\delta,\theta)$ as function of $\{k_n(\theta_j)\}$.
\end{theorem}

\begin{proof}[Proof sketch]
Apply implicit function theorem: if $\partial f/\partial k \ne 0$ and partial derivative submatrix with respect to $(\alpha_\delta,\theta)$ full rank, can solve $k = k(\alpha_\delta,\theta)$ locally, then construct composite map for multiple $\theta_j$, obtain local invertibility. For multiple eigenvalues, stack components; if combined Jacobian full rank, overall identifiability holds. See Appendix B.2 for details.
\end{proof}

\subsection{Pathological Domains and Condition Numbers}

Through explicit computation

$$
\frac{\partial k}{\partial \alpha_\delta} = -\frac{f_{\alpha_\delta}}{f_k},
$$

where $f_{\alpha_\delta} = (\sin(kL))/k$, $f_k = -L\sin(kL) + \alpha_\delta(\dots)$, can define pathological condition number region $f_k\approx 0$, corresponding to spectral quantization curve highly sensitive to parameters or non-invertible. Under computational universe--complexity geometry perspective, these pathological regions correspond to regions on control manifold with large curvature, spectral--phase information's ``geodesic sensitivity'' to parameters dramatically amplified, requiring window functions and experimental design to avoid or specially handle within complexity budget.

\section{FRB and $\delta$-Ring Unified Phase-Frequency Metrology}

This section embeds FRB and $\delta$-ring scattering in same phase--frequency metrology framework, gives geometric conditions for ``cross-platform scale unification''.

\subsection{Juxtaposition of Phase--Frequency Readout Objects}

For FRB and $\delta$-ring, we respectively obtain phase--frequency data objects

$$
U_{\mathrm{PhFr}}^{\mathrm{FRB}} = (\Omega_{\mathrm{FRB}},\Theta_{\mathrm{FRB}}(\omega),\kappa_{\mathrm{FRB}}(\omega)),
$$

$$
U_{\mathrm{PhFr}}^{\delta} = (\Omega_{\delta},\Theta_{\delta}(\omega),\kappa_{\delta}(\omega)).
$$

Under unified time scale hypothesis, exists ``master scale density'' $\kappa_{\mathrm{univ}}(\omega)$ such that effective time scales corresponding to FRB and $\delta$-ring respectively

$$
\kappa_{\mathrm{FRB}}(\omega) = g_{\mathrm{FRB}}(\omega)\kappa_{\mathrm{univ}}(\omega),
$$

$$
\kappa_{\delta}(\omega) = g_{\delta}(\omega)\kappa_{\mathrm{univ}}(\omega),
$$

where $g_{\mathrm{FRB}},g_{\delta}$ are weight functions determined by geometry and propagation kernels.

\subsection{Cross-Platform Scale Unification Condition}

\begin{definition}[Cross-Platform Scale Unification]
FRB and $\delta$-ring scattering called scale-unified on unified time scale if there exist master scale density $\kappa_{\mathrm{univ}}$ and weights $g_{\mathrm{FRB}},g_{\delta}$ such that windowed phase residuals satisfy consistent interpretation:

$$
R_{\mathrm{FRB}}(W_{\mathrm{FRB}}) \approx \int W_{\mathrm{FRB}}(\omega)\delta g_{\mathrm{FRB}}(\omega)\kappa_{\mathrm{univ}}(\omega)\,\mathrm{d}\omega,
$$

$$
R_{\delta}(W_{\delta}) \approx \int W_{\delta}(\omega)\delta g_{\delta}(\omega)\kappa_{\mathrm{univ}}(\omega)\,\mathrm{d}\omega,
$$

for window function family $W_{\mathrm{FRB}},W_{\delta}$.
\end{definition}

\begin{theorem}[Cross-Platform Consistency Test of Unified Time Scale]
\label{thm:cross-platform-test}
If there exist master scale density $\kappa_{\mathrm{univ}}$ and weights $g_{\mathrm{FRB}},g_{\delta}$ such that for PSWF/DPSS type window function family $\{W_j\}$, windowed residuals of FRB and $\delta$-ring satisfy

$$
R_{\mathrm{FRB}}(W_j) = \lambda_j R_{\delta}(W_j) + \mathcal{O}(\varepsilon_j),
$$

where $\lambda_j$ are ratios precomputable from geometric factors, when $\varepsilon_j$ acceptable within experimental error, then phase--frequency data of FRB and $\delta$-ring scattering consistent with unified time scale model.

Conversely, if for some window functions $W_j$ systematic deviation exists exceeding error tolerance, can determine unified time scale model has inconsistency in this frequency band and scale, need to correct $\kappa_{\mathrm{univ}}$ or weight model.
\end{theorem}

\begin{proof}[Proof sketch]
Using completeness of PSWF/DPSS, expand $\kappa_{\mathrm{univ}}$ in window function space, write FRB and $\delta$-ring residuals as coefficient vectors for same basis, test whether they satisfy prespecified linear relationship. See Appendix B.3 for details.
\end{proof}

\section{Joint Variation of Window Functions and Control Parameters: Optimal Cross-Platform Metrology Strategy}

This section incorporates FRB and $\delta$-ring window function and control parameter choices into time--information--complexity joint variational principle, gives variational form of ``optimal cross-platform metrology under finite complexity budget''.

\subsection{Extended Joint Manifold}

Previous joint manifold $\mathcal{E}_Q = \mathcal{M}\times\mathcal{S}_Q$. Now introduce two types of additional degrees of freedom:

\begin{enumerate}
\item FRB window function parameter space $\mathcal{W}_{\mathrm{FRB}}$, e.g., spanned by linear coefficients of several PSWF modes;

\item $\delta$-ring control parameter space $\mathcal{M}_\delta = \{(L,\alpha_\delta,\theta)\}$ and corresponding window function space $\mathcal{W}_{\delta}$.
\end{enumerate}

Extended joint configuration as

$$
\widehat{z}(t) = (\theta(t),\phi(t);\theta_\delta(t),\phi_\delta(t);\{W_{\mathrm{FRB},j}\},\{W_{\delta,j}\}).
$$

In actual metrology problems, FRB window functions mostly time-independent parameters, $\delta$-ring control can vary with experimental steps, $t$ only formalized ``execution parameter''.

\subsection{Joint Action and Extremality Conditions}

Define extended action

\begin{align*}
\mathcal{A}_{\mathrm{cal}}[\widehat{z}] &= \mathcal{A}_{\mathrm{dyn}}[\theta,\phi,\theta_\delta,\phi_\delta] + \mu_{\mathrm{FRB}}\mathcal{E}_{\mathrm{win}}^{\mathrm{FRB}}(\{W_{\mathrm{FRB},j}\})\\
&\quad + \mu_{\delta}\mathcal{E}_{\mathrm{win}}^{\delta}(\{W_{\delta,j}\}) - \lambda_{\mathrm{univ}}\mathcal{I}_{\mathrm{cons}}[\kappa_{\mathrm{univ}}],
\end{align*}

where:

\begin{itemize}
\item $\mathcal{A}_{\mathrm{dyn}}$ is control--information--complexity part's kinetic--potential energy;
\item $\mathcal{E}_{\mathrm{win}}^{\mathrm{FRB}},\mathcal{E}_{\mathrm{win}}^{\delta}$ are window function error functionals for both sides;
\item $\mathcal{I}_{\mathrm{cons}}[\kappa_{\mathrm{univ}}]$ measures FRB and $\delta$-ring consistency for unified time scale, e.g., residual sum of squares;
\item $\mu_{\mathrm{FRB}},\mu_{\delta},\lambda_{\mathrm{univ}}$ are weights.
\end{itemize}

Variation with respect to $\theta,\theta_\delta$ yields geodesic--potential equations for control parameters, variation with respect to $W_{\mathrm{FRB},j},W_{\delta,j}$ yields extremality conditions in window function direction (knowable from previous work as solution being PSWF/DPSS subspace), variation with respect to $\kappa_{\mathrm{univ}}$ yields optimal estimation condition for unified time scale.

At minimal solution of joint variation, obtain following sense of ``optimal metrology strategy'':

\begin{enumerate}
\item FRB and $\delta$-ring each use optimal window functions (PSWF/DPSS) to achieve error control;

\item $\delta$-ring control parameters chosen in regions that both enhance time scale sensitivity and avoid identifiability pathological domains;

\item FRB observation band and $\delta$-ring experimental band maximally complementary under unified time scale model, providing maximum information over full domain of $\kappa_{\mathrm{univ}}$.
\end{enumerate}

\appendix

\section{Time--Frequency Concentration and Optimality of PSWF/DPSS}

This appendix briefly reviews classical results of PSWF/DPSS, gives variational formulation needed for use in this paper's framework.

\subsection{Variational Formulation of PSWF}

Consider band-limited function space

$$
\mathcal{B}_W = \{ f\in L^2(\mathbb{R}) : \widehat{f}(\omega)=0\ \text{for}\ |\omega|>W \}.
$$

In $\mathcal{B}_W$, define energy concentration on time interval $[-T,T]$

$$
\alpha(f) = \frac{\int_{-T}^{T}|f(t)|^2\,\mathrm{d}t}{\int_{-\infty}^{\infty}|f(t)|^2\,\mathrm{d}t}.
$$

Variational problem: in $\mathcal{B}_W$ seek $f\ne 0$ maximizing $\alpha(f)$. Standard method writes this as Rayleigh quotient, corresponding to integral operator

$$
(\mathcal{K} f)(t) = \int_{-T}^{T}\frac{\sin W(t-s)}{\pi(t-s)}f(s)\,\mathrm{d}s.
$$

Eigenvalue $\lambda$ of eigenequation $\mathcal{K} f = \lambda f$ is energy concentration $\alpha(f)$. Ordering eigenvalues from large to small yields $\lambda_0\ge\lambda_1\ge\cdots$, corresponding eigenfunctions $\psi_0,\psi_1,\dots$ are PSWF.

Optimality conclusion: for any dimension-$N$ subspace $V\subset\mathcal{B}_W$,

$$
\sum_{f\in\ONB(V)}\alpha(f) \le \sum_{n=0}^{N-1}\lambda_n.
$$

In particular, equality holds when $V$ spanned by $\{\psi_0,\dots,\psi_{N-1}\}$.

\subsection{Discrete Analogue of DPSS}

In sequence space $\mathbb{C}^N$ of length $N$, define Toeplitz matrix

$$
K_{mn} = \frac{\sin 2\pi W(m-n)}{\pi(m-n)},\ 0\le m,n\le N-1,
$$

diagonal elements taking limit $K_{mm} = 2W$. Eigenvalue--eigenvector problem

$$
\sum_{n=0}^{N-1}K_{mn}v_n^{(k)} = \lambda_k v_m^{(k)}
$$

defines DPSS $v^{(k)}$ and their energy concentration $\lambda_k$. DPSS have optimality corresponding to PSWF under discrete frequency band $[-W,W]$ and finite length $N$: among all dimension-$K$ subspaces, DPSS subspace maximizes total energy concentration within frequency band.

\section{$\delta$-Ring Spectral-Scattering Equivalence and Identifiability Proof Details}

\subsection{Proof Points of Theorem~\ref{thm:spectral-phase-equiv}}

Starting from $\delta$-ring boundary conditions, write plane wave solution $\psi(x) = A\mathrm{e}^{ikx} + B\mathrm{e}^{-ikx}$ in interval $x\in(0,L)$, $\delta$--potential introduces discontinuity conditions at $x=0$ and $x=L^-$; AB flux represented through phase factor $\mathrm{e}^{i\theta}$ imposed between $x=0$ and $x=L^-$. Collective conditions can be organized as eigenvalue equation of $2\times 2$ transfer matrix, whose consistency condition gives spectral quantization formula

$$
\cos(kL) + (\alpha_\delta/k)\sin(kL) = \cos\theta.
$$

On other hand, through constructing corresponding scattering matrix $S(k)$ and transmission amplitude $t(k)$, computing their phase $\gamma(k)$ and modulus $|t(k)|$, consistency condition can be rewritten as

$$
\cos\gamma(k) = |t(k)|\cos\theta.
$$

Algebraic equivalence between them obtained through elementary trigonometric identities and transfer matrix--scattering matrix conversion relations, details omitted.

\subsection{Implicit Function Theorem Application for Theorem~\ref{thm:delta-ring-identif}}

Partial derivative matrix of spectral quantization equation $f(k,\alpha_\delta,\theta)=0$ with respect to three variables at general point is

$$
\partial f/\partial k,\ \partial f/\partial \alpha_\delta,\ \partial f/\partial\theta.
$$

If $\partial f/\partial k\ne 0$, can locally write $k = k(\alpha_\delta,\theta)$. For multiple groups $\theta_j$ and corresponding eigenvalues $k_n(\theta_j)$, stack these functions to obtain map from parameter space $(\alpha_\delta,\theta)$ to observation space $\{k_n(\theta_j)\}$. If this map has full-rank Jacobian at some point, local inverse map exists, parameter identifiability holds.

Pathological regions correspond to $\partial f/\partial k=0$, i.e., spectral curve has horizontal tangent line; at this time small parameter change can cause dramatic $k$ change, conversely inferring parameters from finite precision spectral data unstable.

\subsection{Linear Algebra Form of Theorem~\ref{thm:cross-platform-test}}

Choose window function family $\{W_j\}$ (like PSWF basis), expand unified scale perturbation in this basis

$$
\delta\kappa_{\mathrm{univ}}(\omega) = \sum_j c_j W_j(\omega).
$$

Windowed residuals of FRB and $\delta$-ring respectively written as

$$
R_{\mathrm{FRB}}(W_j) \approx a_j c_j, \quad R_{\delta}(W_j) \approx b_j c_j,
$$

where $a_j,b_j$ determined by propagation kernel and geometric factors. If theory predicts $R_{\mathrm{FRB}} = \lambda_j R_{\delta}$, equation becomes $a_j c_j = \lambda_j b_j c_j$. On one hand, holding simultaneously for all $j$ requires $a_j = \lambda_j b_j$ and $c_j$ nonzero; on other hand, in actual fitting can test through least squares or maximum likelihood estimation whether this linear relationship holds within experimental error, constituting consistency test of unified time scale.

\section{Formalization of Window Function Variational Optimality}

In extended action, window function part's error functional is

$$
\mathcal{E}_{\mathrm{win}}(\{W_j\}) = \sup_{f\in\mathcal{B}_W} \frac{|f - \sum_{j=1}^K\langle f,W_j\rangle W_j|}{|f|}.
$$

This functional on window function subspace $V = \Span\{W_1,\dots,W_K\}$ depends only on $V$ itself. Standard Hilbert space theory shows this error equals $\sqrt{1-\lambda_{\min}(V)}$, where $\lambda_{\min}(V)$ is minimum eigenvalue of $\mathcal{K}$ on $V$. Through minimax principle know when $V$ spanned by first $K$ PSWF eigenfunctions, this minimum eigenvalue maximal, thus error minimal, PSWF subspace is global minimal solution. DPSS discrete case completely analogous, replacing integral operator with matrix spectrum.

This point ensures in time--information--complexity joint variational principle, when incorporating window function degrees of freedom into variation, minimal solution necessarily chooses PSWF/DPSS type window functions, unifying error control and optimality of unified time scale readout within same variational framework.

\end{document}
