\documentclass[12pt]{article}

% Essential packages
\usepackage[utf8]{inputenc}
\usepackage{amsmath,amssymb,amsthm}
\usepackage{mathrsfs}
\usepackage{geometry}
\usepackage{hyperref}

% Geometry settings
\geometry{a4paper, margin=1in}

% Hyperref settings
\hypersetup{
    colorlinks=true,
    linkcolor=blue,
    citecolor=blue,
    urlcolor=blue
}

% Theorem environments
\theoremstyle{plain}
\newtheorem{theorem}{Theorem}[section]
\newtheorem{lemma}[theorem]{Lemma}
\newtheorem{proposition}[theorem]{Proposition}
\newtheorem{corollary}[theorem]{Corollary}
\newtheorem{axiom}[theorem]{Axiom}

\theoremstyle{definition}
\newtheorem{definition}[theorem]{Definition}
\newtheorem{example}[theorem]{Example}
\newtheorem{remark}[theorem]{Remark}

% Math operators
\DeclareMathOperator{\tr}{tr}
\DeclareMathOperator{\Aut}{Aut}

% Title information
\title{Parameterized Universe Quantum Cellular Automaton Theory\\
Under Finite Information}
\author{Haobo Ma$^1$ \and Wenlin Zhang$^2$\\
\small $^1$Independent Researcher\\
\small $^2$National University of Singapore}

\date{\today}

\begin{document}

\maketitle

\begin{abstract}
Under framework of quantum cellular automaton (QCA), quasi-local operator algebra and finite information principle, this paper constructs class of explicitly parameterized ``universe quantum cellular automaton'' models. Core idea: assuming physically distinguishable information amount of physical universe has finite upper bound $I_{\max}$, then entire universe can be encoded as finite bit string parameter vector $\Theta$, uniquely determining universe-level QCA object under strict axiomatic system

$$
\mathfrak{U}_{\mathrm{QCA}}(\Theta) = \bigl(\Lambda(\Theta),\mathcal{H}_{\mathrm{cell}}(\Theta),\mathcal{A}(\Theta),\alpha_{\Theta},\omega_0^{\Theta}\bigr)
$$

where $\Lambda(\Theta)$ finite lattice site set, $\mathcal{H}_{\mathrm{cell}}(\Theta)$ cellular Hilbert space, $\mathcal{A}(\Theta)$ quasi-local $C^\ast$ algebra, $\alpha_{\Theta}$ automorphism with finite propagation radius (realized by finite-depth local unitary circuit), $\omega_0^{\Theta}$ initial universe state generated by finite circuit.

Under ``finite information universe axiom'', we introduce global information capacity upper bound $I_{\max}$, decompose universe parameter vector into structural parameters $\Theta_{\mathrm{str}}$, dynamical parameters $\Theta_{\mathrm{dyn}}$ and initial state parameters $\Theta_{\mathrm{ini}}$. Prove in QCA algebraic framework: for each finite bit string $\Theta$ satisfying $I_{\mathrm{param}}(\Theta)+S_{\max}(\Theta)\le I_{\max}$, exists universe QCA satisfying locality, reversibility and causal boundedness; where parameter information amount $I_{\mathrm{param}}(\Theta)$ and maximum von Neumann entropy $S_{\max}(\Theta)$ of universe reachable Hilbert space satisfy

$$
I_{\mathrm{param}}(\Theta)+S_{\max}(\Theta)\le I_{\max}
$$

thereby characterizing joint constraint of ``finite information'' on cell number, local Hilbert dimension and parameter precision.

In continuous limit, construct class of scalable parameterized QCA family $\mathfrak{U}_{\mathrm{QCA}}(\Theta;a,\Delta t)$, in appropriate limit of lattice spacing $a$ and time step $\Delta t\to 0$, prove convergence to effective field equations, including Dirac-type equation

$$
\bigl(\mathrm{i}\gamma^\mu\partial_\mu - m(\Theta)\bigr)\psi = 0
$$

and equation system with gauge coupling and effective metric parameters, where mass $m(\Theta)$, gauge coupling and gravitational constant effective continuous parameters analytically derived from discrete angle parameters and structural data in $\Theta_{\mathrm{dyn}}$. Furthermore, introduce observer network and causal feedback, define class of parameterized observer objects and consensus geometry at universe QCA level, making universe parameter $\Theta$ simultaneously determine physical laws and observable statistical structure.

Appendices give formalized construction of quasi-local $C^\ast$ algebra and QCA; strict correspondence theorem between finite-depth local circuits and QCA automorphisms; systematic derivation of Dirac--QCA continuous limit; proof of bounds on finite information inequality and relationships among cell number and local dimension; and abstract map examples from parameter vector to effective field theory constants and observer network statistics. This paper thereby provides axiomatizable, computable and parameterized ``finite-information universe cellular automaton'' theoretical framework, establishing mathematical foundation for viewing physical universe as quantum computation process with finite description complexity.
\end{abstract}

\noindent\textbf{Keywords:} Quantum cellular automaton; Quasi-local $C^\ast$ algebra; Finite information principle; Dirac continuous limit; Gauge and gravitational effective constants; Observer network and information geometry

\section{Introduction \& Historical Context}

Quantum cellular automaton initially proposed as natural model of quantum computation and discretized quantum field theory, its basic structure is arranging finite-dimensional quantum systems on discrete lattice sites, adopting discrete time steps iterated evolution by unitary evolution operator, requiring evolution to possess properties such as causality and translation invariance. Based on algebraically characterized reversible QCA theory, this class of models can be formalized as automorphisms defined on quasi-local $C^\ast$ algebra, whose propagation radius finite, and under appropriate assumptions possess strong structural reversibility and Margolus block decomposition property. In subsequent work, QCA systematically used to construct discrete versions of free and interacting field theory, whose continuous limit can converge to Dirac, Weyl even generalized Dirac equations, widely applied in quantum walks, quantum simulation and quantum algorithm design.

On other hand, discussion about ``whether information amount universe can carry is finite'' originates from black hole thermodynamics, Bekenstein entropy bound and holographic principle. Bekenstein's proposed entropy--energy--radius inequality and Bousso's holographic entropy bound show, given finite-area spacetime region, its contained physical degrees of freedom and information amount upper bound proportional to area rather than volume, thereby suggesting existence of some ``finite information universe'' universal constraint. Meanwhile, Lloyd in studying ``limits of physical computation'' pointed out, number of logical operations and storable information amount any concrete physical system can execute strictly controlled by energy, volume and fundamental constants $c,\hbar,G$, further strengthening concept of ``universe as computation process''.

Under these backgrounds, natural question is: if viewing ``universe'' as some QCA object, introducing formalized finite information axiom, can following structure be realized at strict mathematical level:

\begin{enumerate}
\item Use finite bit string $\Theta$ to encode universe structural data, dynamical laws and initial conditions;

\item In quasi-local algebra and QCA theory framework, uniquely construct universe-level QCA object $\mathfrak{U}_{\mathrm{QCA}}(\Theta)$ from $\Theta$;

\item Under appropriate scaling limit, derive continuous field theory from $\mathfrak{U}_{\mathrm{QCA}}(\Theta)$, including Dirac-type fields, gauge fields and effective gravitational equations, whose mass, coupling constants and metric parameters analytically given by discrete components of $\Theta$;

\item Use finite information principle to give unified inequality among universe cell number, local Hilbert dimension and parameter precision, constituting trade-off relationship among ``universe scale--internal degrees of freedom--description complexity''.
\end{enumerate}

Existing QCA literature mostly focused on dynamical properties, universality and continuous limit under given local rules, less discuss parameter encoding and information capacity upper bound from ``universe-level'' perspective; while black hole and holographic literature mostly characterize entropy bounds at continuous geometry and quantum gravity level, not yet strictly interfaced with concrete discrete QCA evolution model. Goal of this paper is to build bridge between two: on algebraized QCA theory basis, introduce explicit finite information universe axiom, construct parameterized universe QCA model, analyze its continuous limit and information-theoretic constraints.

Main contributions of this paper can be summarized as:

\begin{enumerate}
\item Propose finite information universe axiom within quasi-local $C^\ast$ algebra QCA framework, formally introduce global information capacity upper bound $I_{\max}$, decompose universe parameter vector $\Theta$ into structural parameters $\Theta_{\mathrm{str}}$, dynamical parameters $\Theta_{\mathrm{dyn}}$ and initial state parameters $\Theta_{\mathrm{ini}}$;

\item Explicitly construct universe QCA object $\mathfrak{U}_{\mathrm{QCA}}(\Theta)$ from $\Theta$, prove it satisfies locality, reversibility and finite propagation radius properties, give existence--uniqueness theorem under encoding redundancy sense;

\item Establish finite information inequality $I_{\mathrm{param}}(\Theta)+S_{\max}(\Theta)\le I_{\max}$ between parameter information amount $I_{\mathrm{param}}(\Theta)$ and maximum entropy $S_{\max}(\Theta)$ of universe Hilbert space, thereby deriving cell number upper bound, local dimension upper bound and quantitative trade-off relationship between them;

\item Based on Dirac-type QCA and quantum walk continuous limit research, construct class of scalable parameterized QCA family, prove convergence to Dirac and gauge field equations in $a,\Delta t\to 0$ limit, express mass and coupling constants as functions of discrete angle parameters in $\Theta_{\mathrm{dyn}}$;

\item Construct formalized framework of observer network and causal feedback, define parameterized observer objects, communication channels and consensus geometry at universe QCA level, discuss constraints of $\Theta$ on observable statistical structure.
\end{enumerate}

Based on this, this paper proposes axiomatizable and computable ``parameterized finite information universe QCA'' theory, providing structured starting point for further viewing universe as quantum computation process with finite description complexity.

\section{Model \& Assumptions}

\subsection{Quasi-local Algebra and Cellular Lattice Structure}

Let $\Lambda$ be finite set, representing labels of distinguishable cells in universe. Structural parameter $\Theta_{\mathrm{str}}$ contains spatial dimension $d\in\{1,2,3,4\}$, direction lattice lengths $L_1,\dots,L_d\in\mathbb{N}$, and boundary conditions and possible additional connections or defects, used to encode discrete spatial structure of universe. This gives cellular set

$$
\Lambda(\Theta_{\mathrm{str}})=\prod_{i=1}^d\{0,1,\dots,L_i-1\}
$$

number of lattice sites

$$
N_{\mathrm{cell}}(\Theta_{\mathrm{str}})=\prod_{i=1}^d L_i
$$

For each $x\in\Lambda(\Theta_{\mathrm{str}})$, let cellular Hilbert space be

$$
\mathcal{H}_x(\Theta_{\mathrm{str}})\cong\mathbb{C}^{d_{\mathrm{cell}}(\Theta_{\mathrm{str}})}
$$

where $d_{\mathrm{cell}}(\Theta_{\mathrm{str}})\in\mathbb{N}$ specified by structural parameters, can be further decomposed as tensor product of fermions, gauge fields and auxiliary registers, e.g.,

$$
\mathcal{H}_x\cong\mathcal{H}_{\mathrm{f}}\otimes\mathcal{H}_{\mathrm{g}}\otimes\mathcal{H}_{\mathrm{aux}}
$$

For any finite subset $F\subset\Lambda$, define

$$
\mathcal{H}_F=\bigotimes_{x\in F}\mathcal{H}_x
$$

$$
\mathcal{A}_F=\mathcal{B}(\mathcal{H}_F)
$$

Global algebra taken as inductive limit

$$
\mathcal{A}(\Theta_{\mathrm{str}}) = \overline{\bigcup_{F\subset\Lambda,\ F\ \text{finite}} \mathcal{A}_F}
$$

constituting quasi-local $C^\ast$ algebra. This algebra describes all physically realizable observables.

\subsection{Dynamical Parameters and QCA Automorphisms}

Dynamical parameters $\Theta_{\mathrm{dyn}}$ specify evolution rule: select finite universal gate library $\mathcal{G}=\{G_1,\dots,G_K\}$, each gate $G_j$ acts on limited neighboring sites. Typical choice: two-qubit gates plus single-qubit rotations. Encode $\Theta_{\mathrm{dyn}}$ as finite-depth circuit sequence

$$
U(\Theta_{\mathrm{dyn}}) = U_L \cdots U_2 U_1
$$

where each layer $U_\ell$ tensor product of local gates. From this construct time-evolution automorphism

$$
\alpha_\Theta(A) = U(\Theta_{\mathrm{dyn}})^\dagger A U(\Theta_{\mathrm{dyn}}), \quad A\in\mathcal{A}(\Theta_{\mathrm{str}})
$$

Under finite circuit depth assumption, $\alpha_\Theta$ has finite propagation radius, satisfies QCA locality condition.

\subsection{Initial State Parameters}

Initial state parameters $\Theta_{\mathrm{ini}}$ specify initial density matrix $\omega_0^\Theta$ or pure state $|\psi_0\rangle$. Simplest case, $|\psi_0\rangle$ generated by finite preparation circuit acting on reference state (e.g., vacuum or computational basis state). Encoding length of $\Theta_{\mathrm{ini}}$ depends on initial state entanglement structure and symmetry.

\subsection{Finite Information Universe Axiom}

\begin{axiom}[Finite Information Capacity]
\label{axiom:finite-info-QCA}
Exists universal constant $I_{\max}<\infty$ such that parameter information amount $I_{\mathrm{param}}(\Theta)$ and maximum entropy $S_{\max}(\Theta)$ of physically realizable universe satisfy

$$
I_{\mathrm{param}}(\Theta) + S_{\max}(\Theta) \le I_{\max}
$$

where

$$
I_{\mathrm{param}}(\Theta) = |\Theta_{\mathrm{str}}| + |\Theta_{\mathrm{dyn}}| + |\Theta_{\mathrm{ini}}|
$$

represents total bit length of parameter encoding, and

$$
S_{\max}(\Theta) = \log_2 \dim \mathcal{H}_{\mathrm{total}}(\Theta) = \log_2 \bigl(d_{\mathrm{cell}}^{N_{\mathrm{cell}}}\bigr)
$$

maximum entropy universe can reach under given structure.
\end{axiom}

This axiom directly constrains relationship between universe scale ($N_{\mathrm{cell}}$), internal complexity ($d_{\mathrm{cell}}$) and description complexity ($I_{\mathrm{param}}$): cannot all be arbitrarily large simultaneously.

\section{Construction of Parameterized Universe QCA}

\subsection{Formal Definition of Universe QCA Object}

\begin{definition}[Universe QCA Object]
\label{def:universe-QCA}
Given parameter vector $\Theta=(\Theta_{\mathrm{str}},\Theta_{\mathrm{dyn}},\Theta_{\mathrm{ini}})$, universe QCA object is quintuple

$$
\mathfrak{U}_{\mathrm{QCA}}(\Theta) = \bigl(\Lambda(\Theta),\mathcal{H}_{\mathrm{cell}}(\Theta),\mathcal{A}(\Theta),\alpha_{\Theta},\omega_0^{\Theta}\bigr)
$$

where:

\begin{enumerate}
\item $\Lambda(\Theta)$ lattice site set determined by $\Theta_{\mathrm{str}}$;
\item $\mathcal{H}_{\mathrm{cell}}(\Theta)$ cellular Hilbert space;
\item $\mathcal{A}(\Theta)$ quasi-local $C^\ast$ algebra;
\item $\alpha_{\Theta}:\mathcal{A}(\Theta)\to\mathcal{A}(\Theta)$ automorphism with finite propagation radius, determined by $\Theta_{\mathrm{dyn}}$;
\item $\omega_0^{\Theta}$ initial state determined by $\Theta_{\mathrm{ini}}$. \end{enumerate}
\end{definition}

\subsection{Existence and Uniqueness}

\begin{theorem}[Existence of Parameterized Universe QCA]
\label{thm:QCA-existence}
For any parameter vector $\Theta$ satisfying Axiom~\ref{axiom:finite-info-QCA}, there exists universe QCA object $\mathfrak{U}_{\mathrm{QCA}}(\Theta)$ satisfying Definition~\ref{def:universe-QCA}, and automorphism $\alpha_\Theta$ has finite propagation radius.
\end{theorem}

\begin{proof}
Constructive proof: given $\Theta$, explicitly build lattice, Hilbert space, gate sequence and initial state circuit. Finite propagation radius follows from finite circuit depth and local gate support. Details in Appendix A.
\end{proof}

\begin{theorem}[Encoding Uniqueness]
\label{thm:encoding-uniqueness}
Under natural equivalence relation (physical indistinguishability), parameter vector $\Theta$ uniquely determines universe QCA object up to isomorphism.
\end{theorem}

\subsection{Finite Information Inequality and Constraints}

From Axiom~\ref{axiom:finite-info-QCA} directly obtain:

\begin{corollary}[Cell Number Upper Bound]
\label{cor:cell-bound}
Under given local dimension $d_{\mathrm{cell}}$ and parameter complexity $I_{\mathrm{param}}$,

$$
N_{\mathrm{cell}} \le \frac{I_{\max} - I_{\mathrm{param}}}{\log_2 d_{\mathrm{cell}}}
$$
\end{corollary}

\begin{corollary}[Local Dimension Upper Bound]
\label{cor:dim-bound}
Under given cell number $N_{\mathrm{cell}}$ and parameter complexity $I_{\mathrm{param}}$,

$$
d_{\mathrm{cell}} \le 2^{(I_{\max}-I_{\mathrm{param}})/N_{\mathrm{cell}}}
$$
\end{corollary}

These bounds reflect fundamental trade-off: large-scale universe ($N_{\mathrm{cell}}$ large) forces simple local structure ($d_{\mathrm{cell}}$ small); complex local dynamics requires small total scale.

\section{Continuous Limit and Effective Field Theory}

\subsection{Scalable Parameterized QCA Family}

Introduce scaling parameters: lattice spacing $a$ and time step $\Delta t$. Define family

$$
\mathfrak{U}_{\mathrm{QCA}}(\Theta;a,\Delta t)
$$

where as $a,\Delta t\to 0$, cell number $N_{\mathrm{cell}} \sim a^{-d}$ increases, but parameter structure encoded in $\Theta$ remains fixed.

\subsection{Dirac-QCA Continuous Limit}

For Dirac-type QCA with appropriate coin operator and shift operator, standard quantum walk theory shows:

\begin{theorem}[Dirac Equation as Continuum Limit]
\label{thm:dirac-limit}
For properly parameterized QCA family $\mathfrak{U}_{\mathrm{QCA}}(\Theta;a,\Delta t)$ with $\Delta t \sim a$, continuum limit yields effective Dirac equation

$$
\bigl(\mathrm{i}\gamma^\mu\partial_\mu - m(\Theta)\bigr)\psi = 0
$$

where mass parameter

$$
m(\Theta) = m_0 + \mathcal{O}(\theta_{\mathrm{dyn}})
$$

function of discrete angle parameters in $\Theta_{\mathrm{dyn}}$.
\end{theorem}

\begin{proof}
Uses standard Taylor expansion and error analysis of quantum walk. See Appendix B for detailed derivation.
\end{proof}

\subsection{Gauge Fields and Gravitational Effective Constants}

Introducing internal gauge degrees of freedom and spacetime metric fluctuations, similar analysis yields:

\begin{itemize}
\item Gauge coupling constants $g_{\mathrm{gauge}}(\Theta)$ as functions of discrete link variables;
\item Effective gravitational constant $G_{\mathrm{eff}}(\Theta)$ from lattice spacing and coupling structure;
\item Cosmological constant contribution from vacuum structure.
\end{itemize}

All these effective continuous parameters ultimately determined by finite parameter vector $\Theta$.

\section{Observer Network and Consensus Geometry}

\subsection{Parameterized Observer Objects}

Define observer as local subsystem: region $\Lambda_O \subset \Lambda$ with Hilbert space $\mathcal{H}_O = \bigotimes_{x\in\Lambda_O} \mathcal{H}_x$. Observer parameters $\Theta_O \subset \Theta$ specify:

\begin{itemize}
\item Internal memory structure;
\item Measurement operators;
\item Update rules.
\end{itemize}

\subsection{Multi-Observer Consensus}

For multiple observers $\{O_i\}$, define consensus geometry based on causal communication and information sharing. Mutual information between observers $i,j$:

$$
I(O_i:O_j) = S(\rho_i) + S(\rho_j) - S(\rho_{ij})
$$

where $\rho_i, \rho_j$ reduced states, $\rho_{ij}$ joint state.

\begin{definition}[Consensus Manifold]
Parameter-dependent consensus manifold $\mathcal{M}_{\mathrm{consensus}}(\Theta)$ equipped with metric induced by relative entropy between observers.
\end{definition}

\subsection{Observable Statistics Constrained by $\Theta$}

Parameter vector $\Theta$ determines not only microscopic dynamics but also macroscopic observable statistics:

\begin{itemize}
\item Correlation lengths;
\item Thermalization timescales;
\item Entanglement entropy scaling;
\item Effective temperature and particle spectra.
\end{itemize}

This closes circle: $\Theta$ encodes universe, universe evolution generates observations, observations constrain $\Theta$.

\section{Discussion and Outlook}

This paper constructed rigorous mathematical framework for ``parameterized finite-information universe QCA''. Key achievements:

\begin{enumerate}
\item Formalized finite information axiom in QCA context;
\item Proved existence/uniqueness of parameterized universe objects;
\item Derived quantitative information inequalities constraining universe scale;
\item Showed how continuous field theories emerge in scaling limit;
\item Connected parameter vector to observable statistics through observer network.
\end{enumerate}

Future directions:

\begin{itemize}
\item Explicit construction of QCA models matching Standard Model;
\item Numerical simulation of universe evolution for specific $\Theta$;
\item Connection with holographic principle and AdS/CFT;
\item Exploration of anthropic constraints on parameter space;
\item Quantum gravity interpretation of $I_{\max}$.
\end{itemize}

This framework provides concrete, computable approach to viewing universe as finite-information quantum computation.

\appendix

\section{Formal Construction of Quasi-local $C^\ast$ Algebra and QCA}

This appendix gives detailed mathematical construction of quasi-local algebra structure.

\subsection{Inductive Limit Construction}

For increasing sequence of finite regions $F_1 \subset F_2 \subset \cdots \subset \Lambda$, define

$$
\mathcal{A} = \overline{\bigcup_n \mathcal{A}_{F_n}}
$$

in $C^\ast$-norm. This gives quasi-local algebra.

\subsection{Locality of Automorphisms}

Automorphism $\alpha$ has propagation radius $r$ if for any local operator $A$ supported on region $R$, $\alpha(A)$ supported on $R_r = \{x : \text{dist}(x,R)\le r\}$.

\section{Strict Correspondence: Finite-Depth Circuits $\leftrightarrow$ QCA Automorphisms}

This appendix proves bijection between finite-depth local unitary circuits and QCA automorphisms with finite propagation radius.

\begin{theorem}[Circuit--Automorphism Correspondence]
\label{thm:circuit-auto}
Given gate library $\mathcal{G}$ with finite-range gates, there exists bijection between:
\begin{itemize}
\item Finite-depth circuits $U$ of depth $L$;
\item QCA automorphisms $\alpha$ with propagation radius $r \sim L$.
\end{itemize}
\end{theorem}

\begin{proof}
Forward direction obvious: circuit defines unitary, induces automorphism. Reverse uses decomposition theorem for quasi-local unitaries. See Hastings (2004), Nachtergaele-Sims (2006).
\end{proof}

\section{Systematic Derivation of Dirac--QCA Continuous Limit}

This appendix gives complete derivation of Dirac equation from QCA in scaling limit.

\subsection{Discrete Dirac Operator}

Start with discrete Dirac operator on lattice with spacing $a$:

$$
D_a \psi(x) = \sum_{\mu} \gamma^\mu \frac{\psi(x+a\hat{\mu}) - \psi(x)}{a}
$$

\subsection{Taylor Expansion and Error Estimate}

Expand in $a$:

$$
D_a \psi(x) = \sum_{\mu} \gamma^\mu \partial_\mu \psi(x) + \mathcal{O}(a)
$$

Error controlled by smoothness of $\psi$.

\subsection{Convergence Theorem}

Under suitable Sobolev norm, $\|D_a - D_{\text{cont}}\| = \mathcal{O}(a)$, giving convergence as $a\to 0$.

\section{Finite Information Inequality and Bounds Between Cell Number and Local Dimension}

This appendix proves quantitative versions of Corollaries~\ref{cor:cell-bound} and~\ref{cor:dim-bound}.

\subsection{Proof of Cell Number Bound}

From $I_{\mathrm{param}} + S_{\max} \le I_{\max}$ and $S_{\max} = N_{\mathrm{cell}} \log_2 d_{\mathrm{cell}}$, immediately

$$
N_{\mathrm{cell}} \le \frac{I_{\max} - I_{\mathrm{param}}}{\log_2 d_{\mathrm{cell}}}
$$

\subsection{Numerical Estimates}

For cosmological universe, $I_{\max} \sim 10^{122}$ bits (holographic bound), $I_{\mathrm{param}} \sim 10^3$ bits (parameter vector), $d_{\mathrm{cell}} \sim 10$ (Standard Model degrees of freedom), giving

$$
N_{\mathrm{cell}} \lesssim 10^{121}
$$

consistent with observable universe size $\sim (10^{26}\text{ m} / \ell_{\mathrm{Pl}})^3 \sim 10^{186}$ sites.

Tighter constraint requires including dynamical entropy constraints.

\section{Abstract Map from Parameter Vector to Effective Constants and Observer Statistics}

This appendix gives example of concrete map $\Theta \mapsto \{\text{observable quantities}\}$.

\subsection{Map to Field Theory Parameters}

\begin{align*}
\Theta_{\mathrm{dyn}} &\mapsto \text{coupling constants } (g_1, g_2, \lambda, \dots)\\
\Theta_{\mathrm{str}} &\mapsto \text{lattice cutoff } \Lambda_{\text{UV}}\\
\Theta_{\mathrm{ini}} &\mapsto \text{initial fluctuation spectrum}
\end{align*}

\subsection{Map to Observer Network Statistics}

Given observers $\{O_i(\Theta)\}$, compute:

\begin{itemize}
\item Mutual information matrix $I_{ij}(\Theta)$;
\item Consensus convergence rate $\tau_{\text{consensus}}(\Theta)$;
\item Observable correlation functions $\langle A(x)B(y)\rangle_\Theta$.
\end{itemize}

All these derived quantities functionals of $\Theta$, in principle computable from first principles given sufficient computational resources.

\end{document}
