\documentclass[11pt]{article}
\usepackage[utf8]{inputenc}
\usepackage[T1]{fontenc}
\usepackage{amsmath,amssymb,amsthm}
\usepackage{mathtools}
\usepackage{geometry}
\geometry{margin=1in}
\usepackage{hyperref}
\usepackage{cite}
\usepackage{braket}

\newtheorem{theorem}{Theorem}
\newtheorem{lemma}[theorem]{Lemma}
\newtheorem{proposition}[theorem]{Proposition}
\newtheorem{corollary}[theorem]{Corollary}
\theoremstyle{definition}
\newtheorem{definition}[theorem]{Definition}
\newtheorem{axiom}[theorem]{Axiom}
\theoremstyle{remark}
\newtheorem{remark}[theorem]{Remark}

\title{Unified Characterization of ``My Mind Is the Universe'' in Matrix Universe THE-MATRIX}

\author{Haobo Ma$^1$ \and Wenlin Zhang$^2$\\
\small $^1$Independent Researcher\\
\small $^2$National University of Singapore}

\date{}

\begin{document}

\maketitle

\begin{abstract}
Within framework of unified time scale, boundary time geometry, causal manifolds, and matrix universe THE-MATRIX, this paper provides axiomatizable, theorem-provable mathematical characterization of philosophical proposition ``my mind is the universe''. First, physical universe is characterized on one hand as Lorentz causal manifold $U_{\rm geo}$ with small causal diamond generalized entropy structure and boundary time geometry; on the other hand as matrix universe $U_{\rm mat}$ controlled by scattering matrix family $S(\omega)$, Wigner--Smith time-delay matrix $Q(\omega)$, and spectral shift function. Through Birman--Kreĭn formula and Wigner--Smith theory, we define unified time scale density $\kappa(\omega)=\varphi'(\omega)/\pi=\rho_{\rm rel}(\omega)=(2\pi)^{-1}\mathrm{tr}\,Q(\omega)$ and align it with generalized entropy variation and modular flow time. Second, individual ``self'' is formalized as observer triple $I=(\gamma,\mathcal{A}_O,\{\omega_O^{(\tau)}\}_{\tau\in\mathbb{R}})$ along timelike worldline, where ``mind'' is modeled as family of statistical models for universe parameters $\theta\in\Theta$ and their posterior trajectory $\{\pi_\tau\}$ under Bayesian update, carrying Fisher--Rao metric $g^{\rm Fisher}$ in Eguchi--Amari information geometry sense. Under appropriate assumptions of identifiability, regularity, and Bayesian consistency, we construct ``physical metric'' $g^{\rm phys}$ induced from matrix universe spectral data, proving $g^{\rm phys}=g^{\rm Fisher}$, thus information geometry of ``my mind'' and universe's parameter geometry are isometric in limiting sense. Main theorem shows: when unified time scale equivalence class $[\tau]$ unifies scattering time, modular time, geometric time with observer's proper time and cognitive time, and posterior $\pi_\tau\Rightarrow\delta_{\theta^\ast}$, matrix universe structure of universe is isomorphic to model manifold of ``my mind'' in information geometry, allowing strict statement ``my mind is the universe''. Appendices provide categorical equivalence outline between matrix universe and geometric universe, technical proof of information geometry and spectral data alignment, and toy model example of one-dimensional $\delta$ potential ring with Aharonov--Bohm flux.
\end{abstract}

\section*{Keywords}
Matrix universe THE-MATRIX; Unified time scale; Boundary time geometry; Generalized entropy and QNEC; Tomita--Takesaki modular theory; Thermal time hypothesis; Wigner--Smith time delay; Information geometry; Bayesian consistency; Observer and ``mind''

\section{Introduction \& Historical Context}

Proposition ``my mind is the universe'' in East Asian philosophical tradition commonly expresses identity and inseparability of subject and world. However, lacking explicit structural language, it difficultly interfaces directly with contemporary mathematical physical theories of spacetime, quantum fields, and information. On other hand, modern physics gradually constructs structural pictures like ``static universe'', ``boundary priority'', ``time as state-dependent parameter'' across multiple intertwining routes: e.g., block universe in general relativity, holographic gravity and generalized entropy, Tomita--Takesaki modular theory in operator algebras and Connes--Rovelli thermal time hypothesis, and time readings characterized by Birman--Kreĭn formula and Wigner--Smith time delay in scattering theory.

At intersection of gravity and quantum information, research through generalized entropy $S_{\rm gen}$ and quantum energy conditions (such as quantum null energy condition QNEC, quantum focusing conjecture QFC) shows deep connection between local geometric dynamics and second variation of boundary quantum entropy; arrow of time can be intrinsically extracted from entropy monotonicity without adding ``absolute time''.

Meanwhile, information geometry from work of Eguchi, Amari et al. develops viewpoint of viewing statistical models as manifolds with natural metric and connection, where Umegaki relative entropy and broader divergence families provide unified source for Fisher--Rao metric and $\alpha$ connections. In this perspective, ``rational observer'''s ``mind'' is naturally modeled as point on parameter manifold and its trajectory driven by observations. Bayesian posterior consistency theorems guarantee that under identifiability and moderate regularity conditions, posterior distributions converge almost surely to true parameter point, thus observer's internal model ``learns'' true world in limiting sense.

This paper's core claim is: in universe with boundary time geometry and matrix universe THE-MATRIX unified scale, one can restate ``my mind is the universe'' in strict mathematical sense as ``matrix structure of universe and model manifold of observer's mind are information-geometrically isometric under unified time scale''. Specifically:

\begin{enumerate}
\item Universe's \textbf{external characterization}: on one hand Lorentz manifold $U_{\rm geo}$ with generalized entropy and causal partial order; on other hand matrix universe $U_{\rm mat}$ controlled by scattering matrix family $S(\omega)$, Wigner--Smith time delay $Q(\omega)$, and spectral shift function $\xi(\omega)$; both equivalent in appropriate category through Birman--Kreĭn formula and heat kernel--spectral flow tools.

\item ``Self'''s \textbf{internal characterization}: observer along timelike worldline $\gamma$, whose ``mind'' is modeled as probability distribution $\pi_\tau$ on parameter space $\Theta$, using Bayesian update when observing matrix universe scattering data. Relative entropy $D(\theta\Vert\theta_0)$ induces Fisher--Rao metric $g^{\rm Fisher}$, thus ``mind'' itself carries information geometric structure.

\item ``Is'''s \textbf{structural meaning}: under unified time scale, metric $g^{\rm phys}$ constructed from matrix universe spectral data coincides with $g^{\rm Fisher}$ obtained from relative entropy Hessian, and posterior $\pi_\tau\Rightarrow\delta_{\theta^\ast}$. In this limit, local geometry of ``my mind'''s model manifold completely coincides with universe's parameter geometry, allowing both to be viewed as same geometric object with two coordinate systems.
\end{enumerate}

This paper's goal is not making metaphysical declarations, but providing explicit mathematical framework where ``my mind is the universe'' becomes set of theorems that can be stated, analyzed, even tested in simplified models. Below we give corresponding axiomatic setting, main theorems and proofs, demonstrating operability of this framework in one-dimensional scattering toy models and multi-port electromagnetic scattering network engineering proposals.

\section{Model \& Assumptions}

This section presents unified model of universe, matrix universe, and ``my mind'', listing key assumptions needed for subsequent theorems.

\subsection{Geometric Universe and Boundary Time Geometry}

Let $(M,g)$ be four-dimensional globally hyperbolic Lorentz manifold, $\prec$ causal partial order induced by light cone structure. For each point $p\in M$ and small scale parameter $r$, choose small causal diamond $D_{p,r}$ whose boundary is generated by two families of null geodesics. Choosing affine parameter $\lambda$ in one null direction, let $\Sigma_\lambda$ be cross-section; define generalized entropy for each cross-section:
$$
S_{\rm gen}(\lambda)
=\frac{A(\Sigma_\lambda)}{4G\hbar}+S_{\rm out}(\lambda),
$$
where $A$ is cross-section area, $S_{\rm out}$ von Neumann entropy of exterior quantum field. Quantum null energy condition (QNEC) gives lower bound on stress tensor along null direction, expressed in terms of second derivative of $S_{\rm out}(\lambda)$; under appropriate assumptions, this condition can be viewed as local projection of quantum focusing conjecture (QFC), which links generalized entropy variation with Einstein-like equations.

In cases with boundary $\partial M$, gravitational action is
$$
I[g,\Psi]
=\frac{1}{16\pi G}\int_M R\sqrt{-g}\,\mathrm{d}^4x
+\frac{1}{8\pi G}\int_{\partial M} K\sqrt{|h|}\,\mathrm{d}^3x
+I_{\rm matter}[\Psi,g]
+\text{(corner \& null-like boundary terms)},
$$
where $K$ is extrinsic curvature trace, $h$ induced boundary metric. Gibbons--Hawking--York boundary term ensures field equations well-posed under variation fixing boundary geometric data; Brown--York quasilocal stress tensor $T^{\rm BY}_{ab}$ is Hamiltonian generator of boundary time translation, yielding geometric time parameter $\tau_{\rm geom}$ after choosing family of time translation vector fields on boundary.

Synthesizing above structures, we define \textbf{geometric universe} as
$$
U_{\rm geo}
=(M,g,\prec,\mathcal{A}_\partial,\omega_\partial,S_{\rm gen},\kappa),
$$
where $\mathcal{A}_\partial$ is boundary observable algebra, $\omega_\partial$ boundary state, $\kappa$ unified time scale density introduced from scattering theory shortly and aligned with $\tau_{\rm geom}$.

\subsection{Matrix Universe THE-MATRIX and Unified Time Scale}

At spectral--scattering end, consider pair of self-adjoint operators $(H,H_0)$ satisfying relative trace-class perturbation condition and Birman--Kreĭn assumptions. Denote scattering matrix on absolutely continuous spectrum as $S(\omega)$, total scattering determinant
$$
\det S(\omega)=\mathrm{e}^{\mathrm{i}\Phi(\omega)},\qquad
\varphi(\omega)=\frac{1}{2}\Phi(\omega),
$$
and spectral shift function $\xi(\omega)$. Birman--Kreĭn formula gives
$$
\det S(\omega)=\exp\bigl(-2\pi\mathrm{i}\,\xi(\omega)\bigr),
$$
thus $\Phi'(\omega)=-2\pi\xi'(\omega)$; define relative density of states
$$
\rho_{\rm rel}(\omega)=-\xi'(\omega)
=\frac{\Phi'(\omega)}{2\pi}
=\frac{\varphi'(\omega)}{\pi}.
$$

On other hand, Wigner--Smith time-delay matrix is defined as
$$
Q(\omega)
=-\mathrm{i} S(\omega)^\dagger\partial_\omega S(\omega),
$$
whose trace's real part under appropriate normalization gives sum of time delays; in multi-channel scattering systems, eigenvalues of $Q(\omega)$ are so-called proper delay times.

Under standard conditions,
$$
\frac{1}{2\pi}\mathrm{tr}\,Q(\omega)=\rho_{\rm rel}(\omega)
=\frac{\varphi'(\omega)}{\pi},
$$
thus we can introduce unified time scale density
$$
\kappa(\omega)
=\frac{\varphi'(\omega)}{\pi}
=\rho_{\rm rel}(\omega)
=\frac{1}{2\pi}\mathrm{tr}\,Q(\omega).
$$

For reference frequency $\omega_0$, define time scale
$$
\tau_{\rm scatt}(\omega)-\tau_{\rm scatt}(\omega_0)
=\int_{\omega_0}^{\omega}\kappa(\tilde{\omega})\,\mathrm{d}\tilde{\omega}.
$$

We call
$$
U_{\rm mat}
=\bigl(\mathcal{H}_{\rm chan},S(\omega),Q(\omega),\kappa(\omega),\mathcal{A}_\partial,\omega_\partial\bigr)
$$
a \textbf{matrix universe THE-MATRIX} if and only if:

\begin{enumerate}
\item $\mathcal{H}_{\rm chan}$ is direct sum of Hilbert spaces of all ``in/out'' boundary channels; $S(\omega)$ is unitary and sufficiently differentiable matrix-valued function on each energy window;

\item For each small causal diamond $D_{p,r}$, scattering data of its boundary observable subalgebra can be embedded in finite-dimensional matrix block of $S(\omega)$, with such embedding compatible with corresponding $K^1$ class at family level;

\item Unified time scale density $\kappa(\omega)$ aligns with $\tau_{\rm geom}$ at geometric end, stated more precisely in Section 3.1 shortly.
\end{enumerate}

Intuitively, $U_{\rm mat}$ compresses all observable causal and temporal structures of universe into family of scattering matrices varying with frequency, and time delay and spectral shift data derived from them.

\subsection{Modular Flow and Thermal Time}

At operator algebra end, Tomita--Takesaki theory shows: given von Neumann algebra $\mathcal{M}$ with faithful state $\omega$, one can construct modular operator $\Delta$ and one-parameter group $\sigma_t^\omega$ of modular automorphisms
$$
\sigma_t^\omega(A)=\Delta^{\mathrm{i} t}A\Delta^{-\mathrm{i} t},\qquad A\in\mathcal{M},
$$
this modular flow satisfies KMS condition and plays role of ``time evolution'' in many quantum field theories and curved spacetime backgrounds.

Connes--Rovelli thermal time hypothesis proposes: for given physical state $\omega$, its modular flow parameter can be interpreted as ``intrinsic time'' under that state; in context of generally covariant quantization of general relativity, thermal time provides scheme for extracting time parameter in Hamiltonian constraint system with no overall time.

This paper utilizes this idea to align modular time $\tau_{\rm mod}$ with scattering time $\tau_{\rm scatt}$ and geometric time $\tau_{\rm geom}$ on boundary algebra $\mathcal{A}_\partial$, constructing unified time scale equivalence class $[\tau]$.

\subsection{Formalization of ``Self'' and ``Mind''}

\subsubsection{``Self'' as Worldline Compression}

Given geometric universe $U_{\rm geo}$, define:

\begin{definition}[``Self'']
A ``self'' is triple
$$
I=(\gamma,\mathcal{A}_O,\{\omega_O^{(\tau)}\}_{\tau\in\mathbb{R}}),
$$
where:

\begin{enumerate}
\item $\gamma:\mathbb{R}\to M$ is timelike worldline, whose parameter $\tau$ takes value in proper time and unified time scale equivalence class $[\tau]$;

\item $\mathcal{A}_O\subset\mathcal{A}_\partial$ is subalgebra obtained through some completely positive compression map $\Phi:\mathcal{A}_\partial\to\mathcal{A}_O$, representing boundary observables accessible to ``self'';

\item $\omega_O^{(\tau)}$ is internal state of ``self'' at scale $\tau$, satisfying existence of completely positive map $\Psi_\tau:\mathcal{A}_O\to\mathcal{A}_\partial$ such that
$$
\omega_O^{(\tau)}(A)=\omega_\partial(\Psi_\tau(A)),\qquad A\in\mathcal{A}_O.
$$
\end{enumerate}

This ensures ``self'''s internal state is compatible with universe boundary state.
\end{definition}

\subsubsection{``Mind'' as Information Geometric Model Manifold}

``Mind'' is not single instantaneous state, but family of learnable models about universe and their update trajectory.

Let $\Theta\subset\mathbb{R}^d$ be smooth parameter manifold, $\{P_\theta\}_{\theta\in\Theta}$ statistical model family from implementable experiments in matrix universe (such as multi-frequency scattering experiments), each $P_\theta$ probability measure on some outcome space. Umegaki relative entropy is defined as
$$
D(\theta\Vert\theta_0)
=\int \log\frac{\mathrm{d} P_\theta}{\mathrm{d} P_{\theta_0}}\,\mathrm{d} P_\theta,
$$
under appropriate differentiability and convexity conditions, Eguchi's information geometry theory shows its Hessian
$$
g_{ij}(\theta_0)
=\left.\frac{\partial^2}{\partial\theta^i\partial\theta^j}D(\theta\Vert\theta_0)\right|_{\theta=\theta_0}
$$
gives Fisher--Rao metric, while third-order derivatives define Amari--Chentsov tensor, yielding family of $\alpha$ connections.

\begin{definition}[Mind's Model Space]
``Mind'''s model space is manifold with metric and connection
$$
U_{\rm heart}
=(\Theta,g^{\rm Fisher},\nabla^{(\alpha)},\{\pi_\tau\}_{\tau\in\mathbb{R}}),
$$
where $\pi_0$ is prior distribution, $\pi_\tau$ posterior for parameter $\theta$ at scale $\tau$.
\end{definition}

\subsubsection{Observation and Update}

Under unified time scale, ``self'' performs series of experiments in matrix universe, obtaining observation data stream $\mathcal{D}_{[0,\tau]}$. With likelihood function $\mathcal{L}(\mathcal{D}_{[0,\tau]}\mid\theta)$ sufficiently regular in $\theta$, posterior update is
$$
\pi_\tau(\theta)\propto \pi_0(\theta)\,\mathcal{L}(\mathcal{D}_{[0,\tau]}\mid\theta).
$$

In classical case this is standard Bayesian update; in quantum case generalized measurements and quantum operations can be used, but this paper focuses on statistical level abstracted by $P_\theta$.

\subsection{Key Assumptions}

Subsequent main theorems rely on following assumptions:

\begin{enumerate}
\item \textbf{Unified time scale assumption}: There exists unified time scale equivalence class $[\tau]$ such that scattering time $\tau_{\rm scatt}$, geometric time $\tau_{\rm geom}$, modular time $\tau_{\rm mod}$, and observer's proper time and cognitive time all belong to $[\tau]$.

\item \textbf{Identifiability}: Statistical model family $\{P_\theta\}$ is identifiable, i.e., if $P_{\theta_1}=P_{\theta_2}$ for all implementable experiments, then $\theta_1=\theta_2$.

\item \textbf{Model completeness}: True matrix universe corresponds to some $\theta^\ast\in\Theta$.

\item \textbf{Bayesian consistency}: Prior $\pi_0$ has positive mass on $\theta^\ast$; observation process satisfies standard conditions of theorems like Doob--Barron, thus posterior converges almost surely to $\delta_{\theta^\ast}$.

\item \textbf{Eguchi regularity}: Relative entropy $D(\theta\Vert\theta_0)$ has sufficient differentiability and strict convexity in parameters, thus information geometric structure is well-behaved.
\end{enumerate}

Under these assumptions, we can precisely state and prove main theorem of ``my mind is the universe''.

\section{Main Results (Theorems and Alignments)}

This section states three main results: unified time scale theorem, information geometric isometry theorem, and ``my mind is the universe'' theorem. Rigorous proofs given in Section 4 and appendices.

\subsection{Unified Time-Scale Theorem}

\begin{theorem}[Unified Time Scale]
Let $U_{\rm geo}$ satisfy generalized entropy monotonicity condition given by QNEC/QFC, $U_{\rm mat}$ satisfy Birman--Kreĭn and Wigner--Smith assumptions; boundary algebra $\mathcal{A}_\partial$ has Tomita--Takesaki modular structure and corresponding thermal time flow. Let $I$ be observer along timelike worldline $\gamma$ with proper time $\tau_{\rm prop}$; cognitive time $\tau_{\rm cog}$ defined as parameter making relative entropy increment $D(\omega_O^{(\tau+\Delta\tau)}\Vert\omega_O^{(\tau)})$ take fixed value. Then there exists unified time scale equivalence class $[\tau]$ such that
$$
\tau_{\rm scatt}\sim \tau_{\rm geom}\sim \tau_{\rm mod}\sim \tau_{\rm prop}\sim \tau_{\rm cog},
$$
where ``$\sim$'' denotes affine equivalence relation.
\end{theorem}

This theorem shows that under appropriate physical conditions, scattering time, geometric time, modular time, and observer's internal time can be unified into same scale equivalence class, providing foundation for subsequent alignment of ``mind'' with universe geometry.

\subsection{Information-Geometric Isometry Theorem}

To introduce isometry between information geometry and physical geometry, need to define metric on universe end.

\begin{definition}[Physical Parameter Metric]
Consider parametrized scattering family $S(\omega;\theta)$ in matrix universe, with spectral shift function $\xi(\omega;\theta)$ and relative DOS $\rho_{\rm rel}(\omega;\theta)$. For energy window $I\subset\mathbb{R}$ and weight function $W(\omega)\ge 0$, define
$$
g^{\rm phys}_{ij}(\theta)
=\int_I W(\omega)\,
\partial_i\log \rho_{\rm rel}(\omega;\theta)\,
\partial_j\log \rho_{\rm rel}(\omega;\theta)\,
\mathrm{d}\omega.
$$

Under finite-order Euler--Maclaurin and Poisson sum distribution theory framework, $g^{\rm phys}$ can be understood as spectral expression of second-order deformation of relative entropy.
\end{definition}

\begin{theorem}[Information Geometric Isometry]
Under assumptions of Section 2.5, there exist weight function $W(\omega)$ and energy window $I$ such that:

\begin{enumerate}
\item Metric $g^{\rm phys}$ constructed from matrix universe spectral data equals Fisher--Rao metric $g^{\rm Fisher}$ in Eguchi information geometry;

\item This equality holds in neighborhood around $\theta^\ast$.
\end{enumerate}

In other words, near true parameter, information geometry of ``mind'' and spectral--scattering geometry of universe are isometric.
\end{theorem}

\subsection{``My Mind Is the Universe'' Theorem}

Based on unified time scale and information geometric isometry theorems, we can give rigorous version of ``my mind is the universe''.

\begin{definition}[``My Mind Is the Universe'']
In given universe $(U_{\rm geo},U_{\rm mat})$ and observer--mind structure $I,U_{\rm heart}$, if satisfying:

\begin{enumerate}
\item \textbf{Unified time scale}: There exists $[\tau]$ such that all physical and cognitive time scales belong to one equivalence class;

\item \textbf{Identifiability and Bayesian consistency}: True parameter $\theta^\ast$ has positive prior mass, posterior $\pi_\tau\Rightarrow\delta_{\theta^\ast}$;

\item \textbf{Geometric isometry}: In neighborhood of $\theta^\ast$, $g^{\rm phys}=g^{\rm Fisher}$;
\end{enumerate}

then ``my mind is the universe'' is said to hold between this observer and this universe.
\end{definition}

\begin{theorem}[``My Mind Is the Universe'']
Under assumptions of Section 2.5, for matrix universe THE-MATRIX and observer--mind structure along worldline $\gamma$, there exist unified time scale equivalence class $[\tau]$ and parameter manifold $(\Theta,g)$ such that in Bayesian consistency limit, model manifold of ``my mind'' is isometric to universe's parameter geometry, thus satisfying all conditions of Definition 3.2. In other words, in this sense
$$
\text{``My mind''} \simeq \text{``Universe''},
$$
where $\simeq$ denotes structural isomorphism under information geometry and unified time scale.
\end{theorem}

\section{Proofs}

This section provides proof ideas and key steps of main theorems; complete technical details and partial technical constructions placed in appendices.

\subsection{Proof of Theorem 3.1 (Unified Time-Scale)}

Unified time scale theorem needs to align four types of time: scattering time $\tau_{\rm scatt}$, geometric time $\tau_{\rm geom}$, modular time $\tau_{\rm mod}$, and observer's proper time $\tau_{\rm prop}$ and cognitive time $\tau_{\rm cog}$.

\begin{enumerate}
\item \textbf{Scattering time and DOS time}

By Birman--Kreĭn formula, $\xi(\omega)$ linked to phase $\Phi(\omega)$ of $\det S(\omega)$, thus relative DOS $\rho_{\rm rel}(\omega)=-\xi'(\omega)=\Phi'(\omega)/(2\pi)$. On other hand, trace of Wigner--Smith time-delay matrix satisfies $\mathrm{tr}\,Q(\omega)=2\pi\rho_{\rm rel}(\omega)$. Therefore unified time scale density
$$
\kappa(\omega)
=\rho_{\rm rel}(\omega)
=\frac{1}{2\pi}\mathrm{tr}\,Q(\omega).
$$

Define $\tau_{\rm scatt}$ as integral of $\kappa(\omega)$; then $\tau_{\rm scatt}$ is unique up to affine rescaling within given energy window (ignoring integer spectral flow and boundary terms).

\item \textbf{Geometric time and generalized entropy}

On small causal diamonds, QNEC and QFC show generalized entropy second variation related to energy density along null direction and geometric contraction rate. When gravitational action includes GHY boundary term, boundary time translation Hamiltonian determined by Brown--York tensor; requiring generalized entropy flow consistent with ADM/Bondi time parameter yields geometric time scale $\tau_{\rm geom}$. In holographic or semiclassical window, boundary scattering data and DOS difference can be reconstructed from bulk geometry, so $\tau_{\rm geom}$ and $\tau_{\rm scatt}$ differ at most by affine transformation.

\item \textbf{Modular time and thermal time}

Tomita--Takesaki theory endows boundary algebra $\mathcal{A}_\partial$ with modular flow $\sigma_t^\omega$. In thermal time hypothesis, choosing some physical state $\omega_\partial$ as reference, modular parameter $t$ is interpreted as ``physical time'' under that state. In equilibrium states satisfying KMS condition, modular time and Hamiltonian time related by temperature factor; in general curvature backgrounds, need moderate assumptions ensuring compatibility of modular flow with geometric time. This paper assumes constants $a,b$ exist such that
$$
\tau_{\rm mod}=a\,\tau_{\rm geom}+b,
$$
thus modular time belongs to same equivalence class as geometric time.

\item \textbf{Proper time and cognitive time}

Proper time $\tau_{\rm prop}$ along timelike worldline $\gamma$ defined through line element of $g_{\mu\nu}$, usually equivalent locally to appropriately chosen geometric time parameter. Cognitive time $\tau_{\rm cog}$ defined as time scale when relative entropy increment is fixed: choosing constant $\Delta D>0$, require
$$
D\bigl(\omega_O^{(\tau+\Delta\tau)}\Vert\omega_O^{(\tau)}\bigr)=\Delta D,
$$
then $\Delta\tau$ defines unit cognitive time. In unified framework of entropy--energy--time, second-order deformation of $D$ controlled by energy--stress tensor and geometric time evolution, thus $\tau_{\rm cog}$ and $\tau_{\rm mod}$ can be proved to differ by constant factor.
\end{enumerate}

In summary, there exists unified time scale equivalence class $[\tau]$ such that all time scales can be converted to each other through affine transformations; Theorem 3.1 proved. More detailed construction and technical conditions given in Appendix A.

\subsection{Proof of Theorem 3.2 (Information-Geometric Isometry)}

Core of information geometric isometry theorem is proving $g^{\rm phys}$ constructed from spectral data coincides with $g^{\rm Fisher}$ constructed from relative entropy Hessian in neighborhood of $\theta^\ast$.

\begin{enumerate}
\item \textbf{Relative entropy and DOS expression}

For implementable experiments in matrix universe (e.g., statistics of multi-frequency scattering phase and time delay), likelihood $P_\theta$ can be constructed. In appropriate windowed limit, relative entropy $D(\theta\Vert\theta_0)$ can be rewritten in integral expression form using spectral shift function and DOS difference:
$$
D(\theta\Vert\theta_0)
\approx \int_I F\bigl(\rho_{\rm rel}(\omega;\theta),\rho_{\rm rel}(\omega;\theta_0)\bigr)\,\mathrm{d}\omega,
$$
where $F$ is some local function satisfying regularity conditions, $I$ energy window. This rewriting relies on standard connections among heat kernel--spectral shift--relative determinant.

\item \textbf{Hessian and Fisher--Rao metric}

In Eguchi theory, Fisher--Rao metric given by second derivative of $D$ with respect to parameters. Substituting above integral expression, obtain
$$
g^{\rm Fisher}_{ij}(\theta_0)
=\int_I W(\omega)\,
\partial_i\log\rho_{\rm rel}(\omega;\theta_0)\,
\partial_j\log\rho_{\rm rel}(\omega;\theta_0)\,
\mathrm{d}\omega,
$$
where weight function $W$ comes from second-order expansion of $F$ at reference point. Comparing with $g^{\rm phys}$ in Definition 3.1, both equal when $W$ chosen appropriately.

\item \textbf{Regularity and locality}

Eguchi regularity ensures existence and positive-definiteness of Hessian. In neighborhood of $\theta^\ast$, DOS difference varies smoothly with parameter, and $\rho_{\rm rel}(\omega;\theta^\ast)$ is nonzero, thus logarithmic derivative well-behaved. Weight function $W$ can be chosen as experimentally realizable frequency window, e.g., smooth compactly supported function, thus $g^{\rm phys}$ also well-defined.

\item \textbf{Conclusion}

Therefore in neighborhood of $\theta^\ast$, $g^{\rm phys}=g^{\rm Fisher}$; Theorem 3.2 proved. Detailed functional analysis and windowed Tauberian arguments in Appendix B.
\end{enumerate}

\subsection{Proof of Theorem 3.3 (``My Mind Is the Universe'')}

Theorem 3.3 is direct synthesis of Theorems 3.1 and 3.2 with Bayesian consistency.

\begin{enumerate}
\item By Theorem 3.1, there exists unified time scale equivalence class $[\tau]$ aligning all physical and cognitive times at universe end and observer end.

\item By Theorem 3.2, there exists metric $g$ such that universe parameter geometry $(\Theta,g^{\rm phys})$ and ``mind'''s information geometry $(\Theta,g^{\rm Fisher})$ are isometric in neighborhood of $\theta^\ast$.

\item By Bayesian consistency, posterior $\pi_\tau\Rightarrow\delta_{\theta^\ast}$; thus as $\tau\to\infty$, model manifold region actually accessed by ``mind'' contracts to neighborhood of $\theta^\ast$, where both geometries completely coincide.

\item Synthesizing three points satisfies all conditions of Definition 3.2, thus in unified time scale and information geometric sense, ``my mind is the universe'' holds.
\end{enumerate}

\section{Model Applications}

This section demonstrates concrete development of above framework in simplified situations through one-dimensional toy model and abstract EBOC perspective.

\subsection{One-Dimensional $\delta$ Potential Ring with Aharonov--Bohm Flux}

Consider one-dimensional ring of radius $L$ with $\delta$ potential of strength $\alpha$ at one point, and magnetic flux $\theta$ (Aharonov--Bohm flux) threading through ring. For energy $E=k^2$ scale, periodic boundary condition and jump condition give dispersion relation
$$
\cos\theta=\cos(kL)+\frac{\alpha}{k}\sin(kL).
$$

Scattering matrix $S(\omega;\alpha,\theta)$ of this system can be explicitly constructed; from its determinant phase and time delay obtain relative DOS $\rho_{\rm rel}(\omega;\alpha,\theta)$. Thus matrix universe parameter space is $\Theta=\mathbb{R}\times S^1$ with natural coordinates $\theta=(\alpha,\theta_{\rm AB})$.

``Self'' is set to be at some point on ring, continuously sending wave packets to system, measuring transmission and reflection probabilities and delays, constructing likelihood $P_\theta$ and posterior $\pi_\tau$ from observation data stream. Under standard regularity and identifiability, posterior converges to true parameter $\theta^\ast$.

Fisher--Rao metric $g^{\rm Fisher}$ given by Eguchi information geometry is equivalent to $g^{\rm phys}$ constructed from $\rho_{\rm rel}$ in neighborhood of $\theta^\ast$. Thus in this toy model, ``my mind'''s model manifold is isomorphic to universe parameter space, directly embodying content of main theorem.

\subsection{Matrix Universe and Mind in EBOC Perspective}

In EBOC (Eternal-Block Observer-Computing) perspective, universe as whole is viewed as static ``eternal block'': all events and causal relations already completely given; time is merely reading parameter of observer on this block. Matrix universe THE-MATRIX provides operator representation of this block universe: entire universe's boundary behavior encoded in frequency domain as scattering matrix $S(\omega)$ and its derivative $Q(\omega)$.

``Mind'''s dynamics is trajectory of observer performing computation and update on this static matrix universe: given initial prior $\pi_0$, as scale $\tau$ increases, observer samples and updates in matrix universe, producing trajectory $\{\pi_\tau\}$ on parameter manifold. As $\tau\to\infty$, this trajectory converges information-geometrically near true parameter point, thus ``mind'''s geometry coincides with universe's geometry in that region.

This picture shows ``my mind is the universe'' does not reduce universe to mind, nor dissolve mind into matter, but views both as unfoldings of single static structure in two aspects of ``boundary matrix'' and ``information manifold''.

\section{Engineering Proposals}

Although this paper mainly focuses on theoretical structure, ideas of matrix universe and information geometric alignment have some operability at engineering and experimental level. This section proposes several conceptual experimental and engineering schemes for simulating core structure of ``my mind is the universe'' in finite systems.

\subsection{Multi-Port Electromagnetic Scattering Networks}

Wigner--Smith time-delay matrix has been generalized to electromagnetic scattering networks for studying time response of complex multi-port systems. Imagine constructing multi-port network composed of waveguides and tunable loadings, whose scattering matrix $S(\omega;\theta)$ depends on set of controllable parameters $\theta$ (such as loading reactance, coupling strength). In radio frequency or microwave bands, $S(\omega)$ and $Q(\omega)$ can be precisely measured, experimentally constructing finite-dimensional version of matrix universe $U_{\rm mat}^{\rm lab}$.

Engineering steps include:

\begin{enumerate}
\item \textbf{Hardware level}: Design several ports; control scattering matrix parameters through programmable elements (such as tunable capacitor arrays or superconducting quantum interfaces).

\item \textbf{Observation level}: Measure $S(\omega)$ and $Q(\omega)$ at multiple frequency points, estimate $\rho_{\rm rel}(\omega;\theta)$.

\item \textbf{Inference level}: Construct statistical models $\{P_\theta\}$, implement online Bayesian update for parameter $\theta$, compute Fisher--Rao metric $g^{\rm Fisher}$ and compare with $g^{\rm phys}$ constructed from DOS difference.
\end{enumerate}

If consistency of two metrics observed under experimental error control, can be viewed as verification of main theorem in finite-dimensional experimental system.

\subsection{Implementation of Artificial ``Mind'' and Information Geometric Readout}

Based on above network, artificial intelligence or algorithmic agent can play role of ``mind'', whose internal representation is probability distribution $\pi_\tau$ for parameter $\theta$. Engineering can adopt variational Bayes or stochastic gradient MCMC methods to update $\pi_\tau$ online, estimating $g^{\rm Fisher}$ through Fisher information matrix.

If system hardware parameter $\theta$ drifts slowly, then ``universe'''s physical geometry evolves slowly over time, while artificial ``mind'' tracks this evolution through Bayesian updates. This allows studying dynamic process of ``mind'' and ``universe'' geometric alignment in controlled environment.

\subsection{Multiple Observers and Consensus Structure}

By placing multiple inference agents with limited communication on same scattering network, can simulate multi-observer causal network: each agent can only access partial ports and frequency intervals, thus their model families $\mathcal{M}_i$ and metrics $g_i$ differ locally. Through finite-bandwidth communication protocol synchronizing partial posteriors, can study consensus conditions of ``multiple minds same universe'', i.e., under what conditions all agents' parameter estimates and metrics converge to same geometric structure.

Such engineering research can provide concrete testable platform for abstract ``causal consensus geometry''.

\section{Discussion (Risks, Boundaries, Past Work)}

\subsection{Strength of Assumptions and Applicability}

This paper's theory relies on several strong assumptions:

\begin{enumerate}
\item Universe can be completely characterized by matrix universe THE-MATRIX, i.e., all observable structures viewable as functions of scattering matrix family. This is reasonable at local field theory and low-energy effective theory level, but needs caution in extreme cases with strong gravity and significant non-local effects.

\item Statistical model family $\{P_\theta\}$ is identifiable and satisfies Eguchi regularity. Real observers' cognitive mechanisms may be far from Bayesian optimal, and model families typically incomplete.

\item Bayesian consistency requires infinitely long observation time and idealized computational capacity. For observers with finite lifetime and finite computing power, ``my mind is the universe'' can only hold in approximate sense.
\end{enumerate}

These assumptions limit strict applicability domain of this paper's theorems, but provide clear limiting benchmark in abstract theoretical research.

\subsection{Relation to Existing Work}

At philosophical level, this paper's structural characterization does not completely overlap with traditional idealism or phenomenology: it does not claim material world is reduced to ``mind'', but emphasizes that under unified time scale and information geometric perspective, ``mind'' and ``universe'' are two unfoldings of same mathematical structure.

At physical level, this work relates to following directions:

\begin{enumerate}
\item \textbf{Block universe and timeless gravity theory}: Thermal time hypothesis and modular flow time incorporated into scattering--geometry--entropy framework through unified time scale in this paper.

\item \textbf{Holographic gravity and generalized entropy}: QNEC/QFC and their relation to Einstein equations provide foundation for boundary time geometry.

\item \textbf{Information geometry and statistical learning}: Divergence geometry of Eguchi--Amari and Bayesian consistency of Doob--Barron provide standard tools for mathematical model of ``mind''.
\end{enumerate}

This paper's contribution: unifying these structures originally belonging to different fields under specific proposition ``matrix universe THE-MATRIX and mind's model manifold are isometric'', providing theorem-provable framework for discussing ``observer--universe relation''.

\subsection{Risks and Open Problems}

Main risks include:

\begin{enumerate}
\item \textbf{Mathematical rigor}: Precise definition of matrix universe and its categorical equivalence with geometric universe need more systematic spectral and scattering theory tools. This paper only provides equivalence outline.

\item \textbf{Physical reasonableness}: In strong-field gravity, early universe, or non-equilibrium quantum field theory backgrounds, scattering description may not apply; need extension to more general frameworks.

\item \textbf{Realism of cognitive models}: Whether real biological or artificial observers approach Bayesian optimality is inconclusive; thus applicability of ``my mind is the universe'' at cognitive science level awaits further analysis.
\end{enumerate}

Open problems include: how to extend this framework when multiple interacting universe sectors or non-trivial topology exists; how to rewrite matrix universe and mind's isometric relation in discrete spacetime or causal set models; how to experimentally construct more complex multi-observer matrix universe simulation platforms.

\section{Conclusion}

Within framework of unified time scale, boundary time geometry, and matrix universe THE-MATRIX, this paper provides rigorous mathematical version of philosophical proposition ``my mind is the universe''. By characterizing universe as unified structure with generalized entropy and scattering matrices, characterizing ``my mind'' as information geometric model manifold and its Bayesian update trajectory, and proving under appropriate assumptions that universe parameter geometry is isometric to ``mind'''s information geometry under unified time scale, this paper gives following conclusions:

\begin{enumerate}
\item Time can be viewed as unified scale parameter of scattering phase gradient, time delay, modular flow, and generalized entropy variation;

\item Universe's matrix representation is matrix universe THE-MATRIX, whose spectral data provides ``external'' encoding for time and causality;

\item Observer ``mind'''s information geometry provides ``internal'' encoding for universe parameter structure;

\item Under identifiability and Bayesian consistency conditions, these two encodings are geometrically isometric, thus one can strictly say ``my mind is the universe''.
\end{enumerate}

This framework does not attempt to favor idealism or materialism in philosophical stance, but points out: in language provided by modern mathematical physics and information geometry, there exists precisely writable structural equivalence path between subject and world. Future work will be devoted to relaxing assumptions, introducing more physical and cognitive details, and exploring approximate manifestation of this structure under finite dimensions and finite observation time in experimental systems.

\section*{Acknowledgements \& Code Availability}

This work is based on existing achievements in scattering theory, operator algebras, quantum information, and information geometry for theoretical synthesis. Inference and metric calculations involved in principle implementable by standard numerical linear algebra and Bayesian inference libraries. Specific code implementation depends on chosen toy models and experimental platforms; this paper does not provide independent code package.

\section*{Appendix A: Equivalence Between Geometric Universe and Matrix Universe}

This appendix provides outline of equivalence structure between geometric universe $U_{\rm geo}$ and matrix universe $U_{\rm mat}$ in appropriate category, completing technical details of time scale alignment in Theorem 3.1.

\subsection*{A.1 From Geometry to Matrix}

On given Lorentz manifold $(M,g)$ with boundary, for each small causal diamond $D_{p,r}$ consider Klein--Gordon wave operator
$$
\Box_g+m^2+\xi R
$$
propagation under appropriate boundary conditions. Can define incoming and outgoing subspaces, constructing corresponding scattering operator $S_{D_{p,r}}(\omega)$. These scattering operators at family level form scattering family indexed by $(p,r)$; under appropriate truncation and normalization, can be embedded in diagonal blocks of total scattering matrix $S(\omega)$.

For entire topological space, using heat kernel methods and spectral shift function, difference of wave operator family can be encoded as spectral shift function $\xi(\omega)$, obtaining scattering determinant phase through Birman--Kreĭn formula. This process gives functor from geometric universe to matrix universe in suitable operator category.

\subsection*{A.2 From Matrix to Geometry}

Reverse construction more subtle, but in specific classes (e.g., non-compact asymptotically flat or scattering manifolds), by relating DOS slope, high-frequency expansion of phase shift to geometric invariants (volume, curvature integrals), can reconstruct equivalence class of metric from scattering data. This aspect has extensive research in spectral geometry and inverse scattering theory.

In this paper's framework, we only require existence of right inverse functor from matrix universe to geometric universe, such that composition is equivalent to original geometric universe in appropriate sense. This suffices to support transfer of time scale and generalized entropy structure.

\subsection*{A.3 Technical Alignment of Unified Time Scale}

Formally, unified time scale needs to satisfy:

\begin{enumerate}
\item Definition of $\kappa(\omega)$ consistent with DOS difference;

\item Choice of geometric time $\tau_{\rm geom}$ such that ADM/Bondi time pairs with boundary scattering data in energy flux and entropy flux;

\item Modular time $\tau_{\rm mod}$ under boundary state $\omega_\partial$ satisfies KMS condition, related to Hamiltonian time through temperature factor.
\end{enumerate}

In this background, unified time scale equivalence class $[\tau]$ can be constructed through following steps:

\begin{enumerate}
\item Fix reference energy window and reference state; define $\tau_{\rm scatt}$ through scattering data;

\item Use generalized entropy variation and QNEC/QFC to connect $\tau_{\rm geom}$ with $\tau_{\rm scatt}$;

\item Introduce modular flow on boundary algebra, aligning modular time with geometric time through thermal time hypothesis.
\end{enumerate}

Affine freedom (scale and origin) introduced in this process defines scale equivalence class $[\tau]$.

\section*{Appendix B: Relative Entropy, Eguchi Geometry and Spectral Data}

This appendix supplements technical proof of information geometric isometry in Theorem 3.2.

\subsection*{B.1 Spectral Expression of Relative Entropy}

In matrix universe, assume experimental results viewable as spectral measurements of certain function class $f$, e.g., measuring weighted integrals of scattering phase or time delay in some energy window. For each parameter $\theta$, probability distribution $P_\theta$ can be expressed as function pushforward of some spectral measure.

Under appropriate assumptions (such as absolutely continuous spectrum and finite-order Euler--Maclaurin expansion), Umegaki relative entropy can be written as
$$
D(\theta\Vert\theta_0)
=\int_I G\bigl(\rho_{\rm rel}(\omega;\theta),\rho_{\rm rel}(\omega;\theta_0)\bigr)\,\mathrm{d}\omega
+R(\theta,\theta_0),
$$
where $R$ is higher-order remainder. Through second-order expansion of $G$ near reference point $(\rho_0,\rho_0)$, obtain
$$
D(\theta\Vert\theta_0)
\approx \frac{1}{2}
\int_I \frac{\bigl(\rho_{\rm rel}(\omega;\theta)-\rho_{\rm rel}(\omega;\theta_0)\bigr)^2}
{\rho_{\rm rel}(\omega;\theta_0)}\,\mathrm{d}\omega.
$$

\subsection*{B.2 Fisher--Rao Metric and Spectral Fisher Information}

Taking second derivative of above expression with respect to parameters:
$$
g^{\rm Fisher}_{ij}(\theta_0)
=\int_I \frac{1}{\rho_{\rm rel}(\omega;\theta_0)}\,
\partial_i\rho_{\rm rel}(\omega;\theta_0)\,
\partial_j\rho_{\rm rel}(\omega;\theta_0)\,
\mathrm{d}\omega.
$$

Writing derivatives in logarithmic derivative form:
$$
g^{\rm Fisher}_{ij}(\theta_0)
=\int_I \rho_{\rm rel}(\omega;\theta_0)\,
\partial_i\log\rho_{\rm rel}(\omega;\theta_0)\,
\partial_j\log\rho_{\rm rel}(\omega;\theta_0)\,
\mathrm{d}\omega,
$$
thus consistent with $g^{\rm phys}$ in Definition 3.1 (corresponding to $W(\omega)=\rho_{\rm rel}(\omega;\theta_0)$).

\subsection*{B.3 Regularity and Windowing}

To ensure legality of above steps, need:

\begin{enumerate}
\item $\rho_{\rm rel}(\omega;\theta)$ sufficiently smooth in $\theta$ and $\omega$, positive-valued in energy window $I$;

\item High-frequency and low-frequency ends truncated by window function $W(\omega)$, ensuring integral convergence;

\item Remainders of Euler--Maclaurin and Poisson sums controllable, making contribution of $R(\theta,\theta_0)$ to second derivative negligible.
\end{enumerate}

Such conditions have systematic discussion in scattering theory and spectral geometry literature. Combined with general theory of Eguchi information geometry, completes proof of Theorem 3.2.

\section*{Appendix C: A Toy Model of ``Self--Heart--Universe'' Equivalence}

This appendix provides simplified model of one-dimensional $\delta$ potential ring with Aharonov--Bohm flux, illustrating concrete realization of ``my mind is the universe'' theorem in finite-dimensional case.

\subsection*{C.1 Model Definition}

Consider circle of radius $L$ with coordinate $x\in[0,2\pi L)$; place potential $V(x)=\alpha\delta(x)$ at $x=0$, threading magnetic flux $\Phi$ through circle, corresponding dimensionless flux $\theta_{\rm AB}=2\pi\Phi/\Phi_0$ where $\Phi_0$ is flux quantum. Boundary condition with flux is
$$
\psi(x+2\pi L)=\mathrm{e}^{\mathrm{i}\theta_{\rm AB}}\psi(x).
$$

At energy scale $E=k^2$, wave function satisfies free equation except at $x=0$; at $x=0$ satisfies jump condition
$$
\psi'(0^+)-\psi'(0^-)=\alpha\psi(0).
$$

Solving gives eigenequation
$$
\cos\theta_{\rm AB}
=\cos(kL)+\frac{\alpha}{k}\sin(kL).
$$

\subsection*{C.2 Matrix Universe and Parameter Space}

Scattering matrix of this system viewable as $2\times 2$ matrix (corresponding to clockwise/counterclockwise two channels); parameter space is $\Theta=\mathbb{R}\times S^1$ with coordinates $\theta=(\alpha,\theta_{\rm AB})$. From eigenvalue equation, scattering phase and time delay can be explicitly written, obtaining $\rho_{\rm rel}(\omega;\theta)$ and unified time scale density $\kappa(\omega;\theta)$.

\subsection*{C.3 Mind's Model Manifold and Fisher--Rao Metric}

Assume ``self'' can measure transmission and reflection probabilities at multiple frequency points, constructing likelihood $P_\theta$. In case where log-likelihood approximates Gaussian, Fisher information matrix equivalent to squared average of scattering amplitudes after parametric differentiation, allowing explicit calculation of Fisher--Rao metric $g^{\rm Fisher}(\theta)$.

On other hand, when constructing $g^{\rm phys}(\theta)$ from DOS difference and time delay, can use spectral expression given in previous section. In neighborhood of $\theta^\ast$, two metrics coincide, demonstrating concrete realization of Theorem 3.2.

\subsection*{C.4 Concretization of ``My Mind Is the Universe''}

In this model, ``universe'' is circular scattering system parametrized by $(\alpha,\theta_{\rm AB})$; ``my mind'' is probability distribution for these two parameters and its information geometric manifold. Through sufficiently many observations, posterior converges near true parameter point; in that region ``mind'''s information geometry coincides with universe's parameter geometry, concretely embodying structural meaning of ``my mind is the universe'' in simplified case.

\end{document}


