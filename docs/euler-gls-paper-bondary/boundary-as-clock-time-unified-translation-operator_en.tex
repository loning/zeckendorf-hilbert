\documentclass[12pt]{article}
\usepackage[utf8]{inputenc}
\usepackage[T1]{fontenc}
\usepackage{amsmath,amssymb,amsthm}
\usepackage{mathrsfs}
\usepackage{geometry}
\usepackage{hyperref}
\usepackage{braket}
\usepackage{graphicx}

\geometry{a4paper, margin=1in}
\hypersetup{colorlinks=true,linkcolor=blue,citecolor=blue,urlcolor=blue}

\theoremstyle{plain}
\newtheorem{theorem}{Theorem}[section]
\newtheorem{lemma}[theorem]{Lemma}
\newtheorem{proposition}[theorem]{Proposition}
\newtheorem{corollary}[theorem]{Corollary}
\newtheorem{axiom}{Axiom}

\theoremstyle{definition}
\newtheorem{definition}[theorem]{Definition}
\newtheorem{example}[theorem]{Example}
\newtheorem{remark}[theorem]{Remark}

\title{Boundary as Clock:\\
Time as Unified Translation Operator\\
of Phase--Spectral Shift--Modular Flow}

\author{Haobo Ma$^1$ \and Wenlin Zhang$^2$\\
\small $^1$Independent Researcher\\
\small $^2$National University of Singapore}

\date{\today}

\begin{document}

\maketitle

\begin{abstract}
Against background of general $C^\ast$-algebras and operator scattering theory, construct framework of ``time = boundary translation.'' Time not viewed as pre-given flow parameter in bulk domain but defined as unique translation scale generated by boundary spectral data, maintaining self-consistency among phase--spectral shift--modular flow triple readings.

Specifically: First, in scattering systems satisfying Birman--Krein conditions, take total scattering phase $\Phi(\omega)=\arg\det S(\omega)$, spectral shift function $\xi(\omega)$, relative state density $\Delta\rho(\omega)$, Wigner--Smith operator $Q(\omega)=-iS(\omega)^\dagger\partial_\omega S(\omega)$ as core; establish scale identity
$$
\frac{\varphi'(\omega)}{\pi}=\rho_{\mathrm{rel}}(\omega)=\frac{1}{2\pi}\operatorname{Tr}Q(\omega),\qquad\varphi(\omega):=\tfrac{1}{2}\Phi(\omega),
$$
interpreting as spectral ruler of ``scaling time by boundary phase gradient.''

Second, for given boundary observable algebra $\mathcal{A}_{\partial}$ and faithful state $\omega$'s GNS representation, introduce Tomita--Takesaki modular operator $\Delta$ and modular flow $\sigma_t^\omega$; under ``scattering--KMS consistency'' assumption prove: time parameter $t$ inferred from $S(\omega)$ and $Q(\omega)$ identical with modular time parameter under appropriate normalization.

Finally propose ``time equivalence principle'' and ``scale identity axiom'': any physical evolution of bulk--exterior data pairs equivalently rewritable as translation $U(t)=e^{-itH_{\partial}}$ generated by boundary generator $H_{\partial}$, where $t$ uniquely determined by boundary spectral measure readout function $\mathcal{T}$.

Under natural monotonicity and regularity assumptions, prove time scale satisfying these axioms unique in sense of additive and proportional transformations. Thus at purely operator--geometric and scattering information level, time characterized as ``boundary translation parameter with phase--spectral shift--modular flow triple reading self-consistent,'' providing testable theoretical basis for reconstructing spacetime and dynamics from boundary information.
\end{abstract}

\noindent\textbf{Keywords:} Time Essence; Scattering Phase; Spectral Shift Function; Wigner--Smith Operator; Modular Flow; Boundary Algebra; KMS State; Time Equivalence Principle

\noindent\textbf{MSC 2020:} 81U40, 81Q10, 46L55, 58J40

---

\section{Introduction and Historical Context}

Classical mechanics views time as absolutely uniformly flowing external parameter; in general relativity, time embedded as coordinate function with causal cone structure; standard quantum theory mostly uses internal--external parameter split, taking time as continuous parameter in Schrödinger equation.

In contrast, development of operator algebras and quantum statistical mechanics shows: given observable algebra $\mathcal{A}$ and state $\omega$, can construct intrinsic automorphism group family $\sigma_t^\omega$ via Tomita--Takesaki modular theory, naturally interpreted as ``modular time.'' This structure plays central role in KMS conditions and equilibrium state theory.

On other hand, in scattering theory, Wigner--Smith time delay concept interprets total scattering phase frequency derivative as average ``residence time'' particle experiences in potential field; this concept extensively generalized and verified in random media, chaotic scattering, electromagnetic, acoustic systems.

In rigorous operator scattering framework, Birman--Krein formula connects scattering determinant with spectral measure using spectral shift function $\xi(\lambda)$, giving
$$
\det S(\lambda)=\exp(-2\pi i\,\xi(\lambda)),
$$
while Krein trace formula connects spectral shift function between two operators with test function difference trace. These chains unify ``phase--spectral shift--relative state density'' as different aspects of same object.

Connes--Rovelli ``thermal time hypothesis'' further proposes: in generally covariant quantum theories, physical time flow shouldn't be given by preset external parameter but jointly determined by system's statistical state and observable algebra; time flow realized by state's modular automorphism group. This makes ``time = modular flow'' powerful candidate answer.

Above three threads respectively answer ``how to read out time from scattering phase,'' ``how to scale state density by spectral shift function,'' ``how to construct time flow from state and algebra.''

This paper's goal: within single, geometrically minimal-structure framework, unify these three; give rigorous existence and uniqueness conclusions. Core idea: introduce boundary algebra $\mathcal{A}_{\partial}$ as ``inside--outside'' information interface, requiring:

1. All observable outputs ultimately land on $\mathcal{A}_{\partial}$;
2. Given faithful state $\omega$ and scattering data $S(\omega)$, exists unique (up to affine) time translation group $\alpha_t$ such that:
   $\bullet$ $\alpha_t$ generated by self-adjoint operator $H_{\partial}$ in GNS representation;
   $\bullet$ $\alpha_t$ consistent with $\sigma_t^\omega$;
   $\bullet$ Time ruler under $\alpha_t$ normalized by scale identity.

From this perspective, time no longer ``flow variable in bulk domain'' but characterized as ``unique translation parameter realizing alignment between boundary spectral data and modular flow.''

This characterization maintains spiritual continuity with thermal time hypothesis but requires additional observable scattering data as scale anchor, making time have direct experimental readout.

---

\section{Model and Assumptions}

Give model structure and basic assumptions in abstract framework. Goal: obtain minimal condition family sufficient for applying Birman--Krein formula, spectral shift function, modular theory without relying on specific spacetime geometry.

\subsection{Boundary Algebra and State}

Let $\mathcal{A}_{\partial}$ be separable $C^\ast$-algebra representing ``boundary observables.'' Select faithful state $\omega:\mathcal{A}_{\partial}\to\mathbb{C}$; GNS representation denoted $(\pi_\omega,\mathcal{H}_\omega,\Omega_\omega)$ satisfying
$$
\omega(A)=\langle\Omega_\omega,\pi_\omega(A)\Omega_\omega\rangle,\qquad A\in\mathcal{A}_{\partial},
$$
$\Omega_\omega$ is cyclic and separating vector.

Assume strongly continuous $C^\ast$-automorphism family exists:
$$
\alpha_t:\mathcal{A}_{\partial}\to\mathcal{A}_{\partial},\qquad t\in\mathbb{R},
$$
realized on GNS space by unitary group $U(t)$:
$$
\pi_\omega(\alpha_t(A))=U(t)\pi_\omega(A)U(t)^{-1},\qquad U(t)\Omega_\omega=\Omega_\omega.
$$

Generator $H_{\partial}$ is self-adjoint operator satisfying $U(t)=e^{-itH_{\partial}}$.

Call $(\mathcal{A}_{\partial},\omega,\alpha_t)$ **boundary dynamical system**.

\subsection{Scattering System and Birman--Krein Setting}

Let $H_0,H$ be self-adjoint operators on separable Hilbert space $\mathcal{H}$ satisfying typical scattering assumptions:
1. $V:=H-H_0$ is trace-class perturbation;
2. $H_0$'s absolutely continuous spectrum dominates on energy axis $I\subset\mathbb{R}$;
3. Wave operators $W_\pm$ exist making
   $$
   W_\pm=\operatorname*{s-lim}_{t\to\pm\infty}e^{itH}e^{-itH_0}P_{\mathrm{ac}}(H_0);
   $$
4. Scattering operator $S:=W_+^\ast W_-$ is unitary on $P_{\mathrm{ac}}(H_0)\mathcal{H}$.

In energy representation, $S$ decomposable as fixed-energy scattering matrix family
$$
S(\lambda):\mathcal{K}(\lambda)\to\mathcal{K}(\lambda),\qquad\lambda\in I,
$$
where $\mathcal{K}(\lambda)$ is channel space at each energy. Assume $\lambda\mapsto S(\lambda)$ sufficiently smooth on $I$.

Under these assumptions, spectral shift function $\xi(\lambda)\in L^1_{\mathrm{loc}}(I)$ exists satisfying Krein trace formula
$$
\operatorname{Tr}(f(H)-f(H_0))=\int_I f'(\lambda)\,\xi(\lambda)\,d\lambda
$$
for sufficiently large function class.

Simultaneously, Birman--Krein formula gives scattering determinant--spectral shift function relation:
$$
\det S(\lambda)=\exp(-2\pi i\,\xi(\lambda))\quad\text{almost everywhere on }I.
$$

\subsection{Relative Density of States, Scattering Phase, Wigner--Smith Operator}

Define total scattering phase
$$
\Phi(\lambda):=\arg\det S(\lambda),\qquad\varphi(\lambda):=\tfrac{1}{2}\Phi(\lambda).
$$

From Birman--Krein formula:
$$
\Phi(\lambda)\equiv-2\pi\xi(\lambda)\pmod{2\pi},
$$
thus on locally continuous representative
$$
\Phi'(\lambda)=-2\pi\xi'(\lambda).
$$

Define relative state density (DOS difference)
$$
\Delta\rho(\lambda):=\rho(\lambda)-\rho_0(\lambda),
$$
where $\rho,\rho_0$ are state density functions of $H,H_0$. Under standard setting, $\Delta\rho$ and spectral shift function derivative satisfy
$$
\Delta\rho(\lambda)=-\xi'(\lambda)\quad\Rightarrow\quad\frac{1}{2\pi}\Phi'(\lambda)=\Delta\rho(\lambda),
$$
yielding
$$
\frac{\varphi'(\lambda)}{\pi}=\Delta\rho(\lambda).
$$

On other hand, for each energy, define Wigner--Smith delay operator
$$
Q(\lambda):=-iS(\lambda)^\dagger\partial_\lambda S(\lambda)
$$
on $\mathcal{K}(\lambda)$. $Q(\lambda)$ is self-adjoint operator; trace
$$
\operatorname{Tr}Q(\lambda)
$$
equals total scattering phase derivative under standard scattering framework:
$$
\Phi'(\lambda)=\operatorname{Tr}Q(\lambda).
$$

Merging above relations gives scale identity
$$
\frac{\varphi'(\lambda)}{\pi}=\Delta\rho(\lambda)=\frac{1}{2\pi}\operatorname{Tr}Q(\lambda).
$$

To avoid notation confusion, uniformly denote energy variable as $\omega$; write scale identity as
$$
\frac{\varphi'(\omega)}{\pi}=\rho_{\mathrm{rel}}(\omega)=\frac{1}{2\pi}\operatorname{Tr}Q(\omega),
$$
where $\rho_{\mathrm{rel}}(\omega):=\Delta\rho(\omega)$.

\subsection{Modular Flow, KMS Condition, Thermal Time}

On GNS representation $(\pi_\omega,\mathcal{H}_\omega,\Omega_\omega)$, define Tomita operator
$$
S_0\pi_\omega(A)\Omega_\omega=\pi_\omega(A)^\ast\Omega_\omega,\qquad A\in\mathcal{A}_{\partial}.
$$

Closure denoted $S$; polar decomposition $S=J\Delta^{1/2}$ gives antilinear unitary conjugation $J$ and modular operator $\Delta$. Tomita--Takesaki theorem asserts modular automorphism family exists:
$$
\sigma_t^\omega(A):=\Delta^{it}A\Delta^{-it},\qquad t\in\mathbb{R},
$$
forming one-parameter automorphism group on $\mathcal{A}_{\partial}$; $\omega$ satisfies KMS condition for $\sigma_t^\omega$.

Formally write modular generator
$$
K_\omega:=-\log\Delta,\qquad\sigma_t^\omega(A)=e^{itK_\omega}Ae^{-itK_\omega}.
$$

Connes--Rovelli thermal time framework proposes: in generally covariant field theories, physical time flow determinable by given state and observable algebra, specifically modular flow $\sigma_t^\omega$.

This paper restricts this idea to boundary algebra $\mathcal{A}_{\partial}$; requires modular time consistent with time parameter scaled from scattering phase--spectral shift--Wigner--Smith operator.

---

\section{Main Results (Theorems and Alignments)}

Propose ``time equivalence principle,'' ``scale identity axiom,'' ``modular consistency axiom''; give existence and uniqueness theorem for unified time scale.

\subsection{Time Equivalence Principle and Boundary Generator}

\begin{axiom}[Time Equivalence Principle]
Consider realizable bulk--exterior data pairs $(D_{\mathrm{in}},D_{\mathrm{out}})$ as vectors or states in $\mathcal{H}_{\mathrm{in}},\mathcal{H}_{\mathrm{out}}$. Boundary Hilbert space $\mathcal{H}_\partial$, self-adjoint operator $H_{\partial}$, unitary group
$$
U(t)=e^{-itH_{\partial}},\qquad t\in\mathbb{R}
$$
exist such that for any realizable data pair, real number $t$ exists satisfying
$$
K_{\partial}^{\mathrm{out}}D_{\mathrm{out}}=U(t)\,K_{\partial}^{\mathrm{in}}D_{\mathrm{in}}.
$$
\end{axiom}

\begin{axiom}[Boundary Generator Axiom]
$C^\ast$-algebra $\mathcal{A}_{\partial}$ and faithful state $\omega$ exist making above $U(t)$ realize boundary dynamics on $\mathcal{H}_\omega$:
$$
\alpha_t(A)=U(t)AU(t)^{-1},\qquad A\in\mathcal{A}_{\partial},
$$
and $U(t)\Omega_\omega=\Omega_\omega$.
\end{axiom}

Call $(\mathcal{A}_{\partial},\omega,U(t))$ **boundary as clock** data.

\subsection{Gauge Fixing by Phase--Spectral-Shift--WS Trace}

\begin{axiom}[Scale Identity Axiom]
Scattering system satisfies aforementioned Birman--Krein and Wigner--Smith conditions. In energy window $I$, phase $\varphi(\omega)$, relative state density $\rho_{\mathrm{rel}}(\omega)$, Wigner--Smith operator $Q(\omega)$ exist satisfying
$$
\frac{\varphi'(\omega)}{\pi}=\rho_{\mathrm{rel}}(\omega)=\frac{1}{2\pi}\operatorname{Tr}Q(\omega),\qquad\omega\in I.
$$

Define time differential as
$$
dt:=\frac{1}{2\pi}\operatorname{Tr}Q(\omega)\,d\omega.
$$

Given reference point $\omega_0,t_0$, time scale determined by
$$
t(\omega)-t_0=\int_{\omega_0}^{\omega}\frac{1}{2\pi}\operatorname{Tr}Q(\tilde{\omega})\,d\tilde{\omega}.
$$
\end{axiom}

Scale identity axiom transforms scattering spectral structure on frequency axis into boundary translation scale on time axis.

\subsection{Modular Consistency and Unified Time Flow}

\begin{axiom}[Modular Consistency Axiom]
For boundary dynamical system $(\mathcal{A}_{\partial},\omega,\alpha_t)$, assume constant $c>0$ exists such that for all $A\in\mathcal{A}_{\partial}$
$$
\alpha_t(A)=\sigma_{ct}^\omega(A),
$$
where $\sigma_t^\omega$ is Tomita--Takesaki modular flow. Absorbing $c$ into time unit, can losslessly rewrite as
$$
\alpha_t(A)=\sigma_t^\omega(A).
$$

Thus boundary generator $H_{\partial}$ and modular generator $K_\omega$ differ only by constant shift:
$$
H_{\partial}=K_\omega+\lambda\mathbf{1},\qquad\lambda\in\mathbb{R}.
$$
\end{axiom}

\begin{definition}[Time Structure]
Call quadruple
$$
\mathscr{T}=(\mathcal{A}_{\partial},\omega,\alpha_t,S(\omega))
$$
time structure if satisfying:
1. $\omega$ is faithful normal state on $\mathcal{A}_{\partial}$;
2. $\alpha_t$ realized on GNS representation by $U(t)=e^{-itH_{\partial}}$, $H_{\partial}$ self-adjoint;
3. Scattering matrix family $S(\omega)$ and corresponding $\varphi(\omega),Q(\omega),\rho_{\mathrm{rel}}(\omega)$ exist satisfying scale identity;
4. $\alpha_t=\sigma_t^\omega$ as automorphism groups consistent.
\end{definition}

\begin{theorem}[Time Scale Existence]
Let $\mathscr{T}$ be time structure; assume in energy window $I$, $\rho_{\mathrm{rel}}(\omega)$ is integrable continuous function nonzero on some interval. Then local bijection
$$
\omega\longleftrightarrow t(\omega)
$$
exists given by scale identity axiom such that:

1. For all $A\in\mathcal{A}_{\partial}$,
   $$
   \alpha_{t(\omega)}(A)=\sigma_{t(\omega)}^\omega(A)=\Delta^{it(\omega)}A\Delta^{-it(\omega)};
   $$

2. For scattering side, can view $S(\omega)$ as $S(t)$ satisfying
   $$
   \frac{d}{dt}\varphi(\omega(t))=\pi\,\rho_{\mathrm{rel}}(\omega(t))=\tfrac{1}{2}\operatorname{Tr}Q(\omega(t)),
   $$
   rewriting phase gradient, relative state density, Wigner--Smith trace as time derivatives.

In other words, time parameter $t$ simultaneously parametrizes modular flow and scattering time readouts, making latter ``observable scale'' of former.
\end{theorem}

\begin{theorem}[Additive--Proportional Uniqueness of Scale]
Under Theorem assumptions, further assume:
1. $\rho_{\mathrm{rel}}(\omega)$ strictly positive or strictly negative in considered energy window;
2. Modular flow $\sigma_t^\omega$ non-trivial: no nonzero time $t$ makes $\sigma_t^\omega$ identity.

If another time parameter $\tilde{t}$ and map $\omega\mapsto\tilde{t}(\omega)$ exist such that:
1. $\alpha_{\tilde{t}}$ also realizes as modular flow: $\alpha_{\tilde{t}}=\sigma_{\tilde{t}}^\omega$;
2. Scale identity holds under $\tilde{t}$ in same energy window.

Then constants $a>0$ and $b\in\mathbb{R}$ exist making
$$
\tilde{t}=at+b.
$$

Time scale satisfying axiom system unique in affine transformation sense; time reversal $(a<0)$ excluded.
\end{theorem}

---

\section{Proofs}

Provide proof structure of main theorems; concentrate technical operator scattering and modular theory details in appendices.

\subsection{Birman--Krein Identity and Phase--Spectral-Shift Relation}

Under previous assumptions, spectral shift function $\xi(\lambda)$ satisfies Krein trace formula. Taking smoothed approximation of $f(\lambda)=\chi_{(-\infty,E]}(\lambda)$ yields
$$
\xi(E)=\operatorname{Tr}(P_H((-\infty,E])-P_{H_0}((-\infty,E])),
$$
thus
$$
\xi'(\lambda)=-(\rho(\lambda)-\rho_0(\lambda))=-\Delta\rho(\lambda)
$$
holds in distributional sense.

On other hand, Birman--Krein formula gives $\det S(\lambda)=\exp(-2\pi i\,\xi(\lambda))$. Taking continuous branch and differentiating with respect to $\lambda$:
$$
\Phi'(\lambda)=-2\pi\xi'(\lambda)=2\pi\Delta\rho(\lambda),
$$
i.e.,
$$
\frac{1}{2\pi}\Phi'(\lambda)=\Delta\rho(\lambda).
$$

With $\varphi=\Phi/2$ obtain
$$
\frac{\varphi'(\lambda)}{\pi}=\Delta\rho(\lambda).
$$

\subsection{Wigner--Smith Trace and Relative Density of States}

Wigner--Smith delay operator defined as
$$
Q(\lambda)=-iS(\lambda)^\dagger\partial_\lambda S(\lambda).
$$

In momentum or channel basis, $Q(\lambda)$ is finite or countable-dimensional matrix satisfying
$$
\operatorname{Tr}Q(\lambda)=-i\operatorname{Tr}(S(\lambda)^\dagger\partial_\lambda S(\lambda)).
$$

On other hand, logarithmic derivative of $\det S(\lambda)$ satisfies
$$
\partial_\lambda\log\det S(\lambda)=\operatorname{Tr}(S(\lambda)^{-1}\partial_\lambda S(\lambda))=\operatorname{Tr}(S(\lambda)^\dagger\partial_\lambda S(\lambda)),
$$
using $S(\lambda)$'s unitarity. Taking imaginary part yields
$$
\partial_\lambda\Phi(\lambda)=\operatorname{Tr}Q(\lambda),
$$
thus
$$
\Delta\rho(\lambda)=\frac{1}{2\pi}\operatorname{Tr}Q(\lambda).
$$

This directly verifiable through spectral representation construction of $H,H_0$ and $S(\lambda)$ in rigorous scattering theory; widely used in multiphysics applications.

---

\section{Model Applications}

Give several concrete models illustrating ``boundary as clock'' realization in different physical scenarios.

\subsection{One-Dimensional Schrödinger Scattering}

Consider 1D Schrödinger operator
$$
H_0=-\frac{d^2}{dx^2},\qquad H=-\frac{d^2}{dx^2}+V(x)
$$
on $\mathcal{H}=L^2(\mathbb{R})$; assume $V\in L^1(\mathbb{R},(1+|x|)dx)$ real-valued. Scattering theory completely solvable; reflection, transmission amplitudes $r(k),t(k)$ exist; energy $E=k^2$.

Select boundary Hilbert space as momentum space channels
$$
\mathcal{H}_\partial\simeq L^2(\mathbb{R}_k)\oplus L^2(\mathbb{R}_k);
$$
boundary algebra $\mathcal{A}_{\partial}$ as closure of bounded multiplication operators and finite-rank perturbations; $\omega$ as equilibrium state (e.g., Fermi--Dirac or Boltzmann weight).

Under appropriate thermal equilibrium limit, modular flow of $\mathcal{A}_{\partial}$ and $\omega$ can correspond to Schrödinger evolution, realizing consistency between modular time and scattering time scale in energy window.

Time readout $t$ given by
$$
t(k)-t_0=\int_{k_0}^{k}\frac{1}{2\pi}\operatorname{Tr}Q(\tilde{k})\,\frac{dE}{d\tilde{k}}\,d\tilde{k}=\int_{E_0}^{E}\Delta\rho(\tilde{E})\,d\tilde{E},
$$
transforming energy axis into boundary time axis.

\subsection{Local Algebras and Rindler Wedge}

In algebraic quantum field theory, von Neumann algebra $\mathcal{A}(W)$ associated with Minkowski space wedge region $W$'s modular flow in vacuum state given by Bisognano--Wichmann theorem as Lorentz boost preserving wedge.

This means: for Rindler observer, proper time flow proportional to modular time on $\mathcal{A}(W)$; Connes--Rovelli thermal time hypothesis generalizes this as ``time = modular flow'' paradigm in generally covariant field theories.

In this background, can view wedge boundary (or more generally double cone boundary) as this paper's boundary algebra $\mathcal{A}_{\partial}$; scattering matrix constructed from far-region field incident/outgoing modes; Wigner--Smith delay matrix characterizes field residence time near wedge.

Through scale identity, can align ``geometrically defined proper time'' with ``scattering phase derivative,'' realizing boundary as clock in concrete quantum field theory models.

---

\section{Engineering Proposals}

Propose several experimental and engineering schemes for testing key equations and scale identity construction of ``boundary as clock'' on controllable platforms.

\subsection{Microwave Network with Vector Network Analyzer}

In microwave engineering, complex networks (waveguides, resonant cavities, couplers) commonly described by multi-port scattering matrix $S(\omega)$, directly measurable by vector network analyzer (VNA).

Construct multi-port network approximating dissipationless in working frequency band satisfying scattering theory regularity requirements:

1. Measure $S(\omega)$ with VNA; numerically differentiate to get $\partial_\omega S(\omega)$;
2. Construct Wigner--Smith matrix $Q(\omega)=-iS(\omega)^\dagger\partial_\omega S(\omega)$; compute $\operatorname{Tr}Q(\omega)$;
3. Through appropriate energy--frequency normalization, map $\omega$ axis to time axis
   $$
   t(\omega)-t_0=\int_{\omega_0}^{\omega}\frac{1}{2\pi}\operatorname{Tr}Q(\tilde{\omega})\,d\tilde{\omega};
   $$
4. View network as concrete realization of boundary algebra: port modes span $\mathcal{H}_\partial$; network interior ``bulk--exterior domain'' dynamics project onto ports giving $S(\omega)$;
5. Under statistical steady state, construct empirical state $\omega_{\mathrm{exp}}$ for port excitation and output; approximately recover effective modular flow through energy flow conservation and equilibrium conditions; test consistency with time translation defined by $t(\omega)$ in correlation functions.

If measured $\operatorname{Tr}Q(\omega)$ frequency integral and network interior average residence time plus energy storage rate satisfy scale identity, viewable as engineering-level verification of ``boundary phase gradient scales time.''

---

\section{Discussion (Risks, Boundaries, Past Work)}

Discuss applicability domain, potential risks, relation to existing work of ``boundary as clock'' framework.

1. **Dependence on scattering regularity**: Scale identity depends on Birman--Krein formula and well-defined spectral shift function, requiring $H-H_0$ at least trace-class perturbation; scattering matrix smooth in energy window. For strong coupling, many-body pure point spectrum dominant systems, framework requires modification or generalization.

2. **Dynamical interpretation of modular flow**: Thermal time hypothesis criticism points out modular flow may not always obtain natural dynamical interpretation, especially lacking geometric background or equilibrium state assumptions. This paper by requiring modular flow consistent with scattering time scale actually selects family of states and algebras with good dynamical meaning; however, this selection itself requires additional physical input and experimental calibration.

3. **Locality and causal structure**: This paper doesn't explicitly introduce spacetime causal structure, only working at boundary algebra and scattering channel level. To elevate ``boundary as clock'' to complete ``time geometry,'' requires further introducing local subalgebras, causal embedding, macroscopic geometry reconstruction procedure. Closely related to algebraic quantum field theory research on reconstructing spacetime structure through local algebras.

4. **Time arrow and irreversibility**: Scale identity only characterizes time parameter scale and direction, not directly explaining time arrow origin. Binding time arrow with relative entropy monotonicity, generalized entropy conditions, quantum focusing conditions is natural direction for further work.

5. **Relation to thermal time hypothesis**: Connes--Rovelli framework proposes time given by modular flow; this paper unifies modular flow with scattering spectral data scale, forming ``boundary as clock.'' Both consistent on point ``time determined by state and algebra,'' but this paper emphasizes: time not only determined by modular flow existence but also fixed by scattering spectral data ruler; latter has direct experimental readability.

Overall, ``boundary as clock'' framework organizes seemingly scattered theoretical tools—spectral shift function, Wigner--Smith delay matrix, modular flow, thermal time hypothesis—into unified ``boundary time'' picture, with applicability domain jointly constrained by scattering regularity and modular flow's geometrically interpretable character.

---

\section{Conclusion}

Within purely operator--geometric and scattering spectral theory framework, constructed unified scale of ``time = boundary translation.''

Core achievements:
1. Under Birman--Krein and Wigner--Smith conditions, scattering phase derivative, relative state density, Wigner--Smith delay matrix trace satisfy scale identity, transforming frequency axis into time axis.

2. In GNS representation of boundary algebra $\mathcal{A}_{\partial}$ and faithful state $\omega$, introduce modular operator $\Delta$ and modular flow $\sigma_t^\omega$; through modular consistency axiom unify boundary dynamics $\alpha_t$ with $\sigma_t^\omega$, fixing time scale as ``modular time.''

3. Define time structure $\mathscr{T}=(\mathcal{A}_{\partial},\omega,\alpha_t,S(\omega))$; prove under natural regularity and monotonicity assumptions, unique (up to affine) time scale $t$ exists simultaneously parametrizing modular flow and scattering time readouts. This time scale interpreted as ``boundary as clock'' reading.

4. Demonstrate modeled realization and engineering-level verification paths in 1D Schrödinger scattering, wedge local algebras, microwave/acoustic/photonic scattering networks.

In this framework, time no longer viewed as preset continuous parameter in bulk domain but boundary spectral data and modular flow coordinated translation scale, unique parameter where ``phase--spectral shift--modular flow'' triple reading self-consistent. This provides rigorous and testable theoretical base point for expanding spacetime and dynamics reconstruction from boundary information.

---

\begin{thebibliography}{99}
\bibitem{ref1} Birman--Krein formula: standard scattering theory literature.
\bibitem{ref2} Wigner--Smith time delay: various reviews.
\bibitem{ref3} Tomita--Takesaki modular theory: operator algebra texts.
\bibitem{ref4} Connes--Rovelli thermal time: Class. Quantum Grav. 11 (1994).
\bibitem{ref5} Spectral shift function applications: Gesztesy--Simon et al.
\end{thebibliography}

\appendix

\section{Operator Scattering and Scale Identity Rigorous Framework}
[Further refinement of operator scattering and spectral shift function theory framework...]

\section{Modular Flow and KMS Condition}
[Tomita--Takesaki theorem, KMS condition, time flow uniqueness...]

\section{Explicit Time Scale Calculation in 1D Schrödinger Model}
[Jost solutions, spectral shift function, DOS, Wigner--Smith delay time...]

\end{document}

