\documentclass[12pt]{article}
\usepackage[utf8]{inputenc}
\usepackage[T1]{fontenc}
\usepackage{amsmath,amssymb,amsthm}
\usepackage{mathrsfs}
\usepackage{geometry}
\usepackage{hyperref}
\usepackage{braket}
\usepackage{graphicx}

\geometry{a4paper, margin=1in}
\hypersetup{colorlinks=true,linkcolor=blue,citecolor=blue,urlcolor=blue}

\theoremstyle{plain}
\newtheorem{theorem}{Theorem}[section]
\newtheorem{lemma}[theorem]{Lemma}
\newtheorem{proposition}[theorem]{Proposition}
\newtheorem{corollary}[theorem]{Corollary}
\newtheorem{postulate}{Postulate}

\theoremstyle{definition}
\newtheorem{definition}[theorem]{Definition}
\newtheorem{example}[theorem]{Example}
\newtheorem{remark}[theorem]{Remark}

\title{Boundary as Unified Stage:\\
Variational Completeness, Time Scale, Topological Branching}

\author{Haobo Ma$^1$ \and Wenlin Zhang$^2$\\
\small $^1$Independent Researcher\\
\small $^2$National University of Singapore}

\date{\today}

\begin{document}

\maketitle

\begin{abstract}
From first principles, elevate ``boundary'' from passive geometric appendage to unified physical stage. Propose axiomatic framework with boundary as fundamental object, gluing three seemingly separate structures—Gibbons--Hawking--York (GHY) boundary term and Brown--York quasilocal quantities in gravitational variation, spectral shift function and Wigner--Smith time delay in scattering theory, modular flow and relative entropy monotonicity in operator algebras—as different projections of same ``boundary time geometry.''

Core viewpoint: boundary not merely ``separating'' bulk domains but compressing bulk continuous changes into finite measurable differences (energy difference, time difference, topological class difference, causal orientation).

On geometric--variational side, review and refine complete variational structure of Einstein--Hilbert--GHY--corner--null boundary terms, proving under fixed induced metric condition, requiring well-defined variational principle uniquely selects ``variationally complete'' boundary geometry class, deriving Brown--York boundary stress--energy tensor as boundary readout of bulk ``difference.''

On spectral--scattering side, under standard trace-class perturbation assumptions, starting from Birman--Krein formula, give scale identity relating total scattering phase derivative, relative state density, Wigner--Smith time delay trace, interpreting boundary scattering phase tiny variations uniformly as ``state number changes'' and ``residence time changes.''

On operator algebra--information side, introduce modular flow under general boundary observable algebra and faithful state context, write modular Hamiltonian as energy flow integral along null boundary (or wedge boundary), characterize boundary time arrow unidirectionality using relative entropy monotonicity and quantum energy conditions.

Moreover, define ``discriminant boundary'' and $\mathbb{Z}_2$ branch index in parameter space, explaining after excluding spectral anomaly and topological phase transition hypersurfaces, spin structure defined by scattering matrix square root forms Null--Modular double cover on parameter space. Double cover's non-triviality memorized by boundary as spectral flow parity and intersection number parity, compressing ``going around once'' continuous deformation into discrete topological class difference.

Finally, give unified ``boundary time geometry'' definition: boundary carries geometric--spectral--information--topological data set making (i) variation well-defined, (ii) time scale identity holds, (iii) modular flow and generalized entropy monotonicity determine time arrow, (iv) $\mathbb{Z}_2$ index defined on discriminant boundary gives topological branching.

Main theorem: under appropriate assumptions, can select unique (in affine rescaling sense) time parameter on boundary making these four structure time parameters belong to same scale equivalence class, thus precisely formalizing ``boundary generates difference'' into testable, computable unified framework.
\end{abstract}

\noindent\textbf{Keywords:} Boundary Geometry; Variational Completeness; Spectral Shift; Time Delay; Modular Flow; Relative Entropy; Topological Index; $\mathbb{Z}_2$ Branching

---

\section{Introduction}

Boundaries appear ubiquitously in almost all corners of physics and mathematics: spatial infinity and black hole horizons in general relativity, Cauchy surfaces and causal diamond boundaries in quantum field theory, material interfaces and topological defects in condensed matter, incoming/outgoing asymptotic boundaries in scattering theory, even discriminant hypersurfaces of topological phase transitions in parameter space.

Traditional treatments mostly view boundary as ``geometric appendage of bulk domain'': in field theory need to specify boundary conditions, in geometry need to supplement boundary terms to correct variation, in scattering theory boundary merely way of imposing asymptotic conditions at infinity.

This paper attempts to advance more radical viewpoint: in unified framework, **boundary should be viewed as true stage of physical structure**. Bulk ``continuous changes'' only become measurable, comparable, optimizable objects when translated on boundary into finite-dimensional ``scale differences.''

More specifically:
$\bullet$ On **geometric--variational** level, requiring Einstein--Hilbert action variation well-defined in boundary case forces us to introduce Gibbons--Hawking--York term and corner/null boundary terms on boundary; Brown--York quasilocal stress--energy tensor naturally appears as boundary ledger of ``how much'' bulk geometry and matter distribution differ.

$\bullet$ On **spectral--scattering** level, Birman--Krein spectral shift function and Wigner--Smith time delay translate ``tiny phase changes with frequency'' into ``state density difference'' and ``residence time difference''; this structure naturally is boundary structure since all scattering readouts measured on boundary (or at infinity).

$\bullet$ On **operator algebra--information** level, Tomita--Takesaki modular theory and relative entropy monotonicity show: given boundary observable algebra and faithful state, can define modular flow and its generator (modular Hamiltonian), often writable as energy flow integral along boundary; flow parameter after appropriate normalization interpretable as ``intrinsic time'' on boundary.

$\bullet$ On **topological--parameter space** level, scattering matrix spectrum and phase in parameter space often have discriminant hypersurfaces; excluding these anomalous points, spin structure defined by square-root scattering matrix forms $\mathbb{Z}_2$ double cover on parameter space, non-triviality manifested on boundary as ``extra minus sign after going around once.''

These seemingly scattered phenomena point to common structure: **boundary responsible for compressing invisible bulk changes into visible difference scales**. This paper's goal: starting from this intuition, construct rigorous axiomatic boundary framework; give theorem series unifying variational completeness, time scale identity, modular flow time arrow, topological branching into ``boundary time geometry'' context.

---

\section{Preliminaries and Notation}

\subsection{Geometry and Variation}

Let $(M,g)$ be four-dimensional Lorentzian manifold with boundary $\partial M$. On non-null (spacelike or timelike) boundary, denote induced metric $h_{ab}$, outward normal $n^a$, extrinsic curvature $K_{ab}=h_a^{\ c}h_b^{\ d}\nabla_cn_d$, trace $K=h^{ab}K_{ab}$.

Einstein--Hilbert action defined as
$$
S_{\mathrm{EH}}(g)=\frac{1}{16\pi G}\int_M R(g)\sqrt{-g}\,d^4x,
$$
where $R$ is scalar curvature. Well-known: in boundary case, $S_{\mathrm{EH}}$ alone not well-defined under variation fixing $h_{ab}$; variation produces boundary term depending on $\delta(\partial g)$. To correct, introduce Gibbons--Hawking--York boundary term
$$
S_{\mathrm{GHY}}(g)=\frac{\varepsilon}{8\pi G}\int_{\partial M}K\sqrt{|h|}\,d^3x,
$$
where $\varepsilon=+1$ for spacelike boundary, $\varepsilon=-1$ for timelike boundary. For corner and null boundary cases, need introduce additional corner and null boundary terms; see Appendix A.

\subsection{Scattering Theory and Time Delay}

Let $\mathcal{H}$ be complex Hilbert space, $H_0$ and $H=H_0+V$ self-adjoint operators. Assume perturbation $V$ is $H_0$-relative trace-class making wave operators
$$
W_\pm=\mathrm{s}\text{-}\lim_{t\to\pm\infty}e^{iHt}e^{-iH_0t}
$$
exist and complete; then scattering operator defined as
$$
S=W_+^\ast W_-.
$$

Under appropriate conditions, $S$ writable in energy representation as fiber decomposition
$$
S=\int^\oplus S(\omega)\,d\mu(\omega),
$$
where $S(\omega)$ is finite-dimensional (or separable) scattering matrix at energy $\omega$. Define Wigner--Smith time delay operator
$$
Q(\omega)=-i\,S(\omega)^\dagger\partial_\omega S(\omega).
$$

Birman--Krein spectral shift function $\xi(\lambda)$ satisfies
$$
\det S(\lambda)=\exp(-2\pi i\,\xi(\lambda)).
$$
Under appropriate regularity assumptions, differentiable; derivative $\xi'(\lambda)$ related to state density difference. Adopt notation Krein spectral shift density $\rho_{\mathrm{rel}}(\omega)=\xi'(\omega)$.

\subsection{Modular Theory, Modular Flow, Relative Entropy}

Let $\mathcal{A}$ be $C^\ast$ algebra or von Neumann algebra, $\omega$ faithful normal state on it. GNS construction gives triple $(\pi_\omega,\mathcal{H}_\omega,\Omega_\omega)$ where $\Omega_\omega$ is cyclic vector. Tomita operator $S_\omega$'s polar decomposition produces modular operator $\Delta_\omega$ and conjugation $J_\omega$. Modular flow defined as
$$
\sigma_t^\omega(A)=\Delta_\omega^{it}A\Delta_\omega^{-it},\quad A\in\mathcal{A}.
$$

If self-adjoint operator $K_\omega$ exists satisfying $\Delta_\omega=e^{-K_\omega}$, formally
$$
\sigma_t^\omega(A)=e^{iK_\omega t}Ae^{-iK_\omega t}.
$$

For two states $\omega,\varphi$, relative entropy defined as
$$
S(\omega\|\varphi)=\operatorname{tr}(\rho_\omega(\log\rho_\omega-\log\rho_\varphi)).
$$
Under appropriate generality satisfies monotonicity: non-increasing under restriction to subalgebra or subregion.

In relativistic QFT double cone/wedge region cases, modular Hamiltonian $K_\omega$ often writable as energy--momentum tensor integral along boundary direction, giving modular time geometric meaning.

\subsection{Spectral Flow and $\mathbb{Z}_2$ Index}

Let $\{A_t\}_{t\in[0,1]}$ be family of self-adjoint Fredholm operator paths; spectral flow $\mathrm{Sf}(\{A_t\})$ defined as oriented number of eigenvalues crossing zero. If only caring about parity, define $\mathbb{Z}_2$ index
$$
\nu_{\mathbb{Z}_2}(\{A_t\})=(-1)^{\mathrm{Sf}(\{A_t\})}\in\{\pm1\}.
$$

On parameter space $X$, can connect spectral flow parity with loop intersection number parity for discriminant set $D\subset X$, forming $\mathbb{Z}_2$ topological index. Will use this language to describe scattering square-root branch structure.

---

\section{Axiomatic Definition of Physical Boundary System}

Give ``physical boundary system'' definition adopted in this paper, unifying geometric, scattering, modular flow, topological data into same boundary framework.

\subsection{Boundary Data Quadruple}

\begin{definition}[Physical Boundary System]
Physical boundary system consists of quadruple
$$
\mathfrak{B}=(\partial M,\mathcal{A}_{\partial},\omega_{\partial},\mathcal{S}_{\partial})
$$
where:

1. $\partial M$ is codimension-one boundary of four-dimensional spacetime $M$, equipped with induced metric $h_{ab}$ and extrinsic curvature $K_{ab}$, plus possible corners and null sheet segments;

2. $\mathcal{A}_{\partial}$ is boundary observable algebra associated with $\partial M$ (e.g., boundary-restricted field operator algebra or scattering channel algebra);

3. $\omega_{\partial}$ is faithful normal state on $\mathcal{A}_{\partial}$, giving modular flow $\sigma_t^{\omega_\partial}$;

4. $\mathcal{S}_{\partial}$ is set of boundary scattering and topological data, including:
   $\bullet$ Scattering matrix $S(\omega)$ in energy representation;
   $\bullet$ Frequency measure compatible with $\mathcal{A}_{\partial}$;
   $\bullet$ Discriminant subset $D\subset X$ on parameter space $X$ and $\mathbb{Z}_2$ index defined from it.
\end{definition}

In concrete models, $\partial M$ can be artificial boundary of finite region, small causal diamond boundary, black hole horizon, AdS asymptotic boundary, material interface, even ``discriminant boundary in parameter space'' in abstract sense.

\subsection{Four Fundamental Postulates}

\begin{postulate}[A1: Variational Completeness]
Exists action functional composed of bulk and boundary
$$
S_{\mathrm{tot}}=S_{\mathrm{bulk}}[g,\Phi]+S_{\mathrm{bdy}}[h,K,\Phi|_{\partial M}],
$$
where specific boundary term $S_{\mathrm{bdy}}$ (including GHY, corner, null boundary terms) makes under variation fixing boundary induced metric and matter boundary data $(h_{ab},\Phi|_{\partial M})$, first-order variation of $S_{\mathrm{tot}}$ depends only on bulk variation and equivalent to given field equations (e.g., Einstein--matter equations).
\end{postulate}

\begin{postulate}[A2: Scale Identity]
Exists frequency variable $\omega$ and corresponding scattering matrix $S(\omega)$ making following scale identity hold:
$$
\frac{\varphi'(\omega)}{\pi}=\rho_{\mathrm{rel}}(\omega)=\frac{1}{2\pi}\operatorname{tr}Q(\omega),
$$
where $\varphi(\omega)=\tfrac{1}{2}\arg\det S(\omega)$, $\rho_{\mathrm{rel}}(\omega)$ is relative state density, $Q(\omega)=-iS(\omega)^\dagger\partial_\omega S(\omega)$ is Wigner--Smith time delay matrix. Function defined from this
$$
\kappa(\omega):=\frac{\varphi'(\omega)}{\pi}
$$
called boundary time scale density.
\end{postulate}

\begin{postulate}[A3: Modular Flow Orientation and Time Arrow]
Modular flow $\sigma_t^{\omega_\partial}$ generator $K_{\partial}$ writable as integral of energy--momentum tensor projection along boundary
$$
K_{\partial}=\int_{\partial M}f(x)T_{ab}(x)\chi^a(x)n^b(x)\,d\Sigma_x,
$$
where $\chi^a$ is Killing-like or normalized boundary time translation vector field, $n^b$ is normal, $f$ is positive weight function. Relative entropy $S(\omega_\partial\|\varphi_\partial)$ monotonically non-decreasing along modular flow ``future'' direction, defining time arrow on boundary.
\end{postulate}

\begin{postulate}[A4: Topological Branching and $\mathbb{Z}_2$ Index]
Discriminant subset $D\subset X$ exists in parameter space $X$ such that on $X^\circ=X\setminus D$ can continuously select scattering matrix square root $S^{1/2}$. Any closed loop $\gamma\subset X^\circ$ lifting may return to opposite branch of original point, giving $\mathbb{Z}_2$ index
$$
\nu(\gamma)\in\{\pm1\},
$$
this index equivalent to spectral flow parity of some self-adjoint family or intersection number parity with $D$.
\end{postulate}

---

\section{Variational Completeness and Geometric Boundary}

Prove: Postulate A1's variational completeness requirement introduces GHY term and Brown--York boundary stress--energy tensor on non-null boundary; when corners and null boundaries exist, need additional corner and null boundary terms, completely compressing bulk ``difference'' into readouts on boundary geometry and surface stress.

\begin{theorem}[Variational Completeness on Non-Null Boundary]
Under variation fixing boundary induced metric $h_{ab}$, total action $S[g]=S_{\mathrm{EH}}[g]+S_{\mathrm{GHY}}[g]$ first-order variation is
$$
\delta S[g]=\frac{1}{16\pi G}\int_M(G_{ab}+\Lambda g_{ab})\,\delta g^{ab}\sqrt{-g}\,d^4x,
$$
i.e., all boundary terms completely cancel. Thus under given $h_{ab}$ condition, variational principle well-defined, deriving Einstein equation $G_{ab}+\Lambda g_{ab}=8\pi GT_{ab}$.
\end{theorem}

\textbf{Brown--York Quasilocal Stress--Energy Tensor}: After introducing matter action $S_{\mathrm{matter}}[g,\Phi]$, define on boundary
$$
T_{ab}^{\mathrm{BY}}=-\frac{2}{\sqrt{|h|}}\frac{\delta S_{\mathrm{GHY}}}{\delta h^{ab}}=\frac{1}{8\pi G}(K_{ab}-Kh_{ab}).
$$

Integrating over spatial slice $\Sigma\subset\partial M$ gives Brown--York energy $E_{\mathrm{BY}}(\Sigma)$ interpretable as quasilocal energy relative to reference background, acting as boundary time translation generator in Hamilton formalism.

\textbf{Corners and Null Boundaries}: When timelike and spacelike boundaries intersect forming corners, or null boundaries (like horizons, null hypersurfaces) exist, GHY term alone insufficient to ensure variational completeness. Need introduce corner term $S_{\mathrm{corner}}$ and null boundary term $S_{\mathcal{N}}$.

---

\section{Spectral--Scattering Side Scale Identity and Boundary Time}

Realize Postulate A2; give scale identity sufficient conditions; explain how it uniformly reads boundary phase tiny changes as ``state density difference'' and ``time delay difference.''

\subsection{Birman--Krein Spectral Shift and Scattering Phase}

Under Section 2.2 assumptions, Krein spectral shift function $\xi(\lambda)$ defined making
$$
\operatorname{tr}(f(H)-f(H_0))=\int_{-\infty}^{+\infty}f'(\lambda)\,\xi(\lambda)\,d\lambda
$$
hold for sufficiently many test functions $f$. Birman--Krein formula gives
$$
\det S(\lambda)=\exp(-2\pi i\,\xi(\lambda)).
$$

Define total scattering phase $\Phi(\lambda)=\sum_j\delta_j(\lambda)$, half-phase $\varphi(\lambda)=\frac{1}{2}\Phi(\lambda)$. Then
$$
\frac{\varphi'(\lambda)}{\pi}=\rho_{\mathrm{rel}}(\lambda).
$$

\subsection{Wigner--Smith Time Delay Operator}

Recall Wigner--Smith operator
$$
Q(\lambda)=-iS(\lambda)^\dagger\partial_\lambda S(\lambda).
$$
Taking trace yields
$$
\operatorname{tr}Q(\lambda)=4\varphi'(\lambda).
$$

Combining with previous subsection relation, arranging notation, select unified convention: define time scale density as
$$
\kappa(\omega):=\frac{\varphi'(\omega)}{\pi}.
$$

\begin{theorem}[Scale Identity]
Under standard assumptions (trace-class perturbation, wave operator completeness), normalization selection exists making for almost all $\omega$
$$
\kappa(\omega)=\frac{\varphi'(\omega)}{\pi}=\rho_{\mathrm{rel}}(\omega)=\frac{1}{2\pi}\operatorname{tr}Q(\omega).
$$
\end{theorem}

\textbf{Physical interpretation}: Scale identity shows boundary phase perturbation $\delta\varphi$ frequency change $\partial_\omega\varphi$ readable by three equivalent ways:
$\bullet$ As ``state density difference'' $\rho_{\mathrm{rel}}(\omega)$ per unit frequency;
$\bullet$ As total residence time ``time delay density'' $(2\pi)^{-1}\operatorname{tr}Q(\omega)$;
$\bullet$ As time scale density $\kappa(\omega)$.

This realizes Postulate A2, directly interfacing boundary scattering data with time scale.

---

\section{Modular Flow, Generalized Entropy, Boundary Time Arrow}

Explain how Postulate A3 orients time scale on boundary, aligning with scattering scale identity.

\subsection{Geometric Expression of Modular Hamiltonian}

Let $\mathcal{A}_{\partial}$ be local operator algebra associated with wedge region or causal diamond boundary, $\omega_{\partial}$ vacuum or KMS state on it. In many models, modular Hamiltonian $K_{\partial}$ writable as
$$
K_{\partial}=2\pi\int_{\partial M}\xi^aT_{ab}n^b\,d\Sigma,
$$
where $\xi^a$ is appropriately normalized timelike Killing vector or diamond-like boost vector, $T_{ab}$ is energy--momentum tensor. Modular flow
$$
\sigma_t^{\omega_\partial}(A)=e^{iK_{\partial}t}Ae^{-iK_{\partial}t}
$$
thus interpretable as ``thermal time'' or ``modular time'' along boundary direction.

\subsection{Relative Entropy Monotonicity and Time Arrow}

Let $\omega_\partial,\varphi_\partial$ be two boundary states; corresponding relative entropy $S(\omega_\partial\|\varphi_\partial)$. In local QFT framework, provable under restriction to nested region family, relative entropy monotonically non-increasing with region expansion. Translating to boundary geometry, this monotonicity governs generalized entropy growth along certain ``future directions.''

\begin{proposition}[Modular Time Arrow]
Assume for nested boundary cross-section family $\{\partial M_t\}$ (e.g., sections advancing along null direction)
$$
\frac{d}{dt}S_{\mathrm{gen}}(\partial M_t)\ge0,
$$
where $S_{\mathrm{gen}}$ is generalized entropy; then parameter $t$ selectable as time arrow parameter on boundary. If rescaling $t$ proportionally to align with scattering time scale $\kappa(\omega)$, can simultaneously view on boundary as ``modular time'' and ``scattering time.''
\end{proposition}

---

\section{Topological Branching, Discriminant Boundary, $\mathbb{Z}_2$ Index}

Realize Postulate A4, connecting discriminant boundary in parameter space, spectral flow parity, scattering matrix square root branch structure.

\subsection{Discriminant Boundary and Parameter Space}

Consider parameter space $X$; each point $x\in X$ corresponds to scattering system with scattering matrix $S(\omega;x)$. Discriminant subset $D\subset X$ exists such that if and only if $x\in D$, $S(\omega;x)$ has eigenvalue $-1$ near some energy, degenerate eigenvalues, or other spectral anomalies. Define $X^\circ=X\setminus D$.

On $X^\circ$, spectral anomalies excluded; can select ``principal branch square root'' of scattering matrix $S^{1/2}(\omega;x)$ satisfying
$$
(S^{1/2}(\omega;x))^2=S(\omega;x),\quad S^{1/2}(\omega;x_0)\ \text{given}.
$$

\subsection{$\mathbb{Z}_2$ Index and Spectral Flow Parity}

Take any closed loop $\gamma:[0,1]\to X^\circ$; parallel transporting square root $S^{1/2}$ along $\gamma$ may have overall sign flip
$$
S^{1/2}(\omega;\gamma(1))=\pm S^{1/2}(\omega;\gamma(0)).
$$

Define
$$
\nu(\gamma)=\begin{cases}
+1,&S^{1/2}\ \text{no flip},\\
-1,&S^{1/2}\ \text{flips}.
\end{cases}
$$

\begin{proposition}[$\mathbb{Z}_2$ Index and Spectral Flow Parity]
Under appropriate differentiability and spectral gap assumptions, $\nu(\gamma)$ equals spectral flow parity of some self-adjoint family:
$$
\nu(\gamma)=(-1)^{\mathrm{Sf}(\{A_t\})},
$$
where $\{A_t\}$ is related self-adjoint operator family constructed along $\gamma$; spectral flow $\mathrm{Sf}(\{A_t\})$ records number of eigenvalues crossing zero.
\end{proposition}

\textbf{Geometric interpretation}: Discriminant $D$ as parameter space ``topological boundary'' divides $X^\circ$ into different sectors; $\nu(\gamma)$ records whether closed loop ``passes around'' boundary odd number of times. Thus ``going around once'' continuous deformation compressed by boundary into simple discrete label $\pm1$.

---

\section{Unified Theorem of Boundary Time Geometry}

Glue above geometric, spectral--scattering, modular flow, topological structures into unified ``boundary time geometry'' framework; prove time scale uniqueness result.

\subsection{Definition of Boundary Time Geometry}

\begin{definition}[Boundary Time Geometry]
Boundary time geometry consists of data
$$
\mathfrak{G}_{\partial}=(\partial M,h_{ab},K_{ab};\mathcal{A}_{\partial},\omega_{\partial};S(\omega);D,\nu)
$$
satisfying:

1. $(\partial M,h_{ab},K_{ab})$ makes gravitational and matter action variation well-defined under fixed $h_{ab}$ condition, giving Brown--York boundary stress--energy tensor;

2. $(\mathcal{A}_{\partial},\omega_{\partial})$ gives modular flow $\sigma_t^{\omega_\partial}$ and modular Hamiltonian $K_{\partial}$; defines time arrow through generalized entropy monotonicity;

3. Scattering matrix $S(\omega)$ and spectral shift data satisfy scale identity; time scale density $\kappa(\omega)$ well-defined;

4. Discriminant $D$ and $\nu$ define topological branching $\mathbb{Z}_2$ index.
\end{definition}

Call time parameter $t$ unified scale parameter of this boundary time geometry if simultaneously scales three time structures:
$\bullet$ \textbf{Gravity--geometric side}: $t$ is parameter along boundary time translation vector field $\chi^a$ making Brown--York energy change rate under $t$ consistent with bulk energy flow;
$\bullet$ \textbf{Scattering side}: $t$ related to frequency $\omega$ via bijection $t=t(\omega)$ making $\kappa(\omega)$ interpretable as $dt$ density;
$\bullet$ \textbf{Modular flow side}: $t$ is modular flow parameter making modular Hamiltonian $K_{\partial}$ and Brown--York energy generator differ only by constant factor.

\begin{theorem}[Boundary Time Scale Uniqueness Theorem]
Let $\mathfrak{B}=(\partial M,\mathcal{A}_{\partial},\omega_{\partial},\mathcal{S}_{\partial})$ be physical boundary system satisfying Postulates A1--A4 with technical assumptions:

1. Brown--York boundary energy $E_{\mathrm{BY}}(t)$ change with parameter $t$ writable as boundary energy flow integral;

2. Modular Hamiltonian $K_{\partial}$ compatible with time translation generated by $E_{\mathrm{BY}}$; positive constant $\beta>0$ exists making
   $$
   K_{\partial}=\beta E_{\mathrm{BY}}+\text{constant}.
   $$

3. Scattering matrix $S(\omega)$ frequency dependence reparametrizable as $\omega=\omega(t)$; scale identity maintains form under this reparametrization.

Then unique (in affine rescaling sense) time parameter $t$ exists making:
$\bullet$ Gravity--geometric side boundary time translation;
$\bullet$ Scattering side time delay scale;
$\bullet$ Modular flow side modular time flow
belong to same scale equivalence class.
\end{theorem}

\textbf{Interpretation}: Theorem 8.2 shows in boundary system satisfying Postulates A1--A4, ``geometric time,'' ``scattering time,'' ``modular time'' not independently introduced but scalable using unified parameter $t$. Boundary thus becomes time itself generator: gravity evolution, scattering processes, information flow ``differences'' in bulk all read out on this same time scale.

---

\section{Typical Models and Application Sketches}

Briefly outline how to instantiate this paper's boundary framework in several typical cases.

\subsection{1D Potential Scattering and Finite Interval Boundary}

Consider 1D Schrödinger equation; background free Hamiltonian $H_0=-\partial_x^2$; add potential well and boundary conditions on finite interval; concentrate scattering readouts at interval endpoint phase shift and time delay.

$\bullet$ \textbf{Geometric side}: Interval endpoints as ``artificial boundaries''; need specify endpoint conditions during variation;
$\bullet$ \textbf{Scattering side}: Phase shift and time delay directly given by endpoint reflection coefficient; scale identity explicitly holds;
$\bullet$ \textbf{Topological side}: Through adjusting potential shape and boundary conditions, can form discriminant in parameter space making half-phase square root branch flip correspond to bound state appearance/disappearance.

\subsection{Rindler Wedge, Modular Flow, Unruh Thermal Time}

Consider Rindler wedge region in Minkowski spacetime; region boundary is null plane. Vacuum state modular flow on wedge subalgebra corresponds to translation along Rindler time; modular Hamiltonian proportional to boost generator; energy--momentum tensor flow on boundary gives Unruh temperature.

$\bullet$ \textbf{Geometric side}: Null boundary term and small causal diamond generalized entropy extremality condition give local Einstein equation;
$\bullet$ \textbf{Modular flow side}: Modular time arrow consistent with Rindler time direction;
$\bullet$ \textbf{Scattering side}: For appropriate mirror scattering model, phase and time delay at Rindler frequency characterizable by scale identity.

\subsection{Interface in Topological Matter States and $\mathbb{Z}_2$ Boundary Index}

In 2D or 3D topological insulator/superconductor, material interface carries topologically protected edge/surface states. Organize system parameters (like mass term, local potential, magnetic flux) as parameter space $X$; interface spectrum forms discriminant $D$ in $X$; winding behavior memorized by $\mathbb{Z}_2$ index. This structure precisely instantiates general framework of Section 7.

Above examples show this paper's ``boundary time geometry'' applicable to pure gravity and field theory, extendable to condensed matter and scattering experimental systems.

---

\begin{thebibliography}{99}
\bibitem{ref1} GHY boundary term: Gibbons, Hawking, York (1977).
\bibitem{ref2} Brown--York quasilocal energy: Brown \& York, PRD (1993).
\bibitem{ref3} Birman--Krein formula: standard scattering theory literature.
\bibitem{ref4} Wigner--Smith time delay: various reviews.
\bibitem{ref5} Tomita--Takesaki modular theory: operator algebra literature.
\bibitem{ref6} Spectral flow and index theory: Atiyah--Singer and related works.
\end{thebibliography}

\appendix

\section{Complete Boundary Variation of Gravitational Action}
[Detailed Einstein--Hilbert and GHY term variation calculations...]

\section{Scale Identity and Modular Flow Technical Details}
[Krein spectral shift function details, Wigner--Smith operator trace formula, relative entropy monotonicity...]

\section{$\mathbb{Z}_2$ Index and Discriminant Boundary}
[Cayley transform and spectral flow parity, discriminant as topological boundary...]

\end{document}

