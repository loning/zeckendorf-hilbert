\documentclass[12pt]{article}
\usepackage[utf8]{inputenc}
\usepackage[T1]{fontenc}
\usepackage{amsmath,amssymb,amsthm}
\usepackage{mathrsfs}
\usepackage{geometry}
\usepackage{hyperref}
\usepackage{braket}
\usepackage{graphicx}

\geometry{a4paper, margin=1in}
\hypersetup{colorlinks=true,linkcolor=blue,citecolor=blue,urlcolor=blue}

\theoremstyle{plain}
\newtheorem{theorem}{Theorem}[section]
\newtheorem{lemma}[theorem]{Lemma}
\newtheorem{proposition}[theorem]{Proposition}
\newtheorem{corollary}[theorem]{Corollary}
\newtheorem{axiom}{Axiom}

\theoremstyle{definition}
\newtheorem{definition}[theorem]{Definition}
\newtheorem{example}[theorem]{Example}
\newtheorem{remark}[theorem]{Remark}

\title{Boundary Time Geometry:\\
Unified Theory of Time Scale, Resolution Hierarchy, and Interaction}

\author{Haobo Ma$^1$ \and Wenlin Zhang$^2$\\
\small $^1$Independent Researcher\\
\small $^2$National University of Singapore}

\date{\today}

\begin{document}

\maketitle

\begin{abstract}
Construct unified theoretical system with boundary as ontology and time as geometric scale. Basic assumption: physical reality first manifests as boundary observable algebra and its spectral data; bulk dynamics are extensions determined by boundary data. All observable time scales—scattering time, modular time, geometric time—belong to same equivalence class. Observer's finite resolution geometrically manifests as resolution fiber bundle with connection and curvature.

Mathematically, introduce noncommutative geometric structure of spectral triple with boundary; unify Brown--York boundary stress tensor with AdS/CFT boundary stress tensor, Wigner--Smith time delay matrix with Birman--Krein spectral shift function, Tomita--Takesaki modular flow with thermal time hypothesis within single ``Boundary Time Geometry'' (BTG) framework.

Prove under appropriate matching conditions, exists unique (up to affine rescaling) boundary time generator making scattering time, modular time, geometric time define same time scale equivalence class. All classical ``forces'' manifest as projections of unified boundary connection curvature in different fiber directions, no longer fundamental objects but emergent properties of boundary geometry and resolution structure.

Further establish phenomenal hierarchy emergence theorem on resolution fiber bundle, clarifying how high-resolution quantum scattering and modular time structures degenerate into macroscopic gravity and classical mechanics via completely positive coarse-graining maps.

Finally provide BTG reformulations of black hole thermodynamics, cosmological redshift, mesoscopic transport; propose experimental verification protocols implementable in microwave networks, atomic clock networks, mesoscopic conductors.
\end{abstract}

\noindent\textbf{Keywords:} Boundary Time Geometry; Noncommutative Geometry; Spectral Triple; Wigner--Smith Time Delay; Birman--Krein Spectral Shift; Brown--York Stress Tensor; Thermal Time Hypothesis; Resolution Fiber Bundle; Holographic Boundary--Bulk Correspondence; Renormalization Group

---

\section{Introduction and Historical Context}

In general relativity, Gibbons--Hawking--York boundary term and Brown--York quasilocal stress--energy tensor show that well-defined variation of gravitational action and definition of quasilocal energy--momentum fundamentally depend on boundary geometry and conjugate variables. Variation of boundary three-metric derives surface stress tensor $T^{ab}_{\mathrm{BY}}$ recovering ADM energy in appropriate limits, providing quasilocal energy meaning for black hole thermodynamics.

In AdS/CFT holographic framework, Balasubramanian--Kraus stress tensor views renormalized boundary stress--energy as energy--momentum tensor of dual conformal field theory, further deepening ``boundary-dominated'' perspective.

On scattering theory side, Wigner--Smith time delay matrix
$$
Q(\omega)=-iS(\omega)^\dagger\partial_\omega S(\omega)
$$
characterizes average residence time of wave packets in scattering region; its trace tightly connected to derivative of spectral shift function via Birman--Krein formula: total scattering phase derivative, Wigner--Smith group delay trace, and relative state density are different manifestations of same object.

In algebraic quantum field theory and quantum statistics, Tomita--Takesaki modular theory reveals: given observable algebra and state, naturally exists one-parameter automorphism group $\sigma_t^\omega$ whose parameter $t$ interpretable as ``modular time.'' Connes--Rovelli thermal time hypothesis further proposes physical time understandable as modular flow parameter determined by state--algebra pair; traditional time becomes derived concept.

Noncommutative geometry provides language defining geometry via spectral data: spectral triple $(\mathcal{A},\mathcal{H},D)$ consists of algebra, Hilbert space, Dirac-type operator; for compact Riemannian manifolds, metric structure uniquely reconstructible from Dirac spectrum; this framework provides natural platform unifying ``boundary geometry'' with ``boundary observable algebra.''

This paper's basic stance: glue above three threads—boundary gravity, scattering time, modular time—in unified ``Boundary Time Geometry'' framework, taking boundary as ontology, time as scale, resolution as fiber, constructing unified theoretical system accommodating existing theories while yielding new predictions.

---

\section{Model and Assumptions}

\subsection{Axioms: Boundary Priority, Time Equivalence, Resolution Hierarchy}

\begin{axiom}[Boundary Priority]
Given spacetime region $(M,g)$ with good causal structure, containing topologically well-behaved boundary $\partial M$ (including timelike, spacelike, or null boundaries), fundamental description of physical observables given by boundary observable algebra $\mathcal{A}_\partial$ and state set $\mathcal{S}_\partial$; bulk observables and dynamics viewable as extensions determined by $(\mathcal{A}_\partial,\mathcal{S}_\partial)$ in appropriate sense.
\end{axiom}

\begin{axiom}[Time Scale Equivalence]
Exists time scale equivalence class $[\tau]$ whose elements are time parameters under different constructions: scattering time $\tau_{\mathrm{scatt}}$, modular time $\tau_{\mathrm{mod}}$, geometric time $\tau_{\mathrm{geom}}$. Any two time scales equivalent via affine transformation $\tau^{(2)}=a\tau^{(1)}+b$ ($a>0$) on common domain.
\end{axiom}

\begin{axiom}[Resolution Hierarchy]
For each concrete experimental arrangement or observer, exists resolution parameter $\Lambda$ (understandable as UV cutoff, coarse-graining stage, or RG scale) such that at different $\Lambda$, same boundary geometric data projects via completely positive map to different coarse-grained effective algebras $\mathcal{A}_\Lambda\subseteq\mathcal{A}_\partial$.
\end{axiom}

\subsection{Boundary Spectral Data}

\begin{definition}[Boundary Spectral Triple]
Boundary spectral triple is tuple $(\mathcal{A}_\partial,\mathcal{H}_\partial,D_\partial)$ where:

1. $\mathcal{A}_\partial$ is dense $\ast$-algebra defined on boundary (typically $C^\infty(\partial M)$ or noncommutative generalization);

2. $\mathcal{H}_\partial$ is $\mathbb{Z}_2$-graded Hilbert space carrying $\ast$-representation of $\mathcal{A}_\partial$;

3. $D_\partial$ is self-adjoint, first-order elliptic operator (Dirac-type) with compact resolvent, satisfying commutator $[D_\partial,a]$ bounded for any $a\in\mathcal{A}_\partial$.
\end{definition}

This is boundary version of Connes (even) spectral triple.

\begin{theorem}[Spectral Reconstruction of Boundary Metric]
If $\partial M$ is compact spin Riemannian manifold, triple
$$
(\mathcal{A}_\partial,\mathcal{H}_\partial,D_\partial)=(C^\infty(\partial M),L^2(S_\partial),D_\partial)
$$
determines unique Riemannian metric $h_{ab}$ such that Connes distance
$$
d(x,y)=\sup\{|a(x)-a(y)|:a\in C^\infty(\partial M),\ |[D_\partial,a]|\le 1\}
$$
equals geodesic distance on $(\partial M,h_{ab})$.
\end{theorem}

Thus in BTG, boundary ``metric'' need not be given a priori but defined by spectral structure of $D_\partial$; this provides natural channel embedding time scale into Dirac spectrum.

\subsection{Boundary Stress Tensor and Quasilocal Hamiltonian}

In four-dimensional general relativity, after introducing GHY boundary term, variation of action with respect to boundary three-metric $h_{ab}$ defines Brown--York surface stress tensor
$$
T^{ab}_{\mathrm{BY}}:=\frac{2}{\sqrt{-h}}\frac{\delta S_{\mathrm{grav}}}{\delta h_{ab}}.
$$

Its zero component's appropriate projection gives quasilocal energy density; integrated quasilocal energy equals Hamiltonian generating unit proper time translation on boundary.

In AdS scenario, holographic renormalization process derives renormalized boundary stress tensor $T^{ab}_{\mathrm{ren}}$, interpretable as dual CFT expectation value $\langle T^{ab}\rangle$. These results show: boundary stress tensor naturally carries Hamiltonian generating boundary ``time flow.''

---

\section{Main Results (Theorems and Alignments)}

\subsection{Unified Time Scales on Boundary}

Define time scales from scattering, modular flow, geometric perspectives respectively:

\textbf{1. Scattering time $\tau_{\mathrm{scatt}}$}

Consider finite-channel scattering matrix $S(\omega)$ at fixed energy; define Wigner--Smith time delay matrix
$$
Q(\omega)=-iS(\omega)^\dagger\partial_\omega S(\omega).
$$

Its trace $\tau_{\mathrm{W}}(\omega):=\operatorname{tr}Q(\omega)$ gives total group delay. In Birman--Krein framework, spectral shift function $\xi(\omega)$ satisfies
$$
\det S(\omega)=\exp(-2\pi i\xi(\omega)),
$$
thus
$$
\xi'(\omega)=\frac{1}{2\pi}\operatorname{tr}Q(\omega).
$$

Given reference energy $\omega_0$ and window $I\subset\mathbb{R}$, define scattering time scale
$$
\tau_{\mathrm{scatt}}(\omega):=\int_{\omega_0}^\omega\xi'(\tilde{\omega})\,d\tilde{\omega}=\xi(\omega)-\xi(\omega_0).
$$

\textbf{2. Modular time $\tau_{\mathrm{mod}}$}

For boundary observable algebra $\mathcal{A}_\partial$ and state $\omega$, assuming separating--cyclic vector exists making Tomita--Takesaki modular data $(J,\Delta_\omega)$ well-defined; modular group
$$
\sigma_t^\omega(A):=\Delta_\omega^{it}A\Delta_\omega^{-it}
$$
defines one-parameter automorphism group. Thermal time hypothesis suggests appropriate physical time parameter $\tau_{\mathrm{mod}}$ differs from modular parameter $t$ only by constant factor; $\sigma_t^\omega$ plays ``time evolution'' role in equilibrium states.

\textbf{3. Geometric time $\tau_{\mathrm{geom}}$}

In general relativity with boundary, choose unit timelike vector field $u^a$ on boundary with corresponding Killing or approximate Killing generator $\xi^a$; Brown--York Hamiltonian writable as
$$
H_\partial[\xi]=\int_{\Sigma\cap\partial M}\sqrt{\sigma}\,u_aT_{\mathrm{BY}}^{ab}\xi_b\,d^{d-2}x,
$$
where $\sigma$ is induced metric on cross-section. Canonical evolution parameter generated by $H_\partial$ defines geometric time scale $\tau_{\mathrm{geom}}$.

In BTG framework, we don't presuppose three time scales mutually independent, but unify via following theorem:

\begin{theorem}[Boundary Time Scale Equivalence Theorem]
Let $\partial M$ be ``benign boundary'' satisfying:

1. Exists boundary spectral triple $(\mathcal{A}_\partial,\mathcal{H}_\partial,D_\partial)$ and Brown--York boundary stress tensor $T_{\mathrm{BY}}^{ab}$;

2. Boundary admits scattering process with scattering matrix $S(\omega)$ continuously differentiable in energy $\omega$, satisfying Hilbert--Schmidt locality and BK conditions on energy window $I$;

3. For same boundary region exists von Neumann algebra $\mathcal{A}_\partial''$ and KMS state $\omega$ whose modular group $\sigma_t^\omega$ physically represents thermal equilibrium time evolution;

4. Brown--York Hamiltonian $H_\partial[\xi]$ generated boundary time translation induces automorphism group $\alpha_\tau$ on observable algebra ``comparable'' to scattering evolution and modular flow in same energy--frequency window, i.e., exists common invariant subalgebra $\mathcal{A}_{\mathrm{com}}\subset\mathcal{A}_\partial$.

Then exists unique time scale equivalence class $[\tau]$, plus three positive constants $a_{\mathrm{scatt}},a_{\mathrm{mod}},a_{\mathrm{geom}}>0$ and three translation constants $b_{\mathrm{scatt}},b_{\mathrm{mod}},b_{\mathrm{geom}}$, such that on common domain:
$$
\tau_{\mathrm{scatt}}=a_{\mathrm{scatt}}\tau+b_{\mathrm{scatt}},\quad
\tau_{\mathrm{mod}}=a_{\mathrm{mod}}\tau+b_{\mathrm{mod}},\quad
\tau_{\mathrm{geom}}=a_{\mathrm{geom}}\tau+b_{\mathrm{geom}}.
$$

In other words, scattering time, modular time, geometric time in BTG only represent different normalizations and zero-point choices of same time scale.
\end{theorem}

Rigorous proof given in Proofs section and appendices, core being:
$\bullet$ Use BK--Wigner--Smith identity to express scattering time as integral of relative spectral density;
$\bullet$ Via thermal time hypothesis and boundary KMS state, align modular parameter with relative spectral density;
$\bullet$ Via Brown--York Hamiltonian and boundary stress tensor's spectral representation, associate geometric time flow generator with same spectral measure.

\subsection{No Fundamental Forces: Curvature of Unified Boundary Connection}

\begin{definition}[Boundary Total Bundle and Unified Connection]

1. On geometric--gauge--resolution three-layer degrees of freedom, define boundary total bundle $\pi:\mathcal{B}\to\partial M$ with fiber
$$
\mathcal{F}=\mathcal{F}_{\mathrm{int}}\times\mathcal{F}_{\mathrm{res}},
$$
carrying internal gauge degrees of freedom and resolution scale degrees of freedom respectively.

2. Structure group taken as
$$
G_{\mathrm{tot}}=\mathrm{SO}(1,3)^\uparrow\times G_{\mathrm{YM}}\times G_{\mathrm{res}},
$$
where $G_{\mathrm{res}}$ is scale group equivalent to renormalization group or coarse-graining transformations.

3. Unified boundary connection defined as
$$
\Omega_\partial=\omega_{\mathrm{LC}}\oplus A_{\mathrm{YM}}\oplus\Gamma_{\mathrm{res}},
$$
corresponding to Levi--Civita spin connection, Yang--Mills connection, resolution connection; corresponding curvature
$$
\mathcal{R}_\partial=R_\partial\oplus F_\partial\oplus\mathcal{R}_{\mathrm{res}}.
$$
\end{definition}

\begin{theorem}[No Fundamental Forces Theorem]
Under BTG framework, consider charged--colored test particle or effective mass trajectory lift $\gamma(\tau)\subset\mathcal{B}$ on boundary; its projection to $\partial M$ is $x^\mu(\tau)$, intrinsic degrees of freedom via representation $\rho:G_{\mathrm{YM}}\to\mathrm{Aut}(\mathcal{F}_{\mathrm{int}})$ and $G_{\mathrm{res}}$ one-dimensional representation. Then following holds:

1. ``Force-free motion'' of trajectory $\gamma(\tau)$ is parallel transport for unified connection $\Omega_\partial$: $D_\tau\dot{\gamma}=0$.

2. Its base trajectory $x^\mu(\tau)$ satisfies equation writable as
$$
m\frac{D^2x^\mu}{D\tau^2}=qF^\mu{}_\nu\dot{x}^\nu+f_{\mathrm{res}}^\mu,
$$
where $F^\mu{}_\nu$ is Yang--Mills curvature projection under representation $\rho$, $f_{\mathrm{res}}^\mu$ is resolution curvature $\mathcal{R}_{\mathrm{res}}$ projection in appropriate effective action.

3. Classical gravitational ``force'' corresponds to geodesic deviation effect of $R_\partial$; thus all ``forces'' understandable as different projections and representations of unified boundary connection curvature, no longer fundamental objects.
\end{theorem}

Therefore in BTG theory, ``forces'' not independent axiomatic entities but emergent manifestations of boundary time geometry; all interactions—including gravity, gauge interactions, resolution-driven ``entropic forces''—jointly arise from unified connection curvature.

\subsection{Resolution Hierarchy and Emergent Phenomena}

\begin{definition}[Resolution Fiber Bundle and Coarse-Graining Maps]

1. On boundary total bundle define ``resolution fiber bundle'' $P_{\mathrm{res}}=(\mathcal{B},\partial M,G_{\mathrm{res}},\pi_{\mathrm{res}})$ with fiber coordinate understandable as resolution or renormalization scale $\Lambda$.

2. Each $\Lambda$ induces completely positive, unit-preserving map
$$
\Phi_\Lambda:\mathcal{A}_\partial\to\mathcal{A}_\Lambda\subseteq\mathcal{A}_\partial,
$$
viewable as coarse-graining from high-resolution boundary algebra to low-resolution effective algebra.
\end{definition}

\begin{theorem}[Phenomenal Hierarchy Emergence Theorem]
When satisfying:

1. $\{\Phi_\Lambda\}_\Lambda$ forms normal $\ast$-homomorphism family of semigroup (or group) satisfying $\Phi_{\Lambda_1}\circ\Phi_{\Lambda_2}=\Phi_{\Lambda_1\circ\Lambda_2}$;

2. For any local observable $A\in\mathcal{A}_\partial$, its image $\Phi_\Lambda(A)$ in $\Lambda\to 0$ limit (coarsest) converges to classical function or operator $A_{\mathrm{cl}}$;

3. Unified connection $\Omega_\partial$'s connection form $\Gamma_{\mathrm{res}}$ in resolution direction satisfies Callan--Symanzik-like equation: parallel transport along $\Lambda$ flow equivalent to renormalization group flow.

Then:

1. In high-resolution limit, description of $\mathcal{A}_\partial$ is full quantum scattering and modular time structure;

2. At medium resolution, curvature expectation values in coarse-grained algebra $\mathcal{A}_\Lambda$ manifest as gauge forces, entropic forces, topological effects;

3. In $\Lambda\to 0$ macroscopic limit, geometric curvature and Brown--York tensor dominate, dynamics degenerating to classical gravity and thermodynamics; all ``forces'' effectively viewable as geometric effects of metric and effective potentials.
\end{theorem}

---

\section{Proofs}

This section provides proof skeletons of main theorems; details and technical lemmas in appendices.

\subsection{Preliminaries: Scattering, Time Delay, Spectral Shift}

Let $H_0$ and $H=H_0+V$ be self-adjoint operators on some Hilbert space satisfying wave operator existence conditions of general scattering theory. Birman--Krein theory provides spectral shift function $\xi(\omega)$ giving trace formula for smooth functions $f$ of $H,H_0$:
$$
\operatorname{Tr}(f(H)-f(H_0))=\int f'(\omega)\,\xi(\omega)\,d\omega.
$$

Under suitable conditions, scattering matrix $S(\omega)$ satisfies
$$
\det S(\omega)=\exp(-2\pi i\xi(\omega)).
$$

For Theorem 2, only need local BK formula on energy window $I$ and differentiability of Wigner--Smith matrix. Let
$$
Q(\omega)=-iS(\omega)^\dagger\partial_\omega S(\omega),
$$
then $\xi'(\omega)=(2\pi)^{-1}\operatorname{tr}Q(\omega)$, thus scattering time scale
$$
\tau_{\mathrm{scatt}}(\omega)=\xi(\omega)-\xi(\omega_0)
$$
well-defined on $I$.

\subsection{Proof Sketch of Theorem 2 (Time Scale Equivalence)}

\textbf{Step 1: Unified Spectral Measure}

On energy window $I$, define spectral measure via BK:
$$
\mu_{\mathrm{scatt}}(d\omega):=\frac{1}{2\pi}\operatorname{tr}Q(\omega)\,d\omega.
$$

On other hand, KMS state $\omega$ on boundary von Neumann algebra $\mathcal{A}_\partial''$ induces modular operator $\Delta_\omega$ whose spectral measure $\mu_{\mathrm{mod}}$ determines modular group $\sigma_t^\omega$ generator $K_\omega:=-\log\Delta_\omega$. Thermal time hypothesis requires constant $c_{\mathrm{mod}}>0$ exists making physical Hamiltonian $H_{\mathrm{mod}}=c_{\mathrm{mod}}K_\omega$.

On geometric end, Brown--York Hamiltonian writable as functional on Dirac or Laplace operator spectrum: under appropriate boundary conditions, its expectation value on energy eigenstate $|E\rangle$ gives spectral function $h_{\mathrm{geom}}(E)$, introducing measure $\mu_{\mathrm{geom}}(dE)=h_{\mathrm{geom}}(E)\,dE$. In AdS/CFT context, this measure equivalent to boundary CFT energy--momentum tensor spectral measure.

\textbf{Step 2: Matching Conditions and Measure Equivalence}

Assume scattering process, modular flow, geometric time translation act on common decomposable subalgebra $\mathcal{A}_{\mathrm{com}}$; within energy window $I$, three dynamics' spectral decompositions representable in same Hilbert space; this is ``comparability'' condition in theorem statement.

Under this condition, can prove:

1. Exists family of monotonic differentiable energy rescaling functions making $\mu_{\mathrm{scatt}}$, $\mu_{\mathrm{mod}}$, $\mu_{\mathrm{geom}}$ mutually absolutely continuous on $I$ with constant Radon--Nikodym derivatives;

2. This ensures three time generators equivalent on $L^2(I,\mu)$, differing only by constant factors and additive constants.

\textbf{Step 3: Uniqueness}

If another time scale $\tilde{\tau}$ affinely equivalent to all three above, then $\tilde{\tau}$ also affinely equivalent to $\tau$; thus equivalence class $[\tau]$ unique.

Complete proof involves fine control of spectral decomposition, KMS conditions, Brown--York Hamiltonian spectral representation; see Appendix A.

\subsection{Proof Sketch of Theorem 3 (No Fundamental Forces)}

Under unified connection $\Omega_\partial$, consider curve $\gamma(\tau)$ on total bundle $\mathcal{B}$. Its covariant derivative
$$
D_\tau=\frac{d}{d\tau}+\Omega_\partial(\dot{\gamma}).
$$

Define ``free motion'' as $D_\tau\dot{\gamma}=0$.

Expanding this condition in different fiber direction components yields:

1. In $\mathrm{SO}(1,3)$ part: standard geodesic equation;
2. In $G_{\mathrm{YM}}$ part: Wong-equation-like gauge force term: particle parallel transport in internal space induces $F_{\mu\nu}\dot{x}^\nu$ term on base trajectory;
3. In $G_{\mathrm{res}}$ part: resolution connection $\Gamma_{\mathrm{res}}$ curvature via effective action's scale dependence gives ``entropic force'' or ``information force'', specific form depending on chosen effective free energy functional.

Thus any seemingly ``forced'' motion viewable as parallel transport under some unified connection, we simply ignore certain fiber directions in projection. Theorem 3 content merely formalizes this geometric fact.

---

\section{Model Applications}

\subsection{Black Hole Thermodynamics in BTG}

For spacetime with event horizon, treat horizon as special null boundary; introduce null Brown--York stress tensor and corresponding quasilocal energy, rewriting black hole thermodynamics four laws in pure boundary language.

\begin{proposition}[Boundary Restatement of Black Hole Thermodynamics]

1. Hawking temperature $T_H=\kappa/(2\pi)$ comes from modular flow period $2\pi/\kappa$ on horizon, where $\kappa$ is surface gravity;

2. Bekenstein--Hawking entropy $S_{\mathrm{BH}}=A/(4G)$ interpretable as von Neumann entropy of horizon boundary algebra or entropy density on type factor;

3. Hawking radiation ``purity problem'' formulable in BTG as Markov property stability problem between horizon and infinity boundary algebras: if boundary relative entropy slice-independent satisfying appropriate quantum focusing conditions, overall evolution can preserve pure states.
\end{proposition}

In BTG language, black hole thermodynamics no longer mixture of bulk singularity and horizon structure, but completely described by boundary time geometry and modular time scale.

\subsection{Cosmological Redshift as Boundary Time Rescaling}

In FRW universe, standard redshift formula $1+z=a(t_0)/a(t_e)$ rewritable in BTG as:

\begin{proposition}[Boundary Interpretation of Cosmological Redshift]
Choose cosmological boundary as conformal infinity or comoving observer family worldtube boundary, with time scale defined by boundary time scale $\tau_\partial$; exists affine transformation making
$$
1+z=\tau_\partial(t_0)/\tau_\partial(t_e).
$$
\end{proposition}

This shows redshift viewable as overall rescaling of boundary time scale, not bulk ``proper time'' difference; BTG directly connects redshift to boundary spectral data evolution.

\subsection{Mesoscopic Transport and Friedel--Wigner Consistency}

In mesoscopic conductors or AB rings, Wigner--Smith time delay matrix and Friedel sum rule provide connections between local density of states, phase shift, transport properties. In BTG, these results interpretable as boundary spectral triple projections at finite resolution:

$\bullet$ Phase shift derivative $\partial_\omega\phi(\omega)$ ratio to local density of states directly gives scattering time scale;
$\bullet$ Via Theorem 2, this scale equivalent to modular and geometric time, making mesoscopic transport experiments direct verification platform for BTG time scale equivalence.

---

\section{Engineering Proposals}

\subsection{Microwave Scattering Networks as Discrete Boundary Models}

Construct multi-port microwave network viewing as discretized boundary $\partial M$ model:

1. Measure multi-port scattering matrix $S(\omega)$ via vector network analyzer; numerically construct Wigner--Smith matrix $Q(\omega)$ and spectral shift function $\xi(\omega)$, define scattering time scale $\tau_{\mathrm{scatt}}$.

2. Introduce tunable ``geometric parameters'' at network nodes (e.g., electrical length, lossy elements); reconstruct $\tau_{\mathrm{geom}}$ via network Lagrangian or effective RLC model inversion.

3. Place network in controlled noise environment; define statistical steady state and construct equivalent modular flow; measure $\tau_{\mathrm{mod}}$ proxy quantities (e.g., correlation function decay parameters).

BTG prediction: Within energy window and resolution conditions satisfying Theorem 2 assumptions, ratios of three time scales should be constant; deviations attributable to resolution connection $\Gamma_{\mathrm{res}}$ curvature and experimental non-idealities.

\subsection{Atomic Clock Networks and Gravitational Redshift}

Deploy atomic clock network at different gravitational potentials; use two-way time transfer protocol to measure frequency ratio $\nu_2/\nu_1$. In BTG language, this frequency ratio should equal corresponding boundary modular Hamiltonian eigenvalue ratio:
$$
\nu_2/\nu_1=\lambda_{\mathrm{mod}}(x_2)/\lambda_{\mathrm{mod}}(x_1).
$$

Experimental tasks:

1. Via precise gravitational redshift measurement, determine relation between macroscopic geometric time scale $\tau_{\mathrm{geom}}$ and frequency ratio;

2. Using quantum optics and quantum thermodynamics techniques, construct effective KMS state and modular flow for atomic clock system, characterize $\tau_{\mathrm{mod}}$;

3. Test whether they are affinely equivalent; estimate deviations from resolution curvature.

\subsection{Mesoscopic Conductors and Phase--Delay Fingerprints}

In quantum dot or AB ring systems, via phase-sensitive transport and local density of states measurements, construct:

$\bullet$ Phase shift $\phi(\omega)$;
$\bullet$ Wigner--Smith matrix $Q(\omega)$;
$\bullet$ Local density of states $\rho(\omega)$.

BTG prediction: At sufficiently low temperature and within coherence length, frequency-dependent
$$
\partial_\omega\phi(\omega)/\pi,\quad(2\pi)^{-1}\operatorname{tr}Q(\omega)
$$
and relative state density should agree, belonging to same equivalence class as geometric time scale experimental proxy (e.g., interference fringe drift with flux).

---

\section{Discussion (Risks, Boundaries, Past Work)}

1. \textbf{Domain restrictions}: Theorem 2 time scale equivalence only holds on energy window $I$ and corresponding boundary region; equivalence relations between different windows or boundary patches may require new rescaling. BTG only claims ``local equivalence classes,'' not global unique time.

2. \textbf{Relation to AdS/CFT and holographic RG}: BTG structurally similar to holographic principle: taking boundary stress tensor and spectral triple as fundamental objects, implementing scale flow via RG/resolution connection. But BTG not dependent on AdS geometry or strict CFT, allowing more general spacetime and algebraic structures; complementary to recent research on stress tensor flow and holographic RG.

3. \textbf{Relation to thermal time hypothesis and quantum thermodynamics}: BTG confines thermal time hypothesis to boundary algebra, aligning with scattering and geometric time via Theorem 2. Consistent with recent work on thermal time as ``unsharp observable,'' also more geometric realization of ``time from thermal equilibrium.''

4. \textbf{Potential risks and open questions}:
$\bullet$ In highly non-equilibrium states, modular flow no longer simply corresponds to thermal time; how to maintain equivalence with scattering and geometric time remains to be studied;
$\bullet$ For non-compact, singular boundaries (e.g., cusps, topological defects) need more refined spectral geometry and index theorem tools;
$\bullet$ Relation between resolution fiber bundle characteristic classes and physical observables not fully clarified.

---

\section{Conclusion}

Boundary Time Geometry theory based on three core insights:

1. \textbf{Boundary ontology}: Physical reality first given by boundary observable algebra and spectral data; bulk only one extension;

2. \textbf{Time scale unification}: Scattering time, modular time, geometric time belong to same time scale equivalence class, differing only in normalization and zero-point choice;

3. \textbf{Interaction emergence}: All ``forces'' are projections of unified boundary connection curvature; resolution structure makes quantum scattering and modular flow emerge as macroscopic gravity and classical mechanics via coarse-graining.

In this framework, gravity, quantum, information no longer three isolated levels, but manifestations of same boundary time geometry at different resolutions and projections. BTG not only provides new interpretations for black hole thermodynamics and cosmological redshift, but also gives series of testable predictions for mesoscopic transport and precision timing experiments.

---

\begin{thebibliography}{99}
\bibitem{ref1} A. Connes, \textit{Noncommutative Geometry}, Academic Press (1994).
\bibitem{ref2} J. D. Brown, J. W. York, ``Quasilocal energy and conserved charges derived from the gravitational action,'' \textit{Phys. Rev. D} \textbf{47}, 1407 (1993).
\bibitem{ref3} V. Balasubramanian, P. Kraus, ``A stress tensor for Anti-de Sitter gravity,'' \textit{Commun. Math. Phys.} \textbf{208}, 413 (1999).
\bibitem{ref4} Birman--Krein formula and Wigner--Smith time delay literature.
\bibitem{ref5} T. T. Paetz, \textit{An Analysis of the ``Thermal Time Concept'' of Connes and Rovelli}, diploma thesis (2010).
\bibitem{ref6} Spectral triple, Wikipedia and related literature.
\bibitem{ref7} Holographic RG and stress tensor flow references.
\end{thebibliography}

\appendix

\section{Spectral Measures and Time-Delay Relations}
[Detailed derivation of spectral measure construction and BK--Wigner--Smith relations in Theorem 2...]

\section{Geometry of Unified Boundary Connection}
[Geometric proof of Theorem 3...]

\section{Resolution Bundle, RG Flow and Emergence}
[Mathematical structure of resolution fiber bundle and RG flow in Theorem 4...]

\end{document}

