\documentclass[12pt]{article}

% Essential packages
\usepackage[utf8]{inputenc}
\usepackage[T1]{fontenc}
\usepackage{amsmath,amssymb,amsthm}
\usepackage{mathrsfs}
\usepackage{geometry}
\usepackage{hyperref}
\usepackage{braket}
\usepackage{graphicx}

% Geometry settings
\geometry{a4paper, margin=1in}

% Hyperref settings
\hypersetup{
    colorlinks=true,
    linkcolor=blue,
    citecolor=blue,
    urlcolor=blue
}

% Theorem environments
\theoremstyle{plain}
\newtheorem{theorem}{Theorem}[section]
\newtheorem{lemma}[theorem]{Lemma}
\newtheorem{proposition}[theorem]{Proposition}
\newtheorem{corollary}[theorem]{Corollary}

\theoremstyle{definition}
\newtheorem{definition}[theorem]{Definition}
\newtheorem{example}[theorem]{Example}
\newtheorem{remark}[theorem]{Remark}
\newtheorem{axiom}[theorem]{Axiom}

% Title information
\title{Boundary Language as a Unified Physical Framework:\\
From Scattering Phase, GHY Boundary Term to Modular Flow and Generalized Entropy as a Single Structure}

\author{Haobo Ma$^1$ \and Wenlin Zhang$^2$\\
\small $^1$Independent Researcher\\
\small $^2$National University of Singapore}

\date{\today}

\begin{document}

\maketitle

\begin{abstract}
This paper proposes and systematizes the concept of ``boundary language,'' rewriting physical theories as algebraic--geometric structures about ``what is allowed to be exchanged'' on causal cut surfaces, with bulk field theory being merely one realization of this structure. Taking the boundary observable algebra and boundary state as fundamental objects, we unify three seemingly independent theoretical paradigms into the same framework: (1) in scattering theory, the spectral shift function, total scattering phase, and Wigner--Smith group delay; (2) in general relativity, the Gibbons--Hawking--York (GHY) boundary term and Brown--York quasilocal energy; (3) in operator algebras, the Tomita--Takesaki modular flow and relative entropy monotonicity. The core idea is: time is not a parameter of flow within the bulk that is given a priori, but rather a unified translation parameter generated by ``what is allowed in terms of flux balance and information monotonicity'' in the ``boundary language;'' all observable delays, energies, and generalized entropy variations are different projections of the same boundary structure.

Mathematically, we formalize this framework as three ``boundary language axioms'': (A1) Conservation and Flux Axiom, viewing the boundary as a balancing interface for energy, charge, and information flux; (A2) Time Generation Axiom, viewing the one-parameter ${}^\ast$-automorphism group defined on the boundary and its generator as the source of time scale; (A3) Monotonicity and Consistency Axiom, represented by relative entropy monotonicity and its geometric forms (quantum focussing, quantum null energy condition, etc.), excluding supercausality and negative entropy transport.

In the scattering realization, we prove: the boundary language satisfying A1--A3 necessarily induces the scale identity on a well-posed short-range scattering system
$$
\kappa(\omega)=\frac{\varphi'(\omega)}{\pi}=\rho_{\mathrm{rel}}(\omega)=\frac{1}{2\pi}\operatorname{tr}Q(\omega),
$$
where $\varphi(\omega)$ is the total scattering half-phase, $\rho_{\mathrm{rel}}(\omega)$ is the relative state density, and $Q(\omega)$ is the Wigner--Smith group delay operator. This identity unifies the phase gradient, spectral shift density, and group delay trace as a single boundary object called ``time scale.''

In the gravity realization, we show: the GHY boundary term and the Brown--York quasilocal energy positivity are necessary conditions for the boundary language A1--A2 on the geometric side; thus ADM time, proper time, and Killing time are restated as translations generated by the boundary Hamiltonian. In the operator algebra realization, we provide the canonical model of boundary language through Tomita--Takesaki modular theory and implement A3 through relative entropy monotonicity, thereby characterizing the ``time arrow'' as the monotonic evolution of relative entropy with modular time under a natural class of conditions.

Finally, through three model classes---one-dimensional potential scattering, static black hole exterior regions, and Rindler wedges---we demonstrate how the boundary language produces experimentally observable time delays, quasilocal energy, and Unruh temperature, and give several testable spectral--geometric--information theoretical predictions. Detailed appendices provide proofs of key propositions such as the scattering--spectral shift--group delay scale identity, the variational completeness of the GHY term, and relative entropy monotonicity, and introduce the ``error geometry'' framework of finite-order Euler--Maclaurin and Poisson discipline to ensure a controllable mapping from boundary readings to experimental data.
\end{abstract}

\noindent\textbf{Keywords:} Boundary Language; Scattering Phase; Wigner--Smith Group Delay; GHY Boundary Term; Brown--York Quasilocal Energy; Tomita--Takesaki Modular Flow; Relative Entropy; Time Scale

---

\section{Introduction}

\subsection{Research Motivation and Overall Structure}

Traditional field theory and gravitational theory often start from the bulk: given a manifold $(M,g)$ with a metric, a bulk action $S[\Phi,g]$ and its variational equations, then supplemented by boundary conditions. However, in three classes of seemingly unrelated theories, the boundary has long played the role of ``truly observable'':

\begin{enumerate}
\item In scattering and spectral theory, the total scattering phase $\varphi(\omega)$, the Birman--Kreĭn spectral shift function $\xi(\omega)$, and the Wigner--Smith group delay operator $Q(\omega)=-iS^\dagger\partial_\omega S$ are completely defined by the ``in/out asymptotic infinite boundaries.''

\item In general relativity, the variation of the Einstein--Hilbert bulk action requires introducing the Gibbons--Hawking--York (GHY) boundary term to achieve variational completeness, while the Brown--York quasilocal energy and quasilocal stress tensor are strictly boundary data.

\item In operator algebras and algebraic quantum field theory, the Tomita--Takesaki modular flow $\sigma_t^\omega$ is completely determined by the algebra--state pair $(\mathcal M,\omega)$, with its parameter $t$ viewed as ``intrinsic time''; the monotonicity of relative entropy and generalized entropy inequalities depend only on the inclusion relationships between boundary accessible algebras.
\end{enumerate}

These facts suggest: \textbf{the boundary itself, rather than the bulk, is the natural stage for unified physical structure}. Based on this, this paper proposes the concept of ``boundary language,'' defining physical theory as a triple structure on a certain causal cut surface $\Sigma \subset M$
$$
\mathfrak L_\Sigma=(\mathcal A_\partial,\omega,\mathcal F),
$$
where $\mathcal A_\partial$ is the boundary observable algebra, $\omega$ is the boundary state, and $\mathcal F$ is a family of flux functionals used to characterize the exchange of energy, charge, and information across $\Sigma$. Bulk field theory, geometry, and scattering constructions are merely different ways of realizing this triple.

\subsection{Core Point: Time as Boundary Translation}

The core claim of this paper can be summarized as:

\begin{itemize}
\item Given a causal cut surface $\Sigma$, physical theory first gives ``allowed exchanges across $\Sigma$''---this is the boundary language;
\item Time is not a continuous parameter given to the bulk a priori, but is derived from a one-parameter ${}^\ast$-automorphism group $\{\alpha_t\}_{t\in\mathbb R}$ internal to the boundary language and its generator;
\item When conservation and monotonicity conditions are satisfied, this time belongs to the same ``time scale equivalence class'' as the frequency derivative of the scattering phase, the gravitational boundary Hamiltonian, and the modular flow parameter.
\end{itemize}

More specifically, we will prove the scale identity in the scattering realization
$$
\kappa(\omega)=\frac{\varphi'(\omega)}{\pi}=\rho_{\mathrm{rel}}(\omega)=\frac{1}{2\pi}\operatorname{tr}Q(\omega),
$$
and show in gravity and modular flow realizations: ADM/proper time and modular time can be mapped into the same equivalence class through this scale.

\subsection{Article Structure}

The structure of the full text is as follows: Section 2 gives the strict definition of boundary language and the three axioms. Section 3 realizes the boundary language in short-range scattering theory and proves the scale identity. Section 4 shows how the GHY term and Brown--York quasilocal energy implement A1--A2 in gravitational theory with boundaries. Section 5 discusses modular flow and relative entropy monotonicity, demonstrating the implementation of A3 on the algebraic side. Section 6 integrates the three-terminal structure, introducing the unified time scale equivalence class. Section 7 gives several typical models and testable predictions. Appendices A--D provide detailed proofs of key propositions and the error geometry framework.

---

\section{Boundary Language and Three Axioms}

\subsection{Causal Cut Surface and Boundary Algebra}

Let $(M,g)$ be a Lorentzian manifold with good causal structure, and $\Sigma\subset M$ be a codimension-one submanifold dividing spacetime into ``inside'' $M_{\mathrm{in}}$ and ``outside'' $M_{\mathrm{out}}$. We call $\Sigma$ a causal cut surface.

\begin{definition}[Boundary Observable Algebra]
The boundary observable algebra associated with the causal cut surface $\Sigma$ is a $C^\ast$-algebra $\mathcal A_\partial$, whose elements correspond to observables that can be determined solely through readings on $\Sigma$, such as:
\begin{itemize}
\item In/out field operators and functions of the S-matrix on scattering channels;
\item Boundary-induced metric, extrinsic curvature, and geometric quantities composed of them;
\item Local algebras on wedge regions or Cauchy surface boundaries in algebraic quantum field theory.
\end{itemize}
\end{definition}

\begin{definition}[Boundary State]
A boundary state is a positive normalized linear functional $\omega:\mathcal A_\partial\to\mathbb C$ on $\mathcal A_\partial$, representing the expectation value of boundary observables under given physical configurations (bulk fields, metric, external sources, etc.).
\end{definition}

\subsection{Boundary Language Triple}

\begin{definition}[Boundary Language]
A boundary language is the triple
$$
\mathfrak L_\Sigma=(\mathcal A_\partial,\omega,\mathcal F),
$$
where $\mathcal A_\partial,\omega$ are as described above, and $\mathcal F\subset\mathcal A_\partial^\ast$ is a family of real-valued linear functionals called flux functional family, used to characterize boundary readings of exchangeable quantities such as energy, charge, entropy, or information.
\end{definition}

Typical examples include:
\begin{itemize}
\item In scattering theory, functionals related to probability current, energy flow, or time delay;
\item In gravitational theory, functionals related to quasilocal energy, momentum, angular momentum, and generalized entropy;
\item In algebraic quantum theory, functionals related to modular Hamiltonian, relative entropy, and information flow.
\end{itemize}

\subsection{Axiom A1: Conservation and Flux}

\begin{axiom}[Conservation and Flux]
For a bulk action $S_{\mathrm{bulk}}$ and boundary action $S_{\mathrm{bdry}}$ satisfying appropriate regularity conditions, there exists a flux functional $F\in\mathcal F$ such that for any compactly supported bulk variation $\delta\Phi,\delta g$,
$$
\delta(S_{\mathrm{bulk}}+S_{\mathrm{bdry}})=\text{(volume integral)}+F(\delta X_\Sigma),
$$
where $\delta X_\Sigma\in\mathcal A_\partial$ is the corresponding boundary source variation. It is required that for all boundary condition-satisfying physical variations, when the bulk equations hold,
$$
F(\delta X_\Sigma)=0
$$
and $F$ is linear for the allowed boundary variation family, so that boundary variation can be interpreted as an expression of the ``flux conservation condition.''
\end{axiom}

Intuitively, A1 requires that any residual term of bulk variation on the boundary can be identified as a flux functional acting on boundary data variation, thereby restating ``variational completeness'' as the condition that ``flux can be attributed to boundary language.''

\subsection{Axiom A2: Time Generation}

\begin{axiom}[Time Generation]
On the boundary observable algebra $\mathcal A_\partial$, there exists a one-parameter ${}^\ast$-automorphism group
$$
\{\alpha_t\}_{t\in\mathbb R}\subset\operatorname{Aut}(\mathcal A_\partial),
$$
whose generator is a closed unbounded derivation $\delta$, satisfying:
\begin{enumerate}
\item There exists a family of ``time observables'' $\mathcal T\subset\mathcal A_\partial$ such that for each $T\in\mathcal T$, $t\mapsto\omega(\alpha_t(T))$ is continuously differentiable;
\item The flux functionals satisfy appropriate invariance or conservation properties under $\alpha_t$, e.g., for energy functional $F_E\in\mathcal F$,
   $$F_E\circ\alpha_t=F_E;$$
\item The corresponding generator $\delta$ can be represented through a certain ``boundary Hamiltonian'' $H_\partial\in\mathcal A_\partial^{\prime\prime}$ as
$$
\frac{\mathrm d}{\mathrm dt}\omega(\alpha_t(A))=i\,\omega([H_\partial,\alpha_t(A)])
$$
   at least on a dense domain.
\end{enumerate}
\end{axiom}

We call the parameter $t\in\mathbb R$ the time scale generated by the boundary language. It is conjugate to frequency $\omega$ in the scattering realization, conjugate to ADM/proper time in the gravity realization, and conjugate to the modular parameter in the modular flow realization.

\subsection{Axiom A3: Monotonicity and Consistency}

\begin{axiom}[Monotonicity and Consistency]
Given boundary language $\mathfrak L_\Sigma$, for any two boundary states $\omega,\omega'$, there exists a non-negative function $S_{\mathrm{rel}}(\omega'\Vert\omega)$ called relative entropy or generalized entropy, satisfying:
\begin{enumerate}
\item Non-negativity: $S_{\mathrm{rel}}(\omega'\Vert\omega)\ge0$, with equality if and only if $\omega'=\omega$;
\item Monotonicity: for any subalgebra inclusion $\mathcal A_{\partial,1}\subset\mathcal A_{\partial,2}\subset\mathcal A_\partial$,
$$
S_{\mathrm{rel}}^{(1)}(\omega'\Vert\omega)\le S_{\mathrm{rel}}^{(2)}(\omega'\Vert\omega);
$$
\item Time consistency: under appropriate physical conditions (such as energy conditions or KMS conditions), if evolving along the flow $\omega_t,\omega_t'$ of the time generation axiom A2, then
$$
\frac{\mathrm d}{\mathrm dt}S_{\mathrm{rel}}(\omega_t'\Vert\omega_t)\le0
$$
   at least holds in one direction (defining the time arrow).
\end{enumerate}
\end{axiom}

A3 characterizes causal consistency and ``information cannot increase'' as intrinsic properties of the boundary language, providing a basis for the boundary geometric--algebraic definition of the time arrow.

---

\section{Boundary Language Realization in Scattering Theory}

\subsection{Short-Range Scattering and S-Matrix}

Consider self-adjoint operators $H_0$ and $H=H_0+V$ on Hilbert space $\mathcal H$, where $V$ is a short-range perturbation satisfying standard assumptions such that the wave operators
$$
\Omega_\pm=\operatorname{s-}\lim_{t\to\pm\infty}e^{itH}e^{-itH_0}
$$
exist and are complete. The S-matrix is defined as
$$
S=\Omega_+^\ast\Omega_-,
$$
which can be decomposed in the energy representation as
$$
S=\int^\oplus S(\omega)\,\mathrm d\mu(\omega),
$$
where $S(\omega)$ is a unitary matrix acting on the channel space.

Here we take the causal cut surface to be the ``asymptotic infinite in/out boundary,'' with the boundary algebra being
$$
\mathcal A_\partial=B(\mathcal H_{\mathrm{in}})\vee B(\mathcal H_{\mathrm{out}})
$$
or an appropriate subalgebra (e.g., algebra generated by asymptotic behavior). The boundary state $\omega$ can be taken as an incident wave packet or thermal equilibrium state.

\begin{definition}[Total Scattering Phase and Group Delay]
Let
$$
\Phi(\omega)=\arg\det S(\omega),\qquad\varphi(\omega)=\frac{1}{2}\Phi(\omega).
$$
Define the Wigner--Smith group delay operator as
$$
Q(\omega)=-iS(\omega)^\dagger\partial_\omega S(\omega).
$$
Denote the trace over the channel space as $\operatorname{tr}Q(\omega)$.

Also let $\xi(\omega)$ be the Birman--Kreĭn spectral shift function, and define the relative state density as
$$
\rho_{\mathrm{rel}}(\omega)=\frac{\mathrm d}{\mathrm d\omega}\xi(\omega)
$$
(sign convention see Appendix A).
\end{definition}

\subsection{Scale Identity and A2 Scattering Realization}

In the scattering context, we select ``time observables'' as observables generated by $S(\omega)$ in frequency space, and flux functionals include probability current, energy flow, and delay functionals.

\begin{theorem}[Scattering--Spectral Shift--Group Delay Scale Identity]
In a scattering system satisfying short-range and regularity assumptions, the scale function
$$
\kappa(\omega)=\frac{\varphi'(\omega)}{\pi}
$$
satisfies the identity with the relative state density $\rho_{\mathrm{rel}}(\omega)$ and the group delay trace $(2\pi)^{-1}\operatorname{tr}Q(\omega)$:
$$
\kappa(\omega)=\frac{\varphi'(\omega)}{\pi}=\rho_{\mathrm{rel}}(\omega)=\frac{1}{2\pi}\operatorname{tr}Q(\omega).
$$
This identity unifies the frequency derivative of the total phase, the spectral shift density, and the group delay into the same scale function $\kappa(\omega)$, which can be viewed as the concrete expression of the time scale generated by the boundary language at the scattering end.
\end{theorem}

\textit{Proof outline}: Using the Birman--Kreĭn formula
$$
\det S(\omega)=e^{-2\pi i\xi(\omega)},
$$
differentiating with respect to $\omega$ and combining with $\det S(\omega)=e^{i\Phi(\omega)}$ and the definition of $Q(\omega)=-iS^\dagger\partial_\omega S$, we obtain
$$
\operatorname{tr}Q(\omega)=2\pi\xi'(\omega),\quad\Phi(\omega)=-2\pi\xi(\omega),
$$
thus
$$
\varphi'(\omega)/\pi=-\xi'(\omega),\quad \rho_{\mathrm{rel}}(\omega)=-\xi'(\omega),
$$
synthesizing to yield the stated identity. Detailed proof in Appendix A.

\begin{corollary}[Scattering Time Scale]
For an incident wave packet frequency-localized in the neighborhood of $\omega_0$, its average group delay $\tau_{\mathrm{WS}}(\omega_0)$ gives an experimental reading of the boundary time scale:
$$
\tau_{\mathrm{WS}}(\omega_0)=\frac{\int\mathrm d\omega\,|a(\omega)|^2\operatorname{tr}Q(\omega)}{\int\mathrm d\omega\,|a(\omega)|^2}\approx 2\pi\,\kappa(\omega_0).
$$
\end{corollary}

\subsection{Boundary Language Perspective on Spectral Flow and Topological Branches}

Consider a scattering system $S(\omega;\lambda)$ that varies continuously with parameter $\lambda\in\mathbb R$. Under appropriate Fredholm and regularity conditions, the spectral shift function $\xi(\omega;\lambda)$ is a continuous function of $(\omega,\lambda)$, and its integral over $\omega$ gives the spectral flow:

\begin{proposition}[Spectral Flow as Boundary Topological Branch Change]
Over the parameter interval $[\lambda_1,\lambda_2]$, the cumulative spectral flow produced by eigenvalue crossings of the spectral threshold within the energy interval $[\omega_1,\omega_2]$
$$
\Delta N=\int_{\omega_1}^{\omega_2}\mathrm d\omega\,\rho_{\mathrm{rel}}(\omega;\lambda)
$$
is an integer and corresponds to a topological branch change of the boundary language. Its physical meaning is: parameter changes cause changes in the number of states allowed to cross the boundary, corresponding to transitions in topological classes or number of bound states.
\end{proposition}

---

\section{Boundary Language in Gravity and GHY Term}

\subsection{Variational Problem of Einstein--Hilbert Action}

Let $(M,g)$ be a four-dimensional spacetime with boundary $\partial M$. The Einstein--Hilbert bulk action is
$$
S_{\mathrm{EH}}[g]=\frac{1}{16\pi G}\int_M\mathrm d^4x\,\sqrt{-g}\,R.
$$

For metric variation $\delta g_{\mu\nu}$, the scalar curvature variation can be written as
$$
\delta R=g^{\mu\nu}\delta R_{\mu\nu}+R_{\mu\nu}\delta g^{\mu\nu},
$$
and $\delta R_{\mu\nu}$ contains first-order derivatives of $\delta g_{\mu\nu}$. After integration by parts, $\delta S_{\mathrm{EH}}$ includes volume and boundary terms, with boundary terms containing $\partial_n\delta g$ terms that cannot be expressed solely through variations of the induced metric $h_{ij}=g_{ij}|_{\partial M}$, meaning $S_{\mathrm{EH}}$ alone cannot give a well-defined Dirichlet boundary variational problem.

\begin{proposition}[GHY Term as Necessary Condition for A1]
If only $S_{\mathrm{EH}}$ is included without adding boundary action $S_{\mathrm{GHY}}$, then the boundary contribution to the metric variation contains terms that cannot be written as a flux functional $F\in\mathcal F$ acting linearly on boundary source variation $\delta h_{ij}$, thus violating the boundary language axiom A1. Adding the GHY boundary term
$$
S_{\mathrm{GHY}}[g]=\frac{1}{8\pi G}\int_{\partial M}\mathrm d^3x\,\sqrt{|h|}\,K,
$$
where $h_{ab}$ is the boundary-induced metric, $K$ is the trace of extrinsic curvature, $\epsilon=\pm 1$ depends on normal type. The boundary variation of the combined action $S_{\mathrm{tot}}=S_{\mathrm{EH}}+S_{\mathrm{GHY}}$ can be written as
$$
\delta S_{\mathrm{tot}}[g]=\text{(volume integral)}+\frac{1}{16\pi G}\int_{\partial M}\mathrm d^3x\,\sqrt{|h|}\,(K_{ij}-Kh_{ij})\delta h^{ij},
$$
where $K_{ij}$ is the extrinsic curvature, $K=K_{ij}h^{ij}$. Therefore
$$
F(\delta X_\Sigma)=\frac{1}{16\pi G}\int_{\partial M}\mathrm d^3x\,\sqrt{|h|}\,(K_{ij}-Kh_{ij})\delta h^{ij}
$$
constitutes a flux functional, implementing A1. Detailed derivation in Appendix B.
\end{proposition}

\subsection{Brown--York Quasilocal Energy and Time Generation}

From the variation of the total action with respect to boundary metric, the Brown--York quasilocal energy-momentum tensor can be defined:
$$
T_{ij}^{\mathrm{BY}}=\frac{2}{\sqrt{|h|}}\frac{\delta S_{\mathrm{tot}}}{\delta h^{ij}}=\frac{1}{8\pi G}(K_{ij}-Kh_{ij})+\text{(reference term)}.
$$

For a given time slice $\Sigma_t\subset\partial M$ and its unit timelike vector field $u^i$, the quasilocal energy can be defined:
$$
E_{\mathrm{BY}}[\Sigma_t]=\int_{\Sigma_t}\mathrm d^2x\,\sqrt{\sigma}\,u^iu^jT_{ij}^{\mathrm{BY}},
$$
where $\sigma$ is the induced two-dimensional metric on $\Sigma_t$.

\begin{theorem}[Boundary Hamiltonian and Geometric Time Generation]
In spacetime with ADM decomposition, there exists a boundary Hamiltonian $H_\partial$ which, under appropriate boundary conditions, generates time evolution on the boundary algebra through Poisson brackets or commutators: for any boundary observable $A\in\mathcal A_\partial$,
$$
\frac{\mathrm d}{\mathrm dt}\omega_t(A)=i\,\omega_t([H_\partial,A]),
$$
where $\omega_t$ is the state along time slice $\Sigma_t$. Moreover, $H_\partial$ can be obtained from the integral of Brown--York quasilocal energy, so geometric time is completely generated by the boundary language, satisfying A2.

In the static case (with timelike Killing vector), $H_\partial$ is equivalent to ADM mass or Komar energy, so Killing time, ADM time, and the time scale generated by boundary language belong to the same equivalence class.
\end{theorem}

---

\section{Operator Algebras, Modular Flow, and Relative Entropy}

\subsection{Standard Form and Modular Flow}

Let $\mathcal M$ be a von Neumann algebra acting on Hilbert space $\mathcal H$, and $\omega$ a faithful normal state on $\mathcal M$. The GNS representation yields vector $\Omega\in\mathcal H$ such that
$$
\omega(A)=\langle\Omega,A\Omega\rangle.
$$

The Tomita operator $S$ is defined by
$$
SA\Omega=A^\ast\Omega,\quad A\in\mathcal M
$$
whose polar decomposition gives
$$
S=J\Delta^{1/2},
$$
where $J$ is conjugation and $\Delta$ is the modular operator. The Tomita--Takesaki modular flow is defined as
$$
\sigma_t^\omega(A)=\Delta^{it}A\Delta^{-it},\quad t\in\mathbb R.
$$

In the boundary language framework, we take
$$
\mathcal A_\partial=\mathcal M,\quad \alpha_t=\sigma_t^\omega,\quad \omega\ \text{given}.
$$

\begin{proposition}[Modular Flow as Canonical Realization of A2]
For any standard form $(\mathcal M,\mathcal H,\Omega)$ and faithful normal state $\omega$, the Tomita--Takesaki modular flow $\{\sigma_t^\omega\}$ is a one-parameter ${}^\ast$-automorphism group satisfying:
\begin{enumerate}
\item $\omega\circ\sigma_t^\omega=\omega$ (i.e., $\omega$ is a KMS state for the modular flow);
\item For any $A\in\mathcal M$, the map $t\mapsto\omega(\sigma_t^\omega(A))$ is continuous.
\end{enumerate}
Therefore, $(\mathcal M,\omega,\{\sigma_t^\omega\})$ naturally satisfies the time generation axiom A2, providing a rigorous mathematical realization of ``modular time.''
\end{proposition}

\subsection{Relative Entropy and Monotonicity}

For two faithful normal states $\omega,\omega'$, the relative modular operator $\Delta_{\omega',\omega}$ can be defined, with the Araki relative entropy defined as
$$
S(\omega'\Vert\omega)=-\langle\Omega',\log\Delta_{\omega',\omega}\,\Omega'\rangle,
$$
where $\Omega'$ is the GNS vector for $\omega'$.

\begin{theorem}[Relative Entropy Monotonicity]
If $N\subset M$ is a von Neumann algebra inclusion and $\omega,\omega'$ are states on $M$, then
$$
S(\omega'|_N\Vert\omega|_N)\le S(\omega'\Vert\omega).
$$
\end{theorem}

This theorem expresses that ``reducing the accessible algebra does not increase distinguishability,'' which in the boundary language framework directly corresponds to the second condition of axiom A3.

When combined with geometry--holography, relative entropy monotonicity can be equivalent to certain monotonicity properties of generalized entropy, as well as the quantum null energy condition, quantum focussing condition, etc., thereby providing a unified ``information non-increase'' boundary law.

\subsection{Time Arrow and Modular Time}

Further, consider states evolving along the modular flow
$$
\omega_t=\omega\circ\sigma_t^\omega,\quad \omega_t'=\omega'\circ\sigma_t^\omega.
$$

Under appropriate energy conditions, KMS conditions, and locality assumptions, it can be proven or strongly expected that
$$
\frac{\mathrm d}{\mathrm dt}S(\omega_t'\Vert\omega_t)\le0,
$$
thereby defining the time arrow as the direction of monotonic decrease in relative entropy. Combined with the scattering-end time scale and gravity-end geometric time, a unified time geometric picture can be obtained.

---

\section{Unified Time Scale and Scale Equivalence Class}

\subsection{Trinity Scale Function}

Synthesizing the scattering realization of Section 3 and the gravity and modular flow realizations of Sections 4 and 5, we introduce the unified scale function
$$
\kappa(\omega)=\frac{\varphi'(\omega)}{\pi}=\rho_{\mathrm{rel}}(\omega)=\frac{1}{2\pi}\operatorname{tr}Q(\omega).
$$

At the scattering end, $\kappa(\omega)$ is directly given by the S-matrix; at the gravity end, $\kappa$ is related to phase delay and Shapiro delay in curved background propagation; at the modular flow end, $\kappa$ can be connected to the logarithmic spectral density of the modular operator $\Delta$ through spectral decomposition.

\begin{definition}[Time Scale Equivalence Class]
If two time parameters $t_1,t_2$ satisfy the existence of a strictly monotonic positive function $f:\mathbb R\to\mathbb R$ such that
$$
t_2=f(t_1)
$$
and for all observable time sequences generated by the boundary language, only scale rescaling exists between $t_1,t_2$ without changing the causal order and monotonicity, then $t_1,t_2$ are said to belong to the same time scale equivalence class, denoted as $[t]$.
\end{definition}

\begin{proposition}[Scale Unification of Scattering--Geometric--Modular Time]
In the boundary language framework satisfying A1--A3, scattering time (defined by $\kappa(\omega)$), geometric time (generated by boundary Hamiltonian $H_\partial$), and modular time (defined by modular flow $\sigma_t^\omega$) belong to the same scale equivalence class under appropriate conditions. The scale mapping is given by the scale function $\kappa$ and the geometric--algebraic bridge.
\end{proposition}

\subsection{Null--Modular Double Cover Boundary Sketch}

In null boundary and wedge modular flow, there exists the so-called Null--Modular double cover structure: the affine parameter of null geodesics can be obtained through appropriate rescaling of modular flow parameters, and there is a bilinear relationship between the modular Hamiltonian and the geometric generator. This structure shows that, in a limiting sense, time on the null boundary can be viewed as the geometrization of modular time; its generalized entropy variation satisfies the quantum focussing inequality, which is again the geometric projection of relative entropy monotonicity. Due to complex technical details, this paper only gives a conceptual statement here.

---

\section{Typical Models and Testable Predictions}

\subsection{One-Dimensional Potential Scattering Model}

Consider one-dimensional potential scattering $H_0=-\frac{\mathrm d^2}{\mathrm dx^2}$, $H=H_0+V(x)$ where $V(x)$ is a short-range potential. In/out channels are plane waves, S-matrix is a $2\times2$ unitary matrix, which can be written as
$$
S(\omega)=\begin{pmatrix}t(\omega)&r'(\omega)\\r(\omega)&t'(\omega)\end{pmatrix}.
$$

In this model:
\begin{itemize}
\item The total scattering phase $\varphi(\omega)$ is determined by the argument of $\det S(\omega)$;
\item The trace of the group delay operator $Q(\omega)$ gives the channel-averaged Wigner--Smith delay;
\item The relative state density $\rho_{\mathrm{rel}}(\omega)$ describes the correction to state density from bound states and resonant states.
\end{itemize}

Through the identity in Appendix A
$$
\kappa(\omega)=\frac{\varphi'(\omega)}{\pi}=\rho_{\mathrm{rel}}(\omega)=\frac{1}{2\pi}\operatorname{tr}Q(\omega),
$$
the boundary time scale can be defined on frequency-localized wave packets. Numerically, measuring $\operatorname{tr}Q(\omega)$ or phase jumps near resonance peaks can directly give experimental proxies for $\kappa(\omega)$.

\subsection{Static Black Hole Exterior and Quasilocal Energy}

In the static black hole exterior, selecting a large sphere $r=R$ as the boundary, define Brown--York quasilocal energy $E_{\mathrm{BY}}(R)$. When $R\to\infty$, $E_{\mathrm{BY}}$ approaches the ADM mass. For finite $R$, the ``Shapiro delay,'' light deflection phase, and $E_{\mathrm{BY}}$ can be connected to obtain consistency conditions between geometric time and scattering scale $\kappa(\omega)$. This provides a way to understand ``gravitational time dilation'' from the boundary language perspective.

\subsection{Rindler Wedge, Unruh Temperature, and Modular Time}

In Minkowski spacetime, take the Rindler wedge $x>|t|$. The modular flow on the wedge local algebra is exactly equivalent to Rindler time translation, with the vacuum state being a KMS state corresponding to temperature $T=a/2\pi$, where $a$ is acceleration. Thus the Unruh temperature can be restated as the temperature interpretation of ``modular time scale'' in the boundary language: for an accelerated observer, the modular flow is its physical time evolution, and the modular Hamiltonian characterizes the energy spectrum structure of its accessible algebra.

---

\section{Conclusion and Outlook}

The boundary language framework proposed in this paper, based on three axioms (conservation and flux, time generation, monotonicity and consistency), unifies the phase--spectral shift--time delay in scattering theory, the GHY boundary term and quasilocal energy in gravity, and the modular flow and relative entropy monotonicity in operator algebras under the same structure. The key technical result is the scale identity
$$
\kappa(\omega)=\frac{\varphi'(\omega)}{\pi}=\rho_{\mathrm{rel}}(\omega)=\frac{1}{2\pi}\operatorname{tr}Q(\omega),
$$
and the necessity of the GHY term in variational completeness and the canonicity of modular flow in time generation.

Future work directions include: complete axiomatization of Null--Modular double cover and quantum focussing inequality; realization of boundary language in time crystals, self-referential scattering networks, and causal networks; and systematic testing of scale identity and error geometry predictions at experimental and numerical levels.

---

\section*{Appendix A: Rigorous Derivation of Scattering--Spectral Shift--Group Delay Scale Identity}

This appendix gives detailed proof of Theorem 3.2.1.

\subsection*{A.1 Spectral Shift Function and Birman--Kreĭn Formula}

Let $H_0,H=H_0+V$ be self-adjoint operators on Hilbert space $\mathcal H_{\rm scatt}\in\mathbf{Hilb}_\mathcal U$, with $V$ being a trace class perturbation. The spectral shift function $\xi(\lambda)$ is defined through: for sufficiently smooth and well-decaying functions $f$,
$$
\operatorname{tr}\left(f(H)-f(H_0)\right)=\int_{-\infty}^{\infty}f'(\lambda)\,\xi(\lambda)\,\mathrm d\lambda.
$$

The Birman--Kreĭn formula shows that for almost every energy $\lambda$, the determinant of the S-matrix satisfies
$$
\det S(\lambda)=e^{-2\pi i\xi(\lambda)}.
$$

\subsection*{A.2 Differentiation with Respect to Energy and Group Delay}

Differentiating with respect to $\lambda$ yields
$$
\frac{\partial_\lambda\det S(\lambda)}{\det S(\lambda)}=-2\pi i\,\xi'(\lambda).
$$

On the other hand, for invertible matrix $S(\lambda)$, there is the logarithmic derivative formula
$$
\frac{\partial_\lambda\det S}{\det S}=\operatorname{tr}\left(S^{-1}\partial_\lambda S\right).
$$

The unitarity of the scattering matrix gives $S^{-1}=S^\dagger$. Define
$$
Q(\lambda)=-iS^\dagger(\lambda)\partial_\lambda S(\lambda),
$$
then
$$
\operatorname{tr}Q(\lambda)=-i\,\operatorname{tr}\left(S^\dagger\partial_\lambda S\right)=-i\,\frac{\partial_\lambda\det S}{\det S}=2\pi\,\xi'(\lambda).
$$

Thus
$$
\xi'(\lambda)=\frac{1}{2\pi}\operatorname{tr}Q(\lambda).
$$

\subsection*{A.3 Phase and Relative State Density}

Let
$$
\Phi(\lambda)=\arg\det S(\lambda).
$$

From the Birman--Kreĭn formula
$$
i\Phi(\lambda)=-2\pi i\xi(\lambda)\quad\Rightarrow\quad\Phi(\lambda)=-2\pi\xi(\lambda).
$$

The total scattering half-phase
$$
\varphi(\lambda)=\frac{1}{2}\Phi(\lambda)=-\pi\xi(\lambda).
$$

Differentiating with respect to $\lambda$ gives
$$
\frac{\varphi'(\lambda)}{\pi}=-\xi'(\lambda).
$$

According to the definition, the relationship between relative state density $\rho_{\mathrm{rel}}(\lambda)$ and the derivative of the spectral shift function can be chosen as
$$
\rho_{\mathrm{rel}}(\lambda)=-\xi'(\lambda),
$$
with the sign convention reflecting consistency in scattering and bound state counting. Thus
$$
\frac{\varphi'(\lambda)}{\pi}=\rho_{\mathrm{rel}}(\lambda).
$$

Comparing with the result from the previous subsection gives
$$
\rho_{\mathrm{rel}}(\lambda)=\frac{1}{2\pi}\operatorname{tr}Q(\lambda),
$$
ultimately yielding
$$
\frac{\varphi'(\lambda)}{\pi}=\rho_{\mathrm{rel}}(\lambda)=\frac{1}{2\pi}\operatorname{tr}Q(\lambda).
$$

---

\section*{Appendix B: Derivation of GHY Boundary Term and Variational Completeness}

This appendix provides detailed explanation of Proposition 4.1.1.

\subsection*{B.1 Boundary Variation of Einstein--Hilbert Action}

Starting from
$$
S_{\mathrm{EH}}[g]=\frac{1}{16\pi G}\int_M\mathrm d^4x\,\sqrt{-g}\,R
$$
and varying with respect to $g_{\mu\nu}$, using
$$
\delta R_{\mu\nu}=\nabla_\rho\delta\Gamma^\rho_{\mu\nu}-\nabla_\mu\delta\Gamma^\rho_{\rho\nu},
$$
$\delta S_{\mathrm{EH}}$ can be written as the sum of volume and boundary terms. The boundary term can be organized as
$$
\delta S_{\mathrm{EH}}\big|_{\partial M}=\frac{1}{16\pi G}\int_{\partial M}\mathrm d^3x\,\sqrt{|h|}\,n_\mu\left(g^{\nu\rho}\delta\Gamma^\mu_{\nu\rho}-g^{\mu\nu}\delta\Gamma^\rho_{\rho\nu}\right),
$$
where $n_\mu$ is the boundary unit normal vector. Since $\delta\Gamma$ contains $\partial\delta g$ terms, the boundary variation cannot be expressed solely through $\delta h_{ij}$, thus not conforming to the form of ``flux functional acting linearly on source variation.''

\subsection*{B.2 Compensating Effect of GHY Term}

Introduce the GHY boundary action
$$
S_{\mathrm{GHY}}[g]=\frac{1}{8\pi G}\int_{\partial M}\mathrm d^3x\,\sqrt{|h|}\,K.
$$

The extrinsic curvature is defined as
$$
K_{ij}=h_i^{\ \mu}h_j^{\ \nu}\nabla_\mu n_\nu,\quad K=K_{ij}h^{ij}.
$$

Varying $g_{\mu\nu}$, the variation of $K$ can be expressed through combinations of $\delta h_{ij}$ and normal derivatives. After careful organization, $\delta S_{\mathrm{GHY}}$ produces exactly compensating terms such that:
\begin{enumerate}
\item All terms containing $\partial_n\delta g$ cancel each other in $\delta S_{\mathrm{EH}}+\delta S_{\mathrm{GHY}}$;
\item The remaining boundary variation only includes $\delta h_{ij}$ without its derivatives.
\end{enumerate}

The final result is
$$
\delta S_{\mathrm{tot}}=\delta S_{\mathrm{EH}}+\delta S_{\mathrm{GHY}}=\frac{1}{16\pi G}\int_M\mathrm d^4x\,\sqrt{-g}\,G_{\mu\nu}\delta g^{\mu\nu}+\frac{1}{16\pi G}\int_{\partial M}\mathrm d^3x\,\sqrt{|h|}\,(K_{ij}-Kh_{ij})\delta h^{ij}.
$$

If choosing Dirichlet boundary conditions $\delta h_{ij}=0$, the boundary term automatically vanishes, achieving variational completeness; if allowing $\delta h_{ij}\ne0$, the boundary term can be interpreted as flux functional
$$
F(\delta X_\Sigma)=\frac{1}{16\pi G}\int_{\partial M}\mathrm d^3x\,\sqrt{|h|}\,(K_{ij}-Kh_{ij})\delta h^{ij},
$$
where $\delta X_\Sigma\equiv\delta h_{ij}$. Thus axiom A1 is realized.

---

\section*{Appendix C: Technical Details of Modular Flow, Relative Entropy, and Time Arrow}

\subsection*{C.1 Construction of Standard Form and Modular Operator}

Given $(\mathcal M,\omega)$, the GNS representation $(\pi_\omega,\mathcal H_\omega,\Omega_\omega)$ gives
$$
\omega(A)=\langle\Omega_\omega,\pi_\omega(A)\Omega_\omega\rangle.
$$

Without confusion, omit $\pi_\omega$ and denote $A\Omega_\omega$ as the GNS vector.

The Tomita operator $S$ is defined by
$$
SA\Omega_\omega=A^\ast\Omega_\omega
$$
whose closure's polar decomposition $S=J\Delta^{1/2}$ gives the modular operator $\Delta$ and modular conjugation $J$. The modular flow
$$
\sigma_t^\omega(A)=\Delta^{it}A\Delta^{-it}
$$
satisfies the Tomita--Takesaki theorem: for any $t\in\mathbb R$, $\sigma_t^\omega$ is a ${}^\ast$-automorphism of $\mathcal M$, and
$$
\omega(A\sigma_{i}^\omega(B))=\omega(BA)
$$
gives the KMS condition.

\subsection*{C.2 Relative Modular Operator and Relative Entropy}

Given two states $\omega,\omega'$, the relative modular operator $\Delta_{\omega',\omega}$ is constructed through the relative Tomita operator:
$$
S_{\omega',\omega}A\Omega_\omega=A^\ast\Omega',
$$
whose polar decomposition is $S_{\omega',\omega}=J_{\omega',\omega}\Delta_{\omega',\omega}^{1/2}$.
The Araki relative entropy is
$$
S(\omega'\Vert\omega)=-\langle\Omega',\log\Delta_{\omega',\omega}\,\Omega'\rangle.
$$

Relative entropy monotonicity can be proven through completely positive trace-preserving maps and Stinespring representation: if $\Phi:\mathcal M\to\mathcal N$ is completely positive trace-preserving, then
$$
S(\omega'\circ\Phi\Vert\omega\circ\Phi)\le S(\omega'\Vert\omega).
$$

Taking $\Phi$ as conditional expectation or subalgebra restriction yields Theorem 5.2.1.

\subsection*{C.3 Inequality Form of Time Arrow}

In some cases, the time arrow statement can be strengthened to the following inequality: let $\omega_t,\omega_t'$ be states evolving along the modular flow, then
$$
\frac{\mathrm d^2}{\mathrm dt^2}S(\omega_t'\Vert\omega_t)\ge0,
$$
i.e., relative entropy is convex in modular time. This type of property is closely related to the second-order variation of generalized entropy in holographic and QNEC/QFC literature, corresponding to quantum focussing conditions on null boundaries.

---

\section*{Appendix D: Error Geometry of Finite-Order Euler--Maclaurin and Poisson Discipline}

The scale function $\kappa(\omega)$ generated by boundary language and experimental readings are often connected through frequency integration and discrete sampling. To ensure rigorous ``theory--experiment docking,'' an ``error geometry'' framework controlling aliasing and truncation errors is needed.

\subsection*{D.1 Finite-Order Euler--Maclaurin Formula}

For sufficiently smooth functions $f$, on interval $[a,b]$, the Euler--Maclaurin formula is
$$
\sum_{n=a}^{b}f(n)=\int_a^b f(x)\,\mathrm dx+\frac{f(a)+f(b)}{2}+\sum_{k=1}^m\frac{B_{2k}}{(2k)!}\left(f^{(2k-1)}(b)-f^{(2k-1)}(a)\right)+R_m,
$$
where $B_{2k}$ are Bernoulli numbers and $R_m$ is the remainder. We enforce taking only finite order $m$ and view the upper bound of $R_m$ on a given function class as part of ``error geometry'': its magnitude is controlled by high-order derivatives of $f$ and principal singularities.

In principle, we require:
\begin{enumerate}
\item Any polynomial or rational approximation fitting the scale function $\kappa(\omega)$ controls the growth of high-order derivatives within physically relevant frequency intervals;
\item All errors $R_m$ introduced by Euler--Maclaurin truncation do not produce new singularities, i.e., ``singularity does not increase, poles = principal scales.''
\end{enumerate}

\subsection*{D.2 Poisson Summation and Aliasing Control}

The Poisson summation formula
$$
\sum_{n\in\mathbb Z}f(n)=\sum_{k\in\mathbb Z}\hat f(2\pi k)
$$
connects discrete sampling with frequency space. For numerical calculations of scale functions and related response functions, sampling step $\Delta\omega$ and frequency domain support determine aliasing errors.

In the boundary language framework, we require:
\begin{enumerate}
\item Sampling satisfies the Nyquist--Shannon condition so that aliasing error can be upper bounded;
\item For each experimental sampling scheme, give explicit aliasing error estimates and prove they do not introduce new singularities, only changing weights or distributions;
\item Error analysis follows ``finite-order'' discipline: not relying on formally infinite sums or infinite differentiation, but forming closed error geometry through finite-order truncation and rigorous error bounds.
\end{enumerate}

This framework makes the mapping from theoretical scale function $\kappa(\omega)$ to discrete measurement data mathematically controllable, both ensuring consistency of boundary language three axioms in numerical implementation and providing direct error budget tools for experimental design.

\end{document}

