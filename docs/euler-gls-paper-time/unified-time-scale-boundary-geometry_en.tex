\documentclass[12pt]{article}

% Essential packages
\usepackage[utf8]{inputenc}
\usepackage[T1]{fontenc}
\usepackage{amsmath,amssymb,amsthm}
\usepackage{mathrsfs}
\usepackage{geometry}
\usepackage{hyperref}
\usepackage{braket}
\usepackage{graphicx}

% Geometry settings
\geometry{a4paper, margin=1in}

% Hyperref settings
\hypersetup{
    colorlinks=true,
    linkcolor=blue,
    citecolor=blue,
    urlcolor=blue
}

% Theorem environments
\theoremstyle{plain}
\newtheorem{theorem}{Theorem}[section]
\newtheorem{lemma}[theorem]{Lemma}
\newtheorem{proposition}[theorem]{Proposition}
\newtheorem{corollary}[theorem]{Corollary}

\theoremstyle{definition}
\newtheorem{definition}[theorem]{Definition}
\newtheorem{example}[theorem]{Example}
\newtheorem{remark}[theorem]{Remark}

% Title information
\title{Unified Time Scale and Boundary Time Geometry:\\
Single Structural Framework of Scattering Phase, Modular Flow, and Gravitational Boundary Terms}

\author{Haobo Ma$^1$ \and Wenlin Zhang$^2$\\
\small $^1$Independent Researcher\\
\small $^2$National University of Singapore}

\date{\today}

\begin{document}

\maketitle

\begin{abstract}
We construct a time unification framework with boundary as fundamental stage, integrating three originally separate time structures into different projections of the same ``boundary time geometry'':
(1) On scattering and spectral theory end, based on Birman--Kreĭn formula and Wigner--Smith time delay, prove scale identity among total scattering phase derivative, relative state density, and group delay trace;
(2) On operator algebra and information end, based on Tomita--Takesaki modular theory and Connes--Rovelli thermal time hypothesis, characterize modular flow parameter as intrinsic time determined by state--algebra pair, introducing time scale equivalence class;
(3) On gravity and geometry end, based on Einstein--Hilbert--Gibbons--Hawking--York action and its boundary variation, unify extrinsic curvature and time translation generated by boundary Hamiltonian into same boundary time geometry.

In unified model, boundary is described by triple structure: intrinsic metric and extrinsic curvature of geometric boundary $\partial M$, quantum boundary algebra $\mathcal{A}_\partial$ with state $\omega$, and scattering matrix $S(\omega)$ defined in external region. Under well-posed traceable scattering assumptions, construct ``scale identity''
$$
\frac{\varphi'(\omega)}{\pi}=\rho_{\mathrm{rel}}(\omega)=\frac{1}{2\pi}\operatorname{tr}Q(\omega),
$$
where $\varphi(\omega)=\tfrac{1}{2}\arg\det S(\omega)$ is total scattering phase, $\rho_{\mathrm{rel}}$ is derivative of spectral shift density, $Q(\omega)=-iS(\omega)^\dagger\partial_\omega S(\omega)$ is Wigner--Smith time delay matrix. This scale is standardized at modular time and geometric time ends respectively through modular Hamiltonian operator $K_\omega=-\log\Delta_\omega$ and Hamilton--Jacobi functional of GHY boundary action, thus defining single boundary time scale equivalence class $[\tau]$.

Based on this, we give several unification theorems: (i) Categorical existence uniqueness of time scale equivalence class: on common domain of given boundary algebra, scattering data, and gravitational boundary geometry, all acceptable time parameters are monotonic rescalings of one fundamental boundary time; (ii) Cosmological redshift relation $1+z=1/a(t)$ can be interpreted as global rescaling of this time equivalence class on large scales, thus unifying local scattering time delay with conformal time in FRW background; (iii) In cases with horizons (Rindler wedge and black hole exterior), modular flow time, proper time of accelerated observer, and geometric outward normal translation time fall into same equivalence class.

Full text gives explicit assumptions and theorems at scattering--spectral, modular flow--information, and gravity--boundary geometry ends respectively, with detailed proofs of scale identity, modular time equivalence, and boundary Hamiltonian generated time in appendices, finally proposing engineered measurement schemes based on waveguides, microwave cavities, and Aharonov--Bohm rings to cross-calibrate three types of time scales experimentally.
\end{abstract}

\noindent\textbf{Keywords:} Boundary Time Geometry; Birman--Kreĭn Formula; Wigner--Smith Time Delay; Spectral Shift Function; Tomita--Takesaki Modular Flow; Thermal Time Hypothesis; Gibbons--Hawking--York Boundary Term; Time Scale Equivalence Class; Cosmological Redshift

---

\section{Introduction and Historical Context}

Time appears in physical theories in multiple guises: as coordinate parametrizing worldlines in classical and relativistic physics, as continuous variable generating unitary evolution in quantum mechanics, as imaginary time period inversely proportional to temperature in statistical physics, as delay of particle dwelling in interaction region in scattering theory, and in general relativity reflecting as geometric structure through metric and extrinsic curvature. Different time perspectives often based on different fundamental objects and measurement processes, thus difficult to unify within single mathematical framework for long time.

In scattering and spectral theory aspects, Lifshits--Kreĭn spectral shift function $\xi(\lambda)$ and its derivative play central role in describing state density difference before and after interaction; Birman--Kreĭn formula gives precise relationship between scattering matrix determinant and spectral shift function $\det S(\lambda)=\exp(-2\pi i\xi(\lambda))$. Under appropriate normalization, derivative of spectral shift density can be interpreted as total scattering phase derivative and equivalent to trace of Wigner--Smith time delay matrix, thus viewing ``time delay'' as geometric derivative of phase--spectral structure.

In operator algebra and quantum statistics aspects, Tomita--Takesaki theory shows: given von Neumann algebra $(M,\Omega)$ with cyclic and separating vector, can construct modular automorphism group $\sigma_t^\Omega(x)=\Delta^{it}x\Delta^{-it}$ through polar decomposition of modular operator $\Delta$, whose image on outer automorphism group independent of chosen state. Connes--Rovelli thermal time hypothesis proposes: in generally covariant quantum theory, physical time flow determined by modular flow of state--algebra pair, modular parameter itself is time, thus ``time'' becomes derived object of state structure rather than a priori parameter.

In gravity and geometry aspects, Einstein--Hilbert action on manifold with boundary insufficient to give well-defined variational principle, must add Gibbons--Hawking--York boundary term
$$
S_{\mathrm{GHY}}=\frac{1}{8\pi G}\int_{\partial M}\mathrm{d}^3y\,\epsilon\sqrt{h}\,K,
$$
where $h_{ab}$ is induced boundary metric, $K$ is trace of extrinsic curvature, $\epsilon=\pm 1$ depends on normal type. Variation of EH+GHY combined action under fixed boundary induced geometry gives Einstein equations, boundary term can be viewed as Hamilton--Jacobi functional, whose functional derivative with respect to boundary metric corresponds to conjugate momentum and quasilocal energy, thus encoding ``time generator'' of translation along boundary normal.

Structural equivalences among scattering phase derivative--time delay, modular flow parameter--thermal time, GHY boundary term--geometric time are each highly mature mathematically and physically meaningful, but their structural equivalence only appears scattered in existing literature. For example, scattering phase derivative can be interpreted both as state density difference and as group delay through frequency derivative of S-matrix; in S-matrix statistical mechanics, scattering phase derivative enters state density integral, thus connecting with heat and free energy. Modular flow used in AdS/CFT and entanglement wedge reconstruction to define modular Hamiltonian and energy of geometric region, connecting to propagation time in spacetime through HKLL/Petz reconstruction. GHY boundary term viewed as key object defining gravitational transfer amplitude and boundary time in loop quantum gravity and quasilocal Hamiltonian formalism.

Goal of this paper is on rigorously provable basis to unify above three ends into ``boundary time geometry'' framework. This framework takes boundary algebra, state, scattering matrix, and boundary geometry as fundamental objects, with ``time scale equivalence class'' as core, proving: under appropriate assumptions, scattering time delay, modular time, and geometric boundary time belong to same equivalence class, any physical time reading can be viewed as monotonic rescaling of single boundary time parameter. Furthermore, incorporating cosmological redshift and scale factor evolution in FRW spacetime into same scale structure, connecting macroscopic cosmic time with microscopic scattering time delay.

---

\section{Model and Assumptions}

This section gives mathematical and physical model supporting unified framework, explicitly specifying assumption domain.

\subsection{Scattering and Spectral End}

Consider self-adjoint operator pair $(H,H_0)$ acting on separable Hilbert space $\mathcal{H}$, satisfying following standard scattering assumptions:

\begin{enumerate}
\item[(1)] $H_0$ possesses absolutely continuous spectral subspace $\mathcal{H}_{\mathrm{ac}}(H_0)$, in which energy representation can be established, making $H_0$ multiplication operator $E\mapsto E$ in that representation;

\item[(2)] Perturbation $V=H-H_0$ such that for some $p\leq 1$, $(H+i)^{-p}-(H_0+i)^{-p}\in\mathfrak{S}_1$, thus satisfying basic condition of trace-class scattering theory;

\item[(3)] Wave operators $W_\pm=\operatorname{s}\text{-}\lim_{t\to\pm\infty}e^{itH}e^{-itH_0}P_{\mathrm{ac}}(H_0)$ exist and are complete, thus scattering operator $S=W_+^\dagger W_-$ well-defined on $\mathcal{H}_{\mathrm{ac}}(H_0)$.
\end{enumerate}

In energy representation, $S$ fibers into family of unitary matrices $S(\omega)$, where $\omega$ denotes energy or frequency variable. Assume for almost everywhere $\omega$, $S(\omega)-\mathbb{1}\in\mathfrak{S}_1$, thus determinant $\det S(\omega)$ and Wigner--Smith time delay operator
$$
Q(\omega)=-iS(\omega)^\dagger\partial_\omega S(\omega)
$$
well-defined and traceable.

Define spectral shift function $\xi(\omega;H,H_0)$ as function satisfying Lifshits--Kreĭn trace formula, whose derivative $\xi'(\omega)$ gives relative state density difference: $\rho_{\mathrm{rel}}(\omega):=-\xi'(\omega)$.

\subsection{Modular Flow and Thermal Time End}

Let $M\subset B(\mathcal{H})$ be von Neumann algebra, $\Omega\in\mathcal{H}$ be cyclic and separating vector, $|\Omega|=1$. Denote vector state $\omega(x)=(x\Omega,\Omega)$. Tomita--Takesaki theory gives polar decomposition $S=J\Delta^{1/2}$ of closed operator $S$, where $J$ is modular conjugation, $\Delta$ is modular operator. Modular automorphism group defined as
$$
\sigma_t^\omega(x)=\Delta^{it}x\Delta^{-it},\quad t\in\mathbb{R}.
$$
$\sigma_t^\omega$ is one-parameter automorphism group of $M$, and $\omega$ is its KMS state.

Connes proved: for any two faithful states $\omega,\omega'$, their modular flows' images in outer automorphism group $\mathrm{Out}(M)$ are consistent, i.e., there exists 1--cocycle $u_t\in M$ such that $\sigma_t^{\omega'}=\operatorname{Ad}(u_t)\circ\sigma_t^\omega$, thus obtaining state-independent ``geometric time'' on $\mathrm{Out}(M)$.

Thermal time hypothesis proposes: physical time flow determined by modular flow, i.e., modular parameter $t$ itself is time scale, rather than given by a priori background.

\subsection{Gravity and Boundary Geometry End}

Consider four-dimensional spacetime manifold $(M,g_{\mu\nu})$ with boundary $\partial M$. Gravitational action chooses Einstein--Hilbert plus Gibbons--Hawking--York sum
$$
S_{\mathrm{grav}}=\frac{1}{16\pi G}\int_M\mathrm{d}^4x\sqrt{-g}\,R+\frac{1}{8\pi G}\int_{\partial M}\mathrm{d}^3y\,\epsilon\sqrt{h}\,K.
$$
Assume boundary is smooth three-dimensional manifold of non-zero measure, distinguishing spacelike and timelike boundaries. When varying while keeping boundary induced metric $h_{ab}$ fixed, volume term gives Einstein equations, boundary term variation determines boundary conjugate momentum and quasilocal energy.

In some cases (such as vacuum region, partial LQG construction), volume term vanishes on shell, Hamilton--Jacobi action completely given by GHY boundary term, thus boundary action itself becomes object generating normal time evolution.

\subsection{Unified Boundary System and Time Scale Equivalence}

We call triple data
$$
\mathcal{B}=(\mathcal{A}_\partial,\omega_\partial,S(\omega);h_{ab},K_{ab})
$$
a boundary system, where $\mathcal{A}_\partial$ is boundary algebra generated by scattering channels and near-boundary fields, $\omega_\partial$ is faithful state on it (can be given by scattering incoming state or boundary CFT state), $S(\omega)$ is scattering matrix on energy shell, $h_{ab},K_{ab}$ are intrinsic and extrinsic data of geometric boundary.

On boundary system, we allow three types of one-parameter evolution:

\begin{enumerate}
\item[(1)] ``Phase--delay'' scale on scattering energy parameter $\omega$, determined by $S(\omega)$ and $Q(\omega)$;

\item[(2)] Modular parameter $t_{\mathrm{mod}}$, generated by modular flow of $(\mathcal{A}_\partial,\omega_\partial)$;

\item[(3)] Geometric parameter $t_{\mathrm{geom}}$, generated by translation along boundary normal (or evolution driven by extrinsic curvature).
\end{enumerate}

Core assumption of unified framework is: under appropriate physical situations (such as asymptotically flat or AdS spacetime infinite boundary, Rindler wedge near black hole horizon, conformal boundary of FRW universe), above three types of parameters can all be defined and satisfy common equivalence relation, forming time scale equivalence class $[\tau]$.

---

\section{Main Results (Theorems and Alignments)}

This section states main theorems and correspondences of unified framework, strictly distinguishing known results from newly introduced structures.

\begin{theorem}[Scattering Phase--Spectral Shift--Group Delay Scale Identity (Theorem 1)]
Under aforementioned scattering assumptions, let spectral shift function be $\xi(\omega;H,H_0)$, define relative state density
$$
\rho_{\mathrm{rel}}(\omega):=-\xi'(\omega).
$$
Let total scattering phase
$$
\Phi(\omega):=\arg\det S(\omega),\qquad\varphi(\omega):=\frac{1}{2}\Phi(\omega),
$$
Wigner--Smith delay operator
$$
Q(\omega)=-iS(\omega)^\dagger\partial_\omega S(\omega).
$$
Then for almost everywhere $\omega$, scale identity holds
$$
\frac{\varphi'(\omega)}{\pi}=\rho_{\mathrm{rel}}(\omega)=\frac{1}{2\pi}\operatorname{tr}Q(\omega).
$$
\end{theorem}

Brief: First equality from Birman--Kreĭn formula $\det S(\omega)=\exp(-2\pi i\xi(\omega))$ and relationship between spectral shift function and state density; second equality from connection between derivative of $\ln\det S(\omega)$ with respect to frequency and Wigner--Smith operator trace.

\begin{theorem}[Time Scale Equivalence Class of Modular Flow (Theorem 2)]
Let $(M,\Omega)$ be von Neumann algebra with cyclic separating vector as above, $\omega,\omega'$ be two faithful states, defining modular flows $\sigma_t^\omega,\sigma_t^{\omega'}$ respectively.

\begin{enumerate}
\item[(1)] There exists family of unitary operators $u_t\in M$ satisfying 1--cocycle condition $u_{t+s}=u_t\sigma_t^\omega(u_s)$, such that
$$
\sigma_t^{\omega'}=\operatorname{Ad}(u_t)\circ\sigma_t^\omega.
$$

\item[(2)] In outer automorphism group $\mathrm{Out}(M)$, $[\sigma_t^{\omega}]=[\sigma_t^{\omega'}]$, time parameter $t$ scale only undergoes linear rescaling when state changes.

\item[(3)] If there exists geometric clock (such as inertial or uniformly accelerated observer) whose proper time $\tau_{\mathrm{phys}}$ relates to modular parameter $t_{\mathrm{mod}}$ through KMS temperature $\beta$ (such as Unruh temperature $T=a/2\pi$ or thermal equilibrium state), then $\tau_{\mathrm{phys}}=\alpha t_{\mathrm{mod}}$, where $\alpha$ determined by $\beta$. Therefore modular parameter defines time scale equivalence class $[\tau_{\mathrm{mod}}]$.
\end{enumerate}
\end{theorem}

\begin{theorem}[GHY Boundary Action and Geometric Time (Theorem 3)]
Under Einstein--Hilbert--GHY action, for given boundary $\partial M$ region $\Sigma$, define Hamilton--Jacobi functional
$$
S_{\mathrm{HJ}}[h_{ab}]=\frac{1}{8\pi G}\int_\Sigma\mathrm{d}^3y\,\epsilon\sqrt{h}\,K,
$$
in case where vacuum Einstein equation holds and volume term vanishes on shell, variation of this functional with respect to boundary induced metric $h_{ab}$ gives conjugate momentum
$$
\pi^{ab}=\frac{\delta S_{\mathrm{HJ}}}{\delta h_{ab}}=\frac{\epsilon}{16\pi G}\sqrt{h}\,(K^{ab}-Kh^{ab}),
$$
defining quasilocal energy density and generator of translation along boundary normal.

If choosing timelike direction and unit normal on boundary, decomposing extrinsic curvature as $K=K_{tt}+K_{\mathrm{spatial}}$, then there exists geometric time parameter $t_{\mathrm{geom}}$ such that for small time translation $\delta t_{\mathrm{geom}}$, action change satisfies
$$
\delta S_{\mathrm{HJ}}=E_{\mathrm{q.l.}}\delta t_{\mathrm{geom}},
$$
where $E_{\mathrm{q.l.}}$ is quasilocal energy, thus $t_{\mathrm{geom}}$ uniquely determined by boundary geometry and gravitational action, forming geometric time scale.
\end{theorem}

\begin{definition}[Boundary Time Scale Equivalence Relation (Definition 4)]
On boundary system $\mathcal{B}$, consider three types of time parameters: scattering time parameter $t_{\mathrm{scatt}}$ (e.g., integral of Wigner--Smith delay), modular parameter $t_{\mathrm{mod}}$, and geometric time $t_{\mathrm{geom}}$.

Define equivalence relation $\sim$: if there exist $C^1$ strictly monotonic function $f$ such that
$$
t_{\mathrm{scatt}}=f_{\mathrm{sm}}(t_{\mathrm{mod}}),\quad t_{\mathrm{mod}}=f_{\mathrm{mg}}(t_{\mathrm{geom}}),
$$
and derivative at $0$ finite and non-zero, then three belong to same time scale equivalence class, written $[\,\tau\,]$.
\end{definition}

\begin{theorem}[Existence and (Local) Uniqueness of Time Scale Equivalence Class (Theorem 4)]
Let boundary system $\mathcal{B}$ satisfy:

\begin{enumerate}
\item[(1)] Scattering side satisfies Theorem 1 assumptions, defining group delay trace scale
$$
\mathrm{d}\tau_{\mathrm{scatt}}(\omega)=\frac{1}{2\pi}\operatorname{tr}Q(\omega)\,\mathrm{d}\omega;
$$

\item[(2)] Operator algebra side satisfies Theorem 2 assumptions, there exists modular flow $\sigma_t^\omega$ with corresponding modular Hamiltonian operator $K_\omega=-\log\Delta_\omega$;

\item[(3)] Gravity side satisfies Theorem 3 assumptions, there exists boundary Hamilton--Jacobi functional $S_{\mathrm{HJ}}$, whose parameter for timelike translation is $t_{\mathrm{geom}}$.
\end{enumerate}

And assume in some energy window and geometric region there exists AdS/CFT or scattering--geometry correspondence, such that structure-preserving isomorphism exists between scattering channels and boundary algebra--geometry (such as correspondence between Rindler wedge of spherical region and its CFT modular Hamiltonian).

Then on this common domain there exists time scale equivalence class $[\tau]$ satisfying:

\begin{enumerate}
\item[(1)] For any observation process, its time reading $t_{\mathrm{obs}}$ is $C^1$ monotonic function of $\tau$;

\item[(2)] If introducing another time parameter $\tilde{t}$ and requiring its unit interval equivalent to unit changes of scattering phase derivative, modular Hamiltonian expectation value, and GHY boundary action, then $\tilde{t}$ must locally be linear rescaling of $\tau$, therefore $[\tau]$ locally unique.
\end{enumerate}
\end{theorem}

\begin{theorem}[Cosmological Redshift as Global Rescaling of Time Scale (Theorem 5)]
In spatially isotropic, homogeneous FRW universe, metric can be written as
$$
\mathrm{d}s^2=-\mathrm{d}t^2+a(t)^2\gamma_{ij}\mathrm{d}x^i\mathrm{d}x^j,
$$
where $a(t)$ is scale factor. Introduce conformal time $\eta$ satisfying $\mathrm{d}t=a(\eta)\mathrm{d}\eta$.

Let there be radiation geodesic between source and observer at given redshift $z$, frequency redshift relation
$$
1+z=\frac{1}{a(t_\mathrm{em})},
$$
where $t_\mathrm{em}$ is emission time, at observation $a(t_0)=1$.

If viewing source--observer system scattering as effective ``far-region scattering'' on cosmological background, then there exists boundary time scale $\tau$ such that local observer's conformal time increment $\mathrm{d}\eta$ and Wigner--Smith delay scale $\mathrm{d}\tau_{\mathrm{scatt}}$ belong to same equivalence class, while cosmological redshift manifests as global rescaling of $\tau$
$$
\mathrm{d}\tau_{\mathrm{cosmo}}=(1+z)\,\mathrm{d}\tau_{\mathrm{local}},
$$
thus macroscopic cosmic time evolution can be viewed as scale factor evolution of unified time scale equivalence class.
\end{theorem}

---

\section{Proofs}

This section gives proof skeletons of main theorems, complete technical details placed in appendices.

\subsection{Proof of Theorem 1}

Birman--Kreĭn formula gives
$$
\det S(\omega)=\exp(-2\pi i\xi(\omega)).
$$
Let $\Phi(\omega):=\arg\det S(\omega)$, then there exists continuous branch choice such that
$$
\Phi(\omega)=-2\pi\xi(\omega).
$$
Define half-phase $\varphi(\omega)=\tfrac{1}{2}\Phi(\omega)$, then
$$
\varphi'(\omega)=-\pi\xi'(\omega)=\pi\rho_{\mathrm{rel}}(\omega).
$$
Thus first equality holds.

On other hand, using $\ln\det S(\omega)=\operatorname{tr}\ln S(\omega)$ and chain rule,
$$
\partial_\omega\ln\det S(\omega)=\operatorname{tr}(S(\omega)^{-1}\partial_\omega S(\omega)).
$$
By unitarity of $S(\omega)$, $S(\omega)^{-1}=S(\omega)^\dagger$. Writing Wigner--Smith operator as
$$
Q(\omega)=-iS(\omega)^\dagger\partial_\omega S(\omega),
$$
then
$$
\partial_\omega\ln\det S(\omega)=i\operatorname{tr}Q(\omega).
$$
On other hand, by Birman--Kreĭn formula
$$
\partial_\omega\ln\det S(\omega)=-2\pi i\xi'(\omega)=2\pi i\rho_{\mathrm{rel}}(\omega),
$$
comparing two expressions yields
$$
i\operatorname{tr}Q(\omega)=2\pi i\rho_{\mathrm{rel}}(\omega)\quad\Rightarrow\quad\rho_{\mathrm{rel}}(\omega)=\frac{1}{2\pi}\operatorname{tr}Q(\omega).
$$
Combining with aforementioned $\varphi'/\pi=\rho_{\mathrm{rel}}$ gives scale identity. Technical requirements of above derivation (such as trace-class conditions, logarithm branch choice) rigorously verified in Appendix A.

\subsection{Proof of Theorem 2}

(1) and (2) are standard conclusions of Tomita--Takesaki theory and Connes modular coboundary theory: through closure and polar decomposition of $S_0:m\Omega\mapsto m^\ast\Omega$ construct modular operator $\Delta$, then using $\Delta^{it}$ realize modular flow, can prove for any two faithful states, modular flows' images in $\mathrm{Out}(M)$ are consistent.

(3) When modular flow describes time evolution of KMS state, KMS condition connects modular parameter $t_{\mathrm{mod}}$ with temperature $\beta^{-1}$. In Unruh effect, observer with acceleration $a$ experiences temperature $T=a/2\pi$, corresponding to period $\beta=2\pi/a$. Modular flow period in imaginary time direction is $\beta$, thus fixing proportion constant between modular parameter and observer proper time $\tau_{\mathrm{phys}}$, obtaining $\tau_{\mathrm{phys}}=\alpha t_{\mathrm{mod}}$, $\alpha\propto\beta$.

\subsection{Proof of Theorem 3}

Varying EH+GHY total action, using Palatini identity and $\delta\sqrt{-g}=-\tfrac{1}{2}\sqrt{-g}g_{\mu\nu}\delta g^{\mu\nu}$, volume term variation gives Einstein tensor $G_{\mu\nu}$, boundary term variation under fixed $h_{ab}$ condition organizes into combination of extrinsic curvature and $\delta h_{ab}$. Standard derivation shows
$$
\delta S_{\mathrm{GHY}}=\frac{1}{16\pi G}\int_{\partial M}\mathrm{d}^3y\,\epsilon\sqrt{h}\,(K^{ab}-Kh^{ab})\delta h_{ab}.
$$
Viewing $\delta S_{\mathrm{GHY}}$ as variation of Hamilton--Jacobi functional, obtain conjugate momentum $\pi^{ab}$ as stated in theorem.

Choosing timelike tangent vector field and unit normal on boundary, decomposing $h_{ab}$ into time and space parts, in ADM decomposition $K$ together with lapse function $N$ determine normal translation. Restricting $\delta h_{ab}$ to pure time rescaling (keeping spatial section shape unchanged), can write $\delta S_{\mathrm{HJ}}$ as quasilocal energy times time increment, obtaining $\delta S_{\mathrm{HJ}}=E_{\mathrm{q.l.}}\delta t_{\mathrm{geom}}$.

\subsection{Proof of Theorem 4}

On common domain, by assumption there exists scattering--geometry--modular flow correspondence:

\begin{enumerate}
\item[(1)] Scattering side gives family of frequency scales $\mathrm{d}\tau_{\mathrm{scatt}}(\omega)$, corresponding to external geometry through eikonal approximation and lensing/Shapiro delay;

\item[(2)] Modular flow side through JLMS-type relation or AdS--Rindler correspondence maps modular Hamiltonian to energy operator of geometric region, thus modular time linearly related to geometric time locally;

\item[(3)] Geometry side through Einstein equations and boundary conditions connects GHY action with extrinsic curvature, quasilocal energy, and time translation.
\end{enumerate}

Since scattering time scale determined by $\operatorname{tr}Q$ and $\partial_\omega\varphi$, while modular time and geometric time both scaled by boundary energy (modular Hamiltonian expectation value, quasilocal energy), and all three act on same boundary algebra--geometric structure, their scaling can only differ by positive finite factor, thus there exists fundamental scale $\tau$. Define $\tau$ as time parameter making three unit changes consistent in reference case (such as low-energy limit or fixed reference observer), i.e.,
$$
\delta\tau=\frac{\varphi'(\omega)}{\pi}\delta\omega=\frac{1}{2\pi}\operatorname{tr}Q(\omega)\delta\omega=\frac{1}{E_*}\delta S_{\mathrm{HJ}},
$$
where $E_*$ is reference energy under unit scale. For any other time parameter $t_{\mathrm{obs}}$, if requiring its unit interval consistent with above three readings, must have $\mathrm{d} t_{\mathrm{obs}}=\alpha\,\mathrm{d}\tau$, thus $t_{\mathrm{obs}}$ linearly equivalent to $\tau$.

Local uniqueness stems from: if there exists another parameter $\tilde{t}$ simultaneously satisfying consistency of scattering scale and energy scale, then $\mathrm{d}\tilde{t}/\mathrm{d}\tau$ locally non-zero and constant, thus $\tilde{t}$ locally only affine rescaling of $\tau$.

\subsection{Proof of Theorem 5}

In FRW metric conformal time satisfies $\mathrm{d}t=a(\eta)\mathrm{d}\eta$, therefore conformal time interval $\mathrm{d}\eta$ represents ``light travel time'' pulled back to flat metric. For high-frequency electromagnetic waves, eikonal approximation shows phase varies linearly with conformal time. Viewing cosmic signal (such as FRB or distant quasar) as scattering process from ``emission boundary'' to ``observation boundary'', frequency redshift $1+z=1/a(t_\mathrm{em})$ means in terms of source's proper time scale, observed frequency scaled by $1/(1+z)$ compared to local frequency.

If unified time scale $\tau$ defined as local observer's scattering scale, then between emission and observation ends, time scale needs global rescaling by scale factor $a(t)$, thus
$$
\mathrm{d}\tau_{\mathrm{cosmo}}=\frac{\mathrm{d}\varphi}{\pi}\cdot\frac{1}{\omega_{\mathrm{obs}}}=\frac{\mathrm{d}\varphi}{\pi}\cdot\frac{1}{\omega_{\mathrm{em}}/(1+z)}=(1+z)\,\mathrm{d}\tau_{\mathrm{local}},
$$
where $\mathrm{d}\varphi$ is phase change, $\omega_{\mathrm{em}},\omega_{\mathrm{obs}}$ are emission and observation frequencies. Therefore redshift can be interpreted as global rescaling of unified time scale equivalence class across cosmic light cone.

---

\section{Model Applications}

This section demonstrates concrete implementations of unified framework in several representative models.

\subsection{One-Dimensional Potential Scattering and Time Delay}

Consider one-dimensional Schrödinger operator on real line
$$
H_0=-\frac{\mathrm{d}^2}{\mathrm{d} x^2},\qquad H=H_0+V(x),
$$
where $V$ is compactly supported or rapidly decaying real potential. On energy shell $E=k^2$, S-matrix written as
$$
S(k)=e^{2i\delta(k)},
$$
where $\delta(k)$ is total phase shift. Theorem 1 reduces to
$$
\frac{\delta'(k)}{\pi}=\rho_{\mathrm{rel}}(k)=\frac{1}{2k}\frac{1}{2\pi}\operatorname{tr}Q(E),
$$
right side $Q(E)$ consistent with traditional Wigner time delay $\tau_{\mathrm{W}}(E)=2\partial_E\delta(E)$, time scale equivalence class can be directly defined as
$$
\mathrm{d}\tau_{\mathrm{scatt}}=\frac{\delta'(E)}{\pi}\,\mathrm{d}E.
$$
This scale can be cross-verified in finite volume box through Levinson theorem and eigenvalue counting function.

\subsection{Modular Time and Unruh Time in Rindler Wedge}

In Minkowski spacetime, consider right Rindler wedge region, algebra $M$ defined by accelerated trajectory observers with vacuum state $|0\rangle$. Tomita--Takesaki theory shows Rindler wedge modular flow corresponds to Rindler time translation, its modular Hamiltonian proportional to acceleration generator. Unruh effect gives temperature $T=a/2\pi$ relationship with acceleration $a$, thus modular parameter $t_{\mathrm{mod}}$ and accelerated observer proper time $\tau$ satisfy $\tau=\alpha t_{\mathrm{mod}}$, where $\alpha\propto\beta=1/T$.

In this case, gravitational geometry can be viewed through equivalence principle as local black hole horizon geometry under constant surface gravity, GHY boundary term defines quasilocal energy proportional to surface gravity, thus Rindler time, modular time, and geometric time fall into same scale equivalence class.

\subsection{Cosmological Scattering in FRW Universe}

In FRW background, consider narrow-band electromagnetic wave emitted by distant quasar; for observer, this equivalent to scattering from early universe ``boundary'' to current cosmic epoch. Electromagnetic wave propagates in nearly conformally flat metric, eikonal phase accumulation proportional to conformal time. By redshift--scale factor relation $1+z=1/a(t_\mathrm{em})$, observation frequency $\omega_{\mathrm{obs}}$ and emission frequency $\omega_{\mathrm{em}}$ satisfy $\omega_{\mathrm{obs}}=\omega_{\mathrm{em}}/(1+z)$.

If reconstructing FRW parameters using scattering angle (such as using frequency-dependent delay), unified time parameter of Wigner--Smith delay scale naturally absorbs redshift factor, global rescaling in Theorem 5 specifically manifests as $\mathrm{d}\tau_{\mathrm{cosmo}}=(1+z)\mathrm{d}\tau_{\mathrm{local}}$.

---

\section{Engineering Proposals}

Important value of unified framework is giving engineered schemes for cross-calibrating time scales across multiple platforms:

\subsection{Group Delay Measurement in Microwave Cavities and Waveguides}

In radio frequency and microwave bands, can measure complex S-parameters through vector network analyzer, thus directly constructing numerical approximation of Wigner--Smith operator
$$
Q(\omega)=-iS(\omega)^\dagger\partial_\omega S(\omega).
$$
Through frequency finite difference approximation $\partial_\omega S$, taking trace over multiple ports obtains group delay trace scale $(2\pi)^{-1}\operatorname{tr}Q(\omega)$.

In cavities containing controllable reflection and coupling loss, can design effective media approaching Rindler or FRW effective geometry (such as gradient refractive index structures), making measured group delay simultaneously reflect scattering and geometric effects, experimentally verifying stability of scale identity.

\subsection{Aharonov--Bohm Ring and Phase--Delay Unification}

In one-dimensional AB ring plus local scattering center system, magnetic flux $\Phi_{\mathrm{AB}}$ controls geometric phase on ring, while local potential scattering controls frequency-dependent phase and group delay. Through scanning frequency and flux, can measure total phase derivative and time delay, fitting out unified time scale $\tau$. Tuning ring effective length and applied potential to simulate Rindler or Schwarzschild effective potential barrier, hopefully realizing ``gravity--scattering--modular time'' unified scale experiment on condensed matter platform.

\subsection{Cosmological Signals and FRB Observations}

On cosmological scales, Fast Radio Bursts (FRB) and high-energy bursts provide ``natural scattering experiments'' across Gpc. Through multi-frequency observations, can measure dispersion delay and possibly existing phase/group delay structure, fitting with scale factor evolution under FRW model. Unified time scale framework can distinguish:

\begin{enumerate}
\item[(1)] Pure plasma dispersion causing $\nu^{-2}$ delay;

\item[(2)] Geometric--gravitational effect causing frequency-independent or weakly frequency-dependent delay;

\item[(3)] Potential vacuum dispersion or quantum gravity corrections.
\end{enumerate}

Although current data insufficient to precisely separate these effects, unified scale model provides theoretical benchmark for future fine observations.

---

\section{Discussion (Risks, Boundaries, Past Work)}

\subsection{Applicability Domain and Assumption Strength}

Scattering--spectral end requires trace-class conditions and wave operator completeness, verifiable in many concrete models, but needing generalized versions in strong long-range potentials, non-self-adjoint, or dissipative systems. Modular flow end requires existence of cyclic and separating vector, physically needing interpretation of ``observer--algebra--state'' selection. Gravity end depends on applicability of EH+GHY action, requiring corrected boundary terms in f(R) or more general higher-order gravitational theories.

\subsection{Choice Freedom in Unified Scale}

Definition of time scale equivalence class $[\tau]$ preserves affine rescaling freedom. This similar to different inertial frame time choices in relativity, ratio freedom between temperature reciprocal and time period in statistical mechanics. In unified framework, truly physically meaningful is relative scale within equivalence class, not some absolute unit.

\subsection{Relationship with Existing Work}

Relationships among scattering phase, spectral shift, and time delay deeply developed in scattering theory and statistical mechanics literature, modular flow and thermal time hypothesis extensively discussed in philosophy and quantum gravity fields, role of GHY boundary term and quasilocal energy fundamental tool in gravitational research. Contribution of this paper:

\begin{enumerate}
\item[(1)] Mathematically, using scale identity as link, align scattering--spectral structure with modular flow--information structure at boundary algebra level;

\item[(2)] In context of gravitational boundary term and quasilocal energy, incorporate geometric time into same scale, making it parameter of boundary Hamilton--Jacobi functional;

\item[(3)] In context of cosmological redshift and FRW conformal time, interpret macroscopic time evolution as global rescaling of unified time scale equivalence class.
\end{enumerate}

Related work includes research constructing state density and thermal functions using BK formula and spectral shift function, work connecting modular flow with geometric region energy and entanglement wedge reconstruction, and work on role of GHY boundary term in black hole thermodynamics.

\subsection{Risks and Open Problems}

\begin{enumerate}
\item[(1)] Rigorous AdS/CFT or scattering--geometry correspondence only holds under specific backgrounds and energy scales; unifying scattering matrix with boundary algebra in general curved spacetime may require more complex algebra--geometric tools.

\item[(2)] Although modular flow structure on type III factors mature, its physical interpretation still controversial, especially how to correspond modular time with experimental time in isolated systems without obvious heat bath.

\item[(3)] Unified scale interpretation of cosmological redshift currently more structural than giving new numerical predictions; to become testable theory, needs combining with more detailed FRB/GRB data and gravitational lensing effects.
\end{enumerate}

---

\section{Conclusion}

This paper under unified ``boundary time geometry'' framework integrates three seemingly independent time structures—scattering phase derivative and time delay, modular flow time in operator algebra, and geometric boundary time in gravity—as different realizations of boundary time scale equivalence class $[\tau]$. Core scale identity
$$
\frac{\varphi'(\omega)}{\pi}=\rho_{\mathrm{rel}}(\omega)=\frac{1}{2\pi}\operatorname{tr}Q(\omega)
$$
unifies phase structure of scattering matrix, state density information of spectral shift function, and time meaning of group delay as single boundary scale; modular flow and GHY boundary action respectively provide intrinsic scales of ``time generator'' at operator algebra and geometry ends. Introduction of cosmological redshift and FRW conformal time extends this unified scale from microscopic scattering to macroscopic cosmic evolution.

This framework emphasizes: time not independent parameter in bulk domain, but derived scale of boundary data—including algebra, state, and geometry. Future needs refining this unified structure in more concrete models (such as AdS black holes, quantum many-body systems, and experimental scattering networks), cross-testing it at engineering level through multi-platform group delay measurements and cosmological observations.

---

\section*{Acknowledgements, Code Availability}

This work relies only on public literature and theoretical derivations, using no specialized code. All example models and calculations can be reproduced using standard numerical software according to equations in text.

---

\begin{thebibliography}{99}

\bibitem{ref1}
A. Strohmaier, ``The Birman--Kreĭn formula for differential forms and scattering theory,'' \textit{Ann. Inst. H. Poincaré Anal. Non Linéaire}, 2022.

\bibitem{ref2}
H. Neidhardt, ``Scattering Matrix, Phase Shift, Spectral Shift and Trace Formula,'' \textit{Integr. Equ. Oper. Theory}, 2007.

\bibitem{ref3}
A. Strohmaier et al., ``Applications of Spectral Shift Functions. I: Basic Properties,'' Lecture Notes, 2016.

\bibitem{ref4}
V. Kostrykin, R. Schrader, ``The density of states and the spectral shift density of random Schrödinger operators,'' \textit{Rev. Math. Phys.}, 2000.

\bibitem{ref5}
U. R. Patel, ``Wigner--Smith Time Delay Matrix for Electromagnetics,'' arXiv:2005.03211, 2020.

\bibitem{ref6}
P. Ambichl et al., ``Focusing inside disordered media with the generalized Wigner--Smith operator,'' \textit{Phys. Rev. Lett.} \textbf{119}, 033903, 2017.

\bibitem{ref7}
M. Bugden, ``KMS States in Quantum Statistical Mechanics,'' Honours Thesis, ANU, 2013.

\bibitem{ref8}
``Tomita--Takesaki theory,'' Wikipedia entry, accessed 2025.

\bibitem{ref9}
T. T. Paetz, ``An Analysis of the `Thermal-Time Concept' of Connes and Rovelli,'' Diploma Thesis, 2010.

\bibitem{ref10}
A. Sorce, ``An intuitive construction of modular flow,'' arXiv:2309.16766, 2023.

\bibitem{ref11}
E. Bahiru et al., ``Explicit reconstruction of the entanglement wedge via the Petz map,'' \textit{JHEP}, 2022.

\bibitem{ref12}
``Gibbons--Hawking--York boundary term,'' Wikipedia entry, accessed 2025.

\bibitem{ref13}
S. Chakraborty, ``Boundary terms of the Einstein--Hilbert action,'' arXiv:1607.05986, 2016.

\bibitem{ref14}
D. Bavera, ``The Boundary Terms of the Einstein--Hilbert Action,'' Master Thesis, 2018.

\bibitem{ref15}
``Scale factor (cosmology),'' Wikipedia entry, accessed 2025.

\bibitem{ref16}
R. Wojtak et al., ``Testing the mapping between redshift and cosmic scale factor,'' \textit{Mon. Not. R. Astron. Soc.} \textbf{458}, 3331--3340, 2016.

\bibitem{ref17}
``Friedmann--Lemaître--Robertson--Walker metric,'' Wikipedia entry, accessed 2025.

\bibitem{ref18}
``Unruh effect,'' Wikipedia entry, accessed 2025.

\end{thebibliography}

\appendix

\section{Technical Proof of Scale Identity}

This appendix gives rigorous proof of Theorem 1 under standard scattering theory postulates.

\subsection{Technical Conditions of Spectral Shift Function and BK Formula}

Let $(H,H_0)$ satisfy

\begin{enumerate}
\item[(1)] For some $p\leq 1$, $(H-i)^{-p}-(H_0-i)^{-p}\in\mathfrak{S}_1$;

\item[(2)] For any Borel bounded function $f$ with bounded $f'$, can define
$$
\operatorname{tr}(f(H)-f(H_0))=\int f'(\lambda)\xi(\lambda)\,\mathrm{d}\lambda,
$$
thus introducing spectral shift function $\xi(\lambda)$.
\end{enumerate}

Birman--Kreĭn formula asserts: on absolutely continuous spectrum almost everywhere
$$
\det S(\lambda)=\exp(-2\pi i\xi(\lambda)).
$$
This equality requires S-matrix fiber to be unitary operator of trace-class perturbation, log branch choice maintaining continuity; in current setting can be rigorously proven through limit of finite volume approximation.

\subsection{Phase Derivative and State Density Difference}

Let $\Phi(\lambda)=\arg\det S(\lambda)$, choose branch making $\Phi(\lambda)$ continuous, define half-phase $\varphi(\lambda)=\tfrac{1}{2}\Phi(\lambda)$. By BK formula
$$
\Phi(\lambda)=-2\pi\xi(\lambda)\quad(\mathrm{mod}\,2\pi),
$$
removing mod $2\pi$ ambiguity after can differentiate in smooth interval, having
$$
\varphi'(\lambda)=-\pi\xi'(\lambda)=\pi\rho_{\mathrm{rel}}(\lambda).
$$
On other hand, spectral shift function derivative can be given through limit of eigenvalue counting function difference in finite box approximation, thus $\rho_{\mathrm{rel}}$ definition independent of specific S-matrix implementation, depending only on operator pair $(H,H_0)$.

\subsection{Wigner--Smith Operator Trace and Spectral Shift Density}

In energy representation, $S(\lambda)$ is family of unitary operators on $\mathcal{H}_\lambda$, whose logarithmic derivative satisfies
$$
\partial_\lambda\ln\det S(\lambda)=\operatorname{tr}(S(\lambda)^{-1}\partial_\lambda S(\lambda)).
$$
Since $\det S(\lambda)=\exp(-2\pi i\xi(\lambda))$,
$$
\partial_\lambda\ln\det S(\lambda)=-2\pi i\xi'(\lambda)=2\pi i\rho_{\mathrm{rel}}(\lambda).
$$
On other hand, defining Wigner--Smith operator
$$
Q(\lambda)=-iS(\lambda)^\dagger\partial_\lambda S(\lambda),
$$
then
$$
\operatorname{tr}(S(\lambda)^{-1}\partial_\lambda S(\lambda))=\operatorname{tr}(S(\lambda)^\dagger\partial_\lambda S(\lambda))=i\operatorname{tr}Q(\lambda).
$$
Comparing yields $\rho_{\mathrm{rel}}(\lambda)=(2\pi)^{-1}\operatorname{tr}Q(\lambda)$.

\subsection{One-Dimensional Case and Levinson Theorem}

Under one-dimensional short-range potential, eigenstates in finite box approximation satisfy $k_nR+\delta(k_n)=n\pi$. Changing potential or box length, eigenvalue density difference can be expressed as phase shift derivative, thus
$$
\rho_{\mathrm{rel}}(k)=\frac{1}{\pi}\delta'(k).
$$
Levinson theorem gives $\delta(0)-\delta(\infty)=\pi N_{\mathrm{bound}}$ and other boundary conditions, thus unifying scattering phase and bound state counting. This construction compatible with spectral shift function definition, providing concrete implementation of scale identity in one-dimensional models.

---

\section{Modular Flow, KMS Conditions, and Thermal Time}

\subsection{Tomita--Takesaki Construction}

Starting from $M$ and $\Omega$, define densely defined antilinear operator
$$
S_0:m\Omega\mapsto m^\ast\Omega,\quad m\in M.
$$
Closure $S$ admits polar decomposition $S=J\Delta^{1/2}$, where $J$ is antilinear isometric modular conjugation, $\Delta$ is positive, self-adjoint modular operator. Modular flow defined as
$$
\sigma_t^\omega(m)=\Delta^{it}m\Delta^{-it}.
$$
Tomita--Takesaki theorem asserts: $\sigma_t^\omega$ is one-parameter automorphism group of $M$, and $\omega$ satisfies KMS condition for it, i.e., there exists analytic function in strip region such that
$$
F(t)=\omega(a\sigma_t^\omega(b)),\quad F(t+i)=\omega(\sigma_t^\omega(b)a).
$$

\subsection{Connes 1--Cocycle and State-Independent Time}

Given two faithful states $\omega,\omega'$, can construct Connes 1--cocycle $u_t$ such that
$$
\sigma_t^{\omega'}(m)=u_t\sigma_t^\omega(m)u_t^{-1},
$$
with $u_{t+s}=u_t\sigma_t^\omega(u_s)$. This means modular flow's image in outer automorphism group $\mathrm{Out}(M)=\operatorname{Aut}(M)/\operatorname{Inn}(M)$ independent of state, only defining ``geometric time'' direction.

\subsection{Thermal Time Hypothesis and Temperature--Time Relation}

In traditional quantum statistics, given Hamiltonian $H$ and inverse temperature $\beta$, KMS state under time evolution $\alpha_t(a)=e^{itH}ae^{-itH}$ satisfies
$$
\omega(a\alpha_t(b))=\omega(\alpha_{t+i\beta}(b)a).
$$
Thermal time hypothesis reverses this logic: first given state $\omega$ and algebra $M$, then interpret modular flow $\sigma_t^\omega$ parameter as time. If external observer's physical time $\tau_{\mathrm{phys}}$ exists, can obtain $\tau_{\mathrm{phys}}=\alpha t_{\mathrm{mod}}$ by comparing modular flow with physical Hamiltonian generated evolution, where $\alpha$ given by temperature or acceleration. Unruh effect's $T=a/2\pi$ provides concrete example of this proportion.

---

\section{GHY Boundary Term, Hamilton--Jacobi Functional, and Quasilocal Time}

\subsection{Variation of EH+GHY Action}

Starting from
$$
S_{\mathrm{grav}}=\frac{1}{16\pi G}\int_M\sqrt{-g}\,R\,\mathrm{d}^4x+\frac{1}{8\pi G}\int_{\partial M}\epsilon\sqrt{h}\,K\,\mathrm{d}^3y,
$$
vary with respect to $g_{\mu\nu}$. EH term variation can be written as sum of volume integral and boundary integral, latter exactly canceled by GHY term variation, thus under $\delta h_{ab}=0$ condition, total variation only contains volume integral, giving Einstein equations.

\subsection{Hamilton--Jacobi Functional and Conjugate Momentum}

Viewing $S_{\mathrm{HJ}}[h_{ab}]$ as function of action obtained by solving Einstein equations under given boundary geometry, variation with respect to $h_{ab}$ gives conjugate momentum
$$
\pi^{ab}=\frac{\delta S_{\mathrm{HJ}}}{\delta h_{ab}}=\frac{\epsilon}{16\pi G}\sqrt{h}\,(K^{ab}-Kh^{ab}).
$$
In ADM decomposition, metric written as
$$
\mathrm{d}s^2=-N^2\mathrm{d}t^2+h_{ij}(\mathrm{d}x^i+N^i\mathrm{d}t)(\mathrm{d}x^j+N^j\mathrm{d}t),
$$
extrinsic curvature determined through lapse $N$ and shift $N^i$. Choosing appropriate gauge (such as $N^i=0$), normal time translation corresponds to change of $t$, therefore $\delta S_{\mathrm{HJ}}/\delta t$ gives quasilocal energy.

\subsection{Rindler and Black Hole Cases}

In Rindler wedge or Schwarzschild black hole exterior, near-horizon region extrinsic curvature proportional to surface gravity, GHY boundary term in Euclidean path integral gives black hole free energy and temperature relation. By pairing Euclidean time period $\beta$ with modular flow period, can show geometric time, modular time, and thermal time belong to same scale equivalence class.

---

\section{Categorical Structure of Time Scale Equivalence Class}

\subsection{Objects and Morphisms}

Construct category $\mathbf{BTG}$:

\begin{enumerate}
\item[(1)] Objects are boundary systems $\mathcal{B}=(\mathcal{A}_\partial,\omega_\partial,S;h_{ab},K_{ab})$;

\item[(2)] Morphisms are mappings $\Phi:\mathcal{B}_1\to\mathcal{B}_2$ preserving physical structure, satisfying:

$\bullet$ $\Phi$ gives $\ast$--isomorphism $\phi:\mathcal{A}_{\partial,1}\to\mathcal{A}_{\partial,2}$ on algebra;

$\bullet$ $\Phi$ maps state and S-matrix to objects preserving BK and scale identity structure;

$\bullet$ $\Phi$ maps boundary geometry to embedding preserving metric and extrinsic curvature (or their equivalence class).
\end{enumerate}

In this category, time scale equivalence class $[\tau]$ can be viewed as functor from objects to $\mathbb{R}$, assigning each boundary system set of time parameters, morphisms corresponding to monotonic rescaling of time scales.

\subsection{Categorical Statement of Local Uniqueness}

Theorem 4 can be restated as: in some local subcategory of given object $\mathcal{B}$, if requiring functor $T:\mathbf{BTG}\to\mathbf{Time}$ simultaneously preserve unit intervals of scattering scale, modular time, and geometric time, then unique in sense of natural isomorphism.

This provides basis for future connection of time scale equivalence class with higher-level categorical structures (such as fibration, natural transformation), but these extensions beyond scope of this paper.

\end{document}
