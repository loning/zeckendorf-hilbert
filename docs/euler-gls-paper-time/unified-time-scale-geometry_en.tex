\documentclass[12pt]{article}

% Essential packages
\usepackage[utf8]{inputenc}
\usepackage[T1]{fontenc}
\usepackage{amsmath,amssymb,amsthm}
\usepackage{mathrsfs}
\usepackage{geometry}
\usepackage{hyperref}
\usepackage{braket}
\usepackage{graphicx}

% Geometry settings
\geometry{a4paper, margin=1in}

% Hyperref settings
\hypersetup{
    colorlinks=true,
    linkcolor=blue,
    citecolor=blue,
    urlcolor=blue
}

% Theorem environments
\theoremstyle{plain}
\newtheorem{theorem}{Theorem}[section]
\newtheorem{lemma}[theorem]{Lemma}
\newtheorem{proposition}[theorem]{Proposition}
\newtheorem{corollary}[theorem]{Corollary}

\theoremstyle{definition}
\newtheorem{definition}{Definition}[section]
\newtheorem{example}[theorem]{Example}
\newtheorem{remark}[theorem]{Remark}
\newtheorem{axiom}{Axiom}[section]

% Title information
\title{Unified Time Scale and Time Geometry:\\
Causal Ordering, Unitary Evolution, and Generalized Entropy}

\author{Haobo Ma$^1$ \and Wenlin Zhang$^2$\\
\small $^1$Independent Researcher\\
\small $^2$National University of Singapore}

\date{\today}

\begin{document}

\maketitle

\begin{abstract}
We propose a ``unified time scale equivalence class'' rigorously gluing three ends of relativity, quantum scattering, and information--holography. Core scale identity unifies derivative of total scattering phase, relative state density, and trace of Wigner--Smith group delay as different projections of same object:

$$
\boxed{\ \frac{\varphi'(\omega)}{\pi}\;=\;\rho_{\mathrm{rel}}(\omega)\;=\;\frac{1}{2\pi}\operatorname{tr}Q(\omega)\ },\qquad
Q(\omega)=-\,iS(\omega)^\dagger\partial_\omega S(\omega),\ \varphi=\tfrac{1}{2}\arg\det S.
$$

In geometry end, Killing time, ADM lapse, null geodesic affine parameter, and FRW conformal time proven mutually rescalable within unified equivalence class; in information--holography end, taking Tomita--Takesaki modular flow as ``intrinsic time,'' controlling generalized entropy extremality on small causal diamonds with QFC/QNEC and relative entropy monotonicity, thus deriving Einstein equations in semiclassical--holographic window. Framework obtains three alignments: (i) phase--proper time equivalence $\phi=(mc^2/\hbar)\int \mathrm{d}\tau$; (ii) gravitational time delay = group delay trace $\Delta T=\partial_\omega\Phi=\operatorname{Tr}Q$; (iii) FRW redshift = phase rhythm ratio $1+z=a(t_0)/a(t_e)=[(\mathrm{d}\phi/\mathrm{d}t)_e]/[(\mathrm{d}\phi/\mathrm{d}t)_0]$. This paper under three axioms of \textbf{causal ordering--unitary evolution--entropy monotonicity/extremality}, establishes existence and affine uniqueness of unified time scale, giving realization schemes for experimental and engineering metrology.
\end{abstract}

\noindent\textbf{Keywords:} Time Geometry; Unified Time Scale; Wigner--Smith Group Delay; Spectral Shift Function; Killing/ADM/Null/Conformal/Modular Time; Generalized Entropy; QFC/QNEC

\noindent\textbf{MSC 2020:} 83C45, 81U40, 81T20, 83C57

---

\section{Introduction and Historical Context}

Role of time splits in different theories: general relativity scales causal structure with proper time, quantum theory generates unitary evolution with external parameter, information--holography views modular flow as intrinsic ``thermal time.'' Yet when three end readings ``synchronize,'' still lacking rigorous and metrologically measurable common scale. Wigner and Smith introduced derivative of phase with respect to energy in scattering theory defining time delay, trace of Wigner--Smith group delay matrix $Q=-iS^\dagger\partial_\omega S$ equals derivative of total phase $\Phi=\arg\det S$, making idea of ``time = phase gradient'' first land on experimentally readable ruler. Birman--Kreĭn formula characterizing state density change caused by interaction with spectral shift function tightly connects phase of scattering determinant with spectral geometry, thus deriving ``phase derivative = relative state density.''

In relativity side, redshift/clock rate in static spacetime given by $g_{tt}$ or Tolman--Ehrenfest law; lapse $N$ in ADM $(3{+}1)$ decomposition characterizes ratio of coordinate time to proper time; tortoise coordinates and $(u,v)$ in asymptotically flat exterior provide natural null time at infinity; in FRW cosmology $1+z=a(t_0)/a(t_e)$ linearizes null geodesics with conformal time.

In information--holography side, Tomita--Takesaki modular theory endows any (state, algebra) pair with family of intrinsic one-parameter automorphisms (modular flow); Connes--Rovelli thermal time hypothesis views this modular flow as physical time candidate; relative entropy monotonicity, QFC, and QNEC bind changes of generalized entropy together with stress tensor constraints, connecting to field equations and reconstruction in small causal diamond limit.

Above historical threads suggest: \textbf{unifying ``phase--group delay--spectral shift'' with ``clock rate--redshift--affine/conformal time'' and ``modular time--generalized entropy'' into single scale}, hopeful to obtain cross-scale time--geometry framework.

---

\section{Model and Assumptions}

\textbf{(A) Causal and Global Structure}
Let $(M,g)$ be stably causal Lorentzian manifold; under global hyperbolicity condition there exist smooth time function and smooth decomposition $M\cong \mathbb{R}\times\Sigma$.

\textbf{(B) Scattering and Spectral Shift}
On absolutely continuous spectral energy window $I\subset\mathbb{R}$, scattering matrix $S(\omega)$ unitary and smooth; define total phase $\Phi=\arg\det S$, group delay $Q=-iS^\dagger\partial_\omega S$. There exists spectral shift function $\xi(\omega)$ and relative state density $\rho_{\mathrm{rel}}=-\xi'$; Birman--Kreĭn formula $\det S(\omega)=\exp[-2\pi i\,\xi(\omega)]$ holds.

\textbf{(C) Unified Scale Identity (Core Assumption)}

$$
\boxed{\ \frac{\varphi'(\omega)}{\pi}=\rho_{\mathrm{rel}}(\omega)=\frac{1}{2\pi}\operatorname{tr}Q(\omega)\ },\qquad
\varphi=\tfrac{1}{2}\Phi .
$$

\textbf{(D) Boundary Entropy and Modular Flow}
Take small causal diamond $D_{p,r}$ through point $p$, null generator affine parameter $\lambda$; generalized entropy

$$
S_{\rm gen}(\lambda)=\frac{\mathrm{Area}(\Sigma_\lambda)}{4G\hbar}+S_{\rm out}(\lambda)
$$

satisfies QFC/QNEC type inequalities and relative entropy monotonicity; modular Hamiltonian $K=-\ln\rho$ generates modular flow $\sigma_s$.

\textbf{(E) Unitary Evolution}
On state space $\mathcal{H}$ there exists strongly continuous unitary group $U(t)=e^{-iHt}$; semiclassical worldline limit can relate $t$ and $\tau$ through phase density (see Section 4.1).

---

\section{Unified Time Scale: Definition and Three Axioms}

\subsection{Unified Time Scale Equivalence Class}

\begin{definition}[Unified Time Scale (Definition 3.1)]
There exists equivalence class
$$
[T]\sim\{\tau,\ t,\ t_{\rm K},\ (N,N^i),\ \lambda_{\rm null},\ u,v,\ \eta,\ \omega^{-1},\ z,\ s_{\rm mod}\},
$$
whose members mutually convertible through monotonic rescaling and geometric/entropy structure, making dynamics local, causally ordered, and entropy structure simplest.
\end{definition}

\subsection{Three Axioms}

\begin{axiom}[Causal Ordering (Axiom I)]
In local hyperbolic domain there exists strictly increasing time function, making fundamental equations local (hyperbolic/first-order) form.
\end{axiom}

\begin{axiom}[Unitary Evolution (Axiom II)]
There exists strongly continuous unitary group $U(t)$; in semiclassical limit phase--time relation determined by Lagrangian stationary phase (Section 4.1).
\end{axiom}

\begin{axiom}[Entropy Monotonicity/Extremality (Axiom III)]
Along null cut family $\{\Sigma_\lambda\}$, $S_{\rm gen}$ satisfies relative entropy monotonicity and QFC/QNEC monotonicity/convexity and takes extremum under physical evolution; modular flow parameter $s$ makes organization law of $S_{\rm gen}$ simplest.
\end{axiom}

\begin{theorem}[Mutual Implication in Semiclassical--Holographic Window (Theorem 3.2)]
Under small causal diamond limit and relative entropy monotonicity/QNEC holding:
$$
\text{Axiom I}+\text{Axiom II}\ \Longleftrightarrow\ \text{Scale Identity}\ \Longrightarrow\ \text{Axiom III}\ \Longrightarrow\ \text{Einstein Equations}.
$$
\end{theorem}

Proof in Section 5 and Appendices D/E.

---

\section{Main Results (Theorems and Alignments)}

\subsection{Phase--Proper Time Equivalence}

\begin{theorem}[Worldline Principal Phase (Theorem 4.1)]
For narrow wave packet of mass $m$, in semiclassical limit
$$
\phi=-\frac{1}{\hbar}S[\gamma_{\rm cl}]=\frac{mc^2}{\hbar}\int_{\gamma_{\rm cl}}\mathrm{d}\tau,\qquad \frac{\mathrm{d}\phi}{\mathrm{d}\tau}=\frac{mc^2}{\hbar}.
$$
\end{theorem}

(Proof: worldline path integral stationary phase; see Appendix B.)

\subsection{Gravitational Time Delay = Group Delay Trace}

\begin{theorem}[Eikonal--Scattering Alignment (Theorem 4.2)]
In geometric optics limit of static or asymptotically flat background,
$$
\Delta T(\omega)=\partial_\omega\Phi(\omega)=\operatorname{Tr}Q(\omega).
$$
Weak field limit returns to Shapiro delay.
\end{theorem}

\subsection{Redshift = Phase Rhythm Ratio}

\begin{proposition}[FRW Phase Expression (Proposition 4.3)]
Under flat FRW metric comoving observers measure
$$
1+z=\frac{\nu_e}{\nu_0}=\frac{\left.\frac{\mathrm{d}\phi}{\mathrm{d}t}\right|_{e}}{\left.\frac{\mathrm{d}\phi}{\mathrm{d}t}\right|_{0}}=\frac{a(t_0)}{a(t_e)}.
$$
\end{proposition}

(See Appendix C.)

\subsection{Four ``Bridges'' of GR Time Structure}

\textbf{Bridge B (Killing Time--Clock Rate--Redshift)}
In static metric $\mathrm{d}s^2=-V(\mathbf{x})c^2\mathrm{d}t^2+\cdots$ stationary observers have
$\mathrm{d}\tau=\sqrt{V}\,\mathrm{d}t$, $\sqrt{V}$ is local redshift/clock rate factor (Tolman--Ehrenfest).

\textbf{Bridge C (ADM Lapse--Local Clock Rate)}
ADM decomposition $\mathrm{d}s^2=-N^2 \mathrm{d}t^2+h_{ij}(\mathrm{d}x^i+N^i \mathrm{d}t)(\mathrm{d}x^j+N^j \mathrm{d}t)$; Euler family orthogonal to slicing satisfies $\mathrm{d}\tau=N\,\mathrm{d}t$.

\textbf{Bridge D (Null Affine Parameter--Retarded/Advanced/Conformal Time)}
Asymptotically flat exterior defines tortoise $r^{*}$ and $u=t-r^{*},\ v=t+r^{*}$, monotonically equivalent to null geodesic affine parameter; in FRW $\mathrm{d}\eta=\mathrm{d}t/a(t)$ linearizes null geodesics.

\textbf{Bridge E (Modular Time--Entropy Gradient--Geometric Equations)}
Parameter $s$ of modular flow $\sigma_s$ provides information-theoretic time; relative entropy monotonicity and QNEC/QFC bind extremality/monotonicity of $\partial_s S_{\rm gen}$ with $\langle T_{kk}\rangle$.

\subsection{Entropic Geometric Form of Einstein Equations}

\begin{theorem}[Entropy--Geometry (Theorem 4.4)]
Under Axiom III and Raychaudhuri equation, on small causal diamond have
$$
R_{\mu\nu}-\tfrac{1}{2}Rg_{\mu\nu}+\Lambda g_{\mu\nu}=8\pi G\,\langle T_{\mu\nu}\rangle .
$$
\end{theorem}

(Proof: combining second-order area variation with QNEC/relative entropy, see Appendix D; cf. Jacobson and subsequent holographic arguments.)

\subsection{Unified Scale Identity (Spectral--Scattering--Geometry)}

\begin{corollary}[Corollary 4.5]
$$
\boxed{\ \frac{\varphi'(\omega)}{\pi}=\rho_{\mathrm{rel}}(\omega)=\frac{1}{2\pi}\operatorname{tr}Q(\omega)\ } ,
$$
obtained by combining Birman--Kreĭn and $\operatorname{Tr}Q=\partial_\omega\Phi$ (Appendix A).
\end{corollary}

---

\section{Proofs (Key Points)}

\textbf{5.1  Theorem 4.1:} Worldline action stationary phase along timelike geodesic, fluctuations only change quantum prefactor, principal phase gives $\phi=(mc^2/\hbar)\int \mathrm{d}\tau$ (Appendix B).

\textbf{5.2  Theorem 4.2:} Eikonal phase difference $\Delta S\simeq-\hbar\,\omega\,\Delta T$ aligns with scattering $\partial_\omega\Phi=\operatorname{Tr}Q$, yielding $\Delta T=\operatorname{Tr}Q$; weak field test returns to Shapiro delay.

\textbf{5.3  Proposition 4.3:} Null geodesic and scale factor give $\nu\propto 1/a(t)$, redshift as phase rhythm ratio (Appendix C).

\textbf{5.4  Bridges B--E:} Static clock rate, ADM lapse, null coordinates, and modular flow each consistent with standard conclusions and literature (Appendix E).

\textbf{5.5  Theorem 4.4:} Raychaudhuri's second-order area variation $\propto-\int R_{kk}$ combines with QNEC $S''_{\rm out}\ge (2\pi/\hbar)\!\int\!\langle T_{kk}\rangle$, extremality condition $S'_{\rm gen}(0)=0$ yields tensor equation (Appendix D); QFC provides stronger monotonicity background.

---

\section{Model Applications}

\textbf{A. Frequency-Domain Reconstruction of Solar System Geometric Delay}
Differentiating multi-frequency radar echo phase $\Phi(\omega)$ gives $\Delta T(\omega)=\partial_\omega\Phi=\operatorname{Tr}Q$, compare with Shapiro delay, can parallelly strip plasma dispersion term.

\textbf{B. Phase--Group Delay Unification of Gravitational Lensing}
Image pair $(i,j)$ Fermat potential difference $\Delta t_{ij}$ equals $\partial_\omega[\Phi_i-\Phi_j]$; broadband electromagnetic/gravitational wave joint use for Hubble constant and mass model systematic error suppression.

\textbf{C. Cosmological ``Phase Ruler''}
Directly estimate $a(t)$ using phase rhythm ratio of pulsar/FRB, avoiding specific spectral line systematics; ``phase--redshift'' synchronization guaranteed by Proposition 4.3.

\textbf{D. Effective ``Time Refractive Index'' Tomography}
Invert $n_t=(-g_{tt})^{-1/2}$ from spatial distribution of $\rho_{\mathrm{rel}}(\omega)$ or $\operatorname{tr}Q(\omega)$, cooperating with optical metric--Fermat principle for weak field time delay imaging.

---

\section{Engineering Proposals}

\begin{enumerate}
\item \textbf{On-chip group delay tomography metrology:} Measure $S(\omega)$ in integrated photonics and real-time compute $\operatorname{Tr}Q(\omega)$, generating equivalent ``gravitational time delay'' map for device inversion and robust design.

\item \textbf{Double-height matter wave standard:} COW geometric arrangement compare $\Delta\phi$ and $\Delta\tau$, test linear regime of $\Delta\phi=(mc^2/\hbar)\Delta\tau$.

\item \textbf{Broadband lens group delay spectrum:} Synchronously fit each image arrival time delay and dispersion using $\partial_\omega\Phi$, reducing time delay cosmology systematic errors.

\item \textbf{``Entropy light cone'' platform:} Measure second-order deformation and energy flow of $S_{\rm out}$ on controllable quantum system, test QNEC/QFC coefficients and saturation conditions.
\end{enumerate}

---

\section{Discussion (Risks, Boundaries, Past Work)}

\textbf{(i) Spectral endpoints and regularity:} Scale identity requires $S(\omega)$ differentiable and belonging to appropriate determinant class; near resonances and thresholds need contour displacement and trace-class regularization.

\textbf{(ii) Geometric optics and strong fields:} Strong spin/non-static backgrounds need generalized optical metric and coherent transport; near-horizon regions better use null coordinates and numerical ray tracing.

\textbf{(iii) Entropy--geometry assumption domain:} QNEC has general QFT proofs and holographic proofs continuously strengthening (including latest new proof approaches), but still advancing in high curvature, strong quantum gravity regions.

\textbf{(iv) GR time structure unification:} Killing/ADM/null/conformal/modular times are coordinatizations of unified scale in different projections; Bernal--Sánchez's global time functions and ADM foliation provide rigorous foundation.

---

\section{Conclusion}

Under three axioms of \textbf{causal ordering--unitary evolution--entropy monotonicity/extremality}, obtain unified time scale equivalence class, aligning microscopic (phase/scattering), mesoscopic (group delay/redshift), and macroscopic (entropy--geometry) three-end languages. Core conclusions:

$$
\phi=\frac{mc^2}{\hbar}\!\int \mathrm{d}\tau,\quad
\Delta T(\omega)=\partial_\omega\Phi(\omega)=\operatorname{Tr}Q(\omega),\quad
1+z=\frac{(\mathrm{d}\phi/\mathrm{d}t)_e}{(\mathrm{d}\phi/\mathrm{d}t)_0},\quad
G_{\mu\nu}+\Lambda g_{\mu\nu}=8\pi G\,\langle T_{\mu\nu}\rangle,
$$

and spectral--scattering--geometry scale identity $\varphi'/\pi=\rho_{\mathrm{rel}}=(2\pi)^{-1}\operatorname{tr}Q$. Time thus can be characterized as: \textbf{equivalence class of one-dimensional parameter making dynamics local, causality clear, entropy structure simplest}; its different ``names'' merely coordinates of same object in different projections.

---

\section*{Acknowledgements, Code Availability}

Thanks to related textbooks and literature. Symbolic derivations and numerical scripts for group delay--time delay reconstruction and FRW phase rhythm demonstration available upon request.

---

\begin{thebibliography}{99}

\bibitem{ref1}
E. P. Wigner, ``Lower Limit for the Energy Derivative of the Scattering Phase Shift,'' \textit{Phys. Rev.} \textbf{98} (1955) 145.

\bibitem{ref2}
F. T. Smith, ``Lifetime Matrix in Collision Theory,'' \textit{Phys. Rev.} \textbf{118} (1960) 349.

\bibitem{ref3}
A. Strohmaier and A. Waters, ``The Birman--Krein formula for differential forms and electromagnetic scattering,'' arXiv:2104.13589.

\bibitem{ref4}
A. N. Bernal and M. Sánchez, ``Smoothness of time functions and the metric splitting of globally hyperbolic spacetimes,'' \textit{Commun. Math. Phys.} \textbf{257} (2005) 43.

\bibitem{ref5}
É. Gourgoulhon, \textit{3+1 Formalism and Bases of Numerical Relativity}, Springer (2012); lecture notes (2007).

\bibitem{ref6}
I. I. Shapiro, ``Fourth Test of General Relativity,'' \textit{Phys. Rev. Lett.} \textbf{13} (1964) 789; see also updates (1971).

\bibitem{ref7}
D. W. Hogg, ``Distance measures in cosmology,'' arXiv:astro-ph/9905116 (2000).

\bibitem{ref8}
S. J. Summers, ``Tomita--Takesaki Modular Theory,'' \textit{Encycl. Math. Phys.} (2006).

\bibitem{ref9}
A. Connes and C. Rovelli, ``Von Neumann Algebra Automorphisms and Time--Thermodynamics Relation in General Covariant Quantum Theories,'' \textit{Class. Quant. Grav.} \textbf{11} (1994) 2899.

\bibitem{ref10}
R. Bousso, Z. Fisher, S. Leichenauer, A. C. Wall, ``Quantum Focusing Conjecture,'' \textit{Phys. Rev. D} \textbf{93} (2016) 064044.

\bibitem{ref11}
R. Bousso, Z. Fisher, J. Koeller, S. Leichenauer, A. C. Wall, ``Proof of the Quantum Null Energy Condition,'' \textit{Phys. Rev. D} \textbf{93} (2016) 024017.

\bibitem{ref12}
S. Balakrishnan, T. Faulkner, Z. U. Khandker, H. Wang, ``A General Proof of the QNEC,'' \textit{JHEP} \textbf{09} (2019) 020.

\bibitem{ref13}
D. L. Jafferis, A. Lewkowycz, J. Maldacena, S. J. Suh, ``Relative entropy equals bulk relative entropy,'' \textit{JHEP} \textbf{06} (2016) 004.

\bibitem{ref14}
N. Engelhardt, A. C. Wall, ``Quantum Extremal Surfaces,'' \textit{JHEP} \textbf{01} (2015) 073.

\bibitem{ref15}
T. Jacobson, ``Thermodynamics of Spacetime: The Einstein Equation of State,'' \textit{Phys. Rev. Lett.} \textbf{75} (1995) 1260.

\bibitem{ref16}
T. Faulkner, R. G. Leigh, O. Parrikar, H. Wang, ``Modular Hamiltonians for Deformed Half-Spaces and the ANEC,'' \textit{JHEP} \textbf{09} (2016) 038.

\bibitem{ref17}
T. Hartman, S. Kundu, A. Tajdini, ``Averaged Null Energy Condition from Causality,'' \textit{JHEP} \textbf{07} (2017) 066.

\bibitem{ref18}
V. Perlick, \textit{Ray Optics, Fermat's Principle, and Applications to GR}, Springer (2000).

\bibitem{ref19}
Scholarpedia, ``Bondi--Sachs Formalism'' (for retarded time $u$ and advanced time $v$).

\end{thebibliography}

\appendix

\section{Wigner--Smith Group Delay and Spectral--Scattering--Geometry Identity}

\subsection{Birman--Kreĭn and Spectral Shift}

For self-adjoint pair $(H,H_0)$ with trace-class/quasi-trace-class perturbation, spectral shift function $\xi(\omega)$ satisfies
$$
\det S(\omega)=e^{-2\pi i\xi(\omega)}\Rightarrow
\frac{1}{2\pi}\partial_\omega\Phi(\omega)= -\xi'(\omega)=\rho_{\mathrm{rel}}(\omega).
$$
(See reference 3.)

\subsection{Trace Identity}

From $Q(\omega)=-iS^\dagger\partial_\omega S$ and $\operatorname{Tr}\ln S=\ln\det S$ obtain
$$
\operatorname{Tr}Q(\omega)=\partial_\omega\Phi(\omega).
$$
Combining A.1 gives scale identity:
$$
\boxed{\ \frac{\varphi'(\omega)}{\pi}=\rho_{\mathrm{rel}}(\omega)=\frac{1}{2\pi}\operatorname{tr}Q(\omega)\ } .
$$

\section{``Phase--Proper Time'' Under Worldline Path Integral}

Fermi normal coordinate expansion along timelike geodesic $\gamma_{\rm cl}$
$$
S[\gamma]=-mc^2\!\int \mathrm{d}\tau+\frac{m}{2}\!\int \mathrm{d}\tau\,\delta_{ij}\dot{y}^i\dot{y}^j+\cdots .
$$
Stationary phase gives
$$
\phi=-\frac{1}{\hbar}S[\gamma_{\rm cl}]
=\frac{mc^2}{\hbar}\!\int \mathrm{d}\tau,\qquad \frac{\mathrm{d}\phi}{\mathrm{d}\tau}=\frac{mc^2}{\hbar}.
$$

\section{Phase Expression of Redshift in FRW Cosmology}

Flat FRW: $\mathrm{d}s^2=-\mathrm{d}t^2+a(t)^2 \mathrm{d}\mathbf{x}^2$. Comoving observer $u^\mu=(1,0,0,0)$, photon eikonal phase $\phi$ has $k_\mu=\partial_\mu\phi$ and
$$
\nu=-\frac{1}{2\pi}k_\mu u^\mu=\frac{1}{2\pi}\frac{\mathrm{d}\phi}{\mathrm{d}t}\propto\frac{1}{a(t)} \Rightarrow
1+z=\frac{(\mathrm{d}\phi/\mathrm{d}t)_e}{(\mathrm{d}\phi/\mathrm{d}t)_0}=\frac{a(t_0)}{a(t_e)} .
$$
(Reference 7.)

\section{Generalized Entropy Extremality/Monotonicity and Field Equations}

Let $\{\Sigma_\lambda\}$ be null cut family through $p$. Raychaudhuri: $\dot{\theta}=-\tfrac{1}{2}\theta^2-\sigma^2-R_{kk}$. Second-order area variation
$\displaystyle \left.\mathrm{d}^2A/\mathrm{d}\lambda^2\right|_{0}\propto -\!\int R_{kk}\,\mathrm{d}A$.

QNEC and relative entropy monotonicity: $\displaystyle \left.\mathrm{d}^2S_{\rm out}/\mathrm{d}\lambda^2\right|_{0}\ge \frac{2\pi}{\hbar}\!\int\!\langle T_{kk}\rangle \mathrm{d}A$.

Extremum $S'_{\rm gen}(0)=0$ combines to give $R_{kk}=8\pi G\langle T_{kk}\rangle$, holding for any $k^\mu$, upgrades to tensor equation and gives $\Lambda$ as integration constant.

\section{Refinement of GR Time Bridges}

\textbf{E.1  Static Spacetime (Killing):} $\xi^\mu$ is timelike Killing vector, stationary observer $u^\mu=\xi^\mu/\sqrt{-\xi^2}$, if $g_{tt}=-V$ then $\mathrm{d}\tau=\sqrt{V}\,\mathrm{d}t$. (Tolman--Ehrenfest temperature redshift law same form.)

\textbf{E.2  ADM Lapse:} $\mathrm{d}s^2=-N^2 \mathrm{d}t^2+h_{ij}(\mathrm{d}x^i+N^i \mathrm{d}t)(\mathrm{d}x^j+N^j \mathrm{d}t)$; slicing orthogonal family satisfies $\mathrm{d}x^i+N^i \mathrm{d}t=0\Rightarrow \mathrm{d}\tau=N\,\mathrm{d}t$.

\textbf{E.3  Null Coordinates:} Schwarzschild exterior $r^*=r+2M\ln|r/2M-1|$, $u=t-r^*$, $v=t+r^*$; in FRW $\mathrm{d}\eta=\mathrm{d}t/a(t)$.

\textbf{E.4  Modular Time:} Given (algebra, state) pair $(\mathcal{A},\omega)$ GNS representation, modular flow $\sigma_s$ intrinsically defines time; under half-space and small deformations, $K$ localizes to $\int T_{kk}$, isomorphic with ANEC/QNEC, JLMS/relative entropy.

\section{Shapiro Delay and Group Delay}

Weak field Schwarzschild exterior
$$
\Delta t \simeq \frac{4GM}{c^3}\ln\frac{4r_E r_R}{b^2}+\cdots,
$$
consistent with frequency-domain measured $\partial_\omega\Phi=\operatorname{Tr}Q$; multi-frequency echo fitting can separate dispersion and geometric terms.

\section{Existence and Uniqueness of Unified Time Scale}

Given scattering data $(S(\omega))$ satisfying scale identity. Define
$$
t-t_0=\int_{\omega_0}^{\omega}\frac{1}{2\pi}\operatorname{Tr}Q(\tilde\omega)\,\mathrm{d}\tilde\omega
=\int_{\omega_0}^{\omega}\frac{\varphi'(\tilde\omega)}{\pi}\,\mathrm{d}\tilde\omega
=\int_{\omega_0}^{\omega}\rho_{\mathrm{rel}}(\tilde\omega)\,\mathrm{d}\tilde\omega .
$$
In non-degenerate frequency window derivative positive, $t(\omega)$ locally bijective; if another $\tilde{t}$ satisfies same condition, then $\tilde{t}=a t+b$ ($a>0$), giving affine uniqueness.

---

\subsection*{Unified Diagram (Summary)}

$$
\boxed{
\begin{array}{c}
[T]\sim\{\tau,\ t_{\rm K},\,N,\,\lambda,u,v,\,\eta,\,\omega^{-1},\,z,\,s_{\rm mod}\}\\[2pt]
\Downarrow\\[3pt]
\phi,\ \Phi(\omega)\ \Longleftrightarrow\ \operatorname{Tr}Q(\omega),\ \rho_{\mathrm{rel}}(\omega)\\[3pt]
\Longleftrightarrow\ 1+z\ \ (\text{phase rhythm ratio})\ \Longleftrightarrow\ \partial_s S_{\rm gen}\ (\text{extremality/monotonicity})\\[3pt]
\Longleftrightarrow\ G_{\mu\nu}+\Lambda g_{\mu\nu}=8\pi G\langle T_{\mu\nu}\rangle .
\end{array}}
$$

\end{document}




