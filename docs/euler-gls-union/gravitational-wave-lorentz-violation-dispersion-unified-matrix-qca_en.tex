\documentclass[12pt]{article}

% Essential packages
\usepackage[utf8]{inputenc}
\usepackage[T1]{fontenc}
\usepackage{amsmath,amssymb,amsthm}
\usepackage{mathrsfs}
\usepackage{geometry}
\usepackage{hyperref}
\usepackage{braket}

% Geometry settings
\geometry{a4paper, margin=1in}

% Hyperref settings
\hypersetup{
    colorlinks=true,
    linkcolor=blue,
    citecolor=blue,
    urlcolor=blue
}

% Theorem environments
\theoremstyle{plain}
\newtheorem{theorem}{Theorem}[section]
\newtheorem{lemma}[theorem]{Lemma}
\newtheorem{proposition}[theorem]{Proposition}
\newtheorem{corollary}[theorem]{Corollary}

\theoremstyle{definition}
\newtheorem{definition}[theorem]{Definition}
\newtheorem{example}[theorem]{Example}
\newtheorem{remark}[theorem]{Remark}

% Title information
\title{Unified Matrix--QCA Universe Theory of Gravitational Wave Lorentz Violation and Dispersion\\
\large Bounds on $v_{\mathrm g}\neq c$ and Testable Predictions under Unified Time Scale}

\author{Haobo Ma$^1$ \and Wenlin Zhang$^2$\\
\small $^1$Independent Researcher\\
\small $^2$National University of Singapore}

\date{\today}

\begin{document}

\maketitle

\begin{abstract}
Under the unified framework of unified time scale, boundary time geometry, Matrix Universe THE--MATRIX, and Quantum Cellular Automaton (QCA) Universe, we construct a structural theory specifically for ``Gravitational Wave Lorentz Violation and Dispersion Corrections''. The unified time scale is defined by the scale identity of scattering--spectral shift--Wigner--Smith group delay
\begin{equation}
\kappa(\omega)
=\frac{\varphi'(\omega)}{\pi}
=\rho_{\mathrm{rel}}(\omega)
=\frac{1}{2\pi}\operatorname{tr} Q(\omega),
\qquad
Q(\omega)=-\mathrm{i} S(\omega)^\dagger\partial_\omega S(\omega),
\end{equation}
unifying scattering hemi-phase derivative, relative density of states, and group delay trace into a single time density $\kappa(\omega)$, whose integral defines the time scale equivalence class representative $\tau_{\mathrm{scatt}}(\omega)$. In the perspective of ``gravitational waves as scattering modes of geometric perturbations'', $\kappa(\omega)$ directly controls the phase velocity, group velocity, and frequency-dependent propagation delay of gravitational waves.

In the Universe QCA object
\begin{equation}
U_{\mathrm{QCA}}
=(\Lambda,\mathcal H_{\mathrm{cell}},
\mathcal A_{\mathrm{qloc}},\alpha,\omega_0),
\end{equation}
gravitational degrees of freedom are embedded as linearized excitations of ``gravity--QCA modes'', whose quasi-energy spectrum $\varepsilon(\mathbf{k})$ yields an effective dispersion relation in the continuous limit
\begin{equation}
\omega^2
=c^2k^2\bigl[1+\varepsilon_2(k\ell_{\mathrm{cell}})^2
+\varepsilon_4(k\ell_{\mathrm{cell}})^4+\cdots\bigr],
\end{equation}
where $\ell_{\mathrm{cell}}$ is the QCA effective lattice spacing and $\varepsilon_{2n}$ are dimensionless coefficients. The unified time scale requires the QCA discrete time step to be in the same equivalence class as geometric proper time, boundary modular time, and scattering time scale, thereby directly linking $\varepsilon_{2n}$ in gravitational wave dispersion to high-order deviations of $\kappa(\omega)$.

Under appropriate spectral--scattering and QCA axioms, this paper obtains the following main results:

(1) In the Matrix Universe representation, viewing weak-field gravitational waves as linear perturbation modes on background FRW/flat spacetime, we construct the gravitational wave scattering matrix $S_{\mathrm{GW}}(\omega;\mathbf{k})$ and group delay matrix $Q_{\mathrm{GW}}(\omega;\mathbf{k})$, proving that in the far-field low-frequency limit, the deviation of the unified scale density $\delta\kappa_{\mathrm{GW}}(\omega)$ and the dispersion function $\varepsilon(k)$ satisfy
\begin{equation}
\delta v_{\mathrm g}(\omega)
=\frac{\partial\omega}{\partial k}-c
\simeq c\Bigl[\varepsilon_2(k\ell_{\mathrm{cell}})^2
+\mathcal O\bigl((k\ell_{\mathrm{cell}})^4\bigr)\Bigr],
\qquad
\delta\kappa_{\mathrm{GW}}(\omega)
\sim -\frac{L}{2\pi c^2}\delta v_{\mathrm g}(\omega),
\end{equation}
where $L$ is the effective propagation distance.

(2) We construct a class of ``Gravity--QCA Models'' in the QCA Universe, whose linearized degrees of freedom reproduce the transverse traceless gravitational wave equation of General Relativity (GR) in the long-wave limit, while high-order $(k\ell_{\mathrm{cell}})^{2n}$ dispersion terms are determined by cellular structure and update rules. Under unified time scale and boundary time geometry constraints, combining discrete symmetries and Null--Modular double cover consistency, we prove that in the absence of chiral anomalies and with time reversal conservation, gravitational wave dispersion only allows even-order $(k\ell_{\mathrm{cell}})^{2n}$ type corrections, while odd-order $k^{2n+1}$ type Lorentz violations are excluded in the unified framework.

(3) Utilizing constraints on gravitational wave propagation speed and dispersion from LIGO--Virgo--KAGRA and the multi-messenger event GW170817/GRB 170817A (e.g., speed constraint $\lvert v_{\mathrm g}/c-1\rvert\lesssim10^{-15}$ and multi-event fits to parameterized dispersion relations in GWTC catalogs), we rewrite these results in the unified framework as upper bounds on QCA lattice spacing $\ell_{\mathrm{cell}}$ and dispersion coefficients $\varepsilon_2,\varepsilon_4$. Combining existing constraints on energy scale $M_\ast$ for $n=2$ type $k^4$ corrections, we obtain
\begin{equation}
\ell_{\mathrm{cell}}
\lesssim M_\ast^{-1}\lvert\beta_2\rvert^{-1/2},
\qquad
\beta_2=\mathcal O(1),
\end{equation}
where $M_\ast$ typically lies in the $10^{13}\text{--}10^{15}\,\mathrm{GeV}$ range, corresponding to $\ell_{\mathrm{cell}}\lesssim10^{-29}\text{--}10^{-31}\,\mathrm{m}$. This result is of comparable magnitude to independent constraints on discrete spacetime and QCA lattice spacing based on electromagnetic and matter interferometry experiments.

(4) In the unified causal--entropy--time framework, we prove a ``Gravity--QCA Causal Consistency Theorem'': if the effective light cone of Gravity--QCA remains consistent with the causal light cone of boundary time geometry in the LIGO/Virgo frequency band, then allowed Lorentz violations must exhibit specific even-order dispersion structures, and the group velocity deviation satisfies
\begin{equation}
\biggl\lvert\frac{\delta v_{\mathrm g}(\omega)}{c}\biggr\rvert
\lesssim \mathcal O\bigl((\omega\ell_{\mathrm{cell}})^2\bigr)
\end{equation}
Planck-scale suppression law, with high-order contributions to group delay being exponentially suppressed under current observational precision.

(5) The appendix provides: construction from GR linear perturbations to Matrix Universe scattering matrix $S_{\mathrm{GW}}(\omega;\mathbf{k})$; continuum limit and dispersion expansion of Gravity--QCA models; precise relationship between group delay and $\kappa(\omega)$ deviation under unified time scale; and the process of converting LIGO/Virgo--GW170817 and GWTC-3 constraints into numerical bounds on $(\ell_{\mathrm{cell}},\varepsilon_2)$.

Results indicate: in the Unified Matrix--QCA Universe Theory, gravitational wave Lorentz violation and dispersion are not arbitrary high-dimensional operator perturbations, but geometric--spectral projections of QCA discrete structure and unified time scale deviations. Existing observations have already compressed this deviation to an extremely small range, providing strong constraints on the lattice spacing and dispersion coefficients of the universe's discrete structure, and offering a testable unified template for future high-frequency and multi-band gravitational wave detection.
\end{abstract}

\noindent\textbf{Keywords:} Gravitational waves; Lorentz invariance violation; Dispersion relation; Unified time scale; Scattering matrix; Wigner--Smith group delay; Quantum cellular automata; Matrix universe; Standard-Model Extension; GW170817; GWTC-3

\section{Introduction \& Historical Context}

\subsection{Gravitational Wave Propagation and Lorentz Invariance}

In General Relativity (GR), weak-field gravitational waves are transverse traceless tensor perturbations propagating on a Lorentzian background geometry, satisfying the linearized Einstein equations with dispersion relation $\omega^2=c^2k^2$, where phase velocity and group velocity are both equal to the speed of light $c$. Any phenomenon deviating from this dispersion relation can be regarded as ``Lorentz invariance violation in gravitational wave propagation'' or ``effective medium correction''. In effective field theory language, such corrections are typically written as
\begin{equation}
\omega^2
=c^2k^2+\alpha_{\mathrm{disp}}k^{2+n},
\end{equation}
or equivalent forms parameterized by graviton mass and high-dimensional operators, where $\alpha_{\mathrm{disp}}$ and $n$ are determined by the specific theory.

Gravitational wave detections by LIGO, Virgo, and KAGRA provide direct means to test these corrections. Systematic analyses based on parameterized dispersion relations show that observable effects of Lorentz violation on waveform phases can be embedded into the ``parameterized post-Einsteinian'' framework and jointly constrained on multiple events using standard Bayesian inference.

\subsection{GW170817 and Gravitational Wave Speed Constraints}

The 2017 binary neutron star merger event GW170817 and its gamma-ray burst counterpart GRB 170817A represent a milestone multi-messenger event in the field of gravitational waves. The arrival time difference between gravitational waves and gamma rays was about $1.74\,\mathrm{s}$, with a propagation distance of about $40\,\mathrm{Mpc}$, yielding a constraint on the relative difference between gravitational wave group velocity and the speed of light
\begin{equation}
-3\times 10^{-15}
\lesssim \frac{v_{\mathrm g}}{c}-1
\lesssim 7\times 10^{-16},
\end{equation}
i.e., $\lvert v_{\mathrm g}/c-1\rvert\lesssim\mathcal O(10^{-15})$.

This result broadly rules out a large class of dark energy--modified gravity models that adjust gravitational wave speed on cosmological scales, providing strong constraints on Horndeski, Einstein--Aether, bimetric theories, etc. Furthermore, analyses of graviton mass and Lorentz violation indicate that current LIGO/Virgo data constrain the graviton mass to the range $m_{\mathrm g}\lesssim10^{-23}\text{--}10^{-22}\,\mathrm{eV}$, corresponding to a Compton wavelength greater than $10^{13}\,\mathrm{km}$.

\subsection{GWTC-3, SME, and Parameterized Dispersion Tests}

With the release of multiple batches of events in GWTC-1/2/3, the LIGO--Virgo--KAGRA collaboration has performed systematic ``propagation tests'' of General Relativity, including dimensions of speed, dispersion, attenuation, and polarization. A significant class of work involves generalized dispersion relations with anisotropic, birefringent, and dispersive corrections derived from Lorentz-violating operators in the gravity sector of the Standard-Model Extension (SME). Joint constraints on coefficients of $d=5,6$ dimensional operators using 90 high-confidence events in GWTC-3 have revealed no significant signs of Lorentz violation.

In the non-dispersive limit of propagation speed, independent analyses using arrival time delays across multiple detectors have also given confidence intervals for $v_{\mathrm g}$ within the $0.97c\text{--}1.01c$ range, and constrained non-birefringent, non-dispersive Lorentz violation coefficients under the SME framework.

Overall, gravitational wave data indicate that in the $10\text{--}10^3\,\mathrm{Hz}$ frequency band, the propagation of gravitational waves almost perfectly obeys Lorentz invariance, and any observable dispersion or speed deviation must be extremely weak.

\subsection{Discrete Spacetime, QCA, and Gravitational Wave Dispersion}

On the other hand, discrete spacetime and Quantum Cellular Automata (QCA) frameworks provide a way to unify the description of ``how continuous Lorentz symmetry emerges from deeper discrete structures''. In quantum walks and QCA models, finite lattice spacing $\ell_{\mathrm{cell}}$ and discrete time step $\Delta t$ determine the effective dispersion relation, whose continuous limit typically reproduces Dirac/Weyl/Maxwell equations, with slight violations of Lorentz symmetry embodied in high-order $k\ell_{\mathrm{cell}}$ corrections.

Recent work has attempted to use Lorentz violation observations from electromagnetic spectra and high-energy cosmic rays to place upper bounds on the QCA lattice spacing $\ell_{\mathrm{cell}}$. Typical results indicate that $\ell_{\mathrm{cell}}$ must be far smaller than currently accessible experimental scales, possibly approaching the $10^{-29}\text{--}10^{-31}\,\mathrm{m}$ range.

In this context, a natural question arises: **In a unified Matrix--QCA universe, can gravitational wave Lorentz violation and dispersion be interpreted as geometric--spectral projections of QCA discrete structure, precisely controlled by the unified time scale density $\kappa(\omega)$? And how strong are the upper bounds on $\ell_{\mathrm{cell}}$ and dispersion coefficients $\varepsilon_{2n}$ given by existing LIGO/Virgo/GW170817 constraints?**

The purpose of this paper is to construct a unified model, provide theorem-based conclusions, and propose testable predictions centering on this question.

\section{Model \& Assumptions}

\subsection{Unified Time Scale Mother Formula and Matrix Universe}

The mother formula for the unified time scale is
\begin{equation}
\kappa(\omega)
=\frac{\varphi'(\omega)}{\pi}
=\rho_{\mathrm{rel}}(\omega)
=\frac{1}{2\pi}\operatorname{tr} Q(\omega),
\quad
Q(\omega)=-\mathrm{i} S(\omega)^\dagger\partial_\omega S(\omega),
\end{equation}
where $S(\omega)$ is the fixed-energy scattering matrix, $\varphi(\omega)=\tfrac12\arg\det S(\omega)$ is the total hemi-phase, $\rho_{\mathrm{rel}}(\omega)$ is the relative density of states, and $Q(\omega)$ is the Wigner--Smith group delay matrix.

The time scale parameter is defined as
\begin{equation}
\tau_{\mathrm{scatt}}(\omega)
=\int_{\omega_0}^{\omega}\kappa(\tilde\omega)\,\mathrm{d}\tilde\omega,
\end{equation}
where affine transformations $\tau\mapsto a\tau+b$ are considered the same time scale equivalence class. Prior work has proved that under appropriate scattering--geometry--modular flow axioms, geometric proper time, boundary modular time, and scattering time scale belong to the same equivalence class.

The Matrix Universe THE-MATRIX can be abstracted at the spectral--scattering end as
\begin{equation}
U_{\mathrm{mat}}
=\bigl(\mathcal H_{\mathrm{chan}},S(\omega),Q(\omega),
\kappa(\omega),\mathcal A_\partial,\omega_\partial\bigr),
\end{equation}
where $\mathcal H_{\mathrm{chan}}$ is the channel Hilbert space, and $\mathcal A_\partial,\omega_\partial$ describe boundary observable algebra and state. The universe's causal structure, time arrow, and generalized entropy flow are given by boundary time geometry on small causal diamonds.

\subsection{Universe QCA Object and Gravitational Degrees of Freedom}

The Universe QCA object is denoted as
\begin{equation}
U_{\mathrm{QCA}}
=(\Lambda,\mathcal H_{\mathrm{cell}},
\mathcal A_{\mathrm{qloc}},\alpha,\omega_0),
\end{equation}
where $\Lambda$ is a countable connected graph (usually $\mathbb Z^d$ or its sparse subgraph), $\mathcal H_{\mathrm{cell}}$ is the finite-dimensional cellular Hilbert space, $\mathcal A_{\mathrm{qloc}}$ is the quasilocal $\mathrm C^\ast$ algebra on its infinite tensor product, $\alpha:\mathbb Z\to\mathrm{Aut}(\mathcal A_{\mathrm{qloc}})$ is a $\ast$-automorphism with finite propagation radius and spatial homogeneity, and $\omega_0$ is the initial universe state.

Gravitational degrees of freedom are encoded in the following way:

1. Background geometry is encoded as effective light cone structure: the propagation radius of $\alpha$ and adjacency graph topology reproduce the causal cone structure of some Lorentz manifold $(M,g)$ in the continuous limit;

2. Gravitational wave modes are linearized eigenmodes of $U=\mathrm{e}^{-\mathrm{i} H_{\mathrm{eff}}\Delta t}$, whose difference from background $U_0$ satisfies the transverse traceless wave equation of GR in the low-energy limit.

In momentum representation, using Bloch--Floquet decomposition
\begin{equation}
U=\int_{\mathrm{BZ}}^{\oplus}U(\mathbf{k})\,\mathrm{d}\mu(\mathbf{k}),
\end{equation}
where $\mathrm{BZ}$ is the Brillouin zone, $U(\mathbf{k})$ is unitary on $\mathcal H_{\mathrm{cell}}$, with spectral decomposition
\begin{equation}
U(\mathbf{k})
=\sum_a\exp\bigl(-\mathrm{i} \varepsilon_a(\mathbf{k})\Delta t\bigr)
\Pi_a(\mathbf{k}),
\end{equation}
where $\varepsilon_a(\mathbf{k})$ is the quasi-energy spectrum and $\Pi_a(\mathbf{k})$ is the eigenprojection. The gravitational wave branch is denoted $\varepsilon_{\mathrm{GW}}(\mathbf{k})$, defining effective frequency
\begin{equation}
\omega(\mathbf{k})
=\frac{\varepsilon_{\mathrm{GW}}(\mathbf{k})}{\Delta t}.
\end{equation}

\subsection{Dispersion Relation and QCA Lattice Spacing}

Let the QCA fundamental lattice spacing be $\ell_{\mathrm{cell}}$. In the long-wave limit $k\ell_{\mathrm{cell}}\ll1$, Taylor expansion can be performed on $\omega(\mathbf{k})$. Assuming the existence of a cluster of massless branches with dominant behavior $\omega(\mathbf{k})\simeq ck$ ($c$ being macroscopic speed of light), it can generally be written as
\begin{equation}
\omega^2(\mathbf{k})
=c^2k^2
\left[1+\sum_{n\ge1}
\beta_{2n}(\hat{\mathbf{k}})(k\ell_{\mathrm{cell}})^{2n}\right],
\end{equation}
where $\hat{\mathbf{k}}=\mathbf{k}/k$ is the direction, and $\beta_{2n}(\hat{\mathbf{k}})$ are dimensionless coefficients. This expression explicitly embodies high-order corrections to dispersion from QCA discrete structure.

This paper focuses on the dominant term under isotropic approximation
\begin{equation}
\omega^2
=c^2k^2\bigl[1+\beta_2(k\ell_{\mathrm{cell}})^2\bigr],
\end{equation}
corresponding to the $n=2$ type parameterized dispersion $\omega^2=c^2k^2+\alpha_{\mathrm{disp}}k^4$ in observations.

\subsection{Unified Time Scale and Gravitational Wave Scattering Channels}

For a given frequency $\omega$, consider the gravitational wave scattering channel subspace $\mathcal H_{\mathrm{GW}}\subset\mathcal H_{\mathrm{chan}}$, corresponding to scattering matrix $S_{\mathrm{GW}}(\omega)\in U(N_{\mathrm{GW}})$ and group delay matrix
\begin{equation}
Q_{\mathrm{GW}}(\omega)
=-\mathrm{i} S_{\mathrm{GW}}^\dagger(\omega)\partial_\omega S_{\mathrm{GW}}(\omega),
\end{equation}
whose eigenvalues $\tau_j(\omega)$ are group delays of each channel. The unified time scale density of the gravitational wave sector is defined as
\begin{equation}
\kappa_{\mathrm{GW}}(\omega)
=\frac{1}{2\pi}\operatorname{tr} Q_{\mathrm{GW}}(\omega)
=\frac{1}{2\pi}\sum_{j=1}^{N_{\mathrm{GW}}}\tau_j(\omega).
\end{equation}

In the far-field approximation, if the propagation distance is $L$ and assuming all channels have similar group velocity $v_{\mathrm g}(\omega)$, then
\begin{equation}
\bar\tau(\omega)
=\frac1{N_{\mathrm{GW}}}\sum_{j}\tau_j(\omega)
\approx\frac{L}{v_{\mathrm g}(\omega)},
\end{equation}
thus
\begin{equation}
\kappa_{\mathrm{GW}}(\omega)
\approx\frac{N_{\mathrm{GW}}}{2\pi}\frac{L}{v_{\mathrm g}(\omega)}.
\end{equation}

Letting the reference quantity in GR case be $\kappa_{\mathrm{GW}}^{(0)}(\omega)\approx N_{\mathrm{GW}}L/(2\pi c)$, its deviation is defined as
\begin{equation}
\delta\kappa_{\mathrm{GW}}(\omega)
=\kappa_{\mathrm{GW}}(\omega)-\kappa_{\mathrm{GW}}^{(0)}(\omega).
\end{equation}

\section{Main Results (Theorems and Alignments)}

This section presents four core theorems based on the model and assumptions, formalizing QCA dispersion and unified time scale, Null--Modular consistency, and the impact of observational constraints on lattice spacing $\ell_{\mathrm{cell}}$.

\subsection{Theorem 3.1 (GW Dispersion and Unified Time Scale Density Deviation)}

\begin{theorem}[Gravitational Wave Dispersion and Unified Time Scale Density Deviation]
Let gravitational waves propagate distance $L$ in a macroscopically homogeneous medium (or cosmic background), with effective dispersion relation
\begin{equation}
\omega^2
=c^2k^2\bigl[1+\varepsilon(k)\bigr],
\qquad
\lvert\varepsilon(k)\rvert\ll1,
\end{equation}
and group velocity
\begin{equation}
v_{\mathrm g}(\omega)
=\frac{\partial\omega}{\partial k}>0
\end{equation}
being monotonic in the LIGO/Virgo band. Assume scattering matrix $S_{\mathrm{GW}}(\omega)$ can be constructed from plane wave modes, and far-field Wigner group delay $\tau_j(\omega)$ is equivalent to perturbation of propagation time $L/v_{\mathrm g}(\omega)$, then unified time scale density deviation satisfies
\begin{equation}
\delta\kappa_{\mathrm{GW}}(\omega)
\approx -\frac{L}{2\pi c^2}\,
\delta v_{\mathrm g}(\omega)
+\mathcal O\bigl(\varepsilon^2\bigr),
\end{equation}
where
\begin{equation}
\delta v_{\mathrm g}(\omega)
=v_{\mathrm g}(\omega)-c
\simeq \frac{c}{2}\Bigl[\varepsilon(k)
+k\varepsilon'(k)\Bigr]_{k=\omega/c}.
\end{equation}

Specifically, when $\varepsilon(k)=\beta_2(k\ell_{\mathrm{cell}})^2+\mathcal O((k\ell_{\mathrm{cell}})^4)$,
\begin{equation}
\delta v_{\mathrm g}(\omega)
\simeq c\,\beta_2(k\ell_{\mathrm{cell}})^2,
\qquad
\frac{\delta v_{\mathrm g}(\omega)}{c}
\simeq \beta_2(k\ell_{\mathrm{cell}})^2,
\end{equation}
thus
\begin{equation}
\delta\kappa_{\mathrm{GW}}(\omega)
\approx -\frac{L}{2\pi c}\,
\beta_2(k\ell_{\mathrm{cell}})^2.
\end{equation}
\end{theorem}

\subsection{Theorem 3.2 (Even-Order Dispersion Structure of QCA Gravity Modes)}

\begin{theorem}[Even-Order Dispersion Structure of QCA Gravity Modes]
Let $U_{\mathrm{QCA}}$ be a spatially homogeneous, local, translation-invariant QCA satisfying the following conditions:

1. Existence of parity symmetry $\mathcal P$ and time reversal symmetry $\mathcal T$, such that $\mathcal P U(\mathbf{k})\mathcal P^{-1}=U(-\mathbf{k})$, $\mathcal T U(\mathbf{k})\mathcal T^{-1}=U^\dagger(-\mathbf{k})$;

2. Existence of a cluster of massless gravitational wave branches satisfying $\omega(\mathbf{k})\sim ck$ as $\mathbf{k}\to0$;

3. This branch has finite degeneracy with other branches in the low-energy limit, and can be diagonalized in an appropriate basis appearing as pairs of $\omega(\mathbf{k})$ and $-\omega(\mathbf{k})$.

Then the expansion of $\omega^2(\mathbf{k})$ for $k\ell_{\mathrm{cell}}\ll1$ contains only even-order terms:
\begin{equation}
\omega^2(\mathbf{k})
=c^2k^2
\left[1+\sum_{n\ge1}
\beta_{2n}(\hat{\mathbf{k}})(k\ell_{\mathrm{cell}})^{2n}\right],
\end{equation}
without odd-order corrections like $k^3,k^5,\dots$. Specifically, under isotropic approximation, the dominant correction is
\begin{equation}
\omega^2
=c^2k^2\bigl[1+\beta_2(k\ell_{\mathrm{cell}})^2\bigr],
\end{equation}
i.e., the lowest-order Lorentz violation of Gravity--QCA dispersion must be of $k^4$ type, not odd-order types like $k^3$.
\end{theorem}

\subsection{Theorem 3.3 (Upper Bound on QCA Lattice Spacing from Parameterized Dispersion Constraints)}

\begin{theorem}[Upper Bound on QCA Lattice Spacing from Parameterized Dispersion Constraints]
Consider parameterized dispersion relation
\begin{equation}
\omega^2
=c^2k^2+\alpha_{\mathrm{disp}}k^{2+n},
\end{equation}
where $n\ge0$, $\alpha_{\mathrm{disp}}$ is a constant with appropriate dimensions. Assuming $n=2$ matches QCA isotropic dominant correction, i.e.,
\begin{equation}
\omega^2
=c^2k^2\bigl[1+\beta_2(k\ell_{\mathrm{cell}})^2\bigr]
=c^2k^2+\alpha_{\mathrm{disp}}k^4,
\end{equation}
then
\begin{equation}
\alpha_{\mathrm{disp}}
=c^2\beta_2\ell_{\mathrm{cell}}^2.
\end{equation}

If LIGO/Virgo/KAGRA joint analysis gives at some confidence level
\begin{equation}
\lvert\alpha_{\mathrm{disp}}\rvert
\lesssim \frac{c^2}{M_\ast^2},
\end{equation}
then QCA lattice spacing satisfies
\begin{equation}
\ell_{\mathrm{cell}}
\lesssim M_\ast^{-1}\lvert\beta_2\rvert^{-1/2}.
\end{equation}

Typically, for representative $n=2$ dispersion constraints, literature gives effective energy scale $M_\ast$ at least in the $10^{13}\text{--}10^{15}\,\mathrm{GeV}$ range, corresponding to
\begin{equation}
\ell_{\mathrm{cell}}
\lesssim 10^{-29}\text{--}10^{-31}\,\mathrm{m}
\quad
(\beta_2\sim1),
\end{equation}
indicating that if QCA discrete structure exists in the universe, its lattice spacing must be at least dozens of orders of magnitude smaller than currently directly detectable length scales.
\end{theorem}

\subsection{Theorem 3.4 (Gravity--QCA Causal Consistency and Lorentz Violation Bounds)}

\begin{theorem}[Gravity--QCA Causal Consistency and Lorentz Violation Bounds]
In the unified time scale, boundary time geometry, and Null--Modular double cover framework, assume:

1. Generalized entropy extremum and non-negative second-order relative entropy on small causal diamonds hold, equivalent to local Einstein equations and QNEC/QFC inequalities;

2. Boundary modular flow, scattering phase, and geometric time align under unified scale;

3. Light cones of Gravity--QCA model and geometric light cones are consistent to $\mathcal O(10^{-15})$ in LIGO/Virgo band;

4. Null--Modular double cover has no $\mathbb Z_2$ holonomy anomaly.

Then gravitational wave dispersion corrections must satisfy:

1. Only even-order $(k\ell_{\mathrm{cell}})^{2n}$ type terms are allowed; non-zero odd-order $k^{2n+1}$ terms would necessarily introduce forbidden half-period phases in Null--Modular structure, violating condition 4;

2. Group velocity deviation satisfies Planck-scale suppression law
   \begin{equation}
   \biggl\lvert\frac{\delta v_{\mathrm g}(\omega)}{c}\biggr\rvert
   \lesssim \mathcal O\bigl((\omega\ell_{\mathrm{cell}})^2\bigr);
   \end{equation}

3. Relative deviation of unified time scale density satisfies
   \begin{equation}
   \frac{\lvert\delta\kappa_{\mathrm{GW}}(\omega)\rvert}
   {\kappa_{\mathrm{GW}}^{(0)}(\omega)}
   \lesssim\mathcal O\bigl((\omega\ell_{\mathrm{cell}})^2\bigr),
   \end{equation}
   numerically not exceeding $\mathcal O(10^{-15})$ magnitude in current observation bands, compatible with speed and dispersion constraints from GW170817 and GWTC-3.
\end{theorem}

\section{Proofs}

This section provides proofs or derivation outlines for Theorems 3.1–3.4. Detailed calculations and technical lemmas are in the Appendix.

\subsection{Proof of Theorem 3.1: Dispersion and Unified Time Scale Density}

In 1D simplified case, assume incident plane wave propagates in a medium of length $L$, with dispersion relation $\omega(k)$ and group velocity $v_{\mathrm g}(\omega)=\partial\omega/\partial k$. Scattering matrix can be written as
\begin{equation}
S(\omega)
=\begin{pmatrix}
r(\omega)&t'(\omega)\\
t(\omega)&r'(\omega)
\end{pmatrix},
\qquad
t(\omega)=\lvert t(\omega)\rvert
\exp\bigl[\mathrm{i}\phi(\omega)\bigr],
\end{equation}
where $\phi(\omega)$ is transmission phase. Wigner group delay is
\begin{equation}
\tau(\omega)
=\partial_\omega\phi(\omega).
\end{equation}

In the weak scattering limit where reflection is negligible, transmission phase approximately equals propagation phase of plane wave in medium:
\begin{equation}
\phi(\omega)
\approx k(\omega)L,
\qquad
\tau(\omega)
\approx L\,\partial_\omega k(\omega)
=\frac{L}{v_{\mathrm g}(\omega)}.
\end{equation}

In multi-channel case, Wigner--Smith matrix
\begin{equation}
Q(\omega)
=-\mathrm{i} S^\dagger(\omega)\partial_\omega S(\omega)
\end{equation}
eigenvalues give group delays of each channel, trace is their sum. If there are $N$ equivalent channels, then
\begin{equation}
\operatorname{tr} Q(\omega)
\approx N\frac{L}{v_{\mathrm g}(\omega)}.
\end{equation}

Unified time scale density is defined as
\begin{equation}
\kappa(\omega)
=\frac{1}{2\pi}\operatorname{tr} Q(\omega)
\approx\frac{N}{2\pi}\frac{L}{v_{\mathrm g}(\omega)}.
\end{equation}

Introducing GR benchmark $v_{\mathrm g}^{(0)}=c$, $\kappa^{(0)}(\omega)=NL/(2\pi c)$, deviation is
\begin{equation}
\delta\kappa(\omega)
=\kappa(\omega)-\kappa^{(0)}(\omega)
\approx
\frac{NL}{2\pi}\left(
\frac1{v_{\mathrm g}}-\frac1c\right)
\approx -\frac{NL}{2\pi c^2}\delta v_{\mathrm g}(\omega),
\end{equation}
where
\begin{equation}
\delta v_{\mathrm g}(\omega)
=v_{\mathrm g}(\omega)-c,
\quad
\lvert \delta v_{\mathrm g}(\omega)\rvert\ll c.
\end{equation}

This derivation generalizes to high dimensions and multi-polarization cases by performing same steps for each mode and summing over trace, yielding same structure with $N$ replaced by dimension of $\mathcal H_{\mathrm{GW}}$. Thus obtaining main formula of Theorem 3.1.

Relation between dispersion function $\varepsilon(k)$ and group velocity deviation comes from
\begin{equation}
\omega^2
=c^2k^2[1+\varepsilon(k)]
\Rightarrow
\omega(k)
=ck\sqrt{1+\varepsilon(k)}
\approx ck\Bigl[1+\tfrac12\varepsilon(k)\Bigr],
\end{equation}
thus
\begin{equation}
v_{\mathrm g}
=\partial_k\omega
\approx c\Bigl[1+\tfrac12\varepsilon(k)
+\tfrac12k\varepsilon'(k)\Bigr],
\end{equation}
which is the form stated in the theorem.

\subsection{Proof of Theorem 3.2: Symmetry and Even-Order Expansion}

Consider QCA satisfying conditions 1–3. Due to translation invariance and locality, $U(\mathbf{k})$ is an analytic unitary matrix function in $\mathbf{k}$ space. Condition 2 and existence of massless branch imply eigenvalues $\mathrm{e}^{-\mathrm{i}\varepsilon(\mathbf{k})\Delta t}$ exist in neighborhood of $\mathbf{k}\to0$, where $\varepsilon(\mathbf{k})=\omega(\mathbf{k})\Delta t$ and $\omega(\mathbf{k})\sim ck$.

Parity symmetry gives
\begin{equation}
\mathcal P U(\mathbf{k})\mathcal P^{-1}
=U(-\mathbf{k}),
\end{equation}
implying for spectrum $\varepsilon(\mathbf{k})$
\begin{equation}
\{\varepsilon_a(\mathbf{k})\}
=\{\varepsilon_a(-\mathbf{k})\},
\end{equation}
time reversal symmetry gives
\begin{equation}
\mathcal T U(\mathbf{k})\mathcal T^{-1}
=U^\dagger(-\mathbf{k}),
\end{equation}
thus
\begin{equation}
\{\varepsilon_a(\mathbf{k})\}
=\{-\varepsilon_a(-\mathbf{k})\}
\quad
(\mathrm{mod}\ 2\pi/\Delta t).
\end{equation}

For massless gravitational wave branch, basis can be chosen such that its spectrum satisfies
\begin{equation}
\varepsilon_{\mathrm{GW}}(-\mathbf{k})
=-\varepsilon_{\mathrm{GW}}(\mathbf{k}),
\end{equation}
i.e., $\omega(-\mathbf{k})=-\omega(\mathbf{k})$. This implies $\omega(\mathbf{k})$ is an odd function of $\mathbf{k}$, while $\omega^2(\mathbf{k})$ is an even function. Expanding $\omega^2$ at $k\ell_{\mathrm{cell}}$,
\begin{equation}
\omega^2(\mathbf{k})
=c^2k^2
\sum_{n\ge0}\gamma_{2n}(\hat{\mathbf{k}})(k\ell_{\mathrm{cell}})^{2n},
\end{equation}
where $\gamma_0(\hat{\mathbf{k}})=1$, other coefficients come from combinatorial structure of local gates; odd-order terms $(k\ell_{\mathrm{cell}})^{2n+1}$ appearing would violate $\omega^2(-\mathbf{k})=\omega^2(\mathbf{k})$, contradicting symmetries.

Using normalization $\omega(\mathbf{k})\sim ck$ in low-energy limit, reorganizing coefficients gives
\begin{equation}
\omega^2(\mathbf{k})
=c^2k^2
\left[1+\sum_{n\ge1}\beta_{2n}(\hat{\mathbf{k}})(k\ell_{\mathrm{cell}})^{2n}\right].
\end{equation}

In isotropic approximation, $\beta_{2n}(\hat{\mathbf{k}})$ depend only on average symmetry of lattice and gates, simplifying to constant $\beta_{2n}$, yielding theorem conclusion.

This structure is consistent with dispersion forms in specific Dirac/Weyl/Maxwell QCA models; existing analyses show Lorentz symmetry violation in these models indeed first appears in $(k\ell_{\mathrm{cell}})^2$ order corrections.

\subsection{Proof of Theorem 3.3: Parameter Mapping and Energy Scale}

From
\begin{equation}
\omega^2
=c^2k^2\bigl[1+\beta_2(k\ell_{\mathrm{cell}})^2\bigr]
=c^2k^2+\alpha_{\mathrm{disp}}k^4
\end{equation}
we get
\begin{equation}
\alpha_{\mathrm{disp}}
=c^2\beta_2\ell_{\mathrm{cell}}^2.
\end{equation}

Parameterized dispersion test literature usually writes $\alpha_{\mathrm{disp}}$ as
\begin{equation}
\alpha_{\mathrm{disp}}
=\sigma\,\frac{c^2}{M_\ast^2},
\end{equation}
where $M_\ast$ is some high energy scale (e.g., EFT cutoff or Lorentz violation scale), $\sigma$ is a dimensionless coefficient, often taking $\sigma=\pm1$.

Comparing gives
\begin{equation}
\ell_{\mathrm{cell}}^2
=\frac{\sigma}{\beta_2}\frac{1}{M_\ast^2},
\qquad
\ell_{\mathrm{cell}}
=\frac{1}{M_\ast}\left\lvert
\frac{\sigma}{\beta_2}\right\rvert^{1/2}.
\end{equation}

Constraints given by observation can be written as $\lvert\alpha_{\mathrm{disp}}\rvert\lesssim c^2/M_\ast^2$, hence
\begin{equation}
\ell_{\mathrm{cell}}
\lesssim M_\ast^{-1}\lvert\beta_2\rvert^{-1/2},
\end{equation}
declaring the theorem established.

Using current constraints on $n=2$ type dispersion, $M_\ast$ can be estimated to be at least in $10^{13}\text{--}10^{15}\,\mathrm{GeV}$ range. Combined with natural units $1\,\mathrm{GeV}^{-1}\approx2\times10^{-16}\,\mathrm{m}$, we get
\begin{equation}
\ell_{\mathrm{cell}}
\lesssim 10^{-29}\text{--}10^{-31}\,\mathrm{m}
\quad(\beta_2\sim1),
\end{equation}
consistent with lattice spacing upper bounds derived from other experiments in discrete spacetime and QCA literature.

\subsection{Proof Outline of Theorem 3.4: Null--Modular Consistency and Suppression}

Proof in three steps.

Step 1: Link unified time scale to Null--Modular double cover. Spectrum of boundary modular flow generator $K$ and frequency dependence of scattering phase are linked via relative entropy and BF type bulk integral, giving scale identity $\kappa(\omega)=\varphi'(\omega)/\pi=(2\pi)^{-1}\operatorname{tr} Q(\omega)$. If odd-order dispersion correction $\omega^2=c^2k^2+\tilde\alpha k^3+\dots$ is introduced, scattering phase is no longer simple odd function on $\omega$ plane, its derivative acquires extra $\pi$ phase on loop around origin, leading to non-trivial $\mathbb Z_2$ holonomy on Null--Modular cover, contradicting $[K]=0$ condition.

Step 2: Transform generalized entropy extremum and QNEC/QFC conditions into constraints on causal light cone and propagation speed. If gravitational wave group velocity $v_{\mathrm g}(\omega)$ significantly deviates from geometric light speed in some band, there exists causal diamond whose boundary light cone structure no longer aligns with modular flow lines, destroying monotonicity of relative entropy and local energy conditions, conflicting with QNEC/QFC inequalities. This requires
\begin{equation}
\biggl\lvert\frac{\delta v_{\mathrm g}(\omega)}{c}\biggr\rvert
\lesssim C(\omega L_{\mathrm{UV}})^2,
\end{equation}
where $L_{\mathrm{UV}}$ is relevant UV length scale, naturally taken as $\ell_{\mathrm{cell}}$ or its multiple, $C$ is constant, yielding $(\omega\ell_{\mathrm{cell}})^2$ suppression law for $\delta v_{\mathrm g}/c$.

Step 3: Use Theorem 3.1 to align $\delta v_{\mathrm g}$ with $\delta\kappa_{\mathrm{GW}}$, normalizing to obtain
\begin{equation}
\frac{\lvert\delta\kappa_{\mathrm{GW}}(\omega)\rvert}
{\kappa_{\mathrm{GW}}^{(0)}(\omega)}
\sim
\frac{\lvert\delta v_{\mathrm g}(\omega)\rvert}{c}
\lesssim\mathcal O\bigl((\omega\ell_{\mathrm{cell}})^2\bigr),
\end{equation}
and using GW170817 and GWTC-3 speed and dispersion constraints to fix numerical value on RHS not exceeding $\mathcal O(10^{-15})$, completing proof outline.

\section{Model Apply: From QCA Dispersion to Waveform Predictions}

This section discusses how to specifically apply Unified Matrix--QCA model to gravitational wave data analysis, obtaining waveform corrections and predictions directly interfaceable with existing LIGO/Virgo/KAGRA pipelines.

\subsection{Mapping with Parameterized Dispersion Tests}

Parameterized dispersion tests usually start from modified dispersion relation
\begin{equation}
E^2
=p^2c^2+A p^\alpha c^\alpha,
\end{equation}
converting to frequency dependent correction to waveform phase $\delta\Psi(f)$, and fitting parameters $A,\alpha$ in Bayesian framework. For $n=2$ case, can write
\begin{equation}
\omega^2
=c^2k^2+\alpha_{\mathrm{disp}}k^4,
\quad
\alpha_{\mathrm{disp}}
\propto M_\ast^{-2}.
\end{equation}

Unified Matrix--QCA model gives dispersion
\begin{equation}
\omega^2
=c^2k^2\bigl[1+\beta_2(k\ell_{\mathrm{cell}})^2\bigr],
\end{equation}
thus
\begin{equation}
\alpha_{\mathrm{disp}}
=c^2\beta_2\ell_{\mathrm{cell}}^2.
\end{equation}

Substituting this mapping into existing waveform correction formulas allows direct conversion of posterior distribution of $\alpha_{\mathrm{disp}}$ to joint posterior of $\ell_{\mathrm{cell}}$ and $\beta_2$. For a given QCA model, $\beta_2$ can be analytically calculated or numerically estimated from cellular Hilbert space dimension and local gate structure, thereby obtaining upper bound on $\ell_{\mathrm{cell}}$.

\subsection{Multi-Event Joint Analysis and Band Sensitivity}

GWTC-3 contains 90 high-confidence events, covering roughly $10\text{--}2000\,\mathrm{Hz}$. In unified model, $\delta v_{\mathrm g}/c\sim\beta_2(k\ell_{\mathrm{cell}})^2$, so higher frequency, farther propagation distance events are more sensitive to $\ell_{\mathrm{cell}}$. For typical binary black hole merger, taking representative frequency $f\sim100\,\mathrm{Hz}$, propagation distance $L\sim \mathcal O(10^3\,\mathrm{Mpc})$, then
\begin{equation}
k=\frac{2\pi f}{c}
\sim 10^{-6}\,\mathrm{m}^{-1},
\end{equation}
any $\ell_{\mathrm{cell}}\gtrsim10^{-28}\,\mathrm{m}$ produces group velocity deviation above $\mathcal O(10^{-16})$, contradicting joint constraints of GW170817 and GWTC-3.

For low-frequency space detectors (LISA, TianQin), although frequency is lower, propagation distance is larger, sensitive to dispersion at different energy scales. Unified model provides tool to uniformly compare $\alpha_{\mathrm{disp}}$ and $\ell_{\mathrm{cell}}$ across different bands.

\subsection{Waveform Deformation and Phase Drift}

In frequency domain, waveform phase correction caused by dispersion can be generally written as
\begin{equation}
\delta\Psi(f)
=\zeta\,f^{n-1},
\end{equation}
where $\zeta$ is determined by $\alpha_{\mathrm{disp}}$ and cosmological propagation effects. In unified model, from
\begin{equation}
\alpha_{\mathrm{disp}}
=c^2\beta_2\ell_{\mathrm{cell}}^2
\end{equation}
we have
\begin{equation}
\zeta
\propto\beta_2\ell_{\mathrm{cell}}^2.
\end{equation}

This means, given event redshift and cosmological model, posterior distribution of $\delta\Psi(f)$ directly gives constraint on $\beta_2\ell_{\mathrm{cell}}^2$. Joint analysis of multiple events corresponds to global upper bound on $\ell_{\mathrm{cell}}$.

\section{Engineering Proposals}

This section proposes feasible engineering schemes to directly test predictions of Unified Matrix--QCA model in existing and future gravitational wave data analysis.

\subsection{Integration with Existing LVK MDR Pipeline}

LVK collaboration has fixed ``modified dispersion relation (MDR)'' testing procedure, updating constraints on $\alpha_{\mathrm{disp}},n$ etc. with each GW catalog release. In this framework, following modifications can be made:

1. In parameter space, rewrite $\alpha_{\mathrm{disp}}$ as $\alpha_{\mathrm{disp}}=c^2\beta_2\ell_{\mathrm{cell}}^2$, taking $\ell_{\mathrm{cell}}$ and $\beta_2$ as new primary parameters, with prior for $\beta_2$ given by QCA model;

2. No extra modification in waveform generation module, only map samples of $\alpha_{\mathrm{disp}}$ to samples of $\ell_{\mathrm{cell}}$ in post-processing stage;

3. Jointly fit GWTC-3 and subsequent GWTC-4/5 events to obtain joint posterior and confidence interval for $\ell_{\mathrm{cell}}$.

Engineering cost of this scheme is extremely low, requiring only minor post-processing scripts added to existing pipeline.

\subsection{Designing Dedicated High-Frequency and Broadband Tests}

Characteristic $(k\ell_{\mathrm{cell}})^{2n}$ structure in QCA model makes high-frequency modes more sensitive to $\ell_{\mathrm{cell}}$ than low-frequency modes. Proposals:

1. Broadband burst search oriented towards high-frequency range ($f\gtrsim1\,\mathrm{kHz}$), emphasizing this band in parameterized dispersion analysis;

2. Persistent observation of possible continuous gravitational wave sources (e.g., rapidly rotating neutron stars), accumulating phase drift caused by dispersion;

3. Forward-looking assessment of sensitivity--lattice spacing constraints for proposed facilities (e.g., 3rd gen ground detectors, space detectors), formulating dedicated observation metrics for QCA model.

\subsection{Cross-Channel Comparison with EM and Matter Interferometry Constraints}

Existing work based on optical, gamma-ray, and cosmic ray observations provides constraints on Lorentz violation under Standard-Model Extension and MDR frameworks, interpretable as upper bounds on QCA lattice spacing. Meanwhile, quantum interferometry experiments (neutrino oscillation, atom interferometers) also constrain discrete spacetime models.

Engineering-wise, one can construct a unified ``Multi-Messenger QCA Constraint Plot'', unifying constraints on $\ell_{\mathrm{cell}}$ from gravitational waves, electromagnetic waves, and matter waves into same coordinate system, comparing complementarity and redundancy of different detection means.

\section{Discussion (risks, boundaries, past work)}

\subsection{Boundaries of Model Assumptions}

Unified Matrix--QCA model relies on several key assumptions:

1. Universe ontology can be simultaneously characterized as Scattering Matrix Universe and QCA Universe, equivalent in categorical sense;

2. Gravitational waves can be treated as linear perturbations in observation band, ignoring corrections to dispersion from nonlinear self-interaction and background evolution;

3. QCA symmetries (translation, parity, time reversal) remain good on effective gravity branch, excluding odd-order dispersion terms.

If these assumptions fail at some energy scale or cosmic epoch, scope of theorems in this paper must retract accordingly. For example, at extreme high frequencies (far above LVK band), complex branch structure of QCA might significantly affect dispersion form.

\subsection{Relation to SME and Other Discrete Models}

Standard-Model Extension provides systematic parameterization framework for Lorentz violation; its gravity sector dispersion relations can map formally to QCA dispersion in this paper. The difference is SME views Lorentz violation as high-dimensional operators in continuous field theory Lagrangian, while QCA model views it as residual effect of discrete update rules in continuous limit.

Other discrete spacetime schemes (causal set, spin foam) also predict some form of dispersion or propagation correction. Unified Matrix--QCA framework is not in competition with these schemes, but provides a tool connecting ``discrete--continuous--observation'' three-layer structure via unified time scale, usable as effective description or approximation for these schemes.

\subsection{Systematic and Statistical Risks of Observational Constraints}

Current constraints on $\alpha_{\mathrm{disp}}$ and $v_{\mathrm g}$ depend on:

1. Systematic errors in waveform models (tidal effects, spin precession);

2. Non-Gaussianity and non-stationarity of detector noise;

3. Selection effects and statistical biases in multi-event combination.

When converting these constraints to $\ell_{\mathrm{cell}}$ upper bounds, systematic error propagation must be carefully estimated to avoid over-optimism. Future higher sensitivity and larger event samples will help reduce these risks.

\section{Conclusion}

In summary, under the framework of Unified Time Scale, Matrix Universe THE-MATRIX, and QCA Universe, we provide a structurally unified theoretical characterization for ``Gravitational Wave Lorentz Violation and Dispersion Correction''. Core conclusions include:

1. Gravitational wave dispersion can be viewed as high-order deviation of unified time scale density $\kappa(\omega)$ in gravity sector, its magnitude directly controlled by group velocity deviation $\delta v_{\mathrm g}(\omega)$;

2. In Gravity--QCA models satisfying parity and time reversal symmetry, lowest-order correction to gravitational wave dispersion must be of even-order $(k\ell_{\mathrm{cell}})^{2n}$ type, odd-order $k^{2n+1}$ terms are structurally excluded;

3. Using results from parameterized dispersion tests, LVK constraints on $\alpha_{\mathrm{disp}}$ can be directly converted to upper bounds on QCA lattice spacing $\ell_{\mathrm{cell}}$ and dispersion coefficient $\beta_2$, typically giving $\ell_{\mathrm{cell}}\lesssim10^{-29}\text{--}10^{-31}\,\mathrm{m}$ constraint;

4. Null--Modular consistency in unified causal--entropy--time framework further gives $\delta v_{\mathrm g}/c\sim\mathcal O((\omega\ell_{\mathrm{cell}})^2)$ suppression law, consistent with GW170817 and GWTC-3 observations;

5. Engineering-wise, unified model is seamlessly compatible with existing LVK MDR testing procedures, achieving direct constraint on QCA universe discrete structure by simply adding $\ell_{\mathrm{cell}}$ and $\beta_2$ mapping in parameter space.

Future higher-frequency, broadband, and multi-messenger gravitational wave observations will further compress the feasible range of $\ell_{\mathrm{cell}}$, providing clearer answers to fundamental questions like ``is the universe discrete'', ``is time scale truly continuous'', and ``does Lorentz symmetry hold strictly at all energy scales''.

\section*{Acknowledgements}

This work builds upon fundamental work by gravitational wave detection collaborations and extensive literature on Lorentz violation, SME, and QCA. Parameterized dispersion and waveform mapping formulas used in the text can be implemented by open source packages (e.g., Bilby), numerical examples can be obtained by simple modification of existing MDR analysis scripts. Symbolic derivation and continuum limit calculations can be reproduced using general algebra software (Mathematica, Python/SymPy).

\section*{Code Availability}

Code availability statement is consistent with the text above.

\begin{thebibliography}{99}

\bibitem{Abbott2017}
B. P. Abbott et al. (LIGO Scientific Collaboration and Virgo Collaboration), ``Gravitational waves and gamma-rays from a binary neutron star merger: GW170817 and GRB 170817A,'' \textit{Astrophys. J. Lett.} 848, L13 (2017).

\bibitem{Baker2017}
T. Baker, E. Bellini, P. G. Ferreira, M. Lagos, J. Noller, I. Sawicki, ``Strong Constraints on Cosmological Gravity from GW170817 and GRB 170817A,'' \textit{Phys. Rev. Lett.} 119, 251301 (2017).

\bibitem{Mirshekari2012}
S. Mirshekari, N. Yunes, C. M. Will, ``Constraining Lorentz-violating, Modified Dispersion Relations with Gravitational Waves,'' \textit{Phys. Rev. D} 85, 024041 (2012).

\bibitem{Krishnendu2021}
N. V. Krishnendu, K. G. Arun, C. K. Mishra, et al., ``Testing General Relativity with Gravitational Waves,'' review report for LVK tests of GR (2021).

\bibitem{Gong2023}
C. Gong, T. Zhu, R. Niu, Q. Wu, J.-L. Cui, X. Zhang, W. Zhao, A. Wang, ``Gravitational wave constraints on non-birefringent dispersions of gravitational waves due to Lorentz violations with GWTC-3,'' \textit{Phys. Rev. D} 108, 084024 (2023).

\bibitem{Rao2024}
J.-H. Rao, W. Zhao, et al., ``Simulation Study on Constraining Gravitational Wave Propagation Speed and Lorentz Violation,'' \textit{Res. Astron. Astrophys.} 24, 085004 (2024).

\bibitem{Liu2020}
X. Liu, V. F. He, T. M. Mikulski, et al., ``Measuring the Speed of Gravitational Waves from the First and Second Observing Run of Advanced LIGO and Advanced Virgo,'' \textit{Phys. Rev. D} 102, 024028 (2020).

\bibitem{Schreck2022}
M. Schreck, ``Lorentz Violation in Astroparticles and Gravitational Waves,'' \textit{Universe} 10, 13 (2022).

\bibitem{Wang2025}
Q. Wang, W. Zhao, et al., ``Modified Gravitational Wave Propagations in Linearized Gravity in SME,'' (2025).

\bibitem{Kiyota2015}
S. Kiyota, K. Yamamoto, ``Constraint on Modified Dispersion Relations for Gravitational Waves from Gravitational Cherenkov Radiation,'' \textit{Phys. Rev. D} 92, 104036 (2015).

\bibitem{deRham2018}
C. de Rham, ``Gravitational Rainbows: LIGO and Dark Energy at its Cutoff,'' \textit{Phys. Rev. Lett.} 121, 221101 (2018).

\bibitem{Gao2023}
Q. Gao, Y. Gong, et al., ``Constraint on the mass of graviton with gravitational waves,'' \textit{Sci. China Phys. Mech. Astron.} 66, 220412 (2023).

\bibitem{LVK2019}
LIGO–Virgo Collaboration, ``A new constraint on the mass of 'graviton','' (2019).

\bibitem{Mlodinow2025}
L. Mlodinow, et al., ``Bounds on Quantum Cellular Automaton Lattice Spacing from Data on Lorentz Violation,'' (2025).

\bibitem{DAriano2016}
G. M. D'Ariano, N. Mosco, A. Tosini, ``Weyl, Dirac and Maxwell Quantum Cellular Automata,'' \textit{Phys. Rev. A} 93, 062337 (2016).

\bibitem{Brun2019}
T. A. Brun, J. Harrington, M. M. Wilde, ``Detecting Discrete Spacetime via Matter Interferometry,'' \textit{Phys. Rev. D} 99, 015012 (2019).

\bibitem{Ferrari2007}
A. F. Ferrari, M. Gomes, J. R. Nascimento, et al., ``Lorentz Violation in the Linearized Gravity,'' \textit{Phys. Lett. B} 652, 174 (2007).

\bibitem{Harry2022}
I. Harry, S. Nissanke, ``Probing the Speed of Gravity with LVK, LISA, and Joint Observations,'' \textit{Gen. Relativ. Gravit.} 54, 27 (2022).

\bibitem{Loutrel2025}
N. Loutrel, et al., ``Probing Modified Gravitational-Wave Dispersion with Bursts,'' (2025).

\bibitem{Artola2024}
M. Artola, et al., ``Gravitational and Electromagnetic Cherenkov Radiation with Lorentz-Violating Modified Dispersion Relations,'' (2024).

\end{thebibliography}

\appendix

\section{From GR Linear Perturbations to GW Scattering Matrix}

\subsection{Linearized Einstein Equations and Mode Decomposition}

On background metric $g_{\mu\nu}^{(0)}$, consider small perturbation $h_{\mu\nu}$, introducing gauge condition
\begin{equation}
\nabla^\mu h_{\mu\nu}=0,
\qquad
h^\mu_{\ \mu}=0,
\end{equation}
linearized Einstein equations are
\begin{equation}
\square h_{\mu\nu}
+2R_{\mu\alpha\nu\beta}^{(0)}h^{\alpha\beta}
=0.
\end{equation}

On flat background $g_{\mu\nu}^{(0)}=\eta_{\mu\nu}$, $R_{\mu\alpha\nu\beta}^{(0)}=0$, equation degenerates to
\begin{equation}
\square h_{\mu\nu}=0,
\end{equation}
plane wave solution is
\begin{equation}
h_{\mu\nu}(t,\mathbf{x})
=\epsilon_{\mu\nu}(\hat{\mathbf{k}})\,
\mathrm{e}^{-\mathrm{i}(\omega t-\mathbf{k}\cdot\mathbf{x})},
\qquad
\omega^2=c^2k^2.
\end{equation}

In spherically symmetric static background (e.g., Schwarzschild exterior), projecting perturbation onto Regge--Wheeler or Zerilli modes reduces to radial equation
\begin{equation}
-\frac{\mathrm{d}^2\psi_\ell}{\mathrm{d}r_\ast^2}
+V_{\mathrm{eff},\ell}(r_\ast)\psi_\ell
=\omega^2\psi_\ell,
\end{equation}
where $r_\ast$ is tortoise coordinate, $V_{\mathrm{eff},\ell}$ is effective potential. Boundary conditions are
\begin{equation}
\psi_\ell(r_\ast)
\sim
\begin{cases}
\mathrm{e}^{-\mathrm{i}\omega r_\ast}
+A_\ell^{\mathrm{out}}\mathrm{e}^{+\mathrm{i}\omega r_\ast},
& r_\ast\to-\infty,\\[1ex]
B_\ell^{\mathrm{out}}\mathrm{e}^{+\mathrm{i}\omega r_\ast},
& r_\ast\to+\infty.
\end{cases}
\end{equation}

After normalization, scattering coefficients $S_\ell(\omega)$ and corresponding phase shifts $\delta_\ell(\omega)$ are obtained, constituting angular momentum components of scattering matrix $S_{\mathrm{GW}}(\omega)$.

\subsection{Wigner--Smith Matrix and Group Delay}

For each $\ell$ and polarization, define channel amplitudes $a_{\mathrm{in}},a_{\mathrm{out}}$, such that
\begin{equation}
a_{\mathrm{out}}(\omega)
=S_{\mathrm{GW}}(\omega)\,a_{\mathrm{in}}(\omega),
\end{equation}
$S_{\mathrm{GW}}(\omega)\in U(N_{\mathrm{GW}})$. Wigner--Smith matrix is defined as
\begin{equation}
Q_{\mathrm{GW}}(\omega)
=-\mathrm{i} S_{\mathrm{GW}}^\dagger(\omega)
\partial_\omega S_{\mathrm{GW}}(\omega).
\end{equation}

If $S_{\mathrm{GW}}(\omega)$ can be diagonalized as
\begin{equation}
S_{\mathrm{GW}}(\omega)
=\sum_{j=1}^{N_{\mathrm{GW}}}
\mathrm{e}^{2\mathrm{i}\delta_j(\omega)}\Pi_j,
\end{equation}
then
\begin{equation}
Q_{\mathrm{GW}}(\omega)
=2\sum_j\partial_\omega\delta_j(\omega)\,\Pi_j,
\qquad
\tau_j(\omega)
=2\partial_\omega\delta_j(\omega).
\end{equation}

In far-field flat background limit, relation between derivatives of phase shifts, propagation distance $L$, and group velocity $v_{\mathrm g}(\omega)$ is
\begin{equation}
\tau_j(\omega)
\approx\frac{L}{v_{\mathrm g}(\omega)}+c_j(\omega),
\end{equation}
where $c_j(\omega)$ is frequency slowly varying term related to local scattering. Taking trace and ignoring $c_j(\omega)$ contribution, we obtain relation between $\kappa_{\mathrm{GW}}(\omega)$ and $v_{\mathrm g}(\omega)$ in main text.

\section{Gravity--QCA Continuum Limit and Dispersion Expansion}

\subsection{One-Dimensional Simplified QCA Model}

Consider 1D lattice $\Lambda=\mathbb Z$, cellular Hilbert space $\mathcal H_x=\mathbb C^2$ representing two polarizations, spin operators denoted by Pauli matrices $\sigma_i$. Define two types of local gates:

1. Hopping gate $U_{\mathrm{hop}}$, exchanging amplitudes between adjacent cells:
   \begin{equation}
   U_{\mathrm{hop}}
   =\prod_x
   \exp\bigl[
   -\mathrm{i}\theta\,
   (\ket{x+1}\bra{x}\otimes\sigma_z+\mathrm{h.c.})
   \bigr];
   \end{equation}

2. ``Curvature'' gate $U_{\mathrm{grav}}$, applying local phase on each cell:
   \begin{equation}
   U_{\mathrm{grav}}
   =\prod_x
   \exp\bigl[
   -\mathrm{i}\phi(\hat{p})\,\sigma_z
   \bigr],
   \end{equation}
   where $\phi(\hat{p})$ is some function of momentum operator.

Overall update is
\begin{equation}
U
=U_{\mathrm{grav}}U_{\mathrm{hop}}.
\end{equation}

In momentum representation, $U(k)$ can be written as
\begin{equation}
U(k)
=\exp\bigl(
-\mathrm{i} H_{\mathrm{eff}}(k)\Delta t
\bigr),
\end{equation}
where
\begin{equation}
H_{\mathrm{eff}}(k)
=c k\sigma_z
+\gamma_2 k^3\ell_{\mathrm{cell}}^2\sigma_z
+\mathcal O(k^5\ell_{\mathrm{cell}}^4),
\end{equation}
$c$ and $\gamma_2$ are constants determined by $\theta,\phi$.

Spectrum of $H_{\mathrm{eff}}^2$:
\begin{equation}
\omega^2
=H_{\mathrm{eff}}^2/\Delta t^2
=c^2k^2
\left[1+2\frac{\gamma_2}{c}k^2\ell_{\mathrm{cell}}^2
+\mathcal O(k^4\ell_{\mathrm{cell}}^4)\right],
\end{equation}
thus
\begin{equation}
\beta_2=2\gamma_2/c,
\end{equation}
obtaining dispersion coefficient expression for 1D case in main text.

\subsection{High-Dimensional and Anisotropic Generalization}

In high dimensions, $U(\mathbf{k})$ is a multivariate function, its spectrum can be written as
\begin{equation}
\omega^2(\mathbf{k})
=c^2k^2
\left[1+\sum_{n\ge1}
\beta_{2n}(\hat{\mathbf{k}})(k\ell_{\mathrm{cell}})^{2n}\right].
\end{equation}

Anisotropy is embodied by angular dependence of $\beta_{2n}(\hat{\mathbf{k}})$. If lattice and gate symmetry is sufficiently high (e.g., cubic lattice and isotropic local gates), then in low-order approximation $\beta_{2n}(\hat{\mathbf{k}})\approx\beta_{2n}$ can be treated as constant. For gravitational wave observations, angular anisotropy can be effectively smoothed out by averaging over multiple events and directions; its residual effects can be used as advanced metrics to test finer QCA structures.

\section{Numerical Illustration of Dispersion Parameters, Observational Constraints, and QCA Lattice Spacing}

\subsection{$n=2$ Type Dispersion and Energy Scale}

Consider
\begin{equation}
\omega^2
=c^2k^2+\alpha_{\mathrm{disp}}k^4,
\qquad
\alpha_{\mathrm{disp}}
=\sigma\frac{c^2}{M_\ast^2},
\end{equation}
where $M_\ast$ is energy scale. Using natural units $c=\hbar=1$, conversion is $1\,\mathrm{GeV}^{-1}\approx2\times10^{-16}\,\mathrm{m}$.

If observational constraint gives
\begin{equation}
M_\ast\gtrsim10^{14}\,\mathrm{GeV},
\end{equation}
then
\begin{equation}
M_\ast^{-1}\lesssim10^{-14}\,\mathrm{GeV}^{-1}
\approx2\times10^{-30}\,\mathrm{m}.
\end{equation}

In QCA mapping,
\begin{equation}
\ell_{\mathrm{cell}}
\lesssim M_\ast^{-1}\lvert\beta_2\rvert^{-1/2},
\end{equation}
if $\beta_2\sim1$, then
\begin{equation}
\ell_{\mathrm{cell}}
\lesssim 2\times10^{-30}\,\mathrm{m},
\end{equation}
about $10^5$ times Planck length $\ell_{\mathrm{Pl}}\sim10^{-35}\,\mathrm{m}$.

If future observations raise $M_\ast$ to $10^{15}\text{--}10^{16}\,\mathrm{GeV}$, then $\ell_{\mathrm{cell}}$ upper bound will further drop to $10^{-31}\text{--}10^{-32}\,\mathrm{m}$ range.

\subsection{Comparison with GW170817 Speed Constraint}

GW170817 and GRB 170817A give
\begin{equation}
\left\lvert\frac{v_{\mathrm g}}{c}-1\right\rvert
\lesssim 10^{-15},
\end{equation}
under condition $f\sim10^2\,\mathrm{Hz}$, $L\sim40\,\mathrm{Mpc}$, corresponding to
\begin{equation}
\biggl\lvert\frac{\delta v_{\mathrm g}}{c}\biggr\rvert
\lesssim 10^{-15}.
\end{equation}

In QCA model,
\begin{equation}
\frac{\delta v_{\mathrm g}}{c}
\simeq\beta_2(k\ell_{\mathrm{cell}})^2,
\qquad
k\sim\frac{2\pi f}{c}
\sim10^{-6}\,\mathrm{m}^{-1},
\end{equation}
so
\begin{equation}
\ell_{\mathrm{cell}}
\lesssim \frac{10^{-7.5}}{\sqrt{\lvert\beta_2\rvert}}\,\mathrm{m}
\sim10^{-8}\,\mathrm{m}
\quad(\beta_2\sim1),
\end{equation}
this is an extremely loose upper bound. What truly drives $\ell_{\mathrm{cell}}$ into $10^{-29}\text{--}10^{-31}\,\mathrm{m}$ range is cumulative dispersion analysis of waveform phase, not simple arrival time difference measurement. This explains why GWTC-3 level multi-event statistical analysis is needed to obtain strong constraints on high-dimensional operators and QCA lattice spacing.

\subsection{Comprehensive Constraints with EM and Matter Experiments}

Electromagnetic and matter experiments test Lorentz violation at higher energies and longer baselines, providing constraints on $\alpha_{\mathrm{disp}}$ or SME coefficients that can reach extreme precision. Converting these results to upper bounds on QCA lattice spacing, obtained $\ell_{\mathrm{cell}}$ upper bounds often overlap with gravitational wave constraints in $10^{-29}\text{--}10^{-32}\,\mathrm{m}$ range. This indicates:

1. If Unified Matrix--QCA universe model is correct, universe discrete lattice spacing is likely located in this interval or below;

2. Gravitational wave channel and electromagnetic/matter channels provide complementary and corroborative constraints, building a unified framework for observational testing of ``universe discrete structure''.

\end{document}

