\documentclass[12pt]{article}

% Essential packages
\usepackage[utf8]{inputenc}
\usepackage[T1]{fontenc}
\usepackage{amsmath,amssymb,amsthm}
\usepackage{mathrsfs}
\usepackage{geometry}
\usepackage{hyperref}
\usepackage{braket}
\usepackage{graphicx}

% Geometry settings
\geometry{a4paper, margin=1in}

% Hyperref settings
\hypersetup{
    colorlinks=true,
    linkcolor=blue,
    citecolor=blue,
    urlcolor=blue
}

% Theorem environments
\theoremstyle{plain}
\newtheorem{theorem}{Theorem}[section]
\newtheorem{lemma}[theorem]{Lemma}
\newtheorem{proposition}[theorem]{Proposition}
\newtheorem{corollary}[theorem]{Corollary}

\theoremstyle{definition}
\newtheorem{definition}[theorem]{Definition}
\newtheorem{example}[theorem]{Example}
\newtheorem{remark}[theorem]{Remark}

% Title information
\title{Topological--Scattering Solution of Strong CP Problem and Axion in Unified Matrix--QCA Universe}

\author{Haobo Ma$^1$ \and Wenlin Zhang$^2$\\
\small $^1$Independent Researcher\\
\small $^2$National University of Singapore}

\date{\today}

\begin{document}

\maketitle

\begin{abstract}
Quantum chromodynamics (QCD) in its most general form admits a topological $\theta$-term $\theta\, g_s^2(32\pi^2)^{-1} G_{\mu\nu}^a \tilde G^{a,\mu\nu}$, which violates $P$, $T$ and $CP$. After chiral field redefinitions, the physically observable strong $CP$ angle is $\bar\theta=\theta-\arg\det(Y_u Y_d)$, where $Y_u$ and $Y_d$ are the up- and down-type Yukawa matrices. Naturalness suggests $\bar\theta=\mathcal O(1)$, while neutron electric dipole moment (nEDM) bounds $|d_n|\lesssim 1.8\times 10^{-26}\,e\cdot\mathrm{cm}$ imply $|\bar\theta|\lesssim 10^{-10}$, constituting the strong $CP$ problem.

Peccei--Quinn (PQ) theory promotes $\bar\theta$ to a dynamical vacuum expectation value of an axion field. The axion potential is determined by the QCD topological susceptibility $\chi_{\mathrm{top}}$, with $m_a^2 f_a^2=\chi_{\mathrm{top}}$ and $V(a)\simeq \chi_{\mathrm{top}}[1-\cos(a/f_a-\bar\theta_0)]$. Lattice QCD and chiral effective theory now determine $\chi_{\mathrm{top}}(T)$ with high precision, fixing the QCD axion mass--coupling relation. However, in a broader unified description of the Universe, the origin and robustness of PQ symmetry against gravity, ultraviolet physics and global consistency conditions remain unclear.

Within the unified time-scale, boundary time geometry, matrix universe THE-MATRIX and quantum cellular automaton (QCA) universe framework, this work gives a topological--scattering solution of the strong $CP$ problem. The main ideas are:

1. Introduce a parameter space $X^\circ$ of all low-energy couplings (gauge couplings, $\theta$-angles, Yukawa phases, light scalar parameters) and an extended space $Y=M\times X^\circ$, where $M$ is spacetime. From the family of scattering matrices $S(\omega;\lambda)$ on a channel Hilbert space, construct a determinant line bundle $\mathcal L_{\mathrm{det}}\to Y$ and its square root $\mathcal L_{\mathrm{det}}^{1/2}$. The obstruction to a global smooth square root is encoded in a relative cohomology class $[K]\in H^2(Y,\partial Y;\mathbb Z_2)$, which has the physical meaning of the global $\mathbb Z_2$ holonomy of the "square-root scattering determinant" $\sqrt{\det_p S}$ along parameter loops.

2. Using the previously developed Null--Modular double cover and restricted unitary bundle framework, one has an equivalence between: (i) local Einstein equations with appropriate quantum energy conditions, (ii) small causal diamond generalized entropy extremality and modular flow consistency, and (iii) vanishing of the obstruction class, $[K]=0$. Thus, in any Universe admitting a globally consistent boundary time geometry and semiclassical gravity, allowed physical sectors must satisfy $[K]=0$.

3. Embed QCD and its $\theta$-angle into this unified structure by identifying the QCD sector contribution $[K_{\mathrm{QCD}}]$ of $[K]$. The physical strong $CP$ angle $\bar\theta=\theta-\arg\det(Y_u Y_d)$ reappears as the phase holonomy of $\sqrt{\det_p S_{\mathrm{QCD}}}$ along loops in $X^\circ$. One shows that $[K_{\mathrm{QCD}}]=0$ is equivalent to the triviality (modulo $2\pi$) of all such holonomies; in particular, in any physically realized sector compatible with $[K]=0$ one has $\bar\theta_{\mathrm{eff}}\approx 0$ without requiring a vanishing quark mass.

4. In the matrix universe representation, the global Universe is a gigantic but structured unitary matrix whose block-sparse pattern encodes causal relations and whose spectral data encode the unified time-scale. In this picture, $\bar\theta$ is a "topological phase" of the QCD block of THE-MATRIX, and $[K]=0$ requires that the square-root determinant has trivial $\mathbb Z_2$ holonomy across all coupling loops. Strong $CP$ is then rephrased as the requirement that such topological scattering invariants vanish globally.

5. In the QCA universe layer, one constructs an $SU(3)$ gauge QCA with lattice topological charge $Q\in\mathbb Z$. The QCD $\theta$-term corresponds to a weight factor $\exp(\mathrm i\theta Q)$ in the discrete path-sum. By imposing a "topological--Null--Modular consistent QCA" condition that the total phase for all closed gauge-configuration histories be $2\pi\mathbb Z$, one obtains in the continuum limit a joint constraint on $\bar\theta$ and possible gravitational $\theta_G$-terms, thereby simultaneously suppressing strong and gravitational $CP$ violation.

6. The PQ axion is reinterpreted as a relative cohomology modulus of $[K]$. The axion field $a(x)/f_a$ parametrizes local rephasings of $\sqrt{\det_p S}$ along a $U(1)$ fiber of $\mathcal L_{\mathrm{det}}^{1/2}$. Its effective action reproduces the standard form $S[a]\sim\int[\tfrac12 f_a^2 (\partial a)^2+\chi_{\mathrm{top}}(1-\cos(a/f_a-\bar\theta_0))]\sqrt{-g}\,\mathrm d^4x$, where $\chi_{\mathrm{top}}$ is the QCD topological susceptibility determined from first-principles QCD. The global condition $[K]=0$ then enforces $\langle a\rangle/f_a=\bar\theta_0$ and $\bar\theta_{\mathrm{eff}}=0$, giving a unified topological–scattering reformulation of the PQ mechanism.

Appendices present standard QCD derivations of $\bar\theta$, the precise construction of $[K]$ from scattering theory, and explicit $SU(3)$ gauge QCA models with discrete topological charge and $\theta$-phase. The resulting picture treats strong $CP$ as a consistency constraint of the full matrix–QCA Universe rather than an independent fine-tuning of a low-energy parameter.
\end{abstract}

\noindent\textbf{Keywords:} Strong $CP$ problem; QCD axion; Peccei–Quinn mechanism; Topological susceptibility; Scattering determinant line bundle; Matrix Universe; Quantum Cellular Automata (QCA); Null–Modular double cover

\section{Introduction \& Historical Context}

\subsection{The Strong \texorpdfstring{$CP$}{CP} Problem}

Four-dimensional $SU(3)$ Yang--Mills theory admits a topological term in its Lagrangian:
\begin{equation}
\mathcal L_\theta = \theta\,\frac{g_s^2}{32\pi^2} G_{\mu\nu}^a \tilde G^{a,\mu\nu},\quad
\tilde G^{a,\mu\nu}=\tfrac12\epsilon^{\mu\nu\rho\sigma}G^a_{\rho\sigma}.
\end{equation}
The space-time integral of this term converts to the product of the topological charge $Q\in\mathbb Z$ and the angle $\theta$, explicitly violating $P$, $T$, and $CP$. For QCD with multiple generations of quarks, after introducing Yukawa matrices $Y_u, Y_d$, considering chiral redefinitions and anomaly effects, the physically observable strong $CP$ angle is
\begin{equation}
\bar\theta = \theta - \arg \det(Y_u Y_d),
\end{equation}
which naturally takes values in $[0, 2\pi)$.

Without additional symmetries or dynamical mechanisms, $\bar\theta$ would be expected to be $\mathcal O(1)$. However, $\bar\theta\neq 0$ induces a neutron electric dipole moment (nEDM). QCD sum rules and chiral effective theory give
\begin{equation}
d_n \simeq c_n \bar\theta\, e\cdot\mathrm{cm},\quad c_n\sim 10^{-16}.
\end{equation}
Current nEDM experiments provide an upper bound $|d_n|<1.8\times 10^{-26}\,e\cdot\mathrm{cm}$, implying
\begin{equation}
|\bar\theta|\lesssim 10^{-10}.
\end{equation}
This is the strong $CP$ problem: why an angle that should naturally be $\mathcal O(1)$ is fine-tuned to the order of $10^{-10}$.

\subsection{Traditional Solutions and the QCD Axion}

Traditional solutions to the strong $CP$ problem generally fall into a few categories:
1.  **Massless Up Quark:** If $m_u=0$, $\bar\theta$ can be removed via chiral rotation. Lattice QCD has essentially ruled out this possibility.
2.  **Spontaneous $P$ or $CP$ Breaking:** $P$ or $CP$ is a fundamental symmetry broken spontaneously at a high energy scale, with the vacuum choosing $\bar\theta=0$. Such models need to address the nEDM induced after spontaneous $CP$ breaking and the characterization of discrete symmetries in quantum gravity.
3.  **Peccei--Quinn Mechanism:** Introduces a global $U(1)_{\mathrm{PQ}}$ symmetry and a scalar field. Its anomaly structure makes the QCD vacuum energy dependent on an angular field $a/f_a$, automatically adjusting $\bar\theta_{\mathrm{eff}}=0$ at the minimum of the vacuum energy. The corresponding approximate Goldstone mode is the QCD axion. While the original PQ model is ruled out, "invisible axion" models like KSVZ and DFSZ remain mainstream solutions.

In the PQ framework, the axion effective potential is determined by the QCD topological susceptibility $\chi_{\mathrm{top}}$:
\begin{equation}
V(a) \simeq \chi_{\mathrm{top}}\bigl[1-\cos(a/f_a - \bar\theta_0)\bigr],
\quad m_a^2 f_a^2 = \chi_{\mathrm{top}}.
\end{equation}
Lattice QCD and chiral effective theory have provided high-precision results for $\chi_{\mathrm{top}}(T)$, precisely determining the relationship between QCD axion mass and coupling constant.

\subsection{Unified Universe Framework and Problem Restatement}

The above solutions mostly address the strong $CP$ problem at the level of low-energy effective field theory, without embedding it into a unified framework containing gravity, causal structure, boundary time geometry, and global consistency of the Universe. Furthermore, quantum gravity and black hole thermodynamics suggest that global symmetries might be violated in quantum gravity, questioning the stability of the global $U(1)_{\mathrm{PQ}}$ in the traditional PQ mechanism.

Previous works introduced the multi-layer objects of unified time scale, boundary time geometry, Matrix Universe THE-MATRIX, and QCA Universe, characterizing the Universe as a single structure highly consistent across scattering, geometry, modular flow, generalized entropy, and causal partial order. Key points include:

1.  **Unified Time Scale Mother Formula:**
    \begin{equation}
    \kappa(\omega) = \frac{\varphi'(\omega)}{\pi}
    = \rho_{\mathrm{rel}}(\omega)
    = (2\pi)^{-1} \operatorname{tr} Q(\omega),
    \end{equation}
    unifying scattering hemi-phase derivative, relative density of states, and Wigner--Smith group delay matrix trace into a single time density.

2.  **Boundary Time Geometry and Null--Modular Double Cover:** Gluing modular flow parameters and scattering phases on the boundary of each causal diamond, constraining geometry, entropy, and scattering mutually.

3.  **Matrix Universe and QCA Universe:** Viewing the entire cosmic evolution as a gigantic but sparse unitary matrix or reversible QCA, where block structure encodes causal partial order, spectral data implements time scale, and feedback loops carry $\mathbb Z_2$ topology.

In this framework, the strong $CP$ problem can be restated as: why is the QCD sector's contribution to the topological--scattering invariants of the whole Universe suppressed to nearly zero? This paper will show that this suppression is no longer a "fine-tuning of low-energy constants" but a mandatory result of "unified universe topological--scattering consistency".

\section{Model \& Assumptions}

This section presents the models and basic assumptions for the unified universe structure, parameter space, scattering determinant line bundle, and QCD sector embedding used in this paper.

\subsection{Universe Object and Parameter Space}

Let the geometric layer of the Universe be described by a globally hyperbolic Lorentzian manifold $(M, g)$ with causal partial order and appropriate boundary structure (including past/future infinity and possible black hole horizons). On this basis, introduce several layers of the unified universe object:
*   Geometric layer $U_{\mathrm{geo}}$: $(M, g, \prec)$;
*   Scattering layer $U_{\mathrm{scat}}$: A family of scattering pairs and unified time scale $\kappa(\omega)$;
*   Boundary layer $U_{\mathrm{BTG}}$: Boundary algebra, modular flow, and Brown--York quasilocal energy;
*   Matrix layer $U_{\mathrm{mat}}$: Scattering matrix universe THE-MATRIX on channel Hilbert space;
*   QCA layer $U_{\mathrm{QCA}}$: SU(3) gauge QCA defined on a countable lattice;
*   Topological layer $U_{\mathrm{top}}$: Relative cohomology class $[K]$ and Null--Modular double cover.

To introduce coupling parameters, define an open parameter space $X^\circ$ whose coordinates include:
1.  Gauge coupling constants $g_s, g, g'$, etc.;
2.  Yukawa matrices and CKM/PMNS phase parameterizations;
3.  QCD $\theta$-angle, possible gravitational $\theta_G$ angle, and other topological term parameters;
4.  Light scalars (including possible axions) and external macroscopic parameters.

Define the extended space
\begin{equation}
Y = M \times X^\circ,
\end{equation}
whose boundary $\partial Y$ includes spacetime boundaries and possible boundaries of the parameter space.

\subsection{Scattering Matrix Family and Determinant Line Bundle}

Under appropriate locality and spectral assumptions, for each parameter point $\lambda \in X^\circ$ and energy $\omega$, consider a scattering pair $(H_0, H(\lambda))$ and its wave operators, yielding a scattering matrix on the channel Hilbert space $\mathcal H_{\mathrm{chan}}(\omega)$:
\begin{equation}
S(\omega;\lambda):\mathcal H_{\mathrm{chan}}(\omega)\to \mathcal H_{\mathrm{chan}}(\omega).
\end{equation}
Assume $S(\omega;\lambda)-\mathbf 1$ is a trace-class operator of appropriate order, such that the modified determinant $\det{}_p S(\omega;\lambda)$ is defined and varies smoothly with $(\omega, \lambda)$ (under spectral gap and appropriate renormalization).

From this family of unitary operators on the vector bundle $\mathcal H_{\mathrm{chan}}$ over $(\omega, \lambda) \in Y$, one can construct a complex line bundle $\mathcal L_{\mathrm{det}} \to Y$, whose fiber is generated by the formal "determinant":
\begin{equation}
\mathcal L_{\mathrm{det}}|_{(\omega,\lambda)}
\cong \mathbb C\cdot \det{}_p S(\omega;\lambda).
\end{equation}
Locally, one can choose a logarithmic phase $\phi(\omega, \lambda)$ such that
\begin{equation}
\det{}_p S(\omega;\lambda)=\exp\bigl(\mathrm i \phi(\omega,\lambda)\bigr).
\end{equation}
Define the square root line bundle $\mathcal L_{\mathrm{det}}^{1/2}$ as formally satisfying
\begin{equation}
(\mathcal L_{\mathrm{det}}^{1/2})^{\otimes 2}
\simeq \mathcal L_{\mathrm{det}}.
\end{equation}
Locally, one can choose $\sqrt{\det{}_p S} = \exp(\tfrac{\mathrm i}{2}\phi)$, but globally there may be a sign ambiguity. The $\mathbb Z_2$ twisting of this ambiguity is characterized by a relative cohomology class
\begin{equation}
[K]\in H^2(Y,\partial Y;\mathbb Z_2).
\end{equation}
$[K]=0$ if and only if there exists a global square root line bundle trivial on $\partial Y$. This construction can be seen as a scattering version generalization of Freed et al.'s work on determinant line bundles and cohomological obstructions.

\subsection{Null--Modular Double Cover and Consistency Conditions}

Boundary time geometry equips the boundary $\partial D$ of each small causal diamond $D \subset M$ with:
1.  Boundary observable algebra $\mathcal A_\partial(D)$ and its state $\omega_\partial(D)$;
2.  Modular flow parameter $s \in \mathbb R$ and modular operator $\Delta^{\mathrm i s}$;
3.  Phase $\varphi_D(\omega)$ and time scale $\kappa_D(\omega)$ given by scattering matrix $S_D(\omega)$ and Wigner--Smith group delay $Q_D(\omega)$.

The Null--Modular double cover glues the modular flow parameter and scattering phase via a $\mathbb Z_2$ cover, ensuring consistency between modular flow direction, time arrow, and phase single-valuedness, avoiding "half-cycle sign reversal" anomalies. Previous work showed that under assumptions of local Einstein equations, QNEC/QFC, and scattering--modular consistency, the following conditions are equivalent:
1.  Existence of a global Null--Modular double cover such that modular flow and scattering phase align smoothly on this cover for all causal diamonds;
2.  Generalized entropy extremality conditions and non-negativity of second-order relative entropy on each causal diamond are equivalent to local gravitational field equations;
3.  The relative cohomology class of the determinant square root line bundle satisfies $[K]=0$.

Thus, for an acceptable physical universe sector, $[K]=0$ is not optional but a consistency requirement.

\subsection{QCD Sector and Embedding of $\bar\theta$}

In the QCD sector, consider $SU(3)$ gauge fields $A_\mu^a$ and six flavors of quarks $\psi_f$. The Lagrangian includes
\begin{equation}
\mathcal L_{\mathrm{QCD}} = -\tfrac14 G_{\mu\nu}^a G^{a,\mu\nu}
+ \bar\psi(i\gamma^\mu D_\mu)\psi
- (\bar u_L Y_u u_R + \bar d_L Y_d d_R + \mathrm{h.c.})
+ \theta \frac{g_s^2}{32\pi^2}G_{\mu\nu}^a \tilde G^{a,\mu\nu}.
\end{equation}
Chiral redefinition and Fujikawa anomaly analysis give the physical angle
\begin{equation}
\bar\theta = \theta - \arg\det(Y_u Y_d),
\end{equation}
appearing in the path integral in the weight factor $\exp(\mathrm i\bar\theta Q)$.

Viewing the QCD sector as a block of the Matrix Universe THE-MATRIX, for each parameter $\lambda$ and energy $\omega$, there is a scattering matrix $S_{\mathrm{QCD}}(\omega;\lambda)$. The determinant and square root of this family on the parameter space define a contribution $[K_{\mathrm{QCD}}]$ to $[K]$. Its physical meaning is: along a closed loop in parameter space containing $\bar\theta$ variation, does $\sqrt{\det_p S_{\mathrm{QCD}}}$ undergo an irremovable sign flip?

\subsection{SU(3) Gauge QCA Universe Layer}

In the QCA layer, construct an SU(3) gauge QCA: lattice vertices carry matter Hilbert spaces, edges carry gauge link variables $U_{x,\mu} \in SU(3)$, and discrete time evolution is implemented by local gauge-covariant unitary gates $U_{\mathrm{QCA}}$. The discrete topological charge $Q \in \mathbb Z$ can be defined by summing $\operatorname{Tr}(U_P \tilde U_P)$ over lattice points.

In this setting, the QCD $\theta$-term corresponds to adding a phase factor in the discrete path sum:
\begin{equation}
\prod_n \exp\bigl(\mathrm i\theta Q_n\bigr),
\end{equation}
or in the Heisenberg picture as $U_{\mathrm{QCA}}(\theta)=\exp(\mathrm i\theta \hat Q) U_{\mathrm{QCA}}(0)$. Topological--Null--Modular consistency requires that the total phase for any closed gauge configuration loop be $2\pi\mathbb Z$, which will be used to constrain $\bar\theta$.

\section{Main Results (Theorems and Alignments)}

This section formalizes the main conclusions about the strong $CP$ problem and axion within the unified Matrix--QCA universe framework into several theorems and corollaries. Proofs are in the Appendix.

\subsection{Theorem 1 (Determinant Square Root and Relative Cohomology Class)}

\begin{theorem}[Determinant Square Root and Relative Cohomology Class]
Under the aforementioned scattering and parameter space assumptions, the scattering matrix family $S(\omega;\lambda)$ defines a complex line bundle $\mathcal L_{\mathrm{det}} \to Y$, generated by the modified determinant $\det{}_p S(\omega;\lambda)$. There exists a natural relative cohomology class
\begin{equation}
[K]\in H^2(Y,\partial Y;\mathbb Z_2),
\end{equation}
satisfying:
1.  $[K]=0$ if and only if there exists a square root line bundle $\mathcal L_{\mathrm{det}}^{1/2}$ trivial on $\partial Y$, such that $(\mathcal L_{\mathrm{det}}^{1/2})^{\otimes 2}\cong \mathcal L_{\mathrm{det}}$;
2.  For any closed parameter loop $\gamma \subset X^\circ$, the evaluation of $[K]$ on the mapping torus $M \times S^1_\gamma$ gives the $\mathbb Z_2$ holonomy of $\sqrt{\det_p S}$ along $\gamma$: if the holonomy is $+1$, a smooth square root can be chosen on the loop; if $-1$, there is an irremovable sign flip.
\end{theorem}

\subsection{Theorem 2 (Null--Modular Consistency and $[K]=0$)}

\begin{theorem}[{Null--Modular Consistency and $[K]=0$}]
Assume the Universe satisfies:
1.  $M$ is globally hyperbolic, with appropriate boundary time geometry structure and modular flow;
2.  Generalized entropy extremality and non-negativity of second-order relative entropy on small causal diamonds hold and are equivalent to local gravitational field equations;
3.  Boundary scattering matrices and modular flows can be locally aligned on a Null--Modular double cover.

Then the following propositions are equivalent:
1.  There exists a global Null--Modular double cover such that modular flow and scattering phase on all causal diamonds align smoothly on this cover without $\mathbb Z_2$ anomaly;
2.  There exists a global square root line bundle $\mathcal L_{\mathrm{det}}^{1/2}$ trivial on $\partial Y$;
3.  $[K]=0$.

Thus, for any physically acceptable universe sector, $[K]=0$ is a necessary condition.
\end{theorem}

\subsection{Theorem 3 (QCD $\bar\theta$ as Scattering--Topological Holonomy)}

\begin{theorem}[QCD $\bar\theta$ as Scattering--Topological Holonomy]
In the QCD sector, define the physical angle
\begin{equation}
\bar\theta(\lambda) = \theta(\lambda) - \arg\det\bigl(Y_u(\lambda) Y_d(\lambda)\bigr),
\end{equation}
and consider a closed loop $\gamma \subset X^\circ$ in parameter space. Then:
1.  The evaluation of $[K_{\mathrm{QCD}}]$ on $M \times S^1_\gamma$ equals the sign of $\exp(\mathrm i\Delta_\gamma\bar\theta/2)$, where $\Delta_\gamma\bar\theta$ is the total phase change along $\gamma$ (modulo $2\pi$);
2.  If $\Delta_\gamma\bar\theta \equiv 0 \pmod{2\pi}$ for all physically realizable loops $\gamma$, then $[K_{\mathrm{QCD}}]=0$;
3.  Conversely, if there exists a loop where $\Delta_\gamma\bar\theta \equiv \pi \pmod{2\pi}$, then $[K_{\mathrm{QCD}}] \neq 0$.

Specifically, in a local parameter neighborhood, $[K_{\mathrm{QCD}}]=0$ requires that a gauge can be chosen such that $\bar\theta_{\mathrm{eff}} \equiv 0 \pmod{2\pi}$.
\end{theorem}

\subsubsection{Corollary 3.1 (Strong $CP$ Suppression in Unified Universe)}
Assume the Universe satisfies the assumptions of Theorem 2, and non-QCD sector contributions to $[K]$ are constrained to zero by independent consistency conditions. Then $[K]=0 \Rightarrow [K_{\mathrm{QCD}}]=0$. Thus:
1.  Physically realizable universe sectors must satisfy $\bar\theta_{\mathrm{eff}} \equiv 0 \pmod{2\pi}$;
2.  The experimentally observed $|\bar\theta| \lesssim 10^{-10}$ is interpreted as an extremely sensitive upper bound on possible residuals of $[K_{\mathrm{QCD}}]$, while the natural expectation is strictly zero.

\subsection{Theorem 4 (Axion as Relative Cohomology Modulus of $[K]$)}

\begin{theorem}[{Axion as Relative Cohomology Modulus of $[K]$}]
Assume $[K]$ lifts to an integer coefficient class $\tilde K \in H^2(Y,\partial Y;\mathbb Z)$. Let the phase of the axion field be
\begin{equation}
\phi(x) = \exp\bigl(\mathrm i a(x)/f_a\bigr)\in U(1),
\end{equation}
and embed it into QCD via the coupling
\begin{equation}
\mathcal L_{aG\tilde G} = \frac{a}{f_a}\frac{g_s^2}{32\pi^2}G_{\mu\nu}^a \tilde G^{a,\mu\nu}.
\end{equation}
Then:
1.  The variation of $\phi$ can be viewed as a local rescaling of the square root line bundle $\mathcal L_{\mathrm{det}}^{1/2}$, and $\tilde K$ is the first Chern class of this $U(1)$ line bundle;
2.  In low-energy effective theory, the axion potential can be written as
    \begin{equation}
    V(a) \simeq \chi_{\mathrm{top}}\bigl[1-\cos(a/f_a-\bar\theta_0)\bigr]+\cdots,
    \end{equation}
    where $\chi_{\mathrm{top}}$ is the QCD topological susceptibility, and $\bar\theta_0$ is the bare angle in the UV frame;
3.  If the unified universe requires $[K]=0$, then globally one must have $\langle a\rangle/f_a = \bar\theta_0$, thus $\bar\theta_{\mathrm{eff}} = \bar\theta_0 - \langle a\rangle/f_a = 0$.

This gives a reinterpretation of the PQ mechanism in the unified topological--scattering framework: the axion is the relative cohomology modulus that truly eliminates $[K]$, and its vacuum selection is enforced by $[K]=0$ rather than being an extra freedom.
\end{theorem}

\subsection{Theorem 5 ($\theta$--Sector and $[K_{\mathrm{QCD}}]$ in SU(3) Gauge QCA)}

\begin{theorem}[{$\theta$--Sector and $[K_{\mathrm{QCD}}]$ in SU(3) Gauge QCA}]
In the SU(3) gauge QCA model, let the evolution operator for each time step be
\begin{equation}
U_{\mathrm{QCA}}(\theta) = \exp\bigl(\mathrm i\theta \hat Q\bigr)\, U_{\mathrm{QCA}}(0),
\end{equation}
where $\hat Q$ is the discrete topological charge operator. Consider the discrete path space $\mathcal C$ composed of all possible gauge configuration paths.
1.  For any closed path $\mathcal C$, the total phase
    \begin{equation}
    \Phi(\mathcal C) = \theta \sum_{n\in\mathcal C} Q_n
    \end{equation}
    has a sign (modulo $2\pi$) consistent with the evaluation of $[K_{\mathrm{QCD}}]$ on the corresponding mapping torus;
2.  If one requires $\Phi(\mathcal C) \in 2\pi\mathbb Z$ for all physically realizable closed paths $\mathcal C$, then the effective $\bar\theta_{\mathrm{eff}} \equiv 0 \pmod{2\pi}$ in the discrete model;
3.  In the continuum limit, this condition forces $[K_{\mathrm{QCD}}]=0$ in the QCD sector, compatible with the conclusions of Theorems 3--4.
\end{theorem}

\section{Proofs}

This section outlines the proof frameworks for the main theorems. Details and rigorous constructions are in the Appendix.

\subsection{Sketch of Proof for Theorem 1}
Determinant line bundles and cohomological obstructions to square roots are standard topics.
1.  For each $(\omega, \lambda)$, fix an orthonormal standard channel basis. The spectrum of $S(\omega;\lambda)$ can be written as $\{e^{\mathrm i\delta_j(\omega,\lambda)}\}_j$. Define local hemi-phase
    \begin{equation}
    \phi(\omega,\lambda) = \sum_j \delta_j(\omega,\lambda).
    \end{equation}
2.  On a good open cover $\{U_\alpha\}$ of $Y$, select continuous branches $\phi_\alpha$, defining local sections $s_\alpha=\exp(\mathrm i\phi_\alpha)$ as local trivializations of $\mathcal L_{\mathrm{det}}$.
3.  On intersections $U_\alpha\cap U_\beta$, transition functions $g_{\alpha\beta}=s_\alpha/s_\beta$ take values in $U(1)$. Their projective square roots $h_{\alpha\beta}=\sqrt{g_{\alpha\beta}}$ form candidate transition functions for $\mathcal L_{\mathrm{det}}^{1/2}$.
4.  Whether $h_{\alpha\beta}$ can be chosen to satisfy the cocycle condition $h_{\alpha\beta} h_{\beta\gamma} h_{\gamma\alpha}=1$ is determined by a $\mathbb Z_2$-valued Čech 2-cocycle, corresponding to a class in $H^2(Y;\mathbb Z_2)$.
5.  Adding the boundary condition that the square root is trivial on $\partial Y$ yields the class $[K]$ in relative cohomology $H^2(Y,\partial Y;\mathbb Z_2)$. Its vanishing is necessary and sufficient for the existence of a global square root trivial on the boundary.
Holonomy along a closed loop $\gamma$ is determined by the net change $\Delta_\gamma\phi$, with sign $\exp(\mathrm i\Delta_\gamma\phi/2)$, consistent with the evaluation of $[K]$ on $M\times S^1_\gamma$.

\subsection{Sketch of Proof for Theorem 2}
Null--Modular double cover links boundary modular flow parameter $s$ and scattering phase $\phi(\omega, \lambda)$ such that for each causal diamond $D$:
1.  Time translation generated by modular flow aligns with scattering time scale defined by $\kappa(\omega)$ on the boundary;
2.  Second variation of generalized entropy and non-negativity of second-order relative entropy are equivalent to local Einstein equations.
If $[K]\neq 0$, there exist closed parameter--geometry loops where the holonomy of $\sqrt{\det_p S}$ is $-1$. This means on the Null--Modular double cover, one cannot continuously choose the relative sign of scattering phase and modular flow phase, leading to a "sign reversal" jump when gluing boundaries of certain causal diamonds. This destroys the unified variational principle of generalized entropy--gravity equations.
Conversely, if $[K]=0$, a smooth $\sqrt{\det_p S}$ can be chosen over the entire $Y$, corresponding one-to-one with modular flow parameters on boundaries, constructing an anomaly-free Null--Modular double cover. Consistency of modular flow, generalized entropy, and scattering phase ensures equivalence between the variational principle on small causal diamonds and Einstein equations. This yields the equivalence in Theorem 2.

\subsection{Sketch of Proof for Theorem 3 and Corollary 3.1}
In the QCD sector, the physical angle
\begin{equation}
\bar\theta = \theta - \arg\det(Y_u Y_d)
\end{equation}
appears in the path integral weight $\exp(\mathrm i\bar\theta Q)$. Its contribution to the scattering matrix determinant phase can be viewed as
\begin{equation}
\det{}_p S_{\mathrm{QCD}} \sim \sum_Q P(Q)\exp(\mathrm i\bar\theta Q)\det S_Q,
\end{equation}
where $P(Q)$ is the topological sector weight.
Along a closed loop $\gamma$ in parameter space, $\bar\theta$ may wind by integer multiples of $2\pi$. Since topological charge $Q \in \mathbb Z$, the total phase change is
\begin{equation}
\Delta_\gamma\Phi = \Delta_\gamma\bar\theta\cdot \langle Q\rangle_\gamma + \cdots.
\end{equation}
Considering modulo $2\pi$, if $\Delta_\gamma\bar\theta \equiv 0 \pmod{2\pi}$, a continuous square root can be chosen on the loop without sign flip. If $\Delta_\gamma\bar\theta \equiv \pi \pmod{2\pi}$, the square root must flip sign once, corresponding to $[K_{\mathrm{QCD}}]\neq 0$.
Thus, $[K_{\mathrm{QCD}}]=0$ is equivalent to $\Delta_\gamma\bar\theta \equiv 0 \pmod{2\pi}$ for all physically realizable loops. Assuming non-QCD sector contributions are suppressed to zero by other consistency conditions, $[K]=0 \Rightarrow [K_{\mathrm{QCD}}]=0$, yielding Corollary 3.1: $\bar\theta_{\mathrm{eff}} \equiv 0 \pmod{2\pi}$.

\subsection{Sketch of Proof for Theorem 4}
The phase of the axion field $\phi(x)=\exp(\mathrm i a(x)/f_a)$ can be viewed as the $U(1)$ fiber coordinate on $\mathcal L_{\mathrm{det}}^{1/2}$. Its coupling
\begin{equation}
\frac{a}{f_a}\frac{g_s^2}{32\pi^2}G\tilde G
\end{equation}
introduces a weight $\exp(\mathrm i aQ/f_a)$ in the path integral, interpreted as a local rescaling of $\sqrt{\det_p S}$.
Chiral effective theory and lattice QCD describe the dependence of QCD vacuum energy on $\theta$ as
\begin{equation}
E_0(\theta) = E_0(0) + \tfrac12\chi_{\mathrm{top}}\theta^2 + \mathcal O(\theta^4),
\end{equation}
generalizing to axion coupling yields
\begin{equation}
V(a) \simeq \chi_{\mathrm{top}}\bigl[1-\cos(a/f_a - \bar\theta_0)\bigr],
\end{equation}
where $\bar\theta_0$ is the bare angle in UV frame.
If $[K]=0$, a global square root line bundle exists. With the axion, this condition is equivalent to requiring the winding of $\phi$ and $\bar\theta$ to cancel each other, so the vacuum must satisfy
\begin{equation}
\frac{\langle a\rangle}{f_a} = \bar\theta_0,
\quad \bar\theta_{\mathrm{eff}} = \bar\theta_0 - \frac{\langle a\rangle}{f_a}=0.
\end{equation}
Thus, the PQ mechanism is reinterpreted as introducing the axion field to lift the representative of the integer class $\tilde K$ to trivial, eliminating the $\mathbb Z_2$ projection of $[K]$.

\subsection{Sketch of Proof for Theorem 5}
In the SU(3) gauge QCA model, the topological charge operator $\hat Q$ takes integer values for each discrete configuration $\{U_{x,\mu}\}$. Introducing the phase $\exp(\mathrm i\theta \hat Q)$, the one-step time evolution operator is
\begin{equation}
U_{\mathrm{QCA}}(\theta) = \exp(\mathrm i\theta \hat Q)U_{\mathrm{QCA}}(0).
\end{equation}
For a closed gauge configuration path $\mathcal C$, the total phase is
\begin{equation}
\Phi(\mathcal C) = \theta \sum_{n\in\mathcal C} Q_n.
\end{equation}
Requiring $\Phi(\mathcal C) \in 2\pi\mathbb Z$ for all physically realizable paths $\mathcal C$ implies the effective $\theta_{\mathrm{eff}} \equiv 0 \pmod{2\pi}$. Mapping this condition to the continuum limit and scattering determinant line bundle yields $[K_{\mathrm{QCD}}]=0$, consistent with Theorems 3--4.

\section{Model Apply}

This section applies the unified topological--scattering framework to specific quantities of the strong $CP$ problem, particularly the nEDM bound, QCD topological susceptibility, and axion parameter space.

\subsection{nEDM Bound and $[K_{\mathrm{QCD}}]$ Constraint}
In the axion-less case, the relation between neutron EDM and $\bar\theta$ is
\begin{equation}
d_n \simeq c_n \bar\theta\, e\cdot\mathrm{cm},
\quad c_n\simeq (2\text{--}3)\times 10^{-16}.
\end{equation}
Combined with the experimental bound $|d_n|<1.8\times 10^{-26}\,e\cdot\mathrm{cm}$, we get $|\bar\theta| \lesssim 10^{-10}$.
In the unified universe perspective, $\bar\theta$ is not a free constant but a representation of $[K_{\mathrm{QCD}}]$ in a local parameter coordinate. If $[K_{\mathrm{QCD}}]=0$, there exists a gauge where $\bar\theta_{\mathrm{eff}} \equiv 0 \pmod{2\pi}$, naturally yielding zero nEDM (leaving only electroweak CKM contributions). The nEDM bound is interpreted as a highly sensitive test for deviation from $[K]=0$, rather than fine-tuning of an arbitrary UV parameter.

\subsection{QCD Topological Susceptibility and Axion Parameters}
Lattice QCD and chiral effective theory give $\chi_{\mathrm{top}}(0)\sim (75\,\mathrm{MeV})^4$ at $T=0$.
Axion mass is determined by
\begin{equation}
m_a^2 f_a^2 = \chi_{\mathrm{top}}(0),
\end{equation}
giving the common approximation
\begin{equation}
m_a \simeq 5.7\,\mu\mathrm{eV}
\left(\frac{10^{12}\,\mathrm{GeV}}{f_a}\right).
\end{equation}
In the unified framework, $\chi_{\mathrm{top}}$ is a specific value of a second-order spectral moment of QCD modes on the unified time scale $\kappa(\omega)$, while $f_a$ relates to the "inertia" of the square root determinant line bundle $\mathcal L_{\mathrm{det}}^{1/2}$. Theorem 4 shows these quantities are not only constrained by internal QCD dynamics but must be consistent with $[K]=0$. Specifically:
1.  Too low $f_a$ would cause large-scale topological defects formed by axions in the early universe to introduce unacceptable holonomy in Null--Modular geometry, violating $[K]=0$;
2.  Too high $f_a$ might prevent the axion phase from being sufficiently uniform on cosmic scales, leading to regions with effective $\bar\theta\neq 0$ in some causal diamonds, destroying local entropy--gravity consistency.

\subsection{Gravitational $\theta_G$ Term and Unified Constraint}
Analogous to QCD $G\tilde G$, the gravitational sector may contain an $R\tilde R$ type topological term with angle $\theta_G$. In the unified framework, this term also contributes to $[K]$. Requiring $[K]=0$ gives a joint constraint on $\theta_G$ and $\bar\theta$, implying the strong $CP$ problem and "gravitational $CP$ problem" are not independent but different projections of the unified topological sector selection.

\section{Engineering Proposals}

\subsection{Next-Generation nEDM Experiments}
Future experiments (e.g., n2EDM@PSI) target sensitivities of $10^{-27}\text{--}10^{-28}\,e\cdot\mathrm{cm}$. In the unified framework, such experiments not only constrain $\bar\theta$ but also directly probe residuals of $[K_{\mathrm{QCD}}]$. Continued non-observation of nEDM would strongly support the exact validity of $[K_{\mathrm{QCD}}]=0$ and $[K]=0$.

\subsection{QCA Simulation and Topological Testbed}
Construct quantum simulations of SU(2) or SU(3) gauge QCA on finite-scale quantum computing platforms:
1.  Construct gauge-covariant QCA update gates on 2D or 3D discrete lattices, defining discrete topological charge $Q$.
2.  Introduce tunable $\theta$-phase $\exp(\mathrm i\theta Q)$ and measure the distribution of total phase $\Phi(\mathcal C)$ for closed loops.
3.  Test under what conditions $\mathbb Z_2$ residuals of $\Phi(\mathcal C)$ can be eliminated by introducing axion-like degrees of freedom or adjusting boundary conditions, realizing a discrete version of $[K]=0$.

\section{Discussion (risks, boundaries, past work)}
The unified topological--scattering framework integrates PQ mechanism, discrete $CP$/$P$ symmetry, and other solutions into a single picture. The PQ mechanism is interpreted as introducing an axion field to lift and eliminate the integer lift of $[K]$, with vacuum selection enforced by $[K]=0$.
Limitations: Rigorous construction of $\mathcal L_{\mathrm{det}}$ is clearer in Euclidean/static backgrounds; generalization to Lorentzian backgrounds with cosmological boundaries needs work. The equivalence between Null--Modular double cover, $[K]=0$, and Einstein equations/QNEC relies on previous series of works.

\section{Conclusion}
Within the unified framework, this paper provides a topological--scattering solution to the strong $CP$ problem:
1.  Existence of the square root of the scattering determinant line bundle is controlled by the relative cohomology class $[K]$.
2.  Null--Modular double cover and generalized entropy--gravity variational consistency require $[K]=0$.
3.  QCD physical angle $\bar\theta$ is rewritten as the holonomy coordinate of the QCD sector scattering determinant square root; $[K_{\mathrm{QCD}}]=0$ is equivalent to $\bar\theta_{\mathrm{eff}}\equiv 0\pmod{2\pi}$.
4.  PQ axion is the relative cohomology modulus of $[K]$, with vacuum $\langle a\rangle/f_a=\bar\theta_0$ fixed by $[K]=0$.
5.  SU(3) gauge QCA realizes a discrete version of $[K_{\mathrm{QCD}}]$, where topological--Null--Modular consistency requires total phase to be $2\pi\mathbb Z$, suppressing $\bar\theta$ in the continuum limit.

\appendix
\section{Standard Derivation of QCD \texorpdfstring{$\theta$}{theta}-Term and nEDM}
(Derivations of chiral rotation, $\bar\theta$ definition, and nEDM relation)

\section{Construction of Relative Cohomology Class \texorpdfstring{$[K]$}{[K]}}
(Details on Čech cohomology, bundle twisting, and relative conditions)

\section{Discrete Topological Charge in SU(3) Gauge QCA}
(Lattice construction of $Q$ and $\theta$-phase implementation)

\end{document}

