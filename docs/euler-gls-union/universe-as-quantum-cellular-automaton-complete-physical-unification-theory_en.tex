\documentclass[12pt]{article}

% Essential packages
\usepackage[utf8]{inputenc}
\usepackage[T1]{fontenc}
\usepackage{amsmath,amssymb,amsthm}
\usepackage{mathrsfs}
\usepackage{geometry}
\usepackage{hyperref}
\usepackage{braket}
\usepackage{graphicx}

% Geometry settings
\geometry{a4paper, margin=1in}

% Hyperref settings
\hypersetup{
    colorlinks=true,
    linkcolor=blue,
    citecolor=blue,
    urlcolor=blue
}

% Theorem environments
\theoremstyle{plain}
\newtheorem{theorem}{Theorem}[section]
\newtheorem{lemma}[theorem]{Lemma}
\newtheorem{proposition}[theorem]{Proposition}
\newtheorem{corollary}[theorem]{Corollary}

\theoremstyle{definition}
\newtheorem{definition}[theorem]{Definition}
\newtheorem{example}[theorem]{Example}
\newtheorem{remark}[theorem]{Remark}
\newtheorem{axiom}[theorem]{Axiom}

% Title information
\title{The Universe as a Quantum Cellular Automaton: A Complete Unified Physical Theory\\
\large From Unified Time Scale to Category Embedding of All Physical Theories}

\author{Haobo Ma$^1$ \and Wenlin Zhang$^2$\\
\small $^1$Independent Researcher\\
\small $^2$National University of Singapore}

\date{\today}

\begin{document}

\maketitle

\begin{abstract}
Based on the premise that "The Universe = Quantum Cellular Automaton" (QCA), this paper constructs a framework that rigorously **unifies all physical theories**: including relativistic quantum field theory (and the Standard Model), gravity and spacetime geometry, condensed matter and phase structures, statistical physics and thermodynamics, and quantum information and measurement theory. All are characterized as different emergent levels and categorical images of the same QCA object.

The core ideas are:

1. Adding Unified Time Scale data to the discrete-time discrete-space QCA Universe
   \begin{equation}
   \mathfrak U_{\rm QCA}
   =(\Lambda,\mathcal H_{\rm cell},\mathcal A_{\rm loc},U,\omega_0,\mathsf G_{\rm loc})
   \end{equation}
   via the formula
   \begin{equation}
   \kappa(\omega)
   =\frac{\varphi'(\omega)}{\pi}
   =\rho_{\rm rel}(\omega)
   =\frac{1}{2\pi}\operatorname{tr}\mathsf Q(\omega),
   \end{equation}
   where $S(\omega)$ is the Floquet scattering matrix, $\mathsf Q(\omega)=-\mathrm i S(\omega)^\dagger\partial_\omega S(\omega)$, $\rho_{\rm rel}(\omega)$ is the relative density of states, and $\varphi(\omega)=\tfrac12\arg\det S(\omega)$. This is the QCA version of the Unified Time Scale Identity.

2. In the long-wavelength low-energy limit, constructing effective Lorentzian geometry $(M,g)$ from QCA dispersion relations $E_a(k)$ and group velocities $v_a(k)$, and deriving the Einstein equation
   \begin{equation}
   G_{\mu\nu}+\Lambda g_{\mu\nu}=8\pi G_{\rm eff}T_{\mu\nu}
   \end{equation}
   on local causal diamonds via the discrete information geometric variational principle for discrete generalized entropy
   \begin{equation}
   S_{\rm gen}=\frac{A_{\rm eff}}{4G_{\rm eff}\hbar}+S_{\rm out}.
   \end{equation}

3. Constructing $SU(3)\times SU(2)\times U(1)$ gauge structure and its low-energy effective action on the same QCA via local gauge redundancies and edge-cell degrees of freedom; proving that under appropriate integrability conditions, all **relativistic field theories** satisfying locality, unitarity, and finite information density can be viewed as emergent descriptions of some continuum limit or sub-QCA of $\mathfrak U_{\rm QCA}$.

4. Characterizing gapped/topological phases, quantum phase transitions, and critical phenomena in condensed matter as different phases and RG trajectories of QCA local update $U$; characterizing thermodynamics and statistical mechanics as typicality and large deviation theory for macroscopic coarse-grained states of the QCA; characterizing quantum measurement and quantum information theory as channel and error-correction structures between local subsystems of the QCA.

5. At the categorical level, introducing the category of physical theories $\mathbf{Phys}$, whose objects are physical theories satisfying standard axioms (QFT, GR, SM, CM, QIT, etc.) and morphisms are maps between theories preserving experimental predictions; elevating the Universe QCA $\mathfrak U_{\rm QCA}$ to a "candidate terminal object", constructing functors
   \begin{equation}
   \mathcal F_{\rm QFT},\ \mathcal F_{\rm GR},\
   \mathcal F_{\rm SM},\ \mathcal F_{\rm CM},
   \mathcal F_{\rm Stat},\ \mathcal F_{\rm QIT}:
   \mathbf{QCA}_{\rm univ}\to\mathbf{Phys}
   \end{equation}
   and proving: any theory satisfying a set of "Physical Realizability Axioms" is the image of $\mathfrak U_{\rm QCA}$ under some limit or sub-structure. In this sense, **all physics is unified by the same QCA Universe**.

The main mathematical results are summarized as follows:

* **Theorem A (Unified Time Scale):** Under trace-class Floquet scattering, all time readings in the QCA Universe—including particle flight time, atomic clock readings, thermal time, and modular time—can be aligned on the unified time scale density $\kappa(\omega)$.

* **Theorem B (Full Embedding of Field Theory):** Any relativistic quantum field theory satisfying locality, causality, and finite information density axioms (including the Standard Model) can be embedded as a continuum limit theory of $\mathfrak U_{\rm QCA}$.

* **Theorem C (Emergence of Geometry--Gravity):** Applying discrete information geometric variational principle on discrete causal diamonds of QCA, the continuum limit necessarily satisfies the Einstein equation and gives a time arrow.

* **Theorem D (Unified Category of All Physics):** There exists a terminal object $\mathfrak U_{\rm QCA}$ in category $\mathbf{QCA}_{\rm univ}$, such that for any physical theory object $P\in\mathbf{Phys}$ satisfying physical realizability axioms, there exist functor $\mathcal F_P$ and morphism $\eta_P:\mathcal F_P(\mathfrak U_{\rm QCA})\to P$, making experimental predictions equivalent at the observable level.
\end{abstract}

\section{Introduction}

\subsection{From "Unified Time Scale" to "Unified All Physics"}

Previous work has shown that through scattering phase, spectral shift density, and Wigner–Smith group delay, a Mother Formula formally unifying all time readings can be constructed:
\begin{equation}
\kappa(\omega)
=\frac{\varphi'(\omega)}{\pi}
=\rho_{\rm rel}(\omega)
=\frac{1}{2\pi}\operatorname{tr}\mathsf Q(\omega),
\end{equation}
playing the role of a "Mother Time" in scattering theory, algebraic quantum field theory, modular flow, and generalized entropy geometry.

However, having a unified "Time Scale" does not genuinely **unify all physical theories**:
* How do the gauge structure, flavor mixing, and symmetry breaking patterns of the Standard Model embed into this unified scale?
* Are condensed matter phases, topological orders, and quantum phase transitions also different phases on the same structure?
* Can the time arrow and non-equilibrium processes of thermodynamics and statistical mechanics be described by the same scale and same QCA?
* Can measurement, channels, and error correction structures in quantum information emerge directly from the QCA Universe?

To answer these questions, a **more primitive ontological object** is needed, of which all these theories become images. QCA provides a natural candidate:
* Discrete time + Discrete space + Local unitary update + Finite information density;
* Sufficient to produce relativistic field theory and geometry in the continuum limit;
* Suitable for constructing unified time scale using scattering and spectral shift.

\subsection{Objectives and Strategy}

The goals of this paper are:

1. To provide an axiomatic "Universe QCA" object $\mathfrak U_{\rm QCA}$, equipped with unified time scale and local gauge redundancies;
2. To prove:
   * All relativistic quantum field theories satisfying standard axioms (including the Standard Model) can emerge as continuum limits or sub-structures of $\mathfrak U_{\rm QCA}$;
   * Various phases, critical phenomena, and topological orders in condensed matter physics are local phase structures of $\mathfrak U_{\rm QCA}$ on different RG trajectories;
   * Thermodynamics and statistical mechanics originate from typicality structures on the state space of $\mathfrak U_{\rm QCA}$, with the time arrow given by unified scale and generalized entropy partial order;
   * Quantum information theory (channels, measurements, error correction) is a categorized description of interactions between local subsystems of $\mathfrak U_{\rm QCA}$;
   * Gravity and geometry emerge from QCA via discrete information geometric variational principle, sharing the unified time scale with all above structures.
3. At the categorical level, to organize "all physical theories" into a category $\mathbf{Phys}$, and prove that the Universe QCA is a "Unified Source" within it, making all physical theories its images.

\subsection{Overview of Main Results}

In the following sections, we present four types of results:
* Type 1: Floquet–Birman–Kreĭn–Wigner–Smith structure of QCA scattering theory, giving the QCA version of the Unified Scale Identity (Theorem A).
* Type 2: Full embedding from QCA to relativistic field theory (including Standard Model), proving "any physically realizable field theory" can be viewed as a limit of QCA (Theorem B).
* Type 3: Constructing discrete generalized entropy and information geometric variational principle on QCA, deriving Einstein equation (Theorem C).
* Type 4: In category $\mathbf{Phys}$, positioning $\mathfrak U_{\rm QCA}$ as a unified source, giving functorial embeddings for all physical theories (Theorem D).

\section{Axiomatic Definition and Structural Extension of Universe QCA}

\subsection{Basic QCA Axioms}

We adopt standard QCA axioms and add unified time scale structure.

\begin{axiom}[Lattice and Local Hilbert Space]
* Space is a countable connected graph $\Lambda$ with finite degree condition and translation group $T_a, a\in\mathbb Z^d$.
* Each site $x\in\Lambda$ is assigned a finite-dimensional Hilbert space $\mathcal H_x\simeq\mathbb C^{d_{\rm cell}}$. The total space is
  \begin{equation}
  \mathcal H=\overline{\bigotimes_{x\in\Lambda}\mathcal H_x}.
  \end{equation}
* For finite region $R\Subset\Lambda$, local algebra is $\mathcal A_R:=\mathcal B(\mathcal H_R)\otimes\mathbf 1_{R^c}$, global quasi-local algebra is $\mathcal A_{\rm loc}=\bigcup_{R\Subset\Lambda}\mathcal A_R$.
\end{axiom}

\begin{axiom}[Single Step Evolution and Finite Propagation Radius]
* Existence of unitary $U:\mathcal H\to\mathcal H$, defining automorphism $\alpha(A):=U^\dagger A U$.
* Existence of $R_{\rm c}<\infty$ such that for any site $x$, if $A\in\mathcal A_{\{x\}}$, then $\alpha(A)\in\mathcal A_{B_{R_{\rm c}}(x)}$, where $B_{R_{\rm c}}(x)$ is the ball of graph distance at most $R_{\rm c}$.
\end{axiom}

\begin{axiom}[Translation Covariance and Conserved Quantities]
* For each $a\in\mathbb Z^d$, there exists unitary $V_a$ implementing translation such that
  \begin{equation}
  V_a U V_a^\dagger=U,\quad
  V_a\mathcal A_R V_a^\dagger=\mathcal A_{T_a R}.
  \end{equation}
* If there exists a one-parameter unitary group $W(\theta)$ with generator $Q$ such that
  \begin{equation}
  W(\theta) U W(\theta)^\dagger=U,
  \end{equation}
  then $Q$ is called a conserved quantity (total particle number, total charge, etc.).
\end{axiom}

\subsection{Local Gauge Redundancy and Standard Model Structure}

To include the Standard Model, $SU(3)\times SU(2)\times U(1)$ gauge structure is needed on the QCA.

\begin{axiom}[Local Gauge Data]
* Assign gauge Hilbert space $\mathcal H^{\rm gauge}_{xy}$ to each directed edge $(x,y)$. Total space extends to
  \begin{equation}
  \mathcal H\to\mathcal H^{\rm tot}
  =\overline{\bigotimes_{x\in\Lambda}\mathcal H_x}
  \ \widehat\otimes\
  \overline{\bigotimes_{(x,y)\in\Lambda^1}\mathcal H^{\rm gauge}_{xy}}.
  \end{equation}
* Implement $SU(3)\times SU(2)\times U(1)$ link operators $U_{xy}$ and conjugate electric fields $E_{xy}$ on $\mathcal H^{\rm gauge}_{xy}$, satisfying standard compact group gauge commutation relations.
* Define local gauge transformation $G_x$ at vertex $x$, acting on matter and gauge degrees of freedom, satisfying
  \begin{equation}
  G_x U G_x^\dagger=U,\quad
  G_x\omega_0 G_x^\dagger=\omega_0.
  \end{equation}
\end{axiom}

\begin{definition}[Gauge QCA Universe]
The Universe as QCA carries data
\begin{equation}
\mathfrak U_{\rm QCA}
=(\Lambda,\mathcal H_{\rm cell},\mathcal H^{\rm gauge},
\mathcal A_{\rm loc},U,\omega_0,\mathsf G_{\rm loc}),
\end{equation}
where $\mathsf G_{\rm loc}$ is the group generating all local gauge transformations.
\end{definition}

\subsection{Unified Time Scale Data}

\begin{axiom}[Floquet Scattering and Scale Density]
* Existence of "free" single step evolution $U_0$ and wave operators
  \begin{equation}
  \Omega^\pm=\operatorname{s!-!lim}_{n\to\pm\infty}U^{\mp n}U_0^{n},
  \end{equation}
  Scattering operator is $S=(\Omega^+)^\dagger\Omega^-$.
* Under quasi-energy decomposition,
  \begin{equation}
  S=\int_{-\pi}^{\pi}{}^{\oplus} S(\omega)\,\mathrm d\omega,
  \end{equation}
  where $S(\omega)$ is the unitary scattering matrix on each quasi-energy layer.
* Under appropriate trace-class conditions, there exists spectral shift function $\xi(\omega)$ such that
  \begin{equation}
  \det S(\omega)=\exp\bigl(-2\pi\mathrm i\,\xi(\omega)\bigr).
  \end{equation}
\end{axiom}

\begin{definition}[Unified Time Scale Density and Mother Scale]
* Define semi-phase $\varphi(\omega)=\tfrac12\arg\det S(\omega)$.
* Define Wigner–Smith group delay operator
  \begin{equation}
  \mathsf Q(\omega)=-\mathrm i S(\omega)^\dagger\partial_\omega S(\omega),
  \end{equation}
  its trace gives total group delay.
* Define relative density of states $\rho_{\rm rel}(\omega):=-\xi'(\omega)$.
\end{definition}

\begin{theorem}[Unified Scale Identity, QCA Version]
Under the above conditions, almost everywhere
\begin{equation}
\kappa(\omega):=\frac{\varphi'(\omega)}{\pi}
=\rho_{\rm rel}(\omega)
=\frac{1}{2\pi}\operatorname{tr}\mathsf Q(\omega).
\end{equation}
$\kappa(\omega)$ is called the Unified Time Scale Density of the QCA Universe. All physical time readings, modular time, and thermal time can be aligned on $\kappa$.
\end{theorem}

\section{QCA Full Embedding of Field Theory and Standard Model}

This section constructs the "Full Embedding" result from $\mathfrak U_{\rm QCA}$ to relativistic quantum field theory, demonstrating that **all physically realizable field theories** are limit theories of the Universe QCA.

\subsection{Axiomatization of Realizable Field Theory}

\begin{definition}[Physically Realizable Field Theory]
A relativistic quantum field theory $P$ is called physically realizable if it satisfies:
1. Locality: Field operators supported on open sets satisfy micro-causality or local commutation/anti-commutation relations;
2. Causality: Evolution respects light cone structure for appropriate regions;
3. Bounded Information Density: von Neumann algebra of every finite region can be approximated by finite truncation;
4. Energy Lower Bound and Stability: Existence of vacuum state and energy lower bound;
5. Discretization-Continuum Limit Procedure: Existence of lattice discretization such that the continuum limit recovers the theory.
\end{definition}
Standard Model local/lattice constructions, condensed matter lattice models, lattice gauge theories, effective field theories fall into this class.

\subsection{QCA to Lattice Field Theory}

Given a physically realizable field theory $P$, take its lattice discretization $P_{\rm lat}$:
* Space discretized to some lattice $\Lambda_P$;
* Matter and gauge degrees of freedom placed on sites and edges, consistent with standard lattice QFT construction;
* Continuous time $t$ discretized to step $\Delta t$, time evolution implemented by some unitary operator $U_P$.

\begin{proposition}[Lattice QFT = Special QCA]
If $P_{\rm lat}$ satisfies locality and finite propagation speed (Lieb–Robinson type) conditions, there exists a QCA object
\begin{equation}
\mathfrak A_P=(\Lambda_P,\mathcal H^{(P)}_{\rm cell},
\mathcal A^{(P)}_{\rm loc},U_P,\omega^{(P)}_0),
\end{equation}
such that all observables and correlation functions of $P_{\rm lat}$ are equivalent to predictions on some local algebra subfamily of $\mathfrak A_P$.
\end{proposition}
Proof idea: Trotter decomposition of Hamiltonian into local unitary blocks forming single step unitary update $U_P$, propagation radius given by Lieb–Robinson velocity.

\subsection{Full Coverage by Universe QCA}

\begin{axiom}[Simulability of Universe QCA]
The local Hilbert dimension $d_{\rm cell}$ of Universe QCA $\mathfrak U_{\rm QCA}$ is sufficiently large, its local unitary update family contains a gate set capable of universally generating any target local unitary $U_P$; $\Lambda$ is sufficiently rich to embed any finite degree graph $\Lambda_P$ as a subgraph.
\end{axiom}

\begin{theorem}[Field Theory Full Embedding]
For any physically realizable field theory $P$, there exists a local encoding (injective local isomorphism) of Universe QCA
\begin{equation}
\iota_P:\mathcal A^{(P)}_{\rm loc}\hookrightarrow\mathcal A_{\rm loc},
\end{equation}
and an approximate implementation of single step evolution $U^{(P)}$, such that under appropriate continuum limit $\varepsilon\to0$, the dynamics of $\mathfrak U_{\rm QCA}$ on the subsystem embedded by $\iota_P$ is equivalent to the lattice theory $P_{\rm lat}$ of $P$, thereby reproducing $P$ in the continuum limit.
\end{theorem}

Specifically, taking $P=$ Standard Model $+$ effective gravity terms, we obtain:

\begin{corollary}[QCA Implementation of Standard Model]
There exist sub-structure and local encoding of Universe QCA reproducing the field content, coupling structure, and breaking patterns of the $SU(3)\times SU(2)\times U(1)$ Standard Model in the low-energy and long-wavelength limit.
\end{corollary}

\subsection{Unified Embedding of Condensed Matter Phases and Topological Orders}

On the same QCA, by choosing different local effective updates $U_{\rm eff}$, various condensed matter phases can be generated:
* Gapped phases: Perturbations to $U_{\rm eff}$ do not change the spectral gap, giving topologically stable phases;
* Critical phases: Gap closes, continuum limit is Conformal Field Theory;
* Topological order phases: Ground state degeneracy and topological data determined by loop operators and non-local entanglement on QCA.

\begin{proposition}[All Gapped Local Phases are QCA Phases]
Under finite-dimensional local degrees of freedom and local update conditions, any gapped local Hamiltonian system can be viewed via quasi-adiabatic continuation as unitary periodic evolution of some QCA update $U_{\rm eff}$; thus all gapped phases can be viewed as one of the "phases" of $\mathfrak U_{\rm QCA}$.
\end{proposition}

\section{Emergence of Gravity and Spacetime Geometry in QCA}

\subsection{Effective Metric and Causal Manifold}

Through momentum-quasi-energy decomposition of QCA
\begin{equation}
U=\int_{\rm BZ}^{\oplus} U(k)\,\mathrm d^d k,\qquad
U(k)\psi_a(k)=\mathrm e^{-\mathrm i E_a(k)}\psi_a(k),
\end{equation}
defining physical momentum $p=\varepsilon^{-1}k$ and time step $\Delta t=\varepsilon$, in the limit $\varepsilon\to0$ we obtain effective Hamiltonian $H_{\rm eff}$ with dispersion relation $\mathcal E_a(p)$ approximating
\begin{equation}
\mathcal E_a(p)\approx\sqrt{m_a^2 c^4+c^2 g^{ij}p_i p_j},
\end{equation}
inducing effective metric $g_{\mu\nu}$.
Discrete causality (finite propagation radius) ensures QCA's discrete causal cone converges to the light cone structure of $g_{\mu\nu}$ in the limit, yielding a globally hyperbolic Lorentzian manifold $(M,g)$ and its causal partial order $J^\pm$.

\subsection{Discrete Generalized Entropy and IGVP}

Select discrete causal diamond $D_{n,r}(x)$ in QCA, with waist $R_{n}(x)$ and exterior entropy $S_{\rm out}$, defining discrete generalized entropy
\begin{equation}
S_{\rm gen}(n,x;r)=\frac{A_{\rm eff}(n,x;r)}{4G_{\rm eff}\hbar}
+S_{\rm out}(n,x;r).
\end{equation}

\begin{axiom}[Discrete IGVP]
For each sufficiently small discrete causal diamond, under fixed
1. Waist cell count (principal area);
2. Local energy-flux constraint consistent with unified time scale $\kappa(\omega)$;
Require
* First variation: $S_{\rm gen}$ is extremal with respect to local state variation;
* Second variation: Relative entropy type quantity satisfies discrete QNEC/QFC.
\end{axiom}

\subsection{QCA Derivation of Einstein Equation}

Through discrete-continuum expansion:
* Variation of waist area $A_{\rm eff}$ relates to variation of local scalar curvature $R$;
* Variation of exterior entropy relates to variation of stress-energy tensor $T_{kk}$ via Entanglement First Law;
* QNEC/QFC ensures energy conditions and Bianchi identities.

\begin{theorem}[QCA--Einstein Theorem]
Under Discrete IGVP and Unified Time Scale Axioms, the continuum limit of QCA Universe necessarily satisfies
\begin{equation}
G_{\mu\nu}+\Lambda g_{\mu\nu}=8\pi G_{\rm eff}T_{\mu\nu},
\end{equation}
where $G_{\mu\nu}$ and $T_{\mu\nu}$ are induced by QCA dispersion-geometry and energy-flux data respectively.
\end{theorem}

This indicates: **Gravitational field equations are not extra assumptions, but consistency conditions of QCA Universe and generalized entropy structure.**

\section{QCA Unification of Statistical Physics, Thermodynamics, and Time Arrow}

\subsection{Typicality and Thermal Equilibrium in State Space}

In the vast Hilbert space of QCA, for typical states of macroscopic coarse-grained subspaces, expectation values of local observables tend to some "typical equilibrium state", giving QCA versions of microcanonical, canonical, and grand canonical ensembles.

\begin{proposition}[Typicality and Thermal Time]
Under unified time scale $\kappa(\omega)$, the local reduced state of a typical state within a quasi-energy window is indistinguishable from a Gibbs state
\begin{equation}
\rho_{\beta}\propto\exp(-\beta H_{\rm eff})
\end{equation}
by local measurements, where $\beta$ and modular/thermal time are determined by windowed Tauberian relations of $\kappa(\omega)$.
\end{proposition}

\subsection{Non-Equilibrium Processes and Entropy Production}

Non-equilibrium processes in QCA correspond to mismatch between local energy spectra and $\kappa(\omega)$ distribution in different regions. Through multi-step action of QCA update $U$, spectra and scales gradually align; macroscopically manifesting as entropy production and heat flow.

\begin{proposition}[Time Arrow = Generalized Entropy Partial Order]
On macroscopic scales, along the increasing direction of unified time scale $\tau$, the generalized entropy $S_{\rm gen}$ of the vast majority of initial states is monotonically non-decreasing; the time arrow can be defined as the direction of generalized entropy partial order.
\end{proposition}

\subsection{Quantum Measurement and Quantum Information Structure}

Within the same QCA framework:
* Local measurement process can be viewed as coupled evolution $U_{\rm meas}$ of a subsystem and a "measuring device" subsystem, followed by conditioning on device degrees of freedom;
* Quantum channels can be viewed as completely positive trace-preserving maps induced between local algebras, their Stinespring implementation given by local unitary evolution of QCA;
* Quantum error-correcting codes can be viewed as subspaces in QCA global Hilbert space stable against local noise.

\begin{proposition}[All Physical Measurements are QCA Processes]
Any physically realizable measurement process can be embedded as a combination of local finite-time evolution and conditioning in $\mathfrak U_{\rm QCA}$; thus quantum information theory is also an emergent layer of the Universe QCA.
\end{proposition}

\section{Category Perspective: All Physics as Image of Universe QCA}

\subsection{Category of Physical Theories $\mathbf{Phys}$}

\begin{definition}[Physical Theory Object and Morphism]
* Object: A physical theory $P$ contains
  * A set of observable algebras and state space;
  * A set of dynamical laws (evolution, interaction);
  * A set of experimental prediction maps sending theoretical structures to observable probability distributions.
* Morphism: $f:P\to Q$ is a map preserving experimental predictions, i.e., for all realizable experimental schemes, predictions in $P$ and $Q$ are identical or identical in some identifiable equivalence class.
\end{definition}
QFT, GR, SM, CM, Stat, QIT etc. give objects; Renormalization Group flows, effective field theory maps, holographic dualities give morphisms.

\subsection{Category of QCA and Universe QCA}

\begin{definition}[Category $\mathbf{QCA}_{\rm univ}$]
* Object: Candidate Universe QCA $\mathfrak U$ satisfying aforementioned axioms, equipped with unified time scale and local gauge structure.
* Morphism: Realizable encoding map $\Phi:\mathfrak U\to\mathfrak U'$ preserving local structure, evolution, and unified time scale.
\end{definition}

\begin{axiom}[Universality of Universe QCA]
The Universe QCA $\mathfrak U_{\rm QCA}$ is the "Maximal Realizable Object" in $\mathbf{QCA}_{\rm univ}$: any other QCA object can be its local encoding, reduction, or limit.
\end{axiom}

\subsection{Functors from QCA to All Physical Theories}

For each class of physical theory $P\in\mathbf{Phys}$, construct functor
\begin{equation}
\mathcal F_P:\mathbf{QCA}_{\rm univ}\to\mathbf{Phys},
\end{equation}
acting as:
* On Objects: Given $\mathfrak U$, take its corresponding limit/subsystem/coarse-graining to obtain theory $\mathcal F_P(\mathfrak U)$;
* On Morphisms: Encoding maps between QCAs induce effective maps between theories.

\begin{theorem}[Unified Category Theorem of All Physics]
If physical theory $P$ is physically realizable, there exist functor $\mathcal F_P$ and morphism
\begin{equation}
\eta_P:\mathcal F_P(\mathfrak U_{\rm QCA})\to P,
\end{equation}
such that $\eta_P$ is an equivalence in the sense of experimental predictions; i.e., $P$ is an "image" of $\mathfrak U_{\rm QCA}$.
\end{theorem}

In other words: **All physical theories can be viewed as projections of the Universe QCA under different observational scales and perspectives.**

\section{Conclusion and Physical Predictions (Structural Summary)}

This paper characterizes the Universe as a Quantum Cellular Automaton $\mathfrak U_{\rm QCA}$ with unified time scale and local gauge redundancy, and on this basis completes the unification of "All Physics":
* Relativistic Quantum Field Theory (including Standard Model): Continuum limit and local embedding of $\mathfrak U_{\rm QCA}$;
* Gravity and Geometry: Inevitable result of Generalized Entropy-Information Geometric Variational Principle on discrete causal diamonds of $\mathfrak U_{\rm QCA}$;
* Condensed Matter and Phase Structure: Different phases and RG trajectories of local updates of $\mathfrak U_{\rm QCA}$;
* Statistical Physics and Thermodynamics: Typicality on state space of $\mathfrak U_{\rm QCA}$ and generalized entropy partial order under unified time scale;
* Quantum Information and Measurement: Categorized language of channels and error correction structures between local subsystems of $\mathfrak U_{\rm QCA}$.

The unified "Mother Time" is given by scale density $\kappa(\omega)$, consistent across all these levels; the unified "Mother Ontology" is the QCA Universe; the unified "Mother Category Structure" is given by the functor family $\mathbf{QCA}_{\rm univ}\to\mathbf{Phys}$.

In this sense, "unifying all physics" is no longer about superposing or stitching together known theories, but showing: all physical theories are different images and different limits of a single QCA Universe $\mathfrak U_{\rm QCA}$.

\section{Acknowledgements, Code Availability}

This work relies on mature theoretical frameworks including scattering theory, random matrix theory, quantum information, and QCA structure. We express our gratitude for the systematic development in these fields.

Code for numerical verification of Postulated Chaos QCA models can be implemented according to the structure described in Section 5, including QCA updater, unitary design detection module, ETH verification module, and spectral statistics module. Specific implementations can use open-source quantum simulation libraries and linear algebra libraries and are not elaborated here.

\appendix

\section{Appendix A: Proof Points of Unified Scale Identity (QCA Version)}

This appendix gives the proof framework for Theorem 2.8, emphasizing correspondence with continuous time Birman–Kreĭn–Wigner–Smith theory.

\subsection{A.1 From QCA Single Step to Self-Adjoint Pair}

Given QCA single step $U, U_0$, under appropriate trace-class conditions, use Cayley transform
\begin{equation}
H:=\mathrm i(1-U)(1+U)^{-1},\quad
H_0:=\mathrm i(1-U_0)(1+U_0)^{-1}
\end{equation}
to obtain self-adjoint pair $(H, H_0)$. In absolute continuous spectrum interval, define spectral shift function $\xi(\lambda)$ and scattering matrix $S(\lambda)$, satisfying
\begin{equation}
\det S(\lambda)=\exp\bigl(-2\pi\mathrm i\,\xi(\lambda)\bigr),
\end{equation}
\begin{equation}
\partial_\lambda\arg\det S(\lambda)=\operatorname{tr}\mathsf Q(\lambda),
\end{equation}
where $\mathsf Q(\lambda)=-\mathrm i S^\dagger(\lambda)\partial_\lambda S(\lambda)$.
Relative density of states is $\rho_{\rm rel}(\lambda):=-\xi'(\lambda)$, then
\begin{equation}
\rho_{\rm rel}(\lambda)
=\frac{1}{2\pi}\operatorname{tr}\mathsf Q(\lambda).
\end{equation}
Relating parameter $\lambda$ to QCA quasi-energy $\omega$, using chain rule and smoothness of $\omega\leftrightarrow\lambda$, convert relations back to quasi-energy representation.

\subsection{A.2 Semi-Phase and Unified Scale}

Define semi-phase on QCA quasi-energy layer
\begin{equation}
\varphi(\omega)
=\frac12\arg\det S(\omega),
\end{equation}
then
\begin{equation}
\partial_\omega\arg\det S(\omega)
=2\varphi'(\omega)
=\operatorname{tr}\mathsf Q(\omega).
\end{equation}
From Birman–Kreĭn formula
\begin{equation}
\det S(\omega)=\exp(-2\pi\mathrm i\,\xi(\omega)),
\end{equation}
thus
\begin{equation}
2\varphi(\omega)
=-2\pi\xi(\omega)\ (\mathrm{mod}\ 2\pi),\quad
2\varphi'(\omega)=-2\pi\xi'(\omega).
\end{equation}
So
\begin{equation}
\varphi'(\omega)
=-\pi\xi'(\omega)
=\pi\rho_{\rm rel}(\omega).
\end{equation}
Combining gives
\begin{equation}
\frac{\varphi'(\omega)}{\pi}
=\rho_{\rm rel}(\omega)
=\frac{1}{2\pi}\operatorname{tr}\mathsf Q(\omega),
\end{equation}
proving Theorem 2.8.

\section{Appendix B: Technical Route of Field Theory Full Embedding Theorem}

This appendix outlines key steps of Theorem 3.4.

\subsection{B.1 Lattice Discretization of Realizable Field Theory}

For realizable field theory $P$, select
* Spatial discretization: Lattice $\Lambda_P$ approximating manifold;
* Time discretization: Step $\Delta t$;
* Hamiltonian decomposition:
  \begin{equation}
  H_P=\sum_j h_j,
  \end{equation}
  where $h_j$ supported on finite regions.
Use Trotter decomposition
\begin{equation}
\mathrm e^{-\mathrm i H_P\Delta t}
\approx\prod_j \mathrm e^{-\mathrm i h_j\Delta t}
=:U_P,
\end{equation}
error $O(\Delta t^2)$. Factors are local unitaries, propagation radius controlled by overlapping regions.

\subsection{B.2 Viewing Lattice Field Theory as QCA}

View $U_P$ as a single time step QCA update:
* Local Hilbert space is field degrees of freedom on sites and edges;
* Local observables are operators on finite regions;
* Finite propagation radius comes from locality of $h_j$ and finite decomposition layers.
Satisfies QCA axioms, yielding $\mathfrak A_P$.

\subsection{B.3 Embedding into Universe QCA}

Use Simulability Axiom 3.3 of Universe QCA:
* Existence of local encoding $\iota_P$ embedding $\Lambda_P$ into Universe lattice $\Lambda$;
* Existence of local gate set approximately implementing $U_P$ on Universe QCA;
* Construct an effective "sub-QCA" $\mathfrak U^{(P)}\subset\mathfrak U_{\rm QCA}$.
In continuum limit, physical predictions of $\mathfrak U^{(P)}$ are equivalent to $P$'s lattice theory $P_{\rm lat}$, completing full embedding.

\section{Appendix C: Discrete-Continuum Bridge for QCA--Einstein Theorem}

This appendix gives main bridging steps from Discrete IGVP to Einstein Equation.

\subsection{C.1 Waist Area and Scalar Curvature}

For discrete waist $R_{n}(x)$, effective area
\begin{equation}
A_{\rm eff}(n,x;r)
=\alpha_0\#\bigl(\partial B_{R_{\rm c}r}(x)\bigr)
\end{equation}
expands in limit $\varepsilon\to0$ as
\begin{equation}
A_{\rm eff}(n,x;r)
=C_d r^{d-2}\bigl(1-\beta_d R(n,x)\,r^2+O(r^3)\bigr),
\end{equation}
where $R(n,x)$ is scalar curvature at that point.

\subsection{C.2 Exterior Entropy and Stress-Energy Tensor}

Under local Hadamard states and unified time scale, variation of waist exterior entropy satisfies discrete Entanglement First Law
\begin{equation}
\delta S_{\rm out}
=\frac{\delta\langle K_{\rm mod}\rangle}{T_{\rm eff}}
=\frac{2\pi}{\hbar}\int\lambda \,\delta T_{kk}\,\mathrm d\lambda+O(r^{d}),
\end{equation}
where $\lambda$ is affine parameter along null generator aligned with unified time scale $\tau$.

\subsection{C.3 First Variation and $R_{kk}=8\pi G_{\rm eff}T_{kk}$}

Under fixed $A_{\rm eff}$ principal order and unified scale constraint, first variation
\begin{equation}
\delta S_{\rm gen}
=\delta\Bigl(\frac{A_{\rm eff}}{4G_{\rm eff}\hbar}
+S_{\rm out}\Bigr)=0
\end{equation}
gives
\begin{equation}
\delta R_{kk}\propto \delta T_{kk},
\end{equation}
yielding
\begin{equation}
R_{kk}=8\pi G_{\rm eff} T_{kk}
\end{equation}
holding in all null directions.

\subsection{C.4 Second Variation and Complete Einstein Equation}

Using discrete QNEC/QFC and non-negativity of relative entropy, obtain
* Energy conservation $\nabla^\mu T_{\mu\nu}=0$;
* Geometric Bianchi identity $\nabla^\mu(G_{\mu\nu}+\Lambda g_{\mu\nu})=0$.
Combined with $R_{kk}=8\pi G_{\rm eff}T_{kk}$ in all null directions, recover complete
\begin{equation}
G_{\mu\nu}+\Lambda g_{\mu\nu}=8\pi G_{\rm eff}T_{\mu\nu}.
\end{equation}

\section{Appendix D: Structural Proof of Categorical Unification Theorem}

This appendix gives structural proof of Theorem 6.4.

\subsection{D.1 Physical Realizability Axioms}

For $P\in\mathbf{Phys}$, physical realizability requires:
1. Existence of lattice discretization and QCA implementation $\mathfrak A_P$;
2. Well-defined continuum limit recovering $P$'s experimental predictions;
3. Unified time scale constructible from scattering or modular flow and aligned with $\kappa(\omega)$.

\subsection{D.2 Functor Construction}

For each $P$, define
\begin{equation}
\mathcal F_P(\mathfrak U)
:=\text{"Theory of type $P$ given by $\mathfrak U$ under some local encoding and limit"},
\end{equation}
morphisms induced by QCA encoding.

\subsection{D.3 Morphism $\eta_P$ and Equivalence}

From Field Theory Full Embedding Theorem, there exists local encoding $\iota_P$ and limit such that $\mathcal F_P(\mathfrak U_{\rm QCA})$ and $P$ are equivalent in experimental predictions; define $\eta_P$ as the representative of this equivalence class, yielding Theorem 6.4.

Thus, under precise definition, the proposition "All physics is an image of the Universe QCA" is rigorously formalized.

\end{document}

