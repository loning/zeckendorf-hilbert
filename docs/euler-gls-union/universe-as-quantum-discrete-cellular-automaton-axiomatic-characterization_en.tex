\documentclass[12pt]{article}

% Essential packages
\usepackage[utf8]{inputenc}
\usepackage[T1]{fontenc}
\usepackage{amsmath,amssymb,amsthm}
\usepackage{mathrsfs}
\usepackage{geometry}
\usepackage{hyperref}
\usepackage{braket}
\usepackage{graphicx}
\usepackage{amsfonts} % For \mathfrak

% Geometry settings
\geometry{a4paper, margin=1in}

% Hyperref settings
\hypersetup{
    colorlinks=true,
    linkcolor=blue,
    citecolor=blue,
    urlcolor=blue
}

% Theorem environments
\theoremstyle{plain}
\newtheorem{theorem}{Theorem}[section]
\newtheorem{lemma}[theorem]{Lemma}
\newtheorem{proposition}[theorem]{Proposition}
\newtheorem{corollary}[theorem]{Corollary}

\theoremstyle{definition}
\newtheorem{definition}[theorem]{Definition}
\newtheorem{example}[theorem]{Example}
\newtheorem{remark}[theorem]{Remark}
\newtheorem{hypothesis}[theorem]{Hypothesis}

% Title information
\title{Universe as Quantum Discrete Cellular Automaton: Axiomatic Characterization}
\author{Haobo Ma$^1$ \and Wenlin Zhang$^2$\\
\small $^1$Independent Researcher\\
\small $^2$National University of Singapore}

\date{\today}

\begin{document}

\maketitle

\begin{abstract}
Within the frameworks of Quantum Cellular Automata (QCA), quasi-local $C^\ast$-algebras, and causal sets, we construct an axiomatic system for "Universe = Single Quantum Discrete Cellular Automaton". Specifically, we use a countable connected graph $\Lambda$ as the discrete space; a finite-dimensional local Hilbert space $\mathcal H_{\rm cell}$ and the quasi-local algebra $\mathcal A$ on its infinite tensor product to describe local quantum degrees of freedom; a $\ast$-automorphism $\alpha:\mathcal A\to\mathcal A$ with finite propagation radius and its unitary implementation $U$ to describe discrete time evolution; and an initial cosmic state $\omega_0$ to describe the universe at time $n=0$. We prove that under these axioms, the naturally induced relation on the event set $E=\Lambda\times\mathbb Z$ constitutes a locally finite partial order, thereby yielding a discrete causal set whose local finiteness strictly corresponds to the finite propagation radius condition of the QCA, aligning with the "locally finite partial order" structure in causal set theory. Furthermore, we define the "Universe QCA Object" $\mathfrak U_{\rm QCA}=(\Lambda,\mathcal H_{\rm cell},\mathcal A,\alpha,\omega_0)$, provide an equivalence theorem of "QCA Locality $\Longleftrightarrow$ Local Finiteness of Causal Partial Order", and construct a 1D Dirac-type QCA in the single-particle limit, demonstrating how to recover the continuous Dirac equation in appropriate scaling limits. Finally, we discuss the expression of observation, entropy, and the arrow of time within this framework, as well as relationships with causal set quantum gravity and discretization schemes for quantum field theory.
\end{abstract}

\noindent\textbf{Keywords:} Quantum Cellular Automata; Quasi-local $C^\ast$-Algebra; Discrete Universe Model; Causal Set; Discrete Time Dynamics; Lattice Implementation of Dirac Equation

\section{Introduction \& Historical Context}

Quantum Cellular Automata (QCA) can be viewed as "discrete-time, local unitary dynamics on infinite quantum lattices". Their representative axiomatization appeared in the work of Schumacher--Werner on reversible QCA: defining QCA as a translation-invariant $\ast$-automorphism on a quasi-local algebra with a strictly finite propagation radius, ensuring that each evolution step propagates support only within a finite neighborhood. Subsequently, Arrighi, Farrelly, and others systematically reviewed structural theorems, computability, quantum simulation, and continuum limits of QCA, demonstrating the broad applicability of QCA as a tool for discretized quantum field theory and topological phase simulation. ([arXiv][1])

On the other hand, the causal set program represented by Sorkin and Surya proposes that spacetime ontology can be replaced by a locally finite partial order set, where the partial order encodes causal structure and local finiteness encodes discreteness, satisfying the "order + number $\sim$ geometry" program. In this scheme, causal partial order and local finiteness jointly characterize a discrete "proto-spacetime", with continuous Lorentzian manifolds being merely limit interpolations. ([Living Reviews][5])

Existing research on QCA mainly treats it as a "simulation tool" or "discretization scheme": to approximate a given continuous quantum field or condensed matter system, and converge back to continuous theory in appropriate limits. This paper chooses the reverse perspective: not treating QCA as a numerical approximation of a continuous universe, but defining the "Universe Ontology" itself as a QCA object satisfying certain axioms; causal structure and "spacetime" are derived from the locality and time iteration of this object, rather than being pre-given.

More specifically, the core questions of this paper are:

1. In the context of QCA, how to define "The Universe" as a global object that simultaneously contains space, local degrees of freedom, dynamics, and initial state?

2. Can we derive a locally finite partial order on the event set starting from QCA locality and graph structure, thereby aligning with the fundamental structure of causal set theory?

3. In this axiomatic system, how to abstract observation, entropy, and the arrow of time, and recover continuous relativistic field equations, such as the Dirac equation, in appropriate limits?

Focusing on these questions, this paper constructs the Universe QCA Object $\mathfrak U_{\rm QCA}$, proves that its induced event set $(E,\preceq)$ is a locally finite partial order, and provides the continuous limit construction of Dirac-type QCA, demonstrating how to obtain the lattice version of the standard Dirac equation and its limit in the single-particle sector.

\section{Model \& Assumptions}

This section provides the basic mathematical structures and axiomatic assumptions used in this paper, including discrete space, local Hilbert space, quasi-local $C^\ast$-algebra, and the definition of QCA, and on this basis defines the "Universe QCA Object".

\subsection{Discrete Space and Graph Structure}

Let $\Lambda$ be a countable set, serving as the "lattice set" or "discrete space". Assume $\Lambda$ carries an undirected connected graph structure with edge set $E_\Lambda\subset\Lambda\times\Lambda$, satisfying:

1. $(x,y)\in E_\Lambda\Rightarrow (y,x)\in E_\Lambda$;

2. No self-loops $(x,x)\in E_\Lambda$.

Based on this, define the graph distance $\operatorname{dist}:\Lambda\times\Lambda\to\mathbb N\cup\{0\}$ as the number of edges in the shortest path connecting $x$ and $y$ (or $+\infty$ if no path exists). The connectivity assumption ensures finite distance between any $x,y\in\Lambda$.

For any $R\in\mathbb N$ and $x\in\Lambda$, define the closed ball
$$
B_R(x):=\{y\in\Lambda:\operatorname{dist}(x,y)\le R\}.
$$

Assume each $B_R(x)$ is a finite set. This assumption holds automatically on standard lattices $\mathbb Z^d$.

When $\Lambda=\mathbb Z^d$, define the translation action $\tau_a:\Lambda\to\Lambda$ as $\tau_a(x):=x+a$, where $a\in\mathbb Z^d$. This structure will be used later when discussing spatial homogeneity.

\subsection{Local Hilbert Space and Quasi-Local $C^\ast$-Algebra}

For each lattice site $x\in\Lambda$, associate a finite-dimensional Hilbert space $\mathcal H_x$, and assume there exists a fixed finite-dimensional Hilbert space $\mathcal H_{\rm cell}$ such that
$$
\mathcal H_x\simeq\mathcal H_{\rm cell}\simeq\mathbb C^d
$$
holds for all $x\in\Lambda$, where $d\in\mathbb N$ is the dimension of local degrees of freedom of the cell.

For any finite subset $F\Subset\Lambda$, define the finite volume Hilbert space
$$
\mathcal H_F:=\bigotimes_{x\in F}\mathcal H_x,
$$
and the corresponding bounded operator algebra is
$$
\mathcal A_F:=\mathcal B(\mathcal H_F).
$$
If $F\subset G\Subset\Lambda$, there is a natural embedding
$$
\iota_{F,G}:\mathcal A_F\hookrightarrow\mathcal A_G,\quad A\mapsto A\otimes\mathbf 1_{G\setminus F},
$$
where $\mathbf 1_{G\setminus F}$ is the identity operator on $\bigotimes_{x\in G\setminus F}\mathcal H_x$.

Let
$$
\mathcal A_{\rm loc}:=\bigcup_{F\Subset\Lambda}\mathcal A_F,
$$
define the quasi-local $C^\ast$-algebra as
$$
\mathcal A:=\overline{\mathcal A_{\rm loc}}^{|\cdot|},
$$
where $|\cdot|$ is the operator norm. The support $\operatorname{supp}(A)\subset\Lambda$ of an element $A\in\mathcal A_{\rm loc}$ is defined as the minimal finite set $F$ such that $A\in\mathcal A_F$. For general $A\in\mathcal A$, support can be defined as the closure of supports of local operators approximating $A$.

This construction is consistent with the standard quasi-local algebra formalism for quantum spin systems and quantum lattice systems. ([SpringerLink][7])

\subsection{States and GNS Representation}

On the $C^\ast$-algebra $\mathcal A$, a state is a positive, normalized linear functional
$$
\omega:\mathcal A\to\mathbb C,\quad \omega(A^\ast A)\ge 0,\ \omega(\mathbf 1)=1.
$$
Physically, $\omega(A)$ gives the expectation value of observable $A$.

By the GNS construction, for any state $\omega$, there exists a triple $(\pi_\omega,\mathcal H_\omega,\Omega_\omega)$, where $\pi_\omega:\mathcal A\to\mathcal B(\mathcal H_\omega)$ is a $\ast$-representation, $\Omega_\omega\in\mathcal H_\omega$ is a cyclic vector, such that
$$
\omega(A)=\langle\Omega_\omega,\pi_\omega(A)\Omega_\omega\rangle,\quad A\in\mathcal A,
$$
and the set $\{\pi_\omega(A)\Omega_\omega:A\in\mathcal A\}$ is dense in $\mathcal H_\omega$.

\subsection{Heisenberg Picture Definition of QCA}

Adopting the algebraic definition from Schumacher--Werner and subsequent works, QCA is viewed as a $\ast$-automorphism on a quasi-local algebra with finite propagation radius and commuting with translations. ([arXiv][1])

\textbf{Definition 2.1 (Quantum Cellular Automaton)} Let $R\in\mathbb N$. A map $\alpha:\mathcal A\to\mathcal A$ is called a Quantum Cellular Automaton with radius at most $R$, if it satisfies:

1. $\alpha$ is a $\ast$-automorphism of the $C^\ast$-algebra, i.e., for any $A,B\in\mathcal A$ and $\lambda\in\mathbb C$,
   $$
   \alpha(AB)=\alpha(A)\alpha(B),\quad \alpha(A^\ast)=\alpha(A)^\ast,\quad \alpha(\mathbf 1)=\mathbf 1,
   $$
   and $\alpha$ is bijective and continuous.

2. Locality: For any finite $F\Subset\Lambda$ and any $A_F\in\mathcal A_F$, there exists a finite set $G\subset\Lambda$ such that $\alpha(A_F)\in\mathcal A_G$, and satisfying
   $$
   G\subset B_R(F):=\bigcup_{x\in F}B_R(x).
   $$

3. If $\Lambda$ carries translation action $\tau_a$, there exists a corresponding translation automorphism $\theta_a:\mathcal A\to\mathcal A$, such that for all $a$ and $A\in\mathcal A$,
   $$
   \alpha\circ\theta_a=\theta_a\circ\alpha.
   $$

If there exists some finite $R$ such that the above conditions hold, $\alpha$ is called a local QCA.

From $\alpha$, integer iterations can be defined
$$
\alpha^n:=\underbrace{\alpha\circ\cdots\circ\alpha}_{n\text{ times}},\quad n\in\mathbb Z,
$$
where $\alpha^0=\mathrm{id}$, $\alpha^{-n}:=(\alpha^{-1})^n$.

\subsection{Schrödinger Picture and Unitary Implementation}

In the Schrödinger picture, states evolve with time. Given QCA $\alpha$, for any state $\omega$, define the state after step $n$ as
$$
\omega_n:=\omega\circ\alpha^{-n},\quad n\in\mathbb Z.
$$
In an appropriate GNS representation, QCA can be implemented by a unitary operator. Let $\omega$ be a faithful and $\alpha$-invariant state, i.e., $\omega\circ\alpha=\omega$, and $\omega(A^\ast A)=0\Rightarrow A=0$. Then in the GNS representation $(\pi_\omega,\mathcal H_\omega,\Omega_\omega)$, there exists a unique unitary operator $U:\mathcal H_\omega\to\mathcal H_\omega$ such that
$$
\pi_\omega(\alpha(A))=U\pi_\omega(A)U^\dagger,\quad A\in\mathcal A,
$$
and $U\Omega_\omega=\Omega_\omega$. This result is a standard conclusion of GNS formalism application; proof is given in Appendix B.

In specific models, a unitary operator $U$ is often directly specified on a given Hilbert space $\mathcal H$, setting $\alpha(A):=U^\dagger A U$, and then verifying it satisfies locality and translation symmetry conditions.

\subsection{Universe QCA Object}

Based on the above structures, define the "Universe QCA Object", summarizing space, local degrees of freedom, dynamics, and initial state into a quintuple.

\textbf{Definition 2.2 (Universe QCA Object)} A set of data
$$
\mathfrak U_{\rm QCA}=(\Lambda,\mathcal H_{\rm cell},\mathcal A,\alpha,\omega_0)
$$
is called a Universe QCA Object if it satisfies:

1. $\Lambda$ is the vertex set of a countable infinite connected graph with graph distance $\operatorname{dist}$, and for any $x\in\Lambda,R\in\mathbb N$, the ball $B_R(x)$ is finite.

2. $\mathcal H_x\simeq\mathcal H_{\rm cell}$ is a finite-dimensional Hilbert space, and $\mathcal A$ is its quasi-local $C^\ast$-algebra.

3. $\alpha:\mathcal A\to\mathcal A$ is a QCA with some finite radius $R$.

4. If $\Lambda$ carries translation action, $\alpha$ commutes with the corresponding translation automorphism.

5. $\omega_0:\mathcal A\to\mathbb C$ is a normalized state, called the initial cosmic state; for $n\in\mathbb Z$, define
   $$
   \omega_n:=\omega_0\circ\alpha^{-n}.
   $$

In this definition, $\Lambda$ and $\mathcal H_{\rm cell}$ fix the "type of discrete space and local degrees of freedom of the universe", $\alpha$ fixes the "dynamical laws", and $\omega_0$ fixes the "initial conditions".

\section{Main Results (Theorems and alignments)}

This section presents the main results: the causal structure derived from the Universe QCA Object, the equivalent characterization of its local finiteness and QCA locality, and the continuous limit of Dirac-type QCA.

\subsection{Event Set and Causal Reachability Relation}

In the Universe QCA Object, the natural event set is
$$
E:=\Lambda\times\mathbb Z,
$$
where $(x,n)$ denotes the event at lattice site $x$ at time step $n$. For $e=(x,n)\in E$, denote spatial coordinate as $\operatorname{sp}(e)=x$, and time as $\operatorname{tm}(e)=n$.

The finite propagation radius of QCA determines the "discrete light cone". Let the propagation radius of $\alpha$ be $R$. For any $n<m$ and local operator $B_y$ supported on a single point $\{y\}$, by locality we have
$$
\operatorname{supp}\bigl(\alpha^{m-n}(B_y)\bigr)\subset B_{R(m-n)}(y).
$$
Define the geometric relation
$$
(x,n)\leq_{\rm geo}(y,m)\iff m\ge n,\ \operatorname{dist}(x,y)\le R(m-n).
$$
To obtain causal relations, we need to link statistical correlation with geometric light cones. Define:

\textbf{Definition 3.1 (Causal Reachability Relation)} For $(x,n),(y,m)\in E$, we say
$$
(x,n)\preceq(y,m)
$$
if:

1. $m\ge n$;

2. There exists a local operator $A_x\in\mathcal A_{\{x\}}$ supported on $\{x\}$, and $B_y\in\mathcal A_{\{y\}}$, and some state $\omega$, such that
   $$
   \omega\bigl(\alpha^n(A_x)\alpha^m(B_y)\bigr)\neq\omega\bigl(\alpha^n(A_x)\bigr)\omega\bigl(\alpha^m(B_y)\bigr).
   $$
   That is, a local perturbation at $x$ at time $n$ can produce a statistical influence at $y$ at time $m$.

\subsection{Theorem 1: QCA Causal Structure is a Locally Finite Partial Order}

\textbf{Theorem 3.2 (Partial Order and Local Finiteness)} In any Universe QCA Object $\mathfrak U_{\rm QCA}$, the relation $\preceq$ induced by Definition 3.1 is equivalent to the geometric relation $\leq_{\rm geo}$, i.e.,
$$
(x,n)\preceq(y,m)\iff (x,n)\leq_{\rm geo}(y,m).
$$
Thus:

1. $(E,\preceq)$ is a partial order set (reflexive, antisymmetric, transitive);

2. $(E,\preceq)$ is locally finite, i.e., for any $e_1,e_2\in E$, the causal interval
   $$
   I(e_1,e_2):=\{e\in E: e_1\preceq e\preceq e_2\}
   $$
   is a finite set.

Therefore, $(E,\preceq)$ shares the same structural type as "locally finite partial order" in causal set theory.

The proof idea relies on two points: First, if $(x,n)$ is within the geometric light cone of $(y,m)$, there exist local operators and states creating non-trivial statistical correlations; Second, if the distance condition is not met, operator supports can be decomposed into disjoint regions, leading to factorization of expectation values and no causal influence. Local finiteness is given by the finiteness of each ball $B_R(x)$ in the graph structure and the finite difference in time steps. Detailed proof in Appendix A.

\subsection{Theorem 2: Equivalent Characterization of QCA Locality and Locally Finite Partial Order}

Theorem 3.2 gives the construction from QCA to causal sets. The reverse statement is: Given a discrete time dynamics and a locally finite partial order on the event set, the propagation radius of QCA can be reconstructed from the "light cone interface" of this partial order.

\textbf{Theorem 3.3 (Equivalent Characterization)} Let $\alpha:\mathcal A\to\mathcal A$ be a $C^\ast$-algebra automorphism, $\Lambda$ be a connected graph, $E=\Lambda\times\mathbb Z$. The following two are equivalent:

1. There exists $R<\infty$, such that for any finite $F\Subset\Lambda$ and $A_F\in\mathcal A_F$, $\operatorname{supp}(\alpha(A_F))\subset B_R(F)$;

2. There exists a relation $\preceq$ such that $(E,\preceq)$ is a locally finite partial order, and satisfies: if $(x,n)\preceq(y,m)$ and $m=n+1$, then $\operatorname{dist}(x,y)\le R$, and for all $(x,n)$, its one-step future reachable point set is contained in $B_R(x)\times\{n+1\}$.

In other words, the finite propagation radius condition of QCA is equivalent to the locally finite partial order structure on the event set where "causal links per time step only connect finite neighborhoods". This equivalent characterization restates the QCA definition of Schumacher--Werner and subsequent works in the language of causal sets as "discrete time + locally finite partial order + automorphism on quasi-local algebra".

\subsection{Theorem 3: Continuous Limit of Dirac-Type QCA}

QCA is widely used as a discretization framework for continuous quantum field theory. Below we present the continuous limit result for 1D Dirac-type QCA, demonstrating how the standard Dirac equation is obtained within the Universe QCA Object.

Consider $\Lambda=\mathbb Z$, each site carrying $\mathcal H_x\simeq\mathbb C^2$, basis vectors denoted $|x,\uparrow\rangle,|x,\downarrow\rangle$. Define the time step evolution operator
$$
U:=S\circ R,
$$
where

1. Local spin rotation
   $$
   R_x=\mathrm e^{-\mathrm i\theta\sigma_y}=\cos\theta\,\mathbf 1-\mathrm i\sin\theta\,\sigma_y
   $$
   acts independently on each site;

2. Conditional translation
   $$
   S|x,\uparrow\rangle=|x+1,\uparrow\rangle,\quad S|x,\downarrow\rangle=|x-1,\downarrow\rangle.
   $$

This $U$ is a unitary operator with propagation radius $R=1$, corresponding to 1D Dirac-type QCA. Denote the single-particle state at time step $n$ as
$$
|\psi_n\rangle=\sum_{x\in\mathbb Z}\bigl(\psi_n^\uparrow(x)|x,\uparrow\rangle+\psi_n^\downarrow(x)|x,\downarrow\rangle\bigr),
$$
evolution equation is
$$
|\psi_{n+1}\rangle=U|\psi_n\rangle.
$$
Let lattice spacing $\varepsilon>0$, define continuous coordinates $X=\varepsilon x$, $T=\varepsilon n$. Take rotation angle scaling as $\theta=\varepsilon m$, where $m>0$ is constant. If assuming the wavefunction is smooth in the limit $\varepsilon\to 0$, satisfying
$$
\psi_n^\uparrow(x)\approx\psi^\uparrow(X,T),\quad \psi_n^\downarrow(x)\approx\psi^\downarrow(X,T),
$$
then retaining up to first order terms in Taylor expansion, we obtain the following continuous equations:
$$
\partial_T\psi^\uparrow=-\partial_X\psi^\uparrow-m\,\psi^\downarrow,\quad
\partial_T\psi^\downarrow=\partial_X\psi^\downarrow+m\,\psi^\uparrow.
$$
Denoting $\Psi=(\psi^\uparrow,\psi^\downarrow)^{\mathsf T}$ as a two-component spinor, the above can be written as
$$
\mathrm i\partial_T\Psi=\bigl(-\mathrm i\sigma_z\partial_X+m\sigma_y\bigr)\Psi,
$$
which is a standard form of the 1D Dirac equation. This result shows that in appropriate scaling limits, the Dirac-type QCA in the Universe QCA Object produces dynamics consistent with the continuous Dirac equation in the single-particle sector. Detailed derivation in Appendix C.

\section{Proofs}

This section provides the proof framework for the main theorems; full details are in appendices.

\subsection{Overview of Proof for Theorem 3.2}

We need to prove two points:

1. Causal relation $\preceq$ is equivalent to geometric relation $\leq_{\rm geo}$;

2. On this basis, $(E,\preceq)$ is a locally finite partial order.

\textbf{Causal Reachability within Geometric Light Cone} If $(x,n)\leq_{\rm geo}(y,m)$, then $m\ge n$ and $\operatorname{dist}(x,y)\le R(m-n)$. By locality, for any $B_y$ supported on $\{y\}$,
$$
\operatorname{supp}\bigl(\alpha^{m-n}(B_y)\bigr)\subset B_{R(m-n)}(y),
$$
so this support contains degrees of freedom near $x$. Choose $A_x$ and $B_y$ such that $\alpha^{m-n}(B_y)$ does not commute with $A_x$ near $x$, and pick a state $\omega$ capable of resolving this non-commutativity, then
$$
\omega\bigl(\alpha^n(A_x)\alpha^m(B_y)\bigr)-\omega\bigl(\alpha^n(A_x)\bigr)\omega\bigl(\alpha^m(B_y)\bigr)
$$
can be constructed to be non-zero, thus $(x,n)\preceq(y,m)$. This construction utilizes the finite dimensionality of local Hilbert spaces and freedom to choose Pauli-type operators.

\textbf{Non-Causality Outside Geometric Light Cone} If $m\ge n$ and $\operatorname{dist}(x,y)>R(m-n)$, then the support of $\alpha^{m-n}(B_y)$ is completely contained in some finite set $G\subset\Lambda$, and $x\notin G$. Select a finite set $F\Subset\Lambda$ containing $x$ and disjoint from $G$, then
$$
A_x\in\mathcal A_F,\quad \alpha^{m-n}(B_y)\in\mathcal A_G,\quad F\cap G=\varnothing.
$$
Under tensor decomposition $\mathcal A_{F\cup G}\cong\mathcal A_F\otimes\mathcal A_G$, they act on different factors. For any product state $\omega_F\otimes\omega_G$,
$$
\omega\bigl(\alpha^n(A_x)\alpha^m(B_y)\bigr)=\omega_F(\dots)\omega_G(\dots),
$$
general states can be constructed by limits of product states, yielding factorization. Thus $(x,n)\not\preceq(y,m)$.

Combining the above two points gives equivalence of $\preceq$ and $\leq_{\rm geo}$. Partial order properties follow directly from reflexivity, antisymmetry, and transitivity of $\leq_{\rm geo}$.

Local finiteness: For $e_1=(x,n)\preceq e_2=(y,m)$, if $m<n$ the interval is empty. If $m\ge n$, any $e=(z,k)\in I(e_1,e_2)$ satisfies
$$
n\le k\le m,\quad \operatorname{dist}(x,z)\le R(k-n),\quad \operatorname{dist}(z,y)\le R(m-k),
$$
thus
$$
\operatorname{dist}(x,z)\le R(m-n),\quad \operatorname{dist}(y,z)\le R(m-n),
$$
i.e., $z\in B_{R(m-n)}(x)\cap B_{R(m-n)}(y)$, which is finite; and $k$ can only take finitely many integer values, so $I(e_1,e_2)$ is a finite set.

Full technical details in Appendix A.

\subsection{Overview of Proof for Theorem 3.3}

The construction from finite propagation condition of QCA to locally finite partial order is done in Theorem 3.2. Reverse direction: Assume $(E,\preceq)$ is a locally finite partial order, and there exists integer $R$ such that for any $(x,n)$, its one-step future $\{(y,n+1):(x,n)\preceq(y,n+1)\}$ satisfies $\operatorname{dist}(x,y)\le R$. Using this condition, one can define "neighborhood propagation bound" for each time step, characterizing the propagation radius of automorphism $\alpha$ as no more than $R$.

Specifically, examine the set of all lattice points that may have correlations after one step of evolution on any finite region $F$, and use local finiteness to ensure this set remains finite. Further prove that for any $A_F\in\mathcal A_F$, support of $\alpha(A_F)$ is contained in $B_R(F)$, obtaining the finite propagation radius condition. Complete proof requires formalizing the relationship between "statistical correlation" and "support containment", see Appendix A.

\subsection{Overview of Proof for Theorem 3.4}

Proof of continuous limit for Dirac-type QCA is based on standard scaling limit analysis. Key steps are:

1. Scale discrete space and time coordinates as $X=\varepsilon x,T=\varepsilon n$, and let rotation angle $\theta=\varepsilon m$ scale with $\varepsilon$;

2. Perform first-order Taylor expansion for $\cos\theta,\sin\theta$ and wavefunction $\psi_n^{\uparrow,\downarrow}(x\pm 1)$;

3. Substitute expansions into discrete evolution equations
   $$
   \psi_{n+1}^\uparrow(x)=\cos\theta\,\psi_n^\uparrow(x-1)-\sin\theta\,\psi_n^\downarrow(x-1),
   $$
   $$
   \psi_{n+1}^\downarrow(x)=\sin\theta\,\psi_n^\uparrow(x+1)+\cos\theta\,\psi_n^\downarrow(x+1)
   $$
   replacing terms with continuous functions and derivatives, ignoring $\varepsilon^2$ and higher terms;

4. Obtain system of first-order partial differential equations, rearrange into spinor form to get 1D Dirac equation.

This construction aligns with typical methods in quantum walks and QCA simulation of Dirac equations. Full expansion calculation in Appendix C.

\section{Model Apply}

This section demonstrates the application of Universe QCA Object in specific models, focusing on the Dirac-type QCA example and its embedding in the Universe QCA framework.

\subsection{1D Dirac-Type Universe Section}

Let $\Lambda=\mathbb Z$, $\mathcal H_{\rm cell}\simeq\mathbb C^2$, $\mathcal A$ be the corresponding quasi-local algebra. Choose Dirac-type QCA evolution $\alpha(A)=U^\dagger A U$, where $U=S\circ R$ as described before. Take a translation-invariant ground state $\omega_0$ as initial state, e.g., spin-up filled state, KMS state, or Gaussian state.

Then the quintuple
$$
\mathfrak U_{\rm QCA}^{\rm Dirac}=(\mathbb Z,\mathcal H_{\rm cell},\mathcal A,\alpha,\omega_0)
$$
constitutes a Universe QCA Object, whose event set $E=\mathbb Z\times\mathbb Z$ under QCA radius $R=1$ induces causal structure
$$
(x,n)\preceq(y,m)\iff m\ge n,\ \operatorname{dist}(x,y)\le m-n.
$$
In the single-particle sector and $\varepsilon\to 0$ scaling limit, the effective dynamics of this model is governed by the 1D Dirac equation, demonstrating how to reconstruct low-energy behavior of relativistic fields within the Universe QCA framework.

\subsection{Generalization to Higher Dimensions and Multi-Field Degrees of Freedom}

On higher dimensional spaces $\Lambda=\mathbb Z^d$, similar constructions can be used, introducing more complex local Hilbert spaces $\mathcal H_{\rm cell}$ (e.g., carrying multiple spins and internal degrees of freedom) to construct corresponding Dirac, Weyl, or Majorana type QCAs. Existing work shows that through appropriate QCA, effective descriptions of various free fields and some interacting field theories can be obtained in the continuum limit. In the Universe QCA perspective, these models correspond to different "local universe sections" or "different universe instances".

\section{Engineering Proposals}

Although this paper constructs an abstract axiomatic framework, QCA itself has clear experimental and engineering implementation paths. In the "Universe = QCA" perspective, existing quantum simulation and quantum information platforms can be viewed as "sub-universe simulations" of this cosmic structure.

\subsection{Implementation of Quantum Walks and QCA on Experimental Platforms}

1. **Optical Lattices and Cold Atoms**

   In 1D or 2D optical lattices, using internal degrees of freedom of atoms as spins and lattice sites as discrete space, Dirac-type QCA can be implemented by alternating spin rotations and conditional translations. Various quantum walks and Dirac-type evolutions have been experimentally realized, whose Hamiltonians approximate the structure of Dirac-type QCA in this paper.

2. **Ion Traps and Superconducting Qubit Arrays**

   In linear ion traps or 2D superconducting qubit arrays, local single and two-qubit gates can be combined into finite-depth quantum circuits to implement QCA with given propagation radius. specifically, finite radius QCA can be physically realized as a combination of finite depth quantum circuits and finite range interactions.

3. **Topological Qubits and Fermionic QCA**

   In topological superconductors or Majorana linear arrays, fermionic QCA with $\mathbb Z_2$ index can be constructed to simulate topological phases and boundary modes. Such models are compatible with the Universe QCA object framework and can obtain consistent descriptions in causal structure and quasi-particle propagation.

\subsection{Verification Ideas for Universe QCA Axioms}

Although "the QCA of the entire universe" cannot be directly manipulated in experiments, the rationality of this framework can be indirectly verified by:

1. **Propagation Radius and "Speed of Light"**

   In QCA models, propagation radius $R$ and time step determine the maximum propagation speed. If universe ontology is indeed QCA, the physically observed maximum propagation speed (e.g., speed of light) should be consistent with this discrete propagation bound, presenting Lorentz invariance in the continuous limit.

2. **Local Finiteness and Information Propagation**

   In actual quantum systems, the Lieb--Robinson bound gives bounded operator propagation with finite velocity. QCA as a strictly local model has clear propagation bounds. Observing information propagation cones in various quantum systems provides indirect evidence for "cosmic dynamics having QCA-type locality".

3. **Continuum Limit and Field Equation Recovery**

   By constructing QCA and simulating its evolution on experimental platforms, observing whether its effective behavior at large scales follows known relativistic field equations is a key path to test whether "Universe QCA Object" suffices to support existing physics.

\section{Discussion (risks, boundaries, past work)}

This section discusses the applicable boundaries, potential risks, and relationship with existing work of the Universe QCA framework.

\subsection{Relationship with Traditional QCA Literature}

Works by Schumacher--Werner, Richter--Werner, et al. define QCA using quasi-local algebras and finite propagation conditions, providing a mature foundation for the dynamics part of the Universe QCA Object in this paper. Arrighi and Farrelly reviewed QCA structural theorems, classification, and quantum simulation applications, showing connections between QCA and topological phases and quantum information processing. These works mainly focus on "discretized descriptions of given physical systems", while this paper elevates QCA to an axiomatic object of universe ontology.

The causal set program starts from causal partial order and local finiteness, treating spacetime as a discrete partial order set and recovering continuous geometry in the limit. This paper shows that Universe QCA Object naturally induces a locally finite partial order $(E,\preceq)$, aligning with causal set structure, thus unifying QCA and causal sets in one framework: the former endows each event with local quantum degrees of freedom and dynamics, the latter characterizes "spacetime structure" via partial order and local finiteness.

\subsection{Conceptual and Technical Boundaries}

1. **Lorentz Invariance and Continuum Limit**

   Implementing exact Lorentz invariance on discrete lattices is difficult. QCA models typically exhibit Lorentz invariant behavior approximately in low-energy and long-wavelength limits, but show "lattice anisotropy" and dispersion at high energy scales. The Universe QCA framework needs to provide continuous limit construction and error estimation for the system to ensure physics at observable energy scales matches existing experiments.

2. **Encoding of Gravity and Curvature**

   This paper discusses causal structure and quantum dynamics only under fixed graph structure $\Lambda$ and fixed propagation radius $R$, not yet incorporating gravity and curvature effects into QCA axioms. To interface with general relativity or quantum gravity schemes, one needs to introduce graph structures and propagation parameters that vary with state or energy-momentum distribution in the QCA framework, or reconstruct effective metrics and curvature in the causal set limit.

3. **Interaction and Renormalization**

   Although QCA is known to simulate free fields and some interacting field theories, complete renormalization group structure and interacting flows in the continuum limit remain open problems. The Universe QCA framework needs to be compatible with renormalization group and effective field theory mechanisms to reproduce physical phenomena across wide energy scales.

4. **Ontology and Testability**

   Defining the universe as a single QCA object is an ontological choice. Its testability depends on whether observable predictions different from continuous theories can be derived from this framework, such as deviations in high-energy scattering or discrete signs in the early universe.

\subsection{Alignment with Past Work}

This paper aligns structurally with the following directions:

1. Using quasi-local $C^\ast$-algebras and finite propagation automorphisms to characterize discrete time dynamics, consistent with reversible QCA literature;

2. Using locally finite partial orders to characterize causal structure, consistent with basic definitions of causal set theory;

3. Using Dirac-type QCA to implement discretization and continuous limit of relativistic fields, consistent with typical models in quantum walks and QCA simulation literature.

The difference lies in: this paper unifies these three threads into a single axiomatic object $\mathfrak U_{\rm QCA}$, defining "The Universe" as the totality satisfying these axioms, rather than constructing QCA on a given continuous spacetime.

\section{Conclusion}

Within the frameworks of Quantum Cellular Automata and Causal Sets, this paper proposes an axiomatic characterization of "Universe as Quantum Discrete Cellular Automaton". Main contents include:

1. Constructing Universe QCA Object $\mathfrak U_{\rm QCA}$ using countable connected graph $\Lambda$, finite-dimensional local Hilbert space $\mathcal H_{\rm cell}$, quasi-local $C^\ast$-algebra $\mathcal A$, and finite propagation radius QCA $\alpha$, and endowing it with statistical structure via initial state $\omega_0$.

2. Deriving causal reachability relation $\preceq$ on event set $E=\Lambda\times\mathbb Z$ from QCA locality and graph structure, proving $(E,\preceq)$ is a locally finite partial order, aligning with fundamental structure of causal set theory.

3. Giving equivalent characterization between QCA locality and locally finite partial order on event set, restating Schumacher--Werner's QCA definition as a unified form of "discrete time + locally finite partial order + automorphism on quasi-local algebra".

4. Constructing 1D Dirac-type QCA and obtaining continuous Dirac equation in appropriate scaling limit, demonstrating Universe QCA Object can reproduce relativistic field theory in low-energy limit.

5. Briefly discussing abstract expressions of observation, entropy, and arrow of time in QCA universe, and potential experimental verification paths and docking with existing QCA and causal set literature.

This framework provides a rigorous operator algebraic and causal structural foundation for discrete universe models. Future work directions include: explicitly introducing gravity and curvature structure in QCA axioms; constructing multi-scale QCA universe compatible with renormalization group; extracting potential observable predictions in cosmology and high-energy physics contexts and comparing with experimental results.

\section*{Acknowledgements, Code Availability}

This work relies on existing results in Quantum Cellular Automata, quasi-local $C^\ast$-algebras, and Causal Set theory. We express gratitude to the relevant research communities for the long-term development of theoretical foundations.

The construction of Dirac-type QCA and Universe QCA Object in the text are analytical forms, without specific software implementation code attached. Any programmable quantum simulation platform (such as quantum circuit simulators, cold atom or superconducting qubit devices) can implement corresponding models according to the evolution operator decomposition given in Appendix C.

\begin{thebibliography}{99}

\bibitem{Schumacher2004} B. Schumacher, R. F. Werner, "Reversible Quantum Cellular Automata", arXiv:quant-ph/0405174, 2004.

\bibitem{Arrighi2019} P. Arrighi, "An Overview of Quantum Cellular Automata", Natural Computing, 18, 885–899, 2019.

\bibitem{Farrelly2020} T. Farrelly, "A Review of Quantum Cellular Automata", Quantum, 4, 368, 2020.

\bibitem{Richter1995} S. Richter, R. F. Werner, "Ergodicity of Quantum Cellular Automata", arXiv:cond-mat/9504001, 1995.

\bibitem{Surya2019} S. Surya, "The Causal Set Approach to Quantum Gravity", Living Reviews in Relativity, 22, 5, 2019.

\bibitem{Reid2008} D. D. Reid, "Causal Sets: An Alternative View of Spacetime Structure", AIP Conf. Proc. 991, 1–9, 2008.

\bibitem{Brightwell1991} G. Brightwell, R. Gregory, "Structure of Random Discrete Spacetime", Physical Review Letters, 66, 260–263, 1991.

\bibitem{Jones2024} C. Jones, "DHR Bimodules of Quasi-Local Algebras and Symmetric Fusion Categories", J. Eur. Math. Soc., 2024.

\bibitem{TosiniPhD} A. Tosini, "A Quantum Cellular Automata Framework for Quantum Field Theory", PhD Thesis, etc.

\end{thebibliography}

\appendix

\section{Appendix A: Detailed Proof of QCA Causal Structure as Locally Finite Partial Order}

This appendix completes technical details for Theorem 3.2 and Theorem 3.3.

\subsection{A.1 Equivalence of Causal Relation and Geometric Relation}

Recall definition: Geometric relation
$$
(x,n)\leq_{\rm geo}(y,m)\iff m\ge n,\ \operatorname{dist}(x,y)\le R(m-n),
$$
Causal relation $\preceq$ defined as existence of local operator and state producing non-trivial statistical correlation.

\textbf{Proposition A.1 (Causality within Geometric Light Cone)} If $(x,n)\leq_{\rm geo}(y,m)$, then $(x,n)\preceq(y,m)$.

*Proof*: $\operatorname{dist}(x,y)\le R(m-n)$ ensures
$$
x\in B_{R(m-n)}(y).
$$
For any $B_y\in\mathcal A_{\{y\}}$, by locality
$$
\operatorname{supp}\bigl(\alpha^{m-n}(B_y)\bigr)\subset B_{R(m-n)}(y).
$$
Thus there exists a neighborhood $N(x)\subset B_{R(m-n)}(y)$ of $x$, such that $\alpha^{m-n}(B_y)$ has non-trivial action in $\mathcal A_{N(x)}$. Choose $A_x\in\mathcal A_{\{x\}}$ such that it does not commute with $\alpha^{m-n}(B_y)$ in $\mathcal A_{N(x)}$. For example, on $\mathcal H_x\simeq\mathbb C^2$, take Pauli operators $\sigma_x,\sigma_z$ etc., construct operators satisfying $[A_x,\alpha^{m-n}(B_y)]\neq 0$ by linear combination.

Under some pure state $\omega$, choose $\omega$ such that
$$
\omega\bigl([A_x,\alpha^{m-n}(B_y)]\bigr)\neq 0,
$$
then
$$
\omega\bigl(\alpha^n(A_x)\alpha^m(B_y)\bigr)
=\omega\bigl(\alpha^n(A_x\alpha^{m-n}(B_y))\bigr),
$$
$$
\omega\bigl(\alpha^n(A_x)\bigr)\omega\bigl(\alpha^m(B_y)\bigr)
=\omega\bigl(\alpha^n(A_x)\bigr)\omega\bigl(\alpha^n(\alpha^{m-n}(B_y))\bigr),
$$
the difference relates to $\omega\bigl([A_x,\alpha^{m-n}(B_y)]\bigr)$ and can be constructed to be non-zero. Thus $(x,n)\preceq(y,m)$. Q.E.D.

\textbf{Proposition A.2 (No Causality Outside Geometric Light Cone)} If $m\ge n$ and $\operatorname{dist}(x,y)>R(m-n)$, then for any local operators $A_x\in\mathcal A_{\{x\}}, B_y\in\mathcal A_{\{y\}}$ and any state $\omega$,
$$
\omega\bigl(\alpha^n(A_x)\alpha^m(B_y)\bigr)
=\omega\bigl(\alpha^n(A_x)\bigr)\omega\bigl(\alpha^m(B_y)\bigr).
$$

*Proof*: We have
$$
\alpha^m(B_y)=\alpha^n\bigl(\alpha^{m-n}(B_y)\bigr),
$$
and
$$
\operatorname{supp}\bigl(\alpha^{m-n}(B_y)\bigr)\subset B_{R(m-n)}(y).
$$
Since $\operatorname{dist}(x,y)>R(m-n)$, we can take finite $F\Subset\Lambda$ satisfying $x\in F$, and $F\cap B_{R(m-n)}(y)=\varnothing$. Then
$$
A_x\in\mathcal A_F,\quad \alpha^{m-n}(B_y)\in\mathcal A_G,
$$
where $G\subset B_{R(m-n)}(y)$, and $F\cap G=\varnothing$.

Under tensor decomposition $\mathcal A_{F\cup G}\cong\mathcal A_F\otimes\mathcal A_G$, we have
$$
A_x=A_F\otimes\mathbf 1_G,\quad \alpha^{m-n}(B_y)=\mathbf 1_F\otimes B_G,
$$
thus
$$
\alpha^n(A_x)\alpha^m(B_y)
=\alpha^n(A_F\otimes\mathbf 1_G)\alpha^n(\mathbf 1_F\otimes B_G)
=\alpha^n(A_F\otimes B_G).
$$
For any product state $\omega=\omega_F\otimes\omega_G$,
$$
\omega\bigl(\alpha^n(A_x)\alpha^m(B_y)\bigr)
=\omega_F\bigl(\alpha_F^n(A_F)\bigr)\omega_G\bigl(\alpha_G^n(B_G)\bigr),
$$
matching
$$
\omega\bigl(\alpha^n(A_x)\bigr)\omega\bigl(\alpha^m(B_y)\bigr)
=\omega_F\bigl(\alpha_F^n(A_F)\bigr)\omega_G\bigl(\alpha_G^n(B_G)\bigr).
$$
General states can be obtained by weak-$^\ast$ limits of product states, yielding factorization relation. Thus no non-trivial statistical correlation exists, i.e., $(x,n)\not\preceq(y,m)$. Q.E.D.

Propositions A.1 and A.2 jointly give
$$
(x,n)\preceq(y,m)\iff(x,n)\leq_{\rm geo}(y,m),
$$
thus causal relation is equivalent to geometric light cone.

\subsection{A.2 Proof of Partial Order Properties and Local Finiteness}

By geometric relation definition:

1. Reflexivity: For any $(x,n)$, $\operatorname{dist}(x,x)=0\le R(0)$, and $n\ge n$, so $(x,n)\leq_{\rm geo}(x,n)$, thus $(x,n)\preceq(x,n)$.

2. Antisymmetry: If $(x,n)\leq_{\rm geo}(y,m)$ and $(y,m)\leq_{\rm geo}(x,n)$, then $m\ge n$ and $n\ge m$ implies $m=n$, then
   $$
   \operatorname{dist}(x,y)\le 0,\quad \operatorname{dist}(y,x)\le 0,
   $$
   so $x=y$, thus $(x,n)=(y,m)$.

3. Transitivity: If $(x,n)\leq_{\rm geo}(y,m)$ and $(y,m)\leq_{\rm geo}(z,k)$, then $m\ge n$, $k\ge m$ and
   $$
   \operatorname{dist}(x,y)\le R(m-n),\quad \operatorname{dist}(y,z)\le R(k-m)
   $$
   hold. By triangle inequality of graph distance,
   $$
   \operatorname{dist}(x,z)\le\operatorname{dist}(x,y)+\operatorname{dist}(y,z)
   \le R(m-n)+R(k-m)=R(k-n),
   $$
   and $k\ge n$, so $(x,n)\leq_{\rm geo}(z,k)$, i.e., $(x,n)\preceq(z,k)$.

Local Finiteness: Let $e_1=(x,n)\preceq e_2=(y,m)$. If $m<n$, interval is empty. If $m\ge n$, then for any $e=(z,k)\in I(e_1,e_2)$, by geometric equivalence
$$
n\le k\le m,
$$
$$
\operatorname{dist}(x,z)\le R(k-n),\quad \operatorname{dist}(z,y)\le R(m-k).
$$
Therefore
$$
\operatorname{dist}(x,z)\le R(m-n),\quad \operatorname{dist}(y,z)\le R(m-n),
$$
i.e., $z\in B_{R(m-n)}(x)\cap B_{R(m-n)}(y)$. By assumption on graph structure, this intersection is a finite set, denote its cardinality $N<\infty$. On the other hand, $k$ can only take values in finite set $\{n,n+1,\dots,m\}$, so possible $(z,k)$ combinations are at most $N\cdot(m-n+1)$, thus $I(e_1,e_2)$ is finite.

This completes the proof of Theorem 3.2.

\section{Appendix B: Proof of QCA Unitary Implementation Theorem}

This appendix proves that QCA can be implemented by a unique unitary operator in the GNS representation of an $\alpha$-invariant faithful state.

Let $\alpha:\mathcal A\to\mathcal A$ be a QCA, $\omega$ be a faithful and $\alpha$-invariant state, i.e., $\omega\circ\alpha=\omega$, and $\omega(A^\ast A)=0\Rightarrow A=0$. Let $(\pi_\omega,\mathcal H_\omega,\Omega_\omega)$ be its GNS representation.

\subsection{B.1 Construction and Isometry of Operator $U$}

Define linear operator on dense subspace $\mathcal D_0:=\{\pi_\omega(A)\Omega_\omega:A\in\mathcal A\}$
$$
U_0:\mathcal D_0\to\mathcal H_\omega,\quad
U_0\bigl(\pi_\omega(A)\Omega_\omega\bigr):=\pi_\omega(\alpha(A))\Omega_\omega.
$$
For any $A,B\in\mathcal A$, we have
$$
\langle U_0\pi_\omega(A)\Omega_\omega,U_0\pi_\omega(B)\Omega_\omega\rangle
=\langle\pi_\omega(\alpha(A))\Omega_\omega,\pi_\omega(\alpha(B))\Omega_\omega\rangle
=\omega\bigl(\alpha(A)^\ast\alpha(B)\bigr).
$$
Using $\alpha$ is a $\ast$-automorphism and $\omega\circ\alpha=\omega$,
$$
\omega\bigl(\alpha(A)^\ast\alpha(B)\bigr)
=\omega\bigl(A^\ast B\bigr)
=\langle\pi_\omega(A)\Omega_\omega,\pi_\omega(B)\Omega_\omega\rangle.
$$
Thus $U_0$ preserves inner product on $\mathcal D_0$, is an isometric linear operator. Since $\mathcal D_0$ is dense in $\mathcal H_\omega$, $U_0$ uniquely extends to bounded operator $U:\mathcal H_\omega\to\mathcal H_\omega$, and $U$ is isometric, i.e.,
$$
\langle U\phi,U\psi\rangle=\langle\phi,\psi\rangle,\quad \phi,\psi\in\mathcal H_\omega.
$$

\subsection{B.2 Surjectivity and Unitarity}

Need to prove $U$ is surjective. For any $B\in\mathcal A$, since $\alpha$ is bijective, there exists $A\in\mathcal A$ such that $B=\alpha(A)$. Thus
$$
\pi_\omega(B)\Omega_\omega=\pi_\omega(\alpha(A))\Omega_\omega=U_0\pi_\omega(A)\Omega_\omega\in\operatorname{Ran}(U_0).
$$
Therefore $\operatorname{Ran}(U_0)$ contains $\{\pi_\omega(B)\Omega_\omega:B\in\mathcal A\}$, which is dense in $\mathcal H_\omega$. Since $U$ is continuous extension of $U_0$, $\operatorname{Ran}(U)$ is closure of $\operatorname{Ran}(U_0)$, both closed and dense, thus $\operatorname{Ran}(U)=\mathcal H_\omega$, $U$ is surjective isometry, i.e., unitary operator.

\subsection{B.3 Conjugate Action Implements $\alpha$}

For any $A,B\in\mathcal A$,
$$
U\pi_\omega(A)U^\dagger\pi_\omega(B)\Omega_\omega
=U\pi_\omega(A)\bigl(U^\dagger\pi_\omega(B)\Omega_\omega\bigr).
$$
Note that
$$
U^\dagger\pi_\omega(B)\Omega_\omega
$$
is some vector, and by definition can be written as linear combination of GNS vectors. More convenient is to calculate directly on $\mathcal D_0$:
$$
\begin{aligned}
U\pi_\omega(A)U^\dagger\pi_\omega(B)\Omega_\omega
&=U\pi_\omega(A)\bigl(U^\dagger\pi_\omega(B)\Omega_\omega\bigr)\\
&=U\pi_\omega(A)\bigl(\pi_\omega(\alpha^{-1}(B))\Omega_\omega\bigr)\\
&=U_0\bigl(\pi_\omega(A\alpha^{-1}(B))\Omega_\omega\bigr)\\
&=\pi_\omega(\alpha(A\alpha^{-1}(B)))\Omega_\omega\\
&=\pi_\omega(\alpha(A)B)\Omega_\omega\\
&=\pi_\omega(\alpha(A))\pi_\omega(B)\Omega_\omega
\end{aligned}
$$
Since $\{\pi_\omega(B)\Omega_\omega:B\in\mathcal A\}$ is dense in $\mathcal H_\omega$, above shows
$$
U\pi_\omega(A)U^\dagger=\pi_\omega(\alpha(A))
$$
holds on entire Hilbert space.

Finally,
$$
U\Omega_\omega=U\pi_\omega(\mathbf 1)\Omega_\omega
=\pi_\omega(\alpha(\mathbf 1))\Omega_\omega
=\pi_\omega(\mathbf 1)\Omega_\omega
=\Omega_\omega,
$$
showing GNS vector is $U$-invariant. This completes proof of unitary implementation theorem.

\section{Appendix C: 1D Dirac-Type QCA and Continuous Limit of Dirac Equation}

This appendix gives specific construction and continuous limit derivation of 1D Dirac-type QCA.

\subsection{C.1 Model Definition and Discrete Evolution Equation}

Take $\Lambda=\mathbb Z$, each site carries $\mathcal H_x\simeq\mathbb C^2$, basis vectors $|x,\uparrow\rangle,|x,\downarrow\rangle$. Overall Hilbert space formally
$$
\mathcal H=\bigotimes_{x\in\mathbb Z}\mathcal H_x.
$$
Define local spin rotation operator
$$
R_x=\mathrm e^{-\mathrm i\theta\sigma_y}
=\cos\theta\,\mathbf 1-\mathrm i\sin\theta\,\sigma_y,
$$
global rotation
$$
R:=\bigotimes_{x\in\mathbb Z}R_x.
$$
Define conditional translation operator $S$ action on single particle basis as
$$
S|x,\uparrow\rangle=|x+1,\uparrow\rangle,\quad S|x,\downarrow\rangle=|x-1,\downarrow\rangle.
$$
Time step evolution operator is
$$
U:=S\circ R.
$$
$U$ is unitary with propagation radius $R=1$.

In single particle sector, write
$$
|\psi_n\rangle=\sum_{x\in\mathbb Z}\bigl(\psi_n^\uparrow(x)|x,\uparrow\rangle+\psi_n^\downarrow(x)|x,\downarrow\rangle\bigr),
$$
discrete evolution is
$$
|\psi_{n+1}\rangle=U|\psi_n\rangle.
$$
Expanding gives recurrence relation:
$$
\psi_{n+1}^\uparrow(x)=\cos\theta\,\psi_n^\uparrow(x-1)-\sin\theta\,\psi_n^\downarrow(x-1),
$$
$$
\psi_{n+1}^\downarrow(x)=\sin\theta\,\psi_n^\uparrow(x+1)+\cos\theta\,\psi_n^\downarrow(x+1).
$$

\subsection{C.2 Continuous Limit and Taylor Expansion}

Introduce lattice spacing $\varepsilon>0$, define continuous variables
$$
X=\varepsilon x,\quad T=\varepsilon n.
$$
Assume smooth functions $\psi^{\uparrow,\downarrow}(X,T)$ exist such that
$$
\psi_n^\uparrow(x)\approx\psi^\uparrow(X,T),\quad \psi_n^\downarrow(x)\approx\psi^\downarrow(X,T)
$$
at $X=\varepsilon x,T=\varepsilon n$. Let rotation angle scale with $\varepsilon$ as
$$
\theta=\varepsilon m,
$$
where $m>0$ is constant.

First order expansion for $\cos\theta$ and $\sin\theta$:
$$
\cos\theta=\cos(\varepsilon m)\approx 1-\frac{1}{2}\varepsilon^2 m^2,
$$
$$
\sin\theta=\sin(\varepsilon m)\approx \varepsilon m.
$$
First order expansion for spatial translation and time step:
$$
\psi_n^\uparrow(x\pm 1)\approx\psi^\uparrow(X\pm\varepsilon,T)\approx\psi^\uparrow(X,T)\pm\varepsilon\partial_X\psi^\uparrow(X,T),
$$
$$
\psi_n^\downarrow(x\pm 1)\approx\psi^\downarrow(X\pm\varepsilon,T)\approx\psi^\downarrow(X,T)\pm\varepsilon\partial_X\psi^\downarrow(X,T),
$$
$$
\psi_{n+1}^\uparrow(x)\approx\psi^\uparrow(X,T+\varepsilon)\approx\psi^\uparrow(X,T)+\varepsilon\partial_T\psi^\uparrow(X,T),
$$
$$
\psi_{n+1}^\downarrow(x)\approx\psi^\downarrow(X,T+\varepsilon)\approx\psi^\downarrow(X,T)+\varepsilon\partial_T\psi^\downarrow(X,T).
$$
Substitute into discrete evolution equation, keep first order terms in $\varepsilon$.

For spin-up component:
$$
\psi^\uparrow(X,T)+\varepsilon\partial_T\psi^\uparrow(X,T)
\approx
\left(1-\frac{1}{2}\varepsilon^2 m^2\right)\bigl[\psi^\uparrow(X-\varepsilon,T)\bigr]
-\varepsilon m\,\psi^\downarrow(X-\varepsilon,T).
$$
RHS expands to
$$
\left(1-\frac{1}{2}\varepsilon^2 m^2\right)\bigl[\psi^\uparrow(X,T)-\varepsilon\partial_X\psi^\uparrow(X,T)\bigr]
-\varepsilon m\bigl[\psi^\downarrow(X,T)-\varepsilon\partial_X\psi^\downarrow(X,T)\bigr].
$$
Ignoring $\varepsilon^2$ terms, get
$$
\psi^\uparrow(X,T)+\varepsilon\partial_T\psi^\uparrow(X,T)
\approx
\psi^\uparrow(X,T)-\varepsilon\partial_X\psi^\uparrow(X,T)-\varepsilon m\,\psi^\downarrow(X,T).
$$
Subtract $\psi^\uparrow(X,T)$ from both sides, rearrange to
$$
\partial_T\psi^\uparrow(X,T)=-\partial_X\psi^\uparrow(X,T)-m\,\psi^\downarrow(X,T).
$$
For spin-down component:
$$
\psi^\downarrow(X,T)+\varepsilon\partial_T\psi^\downarrow(X,T)
\approx
\varepsilon m\,\psi^\uparrow(X+\varepsilon,T)+\left(1-\frac{1}{2}\varepsilon^2 m^2\right)\psi^\downarrow(X+\varepsilon,T).
$$
RHS expands to
$$
\varepsilon m\bigl[\psi^\uparrow(X,T)+\varepsilon\partial_X\psi^\uparrow(X,T)\bigr]
+\bigl[\psi^\downarrow(X,T)+\varepsilon\partial_X\psi^\downarrow(X,T)\bigr],
$$
Ignoring $\varepsilon^2$ terms, get
$$
\psi^\downarrow(X,T)+\varepsilon\partial_T\psi^\downarrow(X,T)
\approx
\psi^\downarrow(X,T)+\varepsilon\partial_X\psi^\downarrow(X,T)+\varepsilon m\,\psi^\uparrow(X,T).
$$
Subtract $\psi^\downarrow(X,T)$ from both sides, rearrange to
$$
\partial_T\psi^\downarrow(X,T)=\partial_X\psi^\downarrow(X,T)+m\,\psi^\uparrow(X,T).
$$
Write in spinor form. Define
$$
\Psi(X,T):=\begin{pmatrix}\psi^\uparrow(X,T)\\\psi^\downarrow(X,T)\end{pmatrix},
$$
then
$$
\partial_T\Psi(X,T)=-\sigma_z\partial_X\Psi(X,T)-m\,\sigma_x\Psi(X,T).
$$
Multiply by $\mathrm i$ and rearrange
$$
\mathrm i\partial_T\Psi(X,T)=\bigl(-\mathrm i\sigma_z\partial_X+m\sigma_y\bigr)\Psi(X,T),
$$
where $-m\sigma_x$ and $m\sigma_y$ can be interchanged by basis redefinition. This is a standard form of 1D Dirac equation.

\subsection{C.3 Relation to Universe QCA Object}

In Universe QCA framework, 1D Dirac-type model corresponds to
$$
\mathfrak U_{\rm QCA}^{\rm Dirac}=(\mathbb Z,\mathcal H_{\rm cell},\mathcal A,\alpha,\omega_0),
$$
where $\alpha(A)=U^\dagger A U$. This object induces event set $E=\mathbb Z\times\mathbb Z$ and causal partial order $\preceq$, QCA propagation radius $R=1$ corresponds to "maximum propagation speed", and Dirac equation in continuous limit is the effective description of this universe at low energy and long wavelength scales.

This example shows: under the premise that the universe is defined as a QCA object, through appropriate scaling and coarse-graining, continuous relativistic field theory can be reproduced within the internal construction, providing concrete mathematical support for rationality of "Universe as Quantum Discrete Cellular Automaton".

\end{document}

