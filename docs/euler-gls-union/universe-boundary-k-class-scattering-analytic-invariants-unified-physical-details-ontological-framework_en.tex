\documentclass[12pt]{article}

% Essential packages
\usepackage[utf8]{inputenc}
\usepackage[T1]{fontenc}
\usepackage{amsmath,amssymb,amsthm}
\usepackage{mathrsfs}
\usepackage{geometry}
\usepackage{hyperref}
\usepackage{braket}
\usepackage{graphicx}
\usepackage{amsfonts} % For \mathfrak
\usepackage{tikz-cd} % For commutative diagrams

% Geometry settings
\geometry{a4paper, margin=1in}

% Hyperref settings
\hypersetup{
    colorlinks=true,
    linkcolor=blue,
    citecolor=blue,
    urlcolor=blue
}

% Theorem environments
\theoremstyle{plain}
\newtheorem{theorem}{Theorem}[section]
\newtheorem{lemma}[theorem]{Lemma}
\newtheorem{proposition}[theorem]{Proposition}
\newtheorem{corollary}[theorem]{Corollary}

\theoremstyle{definition}
\newtheorem{definition}[theorem]{Definition}
\newtheorem{example}[theorem]{Example}
\newtheorem{remark}[theorem]{Remark}
\newtheorem{hypothesis}[theorem]{Hypothesis}

% Title information
\title{Universe Boundary $K$-Class and Scattering Analytic Invariants: Ontological Framework for Unified Physical Details}
\author{Haobo Ma$^1$ \and Wenlin Zhang$^2$\\
\small $^1$Independent Researcher\\
\small $^2$National University of Singapore}

\date{\today}

\begin{document}

\maketitle

\begin{abstract}
In previous works, we have characterized "The Universe" as a maximal, consistent, and complete ontological object $\mathfrak U$ in a given foundational category, internally carrying multiple layers of components such as causal manifolds, quantum field theory, scattering and spectral shift, Tomita--Takesaki modular structure, generalized entropy and quantum null energy conditions, and observer and consensus geometries. However, such works mainly achieved "structural unification"---unifying time scales, boundary time geometry, and causal--entropy--observer axioms---without yet providing a unified encoding method for "specific physical details" (e.g., field content, gauge groups and representations, mass spectra and coupling constants, topological phases and band structures, task graphs and strategy spaces of multi-agent systems).

In this paper, within the framework of the universe ontological object $\mathfrak U$, unified time scale, and boundary time geometry, we introduce a new "detail data type"
$$
\mathcal D = \bigl([E],\{f_\alpha(\omega,\ell)\}_{\alpha\in A}\bigr),
$$
where $[E]$ is the $K$-theory class of the universe boundary channel bundle, unifying all discrete physical details (types of fields, Fermi/Bose statistics, gauge groups and representations, topological phases, etc.), and $\{f_\alpha(\omega,\ell)\}$ is a family of analytic invariants extracted from the scattering matrix, total connection $\Omega_\partial$, and its renormalization group connection $\Gamma_{\mathrm{res}}$, unifying all continuous physical details (mass spectra, coupling constants and $\beta$ functions, band structures, cosmological background evolution, costs and learning rates in multi-agent systems, etc.).

The main results of this paper include:

1. Under unified time scale and boundary time geometry axioms, the boundary scattering data of the universe ontological object $\mathfrak U$ naturally induce a unified detail data $\mathcal D_{\mathfrak U}=([E_{\mathfrak U}],\{f_\alpha^{\mathfrak U}\})$, where $[E_{\mathfrak U}]$ comes from the $K^1$ class of restricted unitary bundles, and $\{f_\alpha^{\mathfrak U}\}$ are extracted from scattering matrix $S(\omega;\ell)$, group delay matrix $Q(\omega;\ell)$, and curvature $F_{\mathrm{res}}$ of the resolution connection.

2. For any local quantum field theory satisfying causality, locality, and entropy controllability conditions, there exists a reconstruction theorem: given appropriate $\mathcal D$, a Haag--Kastler type local field theory can be reconstructed on a substructure of $\mathfrak U$, whose field content, symmetry group, mass spectrum, and interaction vertices are uniquely determined by $[E]$ and $\{f_\alpha\}$ (modulo field redefinitions and equivalence transformations).

3. For condensed matter and topological phase systems, under appropriate gap and locality conditions, $[E]$ and $\{f_\alpha\}$ establish a natural correspondence with band $K$-theory classification, topological invariants, and linear response functions, implying the existence of a family of lattice Hamiltonians $H$ whose spectral and response properties are completely determined by $\mathcal D$.

4. For multi-agent and macroscopic decision networks, agents can be viewed as observer nodes, task graphs and strategy updates as boundary scattering processes and modular flows. We prove that under the unified scale, $\mathcal D$ also encodes the topological--causal structure and cost--learning dynamics of the system, making multi-agent systems a special case of $\mathcal D$.

At the categorical level, we construct a "detail functor tower" starting from the universe object $\mathfrak U$, passing through unified detail data $\mathcal D$, to various categories of physical phenomena $\mathbf{Phys}^{(\mathrm{phen})}$, and prove that under natural conditions, all physical phenomenon theories can be represented as images of $\mathcal D_{\mathfrak U}$, with reduction and equivalence between theories corresponding to natural isomorphisms in this tower. Appendices provide outlines of proofs for boundary $K$-class construction, relation between scattering analytic invariants and unified time scale, local QFT reconstruction theorem, condensed matter system reconstruction theorem, and multi-agent system embedding theorem.
\end{abstract}

\section{Introduction}

\subsection{The "Physical Details Problem" in Unified Theories}

General Relativity, Quantum Field Theory, Quantum Information, and Condensed Matter Physics provide highly precise predictions and experimental agreement in their respective domains. However, when attempting to construct a "universe-level unified theory", traditional paths often focus on:

1. Unifying fundamental equations or actions (e.g., introducing larger gauge groups, extra dimensions, or string/brane structures);

2. Unifying symmetries and geometry (e.g., via gauge--gravity unification, general isomorphism theories, etc.);

3. Unifying information and causal structure (e.g., causal sets, holographic duality, generalized entropy and relative entropy variations, etc.).

In these works, structural issues like "time", "causality", "boundary", "observer", "entropy arrow" receive deep unification, but "specific physical details" are still presented in intrinsic ways of respective theories:

* Gauge groups $SU(3)\times SU(2)\times U(1)$ and their representations in the Standard Model;

* Particle mass spectra and mixing matrices;

* Lattice and band topology in condensed matter systems;

* Cosmological parameters $H_0,\Omega_\Lambda,\Omega_m,\dots$;

* Task graphs, cost functions, and learning rates in multi-agent systems.

In other words, existing unification schemes mostly achieve "structural unification" but have not yet achieved "detail unification": one still needs to manually specify the above details in different theoretical languages.

The goal of this paper is to address this gap by proposing an ontological framework for "detail unification", such that all the above physical details can be forced into the same type of mathematical object, and all can be viewed as different projections of the boundary scattering data of the universe ontological object $\mathfrak U$.

\subsection{Overview of Approach}

The basic idea of this paper can be summarized in three layers:

1. **Type Unification**: Define unified detail data $\mathcal D=([E],\{f_\alpha(\omega,\ell)\})$, where $[E]$ is the $K$-theory class of the boundary channel bundle, unifying all discrete details; $\{f_\alpha\}$ is a family of analytic functions induced by scattering and renormalization group connections, unifying all continuous details.

2. **Embedding Unification**: Prove that the universe ontological object $\mathfrak U$, under unified time scale and boundary time geometry axioms, naturally induces a specific $\mathcal D_{\mathfrak U}$, and any physical system satisfying causality and entropy conditions can be embedded into some boundary substructure of $\mathfrak U$, thereby inheriting a $\mathcal D$.

3. **Reconstruction Unification**: Provide a family of reconstruction theorems, showing that starting from $\mathcal D$, one can reconstruct specific physical theories like local QFT, condensed matter systems, and multi-agent networks within $\mathfrak U$, with all their details determined by $[E]$ and $\{f_\alpha\}$ (modulo natural equivalence and reparameterization).

Combining with the previously constructed unified time scale and causal--entropy--observer axiom system, these three layers elevate "unified universe theory" from the structural level to the "detail level", realizing "unified encoding and reconstruction of all physical details on the universe ontological object".

\subsection{Paper Structure}

Section 2 reviews the core elements of the universe ontological object $\mathfrak U$, unified time scale, and boundary time geometry, and introduces boundary $K$-theory and restricted unitary bundles. Section 3 defines unified detail data $\mathcal D$ and constructs $\mathcal D_{\mathfrak U}$ from boundary scattering data of $\mathfrak U$. Section 4 gives the reconstruction theorem for local quantum field theory. Section 5 discusses the reconstruction of condensed matter and topological phase systems. Section 6 discusses the embedding of multi-agent and macroscopic systems. Section 7 presents the categorified unified structure. Appendices provide proofs of key theorems and technical details.

\section{Preliminaries and Notation}

\subsection{Universe Ontological Object and Unified Time Scale}

We adopt the definition of the universe ontological object from previous work. The universe is characterized as a multi-component object
$$
\mathfrak U=
\bigl(
U_{\mathrm{evt}},U_{\mathrm{geo}},U_{\mathrm{meas}},
U_{\mathrm{QFT}},U_{\mathrm{scat}},
U_{\mathrm{mod}},U_{\mathrm{ent}},
U_{\mathrm{obs}},U_{\mathrm{cat}},U_{\mathrm{comp}}
\bigr),
$$
where $U_{\mathrm{evt}}=(M,g,\prec)$ is a globally hyperbolic Lorentzian manifold with causal partial order, $U_{\mathrm{scat}}$ contains the family of scattering matrices $S(\omega)$ and Wigner--Smith group delay $Q(\omega)$, $U_{\mathrm{mod}}$ is the Tomita--Takesaki modular flow induced by boundary observable algebra $\mathcal A_\partial$ and state $\omega_\partial$. $U_{\mathrm{ent}}$ contains generalized entropy $S_{\mathrm{gen}}$ on small causal diamonds, and $U_{\mathrm{obs}}$ is the family of observers and consensus geometries.

The unified time scale is given by the scale identity:
$$
\kappa(\omega)
=\frac{\varphi'(\omega)}{\pi}
=\rho_{\mathrm{rel}}(\omega)
=\frac{1}{2\pi}\operatorname{tr}Q(\omega),
$$
where $\varphi(\omega)=\arg\det S(\omega)$ is the scattering phase, $\rho_{\mathrm{rel}}(\omega)$ is the derivative of the spectral shift function (relative density of states), and $Q(\omega)=-\mathrm{i}S(\omega)^\dagger\partial_\omega S(\omega)$ is the group delay matrix.

The unified time scale axiom stipulates: all physical time readings in the universe (atomic clock proper time, gravitational time delay, redshift time, modular flow parameter, etc.) are affine transformations of the unified scale class $[\tau]$.

\subsection{Boundary Time Geometry and Total Connection}

Let $M_R\subset M$ be a bulk region with smooth boundary $\partial M_R$. Define induced metric $h_{ab}$, outward normal $n^a$, and second fundamental form $K_{ab}$ on the boundary. The Gibbons--Hawking--York boundary term is
$$
S_{\mathrm{GHY}}
=\frac{1}{8\pi G}
\int_{\partial M_R} K \sqrt{|h|}\,\mathrm{d}^{d-1}x.
$$
Boundary time geometry unifies gravitational connection, gauge connection, and resolution connection into a total connection
$$
\Omega_\partial
=\omega_{\mathrm{LC}}\oplus A_{\mathrm{YM}}\oplus \Gamma_{\mathrm{res}},
$$
where $\omega_{\mathrm{LC}}$ is the Levi--Civita connection, $A_{\mathrm{YM}}$ is the Yang--Mills connection on internal gauge group $G_{\mathrm{int}}$, and $\Gamma_{\mathrm{res}}$ is the connection on resolution space (renormalization group flow parameter).

Its curvature decomposes as
$$
F(\Omega_\partial)
=R\oplus F_{\mathrm{YM}}\oplus F_{\mathrm{res}},
$$
giving spacetime curvature, gauge field strength, and "curvature" of resolution flow respectively.

\subsection{Boundary Observable Algebra and Scattering Matrix}

The boundary observable algebra $\mathcal A_\partial$ is an appropriate $C^\ast$-algebra or von Neumann algebra, with state $\omega_\partial$ defining expectation values on it. Scattering matrix $S(\omega)$ and group delay $Q(\omega)$ can be viewed as families of unitary and self-adjoint operators on a frequency-decomposed channel space $\mathcal H_{\mathrm{chan}}(\omega)$.

We assume that at each frequency $\omega$, $\mathcal H_{\mathrm{chan}}(\omega)$ can be viewed as a finite-dimensional or countable-dimensional Hilbert space, with basis labeled by "boundary channels". Channel structures form a fiber bundle $E\to\partial M_R\times\Lambda$ over frequency and resolution parameter $\ell$, with structure group being some restricted unitary group.

\subsection{$K$-Theory and Restricted Unitary Bundle}

Briefly review $K$-theory. Given a compact space $X$, $K^0(X)$ consists of stable equivalence classes of complex vector bundles, and $K^1(X)$ can be viewed as homotopy classes of maps from $X$ to the restricted unitary group.

In scattering theory, the frequency-dependent unitary family $S(\omega)$, under appropriate conditions, can be viewed as a continuous map from the compactified frequency space $S^1$ to the restricted unitary group $U_{\mathrm{res}}$, thereby defining a $K^1$ class. This is closely related to work on "scattering $K$-theory", "relative scattering determinant", and "relative scattering determiner". ([SpringerLink][5])

In our framework, the $K$-class of the boundary channel bundle $E\to\partial M_R$ and the $K^1$ class of the scattering matrix will jointly constitute the discrete part $[E]$ of the unified detail data.

\section{Unified Detail Data Type on Universe Boundary}

\subsection{Discrete Details: $K$-Class of Channel Bundle}

Consider a boundary patch $\partial M_R$ of the universe ontological object $\mathfrak U$. Under frequency $\omega$ and resolution $\ell$, some representation of the boundary observable algebra $\mathcal A_\partial$ gives channel space $\mathcal H_{\mathrm{chan}}(\omega,\ell)$. As $(\omega,\ell)$ varies, channel spaces form a fiber bundle
$$
E \to \partial M_R\times\Lambda,
$$
where $\Lambda$ is the resolution parameter space. This bundle carries the following structures:

1. Structure group is restricted unitary group $U_{\mathrm{res}}$, defined by Schatten class and normality conditions of the scattering matrix;

2. $\mathbb Z_2$ grading in fibers encodes Fermi/Bose statistics and Null--Modular double cover structure;

3. Possible extra symmetries (lattice symmetry, time reversal, particle--hole, etc.) implemented via corresponding group actions on $E$.

Define
$$
[E] \in K(\partial M_R\times\Lambda)
$$
as the $K$-theory class of this channel bundle. It uniformly contains:

* Types and number of fields (via fiber rank and decomposition);

* Fermi/Bose parity and chirality (via $\mathbb Z_2$ grading and spin structure);

* Gauge groups and their representations (via structure group and associated bundles);

* Topological phases and protected boundary modes (via invariants of $K$ class).

Therefore, we call $[E]$ the unified label for "discrete physical details".

\subsection{Continuous Details: Analytic Invariants Extracted from Scattering and Total Connection}

Given $[E]$, remaining physical details include:

* Continuous parameters like gauge couplings, mass spectra, mixing angles;

* Band structures, gap sizes, response functions;

* Cosmological background $H(\tau),a(\tau)$;

* Cost functions, noise intensities, and learning rates in multi-agent systems.

We encode these uniformly as a family of analytic functions parameterized by frequency $\omega$ and resolution $\ell$
$$
\{f_\alpha(\omega,\ell)\}_{\alpha\in A}.
$$
Construction idea is as follows.

First, pole and branch point positions of scattering matrix $S(\omega;\ell)$ and group delay $Q(\omega;\ell)$ give bound states, resonances, and mass spectra. Second, the component $F_{\mathrm{res}}$ of total connection curvature in resolution direction, together with unified scale $\kappa(\omega)$, determines renormalization group flow of coupling constants $g_a(\ell)$
$$
\beta_a(\ell)
=\frac{\mathrm{d} g_a(\ell)}{\mathrm{d}\ln\ell}.
$$
Third, linear response functions (e.g., conductivity, Hall conductance, dielectric function) can be viewed as linear combinations of scattering amplitudes or Green's functions.

Thus, on a universe boundary patch, we can define an index set $A$, including:

1. Indices $\alpha_{\mathrm{mass},i}$ corresponding to mass spectra and resonance positions;

2. Indices $\alpha_{\mathrm{coupl},a}$ corresponding to coupling constants and $\beta$ functions;

3. Indices $\alpha_{\mathrm{resp},\mu\nu\cdots}$ corresponding to various linear and non-linear responses;

4. Indices $\alpha_{\mathrm{cosm}}$ corresponding to cosmological background and macroscopic fluid parameters;

5. Indices $\alpha_{\mathrm{agent}}$ corresponding to costs and learning rates in multi-agent systems.

For each $\alpha\in A$, we extract analytic function $f_\alpha(\omega,\ell)$ from $\bigl(S(\omega;\ell),Q(\omega;\ell),F(\Omega_\partial)\bigr)$ according to a unified prescription. For example:

* $f_{\mathrm{mass},i}(\omega,\ell)$ is a function of imaginary and real parts of corresponding poles, determining mass and decay width;

* $f_{\mathrm{coupl},a}(\omega,\ell)$ is effective value of coupling constant at given energy scale or resolution;

* $f_{\mathrm{resp},\mu\nu}(\omega,\ell)$ is frequency-resolution dependence of linear response tensor.

These functions satisfy analyticity and growth conditions imposed by unified time scale and causality (e.g., upper half plane analyticity, Paley--Wiener type boundedness).

\subsection{Definition of Unified Detail Data $\mathcal D$}

In summary, we give the definition of unified detail data.

\textbf{Definition 3.1 (Unified Detail Data)}

On a boundary patch $\partial M_R$ of universe ontological object $\mathfrak U$, unified detail data is a pair
$$
\mathcal D
=\bigl([E],\{f_\alpha(\omega,\ell)\}_{\alpha\in A}\bigr),
$$
where:

1. $[E]\in K(\partial M_R\times\Lambda)$ is the $K$-theory class of channel bundle $E$, uniformly encoding all discrete physical details;

2. $\{f_\alpha(\omega,\ell)\}$ is a family of analytic functions extracted from scattering matrix $S(\omega;\ell)$, group delay $Q(\omega;\ell)$, and total connection curvature $F(\Omega_\partial)$, uniformly encoding all continuous physical details;

3. These data satisfy analyticity, growth, and regularity conditions imposed by unified time scale and causality.

We denote the set of all $\mathcal D$ satisfying these conditions as $\mathsf{Detail}(\partial M_R)$.

\subsection{Universe Detail Data $\mathcal D_{\mathfrak U}$}

Scattering data of universe ontological object $\mathfrak U$ on given boundary patch $\partial M_R$ and resolution space $\Lambda$ naturally induce a specific unified detail data
$$
\mathcal D_{\mathfrak U}
=\bigl([E_{\mathfrak U}],\{f_\alpha^{\mathfrak U}(\omega,\ell)\}_{\alpha\in A_{\mathfrak U}}\bigr).
$$

\textbf{Proposition 3.2 (Existence of Universe Detail Data)}

Under unified time scale and boundary time geometry axioms, for any boundary patch $\partial M_R$, there exists a uniquely determined channel bundle $E_{\mathfrak U}\to\partial M_R\times\Lambda$ (modulo stable equivalence), and a family of analytic functions $\{f_\alpha^{\mathfrak U}\}$ constructed from scattering and connection data in $\mathfrak U$, such that
$$
\mathcal D_{\mathfrak U}
\in
\mathsf{Detail}(\partial M_R).
$$
Proof is given in Appendix A, key lies in: continuous deformation of channel space with frequency and resolution gives a restricted unitary bundle, while regularity of scattering matrix and total connection ensures defined $K$ class and analytic functions satisfy unified scale and causal constraints.

\section{Detail Reconstruction Theorem for Local Quantum Field Theory}

This section shows that under appropriate conditions, unified detail data $\mathcal D$ suffices to reconstruct a local quantum field theory, and field content, groups, mass spectra, and coupling constants are determined by $[E]$ and $\{f_\alpha\}$.

\subsection{Axiomatic Framework for Local Quantum Field Theory}

We adopt Haag--Kastler/AQFT style definition of local quantum field theory.

On bulk region $M_R$, a local quantum field theory is characterized by data:

1. Hilbert space $\mathcal H$ and vacuum state $\Omega$;

2. For each bounded region $O\subset M_R$, a von Neumann algebra $\mathcal A(O)\subset\mathcal B(\mathcal H)$, satisfying isotropy, locality, and covariance;

3. One-parameter unitary group $U(a)$ implementing spacetime translation, corresponding to stress--energy tensor $T_{ab}$;

4. $n$-point functions and scattering matrix satisfying spectral condition and cluster property.

We call a set of data $\mathcal Q=(\mathcal H,\Omega,\{\mathcal A(O)\},U)$ a local quantum field theory.

\subsection{Eligible Detail Data and QFT Realizability}

Not all $\mathcal D\in\mathsf{Detail}(\partial M_R)$ can be realized by local QFT. We define "QFT eligibility".

\textbf{Definition 4.1 (QFT Eligible Detail Data)}

Unified detail data $\mathcal D=([E],\{f_\alpha\})$ is called QFT eligible if it satisfies:

1. Causal Analyticity: Corresponding scattering amplitudes satisfy macroscopic causality and boundary value conditions;

2. Local Realizability: Field operators can be defined on local regions of $M_R$ such that $n$-point functions constructed from these fields are compatible with $\{f_\alpha\}$;

3. Energy Bounds and Spectral Condition: Energy spectrum has lower bound under unified time scale, and relative entropy satisfies standard stability conditions;

4. Consistent $K$-Class Constraints: $[E]$ is compatible with $K^1$ class of scattering, allowing construction of corresponding field content and chiral structure.

Denote the set of all QFT eligible detail data as $\mathsf{Detail}_{\mathrm{QFT}}(\partial M_R)\subset \mathsf{Detail}(\partial M_R)$.

\subsection{QFT Reconstruction Theorem}

\textbf{Theorem 4.2 (Local Quantum Field Theory Reconstruction Theorem)}

Let $\mathcal D=([E],\{f_\alpha(\omega,\ell)\})\in\mathsf{Detail}_{\mathrm{QFT}}(\partial M_R)$. Then there exists a local quantum field theory $\mathcal Q=(\mathcal H,\Omega,\{\mathcal A(O)\},U)$, embedded in some substructure of universe ontological object $\mathfrak U$, such that:

1. Field Content and Gauge Group: Field types, Fermi/Bose statistics, gauge groups and representations in $\mathcal Q$ are uniquely determined by $[E]$ (modulo field redefinition and equivalent vector bundle isomorphism);

2. Mass Spectrum and Coupling Constants: Mass spectrum and coupling constant family $g_a(\ell)$ of $\mathcal Q$ are uniquely recovered from corresponding analytic functions in $\{f_\alpha\}$;

3. Scattering Matrix and Response Functions: Restriction of scattering matrix and linear response functions of $\mathcal Q$ on boundary is consistent with scattering and response data of $\mathcal D$;

4. Universality: If there exists another local field theory $\mathcal Q'$ having same detail data $\mathcal D$, then $\mathcal Q$ and $\mathcal Q'$ are locally equivalent on $M_R$.

Proof is given in Appendix B, overall route is:

1. Use $[E]$ to construct appropriate field bundles and spin bundles, determining field content and symmetry groups;

2. Reconstruct $n$-point functions and effective action from $\{f_\alpha\}$ via inverse scattering and light-cone boundary value methods;

3. Construct Hilbert space and field algebra using Wightman function reconstruction theorem;

4. Verify locality, spectral condition, and cluster property, and prove local equivalence.

This theorem shows that under constraint of unified detail data, "physical details" of local quantum field theory are no longer independent inputs, but functions of $[E]$ and $\{f_\alpha\}$.

\section{Detail Reconstruction for Condensed Matter and Topological Phase Systems}

This section shows how unified detail data uniformly encodes topological phases and response characteristics of condensed matter systems.

\subsection{Boundary $K$-Class and Topological Phase Classification}

In condensed matter systems, gapped topological phases can be characterized by $K$-theory class of energy bands. Considering a $d$-dimensional lattice system with Brillouin zone $T^d$, bands form vector bundle $E_{\mathrm{band}}\to T^d$, different topological phases correspond to different $K$ classes $[E_{\mathrm{band}}]\in K(T^d)$.

In universe boundary framework, we view lattice and Brillouin zone as additional structures of some boundary patch $\partial M_R$, channel bundle $E$ restricted to this patch contains band topology information in its $K$ class.

\textbf{Proposition 5.1}

Under appropriate gap and locality conditions, topological phase classification of condensed matter systems is uniquely determined by $[E]$ in $\mathcal D$.

Proof relies on restricting $E$ to lattice direction and Brillouin zone, utilizing bulk--boundary correspondence and $K$-theory isomorphism to embed band $K$ class into $[E]$.

\subsection{Linear Response and Scattering Analytic Invariants}

Linear response of condensed matter systems (e.g., conductivity, Hall conductance, magnetic susceptibility) can be viewed as combination of some boundary scattering amplitudes and Green's functions. Under unified scale, their frequency and resolution dependence are naturally contained in $f_\alpha(\omega,\ell)$.

\textbf{Proposition 5.2}

Under unified time scale and causality constraints, linear response function family of condensed matter systems can be uniquely recovered from corresponding components in $\{f_\alpha\}$, and satisfy Kramers--Kronig relations and Poisson growth conditions.

\subsection{Condensed Matter Reconstruction Theorem}

\textbf{Theorem 5.3 (Condensed Matter System Reconstruction)}

Let $\mathcal D\in\mathsf{Detail}(\partial M_R)$ satisfy:

1. Existence of lattice structure and Brillouin zone such that restriction of $[E]$ gives band $K$ class;

2. Some part of $\{f_\alpha\}$ satisfies causal analyticity and high-frequency decay conditions of linear response;

Then there exists a family of lattice Hamiltonians $H$ (modulo local unitary transformations), defined on some Hilbert space $\mathcal H_{\mathrm{lat}}$, such that:

1. Band topology of $H$ is consistent with restriction of $[E]$;

2. Linear response functions of $H$ are consistent with $\{f_\alpha\}$;

3. If there exists another Hamiltonian $H'$ having same $\mathcal D$, then $H$ and $H'$ are equivalent in condensed matter equivalence sense.

Proof outline in Appendix C, based on combination of $K$-theory classification, bulk--boundary correspondence, and Green's function inverse problem.

\section{Embedding of Multi-Agent and Macroscopic Systems}

This section shows how to embed multi-agent systems and macroscopic decision networks into unified detail data framework.

\subsection{Agents as Observer Nodes}

Consider a set of agents $\{A_i\}$ with task graphs, resource constraints, and strategy update rules. Previous observer axioms allow treating each agent as an observer node
$$
O_i
=\bigl(
C_i,\prec_i,\Lambda_i,
\mathcal A_i,\omega_i,
\mathcal M_i,U_i,u_i,\{\mathcal C_{ij}\}
\bigr),
$$
where $\mathcal M_i$ is strategy or world model space, $U_i$ is update rule, $\mathcal C_{ij}$ is communication structure.

Under unified time scale, "logical time" used by all agents to update strategies and beliefs belongs to scale class $[\tau]$.

\subsection{Task Graphs and Cost Functions as Scattering and Entropy Data}

Task graphs and resource constraints can be viewed as special case of causal net, nodes being tasks or states, edges being dependencies and resource flows, cost or utility functions playing role of generalized entropy or free energy. Strategy update rules can be analogized to modular flow or quantum channels.

We can construct a class of "information scattering matrices", whose channels correspond to agent--task combinations, scattering amplitudes encode strategy update and resource allocation probabilities.

\textbf{Proposition 6.1}

Under unified scale and observer axioms, task graph topology and strategy update rules of multi-agent systems can be encoded as some channel bundle $E$ and scattering family $S(\omega;\ell)$, thus obtaining a special case of $[E]$ and $\{f_\alpha\}$.

\subsection{Multi-Agent System Embedding Theorem}

\textbf{Theorem 6.2 (Multi-Agent Embedding)}

Any multi-agent system satisfying following conditions:

1. Decision and communication follow causality and finite propagation speed constraints;

2. Strategy update rules can be viewed as modular flow along unified time scale, compatible with monotonicity of some entropy or cost function;

can be embedded into boundary scattering structure of universe ontological object $\mathfrak U$, thereby obtaining unified detail data $\mathcal D$. Conversely, any $\mathcal D$ with corresponding structure can be viewed as abstract encoding of some multi-agent system.

Proof in Appendix D.

\section{Categorified Unified Structure}

We finally summarize unified detail framework from categorical perspective.

\subsection{Universe Category and Detail Functor}

Define universe category $\mathbf{Univ}_\mathcal U$, objects being candidate universe structures satisfying unified time scale and boundary time geometry axioms, morphisms being maps preserving these structures. Universe ontological object $\mathfrak U$ is maximal consistent terminal object in this category.

Define detail category $\mathbf{Detail}$, objects being unified detail data $\mathcal D=([E],\{f_\alpha\})$, morphisms being maps preserving $K$ class and analytic structure.

From construction in Section 3, we get a functor from universe category to detail category
$$
\mathcal F_{\mathrm{det}}\colon
\mathbf{Univ}_\mathcal U \to \mathbf{Detail},
$$
such that $\mathcal F_{\mathrm{det}}(\mathfrak U)=\mathcal D_{\mathfrak U}$.

\subsection{Phenomenon Categories and Reconstruction Functors}

For each class of physical phenomena (local QFT, condensed matter, multi-agent systems, etc.), we introduce phenomenon category $\mathbf{Phys}^{(\mathrm{phen})}$, objects being specific theories (e.g., a field theory or Hamiltonian), morphisms being maps preserving physical structures.

Reconstruction theorems in Sections 4--6 yield functors from detail category to phenomenon categories
$$
\mathcal R_{\mathrm{QFT}}\colon
\mathbf{Detail}_{\mathrm{QFT}}\to\mathbf{Phys}^{(\mathrm{QFT})},
$$
$$
\mathcal R_{\mathrm{CM}}\colon
\mathbf{Detail}_{\mathrm{CM}}\to\mathbf{Phys}^{(\mathrm{CM})},
$$
$$
\mathcal R_{\mathrm{MA}}\colon
\mathbf{Detail}_{\mathrm{MA}}\to\mathbf{Phys}^{(\mathrm{MA})},
$$
where $\mathbf{Detail}_{\mathrm{QFT}},\mathbf{Detail}_{\mathrm{CM}},\mathbf{Detail}_{\mathrm{MA}}$ are eligible detail subcategories for corresponding phenomena.

Combining these gives "detail functor tower"
$$
\mathbf{Univ}_\mathcal U
\xrightarrow{\ \mathcal F_{\mathrm{det}}\ }
\mathbf{Detail}
\xrightarrow{\ \mathcal R_{\mathrm{phen}}\ }
\mathbf{Phys}^{(\mathrm{phen})},
$$
where $\mathcal R_{\mathrm{phen}}$ is any of above reconstruction functors.

\subsection{Phenomenon Unified Representation Theorem}

\textbf{Theorem 7.1 (Phenomenon Unified Representation)}

Let $\mathbf{Phys}^{(\mathrm{phen})}$ be some physical phenomenon category, whose objects satisfy causality, locality, and entropy controllability conditions. Then:

1. There exists a faithful functor $\mathcal G\colon\mathbf{Phys}^{(\mathrm{phen})}\to\mathbf{Detail}$ on this category, mapping each phenomenon theory to its unified detail data;

2. There exists a reconstruction functor $\mathcal R_{\mathrm{phen}}$ such that $\mathcal R_{\mathrm{phen}}\circ \mathcal G$ is naturally isomorphic to identity functor;

3. For any universe object $\mathfrak U$, image of $\mathcal R_{\mathrm{phen}}\circ \mathcal F_{\mathrm{det}}(\mathfrak U)$ contains all phenomenon theories realizable by $\mathfrak U$.

This theorem formalizes central claim of this paper: under unified time scale and boundary time geometry axioms, all physical phenomenon theories and their details can be viewed as images of unified detail data $\mathcal D_{\mathfrak U}$ of universe ontological object $\mathfrak U$.

\section{Summary and Outlook}

This paper introduces unified detail data $\mathcal D=([E],\{f_\alpha(\omega,\ell)\})$ on top of universe ontological object, unified time scale, and boundary time geometry framework, and proves:

1. $[E]$ as $K$ class of boundary channel bundle uniformly encodes all discrete physical details;

2. Analytic function family $\{f_\alpha\}$ extracted from scattering matrix and total connection curvature uniformly encodes all continuous physical details;

3. Under appropriate assumptions, any local quantum field theory, condensed matter topological phase, and multi-agent system can be reconstructed from some $\mathcal D$;

4. Categorical structure shows all phenomenon theories are images of universe detail data $\mathcal D_{\mathfrak U}$.

Thus, "unified theory" is no longer just unified equations or geometric structures, but unified to "detail level": any physical detail must be a function of universe boundary $K$ class and scattering analytic invariants.

\begin{thebibliography}{99}

\bibitem{Haag1992} R. Haag, Local Quantum Physics: Fields, Particles, Algebras, Springer.

\bibitem{Weinberg1995} S. Weinberg, The Quantum Theory of Fields, Vols. I--III, Cambridge University Press.

\bibitem{Wald1984} R. Wald, General Relativity, University of Chicago Press.

\bibitem{ReedSimon} M. Reed and B. Simon, Methods of Modern Mathematical Physics, Academic Press.

\bibitem{Atiyah1967} M. Atiyah, K-Theory, Benjamin.

\bibitem{SimonTrace} B. Simon, Trace Ideals and Their Applications, Cambridge University Press.

\bibitem{ArakiQFT} H. Araki, Mathematical Theory of Quantum Fields, Oxford University Press.

\bibitem{NagaosaQFT} N. Nagaosa, Quantum Field Theory in Condensed Matter Physics, Springer.

\bibitem{vonNeumannQM} J. von Neumann, Mathematical Foundations of Quantum Mechanics, Princeton University Press.

\bibitem{Others} Other relevant literature omitted.

\end{thebibliography}

\appendix

\section{Appendix A: Construction of Universe Detail Data $\mathcal D_{\mathfrak U}$ and $K$-Class Properties}

\subsection{A.1 Construction of Channel Bundle}

In universe ontological object $\mathfrak U$, boundary observable algebra $\mathcal A_\partial$ has a representation $\pi_{\omega,\ell}$ at each frequency and resolution, whose GNS construction gives Hilbert space $\mathcal H_{\mathrm{chan}}(\omega,\ell)$.

As $(\omega,\ell)$ varies, these Hilbert spaces glue into a fiber bundle
$$
E_{\mathfrak U}
\to
\partial M_R\times\Lambda,
$$
with fiber $\mathcal H_{\mathrm{chan}}(\omega,\ell)$. Since scattering matrix $S(\omega;\ell)$ varies smoothly over $(\omega,\ell)$, transition functions fall within restricted unitary group $U_{\mathrm{res}}$.

Using standard results, one can prove $E_{\mathfrak U}$ is a restricted unitary bundle, and its stable equivalence class gives a $K$ class $[E_{\mathfrak U}]\in K(\partial M_R\times\Lambda)$.

\subsection{A.2 Compatibility of Scattering $K^1$ Class and Channel Bundle}

Scattering matrix $S(\omega;\ell)$ gives map from compactified frequency space $\tilde{\mathbb R}\cong S^1$ to $U_{\mathrm{res}}$ in frequency direction, defining a class $[S]$ in $K^1(\partial M_R\times\Lambda)$.

In restricted unitary bundle framework, $[S]$ and $[E_{\mathfrak U}]$ must satisfy natural compatibility conditions, i.e., index of some boundary state operator in $K$-theory paired with $K^1$ class of scattering determinant gives integer topological number.

This can be proven via relative scattering determiner and Fredholm index theory. This yields existence and uniqueness of $[E_{\mathfrak U}]$.

\subsection{A.3 Construction of Analytic Invariants}

At each $(\omega,\ell)$, pole and branch structure of scattering matrix $S(\omega;\ell)$ is determined by spectrum of bulk operators. Using Birman--Kreĭn formula and spectral shift function, define relative density of states and group delay trace.

Also, component of total connection curvature $F(\Omega_\partial)$ in resolution direction gives renormalization group flow. By projecting these data onto appropriate basis, one can define analytic functions $f_\alpha(\omega,\ell)$, satisfying analyticity and growth conditions imposed by unified time scale and causality.

\section{Appendix B: Outline of Local QFT Reconstruction Theorem Proof}

\subsection{B.1 Constructing Field Bundles and Symmetry Groups from $[E]$}

Given $[E]\in K(\partial M_R\times\Lambda)$, choose representative vector bundle $E$. By restricting to section of some fixed resolution level $\ell_0$, obtain a family of vector bundles $E_{\ell_0}\to\partial M_R$.

Choose suitable spin structure and Clifford module to construct spin bundle and associated bundles, determining:

1. Fermi/Bose statistics and chiral structure;

2. Gauge group and its representation on fields;

3. Possible topological constraints and boundary conditions.

These data determine "field content" and symmetry group of local quantum field theory.

\subsection{B.2 Reconstructing $n$-Point Functions and Scattering Matrix from $\{f_\alpha\}$}

Given unified scale $[\tau]$ and $f_\alpha(\omega,\ell)$, using multi-dimensional spectral representation and analytic continuation, one can reconstruct frequency domain representation of Wightman functions. Inverse Fourier transform gives time domain Wightman functions, then use Osterwalder--Schrader reconstruction theorem to construct Hilbert space and field operators.

Scattering matrix is given by overlap of in and out states and LSZ limit process, its frequency structure matches $f_\alpha$.

\subsection{B.3 Verifying Locality and Spectral Condition}

Using unified time scale and causality axioms, prove reconstructed $n$-point functions are supported in light cones and their boundaries, satisfying locality. Spectral condition comes from lower bound of energy under unified scale. Cluster property is guaranteed by high frequency decay and relative entropy monotonicity.

Combining these completes proof of Theorem 4.2.

\section{Appendix C: Outline of Condensed Matter System Reconstruction Theorem Proof}

\subsection{C.1 Band $K$-Theory and Boundary $K$-Class}

Under existence of lattice symmetry and energy gap, there is natural map between band vector bundle $E_{\mathrm{band}}$ on Brillouin zone $T^d$ and boundary channel bundle $E$, given by bulk--boundary correspondence and index theorem.

Thus, restriction $[E]|_{T^d}$ is isomorphic to $[E_{\mathrm{band}}]$, so topological phase classification is determined by $[E]$.

\subsection{C.2 Green's Functions and Response Functions}

Linear response functions can be given by frequency domain representation of Green's functions. Unified scale and causality ensure Green's functions satisfy upper half plane analyticity and Kramers--Kronig relations.

Given components in $\{f_\alpha\}$ corresponding to response, use inverse problem techniques to construct a lattice Hamiltonian $H$ whose Green's functions match response functions.

\subsection{C.3 Equivalence Classes and Stability}

If two Hamiltonians have same $[E]$ and $\{f_\alpha\}$, they can be connected by adding trivial bundles, local unitary transformations, and perturbative continuous deformations, thus equivalent in topological phase sense.

\section{Appendix D: Outline of Multi-Agent System Embedding Proof}

\subsection{D.1 From Task Graph to Causal Net}

Task graph of multi-agent system can be represented as weighted directed graph, nodes being tasks or states, edges being causal dependencies and resource flows. Embedding this graph into universe causal manifold $(M,\prec)$ yields a local causal net.

\subsection{D.2 Strategy Update and Modular Flow}

Strategy update rules can be viewed as transformations on probability measures or quantum states. Under unified scale, this family of transformations forms a modular flow or Markov semigroup. Monotonicity of corresponding entropy or cost function is compatible with relative entropy monotonicity.

\subsection{D.3 Channel Bundle and Scattering Matrix}

Viewing each agent--task pair as a channel, strategy update and task completion as scattering processes, construct channel bundle $E$ and scattering matrix $S(\omega;\ell)$.

Under unified scale, strategy update frequency and resolution correspond to $\omega,\ell$. This yields $[E]$ and $\{f_\alpha\}$, where $f_\alpha$ embodies cost function, success probability, and learning rate.

\subsection{D.4 Bidirectional Correspondence}

Given such $\mathcal D$, one can inversely construct an abstract multi-agent system whose task graph and strategy update rules are reinterpreted from $[E]$ and $\{f_\alpha\}$.

This completes outline of Theorem 6.2.

\end{document}

