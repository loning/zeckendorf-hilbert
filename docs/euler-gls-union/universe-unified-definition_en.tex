\documentclass[12pt]{article}

% Essential packages
\usepackage[utf8]{inputenc}
\usepackage[T1]{fontenc}
\usepackage{amsmath,amssymb,amsthm}
\usepackage{mathrsfs}
\usepackage{geometry}
\usepackage{hyperref}
\usepackage{braket}
\usepackage{graphicx}
\usepackage{amsfonts} % For \mathfrak

% Geometry settings
\geometry{a4paper, margin=1in}

% Hyperref settings
\hypersetup{
    colorlinks=true,
    linkcolor=blue,
    citecolor=blue,
    urlcolor=blue
}

% Theorem environments
\theoremstyle{plain}
\newtheorem{theorem}{Theorem}[section]
\newtheorem{lemma}[theorem]{Lemma}
\newtheorem{proposition}[theorem]{Proposition}
\newtheorem{corollary}[theorem]{Corollary}

\theoremstyle{definition}
\newtheorem{definition}[theorem]{Definition}
\newtheorem{example}[theorem]{Example}
\newtheorem{remark}[theorem]{Remark}
\newtheorem{hypothesis}[theorem]{Hypothesis}

% Title information
\title{Unified Mathematical Definition of the Universe}
\author{Haobo Ma$^1$ \and Wenlin Zhang$^2$\\
\small $^1$Independent Researcher\\
\small $^2$National University of Singapore}

\date{\today}

\begin{document}

\maketitle

\section*{Core Definition}

\begin{definition}[Universe]
The **Universe** is a single mathematical structure that is simultaneously maximal, consistent, and complete within a multi-layered category, denoted as
$$
\mathfrak U
=
\Big(
U_{\rm evt},\ U_{\rm geo},\ U_{\rm meas},\ U_{\rm QFT},\ U_{\rm scat},\ U_{\rm mod},\ U_{\rm ent},\ U_{\rm obs},\ U_{\rm cat},\ U_{\rm comp}
\Big)
$$
where each component and the compatibility between them are described below; it is unique up to isomorphism.
\end{definition}

\section{Event and Causal Layer}

\subsection{Event Set and Causal Partial Order}

$$
U_{\rm evt}=(X,\ \preceq,\ \mathcal C)
$$
where
\begin{itemize}
    \item $X$ is a class-set (may be a proper class), with elements called "events";
    \item $\preceq\subseteq X\times X$ is a partial order, satisfying reflexivity, antisymmetry, and transitivity;
    \item $\mathcal C\subseteq\mathcal P(X)$ is a family of "causal patches", such that for each $C\in\mathcal C$, $(C,\preceq|_C)$ is a locally finite partial order, forming a cover $\bigcup_{C\in\mathcal C}C=X$.
\end{itemize}

\subsection{Global Causal Consistency}

$(X,\preceq)$ is **stably causal**: there are no closed causal chains, and there exists a strictly increasing time function
$$
T_{\rm cau}\colon X\to\mathbb R,\quad
x\prec y\Rightarrow T_{\rm cau}(x)<T_{\rm cau}(y).
$$

\subsection{Causal Net and Causal Diamond Family}

Define the set of all bounded causal regions
$$
\mathcal D=\{D\subseteq X:\ D=J^+(p)\cap J^-(q),\ p\preceq q\},
$$
where $J^\pm(\cdot)$ are determined by $\preceq$; each $D\in\mathcal D$ is called a "small causal diamond".

\section{Geometric Layer (Spacetime and Metric)}

\subsection{Lorentzian Manifold Structure}

$$
U_{\rm geo}=(M,\ g,\ \Phi_{\rm evt},\ \Phi_{\rm cau})
$$
where
\begin{itemize}
    \item $M$ is a four-dimensional orientable, time-oriented $C^\infty$ manifold;
    \item $g$ is a Lorentzian metric with signature $(-+++)$;
    \item $\Phi_{\rm evt}:X\to M$ is an event embedding;
    \item $\Phi_{\rm cau}$ pulls the causal partial order back to the light-cone causal structure:
    $$
    x\preceq y\iff \Phi_{\rm evt}(y)\in J^+_g(\Phi_{\rm evt}(x)).
    $$
\end{itemize}

\subsection{Global Hyperbolicity}

$(M,g)$ is globally hyperbolic: there exists a Cauchy hypersurface $\Sigma\subset M$ such that
$$
M\simeq\mathbb R\times\Sigma,\quad
\text{every timelike/null geodesic intersects }\Sigma \text{ exactly once}.
$$

\subsection{Geometric Time Function}

$$
T_{\rm geo}:M\to\mathbb R
$$
is a smooth time function whose gradient is everywhere timelike, encoding the causal structure as
$$
p\in J^+_g(q)\Rightarrow T_{\rm geo}(p)\ge T_{\rm geo}(q).
$$

\section{Measure, Probability, and Statistical Layer}

\subsection{Measure Structure}

$$
U_{\rm meas}=(\Omega,\ \mathcal F,\ \mathbb P,\ \Psi)
$$
where
\begin{itemize}
    \item $(\Omega,\mathcal F,\mathbb P)$ is a complete probability space;
    \item $\Psi:\Omega\to X$ is a random event map, such that observational statistics arise from the push-forward measure of $\Psi$.
\end{itemize}

\subsection{Statistical Time Series}

Define sample paths on worldlines
$$
\Psi_\gamma:\Omega\to X^{\mathbb Z},\quad
\Psi_\gamma(\omega)=(x_n)_{n\in\mathbb Z},
$$
satisfying $x_n\prec x_{n+1}$; inducing a time series process.

\section{Quantum Field and Operator Algebra Layer}

\subsection{Local Observable Algebra Net}

$$
U_{\rm QFT}=(\mathcal O(M),\ \mathcal A,\ \omega)
$$
where
\begin{itemize}
    \item $\mathcal O(M)$ is the family of bounded causally convex open sets on $M$;
    \item $\mathcal A:\mathcal O(M)\to C^\ast\text{Alg},\ O\mapsto\mathcal A(O)$ is a Haag--Kastler type net, satisfying isotony, covariance, and microcausality:
    $$
    O_1\subseteq O_2\Rightarrow \mathcal A(O_1)\subseteq\mathcal A(O_2),\quad
    O_1\perp O_2\Rightarrow [\mathcal A(O_1),\mathcal A(O_2)]=0.
    $$
    \item $\omega$ is a positive, normalized state consistent across all $\mathcal A(O)$.
\end{itemize}

\subsection{GNS Construction}

$$
(\pi_\omega,\ \mathcal H,\ \Omega_\omega)
$$
where
\begin{itemize}
    \item $\pi_\omega:\mathcal A\to B(\mathcal H)$ is a $*$-representation;
    \item $\Omega_\omega$ is a cyclic and separating vector;
    \item $\omega(A)=\langle\Omega_\omega,\pi_\omega(A)\Omega_\omega\rangle$.
\end{itemize}

\section{Scattering, Spectrum, and Time Scale Layer}

\subsection{Scattering Pair and Spectral Shift}

$$
(H,H_0),\quad V:=H-H_0
$$
are a pair of self-adjoint operators satisfying appropriate trace-class/relative trace-class assumptions, ensuring the existence of a spectral shift function $\xi(\omega)$.

\subsection{Scattering Matrix and Wigner--Smith Delay}

$$
S(\omega)\in U(\mathcal H_\omega),\quad
Q(\omega)=-{\rm i}S(\omega)^\dagger\partial_\omega S(\omega).
$$

\subsection{Total Scattering Phase and Relative Density of States}

$$
\Phi(\omega)=\arg\det S(\omega),\quad
\varphi(\omega)=\tfrac12\Phi(\omega),\quad
\rho_{\rm rel}(\omega)=-\xi'(\omega).
$$

\subsection{Unified Time Scale (Mother Ruler)}

Define scale density
$$
\kappa(\omega)
:=\frac{\varphi'(\omega)}{\pi}
=\rho_{\rm rel}(\omega)
=\frac{1}{2\pi}{\rm tr}\,Q(\omega).
$$
For a reference frequency $\omega_0$, define scattering time
$$
\tau_{\rm scatt}(\omega)-\tau_{\rm scatt}(\omega_0)
:=\int_{\omega_0}^{\omega}\kappa(\tilde\omega)\,{\rm d}\tilde\omega.
$$

\subsection{Geometric Time Alignment}

Require the existence of a monotonic bijection $f$ such that
$$
T_{\rm geo}\big(\Phi_{\rm evt}(x)\big)
=f\!\big(\tau_{\rm scatt}(\omega_x)\big)
$$
holds for appropriately defined frequency markers $\omega_x$; i.e., geometric time and scattering time fall into the same scale equivalence class.

\section{Modular Flow and Thermal Time Layer}

\subsection{Modular Operator and Modular Flow}

$$
S_0\pi_\omega(A)\Omega_\omega=\pi_\omega(A)^\ast\Omega_\omega,
$$
closure $S$ has polar decomposition $S=J\Delta^{1/2}$, defining modular flow
$$
\sigma_t^\omega(A)=\Delta^{\mathrm i t}A\Delta^{-\mathrm i t}.
$$

\subsection{Modular Time Scale}

Define modular Hamiltonian
$$
K_\omega:=-\log\Delta,\quad
\sigma_t^\omega(A)=\mathrm e^{\mathrm i tK_\omega}A\mathrm e^{-\mathrm i tK_\omega}.
$$
The modular parameter $t_{\rm mod}\in\mathbb R$ serves as a time parameter; for different faithful states $\omega,\omega'$, their modular flows are conjugate in the outer automorphism group, related by at most an affine rescaling of time.

\subsection{Alignment with Scattering Scale}

Require the existence of constants $a>0,b\in\mathbb R$ such that for some unified boundary algebra $\mathcal A_\partial\subseteq\mathcal A$,
$$
t_{\rm mod}=a\,\tau_{\rm scatt}+b
$$
holds on the common domain.

\section{Generalized Entropy, Energy, and Gravity Layer}

\subsection{Generalized Entropy on Small Causal Diamonds}

For each $D\in\mathcal D$ and its boundary cross-section $\Sigma\subset\partial D$, define
$$
S_{\rm gen}(\Sigma)=\frac{A(\Sigma)}{4G\hbar}+S_{\rm out}(\Sigma),
$$
where $A(\Sigma)$ is the area, and $S_{\rm out}$ is the exterior von Neumann entropy.

\subsection{Generalized Entropy Extremality and Time Arrow}

Deformations along the null generator affine parameter $\lambda$ satisfy
$$
\frac{{\rm d}}{{\rm d}\lambda}S_{\rm gen}(\lambda)\Big|_{\lambda=0}=0,\quad
\frac{{\rm d}^2}{{\rm d}\lambda^2}S_{\rm gen}(\lambda)\ge 0,
$$
uniformly for the family of all small causal diamonds; combined with the Quantum Null Energy Condition (QNEC) $T_{kk}\ge\frac{\hbar}{2\pi}S''_{\rm out}$, this yields the local gravitational field equations.

\subsection{Einstein Field Equations as Geometric Closure}

$$
G_{ab}+\Lambda g_{ab}=8\pi G\,\langle T_{ab}\rangle
$$
holds everywhere on $M$; where $T_{ab}$ is given by $\omega$ and field operator expectations.

\section{Boundary Time Geometry and GHY Term}

\subsection{Boundary Data}

Take a manifold with boundary $(M,g)$, its boundary $\partial M$, induced metric, and extrinsic curvature
$$
h_{ab},\quad K_{ab},\quad K=h^{ab}K_{ab}.
$$

\subsection{EH+GHY Action}

$$
S_{\rm EH}[g]=\frac{1}{16\pi G}\int_M R\sqrt{-g}\,{\rm d}^4x,
\quad
S_{\rm GHY}[g]=\frac{\varepsilon}{8\pi G}\int_{\partial M}K\sqrt{|h|}\,{\rm d}^3x.
$$

\subsection{Brown--York Quasilocal Stress Tensor}

$$
T^{ab}_{\rm BY}=\frac{2}{\sqrt{|h|}}\frac{\delta S}{\delta h_{ab}}
=\frac{\varepsilon}{8\pi G}\big(K^{ab}-Kh^{ab}\big)+\cdots.
$$

\subsection{Geometric Time Generator}

For a timelike Killing vector field $t^a$ on the boundary and a spatial section $\Sigma$, define
$$
H_\partial=\int_\Sigma T^{ab}_{\rm BY}t_an_b\,{\rm d}^{d-1}x,
$$
where $n^b$ is the unit normal of $\Sigma$ in $\partial M$; $H_\partial$ generates boundary time translation $\tau_{\rm geom}$.

\subsection{Alignment with Modular Flow}

Require the existence of a constant $c>0$ such that on the boundary algebra $\mathcal A_\partial$
$$
{\rm Ad}\big(\mathrm e^{-{\rm i}\tau_{\rm geom} H_\partial}\big)
=\sigma_{t_{\rm mod}}^\omega,\quad
t_{\rm mod}=c\,\tau_{\rm geom},
$$
thus geometric time, modular time, and scattering time belong to the same scale equivalence class $[\tau]$.

\section{Observer and Consensus Layer}

\subsection{Observer Objects}

$$
U_{\rm obs}=(\mathcal O,\ {\rm worldline},\ {\rm res},\ {\rm model},\ {\rm update})
$$
where each observer
$$
O_i=(\gamma_i,\ \Lambda_i,\ \mathcal A_i,\ \omega_i,\ \mathcal M_i,\ U_i)
$$
contains: worldline $\gamma_i\subset M$, resolution scale $\Lambda_i$, observable algebra $\mathcal A_i\subseteq\mathcal A$, local state $\omega_i$, candidate model family $\mathcal M_i$, update rule $U_i$.

\subsection{Time Experience Scale}

For each worldline $\gamma_i$, define proper time
$$
\tau_i=\int_{\gamma_i}\sqrt{-g_{\mu\nu}{\rm d}x^\mu{\rm d}x^\nu},
$$
and require the existence of an affine transformation
$$
\tau_i=a_i\tau_{\rm scatt}+b_i=a_i'T_{\rm geo}+b_i'=a_i''t_{\rm mod}+b_i'',
$$
i.e., observer subjective time and the unified scale belong to the same equivalence class.

\subsection{Causal Consensus}

The local partial orders $(C_i,\prec_i)$ of all observers satisfy Čech-like consistency in overlapping regions, implying the existence of a unique global partial order $(X,\preceq)$ (i.e., $U_{\rm evt}$ above), so the "same universe causal net" is the glued limit of all observers' local data.

\section{Category, Topology, and Logic Layer}

\subsection{Universe Category Model}

$$
U_{\rm cat}=(\mathbf{Univ},\ \mathfrak U,\ \Pi)
$$
where
\begin{itemize}
    \item $\mathbf{Univ}$ is a 2-category with "candidate universe structures" as objects and isomorphisms preserving all the above structures as morphisms;
    \item $\mathfrak U$ is the terminal object in $\mathbf{Univ}$: for any object $V$, there exists a unique morphism $V\to\mathfrak U$;
    \item $\Pi$ represents the limit cone composed of projections of each layer (geometric, operator, scattering, modular, entropy, observer, etc.), such that
    $$
    \mathfrak U\simeq
    \varprojlim\big(U_{\rm geo},U_{\rm QFT},U_{\rm scat},U_{\rm mod},U_{\rm ent},U_{\rm obs},\dots\big).
    $$
\end{itemize}

\subsection{Topology and Logic}

$$
\mathscr E={\rm Sh}(M)
$$
is the category of sheaves on $M$ (or a Grothendieck topos), carrying internal higher-order logic; physical propositions correspond to the lattice of subobjects in $\mathscr E$; causality and observability correspond to relations between sublayers and states.

\section{Computation and Realizability Layer}

\subsection{Computable Structure}

$$
U_{\rm comp}=(\mathcal M_{\rm TM},\ {\rm Enc},\ {\rm Sim})
$$
where
\begin{itemize}
    \item $\mathcal M_{\rm TM}$ is the space of Turing machines;
    \item ${\rm Enc}:\mathbf{Univ}\to\mathcal M_{\rm TM}$ is an encoding functor for universe structures (in the upper bound sense);
    \item ${\rm Sim}:\mathcal M_{\rm TM}\rightrightarrows\mathbf{Univ}$ gives a family of simulable sub-universes; the real universe $\mathfrak U$ is the upper bound over all computable models, satisfying consistency but not assuming "computational completeness".
\end{itemize}

\section{Final Compressed Definition of the Universe}

Synthesizing the above, the **Universe** is:

\begin{quote}
Based on a given set theory, a **maximal consistent structure** of all accessible events on $X$, geometric causal structure, quantum fields and operator algebras, scattering and spectral shifts, modular flow and generalized entropy, boundary time geometry and observer networks, etc.
$$
\mathfrak U=
(U_{\rm evt},U_{\rm geo},U_{\rm meas},U_{\rm QFT},U_{\rm scat},U_{\rm mod},U_{\rm ent},U_{\rm obs},U_{\rm cat},U_{\rm comp})
$$
Within it, the unified time scale is given by
$$
\kappa(\omega)
=\frac{\varphi'(\omega)}{\pi}
=\rho_{\rm rel}(\omega)
=\frac{1}{2\pi}{\rm tr}\,Q(\omega)
$$
and is in the same scale equivalence class $[\tau]$ as geometric time $T_{\rm geo}$, modular time $t_{\rm mod}$, and observer proper time $\tau_i$;
All physical laws are compatibility conditions between components of this structure, and the Universe is its unique solution up to isomorphism.
\end{quote}

\section*{Structure Diagram}

The ten-layer structure of the Universe and its interrelations can be illustrated as follows:

\begin{verbatim}
                    U (Terminal Object)
                      |
        +-------------+-------------+
        |                           |
    U_cat <------------------> U_comp
        |                           |
   +----+----+                 +----+----+
   |         |                 |         |
U_obs <-> U_ent            U_QFT <-> U_scat
   |         |                 |         |
   +----+----+                 +----+----+
        |                           |
    U_geo <------------------> U_mod
        |                           |
        +-------------+-------------+
                      |
                   U_evt
                      |
                   U_meas
\end{verbatim}

Time scale unification relation:
$$
[\tau] = \{T_{\rm cau}, T_{\rm geo}, \tau_{\rm scatt}, t_{\rm mod}, \tau_{\rm geom}, \tau_i\}_{\text{affine equivalence}}
$$

All arrows represent structural compatibility constraints, and the Universe $\mathfrak U$ is the unique maximal solution that satisfies all constraints simultaneously.

\section{Mathematical Status}

In the category $\mathbf{Univ}$, the universe $\mathfrak U$ has the following properties:

\begin{enumerate}
    \item **Terminality**: For any candidate universe structure $V$, there exists a unique morphism $V\to\mathfrak U$.
    \item **Limit Property**: $\mathfrak U$ is the inverse limit of all component structures.
    \item **Completeness**: All physical laws are satisfied simultaneously as compatibility conditions in $\mathfrak U$.
    \item **Maximality**: There exists no consistent structure strictly containing $\mathfrak U$.
    \item **Uniqueness**: $\mathfrak U$ is uniquely determined by the above axioms up to isomorphism.
\end{enumerate}

Therefore, the Universe is not "constructed", but **a mathematical object that exists uniquely under consistency constraints**.

\end{document}

