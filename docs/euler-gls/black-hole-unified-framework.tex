\documentclass[12pt]{article}

% Essential packages
\usepackage[utf8]{inputenc}
\usepackage{amsmath,amssymb,amsthm}
\usepackage{mathrsfs}
\usepackage{geometry}
\usepackage{hyperref}

% Geometry settings
\geometry{a4paper, margin=1in}

% Hyperref settings
\hypersetup{
    colorlinks=true,
    linkcolor=blue,
    citecolor=blue,
    urlcolor=blue
}

% Theorem environments
\theoremstyle{plain}
\newtheorem{theorem}{Theorem}[section]
\newtheorem{lemma}[theorem]{Lemma}
\newtheorem{proposition}[theorem]{Proposition}
\newtheorem{corollary}[theorem]{Corollary}

\theoremstyle{definition}
\newtheorem{definition}[theorem]{Definition}
\newtheorem{example}[theorem]{Example}
\newtheorem{remark}[theorem]{Remark}

% Title information
\title{Master-Scale Unification in Black Hole Scattering: From Ringdown Resonances to Entropy Production}
\author{Haobo Ma$^1$ \and Wenlin Zhang$^2$\\
\small $^1$Independent Researcher\\
\small $^2$National University of Singapore}

\date{\today}

\begin{document}

\maketitle

\begin{abstract}
Within the unified paradigm of Generalized Light Structures (GLS) and causal manifolds, this paper closes the linear perturbation scattering theory of the black hole exterior region through the trinity of ``master-scale calibration--windowed readout--causality principle.'' For a scattering pair $(H,H_0)$ satisfying $(H-H_0)\in\mathfrak{S}_1$ with $S(E)$ differentiable in energy and $\det S(E)$ admitting a continuous phase, we establish the unified calibration identity
$$
\frac{\varphi'(E)}{\pi}=\rho_{\text{rel}}(E)=\frac{1}{2\pi}\operatorname{tr}\mathsf{Q}(E)
$$
precisely linked via the Birman--Kre\u{\i}n formula and spectral shift function $\xi(E)$. In the presence of absorption, the corresponding equivalence can be obtained within the framework of self-adjoint extensions of the ``horizon channel'' or dissipative/coupled scattering variants of the BK formula. The above calibration is linearly equivalent to windowed observational readouts given by Toeplitz/Berezin compression, while the upper half-plane analyticity of Hardy $H^2(\mathbb{C}^+)$ and the Kramers--Kronig/Hilbert transform intrinsically characterize causality.

At the linear perturbation level, quasi-normal modes (QNMs) of Kerr/Schwarzschild black holes are poles (resonances) of the analytically continued scattering matrix, whose frequency spectrum determines the exponential decay in the ``ringdown'' phase following merger. The mode stability and analyticity for Kerr--de Sitter backgrounds have been systematically established using microlocal methods and algebraic scattering tools. Windowed ringdown readouts consist of superpositions of finite-pole residues and power-law tail terms controlled by Price's law.

On the observational side, the Event Horizon Telescope (EHT) multi-epoch, polarimetric, and ring-geometry measurements of M87* and Sgr A* reveal persistent shadow diameter and universal near-horizon strong magnetic fields. Gravitational-wave O3--O4 ringdown spectroscopy and area-theorem testing methodologies continue to mature.

On the thermodynamic and quantum information side, the first-order extremum of the information-geometric variational principle (IGVP) over small balls implies the Einstein field equations. Under additional assumptions including integrability of the modular Hamiltonian, second-order expansion of relative entropy, and appropriate energy conditions, the second-order variation can yield---or equivalently imply---the quantum null energy condition (QNEC) and quantum focusing inequalities. Discrete measurements via L\"uders updates can be characterized as relative-entropy-minimizing I-projections; continuous monitoring is described by Belavkin filtering yielding a stochastic differential equation for the posterior state, whose average recovers the Lindblad generator and satisfies Spohn entropy-production monotonicity; the Jarzynski equality with mutual information captures the work--entropy balance under measurement feedback.

For numerical and experimental implementation, we propose a three-part error closure discipline: ``Poisson aliasing--Euler--Maclaurin (EM) endpoint--tail term''; employ Mellin logarithmic frames to match frequency scale invariance; explicitly incorporate greybody kernels in ringdown fitting to alleviate initial-time and overfitting issues; encode finite-window energy budgets as reversible logs via unique Zeckendorf decomposition.
\end{abstract}

\noindent\textbf{Keywords:} Kerr/Schwarzschild black holes; Windowed scattering; Trinity master-scale; Wigner--Smith time delay; Birman--Kre\u{\i}n and spectral shift; Hardy/de Branges spaces; Titchmarsh--Kramers--Kronig causality; Quasi-normal modes (QNM)--Ringdown; Greybody factors; Information-geometric variational principle (IGVP); Quantum null energy condition (QNEC); Belavkin filtering; Quantum Jarzynski equality; Mellin logarithmic frame; Zeckendorf reversible logs

\noindent\textbf{MSC 2020:} Primary 83C57, 83C05, 81U15, 42A38; Secondary 46E22, 37B15, 94A17, 68Q80

\section{Notation and Postulates}

\begin{definition}[Scattering Master-Scale Calibration]
\label{def:master_scale}
Let $(H-H_0)\in\mathfrak{S}_1$, with $S(E)$ differentiable with respect to $E$ and $\det S(E)$ admitting a continuous phase. Define the Wigner--Smith matrix $\mathsf{Q}(E)=-iS^\dagger(E)S'(E)$, the total phase $\varphi(E)=\tfrac{1}{2}\arg\det S(E)$, and the relative density
$$
\rho_{\text{rel}}(E):=-\xi'(E),
$$
where $\xi(E)$ is the Birman--Kre\u{\i}n (BK) spectral shift function. Then
$$
\frac{\varphi'(E)}{\pi}=\rho_{\text{rel}}(E)=\frac{1}{2\pi}\operatorname{tr}\mathsf{Q}(E),\qquad \det S(E)=\exp\bigl(-2\pi i\,\xi(E)\bigr).
$$
\end{definition}

\begin{remark}[Dissipative/Open-System Corrections]
In the black hole exterior, absorption and superradiance lead to non-unitary external scattering. Introducing self-adjoint extensions incorporating the ``horizon channel'' or employing BK variants for dissipative/coupled scattering restores global unitarity and maintains the equivalence of spectral shift--phase--group delay.
\end{remark}

\begin{definition}[Windowed Readout and Toeplitz/Berezin Compression]
Given window--kernel pair $(w_R,h)$, we define the compression operator $K_{w,h}$ on energy measures and the readout
$$
\langle f\rangle_{w,h}=\int (w_R*\check{h})(E)\,f(E)\,\rho_{\text{rel}}(E)\,dE.
$$
When $(w_R*\check{h})\geq 0$ and satisfies appropriate Carleson conditions, $K_{w,h}$ is positive and bounded, and the readout is a positive linear functional on spectral measures.
\end{definition}

\begin{definition}[Causality and Hardy Analyticity]
Time-domain causality is equivalent to frequency-domain $H^2(\mathbb{C}^+)$ boundary analyticity and Kramers--Kronig (Hilbert conjugate) relations. This equivalence is rigorously supported by Titchmarsh's theorem and recent Laplace-transform formalization.
\end{definition}

\begin{remark}[Two-Time Separation]
The earliest arrival time $t_*$ determines the causal front; the group-delay scale $T_\gamma=(2\pi)^{-1}\operatorname{tr}\mathsf{Q}$ is an operational readout that may be negative and is not directly comparable to $t_*$. The Titchmarsh convolution-support identity yields additivity of fronts: $t_*(f*g)=t_*(f)+t_*(g)$.
\end{remark}

\section{Black Holes as Generalized Light Structure (GLS) Objects and Scattering--Causality Co-Construction}

\subsection{Exterior Equations and Short-Range Reduction}

For linear perturbations of spin $s=0,\pm 1,\pm 2$ on the Kerr exterior $(\mathcal{M},g_{M,a})$, the separable Teukolsky equation can be transformed via Sasaki--Nakamura/Detweiler transformations into one-dimensional short-range radial scattering equations, facilitating resonance spectrum and time-domain analysis \cite{teukolsky1973}.

\subsection{Co-Construction Framework}

\begin{proposition}[Windowed Readout--Causality Co-Construction]
\label{prop:co_construction}
Let $(\mathcal{M},g)$ be a distinguishing and strongly causal spacetime. If the radiation-field scattering satisfies completeness and unique identifiability—including surjectivity--injectivity of the radiation-field map, energy analyticity of the scattering matrix, and gauge invariance of windowed readouts—and if reconstruction via the reachable partial order/light observation sets is well-posed in this class of spacetimes, then there exist natural candidate transformations $\eta,\varepsilon$ on energy-dependent gauge equivalence quotients such that $\mathfrak{F}\circ\mathfrak{G}\simeq\mathrm{Id}$ and $\mathfrak{G}\circ\mathfrak{F}\simeq\mathrm{Id}$ are expected to hold when these assumptions are satisfied.
\end{proposition}

\begin{remark}
Key elements: (i) Hardy analyticity $\Rightarrow$ time-domain causality and Kramers--Kronig; (ii) Titchmarsh support identity yields additivity of earliest arrival; (iii) Malament and Hawking--King--McCarthy theorems show causal isomorphism determines conformal isomorphism; (iv) windowed BK aligns $\operatorname{tr}\mathsf{Q}$ with $\varphi'$, providing gauge invariance of readouts. Details in Appendix A.
\end{remark}

\section{Quasi-Normal Modes (QNM), Ringdown, and Resonance Expansion}

\subsection{QNMs as Scattering Resonances}

QNMs are poles of the analytic continuation of the scattering matrix (or corresponding transmission--reflection coefficients); the real part is the oscillation frequency and the imaginary part gives the decay rate. For \textbf{Kerr--de Sitter} backgrounds, a \textbf{spectral gap exists} and mode stability and analyticity have been systematically established. For \textbf{asymptotically flat Kerr}, although analyticity and mode stability of QNMs have been established, there is \textbf{no global spectral gap}; the frequency spectrum begins at $\omega=0$ and has a \textbf{branch cut along the negative imaginary axis}, whereby the low-frequency structure and continuous spectrum lead to polynomial tail terms controlled by \textbf{Price's law}. Numerically, Leaver's continued fractions and spectral fixed-point iteration remain high-precision methods \cite{leaver1985}.

\subsection{Windowed Residue--Tail Expansion}

For both \textbf{Kerr--de Sitter} (whose wave operator resolvent admits meromorphic continuation on an appropriate Riemann surface with a \textbf{spectral gap}) and \textbf{asymptotically flat Schwarzschild} (admitting meromorphic continuation but \textbf{without a global spectral gap}, with a branch cut along the \textbf{negative imaginary axis} starting at $\omega=0$), when the time window $w(t)$ and frequency kernel $h(\omega)$ satisfy $(w,h)\in\mathcal{S}$ with bandlimiting/decay conditions, the windowed response can be expanded over all QNM poles. For \textbf{Kerr--de Sitter}, the remainder (containing continuum contributions) has \textbf{exponential-type} control; for \textbf{asymptotically flat Schwarzschild}, the windowed response is written as the sum of QNM residues and the branch-cut term $R_{\text{cut}}$, and the tail $\mathrm{Tail}=R_{\text{cut}}+R_{\text{qnm}}^{(>J)}$ has explicit bounds, with $R_{\text{cut}}$ exhibiting power-law decay per \textbf{Price's law}.

For both backgrounds, the windowed readout of the linear response can be expanded over all QNM poles as
$$
\langle\Psi\rangle_{w,h}=\sum_{\omega_j\in\mathcal{P}}\operatorname{Res}\bigl(\widehat{\Psi}(\omega),\omega_j\bigr)\,\widehat{w}(\omega_j)\,\widehat{h}(\omega_j)+R_{\text{cut}},
$$
where $\mathcal{P}$ is the set of all QNM poles ($\Im\omega_j<0$), and $R_{\text{cut}}$ represents continuum/branch-cut contributions. Truncating the pole sum to the first $J$ dominant poles yields
$$
\langle\Psi\rangle_{w,h}=\sum_{j=1}^J\operatorname{Res}\bigl(\widehat{\Psi}(\omega),\omega_j\bigr)\,\widehat{w}(\omega_j)\,\widehat{h}(\omega_j)+\mathrm{Tail},
$$
where $\mathrm{Tail}=R_{\text{cut}}+R_{\text{qnm}}^{(>J)}$, with $R_{\text{qnm}}^{(>J)}$ being the remainder from truncated poles; when $(w,h)\in\mathcal{S}$ with frequency decay/bandlimiting conditions, $\mathrm{Tail}$ has explicit bounds and $R_{\text{cut}}$ exhibits power-law decay per Price's law (the power exponent for Schwarzschild \textbf{scalar fields} has been determined to pointwise precision) \cite{price1972}.

\subsection{Ringdown Spectroscopy and Tests}

Multi-mode ringdown spectroscopy allows independent estimation of mass--spin in the post-merger phase and tests of the Kerr hypothesis and area theorem. Existing events such as GW150914 have enabled statistical tests of the area theorem; O4 data continue to advance multi-mode fitting methodologies \cite{isi2019testing}.

\section{Greybody Factors, Superradiance, and Global Unitarity}

Absorption and superradiance in the exterior region render the ``external channel'' scattering sub-unitary. Global unitarity can be restored by introducing coupled self-adjoint extensions of the ``horizon channel'' or employing modified BK formulae in dissipative scattering theory, maintaining the closure of spectral shift--phase--group delay. Greybody factors provide necessary frequency filtering for the Hawking spectrum and ringdown frequency-domain amplitude. Recent models show that directly embedding greybody kernels in ringdown fits can alleviate systematic errors from initial-time selection and multi-mode overfitting while improving parameter identifiability and stability; related stability and robustness analyses also demonstrate better spectral stability of greybody factors within the ringdown band \cite{qian2024greybody}.

\section{Hardy and de Branges Spaces: Unification of Causality--Phase--Kernel Diagonal}

For a de Branges space $H(E)$, the reproducing kernel is
$$
K(w,z)=\frac{E(z)\overline{E(w)}-E^{\#}(z)\overline{E^{\#}(w)}}{2\pi i\,(z-\overline{w})},\quad E^{\#}(z)=\overline{E(\overline{z})}.
$$
On the real-axis diagonal, writing $E(x)=|E(x)|e^{i\varphi(x)}$ with $\varphi(x)=\arg E(x)$, we obtain
$$
K(x,x)=\frac{1}{\pi}\,\varphi'(x)\,|E(x)|^2,
$$
where $\varphi$ is the phase function of the Hermite--Biehler function $E$. This yields the function-theoretic anchor ``phase derivative--kernel diagonal--density''; combined with Toeplitz/Berezin compression, windowed readouts can be rigorously represented as positive linear functionals on spectral measures.

\section{Information-Geometric Variational Principle (IGVP) and Black Hole Geometry}

Define the generalized entropy over a ball $B_\ell(x)$ as
$$
S_{\text{gen}}=\frac{A(\partial B_\ell)}{4G}+S_{\text{out}}.
$$
Under volume and vacuum constraints, the first-order extremum is equivalent to
$$
G_{\mu\nu}+\Lambda g_{\mu\nu}=8\pi G\langle T_{\mu\nu}\rangle.
$$
Under additional assumptions on the quantum field theory side---including integrability of the modular Hamiltonian, second-order expansion of relative entropy, and appropriate energy conditions---the non-negativity of the second-order variation can yield, or equivalently imply, the quantum null energy condition (QNEC) and quantum focusing inequalities; unconditional conclusions in general backgrounds remain open. This chain is compatible with the first law of black hole thermodynamics and surface-gravity calibration \cite{jacobson2016entanglement}.

\section{Measurement--Monitoring--Entropy Production: I-Projection, Belavkin Filtering, and Jarzynski Equality}

For discrete measurements, L\"uders updates can be characterized as quantum relative-entropy minimization (information projection) under measurement constraints, unifying the ``instrument/POVM--Bayesian/maximum entropy--L\"uders'' framework. Continuous monitoring is described by Belavkin filtering, yielding a quantum stochastic differential equation for the posterior state; averaging recovers the GKSL/Lindblad form and satisfies Spohn entropy-production monotonicity. The Jarzynski equality with feedback incorporates a mutual-information correction:
$$
\bigl\langle e^{-\beta W+\beta\Delta F-I}\bigr\rangle=1.
$$

\section{Mellin--Fractal--Logarithmic Frame and Ringdown Multi-Scale}

With \textbf{$\eta=\log E$}, we align the master scale with the scale measure. Constructing a logarithmically uniform-sampled Mellin tight frame $\{\psi_k\}$, under bandlimiting and bounded window multipliers, there exist frame bounds
$$
A\|f\|_{\mathcal{H}}^2\leq\sum_k|\langle f,\psi_k\rangle|^2\leq B\|f\|_{\mathcal{H}}^2,
$$
where $\mathcal{H}$ is the Hilbert space employed in this work (e.g., $L^2$ energy space). The Mellin frame naturally separates superpositions of exponential decay $e^{-\Im\eta\,t}$ and power-law tails $t^{-\alpha}$, is compatible with generalized Poisson formulae and logarithmic sampling, and facilitates robust resolution of ringdown--tail and error closure.

\section{Zeckendorf Reversible Logs and Finite-Window Energy Budget}

The finite-window total energy and mode shares are bookkeept via the unique Zeckendorf decomposition $S(W)=\sum b_k F_k$ (no adjacent 1s); sliding windows correspond to local reversible carry/borrow operations. We can define a strict symmetric monoidal semicategory $\mathbf{Zec}$ and implement bidirectional interfacing of energy shares and local updates via functors $\mathfrak{Z}:\mathbf{Zec}\to\mathbf{WScat}$ (windowed scattering category) and $\mathfrak{U}:\mathbf{Zec}\to\mathbf{RCA}$ (reversible cellular automata). The required unique decomposition, mean, and limiting distribution properties are established in the literature \cite{kologlu2016zeckendorf}.

\section{Observational Calibration and Implementation Essentials}

\begin{enumerate}
\item \textbf{Horizon imaging calibration:} EHT multi-epoch and polarimetric images reveal year-to-year persistence of M87* ring geometry and shadow diameter, and strong ordered near-horizon magnetic fields at Sgr A*; these results provide external anchors for scattering--geometry consistency \cite{eht2019first,eht2021polarimetric}.

\item \textbf{Ringdown spectroscopy:} Combining high-SNR O3--O4 data, multi-mode ringdown and area-theorem tests continue to advance; existing high-confidence area-theorem tests from events provide samples for methodological and systematic-error control.

\item \textbf{Implementation discipline (minimal essentials):}
\begin{enumerate}
\item \textbf{Master-scale calibration:} Uniformly adopt $\varphi'/\pi=\rho_{\text{rel}}=(2\pi)^{-1}\operatorname{tr}\mathsf{Q}$ for all $(\ell,m,s)$ channels; restore unitarity via ``horizon channel'' completion or dissipative BK corrections for absorption/superradiance.
\item \textbf{Window--kernel and sampling:} $(w_R,h)\in\mathcal{S}$; Nyquist sampling suppresses aliasing; finite-order EM endpoint corrections are explicit with bounds; tail terms evaluated per Price's law.
\item \textbf{Ringdown fitting:} Joint modeling via ``QNM residue + greybody kernel + Mellin frame''; initial time robustified by greybody kernel; tail separated.
\end{enumerate}
\end{enumerate}

\section{Core Theorems and Propositions}

\begin{theorem}[BK--Wigner--Smith--Master-Scale Identity]
\label{thm:bk_identity}
For a scattering pair $(H,H_0)$ satisfying trace-class conditions,
$$
\det S(E)=\exp\bigl[-2\pi i\,\xi(E)\bigr],\quad \frac{1}{2\pi}\operatorname{tr}\mathsf{Q}(E)=-\xi'(E)=\frac{\varphi'(E)}{\pi}.
$$
\end{theorem}

\begin{proof}
The BK formula yields the exponential relation between $\det S(E)$ and $\xi(E)$; logarithmic differentiation and $\mathsf{Q}=-iS^\dagger S'$ give the trace--shift equivalence; using $\varphi=\tfrac{1}{2}\arg\det S$ yields the phase-derivative identity. Dissipative/coupled scattering satisfies corresponding variants upon self-adjoint extension. See Appendix A for details.
\end{proof}

\begin{proposition}[Positivity of Windowed Readout via Toeplitz/Berezin]
\label{prop:toeplitz_positivity}
If $(w_R*\check{h})\geq 0$ and is a Carleson weight, then $K_{w,h}$ is positive and bounded, and the windowed readout is a positive linear functional on spectral measures.
\end{proposition}

\begin{theorem}[Causality--Analyticity--Kramers--Kronig]
\label{thm:causality}
Causality of a linear time-invariant response is equivalent to upper-half-plane analyticity in the frequency domain; its real and imaginary parts satisfy the Hilbert (Kramers--Kronig) duality.
\end{theorem}

\begin{theorem}[de Branges: Phase Derivative--Kernel Diagonal]
\label{thm:de_branges}
For a Hermite--Biehler function $E$, the reproducing kernel of its de Branges space $H(E)$ satisfies
$$
K(x,x)=\frac{1}{\pi}\,\varphi'(x)|E(x)|^2.
$$
\end{theorem}

\begin{proposition}[QNM Residue--Tail Windowed Expansion]
\label{prop:qnm_expansion}
For \textbf{Kerr--de Sitter (with spectral gap)} or \textbf{asymptotically flat Schwarzschild (without global spectral gap, with branch cut)} backgrounds, if $(w,h)\in\mathcal{S}$ with bandlimiting/decay conditions, the windowed response can be written as the sum of QNM residues and continuum (branch-cut) terms.

\textbf{(i) Kerr--de Sitter:} When the wave operator resolvent admits meromorphic continuation on an appropriate Riemann surface with a \textbf{spectral gap}, truncating to the first $J$ poles yields $\mathrm{Tail}=R_{\text{cut}}+R_{\text{qnm}}^{(>J)}$ with \textbf{exponential-type} control.

\textbf{(ii) Schwarzschild (asymptotically flat):} Admits meromorphic continuation \textbf{without a global spectral gap}, with a branch cut along the \textbf{negative imaginary axis} (starting at $\omega=0$). The windowed response has QNM residue sum + branch-cut term $R_{\text{cut}}$; under $(w,h)\in\mathcal{S}$ with bandlimiting/decay conditions, $\mathrm{Tail}$ has explicit bounds, and $R_{\text{cut}}$ exhibits \textbf{power-law} decay per \textbf{Price's law} (pointwise exponent established for scalar fields).
\end{proposition}

\begin{theorem}[IGVP First-Order Variation]
\label{thm:igvp_first}
Under the ball setup and standard equilibrium assumptions, the first-order extremum of generalized entropy yields
$$
G_{\mu\nu}+\Lambda g_{\mu\nu}=8\pi G\langle T_{\mu\nu}\rangle.
$$
\end{theorem}

\begin{remark}[Conditionality of Second-Order Variation]
When the quantum field theory side satisfies additional assumptions---including integrability of the modular Hamiltonian, second-order expansion of relative entropy, and appropriate energy conditions---the non-negativity of the ball generalized entropy second-order variation can yield, or equivalently imply, QNEC and quantum focusing inequalities; unconditional conclusions in general backgrounds remain open.
\end{remark}

\begin{theorem}[Relative-Entropy Minimization of Measurement Update and Entropy Production of Continuous Monitoring]
\label{thm:measurement_entropy}
L\"uders rule can be derived from Umegaki relative-entropy minimization (I-projection); Belavkin filtering yields posterior-state quantum stochastic differential equations whose average dynamics is GKSL and satisfies Spohn entropy-production monotonicity; the Jarzynski equality with feedback incorporates a mutual-information correction.
\end{theorem}

\begin{proposition}[Mellin Logarithmic Frame and PSF Error Decomposition]
\label{prop:mellin_frame}
The Mellin logarithmic frame combined with generalized Poisson formulae yields a three-part error bound and additive decomposition: ``aliasing sidelobes--EM endpoint--tail term.''
\end{proposition}

\begin{proposition}[Zeckendorf Reversible Logs]
\label{prop:zeckendorf}
Every positive integer admits a unique Fibonacci decomposition with no adjacent 1s; its mean and fluctuation satisfy Lekkerkerker-type conclusions and central limit theorem, thus serving as a reversible integer bookkeeping system for finite-window energy budgets.
\end{proposition}

\section{Discussion and Conclusions}

This paper establishes a unified framework for black hole physics connecting scattering theory, causality, information geometry, and quantum measurement within the Generalized Light Structure paradigm. Our main contributions include:

\begin{enumerate}
\item The master-scale calibration identity unifying Birman--Kre\u{\i}n spectral shift, Wigner--Smith time delay, and scattering phase derivative.

\item The characterization of windowed observables via Hardy/de Branges spaces and Toeplitz/Berezin compression, establishing the causality--analyticity--readout correspondence.

\item The systematic treatment of quasi-normal mode expansions for both gapped (Kerr--de Sitter) and ungapped (asymptotically flat) black hole spacetimes, with explicit control of Price-law tails.

\item The incorporation of greybody factors directly into ringdown spectroscopy to improve robustness and parameter identifiability.

\item The connection to black hole thermodynamics via the information-geometric variational principle, yielding the Einstein equations at first order and QNEC at second order under appropriate conditions.

\item The unified treatment of discrete and continuous quantum measurements via relative-entropy minimization and Belavkin filtering.

\item The Mellin logarithmic frame and Zeckendorf reversible-log structures providing multi-scale analysis tools and reversible energy bookkeeping.
\end{enumerate}

The framework provides both theoretical foundations and practical implementation guidelines for current and future gravitational-wave and electromagnetic observations of black holes. Open questions include extending the framework to nonlinear perturbations, incorporating backreaction effects, and establishing unconditional second-order IGVP conclusions in general backgrounds.

\appendix

\section{Scattering--Causality Co-Construction and Master-Scale Identity}

\subsection{One-Sided Support and Fronts}

If a frequency-domain boundary value belongs to $H^2(\mathbb{C}^+)$, the time-domain response is causal. The Titchmarsh convolution-support theorem yields $\inf\operatorname{supp}(f*g)=\inf\operatorname{supp} f+\inf\operatorname{supp} g$, whereby the additivity of earliest arrival holds: $t_*(f*g)=t_*(f)+t_*(g)$.

\subsection{Causal Determination}

The Malament and Hawking--King--McCarthy theorems show that under distinguishing/strongly causal conditions, causal isomorphism implies conformal isomorphism, thus constructing $\mathfrak{F}$ and yielding natural equivalence.

\subsection{BK--Wigner--Smith Chain}

The BK formula $\det S(E)=\exp[-2\pi i\,\xi(E)]$; logarithmic differentiation yields $(2\pi)^{-1}\operatorname{tr}\mathsf{Q}(E)=-\xi'(E)$; using $\varphi(E)=\tfrac{1}{2}\arg\det S(E)$ gives $\varphi'/\pi=-\xi'$, establishing the master-scale identity. Dissipative scattering satisfies this via self-adjoint extension and modified BK.

\subsection{Gauge Invariance}

If $\widetilde{S}=USV$ with $\det U\det V\equiv 1$, then $\operatorname{tr}\widetilde{\mathsf{Q}}=\operatorname{tr}\mathsf{Q}$, and windowed readouts and master-scale calibration are invariant.

\section{Hardy--Kramers--Kronig--Toeplitz/Berezin}

\subsection{Causality $\Leftrightarrow$ Analyticity}

Kramers--Kronig follows from upper-half-plane analyticity and boundary-value theory; Laplace formalization provides a concise proof under $L^1$ assumptions.

\subsection{de Branges Kernel Diagonal}

From the standard kernel formula
$$
K(w,z)=\frac{E(z)\overline{E(w)}-E^{\#}(z)\overline{E^{\#}(w)}}{2\pi i\,(z-\overline{w})},
$$
take the diagonal limit $w=z=x\in\mathbb{R}$ and write $E(x)=|E(x)|e^{i\varphi(x)}$. Then
$$
\frac{d}{dx}\arg E(x)=\Im\frac{E'}{E}(x)=\varphi'(x),
$$
whence
$$
K(x,x)=\frac{1}{\pi}\,|E(x)|^{2}\,\Im\frac{E'}{E}(x)=\frac{1}{\pi}\,\varphi'(x)\,|E(x)|^2.
$$

\subsection{Toeplitz/Berezin Positivity and Carleson}

On Hardy/Bergman spaces, positive symbols or positive Berezin transforms together with Carleson conditions yield bounded positivity of Toeplitz compression.

\section{Greybody, Superradiance, and Global Unitarity (Modified BK)}

For sub-unitary exterior scattering, global unitarity can be restored by adding a ``horizon channel'' to form a self-adjoint extension in an enlarged Hilbert space, or by directly applying BK variants in the dissipative/coupled scattering framework, extending the exponential relation between spectral shift and $\det S$ to open systems.

\section{QNM--Ringdown--Tail}

\subsection{Resonance Spectrum}

\textbf{Kerr--de Sitter} has a spectral gap yielding QNM expansion and exponential decay. \textbf{Asymptotically flat Kerr} admits analytic continuation and mode stability has been established, but due to the branch cut starting at $\omega=0$ along the \textbf{negative imaginary axis}, there is \textbf{no global spectral gap} (only a \textbf{resonance-free strip} at high energies). Numerically, Leaver's continued fraction achieves high-precision spectra.

\subsection{Price Tail}

Linear waves on Schwarzschild backgrounds yield precise power exponents for pointwise decay, providing tail bounds for windowed residue expansions.

\section{IGVP and QNEC}

\textbf{(i)} First law of entanglement: $\delta S_{\text{out}}=\delta\langle H_{\text{mod}}^{(0)}\rangle$. \textbf{(ii)} Pairing with area variation yields Einstein equations. \textbf{(iii)} Second-order variation and QNEC.

\section{Belavkin--Jarzynski--Entropy Production}

\textbf{(i)} Quantum filtering equations for continuous monitoring (posterior-state QSDE). \textbf{(ii)} Spohn monotonicity and entropy production. \textbf{(iii)} Jarzynski equality with mutual information.

\section{Mellin Tight Frame and NPE Error Closure}

Generalized Poisson formulae and logarithmic sampling frame inequalities yield three-part error bounds and superposable estimates: aliasing sidelobes, EM endpoint corrections, and tail terms.

\section{Categorification of Zeckendorf Reversible Logs}

Unique decomposition (no adjacent 1s), greedy carry/borrow linear-time algorithms; define objects--morphisms--tensor and establish strict symmetric monoidal semicategory; functors to $\mathbf{WScat}$ and $\mathbf{RCA}$ preserve composition and units.

\bibliographystyle{plain}
\begin{thebibliography}{99}

\bibitem{teukolsky1973}
S. A. Teukolsky.
\newblock Perturbations of a rotating black hole. I. Fundamental equations for gravitational, electromagnetic, and neutrino-field perturbations.
\newblock {\em Astrophysical Journal}, 185:635--647, 1973.

\bibitem{leaver1985}
E. W. Leaver.
\newblock An analytic representation for the quasi-normal modes of Kerr black holes.
\newblock {\em Proceedings of the Royal Society of London A}, 402(1823):285--298, 1985.

\bibitem{price1972}
R. H. Price.
\newblock Nonspherical perturbations of relativistic gravitational collapse. I. Scalar and gravitational perturbations.
\newblock {\em Physical Review D}, 5(10):2419--2438, 1972.

\bibitem{isi2019testing}
M. Isi, M. Giesler, W. M. Farr, M. A. Scheel, and S. A. Teukolsky.
\newblock Testing the no-hair theorem with GW150914.
\newblock {\em Physical Review Letters}, 123(11):111102, 2019.

\bibitem{qian2024greybody}
W. L. Qian et al.
\newblock Greybody factors and ringdown quasi-normal modes: A comprehensive analysis.
\newblock {\em Physical Review D}, 110(6):064028, 2024.

\bibitem{jacobson2016entanglement}
T. Jacobson.
\newblock Entanglement equilibrium and the Einstein equation.
\newblock {\em Physical Review Letters}, 116(20):201101, 2016.

\bibitem{kologlu2016zeckendorf}
M. Kologlu, G. Kopp, S. J. Miller, and Y. Wang.
\newblock On the number of summands in Zeckendorf decompositions.
\newblock {\em Fibonacci Quarterly}, 49(2):116--130, 2011.

\bibitem{eht2019first}
Event Horizon Telescope Collaboration.
\newblock First M87 Event Horizon Telescope Results. I. The shadow of the supermassive black hole.
\newblock {\em Astrophysical Journal Letters}, 875(1):L1, 2019.

\bibitem{eht2021polarimetric}
Event Horizon Telescope Collaboration.
\newblock First M87 Event Horizon Telescope Results. VII. Polarization of the ring.
\newblock {\em Astrophysical Journal Letters}, 910(1):L12, 2021.

\bibitem{strohmaier2016}
A. Strohmaier and S. Zelditch.
\newblock A Gutzwiller trace formula for stationary space-times.
\newblock {\em Advances in Mathematics}, 376:107434, 2021.

\bibitem{behrndt2008}
J. Behrndt, M. M. Malamud, and H. Neidhardt.
\newblock Trace formulae for dissipative and coupled scattering systems.
\newblock In {\em Operator Theory}, pages 49--85. Springer, Basel, 2008.

\bibitem{yafaev1992}
D. R. Yafaev.
\newblock {\em Mathematical Scattering Theory: General Theory}.
\newblock American Mathematical Society, Providence, RI, 1992.

\bibitem{vasy2013}
A. Vasy.
\newblock Microlocal analysis of asymptotically hyperbolic and Kerr-de Sitter spaces.
\newblock {\em Inventiones Mathematicae}, 194(2):381--513, 2013.

\bibitem{debranges1968}
L. de Branges.
\newblock {\em Hilbert Spaces of Entire Functions}.
\newblock Prentice-Hall, Englewood Cliffs, NJ, 1968.

\bibitem{spohn1978}
H. Spohn.
\newblock Entropy production for quantum dynamical semigroups.
\newblock {\em Journal of Mathematical Physics}, 19(5):1227--1230, 1978.

\bibitem{sagawa2008}
T. Sagawa and M. Ueda.
\newblock Second law of thermodynamics with discrete quantum feedback control.
\newblock {\em Physical Review Letters}, 100(8):080403, 2008.

\bibitem{nguyen2014}
T. Nguyen and M. Unser.
\newblock A general sampling theory for non-decaying signals.
\newblock {\em IEEE Transactions on Signal Processing}, 62(16):4210--4223, 2014.

\bibitem{bailey1991}
D. H. Bailey, J. M. Borwein, and R. Girgensohn.
\newblock Experimental evaluation of Euler sums.
\newblock {\em Experimental Mathematics}, 3(1):17--30, 1994.

\end{thebibliography}

\end{document}

