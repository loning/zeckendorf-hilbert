\documentclass[12pt]{article}

% Essential packages
\usepackage[utf8]{inputenc}
\usepackage{amsmath,amssymb,amsthm}
\usepackage{mathrsfs}
\usepackage{geometry}
\usepackage{hyperref}

% Geometry settings
\geometry{a4paper, margin=1in}

% Hyperref settings
\hypersetup{
    colorlinks=true,
    linkcolor=blue,
    citecolor=blue,
    urlcolor=blue
}

% Theorem environments
\theoremstyle{plain}
\newtheorem{theorem}{Theorem}[section]
\newtheorem{lemma}[theorem]{Lemma}
\newtheorem{proposition}[theorem]{Proposition}
\newtheorem{corollary}[theorem]{Corollary}

\theoremstyle{definition}
\newtheorem{definition}[theorem]{Definition}
\newtheorem{example}[theorem]{Example}
\newtheorem{remark}[theorem]{Remark}

% Title information
\title{Friedmann Equations as Geometric Entropy Production: A Thermodynamic Reformulation of Cosmological Dynamics}
\author{Haobo Ma$^1$ \and Wenlin Zhang$^2$\\
\small $^1$Independent Researcher\\
\small $^2$National University of Singapore}

\date{\today}

\begin{document}

\maketitle

\begin{abstract}
In an isotropic and homogeneous FLRW universe, taking the apparent horizon as the natural thermodynamic boundary, we construct the generalized entropy $S_{\text{gen}}(t)=\tfrac{A(t)}{4G}+S_{\text{out}}(t)$ (where $A(t)=4\pi r_A^2$ and $r_A=(H^2+\kappa/a^2)^{-1/2}$), and adopt the quasi-static limit $T=\tfrac{1}{2\pi r_A}$ of the Kodama--Hayward temperature $T=|\kappa_{\text{sg}}|/(2\pi)$. This allows us to equivalently reformulate cosmic dynamics as a geometric entropy production process. Using the Hayward unified first law $dE=A\psi+W\,dV$ and the Clausius relation $\delta Q=T\,dS$ as the core, we obtain on the apparent horizon a set of thermodynamic relations that are mutually equivalent to the two Friedmann equations. From
$$
T\dot{S}_h=4\pi H r_A^3(\rho+p),
$$
we see that the reversible entropy flux is determined solely by the matter/radiation enthalpy flux; curvature and vacuum $\Lambda$ do not produce independent local entropy fluxes (even if $\dot{\Lambda}\neq 0$, its effect is self-consistently absorbed through the geometric calibration $(T,A,r_A)$ and energy exchange). In cases with curvature corrections or non-equilibrium fluids, an intrinsic entropy production term $d_iS$ must be introduced, corresponding to irreversible effects from modified quantum/geometric corrections. This establishes a threefold correspondence: cosmic evolution $=$ geometric entropy production process; Big Bang $=$ phase-transition chain anchored by observations; dark energy $=$ vacuum geometric curvature. The framework rests on the theoretical pillars of Jacobson's ``spacetime thermodynamics,'' Hayward's unified first law, Cai--Kim and Akbar--Cai's derivation of the Friedmann equations from horizon thermodynamics, and Bousso--Engelhardt's cosmological generalized second law (Q-screen form), and unifies the late-time geometric limit via the Gibbons--Hawking de Sitter temperature and horizon entropy calibration.
\end{abstract}

\noindent\textbf{Keywords:} FLRW cosmology; Apparent horizon thermodynamics; Friedmann equations; Geometric entropy production; Generalized second law; Dark energy; Hayward unified first law; Kodama temperature; Non-equilibrium cosmology

\noindent\textbf{MSC 2020:} Primary 83F05, 83C75, 80A10; Secondary 83C05, 94A17

\section{Setup and Notation}

The metric is taken to be FLRW:
$$
ds^2=-dt^2+a(t)^2\bigl((1-\kappa r^2)^{-1}dr^2+r^2 d\Omega^2\bigr),\quad \kappa\in\{-1,0,+1\}.
$$
The Hubble parameter is $H=\dot{a}/a$. The apparent horizon radius is
$$
r_A=(H^2+\kappa/a^2)^{-1/2},
$$
with area $A=4\pi r_A^2$ and volume $V=\tfrac{4}{3}\pi r_A^3$. The surface gravity adopts the Hayward--Kodama dynamical definition; the quasi-static limit temperature is $T=\tfrac{1}{2\pi r_A}$, and the general case $T=|\kappa_{\text{sg}}|/(2\pi)$ is compatible with the dynamical trapping horizon in FRW. The Misner--Sharp energy $E$, energy supply $\psi$, and work density $W=\tfrac{1}{2}(\rho-p)$ enter the unified first law $dE=A\psi+W\,dV$. These objects and definitions are well-defined on spherically symmetric spacetimes, particularly on the apparent horizon in FRW backgrounds, and have been used to reconstruct the Friedmann equations from thermodynamics.

\section{Thermodynamic Reinterpretation of the Friedmann Equations}

\subsection{Horizon First Law $\Longleftrightarrow$ Friedmann Equations}

On the apparent horizon, take the heat flow $\delta Q=A\psi$ and impose the Clausius relation $\delta Q=T\,dS_h$ (where $S_h=\tfrac{A}{4G}$), expressing energy balance via the unified first law $dE=A\psi+W\,dV$. Using energy conservation $\dot{\rho}+3H(\rho+p)=0$, we obtain
$$
\dot{H}-\frac{\kappa}{a^2}=-4\pi G(\rho+p),\qquad H^2+\frac{\kappa}{a^2}=\frac{8\pi G}{3}\rho+\frac{\Lambda}{3},
$$
which are the two Friedmann equations. Conversely, assuming these two equations hold, $\delta Q=T\,dS_h$ holds identically on the apparent horizon, establishing mutual equivalence. This conclusion is rigorous in Einstein gravity and has been extended to Gauss--Bonnet/Lovelock and $f(R)$ theories (non-equilibrium corrections in Section \ref{sec:nonequilibrium}) \cite{cai2005first}.

\subsection{``Entropy Balance Decomposition'' of the First Equation}

Writing $r_A^{-2}=H^2+\kappa/a^2=\tfrac{8\pi G}{3}\rho+\tfrac{\Lambda}{3}$, differentiate with respect to time and use $\dot{\rho}+3H(\rho+p)=-\dot{\Lambda}/(8\pi G)$ (reducing to standard conservation when $\dot{\Lambda}=0$). Combined with $T=(2\pi r_A)^{-1}$ and $S_h=\pi r_A^2/G$, we obtain
$$
T\,\dot{S}_h=4\pi H r_A^3(\rho+p).
\tag{1}
$$
That is, the reversible entropy production on the apparent horizon is completely determined by the matter/radiation enthalpy flux; curvature and $\Lambda$ affect $T\,\dot{S}_h$ only indirectly through the geometric calibration $(r_A,T)$ and do not constitute independent entropy flux terms \cite{cai2005first}.

\section{Threefold Correspondence and Physical Interpretation}

\subsection{Cosmic Evolution = Geometric Entropy Production Process}

The Raychaudhuri equation governs the focusing/divergence evolution of geodesic congruences. Under energy conditions and appropriate boundaries, the monotonicity of apparent/event horizon areas and the cosmological generalized second law (Q-screen form) jointly ensure that the generalized entropy $S_{\text{gen}}$ is non-decreasing along the natural arrow of time. Equation (1) shows that the reversible entropy budget is uniquely determined by the matter/radiation enthalpy flux; curvature and $\Lambda$ affect the evolution of $S_{\text{gen}}$ only indirectly through the geometric calibration $(r_A,T)$ \cite{bousso2016generalized}.

\subsection{Big Bang = Phase-Transition Chain Anchored by Observations}

The electroweak, QCD, recombination/reionization, and BBN stages in thermal history can be viewed as inflection points in the entropy structure corresponding to order-parameter transitions. Using the BBN light-element abundances and CMB acoustic peaks as observational anchors to constrain $\rho(z)$, $p(z)$, and $w(z)$, we can reconstruct the historical trajectory of $\Phi_{\text{m}}(z)=4\pi H r_A^3(\rho+p)$ and test $\dot{S}_{\text{gen}}\ge 0$. Parameter values are anchored by Planck 2018 and PDG reviews \cite{planck2020,pdg2024}.

\subsection{Dark Energy = Vacuum Geometric Curvature}

In the de Sitter limit $H^2=\Lambda/3$, the temperature and entropy are respectively $T_{\text{dS}}=H/(2\pi)$ and $S_{\text{dS}}=A/(4G)=3\pi/(G\Lambda)$, reflecting the reversible setting of the geometric calibration by vacuum curvature. If $\Lambda$ varies slowly with time or contains higher-order curvature corrections, its effect is manifested through changes in $H(t)$ and thus in the quantum/geometric calibration $(T,S_h)$, and may appear as $d_iS$ in non-equilibrium extensions \cite{gibbons1977cosmological}.

\section{Non-Equilibrium Extension and Intrinsic Entropy Production}
\label{sec:nonequilibrium}

When the gravitational entropy functional contains curvature corrections (e.g., the Wald/Noether entropy $S_{\text{Wald}}=\dfrac{A}{4G}f_R$ in $f(R)$ theory, where $f_R\equiv\partial f/\partial R$), the Clausius relation must be modified to $\delta Q=T\,dS_h+T\,d_iS$ (with $d_iS\ge 0$). Eling, Guedens, and Jacobson showed that for polynomial modifications of the Ricci scalar, a non-equilibrium entropy treatment is necessary, and modified field equations can be derived from entropy balance relations. Even in Einstein theory, if horizon shear/bulk viscosity is considered, there exists a positive-definite contribution from $d_iS$. Cosmological viscous fluids (Eckart/Israel--Stewart) provide a macroscopic modeling approach, with non-equilibrium pressure $\Pi$ and $d_iS$ capable of causing additional entropy growth in both early and late epochs \cite{eling2006nonequilibrium}.

\section{Main Theorems}

\begin{theorem}[Equivalence of Horizon First Law and Friedmann Equations]
\label{thm:horizon_friedmann}
In Einstein gravity with FLRW background, the relations $\delta Q=A\psi=T\,dS_h$, $T=\tfrac{1}{2\pi r_A}$, and $S_h=\tfrac{A}{4G}$ on the apparent horizon are mutually equivalent to the two Friedmann equations.
\end{theorem}

\begin{proof}[Proof sketch]
From $r_A=(H^2+\kappa/a^2)^{-1/2}$, we obtain $\dot{r}_A=-H r_A^3(\dot{H}-\kappa/a^2)$. Thus
$$
\dot{S}_h=\frac{d}{dt}\biggl(\frac{\pi r_A^2}{G}\biggr)=\frac{2\pi r_A}{G}\dot{r}_A=-\frac{2\pi H r_A^4}{G}\biggl(\dot{H}-\frac{\kappa}{a^2}\biggr),
$$
whence $T\,\dot{S}_h=-\tfrac{H r_A^3}{G}(\dot{H}-\kappa/a^2)$. The unified first law $dE=A\psi+W\,dV$ with $\delta Q=A\psi$, combined with energy conservation $\dot{\rho}+3H(\rho+p)=0$, and substituting $E=\rho V$, $V=\tfrac{4}{3}\pi r_A^3$, $A=4\pi r_A^2$, and the above expression for $T$, yields
$$
\dot{H}-\frac{\kappa}{a^2}=-4\pi G(\rho+p).
$$
Then from $r_A^{-2}=H^2+\kappa/a^2$ and the energy constraint integral, we obtain
$$
H^2+\frac{\kappa}{a^2}=\frac{8\pi G}{3}\rho+\frac{\Lambda}{3},
$$
completing the sufficiency-necessity proof. See Appendix \ref{app:thm1} for details \cite{akbar2007thermodynamic}.
\end{proof}

\begin{theorem}[Horizon Entropy Budget]
\label{thm:entropy_budget}
Differentiating $r_A^{-2}=H^2+\kappa/a^2$ and using $\dot{\rho}+3H(\rho+p)=-\dot{\Lambda}/(8\pi G)$ (degenerating to standard conservation when $\dot{\Lambda}=0$), we have on the apparent horizon
$$
T\,\dot{S}_h=4\pi H r_A^3(\rho+p).
$$
That is, the reversible entropy production term is completely determined by the enthalpy flux. If higher-order curvature or non-equilibrium fluids are introduced, this should be rewritten as $\delta Q=T\,dS_h+T\,d_iS$, where $d_iS\ge 0$ describes the irreversible contribution from geometric/quantum corrections \cite{cai2005first}.
\end{theorem}

\begin{theorem}[Generalized Second Law and Monotonicity of the ``Phase-Transition Chain'']
\label{thm:gsl}
Along the family of cosmological Q-screens, the generalized entropy $S_{\text{gen}}$ is monotonically non-decreasing. Early thermal-history phase transitions/crossovers appear as non-analytic inflection points in $S_{\text{gen}}$, with their time-scale positions anchored by BBN and CMB observables \cite{bousso2016generalized}.
\end{theorem}

\section{Observational Anchors and Testable Consequences}

\begin{enumerate}
\item \textbf{Late-time geometric calibration:} In the $\Lambda$-dominated era, $T=H/(2\pi)$ and $S_h=3\pi/(G\Lambda)$. If a running vacuum model is considered, a slowly varying $\Lambda(t)$ changes $H(t)$ and thus the quantum/geometric calibration $(T,S_h)$; its irreversible effect (if any) should be manifested in the $d_iS$ term of $\delta Q=T\,dS_h+T\,d_iS$ \cite{gibbons1977cosmological}.

\item \textbf{Enthalpy flux entropy reconstruction:} Using Planck 2018 values for $\Omega_m$, $\Omega_b$, $n_s$, $\tau$ and PDG/latest BBN values for $\eta_B$ as inputs, we can reconstruct $\Phi_{\text{m}}(z)$ as a function of redshift and test $\dot{S}_{\text{gen}}\ge 0$ \cite{planck2020,pdg2024}.

\item \textbf{Curvature modulation effect:} If $|\Omega_k|\neq 0$, curvature modulates the geometric calibration $(T,S_h)$ by changing $r_A=(H^2+\kappa/a^2)^{-1/2}$; its present value is constrained by BAO/CMB measurements of $\Omega_k$. Correspondingly, the influence of curvature on entropy evolution is modulated by the evolution of $a(t)$ \cite{cai2005first}.
\end{enumerate}

\section{Dynamic Consistency and Selection of Temperature}

In the dynamical FRW background, the Hayward--Kodama surface gravity provides a consistent temperature selection, with positive sign for past-inner trapping horizons. This temperature is consistent with the spectral interpretation of the tunneling method and does not depend on the causal nature of the horizon (timelike/spacelike/null). The quasi-static limit $T\simeq(2\pi r_A)^{-1}$ is only a first-order approximation; the rigorous formula contains $\dot{r}_A$ corrections.

\section{Relation to Gravity--Information Unification}

Jacobson viewed the Einstein equations as an equation of state $\delta Q=T\,dS$ on a family of local Rindler horizons. This paper realizes this idea on the cosmological scale via the geometric calibration $(S_h,T)$ of the apparent horizon, providing an entropy decomposition of the Friedmann equations. For higher-order curvature/non-minimal coupling, a non-equilibrium extension with $d_iS$ is needed. In the de Sitter limit, dark energy as vacuum geometric curvature changes the reversible geometric entropy only through the geometric calibration $(T,S_h)$, without producing an independent local entropy flux term (even if $\Lambda$ is time-dependent, its effect is self-consistently absorbed through energy exchange and geometric calibration) \cite{jacobson1995thermodynamics}.

\section{Discussion and Conclusions}

This paper establishes a unified cosmology--thermodynamics framework in which the dynamics of FLRW universes is equivalently reformulated as a geometric entropy production process on the apparent horizon. Our main results include:

\begin{enumerate}
\item The mutual equivalence of the horizon first law and the Friedmann equations, establishing the thermodynamic interpretation of cosmic evolution.

\item The entropy budget identity showing that reversible entropy flux is determined solely by matter/radiation enthalpy flux, with curvature and $\Lambda$ affecting only the geometric calibration $(r_A,T)$.

\item The threefold correspondence: cosmic evolution as geometric entropy production, Big Bang as an observationally anchored phase-transition chain, and dark energy as vacuum geometric curvature.

\item The extension to non-equilibrium scenarios incorporating intrinsic entropy production $d_iS$ for curvature corrections and viscous fluids.

\item Observational anchors via Planck 2018 cosmological parameters and BBN abundances, enabling redshift-dependent reconstruction of entropy evolution and tests of $\dot{S}_{\text{gen}}\ge 0$.
\end{enumerate}

This framework deepens Jacobson's spacetime thermodynamics paradigm and provides a concrete thermodynamic realization on cosmological scales. Future directions include incorporating quantum corrections to the generalized entropy, extending to anisotropic and inhomogeneous cosmologies, and exploring the role of topology change in entropy evolution.

\appendix

\section{Proof Details of Theorem \ref{thm:horizon_friedmann}}
\label{app:thm1}

\textbf{(A.1)} From $r_A=(H^2+\kappa/a^2)^{-1/2}$, we obtain
$$
\dot{r}_A=-H r_A^3\biggl(\dot{H}-\frac{\kappa}{a^2}\biggr).
$$
Thus
$$
\dot{S}_h=\frac{d}{dt}\biggl(\frac{\pi r_A^2}{G}\biggr)=\frac{2\pi r_A}{G}\dot{r}_A=-\frac{2\pi H r_A^4}{G}\biggl(\dot{H}-\frac{\kappa}{a^2}\biggr),
$$
whence
$$
T\,\dot{S}_h=-\frac{H r_A^3}{G}\biggl(\dot{H}-\frac{\kappa}{a^2}\biggr).
$$

\textbf{(A.2)} The unified first law $dE=A\psi+W\,dV$ with $\delta Q=A\psi$, combined with energy conservation $\dot{\rho}+3H(\rho+p)=0$, and substituting $E=\rho V$, $V=\tfrac{4}{3}\pi r_A^3$, $A=4\pi r_A^2$, and the above expression for $T$, yields
$$
\dot{H}-\frac{\kappa}{a^2}=-4\pi G(\rho+p).
$$

\textbf{(A.3)} From $r_A^{-2}=H^2+\kappa/a^2$ and the energy constraint integral, we obtain
$$
H^2+\frac{\kappa}{a^2}=\frac{8\pi G}{3}\rho+\frac{\Lambda}{3},
$$
completing the sufficiency-necessity proof \cite{akbar2007thermodynamic}.

\section{Derivation of Equation (1)}

From $r_A^{-2}=H^2+\kappa/a^2$, we have
$$
-2r_A^{-3}\dot{r}_A=2H\biggl(\dot{H}-\frac{\kappa}{a^2}\biggr).
$$
Multiplying both sides by $-\tfrac{\pi r_A^4}{G}$ and using $S_h=\pi r_A^2/G$ and $T=(2\pi r_A)^{-1}$, we obtain
$$
T\,\dot{S}_h=-\frac{H r_A^3}{G}\biggl(\dot{H}-\frac{\kappa}{a^2}\biggr).
$$
Then using $\dot{H}-\kappa/a^2=-4\pi G(\rho+p)$ (which holds even when $\Lambda(t)$ is time-dependent), we obtain
$$
T\,\dot{S}_h=4\pi H r_A^3(\rho+p).
$$

\section{Non-Equilibrium Entropy Production and Modified Quantum/Geometric Corrections}

When the entropy functional contains curvature corrections, the non-equilibrium entropy balance equation $\delta Q=T\,dS_h+T\,d_iS$ with $d_iS\ge 0$ must be employed. Polynomial modifications of the Ricci scalar require bulk-viscosity-type $d_iS$, whose coefficient is determined through energy-momentum conservation. Even in Einstein theory, horizon shear/bulk viscosity can lead to $d_iS$. These results maintain consistency in deriving modified field equations from thermodynamics \cite{eling2006nonequilibrium}.

\section{Cosmological Generalized Second Law (Q-Screen Form)}

Under backgrounds satisfying the quantum focusing conjecture, the generalized entropy $S_{\text{gen}}$ is monotonically non-decreasing along past/future Q-screens, applicable to cosmological spacetimes without event horizons. This conclusion provides a rigorous foundation for characterizing the thermal-history phase-transition chain as non-analytic structure in $S_{\text{gen}}$ and works in coordination with standard cosmological parameter constraints \cite{bousso2016generalized}.

\bibliographystyle{plain}
\begin{thebibliography}{99}

\bibitem{jacobson1995thermodynamics}
T. Jacobson.
\newblock Thermodynamics of spacetime: The Einstein equation of state.
\newblock {\em Physical Review Letters}, 75(7):1260--1263, 1995.

\bibitem{hayward1997unified}
S. A. Hayward.
\newblock Unified first law of black-hole dynamics and relativistic thermodynamics.
\newblock arXiv preprint gr-qc/9710089, 1997.

\bibitem{cai2005first}
R.-G. Cai and S. P. Kim.
\newblock First law of thermodynamics and Friedmann equations of Friedmann--Robertson--Walker universe.
\newblock {\em Journal of High Energy Physics}, 2005(02):050, 2005.

\bibitem{akbar2007thermodynamic}
M. Akbar and R.-G. Cai.
\newblock Thermodynamic behavior of the Friedmann equation at the apparent horizon of the FRW universe.
\newblock {\em Physical Review D}, 75(8):084003, 2007.

\bibitem{bousso2016generalized}
R. Bousso and N. Engelhardt.
\newblock Generalized second law for cosmology.
\newblock {\em Physical Review D}, 93(2):024025, 2016.

\bibitem{gibbons1977cosmological}
G. W. Gibbons and S. W. Hawking.
\newblock Cosmological event horizons, thermodynamics, and particle creation.
\newblock {\em Physical Review D}, 15(10):2738--2751, 1977.

\bibitem{planck2020}
Planck Collaboration (N. Aghanim et al.).
\newblock Planck 2018 results. VI. Cosmological parameters.
\newblock {\em Astronomy \& Astrophysics}, 641:A6, 2020.

\bibitem{pdg2024}
Particle Data Group (B. D. Fields, P. Molaro, and S. Sarkar).
\newblock Big-bang nucleosynthesis.
\newblock In {\em Review of Particle Physics}, 2024/2025 update.

\bibitem{eling2006nonequilibrium}
C. Eling, R. Guedens, and T. Jacobson.
\newblock Nonequilibrium thermodynamics of spacetime.
\newblock {\em Physical Review Letters}, 96(12):121301, 2006.

\bibitem{kodama1980conserved}
H. Kodama.
\newblock Conserved energy flux for the spherically symmetric system and the backreaction problem in the black hole evaporation.
\newblock {\em Progress of Theoretical Physics}, 63(4):1217--1228, 1980.

\bibitem{hayward1998local}
S. A. Hayward.
\newblock General laws of black-hole dynamics.
\newblock {\em Physical Review D}, 49(12):6467--6474, 1994.

\bibitem{misner1968relativistic}
C. W. Misner and D. H. Sharp.
\newblock Relativistic equations for adiabatic, spherically symmetric gravitational collapse.
\newblock {\em Physical Review}, 136(2B):B571--B576, 1964.

\bibitem{wald1993black}
R. M. Wald.
\newblock Black hole entropy is the Noether charge.
\newblock {\em Physical Review D}, 48(8):R3427--R3431, 1993.

\bibitem{iyer1994properties}
V. Iyer and R. M. Wald.
\newblock Some properties of the Noether charge and a proposal for dynamical black hole entropy.
\newblock {\em Physical Review D}, 50(2):846--864, 1994.

\bibitem{maartens2004dissipative}
R. Maartens.
\newblock Dissipative cosmology.
\newblock {\em Classical and Quantum Gravity}, 12(6):1455--1465, 1995.

\end{thebibliography}

\end{document}

