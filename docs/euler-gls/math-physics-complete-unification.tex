\documentclass[12pt]{article}

% Essential packages
\usepackage[utf8]{inputenc}
\usepackage{amsmath,amssymb,amsthm}
\usepackage{mathrsfs}
\usepackage{geometry}
\usepackage{hyperref}

% Geometry settings
\geometry{a4paper, margin=1in}

% Hyperref settings
\hypersetup{
    colorlinks=true,
    linkcolor=blue,
    citecolor=blue,
    urlcolor=blue
}

% Theorem environments
\theoremstyle{plain}
\newtheorem{theorem}{Theorem}[section]
\newtheorem{lemma}[theorem]{Lemma}
\newtheorem{proposition}[theorem]{Proposition}
\newtheorem{corollary}[theorem]{Corollary}

\theoremstyle{definition}
\newtheorem{definition}{Definition}[section]
\newtheorem{example}[theorem]{Example}
\newtheorem{remark}[theorem]{Remark}

% Title information
\title{Mathematical Structures as Physical Constraints: A Category-Theoretic Unification via Master-Scale Identity}
\author{Haobo Ma$^1$ \and Wenlin Zhang$^2$\\
\small $^1$Independent Researcher\\
\small $^2$National University of Singapore}

\date{\today}

\begin{document}

\maketitle

\begin{abstract}
We propose and prove two unified propositions: first, \textit{mathematical structures} are necessary expressions of \textit{physical constraints}; second, \textit{logical rules} are algebraic characterizations of \textit{causal structures}. The construction path: taking windowed scattering--information geometry--static block observation--reversible logs (WSIG--EBOC--RCA) as the object layer, using the ``master-scale identity'' (MSI) as the observational ruler, establishing a hyperdoctrine composed of realizable predicates and completing it via tripos$\to$topos to obtain internal logic; then characterizing the compatibility of deduction rules and causal partial order via string diagram semantics of dagger symmetric monoidal categories. The core conclusion shows: physical self-consistency satisfying causality, unitarity/complete positivity, energy-spectrum--phase--group-delay co-calibration (MSI), and NPE (Poisson--Euler--Maclaurin--tail) error discipline is equivalent to complete deducibility of some internal logic; and the structural rules of this logic correspond precisely to ``copyable/discardable'' classical wires and quantum channels restricted by resource sensitivity. The paper provides: spectral shift--scattering determinant--group delay co-calibration identity and windowed trace formula; geometrized quantum dynamics on projective Hilbert space; measurement closure via Belavkin filtering and GKSL semigroups; boundary/lens rigidity and metric inversion; integerized calibration via Zeckendorf reversible logs.
\end{abstract}

\noindent\textbf{Keywords:} Mathematics--physics unification; Category theory; Scattering theory; Master-scale identity; Tripos; Topos; Internal logic; Dagger compact categories; CPM construction; Geometrized quantum mechanics; Belavkin filtering; Boundary rigidity; Zeckendorf logs; NPE discipline

\noindent\textbf{MSC 2020:} Primary 18D10, 81U40, 53D50; Secondary 03G30, 81P15, 47A40, 94A17

\section{Notation, Conventions, and Axioms}

\subsection{Notation and Objects}

Observation triple denoted $(\mathcal{H},S(E),\mathsf{K}_{w,h})$. Here $S(E)\in\mathsf{U}(N)$ is differentiable, Wigner--Smith time-delay matrix $\mathsf{Q}(E)=-i\,S(E)^\dagger \partial_E S(E)$, windowed Toeplitz/Berezin compression $\mathsf{K}_{w,h}$ yields readout functional. Relative state density $\rho_{\text{rel}}(E)$ and total phase $\varphi(E)=\tfrac{1}{2}\arg\det S(E)$ co-calibrated via MSI \cite{martin2016wigner}.

\subsection{Two ``Calibration--Error'' Cards}

\begin{definition}[Card I: Calibration Identity (MSI)]
$$
\boxed{\quad \frac{\varphi'(E)}{\pi}\;=\;\rho_{\text{rel}}(E)\;=\;\frac{1}{2\pi}\operatorname{tr}\mathsf{Q}(E)\quad}
$$
This identity is obtained by combining the Birman--Kreĭn formula $\det S(E)=e^{-2\pi i\,\xi(E)}$, Kreĭn--Friedel relation $\xi'(E)=-\rho_{\text{rel}}(E)$, and $\operatorname{tr}\mathsf{Q}(E)=\partial_E\arg\det S(E)$ \cite{birman1962theory}.
\end{definition}

\begin{definition}[Card II: Finite-Order EM and ``Poles = Principal Scales'']
Discrete--continuous bridging employs Poisson summation and finite-order Euler--Maclaurin expansion; remainder uniformly constrained by window decay and master-scale regularity, following the ``singularity non-increase, poles give principal scales'' discipline, forming a fully auditable error ledger.
\end{definition}

\subsection{Testable Axioms}

\begin{description}
\item[\textbf{A0 (Causal--Analytic):}] Time-domain causality $\Rightarrow$ frequency-domain upper half-plane analyticity, mutually determined with Hilbert transform (Kramers--Kronig); convolution support satisfies Titchmarsh endpoint additivity.

\item[\textbf{A1 (Channel Layer):}] Dynamical morphisms are unitary or completely positive trace-preserving (CPTP), closed by Selinger's CPM construction; classical wires realized by copyable/deletable Frobenius structure \cite{selinger2007dagger}.

\item[\textbf{A2 (Observation Layer):}] Effects form effect algebra fibers; predicate transformers have left/right adjoints (existential/universal quantifiers), thus forming a hyperdoctrine in Lawvere's sense.

\item[\textbf{A3 (Ruler Layer):}] MSI and windowed BK trace formula hold (Helffer--Sjöstrand functional calculus ensures windowed measurability) \cite{duchene2015helffer}.

\item[\textbf{A4 (Numerical Layer):}] NPE error discipline: Poisson suppresses aliasing, EM endpoint correction and remainder bounds, tail integrability.
\end{description}

\section{Unified Propositions and Main Theorems}

\subsection{Proposition U1 (Mathematical Structures = Necessity of Physical Constraints)}

\begin{proposition}[U1]
\label{prop:u1}
Let the physical world form a full subcategory $\mathbf{U}$ satisfying A0--A4. Taking each object $X$'s effect algebra $\mathsf{Eff}(X)$ as predicate fiber, morphism pullback $f^*: \mathsf{Eff}(Y)\to\mathsf{Eff}(X)$ with adjoints $\Sigma_f\dashv f^*\dashv \Pi_f$ forms hyperdoctrine $\mathscr{P}\to \mathbf{U}^{\text{op}}$. Applying the Hyland--Johnstone--Pitts tripos$\to$topos construction yields minimal topos $\mathfrak{T}$ such that observable truth values are stable in internal logic; thus any axiomatic system $\mathrm{T}$ claiming ``truth-preservation'' is equivalent to some conservative extension of $\mathfrak{T}$ \cite{hyland2002tripos}.
\end{proposition}

\subsection{Proposition U2 (Logical Rules = Algebra of Causal Structure)}

\begin{proposition}[U2]
\label{prop:u2}
On dagger symmetric monoidal category $\mathbf{Proc}$ generated by systems/processes, string diagram semantics satisfies: cut and implication elimination correspond to sequential composition; exchange/associativity correspond to monoidal constraints; contraction/weakening open only for ``copyable--deletable'' classical wires (no-cloning theorem restricts general objects), thus overall logic is linear--resource logic; CPM construction ensures ``pure--mixed'' closure \cite{selinger2007dagger}.
\end{proposition}

\subsection{Main Theorem (Mutual Interpretation)}

\begin{theorem}[Physical Self-Consistency $\Leftrightarrow$ Deducibility]
\label{thm:main}
$$
\text{Physical self-consistency}~(\text{A0--A4}) \;\Longleftrightarrow\; \text{Deducible}~(\text{Internal logic }{\rm Th}(\mathfrak{T})\ \&\ \text{Linear type theory})
$$
Left direction given by correctness and conservativity of hyperdoctrine--topos; right direction guaranteed by graphical completeness of dagger compact categories and windowed measurability \cite{selinger2007dagger}.
\end{theorem}

\section{Master-Scale Identity and Windowed Trace Formula}

\subsection{Birman--Kreĭn--Kreĭn--Friedel--Wigner--Smith}

Spectral shift function $\xi(E)$ satisfies $\det S(E)=e^{-2\pi i\xi(E)}$ and $\xi'(E)=-\rho_{\text{rel}}(E)$, while $\operatorname{tr}\mathsf{Q}(E)=\partial_E\arg\det S(E)$, thus yielding MSI \cite{birman1962theory}.

\subsection{Helffer--Sjöstrand and Windowed BK}

For quasi-analytic extension of $f$, we have
$$
\mathrm{Tr}\big(f(H)-f(H_0)\big)=-\frac{1}{2\pi i}\!\int f'(E)\,\log\det S(E)\,dE,
$$
holding under windowing $f=h\!\star\!w_R$, thus MSI is measurable at windowed trace level \cite{duchene2015helffer}.

\subsection{Causal--Analytic Consistency}

Kramers--Kronig and Titchmarsh theorems rigorously connect time-domain causality with frequency-domain analyticity and convolution support rules, ensuring ``arrival front--phase response'' consistency.

\section{From Physics to Mathematics: Hyperdoctrine and Tripos$\to$Topos}

\subsection{Effect Fibers and Adjoints}

Each object $X$'s effect algebra $\mathsf{Eff}(X)$ forms a partially ordered structure; Heisenberg pullback $f^*$ combined with conditional expectation yields $\Sigma_f\dashv f^*\dashv \Pi_f$, satisfying Beck--Chevalley and Frobenius properties.

\subsection{Free Completion via Tripos$\to$Topos}

The Hyland--Johnstone--Pitts construction realizes predicate values as subobjects, obtaining $\mathfrak{T}$'s internal intuitionistic logic and quantifier structure; $\mathfrak{T}$ is conservative over windowed truth values, presenting the freeness of ``mathematical structure $=$ closure of physical constraints'' \cite{hyland2002tripos}.

\section{From Causality to Logic: String Diagram Deduction and Linear Resources}

\subsection{Process Category and CPM Closure}

Objects are systems, morphisms are unitary/completely positive processes; CPM internalizes completely positive maps within dagger compact categories, thus graphical equations are complete for $\mathbf{FHilb}$ \cite{selinger2007dagger}.

\subsection{Structural Rules and No-Cloning}

Classical wires realize contraction/weakening via copyable/deletable Frobenius structure; general quantum objects obey no-cloning theorem, thus global logic is linear logic with exponential $!/\,?$ marking copyable resources.

\section{Geometrized Quantum Dynamics}

Projective Hilbert space $\mathbb{P}(\mathcal{H})$ is a Kähler manifold; holonomy of connection one-form $\mathcal{A}=-i\langle\psi|d\psi\rangle$ yields geometric phase; Schrödinger evolution is equivalent to Hamiltonian flow on projective space. This geometric structure is homologous to total derivative of scattering phase $\varphi'(E)$, thus unifying MSI with Berry phase in projective geometry \cite{kibble1979geometrization}.

\section{Measurement Closure and Open Systems}

Continuous monitoring via Belavkin filtering (quantum Itô--QSDE) yields posterior state; averaging obtains master equation with GKSL generator, forming ``recording--dynamics'' closed-loop consistency; Jarzynski equality with mutual-information correction incorporates information readout into energy--work balance \cite{bouten2007introduction}.

\section{Geometric Inversion and Rigidity}

Under simple/near-simple conditions and convexity assumptions, boundary distances/lens data determine metric (up to natural equivalence); Stefanov--Uhlmann--Vasy establish local--global boundary rigidity under ``normal gauge'' and bridge with tensor X-ray transform, providing constructive path for inverting $g$ from travel-time/scattering data \cite{stefanov2021boundary}.

\section{Discretization Calibration and Zeckendorf Reversible Logs}

Sliding-window load $\int_W \rho_{\text{rel}}\,dE$ integerized via unique Zeckendorf decomposition; carry/borrow rules are locally reversible, compatible with RCA logs; their statistics and generalization ($f$-decomposition) yield robust computable encodings and error propagation bounds.

\section{NPE Error Theory and Halting Criteria}

Poisson summation suppresses aliasing; finite-order Euler--Maclaurin gives endpoint corrections and remainder bounds; tail controlled by window decay and master-scale regularity, forming uniform halting criterion at $O(R^{-p})$ scale, compatible with scale--energy co-calibrated numerical pipeline.

\appendix

\section{Proof Chain for MSI and Windowed BK}

\subsection{Spectral Shift--Determinant}

For traceable perturbation pair $(H,H_0)$, Birman--Kreĭn formula gives $\det S(E)=e^{-2\pi i\xi(E)}$ \cite{birman1962theory}.

\subsection{Kreĭn--Friedel Relation}

Relative state density satisfies $\xi'(E)=-\rho_{\text{rel}}(E)$ (see Friedel trace formula), thus $\varphi'(E)/\pi=\rho_{\text{rel}}(E)$.

\subsection{Group Delay--Phase Derivative}

$\mathsf{Q}(E)=-i\,S^\dagger \partial_E S$, thus $\operatorname{tr}\mathsf{Q}(E)=\partial_E\arg\det S(E)$; combining yields MSI \cite{martin2016wigner}.

\subsection{Windowed Trace Formula}

Helffer--Sjöstrand functional calculus holds for $f=h\!\star\!w_R$:
$$
\mathrm{Tr}\big(f(H)-f(H_0)\big)=-\frac{1}{2\pi i}\!\int f'(E)\,\log\det S(E)\,dE,
$$
thus MSI directly falls on measurable window functions \cite{duchene2015helffer}.

\section{Causal--Analytic Prior}

\subsection{Kramers--Kronig}

Upper half-plane analyticity and stability imply real--imaginary part Hilbert duality, equivalent to linear causal response.

\subsection{Titchmarsh Convolution Theorem}

Support endpoint additivity yields strict ``arrival front'' rule, ensuring time--frequency consistency.

\section{Hyperdoctrine and Tripos$\to$Topos}

\subsection{Effect Fibers and Adjoints}

$\mathscr{P}(X)=\mathsf{Eff}(X)$, $f^*$ is Heisenberg pullback; conditional expectation yields existence domains of $\Sigma_f,\Pi_f$ and Beck--Chevalley property.

\subsection{Tripos$\to$Topos}

Following Hyland--Johnstone--Pitts construction, realize predicates as subobjects, obtaining topos $\mathfrak{T}$; its internal logic is conservative over observable truth values \cite{hyland2002tripos}.

\section{Completeness of Logical Rules and Causal Algebra}

\subsection{Categorical Semantics}

Sequential $=$ composition, parallel $=$ tensor; units and symmetry yield graphical proof equality for exchange/associativity laws \cite{selinger2007dagger}.

\subsection{CPM Construction and Graphical Completeness}

CPM incorporates completely positive maps into graphical calculus; completeness of equational reasoning for $\mathbf{FHilb}$ holds \cite{selinger2007dagger}.

\subsection{Structural Rules and No-Cloning}

No-cloning theorem forbids contraction/weakening on general objects; classical wires realize copy/delete via Frobenius comonoid, yielding linear--resource logic.

\section{Geometrized Quantum Mechanics}

Symplectic--Riemannian--complex compatible structure on $\mathbb{P}(\mathcal{H})$ makes Schrödinger evolution a Hamiltonian flow; geometric phase is connection holonomy; connects with scattering phase at $\varphi'(E)$ co-calibration \cite{kibble1979geometrization}.

\section{Measurement Closure and Open Systems}

Belavkin filtering yields posterior QSDE; averaging recovers Lindblad generator, forming recording--dynamics closed loop; mutual-information-corrected Jarzynski equality provides unified information--work accounting \cite{bouten2007introduction}.

\section{Geometric Inversion and Boundary/Lens Rigidity}

Under normal gauge, boundary distances or lens data determine metric up to natural equivalence; ellipticity and stability of related X-ray transform yield inversion and stability estimates \cite{stefanov2021boundary}.

\section{NPE Error and Halting Criteria}

Poisson summation--sampling--periodization are mutually equivalent; finite-order EM gives endpoint corrections and remainder bounds; tail controlled by window decay and master-scale regularity, forming $O(R^{-p})$ uniform halting criterion.

\section{Zeckendorf Logs and Integerization}

Any positive integer has unique non-adjacent Fibonacci decomposition; integerization of sliding-window load via Zeckendorf representation realizes locally reversible carry/borrow updates, compatible with reversible logs and encoding.

\section*{Acknowledgments}

All results cited rely on standard theorems in category theory, scattering theory, functional analysis, and quantum information. References are provided for verifiability.

\bibliographystyle{plain}
\begin{thebibliography}{99}

\bibitem{hyland2002tripos}
J. M. E. Hyland, P. T. Johnstone, and A. M. Pitts.
\newblock Tripos theory.
\newblock {\em Mathematical Proceedings of the Cambridge Philosophical Society}, 88(2):205--232, 1980.

\bibitem{selinger2007dagger}
P. Selinger.
\newblock Dagger compact closed categories and completely positive maps.
\newblock {\em Electronic Notes in Theoretical Computer Science}, 170:139--163, 2007.

\bibitem{martin2016wigner}
P. A. Martin.
\newblock Wigner time delay and related concepts: Application to transport in coherent conductors.
\newblock arXiv:1507.00075, 2015.

\bibitem{birman1962theory}
M. Sh. Birman and M. G. Kreĭn.
\newblock On the theory of wave operators and scattering operators.
\newblock {\em Doklady Akademii Nauk SSSR}, 144(3):475--478, 1962.

\bibitem{duchene2015helffer}
V. Duchêne, J. L. Marzuola, and M. I. Weinstein.
\newblock Wave operator bounds for one-dimensional Schrödinger operators with singular potentials and applications.
\newblock arXiv:1506.04537, 2015.

\bibitem{kibble1979geometrization}
T. W. B. Kibble.
\newblock Geometrization of quantum mechanics.
\newblock {\em Communications in Mathematical Physics}, 65(2):189--201, 1979.

\bibitem{bouten2007introduction}
L. Bouten, R. van Handel, and M. R. James.
\newblock An introduction to quantum filtering.
\newblock {\em SIAM Journal on Control and Optimization}, 46(6):2199--2241, 2007.

\bibitem{stefanov2021boundary}
P. Stefanov, G. Uhlmann, and A. Vasy.
\newblock Boundary rigidity with partial data.
\newblock {\em Journal of the American Mathematical Society}, 29(2):299--332, 2016.

\bibitem{lawvere1970equality}
F. W. Lawvere.
\newblock Equality in hyperdoctrines and comprehension schema as an adjoint functor.
\newblock In {\em Applications of Categorical Algebra}, pages 1--14. American Mathematical Society, 1970.

\bibitem{abramsky2004categorical}
S. Abramsky and B. Coecke.
\newblock A categorical semantics of quantum protocols.
\newblock In {\em Proceedings of the 19th IEEE Symposium on Logic in Computer Science}, pages 415--425, 2004.

\bibitem{coecke2011interacting}
B. Coecke and A. Kissinger.
\newblock The compositional structure of multipartite quantum entanglement.
\newblock In {\em Automata, Languages and Programming}, pages 297--308. Springer, 2010.

\bibitem{wootters1982single}
W. K. Wootters and W. H. Zurek.
\newblock A single quantum cannot be cloned.
\newblock {\em Nature}, 299(5886):802--803, 1982.

\bibitem{berry1984quantal}
M. V. Berry.
\newblock Quantal phase factors accompanying adiabatic changes.
\newblock {\em Proceedings of the Royal Society of London A}, 392(1802):45--57, 1984.

\bibitem{zeckendorf1972representation}
E. Zeckendorf.
\newblock Représentation des nombres naturels par une somme de nombres de Fibonacci ou de nombres de Lucas.
\newblock {\em Bulletin de la Société Royale des Sciences de Liège}, 41:179--182, 1972.

\end{thebibliography}

\end{document}

