\documentclass[12pt]{article}

% Essential packages
\usepackage[utf8]{inputenc}
\usepackage{amsmath,amssymb,amsthm}
\usepackage{mathrsfs}
\usepackage{geometry}
\usepackage{hyperref}
\usepackage{slashed}

% Geometry settings
\geometry{a4paper, margin=1in}

% Hyperref settings
\hypersetup{
    colorlinks=true,
    linkcolor=blue,
    citecolor=blue,
    urlcolor=blue
}

% Theorem environments
\theoremstyle{plain}
\newtheorem{theorem}{Theorem}[section]
\newtheorem{lemma}[theorem]{Lemma}
\newtheorem{proposition}[theorem]{Proposition}
\newtheorem{corollary}[theorem]{Corollary}

\theoremstyle{definition}
\newtheorem{definition}[theorem]{Definition}
\newtheorem{example}[theorem]{Example}
\newtheorem{remark}[theorem]{Remark}
\newtheorem{axiom}{Axiom}

% Title information
\title{Information Geometry as Unifying Foundation: Matter as Excitations, Gauge Fields as Connections, and Gravity from Entropy}
\author{Haobo Ma$^1$ \and Wenlin Zhang$^2$\\
\small $^1$Independent Researcher\\
\small $^2$National University of Singapore}

\date{\today}

\begin{document}

\maketitle

\begin{abstract}
We establish a unified framework in which \textbf{matter fields are information-geometric excitation patterns}, \textbf{gauge fields are connections and curvatures on information bundles}, and \textbf{gravitational field equations emerge from relative entropy extremization}. Core structures include: (1) bi-layered geometry given by Fisher metric characterized by Chentsov uniqueness and quantum geometric tensor, providing fundamental metric--symplectic structure on state space; (2) reformulation of \textbf{fermionic fields} via \textbf{Kähler--Dirac operator} acting on exterior forms: on \textbf{Kähler/flat backgrounds with $\mathrm{Spin}^c$ structure}, spectrally equivalent (with multiplicity) to \textbf{multiple copies} of spinor Dirac fields; while \textbf{bosonic gauge fields} are \textbf{connection 1-forms} on principal bundles, \textbf{independent} of Kähler--Dirac fields; (3) local encoding freedom as ``information code frames'' forming principal bundle with structure group $G$, whose local re-gauging generates \textbf{SU(3)$\times$SU(2)$\times$U(1)} gauge symmetry; Higgs field as \textbf{section} of quotient bundle $P/H$, corresponding to structure group \textbf{reduction}; (4) Einstein equations derived from relative entropy ``first law'' and small-ball extremization, with second variation yielding QNEC-type inequalities; (5) unification of measurable readouts with spectrum--phase--time-delay via windowed scattering--master-scale trinity $\big(\varphi'/\pi=\rho_{\text{rel}}=(2\pi)^{-1}\mathrm{tr}\,Q\big)$; (6) multi-scale resonances organized by Mellin logarithmic scale with tight/frame bounds, viewing discrete information patterns as \textbf{resonance clusters}; (7) measurement--feedback--entropy-production closure via Belavkin filtering and mutual-information-generalized Jarzynski equality. Rigorous theorems and proofs are provided at the end, with one-to-one correspondence to existing mathematical/physical literature indicated. Relevant classical results and recent reviews can be found in literature on information geometry and quantum geometric tensor, Hilbert/Hardy and Kramers--Kronig, causal reconstruction, spectral shift--scattering, Bayesian/filtering and fluctuation relations.
\end{abstract}

\noindent\textbf{Keywords:} Information geometry; Fisher metric; Quantum geometric tensor; Berry/Uhlmann geometry; Kähler--Dirac field; Principal bundle connections; Spontaneous symmetry breaking; SU(3)$\times$SU(2)$\times$U(1); Wigner--Smith time delay; Birman--Kreĭn/spectral shift; Mellin tight frames; Zeckendorf encoding; Belavkin filtering; Jarzynski--Sagawa--Ueda

\noindent\textbf{MSC 2020:} Primary 83C05, 81T13, 81T70; Secondary 53C80, 94A17, 42C40

\tableofcontents

\section{Axiomatic Setup and Master Scale}

\begin{axiom}[Information State Space]
\label{ax:1}
Observable systems are parameterized by families of classical/quantum statistical models $\{\mathcal{P}_\theta\}$ or states $\{\rho_\theta\}$, where parameter $\theta$ belongs to differentiable manifold $\mathfrak{M}$. $\mathfrak{M}$ is equipped with Chentsov monotone metric (classical) or Petz monotone family (quantum); on pure states reduces to Fubini--Study metric and Berry curvature unified as \textbf{quantum geometric tensor} $\displaystyle \mathcal{Q}=g-\tfrac{i}{2}F$ ($F$ being Berry curvature 2-form). Classical counterpart is Fisher metric uniqueness \cite{amari2016information}.
\end{axiom}

\begin{axiom}[Master Scale]
\label{ax:2}
Consider scattering matrix $S(E)$ on energy axis; Wigner--Smith time-delay matrix $Q(E)=-i\,S^\dagger \tfrac{dS}{dE}$. Define master-scale density
$$
\boxed{\,\rho_{\text{rel}}(E)=\dfrac{1}{2\pi}\operatorname{tr}Q(E)=\dfrac{\varphi'(E)}{\pi},}\qquad \varphi(E)=\tfrac{1}{2}\arg\det S(E),
$$
equivalent to Birman--Kreĭn/spectral shift and Friedel sum rule phase derivative. Unify windowed readouts as integrals against spectral shift measure: $d\mu_\varphi=-d\xi(E)$, equivalent to density representation $d\mu_\varphi=\rho_{\text{rel}}\,dE$ as \textbf{signed} measure \cite{cunden2015statistical}.
\end{axiom}

\begin{axiom}[Causality and Analyticity]
\label{ax:3}
Frequency-domain response functions of causal kernels are analytic in upper half-plane, with real/imaginary parts coupled by Hilbert transform (Kramers--Kronig); convolution earliest arrival given by Titchmarsh theorem boundary sharpness \cite{kramers1927diffusion}.
\end{axiom}

\begin{axiom}[Observable Compression]
\label{ax:4}
Toeplitz/Berezin compression of window--kernel pairs $(w,h)$ organizes observable readouts as positive linear functionals on \textbf{positive spectral measures}; when measuring with $d\mu_\varphi$, only assert bounded-variation \textbf{linear} functionals, not ensuring pointwise positivity. Numerical error obeys Poisson--Euler--Maclaurin bounded decomposition. Background in function theory--operator theory given by de Branges spaces and Hardy space materials \cite{debranges1968hilbert}.
\end{axiom}

To ensure positivity in subsequent $L^2$ space and frame bound statements, wherever $L^2$ (or inner product) formulations appear, define via total variation positive measure $|d\mu_\varphi|$ (equivalently $|\rho_{\text{rel}}|dE$).

\section{Information Geometry: Fisher and Quantum Geometric Tensor}

\subsection{Fisher Metric Uniqueness and Monotonicity}

On classical statistical manifolds, Fisher metric is the \textbf{unique} Riemannian metric (up to constant factor) invariant under sufficient statistics (or contractive under Markov maps). This provides canonical benchmark for ``measure--inference--geometry,'' and in quantum generalization is carried by Petz monotone metric family, with pure-state limit equal to Fubini--Study \cite{petz1996monotone}.

\subsection{Quantum Geometric Tensor and Material--Field Theory Connection}

Quantum geometric tensor $\mathcal{Q}_{\mu\nu}=\langle\partial_\mu\psi|(1-|\psi\rangle\langle\psi|)|\partial_\nu\psi\rangle$ has real part giving quantum metric (Bures/FS), imaginary part satisfying $\mathrm{Im}\,\mathcal{Q}=-\tfrac{1}{2}F$ (equivalently $\mathcal{Q}=g-\tfrac{i}{2}F$), extensively controlling superfluid weight, optical response, and topological transport. This framework takes $(g,F)$ as intrinsic \textbf{information--phase} field, subsequently coupled with gauge connections \cite{provost1980riemannian}.

\section{Matter Fields as Information Excitations: Kähler--Dirac Unification}

\subsection{Kähler--Dirac Operator and Unified Field Meaning}

On information manifold $(\mathfrak{M},g)$ with metric, introduce exterior differential $d$ and its adjoint $d^\dagger$, defining Kähler--Dirac operator
$$
\mathcal{D}:=d+d^\dagger,\qquad \mathcal{D}^2=\Delta.
$$
$\mathcal{D}$ acts on direct sum of indefinite-degree differential forms $\Omega^\bullet(\mathfrak{M})$. \textbf{Dirac--Kähler field is merely exterior-form reformulation of fermionic degrees of freedom} (Grassmann-valued); even/odd degrees are just component decomposition, \textbf{not} equivalent to physical unification of ``bosons/fermions.'' \textbf{On four-dimensional Kähler/flat backgrounds with $\mathrm{Spin}^c$ structure}, $\mathcal{D}$ is spectrally equivalent to \textbf{multiple copies} of spinor Dirac fields (in four dimensions, \textbf{4} copies); on general curvature backgrounds, $\mathcal{D}$ is merely Dirac-type and \textbf{not} necessarily spectrally equivalent. \textbf{Gauge potential} $\mathcal{A}$ is \textbf{1-form (bosonic field)} of principal bundle connection, independently characterized by Yang--Mills dynamics, coupled to Dirac--Kähler field through covariant derivative. On lattice discretization, equivalent to \textbf{staggered (Kogut--Susskind)} discretization \cite{rabin1982introduction}.

\subsection{Relation Between Kähler--Dirac and Spinor Dirac}

\begin{proposition}[Kähler--Dirac and Spinor Dirac Relation]
\label{prop:kd-relation}
On four-dimensional \textbf{Kähler manifolds} with $\mathrm{Spin}^c$ structure (or flat background), Kähler--Dirac equation $\mathcal{D}\Psi=0$ on $\Omega^\bullet(\mathfrak{M})$ is spectrally equivalent (up to finite multiplicity) to multiple copies of standard spinor Dirac equations; on general curvature backgrounds, $\mathcal{D}=d+d^\dagger$ is Dirac-type Hodge--de Rham operator acting on exterior forms, typically \textbf{not} spectrally equivalent to spinor Dirac operator.
\end{proposition}

\begin{proof}[Proof Sketch]
In Kähler case with $\mathrm{Spin}^c$ identification $S\simeq\Lambda^{0,\bullet}$, have $D_{\mathrm{Spin}^c}=\sqrt{2}(\bar{\partial}+\bar{\partial}^\dagger)$; flat case reduces to equivalence between Kähler--Dirac and multiple Dirac copies; general manifold only has Dirac-type isomorphism without spectral equivalence. Detailed construction in Appendix \ref{app:kd-proof} \cite{bodwin1988equivalence}.
\end{proof}

\section{Gauge Fields as Principal Bundle Connections: Information Origin of SU(3)$\times$SU(2)$\times$U(1)}

\subsection{Information Code Frames and Principal Bundles}

Abstract distinguishable \textbf{local representation freedom} as ``information code frames'' (feature frames). Their totality forms principal bundle $P\to \mathfrak{X}$ with structure group $G$ ($\mathfrak{X}$ being spacetime or effective base manifold). \textbf{Gauge field} is Ehresmann connection $\mathcal{A}$ on $P$, curvature $\mathcal{F}$ is field strength. Principal bundle--connection--Higgs geometric framework detailed in standard literature \cite{nakahara2003geometry}.

\subsection{From Information Symmetry to Standard Model Group}

Assume local information code frame at each point tensor-factorizes via \textbf{color}, \textbf{weak isospin}, and \textbf{overall phase} into three parts
$$
\mathbb{C}^3\otimes \mathbb{C}^2\otimes \mathbb{C}.
$$
Local transformation group preserving Fisher/FS metric and Berry curvature is
$$
U(3)\times U(2)\times U(1).
$$
Modding out overall phase and one redundant $U(1)$ yields physical
$$
G_{\text{SM}}=SU(3)\times SU(2)\times U(1)_Y,
$$
thus interpreting $G_{\text{SM}}$ as \textbf{local re-gauging} symmetry of ``information code frames.'' Distinguishing ``readable'' from ``invisible re-gauging,'' gauge fields appear as principal bundle connections. This construction is fully isomorphic to traditional ``connection on principal bundle $=$ gauge potential,'' ``curvature $=$ field strength'' geometric formulation \cite{nakahara2003geometry}.

\subsection{Spontaneous Symmetry Breaking = Structure Group Reduction}

Higgs field characterized as \textbf{global section} of quotient bundle $P/H$, whose existence is equivalent to \textbf{reduction} $G\to H$ of structure group $G$. Corresponding geometric theorem: $P/H\to \mathfrak{X}$ admits section if and only if reduction holds. Mass terms and Yukawa couplings geometrized as coupling to this section \cite{sardanashvily2005geometry}.

\section{Higgs as Reduction Section and Mass Generation}

\begin{theorem}[Geometric Characterization of Higgs]
\label{thm:higgs}
For principal bundle $P(\mathfrak{X},G)$, if there exists closed subgroup $H\subset G$ such that quotient bundle $P/H\to \mathfrak{X}$ has global section $h$, then there exists reduced subbundle $P_h(\mathfrak{X},H)$. Matter fields then appear as sections of associated bundles of $P_h$; gauge-invariant Lagrangian factorizes through vertical covariant derivative $P\to P/H$ \cite{sardanashvily2005geometry}.
\end{theorem}

\section{Dynamics: IGVP Derives Gravity; Yang--Mills and Matter Coupling}

\subsection{Information-Geometric Variational Principle (IGVP) and Gravitational Equations}

Generalized entropy on small ball $B_\ell(x)$
$$
S_{\text{gen}}(B_\ell)=\dfrac{A(\partial B_\ell)}{4G}+S_{\text{out}}(\rho_{B_\ell})
$$
with first-order extremization and second-order non-negativity yields
$$
G_{\mu\nu}+\Lambda g_{\mu\nu}=8\pi G\,\langle T_{\mu\nu}\rangle
$$
and family of quantum energy conditions/caustic inequalities. The above ``small-ball extremization + relative entropy first law'' rigorously derives the \textbf{semiclassical/linearized} Einstein equations, with second variation yielding QNEC-type inequalities; concerning \textbf{fully nonlinear} equations, although there exist attempts centered on relative entropy/modular Hamiltonian, additional structural premises (e.g., holographic setup) are required; this paper \textbf{makes no claim of universal validity} \cite{jacobson2016entanglement}.

\subsection{Geometric--Information Rendering of Gauge--Matter Action}

Pair symplectic part of quantum geometric tensor with principal bundle curvature, taking standard Yang--Mills type action
$$
\mathcal{S}_{\text{YM}}=\int \tfrac{1}{2g^2}\,\mathrm{tr}(\mathcal{F}\wedge\star \mathcal{F}),
$$
matter fields given by Kähler--Dirac/Dirac action $\int \bar{\Psi}(i\slashed{D}-m(h))\Psi$ coupled to Higgs section $h$ generating mass. This action together with gravitational term $\tfrac{1}{16\pi G}\int R\sqrt{-g}\,d^4x$ forms unified Lagrangian.

\section{Measurability and Scattering: Windowed Readout, Wigner--Smith and Spectral Shift}

\subsection{BK/Spectral Shift--Phase--Group Delay}

Scattering matrix $S(E)$ determinant phase and spectral shift function $\xi(E)$ satisfy Birman--Kreĭn formula
$$
\det S(E)=e^{-2\pi i\,\xi(E)}.
$$
Friedel total charge/density of states correction equivalent to sum of phase shifts; Wigner--Smith $Q(E)=-iS^\dagger \tfrac{dS}{dE}$ trace gives relative state density. Windowed readouts (Toeplitz/Berezin compression) naturally align experimental sampling with above master scale \cite{pushnitski2002spectral}.

\subsection{Causality--Analyticity Constraints and KK Relations}

Frequency-domain analyticity of causal kernels implies Kramers--Kronig (Hilbert transform) relations; Titchmarsh convolution support theorem gives strict addition for ``earliest arrival.'' These provide mathematical basis for criterion that ``group delay can be negative without violating causality'' \cite{kramers1927diffusion}.

\section{Multi-Scale Resonances: Mellin Tight Frames and Discrete Information Patterns}

Mellin transform diagonalizes \textbf{scale} and naturally adapts to self-similar/fractal spectra. Construct logarithmic-scale wavelet packet family satisfying frame bounds $A,B$, realizing stable decomposition of $L^2\big(|d\mu_\varphi|\big)$ (equivalently $L^2(|\rho_{\text{rel}}|dE)$), thus concretizing ``particle physics $=$ resonances of discrete information patterns'' as \textbf{frame coefficient energy concentration} stable features. Related tight/tunable tight frame construction and error estimation provided by existing literature with mature constant estimates and construction algorithms \cite{daubechies2003framelets}.

\section{Measurement--Feedback--Entropy Production: Belavkin and Jarzynski--Sagawa--Ueda}

Davies--Lewis instruments and Naimark extension fully realize POVMs as compressions of PVMs in enlarged Hilbert space; continuous monitoring via Belavkin equation yields quantum filtering (diffusion/counting-type QSDE) for posterior states. Under feedback, \textbf{mutual-information-corrected} Jarzynski equality
$$
\big\langle e^{-\beta W+\beta\Delta F-I}\big\rangle=1
$$
holds, yielding exact ``information--work'' balance \cite{sagawa2008secondlaw}.

\section{Standard Model Representations and Anomaly Constraints: Geometric--Information Reproduction}

\subsection{Geometric Origin of Representations and Charges}

Take fermions as sections of appropriate associated vector bundles of $G_{\text{SM}}$; left-handed doublets from representation on $\mathbb{C}^2$ factor with chiral connection, right-handed singlets from $U(1)$ representation on $\mathbb{C}$ factor. Electric charge $Q=T_3+\tfrac{Y}{2}$ comes from sum of Cartan of $SU(2)$ and $U(1)_Y$ weight \cite{baez1992introduction}.

\subsection{Anomaly Constraints (Schematic)}

For one fermion generation, satisfy
$$
\sum Y=0,\quad \sum Y^3=0,\quad 
\sum_{R\in \text{rep}(SU(2))} T_2(R)\,Y=0,\quad 
\sum_{R\in \text{rep}(SU(3))} T_3(R)\,Y=0,
$$
respectively corresponding to $[\text{grav}]^2U(1)_Y$, $[U(1)_Y]^3$, $[SU(2)]^2U(1)_Y$, $[SU(3)]^2U(1)_Y$ anomaly cancellation. In information-geometric semantics, these equations correspond to balance condition of ``volume-averaged geodesic deviation of code-frame connection equaling zero''; equivalent to textbook derivation \cite{baez1992introduction}.

\section{Discussion and Testable Predictions}

\begin{enumerate}
\item \textbf{Geometric Unification:} Gravity from information entropy extremization, gauge fields as connections of information code frames; \textbf{fermionic fields} characterized by \textbf{Kähler--Dirac (exterior form reformulation)}, \textbf{bosonic gauge fields} as \textbf{independent} 1-forms of principal bundle connections.

\item \textbf{Experimental--Computational Interface:} Windowed group delay, spectral shift, and KK relations provide universal readout calibration for direct measurement in optical/electrical/acoustic scattering; quantum geometric tensor measurable in solid-state/superconducting quantum circuits \cite{hauke2016measuring}.

\item \textbf{Falsifiability:} If local information code frame invariance deviates from $G_{\text{SM}}$, will manifest as response inconsistent with Yang--Mills coupling in low-energy measurements of Berry curvature and quantum metric.
\end{enumerate}

\appendix

\section{Information Geometry and IGVP: Einstein Equations (Proof)}
\label{app:igvp}

\begin{theorem}[Small-Ball Extremization $\Rightarrow$ Linearized Einstein Equations]
At arbitrary point $x$, take sufficiently small geodesic ball $B_\ell(x)$. If
$$
\delta S_{\text{gen}}(B_\ell)=0,\qquad \delta^2 S_{\text{gen}}(B_\ell)\ge 0,
$$
holds for \textbf{first-order} perturbations of local vacuum state, and assuming relative entropy first law $\delta S_{\text{out}}=\delta\langle H_{\text{mod}}\rangle$, then near this background obtain \textbf{linearized} field equations
$$
\boxed{\,\delta G_{\mu\nu}+\Lambda\,\delta g_{\mu\nu}=8\pi G\,\delta\langle T_{\mu\nu}\rangle\,}.
$$
\end{theorem}

\begin{remark}
To obtain $G_{\mu\nu}+\Lambda g_{\mu\nu}=8\pi G\,\langle T_{\mu\nu}\rangle$ at \textbf{nonlinear} level typically requires additional structural premises (such as holographic/generalized first law, appropriate geometrization of modular Hamiltonian, etc.); this paper \textbf{makes no claim of universal validity}.
\end{remark}

\begin{proof}[Proof Outline]
Using relative entropy first law $\delta S_{\text{out}}=\delta\langle H_{\text{mod}}\rangle$ and localized expression of modular Hamiltonian, pair $\delta A/4G$ with energy-momentum tensor coupling; second variation non-negativity yields QNEC-type inequalities. Relative entropy version equivalent to Jacobson's ``entanglement equilibrium'' at small-ball/linearization level; generalization to \textbf{fully nonlinear} requires additional structure (such as holographic background or integrability); this paper makes no claim of general validity \cite{jacobson2016entanglement}.
\end{proof}

\section{Kähler--Dirac and Spinor Field Equivalence (Proof)}
\label{app:kd-proof}

\begin{proposition}
If $\mathfrak{M}$ is Kähler with $\mathrm{Spin}^c$ structure (or $\mathbb{R}^4$ flat background), then Kähler--Dirac operator $\mathcal{D}=d+d^\dagger$ is spectrally equivalent to direct sum of multiple copies of $\mathrm{Spin}^c$ Dirac operators; in four dimensions multiplicity is \textbf{4}. On general $\mathrm{Spin}^c$ manifolds, $\mathcal{D}$ and spinor Dirac are typically \textbf{not} spectrally equivalent, only both belonging to Dirac-type operators.
\end{proposition}

\begin{proof}[Proof Sketch]
Use Kähler identities and $\mathrm{Spin}^c$ structure to give $S\simeq\Lambda^{0,\bullet}$, pair Kähler decomposition of $\Omega^\bullet$ with Clifford module decomposition, thus identifying action of $\mathcal{D}$ with $D_{\mathrm{Spin}^c}$ on direct sum components and obtaining spectral equivalence (with multiplicity); on non-Kähler/non-flat backgrounds only ensure Dirac-type isomorphism without spectral equivalence. For discretization, refer to Dirac--Kähler and staggered equivalence \cite{bodwin1988equivalence}.
\end{proof}

\section{Gauge Field Principal Bundle Connection--Reduction--Higgs (Proof)}
\label{app:higgs-proof}

\begin{theorem}[Sardanashvily]
For principal bundle $P(\mathfrak{X},G)$, section $h\in\Gamma(P/H)$ exists if and only if structure group reduces to $H$, and matter field Lagrangian is gauge-invariant if and only if it factorizes through vertical covariant derivative $P\to P/H$ \cite{sardanashvily2005geometry}.
\end{theorem}

\section{Windowed Scattering Master Scale and BK/FSR (Proof)}
\label{app:master-scale}

\begin{proposition}
Let $S(E)$ be unitarily differentiable, then
$$
\frac{1}{2\pi}\operatorname{tr}Q(E)=\frac{\varphi'(E)}{\pi}=-\,\xi'(E),
$$
where $\varphi(E)=\tfrac{1}{2}\arg\det S(E)$, $\xi$ is spectral shift function (taking Birman--Kreĭn convention $\det S(E)=e^{-2\pi i\,\xi(E)}$).
\end{proposition}

\begin{proof}[Proof Sketch]
From $\log\det S=-2\pi i\,\xi$ obtain
$$
\xi'=-\frac{1}{2\pi i}\frac{d}{dE}\log\det S,\qquad 
\operatorname{tr}Q=-i\,\frac{d}{dE}\log\det S\Rightarrow \xi'=-\frac{1}{2\pi}\operatorname{tr}Q.
$$
Also from $\varphi=\tfrac{1}{2}\arg\det S$ obtain $\tfrac{1}{2\pi}\operatorname{tr}Q=\varphi'/\pi$ \cite{pushnitski2002spectral}.
\end{proof}

\section{Causality--Analyticity and KK/Titchmarsh (Proof)}
\label{app:kk}

\begin{theorem}
If time-domain kernel is causal and square-integrable, then frequency-domain response is analytic in upper half-plane, with real/imaginary parts as Hilbert transform pairs; convolution support endpoints satisfy
$$
\inf\operatorname{supp}(f*g)=\inf\operatorname{supp} f+\inf\operatorname{supp} g.
$$
\end{theorem}

\begin{proof}
Paley--Wiener and Riesz theorems give analyticity and KK; Titchmarsh gives support endpoint addition equality achievement \cite{titchmarsh1926reciprocal}.
\end{proof}

\section{Mellin Tight Frames and Frame Bounds (Proof)}
\label{app:mellin}

In logarithmic domain construct mother wavelet $\psi(\log E)$ and its translation family; using Parseval and Poisson summation establish frame bounds
$$
A\|f\|_{L^2(|d\mu_\varphi|)}^2\le\sum_k\big|\langle f,\psi_k\rangle\big|^2\le B\|f\|_{L^2(|d\mu_\varphi|)}^2,
$$
where $|d\mu_\varphi|=|\rho_{\text{rel}}(E)|dE$ is Jordan positive measure of $d\mu_\varphi$.
Tight/tunable tight frame construction and error estimation can be given from existing literature \cite{daubechies2003framelets}.

\section{Measurement Unification and Quantum Filtering (Proof)}
\label{app:measurement}

\begin{theorem}[Naimark]
Any POVM can be represented as compression of PVM in extended space.
\end{theorem}

\begin{theorem}[Belavkin]
Under continuous observation, posterior state satisfies filtering equation of QSDE form.
\end{theorem}

\begin{theorem}[Jarzynski--Sagawa--Ueda]
With feedback measurement, $\langle e^{-\beta W+\beta\Delta F-I}\rangle=1$.
\end{theorem}

Proof lines consistent with original literature \cite{naimark1943spectral,belavkin1992quantum,sagawa2008secondlaw}.

\section{Numerical Discipline and Windowed NPE}
\label{app:npe}

Poisson sidelobes, finite-order Euler--Maclaurin endpoint remainder, and tail integrals give total error additive upper bound; under band-limited and Nyquist conditions aliasing is zero; this provides controllable error theory for windowed integration on master scale.

\section*{Acknowledgments}

All results cited rely on standard theorems in information geometry, principal bundle geometry, spectral theory, and quantum measurement theory. References provided for verifiability.

\bibliographystyle{plain}
\begin{thebibliography}{99}

\bibitem{amari2016information}
S. Amari and H. Nagaoka.
\newblock {\em Methods of Information Geometry}.
\newblock American Mathematical Society, 2000.

\bibitem{petz1996monotone}
D. Petz.
\newblock Monotone metrics on matrix spaces.
\newblock {\em Linear Algebra and its Applications}, 244:81--96, 1996.

\bibitem{provost1980riemannian}
J. P. Provost and G. Vallee.
\newblock Riemannian structure on manifolds of quantum states.
\newblock {\em Communications in Mathematical Physics}, 76(3):289--301, 1980.

\bibitem{cunden2015statistical}
F. D. Cunden, F. Mezzadri, N. Simm, and P. Vivo.
\newblock Statistical distribution of the Wigner-Smith time-delay matrix moments for chaotic cavities.
\newblock {\em Physical Review E}, 91(6):060102, 2015.

\bibitem{kramers1927diffusion}
H. A. Kramers.
\newblock La diffusion de la lumière par les atomes.
\newblock {\em Atti del Congresso Internazionale dei Fisici, Como}, 2:545--557, 1927.

\bibitem{debranges1968hilbert}
L. de Branges.
\newblock {\em Hilbert Spaces of Entire Functions}.
\newblock Prentice-Hall, 1968.

\bibitem{rabin1982introduction}
J. M. Rabin.
\newblock Homology theory of lattice fermion doubling.
\newblock {\em Nuclear Physics B}, 201(2):315--332, 1982.

\bibitem{bodwin1988equivalence}
G. T. Bodwin and E. V. Kovacs.
\newblock Equivalence of Dirac-Kähler and staggered lattice fermions in four dimensions.
\newblock {\em Physical Review D}, 38(4):1206, 1988.

\bibitem{nakahara2003geometry}
M. Nakahara.
\newblock {\em Geometry, Topology and Physics}.
\newblock CRC Press, 2nd edition, 2003.

\bibitem{sardanashvily2005geometry}
G. Sardanashvily.
\newblock Geometry of classical Higgs fields.
\newblock arXiv:hep-th/0510168, 2005.

\bibitem{jacobson2016entanglement}
T. Jacobson.
\newblock Entanglement equilibrium and the Einstein equation.
\newblock {\em Physical Review Letters}, 116(20):201101, 2016.

\bibitem{pushnitski2002spectral}
A. Pushnitski.
\newblock The spectral shift function and the invariance principle.
\newblock {\em Journal of Functional Analysis}, 183(2):269--320, 2001.

\bibitem{daubechies2003framelets}
I. Daubechies, B. Han, A. Ron, and Z. Shen.
\newblock Framelets: MRA-based constructions of wavelet frames.
\newblock {\em Applied and Computational Harmonic Analysis}, 14(1):1--46, 2003.

\bibitem{sagawa2008secondlaw}
T. Sagawa and M. Ueda.
\newblock Second law of thermodynamics with discrete quantum feedback control.
\newblock {\em Physical Review Letters}, 100(8):080403, 2008.

\bibitem{baez1992introduction}
J. C. Baez and J. P. Muniain.
\newblock {\em Gauge Fields, Knots and Gravity}.
\newblock World Scientific, 1994.

\bibitem{hauke2016measuring}
P. Hauke, M. Heyl, L. Tagliacozzo, and P. Zoller.
\newblock Measuring multipartite entanglement through dynamic susceptibilities.
\newblock {\em Nature Physics}, 12(8):778--782, 2016.

\bibitem{naimark1943spectral}
M. A. Naimark.
\newblock On a representation of additive operator set functions.
\newblock {\em Comptes Rendus de l'Académie des Sciences de l'URSS}, 41(9):359--361, 1943.

\bibitem{belavkin1992quantum}
V. P. Belavkin.
\newblock Quantum stochastic calculus and quantum nonlinear filtering.
\newblock {\em Journal of Multivariate Analysis}, 42(2):171--201, 1992.

\bibitem{titchmarsh1926reciprocal}
E. C. Titchmarsh.
\newblock The zeros of certain integral functions.
\newblock {\em Proceedings of the London Mathematical Society}, 2(1):283--302, 1926.

\bibitem{berry1984quantal}
M. V. Berry.
\newblock Quantal phase factors accompanying adiabatic changes.
\newblock {\em Proceedings of the Royal Society of London A}, 392(1802):45--57, 1984.

\bibitem{birman1962theory}
M. Sh. Birman and M. G. Kreĭn.
\newblock On the theory of wave operators and scattering operators.
\newblock {\em Doklady Akademii Nauk SSSR}, 144:475--478, 1962.

\bibitem{friedel1952distribution}
J. Friedel.
\newblock The distribution of electrons round impurities in monovalent metals.
\newblock {\em The London, Edinburgh, and Dublin Philosophical Magazine and Journal of Science}, 43(337):153--189, 1952.

\end{thebibliography}

\end{document}

