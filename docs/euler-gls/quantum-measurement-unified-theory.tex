\documentclass[12pt]{article}

% Essential packages
\usepackage[utf8]{inputenc}
\usepackage{amsmath,amssymb,amsthm}
\usepackage{mathrsfs}
\usepackage{geometry}
\usepackage{hyperref}

% Geometry settings
\geometry{a4paper, margin=1in}

% Hyperref settings
\hypersetup{
    colorlinks=true,
    linkcolor=blue,
    citecolor=blue,
    urlcolor=blue
}

% Theorem environments
\theoremstyle{plain}
\newtheorem{theorem}{Theorem}[section]
\newtheorem{lemma}[theorem]{Lemma}
\newtheorem{proposition}[theorem]{Proposition}
\newtheorem{corollary}[theorem]{Corollary}

\theoremstyle{definition}
\newtheorem{definition}[theorem]{Definition}
\newtheorem{example}[theorem]{Example}
\newtheorem{remark}[theorem]{Remark}
\newtheorem{axiom}{Axiom}

% Title information
\title{Axiomatic Quantum Measurement Theory: Static Block Superpositions, Semantic Collapse, and Causal Network Entanglement}
\author{Haobo Ma$^1$ \and Wenlin Zhang$^2$\\
\small $^1$Independent Researcher\\
\small $^2$National University of Singapore}

\date{\today}

\begin{document}

\maketitle

\begin{abstract}
Under the unified perspective of ``discrete--static blocks (EBOC)'' and ``continuous--causal manifolds/gauge systems,'' we construct an axiomatic framework for quantum measurement: wave functions are characterized as complex-amplitude superpositions of compatible SBUs (Static Block Units) in static blocks; collapse is characterized as observer anchor switching and $\pi$-semantic collapse from operator algebras to classical record algebras; entanglement is characterized as conditional mutual information and quantum Markov property in causal networks. Mathematically, quantum instruments and POVMs serve as core objects, with Stinespring/Naimark/Kraus representations, Lüders update, relative entropy minimization principle, and Belavkin filtering closing the ``measurement--update--record'' chain; at the information-geometric level, strong subadditivity and conditional mutual information control causal propagation of entanglement and ``no-signaling.'' At the readout--geometry interface, employing windowed compression $K_{w,h}$ and ``trinity master-scale'' $\displaystyle\frac{\varphi_{\text{m}}'}{\pi}=\rho_{\text{rel}}=\frac{1}{2\pi}\mathrm{tr}\,\mathsf{Q}$ unified readout framework, with non-asymptotic error closure under Nyquist--Poisson--Euler--Maclaurin (finite-order) discipline; this identity holds almost everywhere on absolutely continuous spectrum, thereby rigorously unifying ``phase derivative--relative spectral density--Wigner--Smith group delay.'' Davies--Lewis instruments/POVMs, postulated Lüders rule, Stinespring--Kraus--Naimark representations, and Belavkin continuous measurement stochastic filtering provide testable anchor points; Birman--Kreĭn identity, Kreĭn--Friedel--Lloyd bridge, and de Branges kernel diagonal formula provide spectral/functional-analytic basis for unified readout of ``windowed--spectral--scattering.''
\end{abstract}

\noindent\textbf{Keywords:} Quantum measurement; EBOC; Static block units; POVM; Quantum instruments; Lüders rule; I-projection; Belavkin filtering; Master scale; Wigner--Smith time delay; Birman--Kreĭn; Spectral shift; Windowed readout; Toeplitz/Berezin compression; Causal network; Conditional mutual information

\noindent\textbf{MSC 2020:} Primary 81P15, 81S22, 47A40; Secondary 94A17, 42C40, 81P45

\section{Notation, Axioms, and Conventions}

\subsection{Notation and Conventions}

Units: $\hbar=c=1$. Hilbert space $\mathcal{H}$, observable algebra $\mathcal{A}\subseteq\mathcal{B}(\mathcal{H})$, classical record algebra $\mathcal{O}\cong L^\infty(\mathcal{X})$. Measurement represented by Davies--Lewis instruments $\{\mathcal{I}_\Delta\}_{\Delta\in\mathcal{B}(\mathcal{X})}$ and POVM $E(\cdot)$; Heisenberg image
$$
\pi^\ast(f)=\int_{\mathcal{X}} f(x)\,E(dx),
$$
reducing to $\pi^\ast(f)=\sum_x f(x)E_x$ when $\mathcal{X}$ discrete. Kraus representation equivalent to Stinespring dilation (for CP maps); POVM equivalent to PVM via Naimark extension (for measures). Lüders update covers projection measurement limit; Belavkin filtering for continuous measurement \cite{davies1970operational}.

\subsection{Spectral Measure Notation}

Denote $E_H:\mathcal{B}(\mathbb{R})\to\mathcal{B}(\mathcal{H})$ as spectral projection measure of self-adjoint operator $H$; for any bounded Borel function $f$,
$$
f(H)=\int_{\mathbb{R}} f(E)\,dE_H(E).
$$
Thus $\int w_R(E)\,dE_H$ in \S\ref{sec:windowed} is $w_R(H)$.

\subsection{Convolution and Windowing}

Take window--kernel pair $(w_R,h)$, convolution $(h\star f)(E)=\int_{\mathbb{R}}h(E-E')f(E')\,dE'$. Windowed readout defined as $\langle K_{w,h}\rangle=\int_{\mathbb{R}} w_R(E)\,(h\star\rho_\star)(E)\,dE$, where $\rho_\star\in\{\rho_{\text{abs}},\rho_{\text{rel}}\}$. Here $\rho_{\text{abs}}(E)$ is absolute spectral density/density of states, defined for any $f\in C_0^\infty(\mathbb{R})$ as
$$
\mathrm{Tr}\,f(H)=\int_{\mathbb{R}} f(E)\,\rho_{\text{abs}}(E)\,dE,
$$
with window operator $W_R$ or volume normalization ensuring finiteness as needed. Nyquist--Poisson--Euler--Maclaurin (finite-order) yields non-asymptotic error ledger; under band-limited and Nyquist conditions, aliasing error is zero \cite{higgins1985five}.

\subsection{Master Scale and Scattering}

Define \textbf{master-scale phase}
$$
\varphi_{\text{m}}(E):=\tfrac{1}{2}\arg\det S(E)=-\pi\,\xi(E).
$$
Under standard regularity (relative trace-class perturbation, existence of scattering matrix), almost everywhere on absolutely continuous spectrum
$$
\frac{\varphi'_{\text{m}}(E)}{\pi}=\rho_{\text{rel}}(E)=\frac{1}{2\pi}\mathrm{tr}\,\mathsf{Q}(E),
$$
where $\det S=e^{-2\pi i\xi}$, $\rho_{\text{rel}}=-\xi'$, $\mathsf{Q}=-i\,S^\dagger \tfrac{dS}{dE}$ \cite{pushnitski2010representation}.

\subsection{de Branges Anchor}

Let $\varphi_{\text{dB}}(x):=-\arg E(x)$. Then
$$
\frac{K(x,x)}{|E(x)|^2}=\frac{\varphi'_{\text{dB}}(x)}{\pi}.
$$
\textbf{Bridge condition:} If and only if $E$ is generated by canonical system corresponding to $(H,H_0)$ and aligned with scattering data, $\varphi_{\text{dB}}(x)=\varphi_{\text{m}}(x)$ holds almost everywhere; in general case, this only provides \textbf{function-theoretic calibration scale} for ``phase--density,'' not unconditionally identical to $\rho_{\text{rel}}$ \cite{debranges1968hilbert}.

\section{Wave Function: Compatible SBU Superposition in EBOC Static Blocks}

\begin{definition}[SBU, Decoherence Functional, and Compatible Partition]
\label{def:sbu}
Let $\mathfrak{B}$ be family of history classes in static blocks; interference/decoherence functional $D:\mathfrak{B}\times\mathfrak{B}\to\mathbb{C}$ given. If there exists subfamily $\mathcal{C}\subset\mathfrak{B}$ such that for any $B\neq B'\in\mathcal{C}$, $D(B,B')=0$, then $\mathcal{C}$ is called compatible (decoherent) partition. Wave function $\psi:\mathfrak{B}\to\mathbb{C}$ on compatible partition satisfies $\sum_{B\in\mathcal{C}}|\psi(B)|^2=1$, defining probability $\mathbb{P}(B)=D(B,B)=|\psi(B)|^2$ $(B\in\mathcal{C})$.
\end{definition}

\begin{proposition}[Decoherence Induction]
\label{prop:decoherence}
If interference functional for selected history classes is diagonal, then $\mathbb{P}$ coincides with Born probability.
\end{proposition}

\begin{proof}[Proof Sketch]
Coherence terms vanish, reducing to orthogonal decomposition; see Section \ref{sec:born} for Born--I-projection equivalence.
\end{proof}

\section{Collapse: Anchor Switching and $\pi$-Semantics}

\begin{definition}[Instrument and Classicalization]
\label{def:instrument}
Heisenberg classicalization of instrument $\{\mathcal{I}_\Delta\}_{\Delta\in\mathcal{B}(\mathcal{X})}$ and POVM $E(\cdot)$ is
$$
\pi^\ast(f)=\int_{\mathcal{X}} f(x)\,E(dx),
$$
reducing to $\pi^\ast(f)=\sum_x f(x)E_x$ in discrete case. Choosing record algebra basis is ``anchoring''; anchor switching corresponds to basis change $\mathcal{O}\to\mathcal{O}'$.
\end{definition}

\begin{theorem}[Lüders/Instrument Update]
\label{thm:luders}
When outcome algebra and measurement PVM are co-diagonal and \textbf{$\mathrm{tr}(\rho P_x)>0$}, selective conditional state update is
$$
\rho\mapsto\rho_x=\frac{P_x\rho P_x}{\mathrm{tr}(\rho P_x)}.
$$
\textbf{If $\mathrm{tr}(\rho P_x)=0$, conditional state for this outcome is undefined.} General POVM can be extended to PVM via Naimark extension, then recovered to Schrödinger picture via Stinespring/Kraus \cite{busch2009luders}.
\end{theorem}

\begin{proof}[Proof Sketch]
Naimark gives $E_x=V^\ast P_x V$; Stinespring gives $\Phi(A)=V^\ast\pi(A)V$. Selective update realized by conditioning and contraction $V$; projection case recovers Lüders form.
\end{proof}

\section{Entanglement: Information Correlation and No-Signaling in Causal Networks}

\begin{definition}[Conditional Mutual Information and Quantum Markov]
\label{def:cmi}
Conditional mutual information $I(A:C|B)=S(AB)+S(BC)-S(ABC)-S(B)\ge 0$. Equality if and only if state is short quantum Markov chain (recovery map exists making $A\leftrightarrow C|B$ independent) \cite{lieb1973proof}.
\end{definition}

\begin{proposition}[Operator Condition for No-Signaling]
\label{prop:nosignal}
If regional algebras commute at spacelike separation and local operations are CPTP (CP and trace-preserving), then no superluminal signals exist; in AQFT can be characterized by CP local operations and ``funnel property'' \cite{buchholz2017local}.
\end{proposition}

\begin{remark}
Above connection between information theory and algebraic causality provides criterion for ``entanglement propagation constraint,'' compatible with ``record algebra anchoring.''
\end{remark}

\section{Continuous Measurement and Quantum Trajectories}

\begin{theorem}[Belavkin Filtering]
\label{thm:belavkin}
Under QND and Markov approximation, recorded history yields quantum stochastic differential equation for posterior state; discrete weak measurement limit and trajectory averaging recover Lindblad master equation \cite{belavkin1992quantum}.
\end{theorem}

\begin{proof}[Proof Sketch]
Use quantum stochastic calculus to establish filtering equation for observation process; counting-type and diffusion-type equations are mutual limits; see original work for details \cite{belavkin2005notes}.
\end{proof}

\section{Unified Measurement Variational Principle (I-Projection)}

\begin{theorem}[Relative Entropy Minimization and Lüders/I-Projection Consistency Conditions]
\label{thm:iprojection}
(a) \textbf{Selective update:} Given projection measurement $\{P_x\}$ with outcome $x$, \textbf{if $\mathrm{tr}(\rho P_x)>0$}, unique solution minimizing $D(\rho'\Vert\rho)$ under constraints $\rho'\ge0$, $\mathrm{tr}\rho'=1$, $\mathrm{supp}(\rho')\subseteq\mathrm{ran}(P_x)$ is
$$
\rho'_x=\frac{P_x\rho P_x}{\mathrm{tr}(\rho P_x)}\quad\text{(Lüders conditional state)}.
$$
\textbf{If $\mathrm{tr}(\rho P_x)=0$, constraint set is empty, conditional state undefined.}

(b) \textbf{Non-selective update:} Let prior be \textbf{faithful state} $\rho\succ0$; if $\rho$ not full-rank, \textbf{restrict throughout to} $\mathrm{supp}(\rho)$ and replace with restricted $\log\rho$. If $[\rho,P_x]=0$ and adopting alignment constraints $\mathrm{tr}(\rho'P_x)=\mathrm{tr}(\rho P_x)$ (for all $x$), then I-projection solution is $\rho'=\sum_x P_x\rho P_x$; in general non-commutative case only obtain
$$
\rho' \propto \exp\!\Bigl(\log\rho-\sum_x\lambda_x P_x\Bigr)\quad\text{(on }\mathrm{supp}(\rho)\text{)},
$$
typically not identical to Lüders \cite{csiszar2004information}.
\end{theorem}

\begin{proof}
(a) Direct from constraints and strict convexity; (b) from Lagrangian duality and commutativity condition.
\end{proof}

\begin{corollary}[Born = I-Projection Limit]
\label{cor:born-limit}
For projection measurement, when window/noise scale $\tau\downarrow 0$ (hard limit), softmax weights converge to one-hot kernel limit, yielding $p_x=\mathrm{tr}(\rho P_x)$, i.e., Born rule.
\end{corollary}

\section{Windowed Readout and Three-Part Error Decomposition}
\label{sec:windowed}

\begin{definition}[Windowed Readout and Toeplitz/Berezin Compression]
\label{def:windowed}
Let window operator $W_R=\int w_R(E)\,dE_H$; then Toeplitz/Berezin compression and kernel smoothing induce readout $\langle K_{w,h}\rangle=\int_{\mathbb{R}} w_R(E)\,(h\star\rho_\star)(E)\,dE$. Berezin transform characterizes symbol-to-operator mapping and its compactness/trace-class property.
\end{definition}

\begin{remark}[Compactness Constraint]
Fix window operator as
$$
W_R:=P_R\Bigl(\int_{\mathbb{R}} w_R(E)\,dE_H\Bigr)P_R,
$$
where $P_R$ is finite-rank (or Hilbert--Schmidt) localization projection (Toeplitz/Berezin compression). Thus $W_R$ self-adjoint and compact, with discrete spectrum; Ky Fan extremal characterization of \S\ref{sec:pointer} applies \cite{axler2015berezin}.
\end{remark}

\begin{theorem}[Nyquist--Poisson--Euler--Maclaurin Non-Asymptotic Error]
\label{thm:npe}
For equispaced sampling--finite-sum approximation of $F(E)=w_R(E)\,(h\star\rho_\star)(E)$, have
$\mathcal{E}_R=\mathcal{E}_{\text{alias}}+\mathcal{E}_{\text{EM}}+\mathcal{E}_{\text{tail}}$; if $F$ band-limited and sampling rate satisfies Nyquist, then $\mathcal{E}_{\text{alias}}=0$; finite-order Euler--Maclaurin gives explicit remainder upper bound \cite{higgins1985five}.
\end{theorem}

\begin{proof}[Proof Sketch]
Poisson summation converts sampling error to frequency-domain sidelobes; band-limited and Nyquist kill aliasing; EM bounded remainder given by periodization of Bernoulli polynomials and bounded integrability of $f^{(p)}$ \cite{borwein2013aliasing}.
\end{proof}

\section{Phase--Density--Group Delay ``Trinity'' and Windowed BK Identity}

\begin{theorem}[Birman--Kreĭn and Wigner--Smith]
\label{thm:bk-ws}
Let $(H,H_0)$ be relative trace-class perturbation with scattering matrix $S(E)$ and spectral shift function $\xi(E)$. Then $\det S(E)=e^{-2\pi i\,\xi(E)}$, $\rho_{\text{rel}}(E)=-\xi'(E)$, $\mathsf{Q}(E)=-i\,S^\dagger(E)\frac{dS}{dE}(E)$, and
$$
\mathrm{tr}\,\mathsf{Q}(E)=-2\pi\,\xi'(E),\qquad
\frac{\varphi_{\text{m}}'(E)}{\pi}=\rho_{\text{rel}}(E)=\frac{1}{2\pi}\mathrm{tr}\,\mathsf{Q}(E)
$$
holds almost everywhere \cite{pushnitski2010representation}.
\end{theorem}

\begin{proof}[Proof Outline]
From Lifshits--Kreĭn trace formula and BK identity obtain $\xi'$ as spectral difference density; Wigner--Smith definition gives
$$
\partial_E\arg\det S=\frac{1}{i}\partial_E\log\det S=\frac{1}{i}\mathrm{tr}\,S^\dagger S'=\mathrm{tr}\,\mathsf{Q},
$$
combining yields master identity. For non-unitary extensions see recent literature (complex dissipative case), but this framework takes unitary branch under cosmic closure/unitarity assumption \cite{peller2016lifshits}.
\end{proof}

\begin{theorem}[Windowed BK Identity]
\label{thm:windowed-bk}
Let $f=h\star w_R \in C_0^\infty$. Then
$\mathrm{Tr}\,f(H)-\mathrm{Tr}\,f(H_0)=\int_{\mathbb{R}} f'(E)\,\xi(E)\,dE$
and
$-\frac{1}{2\pi i}\int_{\mathbb{R}} f'(E)\,\log\det S(E)\,dE=\int_{\mathbb{R}} f'(E)\,\xi(E)\,dE$,
thus windowed ``spectrum--scattering--phase'' integral-level bridge holds.
\end{theorem}

\begin{proof}[Proof Sketch]
Via Helffer--Sjöstrand formula and applicability of Kreĭn trace formula for operator Lipschitz class functions \cite{helffer1989equation}.
\end{proof}

\section{Born Rule Completion: I-Projection and Orthogonal Limit}
\label{sec:born}

\begin{theorem}[Born = I-Projection Iff]
\label{thm:born-iprojection}
Under projection measurement, probability vector $p$ generated by unique I-projection solution on linear alignment constraints equals Born probability if and only if Lagrange multipliers of dual problem satisfy spectral diagonal condition (i.e., pointer basis co-diagonal). In this case Lüders update and I-projection coincide \cite{shore1980axiomatic}.
\end{theorem}

\begin{proof}[Proof Sketch]
Under diagonalization, exponential family degenerates to weighting on spectral measure; in limit $\tau\downarrow 0$, softmax converges to projection selection, completing consistency with Lüders.
\end{proof}

\section{Pointer Basis Selection: Spectral Variational Characterization}
\label{sec:pointer}

\begin{theorem}[Ky Fan Spectral Extremal Characterization of Pointer Basis]
\label{thm:kyfan}
Let window operator $W_R$ be self-adjoint compact operator, $P_V$ orthogonal projection onto any $k$-dimensional subspace $V$. Then
$$
\min_{\dim V=k}\mathrm{Tr}(P_V W_R)=\sum_{j=1}^k \lambda_j^{\uparrow}(W_R),
$$
minimum achieved by smallest $k$ eigenprojections of $W_R$. Thus ``pointer basis'' can be characterized as eigenbasis minimizing $\mathrm{Tr}(P_V W_R)$; this formulation equivalent to Ky Fan minimum sum principle \cite{fan1951maximum}.
\end{theorem}

\begin{proof}[Proof Sketch]
Direct from Ky Fan minimum sum principle; minimum achieved by smallest eigenprojections.
\end{proof}

\appendix

\section{de Branges Kernel Diagonal and Phase Density}

de Branges structure function $E=A-iB$ (Hermite--Biehler class) with phase $\varphi_{\text{dB}}(x)=-\arg E(x)$.

\begin{remark}[Notation Supplement]
Let $E^\sharp(z):=\overline{E(\overline{z})}$ (de Branges adjoint entire function).
\end{remark}

Kernel formula written as
$$
K(z,w)=\frac{E(z)\overline{E(w)}-E^\sharp(z)\,\overline{E^\sharp(w)}}{2\pi i\,(z-\overline{w})},
$$
diagonalizing on real axis as
$$
K(x,x)=\frac{\varphi'_{\text{dB}}(x)}{\pi}|E(x)|^2.
$$
This is function-theoretic anchor point for ``phase density scale'' \cite{debranges1968hilbert}.

\section{Kreĭn--Friedel--Lloyd Bridge and Time Delay}

Friedel formula gives relative DOS
$$
\rho_{\text{rel}}(E)=\frac{1}{\pi}\partial_E\varphi_{\text{m}}(E),
$$
consistent with BK identity; where $M:=\dim S(E)$ is number of open channels, $\mathrm{tr}$ denotes trace over channel space. Wigner--Smith gives
$$
\tau_W(E)=\frac{1}{M}\mathrm{tr}\,\mathsf{Q}(E)=-\frac{i}{M}\partial_E\log\det S(E).
$$
Thus ``phase derivative--relative DOS--group delay'' triple unification \cite{friedel1952distribution}.

\section{Nyquist--Poisson--EM Error Closure Details}

Poisson summation expresses sampling error as sum of high-frequency image differences; band-limited and Nyquist condition annihilate aliasing terms. Finite-order Euler--Maclaurin uses periodized Bernoulli functions to give remainder integral expression and upper bound estimate; for analytic/band-limited $F$ can give explicit non-asymptotic upper bound, closing windowed ledger \cite{higgins1985five}.

\section{Conclusion}

(i) With instruments/POVMs--I-projection--Lüders/Belavkin as core, can close ``measurement--update--record'' categorically; (ii) with $\boxed{\varphi'_{\text{m}}/\pi=\rho_{\text{rel}}=(2\pi)^{-1}\mathrm{tr}\,\mathsf{Q}}$ as master scale, and with de Branges kernel diagonal and BK/Helffer--Sjöstrand as anchors, realize testable unification of ``windowed--spectral--scattering''; (iii) Nyquist--Poisson--EM finite-order discipline ensures non-asymptotic error closure; (iv) conditional mutual information and quantum Markov property provide causal constraints on entanglement propagation and no-signaling. These four points unify EBOC static block SBU superposition, anchor-switching $\pi$-semantic collapse, and causal network entanglement within the same measurement theory framework.

\section*{Acknowledgments}

All results cited rely on standard theorems in quantum measurement theory, spectral theory, and information theory. References provided for verifiability.

\bibliographystyle{plain}
\begin{thebibliography}{99}

\bibitem{davies1970operational}
E. B. Davies and J. T. Lewis.
\newblock An operational approach to quantum probability.
\newblock {\em Communications in Mathematical Physics}, 17(3):239--260, 1970.

\bibitem{stinespring1955positive}
W. F. Stinespring.
\newblock Positive functions on C*-algebras.
\newblock {\em Proceedings of the American Mathematical Society}, 6(2):211--216, 1955.

\bibitem{naimark1943spectral}
M. A. Naimark.
\newblock On a representation of additive operator set functions.
\newblock {\em Comptes Rendus de l'Académie des Sciences de l'URSS}, 41(9):359--361, 1943.

\bibitem{kraus1983states}
K. Kraus.
\newblock {\em States, Effects, and Operations: Fundamental Notions of Quantum Theory}.
\newblock Springer, 1983.

\bibitem{busch2009luders}
P. Busch.
\newblock Lüders rule.
\newblock In D. Greenberger, K. Hentschel, and F. Weinert, editors, {\em Compendium of Quantum Physics}, pages 356--358. Springer, 2009.

\bibitem{belavkin1992quantum}
V. P. Belavkin.
\newblock Quantum continual measurements and a posteriori collapse on CCR.
\newblock {\em Communications in Mathematical Physics}, 146(3):611--635, 1992.

\bibitem{belavkin2005notes}
V. P. Belavkin.
\newblock Quantum stochastic calculus and quantum nonlinear filtering.
\newblock {\em Journal of Multivariate Analysis}, 42(2):171--201, 1992.

\bibitem{lieb1973proof}
E. H. Lieb and M. B. Ruskai.
\newblock Proof of the strong subadditivity of quantum-mechanical entropy.
\newblock {\em Journal of Mathematical Physics}, 14(12):1938--1941, 1973.

\bibitem{hayden2004structure}
P. Hayden, R. Jozsa, D. Petz, and A. Winter.
\newblock Structure of states which satisfy strong subadditivity of quantum entropy with equality.
\newblock {\em Communications in Mathematical Physics}, 246(2):359--374, 2004.

\bibitem{buchholz2017local}
D. Buchholz and C. Solveen.
\newblock Local operations and completely positive maps in algebraic quantum field theory.
\newblock arXiv:1704.01229, 2017.

\bibitem{pushnitski2010representation}
A. Pushnitski.
\newblock Representation for the spectral shift function for perturbations of fixed sign.
\newblock {\em St. Petersburg Mathematical Journal}, 9(6):1181--1194, 1998.

\bibitem{debranges1968hilbert}
L. de Branges.
\newblock {\em Hilbert Spaces of Entire Functions}.
\newblock Prentice-Hall, 1968.

\bibitem{csiszar2004information}
I. Csiszár and F. Matúš.
\newblock Information projections revisited.
\newblock {\em IEEE Transactions on Information Theory}, 49(6):1474--1490, 2003.

\bibitem{higgins1985five}
J. R. Higgins.
\newblock Five short stories about the cardinal series.
\newblock {\em Bulletin of the American Mathematical Society}, 12(1):45--89, 1985.

\bibitem{axler2015berezin}
S. Axler and D. Zheng.
\newblock The Berezin transform on the Toeplitz algebra.
\newblock {\em Studia Mathematica}, 127(2):113--136, 1998.

\bibitem{borwein2013aliasing}
D. Borwein, J. M. Borwein, I. E. Leonard, and L. Jörgenson.
\newblock Aliasing error for sampling series derivatives.
\newblock {\em Rendiconti del Circolo Matematico di Palermo}, 52(2):345--360, 2003.

\bibitem{peller2016lifshits}
V. V. Peller.
\newblock The Lifshits--Kreĭn trace formula and operator Lipschitz functions.
\newblock {\em Proceedings of the American Mathematical Society}, 144(12):5207--5215, 2016.

\bibitem{helffer1989equation}
B. Helffer and J. Sjöstrand.
\newblock Equation de Schrödinger avec champ magnétique et équation de Harper.
\newblock In {\em Schrödinger Operators}, pages 118--197. Springer, 1989.

\bibitem{shore1980axiomatic}
J. E. Shore and R. W. Johnson.
\newblock Axiomatic derivation of the principle of maximum entropy and the principle of minimum cross-entropy.
\newblock {\em IEEE Transactions on Information Theory}, 26(1):26--37, 1980.

\bibitem{fan1951maximum}
K. Fan.
\newblock Maximum properties and inequalities for the eigenvalues of completely continuous operators.
\newblock {\em Proceedings of the National Academy of Sciences}, 37(11):760--766, 1951.

\bibitem{friedel1952distribution}
J. Friedel.
\newblock The distribution of electrons round impurities in monovalent metals.
\newblock {\em The London, Edinburgh, and Dublin Philosophical Magazine and Journal of Science}, 43(337):153--189, 1952.

\bibitem{birman1962theory}
M. Sh. Birman and M. G. Kreĭn.
\newblock On the theory of wave operators and scattering operators.
\newblock {\em Doklady Akademii Nauk SSSR}, 144:475--478, 1962.

\bibitem{martin2016wigner}
P. A. Martin.
\newblock Wigner time delay and related concepts: Application to transport in coherent conductors.
\newblock arXiv:1507.00075, 2016.

\end{thebibliography}

\end{document}

