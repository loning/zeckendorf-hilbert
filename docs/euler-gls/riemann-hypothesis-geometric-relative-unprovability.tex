\documentclass[12pt]{article}

% Essential packages
\usepackage[utf8]{inputenc}
\usepackage{amsmath,amssymb,amsthm}
\usepackage{mathrsfs}
\usepackage{geometry}
\usepackage{hyperref}

% Geometry settings
\geometry{a4paper, margin=1in}

% Hyperref settings
\hypersetup{
    colorlinks=true,
    linkcolor=blue,
    citecolor=blue,
    urlcolor=blue
}

% Theorem environments
\theoremstyle{plain}
\newtheorem{theorem}{Theorem}[section]
\newtheorem{lemma}[theorem]{Lemma}
\newtheorem{proposition}[theorem]{Proposition}
\newtheorem{corollary}[theorem]{Corollary}

\theoremstyle{definition}
\newtheorem{definition}[theorem]{Definition}
\newtheorem{example}[theorem]{Example}
\newtheorem{remark}[theorem]{Remark}

% Title information
\title{The Riemann Hypothesis in Band-Limited Subsystems: Geometric Formulation and Relative Unprovability}
\author{Haobo Ma$^1$ \and Wenlin Zhang$^2$\\
\small $^1$Independent Researcher\\
\small $^2$National University of Singapore}

\date{\today}

\begin{document}

\maketitle

\begin{abstract}
Under the unified syntax of ``windowed scattering--Toeplitz/Berezin compression--trinity master scale,'' we geometrize the Riemann Hypothesis equivalently into two global statements: first, that the readout quadratic form induced by the explicit formula is globally non-negative on the self-convolution positive-definite kernel cone; second, that the generator spectrum co-calibrated with it is strictly localized on the critical axis. In an axiomatic subsystem admitting only finite windows, finite bandwidth, and finite-order Euler--Maclaurin corrections, we establish two relative unprovability theorems: (A) no matter how many in-band kernels and finite-order error ledgers are used, global positivity cannot be decided within this subsystem; (B) under the heat-flow threshold formulation, the single-point threshold is likewise undecidable within this subsystem. The proof relies on the master-scale identity
$$
\frac{\varphi'(E)}{\pi}=\rho_{\text{rel}}(E)=\frac{1}{2\pi}\operatorname{tr}\,\mathsf{Q}(E),
$$
the windowed Helffer--Sjöstrand trace formula, the positivity-preservation criterion for Toeplitz/Berezin compression, and the Poisson--finite-order Euler--Maclaurin (NPE) discipline. The master-scale identity is concatenated via the Birman--Kreĭn relation between the Wigner--Smith group delay and Kreĭn spectral shift function, consistent with standard literature on ``phase--density of states--group delay--trace formula.''
\end{abstract}

\noindent\textbf{Keywords:} Riemann Hypothesis geometrization; Explicit formula positivity; Toeplitz/Berezin compression; Windowed Helffer--Sjöstrand trace formula; Nyquist--Poisson--Euler--Maclaurin finite-order discipline; Band-limited incompleteness; Heat-flow threshold

\noindent\textbf{MSC 2020:} Primary 11M26, 42A38, 47B35; Secondary 81U15, 94A20

\section{Notation and Axioms}

\subsection{Observation Triple and Master Scale}

Take energy-parameterized scattering matrix family $S(E)\in\mathsf{U}(N)$ ($E\mapsto S(E)$ differentiable), define Wigner--Smith time-delay matrix
$$
\mathsf{Q}(E)=-\,i\,S(E)^\dagger \partial_E S(E).
$$
Let total scattering phase $\varphi(E)=\tfrac{1}{2}\arg\det S(E)$ and relative state density $\rho_{\text{rel}}(E)$. This paper uses the following master-scale identity as calibration:
$$
\boxed{\ \frac{\varphi'(E)}{\pi}=\rho_{\text{rel}}(E)=\frac{1}{2\pi}\operatorname{tr}\,\mathsf{Q}(E)\ }.
$$
The right-hand side relation ``$\operatorname{tr}\,\mathsf{Q}$ equals (relative) density of states'' is widely used in quantum scattering and mesoscopic transport; the left-hand side is aligned via the Birman--Kreĭn formula $\det S(E)=e^{-2\pi i\,\xi(E)}$ and $\rho_{\text{rel}}=-\xi'$ \cite{cunden2015statistical}.

\subsection{Windowed Trace Formula}

For windowed functional calculus (Helffer--Sjöstrand) of $f\in C_c^\infty(\mathbb{R})$, we have
$$
\mathrm{Tr}\big(f(H)-f(H_0)\big)=\int_\mathbb{R} f'(E)\,\xi(E)\,dE,
$$
which can be equivalently stated as
$$
\mathrm{Tr}\big(f(H)-f(H_0)\big)=\frac{1}{2\pi i}\int_{\mathbb{R}} f(E)\,\partial_E\log\det S(E)\,dE,
$$
obtained from integration by parts and the Birman--Kreĭn relation $\partial_E\log\det S=-2\pi i\,\xi'$. Thus at the windowed level, ``trace--phase--shift--group delay--density of states'' are co-calibrated and measurable \cite{gesztesy1999krein}.

\subsection{Toeplitz/Berezin Compression and Positivity Preservation}

In Hardy/Bergman spaces, let $\Pi_w$ denote the projection associated with window $w$ and $M_h$ denote the multiplier; Toeplitz/Berezin compression is defined as
$$
K_{w,h}:=\Pi_w M_h \Pi_w .
$$
When symbol $h\ge 0$ and the measure satisfies the Carleson condition, $K_{w,h}$ is a positive operator; the corresponding linear readout functional preserves positivity \cite{zhao2014positivity}.

\subsection{NPE (Nyquist--Poisson--Euler--Maclaurin) Finite-Order Discipline}

The discrete--continuous bridge is connected by Poisson summation and Euler--Maclaurin correction of at most order $p$, with tail controlled by window decay and higher-order derivatives. The finite-bandwidth assumption comes from the classical framework of Nyquist--Shannon sampling and Landau density conditions \cite{shannon1949communication}.

\section{Geometrization Propositions and Equivalence}

\subsection{Self-Convolution Positive-Definite Kernel Cone and Readout Quadratic Form}

Let
$$
\mathcal{C}_+:=\{\,k=\varphi\ast\tilde\varphi:\ \varphi\in\mathcal{S}(\mathbb{R})\,\},\qquad
\widehat{k}(\omega)=|\widehat{\varphi}(\omega)|^2\ge 0.
$$
Define readout
$$
\mathcal{D}[k]:=\int_\mathbb{R} k(E)\,\rho_{\text{rel}}(E)\,dE
=\frac{1}{2\pi i}\int_{\mathbb{R}} k(E)\,\partial_E\log\det S(E)\,dE
=-\frac{1}{2\pi i}\int_{\mathbb{R}} k'(E)\,\log\det S(E)\,dE.
$$
Via windowed trace formula and integration by parts, the above three formulations are equivalent \cite{cunden2015statistical}.

\subsection{Two Formulations of Geometrized RH}

\begin{itemize}
\item \textbf{(P) Global Positivity Formulation:} For all $k\in\mathcal{C}_+$, $\mathcal{D}[k]\ge 0$.
\item \textbf{(S) Spectral Localization Formulation:} There exists a generator $\mathcal{D}_{\text{geom}}$ whose spectrum lies entirely on $\{\tfrac{1}{2}+i\mathbb{R}\}$, aligned with $\rho_{\text{rel}}$ under windowed readout.
\end{itemize}

These two are equivalent to Weil's explicit formula positivity criterion: RH is equivalent to non-negativity of the distribution formed by the explicit formula on the cone of convolution squares of test functions \cite{weil1952formules}.

\section{Finite-Resource Subsystem $\mathsf{T}_{\text{win}}(R,\Omega,p)$}

\subsection{Definition}

Given $(R,\Omega,p)\in(0,\infty)^3$, $\mathsf{T}_{\text{win}}(R,\Omega,p)$ admits the following primitives:
\begin{enumerate}
\item[(a)] Only finite sets of window--kernel pairs $\{(w_{R_j},h_j)\}_{j=1}^{m}$ generating kernels
$$
k_j:=w_{R_j}\!\ast \check{h}_j,\qquad \operatorname{supp}\widehat{k_j}\subset[-\Omega,\Omega];
$$
\item[(b)] All readouts realized via master scale and windowed trace formula;
\item[(c)] Discrete--continuous bridge uses only Poisson and Euler--Maclaurin of at most order $p$, errors capped by uniform bounds \cite{miller2003summation}.
\end{enumerate}

\subsection{Interpretation}

$\mathsf{T}_{\text{win}}$ can access evidence only in the form of finitely many in-band readouts $\{\mathcal{D}[k_j]\}_{j=1}^m$ and their finite-order error bounds. Thus it cannot distinguish two ``spectral data'' that agree on $[-\Omega,\Omega]$ but differ out-of-band. This is the core of band-limited incompleteness \cite{landau1967necessary}.

\section{Band-Limited Incompleteness}

\begin{theorem}[Band-Limited Incompleteness]
\label{thm:incompleteness}
Given any finite set $\{k_j\}_{j=1}^m\subset\mathcal{C}_+$ and $\varepsilon>0$, there exist two windowed-equivalent spectral densities $\rho_{\text{rel}}^{\pm}$ satisfying for all $1\le j\le m$,
$$
\Bigl|\int k_j \rho_{\text{rel}}^{+}-\int k_j \rho_{\text{rel}}^{-}\Bigr|<\varepsilon,
$$
but there exists some $k_\star\in\mathcal{C}_+$ such that
$$
\int k_\star \rho_{\text{rel}}^{+}>0,\qquad \int k_\star \rho_{\text{rel}}^{-}<0.
$$
Thus $\mathsf{T}_{\text{win}}(R,\Omega,p)$ cannot decide with finite evidence the truth value of ``for all $k\in\mathcal{C}_+$, $\mathcal{D}[k]\ge 0$.''
\end{theorem}

\begin{proof}
Let uniform bandwidth $\Omega=\max_j\mathrm{band}(\widehat{k_j})$. Take $\widehat{\psi}\in C_c^\infty(\mathbb{R})$ real even and non-negative, with
$$
\operatorname{supp}\widehat{\psi}\subset\{|\omega|\ge 2\Omega\},
$$
and denote its inverse transform by $\psi$. Let $\widehat{\eta}:=|\widehat{\psi}|^2$, denoting its inverse transform by $\eta$. Given baseline $\rho_0\in\mathcal{S}'(\mathbb{R})$ and small parameter $\delta>0$, set
$$
\rho_{\text{rel}}^{\pm}=\rho_0\pm\delta\,\eta.
$$
For each in-band kernel $k_j$, by disjoint Fourier support and Parseval identity,
$$
\int k_j \eta=\frac{1}{2\pi}\int \widehat{k_j}(\omega)\,\widehat{\eta}(\omega)\,d\omega=0,
$$
thus
$$
\int k_j\rho_{\text{rel}}^{+}=\int k_j\rho_{\text{rel}}^{-}=\int k_j\rho_0 .
$$
Moreover, by NPE finite-order discipline, any error ledger formed by combinations of Poisson and at most order $p$ Euler--Maclaurin depends only on in-band information and endpoint derivatives, so we can adjust $\delta$ and window parameters to make the two sides differ by $<\varepsilon$ in all visible evidence \cite{miller2003summation}.

On the other hand, take $\varphi_\star=\psi$ and set $k_\star=\varphi_\star\ast\tilde{\varphi}_\star\in\mathcal{C}_+$. Then
$$
\int k_\star \eta=\frac{1}{2\pi}\int |\widehat{\psi}(\omega)|^4\,d\omega>0,
$$
thus
$$
\int k_\star \rho_{\text{rel}}^{+}-\int k_\star \rho_{\text{rel}}^{-}=2\delta\int k_\star \eta
=\frac{\delta}{\pi}\int |\widehat{\psi}(\omega)|^4\,d\omega\neq 0.
$$
With appropriate sign of $\delta$, one obtains one positive and one negative. Theorem proved.
\end{proof}

\begin{remark}
The above construction essentially employs both Paley--Wiener and Bochner theorems: Fourier images of kernels in $\mathcal{C}_+$ are non-negative; out-of-band perturbations can be orthogonal to them; band-limited measurements are ``blind'' to out-of-band perturbations \cite{paley1934fourier}.
\end{remark}

\section{Relative Unprovability of Heat-Flow Threshold}

Let log-frequency-domain heat kernel $h_\lambda=\exp(\lambda\partial_E^2)$ ($\lambda\ge 0$), define
$$
F(\lambda):=\inf_{\substack{k\in\mathcal{C}_+\\ \int_{\mathbb{R}}k(E)\,dE=1}}\ \int_{\mathbb{R}} (k\ast h_\lambda)(E)\,\rho_{\text{rel}}(E)\,dE.
$$

\begin{lemma}[Monotonicity and Strong Smoothing Limit]
\label{lem:monotonicity}
Let $\mathcal{K}=\{k\in\mathcal{C}_+:\int k=1\}$. Since $h_{\lambda+\tau}=h_\lambda\ast h_\tau$ and $\int h_\tau=1$, the map $k\mapsto k\ast h_\tau$ maps $\mathcal{K}$ back to itself, thus
$$
F(\lambda+\tau)=\inf_{k\in\mathcal{K}}\langle k\ast h_\tau,\rho\ast h_\lambda\rangle
\ \ge\ \inf_{k\in\mathcal{K}}\langle k,\rho\ast h_\lambda\rangle
\ =\ F(\lambda).
$$
If $\int\rho_{\text{rel}}=0$, then $\rho\ast h_\lambda\to 0$, thus $\lim_{\lambda\to\infty}F(\lambda)=0$.
\end{lemma}

\begin{theorem}[Heat-Flow Single-Point Threshold Undecidable in $\mathsf{T}_{\text{win}}$]
\label{thm:threshold}
In $\mathsf{T}_{\text{win}}(R,\Omega,p)$, the truth value of $F(0)\ge 0$ cannot be decided with finite evidence.
\end{theorem}

\begin{proof}
Following the construction of Theorem \ref{thm:incompleteness}, take $\rho_{\text{rel}}^{\pm}=\rho_0\pm \delta\,\eta$ with $\operatorname{supp}\widehat{\eta}\subset\{|\omega|\ge 2\Omega\}$. For each in-band kernel $k_j$ and any $\lambda>0$,
$$
\Bigl|\int (k_j\ast h_\lambda)\,\eta\Bigr|
=\Bigl|\frac{1}{2\pi}\int \widehat{k_j}(\omega)\,e^{-\lambda\omega^2}\,\widehat{\eta}(\omega)\,d\omega\Bigr|
\le C\,e^{-4\lambda\Omega^2},
$$
so when $\lambda$ is sufficiently large, the two data are indistinguishable in $\mathsf{T}_{\text{win}}$'s visible evidence and yield the same non-negative limit (Lemma \ref{lem:monotonicity}). However, at $\lambda=0$,
$$
\int k_\star \rho_{\text{rel}}^{+}-\int k_\star \rho_{\text{rel}}^{-}=\frac{\delta}{\pi}\int |\widehat{\psi}(\omega)|^4\,d\omega\neq 0,
$$
thus there exists $k_\star$ such that the readouts of the two data have opposite signs at the single-point threshold. Since $\mathsf{T}_{\text{win}}$ can only access in-band evidence, it cannot establish the truth value of the proposition $F(0)\ge 0$ within this subsystem with finite evidence. Proof complete.
\end{proof}

\begin{remark}
The ``non-increase of extractable dissipation'' tendency of heat-kernel smoothing is coordinated with time delay--density of states in quantum scattering and entropy production monotonicity of non-equilibrium quantum semigroups \cite{martin2016wigner}.
\end{remark}

\section{Closure Relation with Geometrized RH}

Without imposing ``finite window--finite bandwidth--finite order'' constraints, working over all of $\mathcal{C}_+$:

\begin{itemize}
\item Formulation (P)'s global positivity is equivalent to Weil's explicit formula positivity criterion, thus equivalent to RH;
\item Formulation (S) corresponds to ``generator spectrum on critical axis'' scattering--spectral localization, consistently aligned with $\rho_{\text{rel}}$ under windowed readout via master scale and trace formula.
\end{itemize}

This paper's theorems show: under finite-resource constraints of $\mathsf{T}_{\text{win}}(R,\Omega,p)$, both are merely relatively unprovable propositions \cite{weil1952formules}.

\section{Discussion and Extensions}

\begin{enumerate}
\item \textbf{Band-Limited Incompleteness and Sampling Density:} The ``finite visible evidence'' of $\mathsf{T}_{\text{win}}$ aligns with Landau necessary density (Paley--Wiener spaces): finite in-band information is insufficient to uniquely recover out-of-band features, equivalent to ill-posedness of sampling--reconstruction \cite{landau1967necessary}.

\item \textbf{Toeplitz/Berezin and Carleson:} This paper only uses the basic fact ``non-negative symbol $\Rightarrow$ compression preserves positivity, Carleson $\Rightarrow$ boundedness,'' all valid in large-scale Bergman/Hardy spaces \cite{zhao2014positivity}.

\item \textbf{Universality of Master Scale:} $\rho_{\text{rel}}=(2\pi)^{-1}\operatorname{tr}\mathsf{Q}$ and $\varphi'=\pi\,\rho_{\text{rel}}$ close via the Wigner--Smith--Friedel--Birman--Kreĭn system, the core of scattering--DOS duality \cite{martin2016wigner}.
\end{enumerate}

\appendix

\section{Proof of Master-Scale Identity}

Let $\xi(E)$ be the Kreĭn spectral shift; the Birman--Kreĭn formula yields $\det S(E)=e^{-2\pi i\,\xi(E)}$, thus
$$
\partial_E\log\det S(E)=-2\pi i\,\xi'(E).
$$
On the other hand, Wigner--Smith defines $\mathsf{Q}(E)=-\,i\,S^\dagger S'$, thus
$$
\operatorname{tr}\,\mathsf{Q}(E)=-\,i\,\partial_E\log\det S(E)=\partial_E\arg\det S(E)=-\,2\pi\,\xi'(E).
$$
Setting $\rho_{\text{rel}}=-\xi'$, we obtain
$$
\frac{1}{2\pi}\operatorname{tr}\,\mathsf{Q}(E)=\rho_{\text{rel}}(E),\qquad
\varphi'(E)=\tfrac{1}{2}\partial_E\arg\det S(E)=\pi\,\rho_{\text{rel}}(E).
$$
Consistent with Friedel's rule ``total phase derivative--density of states,'' thus establishing the master-scale identity \cite{birman1962theory}.

\section{Windowed Helffer--Sjöstrand Trace Formula and Readout Co-Calibration}

Helffer--Sjöstrand functional calculus gives for $f\in C_c^\infty(\mathbb{R})$
$$
f(H)=\frac{1}{\pi}\int_{\mathbb{C}}\bar{\partial}\tilde{f}(z)\,(z-H)^{-1}\,dx\,dy,
$$
therefore
$$
\mathrm{Tr}\big(f(H)-f(H_0)\big)=\int f'(E)\,\xi(E)\,dE.
$$
Taking $f'=k$ belonging to the measurable class generated by window--kernel convolutions, combining Appendix A's master-scale identity, we obtain
$$
\int k(E)\,\rho_{\text{rel}}(E)\,dE=\frac{1}{2\pi}\int k(E)\,\operatorname{tr}\,\mathsf{Q}(E)\,dE,
$$
i.e., ``readout $=$ spectral measure linear functional'' \cite{helffer1989equation}.

\section{Positivity Preservation and Carleson Condition for Toeplitz/Berezin Compression}

On Bergman/Hardy spaces, Toeplitz operator $T_\mu f=\Pi (f\,d\mu)$. If $\mu$ is positive and is a Carleson measure for the respective space, then $T_\mu$ is bounded and positive; if symbol $h\ge 0$, then $\Pi M_h\Pi\ge 0$. This ensures positivity of the readout quadratic form induced by $(w_R\!\ast\check{h})$ \cite{pau2015carleson}.

\section{NPE Finite-Order Error Closure}

\begin{itemize}
\item \textbf{Poisson:} For Schwartz function $s$ and its Fourier transform $S$, periodization--sampling are rigorously connected by Poisson formula.
\item \textbf{Euler--Maclaurin (at most order $p$):} Remainder of difference--integral conversion given by endpoint higher-order derivatives and Bernoulli polynomials, with explicit upper bounds; finite-order ledger cannot ``recover'' out-of-band information.
\item \textbf{Conclusion:} Any procedure relying only on $[-\Omega,\Omega]$ in-band spectrum and finite-order endpoint corrections is blind to perturbations $\eta$ with ``support off-band'' \cite{miller2003summation}.
\end{itemize}

\section{High-Frequency Invisible Perturbations and Separating Hyperplanes}

Let $\eta$'s Fourier support be disjoint from $[-\Omega,\Omega]$. For any in-band kernel $k$, $\langle k,\eta\rangle=0$. Meanwhile take $k_\star=\varphi_\star\ast\tilde{\varphi}_\star$, where we first write $\widehat{\eta}=|\widehat{\psi}|^2$ ($\widehat{\psi}\in C_c^\infty$ real even and non-negative, support off-band), then set $\varphi_\star=\psi$. Thus
$$
\langle k_\star,\eta\rangle=\frac{1}{2\pi}\int |\widehat{\psi}(\omega)|^4\,d\omega>0.
$$
Thus from $\rho_{\text{rel}}^{\pm}=\rho_0\pm\delta\,\eta$ one can construct a handle on $\mathcal{C}_+$ such that finite evidence agrees while global positivity yields opposite conclusions. This separation construction relies on Bochner theorem ($\widehat{k}\ge 0$) and Paley--Wiener (controllable support) \cite{bochner1959lectures}.

\section{Heat-Flow Threshold Limit and Monotonicity}

Let $\mathcal{K}=\{k\in\mathcal{C}_+:\int k=1\}$. Using
$$
\int (k\ast h_\lambda)\rho=\int k\,(\rho\ast h_\lambda)
$$
and $\int(k\ast h_\tau)=\int k$, $\mathcal{K}$ is invariant under $*h_\tau$, thus $F(\lambda+\tau)\ge F(\lambda)$; if $\int\rho_{\text{rel}}=0$, then $\rho\ast h_\lambda\to 0$, thus $\lim_{\lambda\to\infty}F(\lambda)=0$ \cite{birman1962theory}.

\section*{Acknowledgments}

The ``windowed scattering--Toeplitz/Berezin compression--master scale'' syntax and several anchor formulas used in this paper can all be found in equivalent forms in standard literature. The term ``unprovability'' in this paper is relative to the explicitly defined finite-resource subsystem $\mathsf{T}_{\text{win}}(R,\Omega,p)$ and does not involve stronger metalogical unprovability.

\bibliographystyle{plain}
\begin{thebibliography}{99}

\bibitem{birman1962theory}
M. Sh. Birman and M. G. Kreĭn.
\newblock On the theory of wave and scattering operators.
\newblock {\em Doklady Akademii Nauk SSSR}, 144:475--478, 1962.

\bibitem{yafaev2010mathematical}
D. R. Yafaev.
\newblock {\em Mathematical Scattering Theory: General Theory}.
\newblock American Mathematical Society, 2010.

\bibitem{helffer1989equation}
B. Helffer and J. Sjöstrand.
\newblock Equation de Schrödinger avec champ magnétique et équation de Harper.
\newblock In {\em Schrödinger Operators}, pages 118--197. Springer, 1989.

\bibitem{dimassi1999spectral}
M. Dimassi and J. Sjöstrand.
\newblock {\em Spectral Asymptotics in the Semi-classical Limit}.
\newblock Cambridge University Press, 1999.

\bibitem{gesztesy1999krein}
F. Gesztesy, A. Pushnitski, and B. Simon.
\newblock On the Koplienko spectral shift function, I. Basics.
\newblock {\em Zhurnal Matematicheskoi Fiziki, Analiza, Geometrii}, 4(1):63--107, 2008.

\bibitem{cunden2015statistical}
F. D. Cunden, F. Mezzadri, N. Simm, and P. Vivo.
\newblock Statistical distribution of the Wigner-Smith time-delay matrix moments for chaotic cavities.
\newblock {\em Physical Review E}, 91(6):060102, 2015.

\bibitem{martin2016wigner}
P. A. Martin.
\newblock Wigner time delay and related concepts: Application to transport in coherent conductors.
\newblock arXiv:1507.00075, 2016.

\bibitem{weil1952formules}
A. Weil.
\newblock Sur les ``formules explicites'' de la théorie des nombres premiers.
\newblock {\em Communications du séminaire mathématique de l'université de Lund}, pages 252--265, 1952.

\bibitem{conrey2019weil}
B. Conrey.
\newblock Weil's explicit formula and positivity criterion.
\newblock Lecture Notes, 2019.

\bibitem{zhao2014positivity}
X. Zhao and D. Zheng.
\newblock Positivity of Toeplitz operators via Berezin transform.
\newblock {\em Journal of Mathematical Analysis and Applications}, 416(2):881--900, 2014.

\bibitem{pau2015carleson}
J. Pau and J. A. Peláez.
\newblock Carleson measures and Toeplitz operators for weighted Bergman spaces on the unit ball.
\newblock {\em Michigan Mathematical Journal}, 64(4):759--796, 2015.

\bibitem{shannon1949communication}
C. E. Shannon.
\newblock Communication in the presence of noise.
\newblock {\em Proceedings of the IRE}, 37(1):10--21, 1949.

\bibitem{landau1967necessary}
H. J. Landau.
\newblock Necessary density conditions for sampling and interpolation of certain entire functions.
\newblock {\em Acta Mathematica}, 117(1):37--52, 1967.

\bibitem{miller2003summation}
S. D. Miller and W. Schmid.
\newblock Summation formulas, from Poisson and Voronoi to the present.
\newblock In {\em Noncommutative Harmonic Analysis}, pages 419--440. Birkhäuser, 2004.

\bibitem{paley1934fourier}
R. E. A. C. Paley and N. Wiener.
\newblock {\em Fourier Transforms in the Complex Domain}.
\newblock American Mathematical Society, 1934.

\bibitem{bochner1959lectures}
S. Bochner.
\newblock {\em Lectures on Fourier Integrals}.
\newblock Princeton University Press, 1959.

\bibitem{connes2021weil}
A. Connes and C. Consani.
\newblock Weil positivity and trace formula.
\newblock arXiv:2104.12075, 2021.

\end{thebibliography}

\end{document}

