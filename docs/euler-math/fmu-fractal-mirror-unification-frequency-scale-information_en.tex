\documentclass[12pt]{article}

% Essential packages
\usepackage[utf8]{inputenc}
\usepackage{amsmath,amssymb,amsthm}
\usepackage{mathrsfs}
\usepackage{geometry}
\usepackage{hyperref}

% Geometry settings
\geometry{a4paper, margin=1in}

% Hyperref settings
\hypersetup{
    colorlinks=true,
    linkcolor=blue,
    citecolor=blue,
    urlcolor=blue
}

% Theorem environments
\theoremstyle{plain}
\newtheorem{theorem}{Theorem}[section]
\newtheorem{lemma}[theorem]{Lemma}
\newtheorem{proposition}[theorem]{Proposition}
\newtheorem{corollary}[theorem]{Corollary}

\theoremstyle{definition}
\newtheorem{definition}[theorem]{Definition}
\newtheorem{example}[theorem]{Example}
\newtheorem{remark}[theorem]{Remark}

% Title information
\title{FMU: Fractal-Mirror Unification of Frequency--Scale--Information}
\author{Haobo Ma$^1$ \and Wenlin Zhang$^2$\\
\small $^1$Independent Researcher\\
\small $^2$National University of Singapore}

\date{\today}

\begin{document}

\maketitle

\begin{abstract}
Within weighted Mellin--Hilbert space $L^2(\mathbb{R}_{+},dt/t)$, we establish rigorous theory of \textbf{fractal mirror} (FM) signal families generated by \textbf{mother function} $M$ through \textbf{multiplicative scale replication}. Around three main threads ``frequency--scale--information'', we provide equivalent characterizations and complete proofs: (A) Under \textbf{Mellin--Calder\'on} condition, \textbf{multiplicative self-similarity $\Longleftrightarrow$ Mellin-domain quasi-periodicity} (critical line exhibits equidistant frequency-shift array); (B) \textbf{spectral power-law $\Longleftrightarrow$} (in logarithmic scale) \textbf{entropy slope first-order approximate linearity}, and in self-affine model derive classical relation $D=(3-\beta)/2$ for image box dimension; (C) With \textbf{Nyquist--Poisson--Euler--Maclaurin (three-fold decomposition)} achieve \textbf{non-asymptotic error closure} and \textbf{separable budget}. Further, with \textbf{spectral density weight measure} $d\mu=(1/\pi)\,\Im m(\omega+i0)\,d\omega$, isometrically embed FM subspace into Paley--Wiener type bandlimited space, thereby deriving \textbf{Landau} type sampling/interpolation \textbf{density thresholds}, \textbf{Wexler--Raz} tight/dual criteria, and \textbf{Balian--Low} impossibility at critical density; prove this spectral density weight consistent with Herglotz representation of Weyl--Titchmarsh $m$-function of one-dimensional self-adjoint canonical systems. Above scales and criteria compatible with \textbf{established standards} including Mellin isometry, Paley--Wiener, Poisson summation, Euler--Maclaurin, Landau density, Wexler--Raz and Balian--Low, Herglotz representation, and de Branges inverse spectral theorem.
\end{abstract}

\noindent\textbf{Keywords:} Fractal mirror; Mellin transform; Quasi-periodicity; Power-law spectrum; Entropy-slope coupling; Nyquist--Poisson--EM three-fold; Landau density; Wexler--Raz; Balian--Low; Herglotz; de Branges

\section*{0. Notation \& Baseplates}

\subsection*{(0.1) Mellin Isometry and Critical Line}

For $x\in L^2(\mathbb{R}_{+},dt/t)$, Mellin transform
\begin{equation}
\mathcal{M}x(s)=\int_{0}^{\infty} x(t)\,t^{s-1}\,dt,
\end{equation}
on critical line $s=\tfrac12+i\omega$ isometric with ``logarithmic Fourier'' unit:
\begin{equation}
|x|_{L^2(dt/t)}^2=\frac{1}{2\pi}\int_{\mathbb{R}}\big|\mathcal{M}x(\tfrac12+i\omega)\big|^2\,d\omega.
\end{equation}
Thus $\widetilde{\mathcal M}:x\mapsto (2\pi)^{-1/2}\mathcal{M}x(\tfrac12+i\cdot)$ is unitary isometry on $L^2(\mathbb{R}_{+},dt/t)\to L^2(\mathbb{R})$.

\subsection*{(0.2) Logarithmic Variable and Scaling}

Take $u=\log t$, let $m(u)=M(e^{u})$. Scale $t\mapsto 2^{k}t$ becomes logarithmic translation $u\mapsto u+k\log 2$. Mellin scaling law yields
\begin{equation}
\mathcal{M}{M(2^k\cdot)}(s)=2^{-ks}\,\mathcal{M}M(s)\quad(s=\tfrac12+i\omega),
\end{equation}
i.e., amplitude factor $2^{-k/2}$ and phase modulation $e^{-ik\omega\log 2}$.

\subsection*{(0.3) Spectral Density Measure and Phase Coordinate}

\textbf{Notation and transform convention:} $\mathcal{M}$ denotes Mellin transform with respect to $t$ evaluated at $s=\tfrac12+i\omega$; $\widehat{\cdot}$ universally denotes Fourier transform with respect to $u=\log t$ (equivalent to $\omega$-domain operator along critical line).

For systems associated with Weyl--Titchmarsh $m$-function, define \textbf{spectral density weight measure}
\begin{equation}
d\mu(\omega)=\frac{1}{\pi}\,\Im m(\omega+i0)\,d\omega.
\end{equation}
When $\Im m\ge 0$, $\mu$ is positive measure. Under special normalization (e.g., certain boundary conditions making $\Re m(\lambda+i0)\equiv 0$ a.e.), can use phase representation $d\mu_\varphi(\omega)=(1/\pi)\,d(\arg m(\omega+i0))$; general case uses spectral density weight $d\mu$. Consistency with spectral measure of self-adjoint systems see §6.

\textbf{Convention ($u$-domain Fourier):} $\widehat{f}(\omega)=(2\pi)^{-1/2}\int_{\mathbb{R}}f(u)\,e^{-i\omega u}\,du$, $f(u)=(2\pi)^{-1/2}\int_{\mathbb{R}}\widehat{f}(\omega)\,e^{i\omega u}\,d\omega$.

\textbf{Notation Convention (Avoiding Conflict):} Throughout fix $u:=\log t$ as logarithmic time variable; spectral density weight coordinate denoted separately as $v_\mu$. $u$-domain Fourier only acts on $u$ variable; isometric maps and density criteria related to spectral density weight stated in $v_\mu$ coordinate.

\subsection*{(0.4) Finite-Order Euler--Maclaurin (EM)}

Throughout use only \textbf{finite-order} EM, decomposing ``sum--integral'' difference into endpoint Bernoulli layer and remainder:
\begin{equation}
\sum_{n=a}^{b}f(n)=\int_{a}^{b}f(x)\,dx+\frac{f(a)+f(b)}{2}+\sum_{r=1}^{p-1}\frac{B_{2r}}{(2r)!}\!\left(f^{(2r-1)}(b)-f^{(2r-1)}(a)\right)+R_p,
\end{equation}
with control $|R_p|\lesssim \tfrac{2\,\zeta(2p)}{(2\pi)^{2p}}\int_a^b|f^{(2p)}|$.

\section{Fractal Mirror (FM) Signal Family: Definition and Basic Properties}

\begin{definition}[FM Generation and Mellin--Calder\'on Condition]
Given mother function $M\in L^2(\mathbb{R}_{+},dt/t)$, weight sequence $\{a_k\}_{k\in\mathbb{Z}}\in \ell^2$, phase sequence $\{\phi_k\}\subset\mathbb{R}$ with $\sup_k|\phi_k|<\infty$. Define
\begin{equation}
x(t)=\sum_{k\in\mathbb{Z}} a_k\, M(2^{k}t)\,e^{i\phi_k},\qquad t>0.
\end{equation}

\textbf{Assumption (H0) Weighted $\ell^2$ Consistency.} Let $b_k:=a_k\,2^{-k/2}e^{i\phi_k}$, require $\{b_k\}_{k\in\mathbb{Z}}\in \ell^2$ (equivalently $\sum_k |a_k|^2\,2^{-k}<\infty$).

\textbf{Assumption (H1) Mellin--Calder\'on Boundedness.} Write $G(\omega):=\mathcal{M}M(\tfrac12+i\omega)$, $P:=2\pi/\log 2$, require Calder\'on sum
\begin{equation}
\mathcal{C}_G(\omega):=\sum_{n\in\mathbb{Z}}\big|G(\omega+nP)\big|^2\in L^{\infty}([0,P]).
\end{equation}
\end{definition}

\begin{proposition}[$L^2$ Unconditional Convergence and Frequency Structure, Under (H0)--(H1)]
Under assumptions (H0)--(H1), series $\sum_{k}a_k M(2^k\cdot)e^{i\phi_k}$ unconditionally converges in $L^2(\mathbb{R}_{+},dt/t)$, and
\begin{equation}
\mathcal{M}x\!\left(\tfrac12+i\omega\right)=G(\omega)\sum_{k\in\mathbb{Z}} b_k\,e^{-ik\omega\log 2},\qquad b_k:=a_k\,2^{-k/2}e^{i\phi_k}.
\end{equation}
With Bessel bound
\begin{equation}
|x|_{L^2(dt/t)}^2\ \le\ \frac{P}{2\pi}\|\mathcal{C}_G\|_{L^\infty([0,P])}\,\sum_{k\in\mathbb{Z}}|b_k|^2,\qquad P=\frac{2\pi}{\log 2}.
\end{equation}
\end{proposition}
\begin{proof}
By (H1)'s Calder\'on upper bound and Plancherel--Mellin isometry, obtain
\begin{equation}
|x|_{L^2(dt/t)}^2\ \le\ \frac{P}{2\pi}\,\|\mathcal{C}_G\|_{L^\infty([0,P])}\,\sum_{k\in\mathbb{Z}}|b_k|^2,\qquad P=\frac{2\pi}{\log2},
\end{equation}
thus series Cauchy converges; frequency expression follows directly from scaling law.
\end{proof}

\section{Main Theorem A: Mellin-Quasi-Periodic Characterization of Multiplicative Self-Similarity}

\begin{theorem}[Self-Similarity $\Longleftrightarrow$ Quasi-Periodicity]
For $x$ from Definition 1.1, following equivalent:
\begin{enumerate}
\item[(i)] $x(t)=\sum_k a_k M(2^k t)e^{i\phi_k}$ (multiplicative self-similar superposition);
\item[(ii)] Exists envelope $G(\omega)=\mathcal{M}M(\tfrac12+i\omega)\in L^2(\mathbb{R})$ such that
\begin{equation}
\mathcal{M}x\!\left(\tfrac12+i\omega\right)=G(\omega)\cdot \underbrace{\sum_{k\in\mathbb{Z}} b_k\,e^{-ik\omega\log 2}}_{\text{Bohr quasi-periodic; frequency-shift lattice spacing }2\pi/\log 2}.
\end{equation}
\end{enumerate}
\end{theorem}
\begin{proof}
``$\Rightarrow$'' by Proposition 1.2. ``$\Leftarrow$'' Apply inverse Mellin ($\sigma=\tfrac12$) to quasi-periodic part and use ``logarithmic-domain translation $\leftrightarrow$ Mellin frequency-shift'' duality, immediately obtain superposition of logarithmic translation family $M(2^k\cdot)$.
\end{proof}

\section{Main Theorem B: Spectral Power-Law--Entropy Slope and Self-Affine Dimension}

\subsection{Power-Law Spectrum and Logarithmic Binned Entropy Linear Coupling (Approximate Relation)}

\begin{proposition}[Entropy--Slope Approximate Linearity, First-Order Regime]
Let power spectrum over wide frequency ratio $\Lambda=f_{\max}/f_{\min}\gg1$ satisfy $S(f)\asymp C f^{-\beta}$, logarithmic uniform binning $I_j=[e^{y_j},e^{y_{j+1}}]$, $\delta y:=y_{j+1}-y_j\ll1$ fixed, bin number $J\sim (\log\Lambda)/\delta y$, probability $P_j\propto \int_{I_j}S(f)\,df$ normalized. Then in first-order approximation
\begin{equation}
H:=-\sum_j P_j\log P_j= c_1(\delta y) + c_2(\delta y)\,\beta + \mathcal{O}\big(\delta y\big) + \mathcal{O}\big((\log\Lambda)^{-1}\big),
\end{equation}
where $c_1,c_2$ depend only on binning step $\delta y$ and window overlap constants, explicitly computable; as $\delta y\to 0$, $\Lambda\to\infty$, main term linearity holds.
\end{proposition}
\begin{proof}[Proof Sketch]
Let $y=\log f$, have $df=f\,dy$, thus $P_j\propto\int_{y_j}^{y_{j+1}}e^{(1-\beta)y}\,dy$. Uniform $y$-binning makes $\{P_j\}$ approximately exponentially distributed; substitute back into entropy and approximate by Riemann sum, $\delta y$ discretization error yields $\mathcal{O}(\delta y)$ term, endpoints/overlap yield $\mathcal{O}((\log\Lambda)^{-1})$ term; both negligible when $\delta y\ll 1/\log\Lambda$.
\end{proof}

\subsection{Self-Affine Spline and Image Dimension (Canonical Model)}

\begin{theorem}[$D=(3-\beta)/2$]
If spline satisfies $x(\lambda t)\overset{d}{=}\lambda^{H}x(t)$ (e.g., fBm), then
\begin{equation}
S(f)\sim f^{-(2H+1)}\quad\Longrightarrow\quad D_{\mathrm{graph}}=2-H=\frac{3-\beta}{2}.
\end{equation}
\end{theorem}
\begin{proof}
Self-affine processes like fBm satisfy $S(f)\propto f^{-(2H+1)}$; their spline graph's Hausdorff/box dimension $D=2-H$ is classical result, jointly eliminate $H$ to obtain above formula. Related conclusions see Flandrin's analysis of fBm spectrum and Xiao's rigorous measure results for image dimension.
\end{proof}

\section{Main Theorem C: Nyquist--Poisson--EM Non-Asymptotic Error Closure}

Let target frequency-domain quantity written as
\begin{equation}
F(\omega)=\widehat{w}(\omega)\,\widehat{h}(\omega)\,X(\omega),\qquad X(\omega):=(2\pi)^{-1/2}\,\mathcal{M}x\!\left(\tfrac12+i\omega\right),
\end{equation}
where $\widehat{w}(\omega)$, $\widehat{h}(\omega)$ are analysis window and interpolation kernel frequency responses after $u$-domain Fourier transform per §0.3 convention, $X(\omega)$ is Mellin image along critical line (normalized by §0.1's unitary isometry).

\begin{theorem}[Nyquist Condition and Three-Fold Decomposition Error]
Suppose exists $B>0$ such that $\mathrm{supp}\,F\subset[-B,B]$, where $B:=\min\{\Omega_w,\Omega_h\}$ (if $X$ non-bandlimited, take effective bandwidth as intersection $\mathrm{supp}(\widehat{w})\cap\mathrm{supp}(\widehat{h})$). If sampling step
\begin{equation}
\Delta\ \le\ \frac{\pi}{B},
\end{equation}
then aliasing energy is zero. In general case, aliasing term written as
\begin{equation}
\varepsilon_{\text{alias}}=\Big|\sum_{\ell\neq 0}F\big(\cdot+2\pi\ell/\Delta\big)\Big|_{L^2([-\pi/\Delta,\, \pi/\Delta])},
\end{equation}
for bandlimited linear reconstruction operator $\mathsf E$, total error decomposes as
\begin{equation}
|x-\mathsf{E}x|_{L^2}\ \le\ \underbrace{\varepsilon_{\text{alias}}}_{\text{periodic superposition}}\ +\ \underbrace{\varepsilon_{\mathrm{EM}}^{(p)}}_{\mathcal{O}(|B_{2p}|\Delta^{2p})}\ +\ \underbrace{\varepsilon_{\text{tail}}}_{\text{band-edge/truncation}},
\end{equation}
and $\varepsilon_{\mathrm{EM}}^{(p)}=\mathcal{O}\!\big(|B_{2p}|\,\Delta^{2p}\big)$.
\end{theorem}
\begin{proof}
Use Poisson summation to convert discretization into spectrum periodization; Nyquist threshold eliminates inter-band overlap. Sum--integral difference decomposed by finite-order EM into endpoint Bernoulli layer and remainder, remainder bound see §0.4.
\end{proof}

\section{Sampling--Interpolation--Stability: Landau--Wexler--Raz--Balian--Low in Spectral Density Coordinate}

\subsection{Isometric Embedding of Spectral Density Weight Coordinate}

In region $\Im m(\omega+i0)>0$, define spectral density weight coordinate
\begin{equation}
v_\mu(\omega)=\frac{1}{\pi}\int_{-\infty}^{\omega}\Im m(\omega'+i0)\,d\omega',\qquad dv_\mu=\frac{1}{\pi}\,\Im m(\omega+i0)\,d\omega,
\end{equation}
then
\begin{equation}
\int_{\mathbb{R}}|X(\omega)|^2\,\frac{1}{\pi}\,\Im m(\omega+i0)\,d\omega=\int_{\mathbb{R}}|X(\omega(v_\mu))|^2\,dv_\mu,
\end{equation}
thus isometrically embed spectral density weight weighted FM-subspace into \textbf{unit bandwidth} Paley--Wiener type space. Under special normalization, can simplify to phase coordinate $v_\mu=\varphi(\omega)/\pi$. Paley--Wiener and Hardy/Mellin-Hardy structure in Mellin context see literature.

\subsection{Landau Type Necessary Condition (FM Version)}

\begin{theorem}[Necessary Density Threshold, $v_\mu$-Domain]
Let $\Omega=\{\omega_n\}$ be sampling sequence (respectively: interpolation sequence). In $v_\mu$ coordinate (defined in §5.1), its Beurling lower (respectively upper) density satisfies
\begin{equation}
\underline{D}_{\mu}(\Omega)\ge 1\qquad(\text{respectively },\overline{D}_{\mu}(\Omega)\le 1).
\end{equation}
\textit{Remark:} Above are necessary conditions. Sufficiency generally requires additional separation/stability structural conditions; in practice can design stable sampling/reconstruction via §5.3's WR/Parseval conditions.
\end{theorem}
\begin{proof}[Proof Sketch]
By 5.1's isometry, problem reduces to non-uniform sampling of unit bandwidth Paley--Wiener space, directly invoke Landau necessary density theorem.
\end{proof}

\subsection{Wexler--Raz and Parseval Tight Frame (Critical Nyquist)}

\begin{theorem}[WR/Parseval Condition]
Under Nyquist (non-aliasing) condition, system generated by multi-windows $\{w_\alpha\}_{\alpha=1}^r$ is Parseval tight frame if and only if
\begin{equation}
\frac{1}{\Delta}\sum_{\alpha=1}^{r}\big|\widehat{w_\alpha}(\xi)\big|^{2}\equiv 1\quad\text{(a.e.)}.
\end{equation}
With aliasing present, Parseval condition becomes
\begin{equation}
\frac{1}{\Delta}\sum_{\alpha=1}^{r}\sum_{m\in\mathbb{Z}}\big|\widehat{w_\alpha}\!\big(\xi+2\pi m/\Delta\big)\big|^{2}\equiv 1.
\end{equation}
Reconstruction with dual windows $\{\widetilde{w}_\alpha\}$ satisfies corresponding biorthogonality formula.
\end{theorem}
\begin{proof}[Proof Sketch]
WR identity yields frequency-domain pointwise orthogonality necessary and sufficient condition; through $u=\log t$ and phase coordinate transformation, losslessly transplants to Mellin/logarithmic model.
\end{proof}

\subsection{Balian--Low Type Impossibility (Critical Density)}

\begin{theorem}[BLT--Mellin Version]
At critical density $D=1$, if single window $w$ ``well-localized'' on both logarithmic time $u$ and frequency $\omega$ sides (e.g., finite second moment), then system generated by $w$ and critical lattice in $v_\mu$ coordinate \textbf{cannot} be Riesz basis; to obtain basis, must relax at least one side's localization or employ oversampling.
\end{theorem}
\begin{proof}[Proof Sketch]
Through $u=\log t$ and §5.1's isometric embedding, problem equivalent to standard Gabor lattice BLT; immediately follows from BLT's Riesz/ONB version.
\end{proof}

\section{Consistency with de Branges--Kreĭn / Weyl--Titchmarsh $m$-Function}

\begin{proposition}[Spectral Density Weight and Herglotz Representation]
In one-dimensional self-adjoint canonical system/Schr\"odinger type operator case, Weyl--Titchmarsh $m$ is Herglotz--Nevanlinna function, exists spectral measure $\mu$ such that
\begin{equation}
m(z)=a z+b+\int_{\mathbb{R}}\left(\frac{1}{\lambda-z}-\frac{\lambda}{1+\lambda^2}\right)\,d\mu(\lambda),
\end{equation}
boundary value satisfies $\Im m(\lambda+i0)=\pi\,\rho(\lambda)$ (a.e.), where $\rho$ is spectral density. Thereby define spectral density weight measure
\begin{equation}
d\mu(\omega)=\frac{1}{\pi}\,\Im m(\omega+i0)\,d\omega=\rho(\omega)\,d\omega,
\end{equation}
consistent with absolutely continuous spectral measure. Under special normalization (e.g., certain boundary conditions making $\Re m(\lambda+i0)\equiv 0$ a.e.), can simplify to phase representation $d\mu_\varphi=(1/\pi)\,d(\arg m)$; general case must use above spectral density weight.
\end{proposition}

\begin{remark}
This consistency ensures spectral density scale defined by $d\mu$ in this paper seamlessly interfaces with spectral theory of de Branges spaces/canonical systems, becoming natural coordinate for density and frame criteria in §5; phase derivative $\varphi'=d(\arg m)/d\omega$ generally also depends on $\Re m$ and $m'$, reduces to $\pi\rho$ only in special cases.
\end{remark}

\section{Reproducible Experimental Paradigm (Verification and Engineering)}

\textbf{P1 | Power-Law/Entropy Coupling.} Over sufficiently wide logarithmic bandwidth, fit $S(f)\propto f^{-\beta}$ and verify with logarithmic binned entropy $H$ that regression slope is linear within error band (§3.1's first-order approximation).

\textbf{P2 | Mellin Peak Array.} Compute $\mathcal{M}x(\tfrac12+i\omega)$, verify equidistant peak array and relative phase stability (§2).

\textbf{P3 | Sampling/Window Design.} Select lattice per §5.2's density threshold; tune windows per §5.3's WR condition to obtain Parseval; perform \textbf{separable budget} for error per §4's three-fold decomposition (Nyquist margin, EM order, band-edge tail term).

\section{Information Trinity $(i_{+},i_{0},i_{-})$ and Model Selection}

\begin{itemize}
\item $i_{+}$: Cross-scale overflow benefit ($\beta$ trending red $\Rightarrow$ low-frequency concentration $\Rightarrow$ compression/prediction benefit).
\item $i_{0}$: Intra-layer rearrangement (phase--coherence ``neutral'' redistribution).
\item $i_{-}$: Sparsity and complexity penalty (avoid overfitting/over-dense lattice).
\end{itemize}

\begin{proposition}[Strategy in Approximate Linear Regime]
In §3.1's first-order approximation regime, $\partial_\beta H\approx c_2$, effective layer number $N_{\mathrm{eff}}\asymp (1+\beta)\log \Lambda$. Accordingly jointly select ``window/lattice density--model complexity'' to maximize $i_{+}-i_{-}$ satisfying §4's three-fold decomposition budget.
\end{proposition}

\section{Interface with S-series / WSIG-QM / UMS}

\subsection{Interface with S24--S26}
\begin{itemize}
\item S24's fiber Gram characterization and Wexler--Raz biorthogonality provide concrete implementation framework for this paper's §5.3 WR condition.
\item S25's non-stationary Weyl--Mellin framework shares mathematical structure with this paper's Mellin isometry (§0.1) and logarithmic translation--frequency shift duality (§0.2).
\item S26's spectral density scale consistent at Herglotz representation level with this paper's §0.3 and §6's spectral density weight measure $d\mu=(1/\pi)\,\Im m(\omega+i0)\,d\omega$; S26's Landau necessary density, Balian--Low impossibility directly correspond to this paper's Theorems 5.1, 5.3.
\end{itemize}

\subsection{Interface with WSIG-QM}
\begin{itemize}
\item WSIG-QM's axiom A2 (finite window readout) shares Nyquist--Poisson--EM three-fold decomposition framework with this paper's §4 windowed reconstruction.
\item WSIG-QM's axiom A5 (phase--density--delay scale) consistent at spectral theory level with this paper's §0.3, §6's spectral density weight measure.
\item WSIG-QM's theorem T6 (window/kernel optimization) shares frame theory criteria with this paper's §5.2--5.3's Landau density threshold, WR condition.
\end{itemize}

\subsection{Interface with UMS}
\begin{itemize}
\item UMS's core unification formula $d\mu = \tfrac{1}{2\pi}\operatorname{tr}\mathsf Q\,dE = \rho_{\mathrm{rel}}\,dE$ in Mellin context corresponds to this paper's §0.3, §6's spectral density weight measure; under special normalization can simplify to phase representation.
\item UMS's axiom A2 (finite window readout) shares framework at numerical implementation level with this paper's Theorem 4.1's three-fold decomposition error closure.
\item UMS's axiom A6 (sampling--frame threshold) completely aligns with this paper's §5's Landau--Wexler--Raz--Balian--Low criteria.
\end{itemize}

\subsection{Interface with Windowed Path Integral Theory}
\begin{itemize}
\item Path integral theory's window--kernel duality (Theorem 2.1) can be rewritten in Mellin domain as this paper's Theorem 2.1's quasi-periodic formulation.
\item Path integral theory's Nyquist--Poisson--EM error closure consistent in discretization framework with this paper's Theorem 4.1's three-fold decomposition.
\end{itemize}

\subsection{Interface with Quantum Gravity Field Theory}
\begin{itemize}
\item Quantum gravity field theory's spectral density scale consistent in spectral shift context with this paper's §0.3, §6's spectral density weight measure $d\mu=(1/\pi)\,\Im m(\omega+i0)\,d\omega$.
\item Quantum gravity field theory's windowed sampling (§6.1) shares frame theory foundation with this paper's §5's Landau--Wexler--Raz criteria.
\end{itemize}

\subsection{Maintaining ``Poles = Principal Scales'' Finite-Order EM Discipline}
\begin{itemize}
\item Throughout all discrete--continuous exchanges, this paper employs \textbf{finite-order} EM (§0.4, Theorem 4.1), ensuring no new singularities introduced.
\item Consistent with S15--S26, WSIG-QM, UMS, path integral theory, quantum gravity field theory: EM remainder serves only as bounded perturbation.
\end{itemize}

\section*{Appendix A: Proof Details and Tools}

\textbf{A.1 Mellin--Hardy and Isometry.} $\widetilde{\mathcal{M}}$ is unitary isometry on $\sigma=\tfrac12$; construction of Mellin--Paley--Wiener and Mellin--Hardy spaces see Bardaro--Butzer--Mantellini--Schmeisser.

\textbf{A.2 Poisson and Aliasing.} Dirac comb of sampling step $\Delta$ in frequency domain is Dirac comb of period $2\pi/\Delta$; aliasing energy equals periodized side-spectrum superposition energy in main band.

\textbf{A.3 Euler--Maclaurin Remainder.} Employ DLMF version EM formula and remainder bound, ensuring finite-order approximation introduces no additional singularities; error given by Bernoulli numbers and step size.

\textbf{A.4 Landau Density.} Sampling (interpolation) sequences of Paley--Wiener space $PW_B$ must satisfy $\underline D\ge B/\pi$ ($\overline D\le B/\pi$); in this paper isometric to threshold 1 at unit bandwidth in $v_\mu$ coordinate.

\textbf{A.5 Wexler--Raz and BLT.} WR identity yields frequency-domain pointwise necessary and sufficient condition for tight/dual frames; BLT shows at critical density ``good double-sided localization + non-redundancy'' incompatible.

\textbf{A.6 Herglotz Representation and Spectral Density.} $m$ Herglotz function $\Rightarrow$ exists spectral measure representation; boundary imaginary part $\Im m(\lambda+i0)=\pi\rho(\lambda)$ (a.e.). Thereby define spectral density weight measure $d\mu=(1/\pi)\,\Im m\,d\omega$; phase derivative $\varphi'=d(\arg m)/d\omega$ generally also depends on $\Re m$ and $m'$, reduces to $\pi\rho$ only under special normalization.

\section*{Conclusion}

\begin{enumerate}
\item Multiplicative self-similarity via Mellin transform equivalent to \textbf{quasi-periodic frequency-shift array}, controlled by envelope $G$ (Theorem 2.1); under weighted $\ell^2$ consistency (H0) and Mellin--Calder\'on condition (H1), series unconditionally converges (Proposition 1.2).
\item Power-law spectrum and logarithmic binned entropy linearly coupled in first-order approximation; in self-affine limit $D=(3-\beta)/2$ holds (Proposition 3.1, Theorem 3.2).
\item Employing \textbf{Nyquist--Poisson--EM} three-fold decomposition, can stably decompose total error into ``aliasing/Bernoulli layer/tail term'' three-term budget (Theorem 4.1).
\item Spectral density weight coordinate $v_\mu$ makes FM subspace isometric with Paley--Wiener space, thereby inheriting \textbf{Landau} density threshold, \textbf{Wexler--Raz} tight/dual criteria, and \textbf{Balian--Low} impossibility; its spectral density weight measure consistent with Herglotz representation of $m$-function (§5--§6).
\end{enumerate}

\begin{thebibliography}{99}

\bibitem{MellinPW}
Bardaro, Butzer, Mantellini, Schmeisser, \textit{On the Paley--Wiener theorem in the Mellin transform setting} (2015) and sequel (2017).

\bibitem{MellinIsometry}
Butzer \& Jansche, \textit{A Direct Approach to the Mellin Transform} (1997).

\bibitem{Poisson}
NIST DLMF §1.8(iv); Cand\`es lecture notes (2021).

\bibitem{EM}
NIST DLMF §2.10(i), §24.17.

\bibitem{Landau}
Landau, \textit{Acta Math.} 117 (1967).

\bibitem{WR}
Daubechies \& Landau, \textit{J. Fourier Anal. Appl.} (1994/95).

\bibitem{BL}
Heil \& Powell, \textit{J. Math. Phys.} (2006).

\bibitem{fBm}
Flandrin (1989); Xiao (1997).

\bibitem{Herglotz}
Kostenko--Teschl et al. surveys and Oberwolfach reports.

\end{thebibliography}

\section*{Reader's Guide}

When embedding this paper into S25 (non-stationary Weyl--Mellin) and S26 (spectral density--de Branges), can directly reuse §5's spectral density weight coordinate $v_\mu$ criteria and §4's \textbf{three-fold decomposition error budget}; window/kernel design implemented via WR necessary and sufficient formula, critical density encountering BLT obstacle circumvented through \textbf{oversampling or relaxing localization}. In implementation note verification of weighted $\ell^2$ consistency (H0) and Mellin--Calder\'on condition (H1) (e.g., Log-Gaussian mother function) and entropy--slope relation's error band control (§3.1's first-order approximation range). Notation convention: logarithmic time $u:=\log t$ and spectral density weight coordinate $v_\mu$ strictly distinguished (§0.2--§0.3).

\end{document}

