\documentclass[12pt]{article}

% Essential packages
\usepackage[utf8]{inputenc}
\usepackage{amsmath,amssymb,amsthm}
\usepackage{mathrsfs}
\usepackage{geometry}
\usepackage{hyperref}

% Geometry settings
\geometry{a4paper, margin=1in}

% Hyperref settings
\hypersetup{
    colorlinks=true,
    linkcolor=blue,
    citecolor=blue,
    urlcolor=blue
}

% Theorem environments
\theoremstyle{plain}
\newtheorem{theorem}{Theorem}[section]
\newtheorem{lemma}[theorem]{Lemma}
\newtheorem{proposition}[theorem]{Proposition}
\newtheorem{corollary}[theorem]{Corollary}

\theoremstyle{definition}
\newtheorem{definition}[theorem]{Definition}
\newtheorem{example}[theorem]{Example}
\newtheorem{remark}[theorem]{Remark}

% Title information
\title{UMMIC: Mother Map--Mellin--de Branges Unified Theory of ``Information Conservation--Phase Density--Sampling Stability'' (With Complete Proofs)}
\author{Haobo Ma$^1$ \and Wenlin Zhang$^2$\\
\small $^1$Independent Researcher\\
\small $^2$National University of Singapore}

\date{\today\\
\small Version: v2.6}

\begin{document}

\maketitle

\begin{abstract}
Starting from mother map kernels satisfying moderate axioms, under the Mellin isometry on $L^2(\mathbb R_{+},dx/x)$ and the spectral dictionary of de Branges--Kreĭn canonical systems, we establish three parallel and closed main lines: (I) a \textbf{Noether-type flux continuity equation} (information conservation) with $\Lambda(s)$'s logarithmic potential as potential function; (II) \textbf{``phase density = spectral shift derivative = relative spectral density''} consistency (CCS) under self-adjoint scattering settings; (III) engineering-oriented \textbf{Nyquist--Poisson--Euler--Maclaurin (EM) three-fold decomposition} with \textbf{non-asymptotic} error closure. Accompanying this, under the parallel framework of positive-weight spectral density scale $d\mu(E)=\rho(E)\,dE$ and phase coordinate $v_\phi(E)=\delta(E)/\pi$, we introduce the relative spectral density $\rho_{\mathrm{rel}}(E)=\xi'(E)=-\tfrac1\pi\delta'(E)$ and its integral coordinate $v_{\mathrm{rel}}(E)=\xi(E)$, unifying engineering and spectral theory scales, and on this basis unify the proofs of \textbf{Landau} sampling/interpolation necessary density thresholds, \textbf{Wexler--Raz} tight/dual sufficiency conditions, and \textbf{Balian--Low} impossibility (Mellin/Weyl version), thereby welding ``mother map--scattering--frame/sampling--error theory'' into a non-asymptotic hard-core system.
\end{abstract}

\noindent\textbf{Keywords:} Mother map; Mellin transform; de Branges space; Information conservation; Phase density; Sampling theory; Wexler--Raz; Balian--Low; Nyquist--Poisson--Euler--Maclaurin

\section{Axioms, Objects, and Notation}

\textbf{A0 (Mother Map and Mellin Embedding)}
Take mother kernel $k_{\mathcal M}\in L^2(\mathbb R_{+},dx/x)$ with Mellin transform
\begin{equation}
Z_{\mathcal M}(s)=\int_{0}^{\infty}k_{\mathcal M}(x)\,x^{s-1}\,dx .
\end{equation}
Along the critical line $s=\tfrac12+i\omega$, there is an isometry
\begin{equation}
\int_{0}^{\infty} |k_{\mathcal M}(x)|^2\,\frac{dx}{x}
=\frac{1}{2\pi}\int_{\mathbb R}\big|Z_{\mathcal M}(\tfrac12+i\omega)\big|^2\,d\omega ,
\end{equation}
and scaling $k_{\mathcal M}(2^k\cdot)$ corresponds to frequency shift and amplitude factor $2^{-k/2}$. The above isometry is the standard statement of the Mellin--Plancherel theorem on $\tfrac12+i\mathbb R$.

\textbf{A1 (Completed Function and Mirror)}
There exists a normalization factor $\gamma_{\mathcal M}(s)$ such that $\Lambda_{\mathcal M}(s)=\gamma_{\mathcal M}(s)Z_{\mathcal M}(s)$ has well-defined phase $\varphi_{\mathcal M}(\omega)$ and modulus $R(\omega)$ on the critical line. Below we use only its boundary phase.

\textbf{A2 (Spectral Density and Weyl--Titchmarsh Dictionary)}
Under self-adjoint canonical system/one-dimensional Schrödinger-type backgrounds, Weyl--Titchmarsh $m$ is a Herglotz function, and the boundary imaginary part yields the absolutely continuous spectral density
\begin{equation}
\rho(E)=\frac1\pi \Im m(E+i0)\quad \text{(a.e.)} ,
\end{equation}
accordingly defining the spectral density weight measure $d\mu(E)=\rho(E)\,dE$.

\textbf{A3 (Density Coordinate, Phase Coordinate, and Isometry)}
Write $X(\omega)=(2\pi)^{-1/2}Z_{\mathcal M}(\tfrac12+i\omega)$. Write $dv_\mu(x)=\rho(x)\,dx$; when $x=E$, $\rho(E)=\tfrac1\pi \Im m(E+i0)$ is the \textbf{absolute} spectral density (positive). Define phase coordinate $v_\phi(E)=\delta(E)/\pi$. The relative spectral density is defined as $\rho_{\mathrm{rel}}(E)=\xi'(E)=-\tfrac1\pi\delta'(E)$.

In the $v_\mu$ coordinate there is an isometry
\begin{equation}
\int_{\mathbb R}|X(\omega)|^2\,\rho(\omega)\,d\omega
=\int_{\mathbb R}|X(\omega(v_\mu))|^2\,dv_\mu .
\end{equation}
Here $d\mu=\rho\,dE$ always takes positive weight, no longer identified with $\delta'$.

On each connected component of the absolutely continuous spectrum, $v_\mu$ is a non-decreasing function with a measurable right inverse for change of variables; accordingly $\int |X(\omega)|^2\rho(\omega)\,d\omega=\int |X(\omega(v_\mu))|^2\,dv_\mu$ holds (a.e.).

Energy formulation takes $v_\phi(E)=\delta(E)/\pi$; log-frequency formulation takes $v_\phi(\omega)=\varphi_{\mathcal M}(\omega)/\pi$. Below we default to $E$-variable, introducing relative density coordinate $v_{\mathrm{rel}}(E)=\xi(E)-\xi(E_\star)$ when necessary.

\textbf{Phase Normalization (Anchor Point):} Choose reference point $E_\star$ such that $\delta(E_\star)=0$ (equivalently $v_\phi(E_\star)=0$); for multi-channel, take $\delta=\tfrac12\arg\det S$ with trace. This normalization does not affect $\underline D_\phi,\overline D_\phi$ and other translation-invariant quantities, but ensures uniqueness of $v_\phi$ and stability of change of variables.

\textbf{Symbol Alignment:} Write $\varphi_{\mathcal M}(\omega)=\arg \Lambda_{\mathcal M}(\tfrac12+i\omega)$, scattering phase $\delta(E)=\tfrac12 \arg\det S(E)$. When embedding the mother map into the scattering model via the de Branges--Kreĭn interface, set $\delta\big(E(\omega)\big)=\varphi_{\mathcal M}(\omega)$. Absolute spectral density $\rho(E)\ge 0$ gives positive $d\mu(E)$; relative spectral density $\rho_{\mathrm{rel}}(E)=\xi'(E)=-\tfrac1\pi\delta'(E)$ can be positive or negative, given by spectral shift minus reference state density difference.

\textbf{A4 (Finite-Order EM Discipline)}
Throughout we use only \textbf{finite-order} Euler--Maclaurin (endpoint Bernoulli layer + explicit remainder upper bound) in parallel with Poisson summation for error accounting, not introducing new singularities.

\textbf{A5 (de Branges--Kreĭn Interface)}
When needed, invoke standard structures of de Branges spaces and canonical systems (kernel, ordering theorem, spectral measure and Hamiltonian duality).

\textbf{A6 (Notation Convention)}
Write $A\lesssim B$ to mean there exists constant $C>0$ (independent of main variables, window scale $R$, sampling step $\Delta$) such that $A\le C B$; write $A\simeq B$ to mean both $A\lesssim B$ and $B\lesssim A$ hold.

\section{Main Theorem I --- Noether-Type Information Conservation (2D Flux Continuity Equation)}

\begin{theorem}[Flux Conservation and Point Source Counting]
Let $\Lambda$ be meromorphic in domain $\mathcal D$, $u(\sigma,\omega)=\log|\Lambda(\sigma+i\omega)|$, $\mathcal J=\partial_\omega u$, $\mathcal H=\partial_\sigma u$. Denote zero and pole sets as $\mathcal Z$, $\mathcal P$.

(i) On $\mathcal D\setminus(\mathcal Z\cup\mathcal P)$, $\partial_\omega \mathcal J+\partial_\sigma \mathcal H=\Delta u=0$.

(ii) In distributional sense
\begin{equation}
\Delta u=2\pi\sum_{z\in\mathcal Z} m_z\,\delta(\cdot-z)-2\pi\sum_{p\in\mathcal P} n_p\,\delta(\cdot-p) ,
\end{equation}
where $m_z,n_p$ are multiplicities of zeros and poles.

(iii) Taking a rectangle $R$ with the critical line as one side,
\begin{equation}
\int_{\partial R}(\mathcal H,\mathcal J)\cdot n\,ds
=2\pi\big(N_{\mathcal Z}(R)-N_{\mathcal P}(R)\big) ,
\end{equation}
thus the interval integral along $\sigma=\tfrac12$ equals ``boundary normal flux + endpoint EM correction + point source counting'' sum.
\end{theorem}

\begin{proof}
(i) Complex harmonicity: in source-free domain, $\log\Lambda$ is holomorphic, $u=\Re\log\Lambda$ is harmonic.
(ii) Distributional source term: Laplacian of logarithmic singularity yields point mass.
(iii) Green's identity: converts interior sources to boundary integral; line integral via finite-order EM gives endpoint correction; Poisson/EM tools for non-asymptotic closure of sum--integral difference.
\end{proof}

\section{Main Theorem II --- CCS Consistency: $-\tfrac{1}{\pi}\delta'=\xi'=\operatorname{tr}(\rho-\rho_0)$}

\begin{theorem}[Phase Density = Spectral Shift Derivative = Relative Spectral Density; Sign Unification]
Let $(H_0,H)$ be a self-adjoint scattering pair, $S(E)$ the scattering matrix (multi-channel with trace). Then almost everywhere
\begin{equation}
-\frac{1}{\pi}\,\delta'(E)=\xi'(E)=\operatorname{tr}\big(\rho(E)-\rho_0(E)\big).
\end{equation}
\end{theorem}

\begin{proof}
(a) Herglotz--Weyl identification: boundary imaginary part of $m$ yields $\rho(E)=\tfrac1\pi \Im m(E+i0)$ (a.e.), similarly $\rho_0$.
(b) Birman--Kreĭn and spectral shift: from $\det S(E)=e^{-2\pi i \xi(E)}$, $\xi'(E)=-\tfrac{1}{2\pi i}\partial_E\log\det S(E)$. Under trace-class assumptions, $\xi'(E)=\operatorname{tr}(\rho-\rho_0)(E)$.
(c) Wigner--Smith delay: $Q(E)=-i\,S(E)^{*}\,\partial_E S(E)$, and $\partial_E\log\det S(E)=\operatorname{tr}(S^{-1}S')=i\,\operatorname{tr}Q(E)$, thus $\xi'(E)=-\tfrac{1}{2\pi}\operatorname{tr}Q(E)$. Single-channel $S=e^{2i\delta}$ gives $\operatorname{tr}Q(E)=2\delta'(E)$, hence $-\tfrac1\pi\delta'(E)=\xi'(E)$.
\end{proof}

\begin{remark}
The above formula is the core of this paper's unified scale: scattering phase derivative, spectral shift function, and relative state density are equivalent under a sign-carrying relation; absolute spectral density $\rho(E)\ge 0$ gives positive weight measure, distinguished from relative density $\rho_{\mathrm{rel}}=\xi'=-\tfrac1\pi\delta'$.
\end{remark}

\section{Main Theorem III --- Non-Asymptotic Error Closure: Nyquist--Poisson--EM Three-Fold}

\textbf{Terminology and Scale ($E$-Domain)}
Take Fourier transform
$$\widehat f(\xi)=\int_{\mathbb R} f(E)\,e^{-i\xi E}\,dE,\qquad(h\star\rho)(E)=\int_{\mathbb R} h(E-t)\,\rho(t)\,dt.$$
Call $g$ \textbf{bandlimited to} $[-B,B]$ if $\operatorname{supp}\widehat g\subset[-B,B]$; call \textbf{nearly bandlimited} to $[-B,B]$ if $\int_{|\xi|>B}|\widehat g(\xi)|^{2}\,d\xi$ is sufficiently small.

Below we stipulate $w_R(E)=w(E/R)$ and $E_n=E_0+n\Delta$, where $w\in\mathcal S(\mathbb R)$ is fixed.

\textbf{Regularity Premise:} Take $w\in\mathcal S(\mathbb R)$; let kernel $h\in W^{2p,1}(\mathbb R)\cap L^1(\mathbb R)$ with $\widehat h\in L^1(\mathbb R)$; absolute spectral density $\rho\in L^1_{\mathrm{loc}}(\mathbb R)$; accordingly $f(E)=w_R(E)\,(h\star\rho)(E)\in C^{2p}(\mathbb R)\cap L^1(\mathbb R)$, with derivatives up to order $(2p-1)$ bounded and integrable at endpoints, $\widehat{h\star\rho}\in L^1(\mathbb R)$. Under this premise, Theorem 3.1's EM remainder and aliasing upper bounds hold.

\begin{theorem}[Three-Fold Decomposition Upper Bound; Symmetric Truncation Version]
For any window $w_R$, kernel $h$, sampling step $\Delta$, truncation radius $T>0$, EM order $p$, there exists constant $C$ such that
\begin{equation}
\Bigg|\int_{|E-E_0|\le T} w_R\big(h\star \rho\big)(E)\,dE
-\Delta\sum_{|n-n_0|\le T/\Delta} w_R\big(h\star\rho\big)(E_n)\Bigg|
\le \varepsilon_{\mathrm{alias}}+\varepsilon_{\mathrm{EM}}^{(p)}+\varepsilon_{\mathrm{tail}}(T,R) .
\end{equation}
If $\widehat{h\star\rho}$ is bandlimited to $[-B,B]$ and $\Delta\le \pi/B$ then $\varepsilon_{\mathrm{alias}}=0$; if only nearly bandlimited,
$$\varepsilon_{\mathrm{alias}}\le \frac{1}{\Delta}\sum_{m\ne0}\int_{|\xi-2\pi m/\Delta|>B}|\widehat{h\star\rho}(\xi)|\,d\xi.$$
$\varepsilon_{\mathrm{EM}}^{(p)}$ is given by finite-order EM's Bernoulli layer explicit upper bound; $\varepsilon_{\mathrm{tail}}(T,R)$ is caused only by $|E-E_0|>T$ and window $w_R$'s decay, controllable by
$$\int_{|E-E_0|>T}|w_R(h\star\rho)(E)|\,dE+\Delta\sum_{|n-n_0|>T/\Delta}|w_R(h\star\rho)(E_n)|.$$
\end{theorem}

\begin{proof}
(i) Poisson summation: for equispaced sampling grid $E_n=E_0+n\Delta$, discrete summation and Poisson summation formula yield periodized spectral superposition; under Nyquist, bands do not overlap, aliasing term is zero.
(ii) Finite-order EM: sum--integral difference endpoint correction given by Bernoulli polynomial layer, remainder $R_p=\mathcal O(\Delta^{2p})$, constant depends only on $p$ and several bounded derivatives.
(iii) Symmetric truncation tail: constituted by integral and summation contributions from $|E-E_0|>T$ region; under symmetric window, $w_R$'s decay and $h\star\rho$'s out-of-band energy upper bound give explicit control. Sum of three terms yields result.
\end{proof}

\section{Sampling--Interpolation--Stability (Phase/Spectral Density Scale)}

The following workspace is a reproducing kernel Hilbert space (such as de Branges space obtained via §A5 interface), so point evaluation functionals are continuous, $|f(E_n)|$ well-defined and satisfies standard kernel estimates.

\begin{definition}[Density and Stability in Phase Coordinate $v_\phi$]
Write sampling points $E_n$'s phase coordinate $v_n=v_\phi(E_n)=\delta(E_n)/\pi$. For any $R>0$ and $v\in\mathbb R$, let $I(v,R)=[v-R,v+R]$.
$$\underline D_\phi=\liminf_{R\to\infty}\inf_{v\in\mathbb R}\frac{\#\{n: v_n\in I(v,R)\}}{2R},\quad\overline D_\phi=\limsup_{R\to\infty}\sup_{v\in\mathbb R}\frac{\#\{n: v_n\in I(v,R)\}}{2R}.$$
Call $\{E_n\}$ a \textbf{stable sampling sequence} if there exist constants $A,B>0$ such that for every $f$ in the workspace
$$A\|f\|_{L^2(d\mu)}^{2}\le \sum_{n}|f(E_n)|^{2}\le B\|f\|_{L^2(d\mu)}^{2}.$$
Call $\{E_n\}$ an \textbf{interpolation sequence} if for any $\{c_n\}\in \ell^2$ there exists $f$ with $f(E_n)=c_n$ and $\|f\|_{L^2(d\mu)}\lesssim \|\{c_n\}\|_{\ell^2}$.
\end{definition}

\begin{theorem}[Landau Necessary Density, Unit Bandwidth Scale]
Via §A3's isometry, embed workspace into $\mathrm{PW}_{1/2}$ (Fourier support in $[-1/2,1/2]$). If node set is a stable sampling sequence, then lower density $\underline D_\phi\ge 1$; if interpolation sequence, then upper density $\overline D_\phi\le 1$.
\end{theorem}

\begin{proof}
After isometry, problem reduces to non-uniform sampling of $\mathrm{PW}_{1/2}$, threshold constant is $1$.
\end{proof}

\begin{theorem}[Parseval Tight Frame Necessary and Sufficient: Shift-Invariant vs. Gabor/WR]

\textbf{(A) Shift-Invariant (Translation Only):} System $\{(w_\alpha(E-n\Delta))_{\alpha,n}\}$ is a Parseval tight frame if and only if
$$\Delta^{-1}\sum_{\alpha=1}^{r}\sum_{m\in\mathbb Z}\big|\widehat{w_\alpha}\big(\xi+2\pi m/\Delta\big)\big|^2\equiv 1\quad\text{(a.e. } \xi\text{)}.$$
If $\widehat{w_\alpha}$ is bandlimited to $[-B,B]$ and $\Delta\le \pi/B$ (no aliasing), the above reduces to
$$\Delta^{-1}\sum_{\alpha=1}^{r}\big|\widehat{w_\alpha}(\xi)\big|^2\equiv 1 .$$

\textbf{(B) Gabor (Translation+Modulation, Wexler--Raz):} System $\{e^{ik\Omega E}\,w_\alpha(E-n\Delta)\}_{\alpha,n,k}\}$ is a Parseval tight frame if and only if (Wexler--Raz identity)
$$\sum_{\alpha=1}^{r}\sum_{m,k\in\mathbb Z}
\widehat{w_\alpha}\left(\xi+\tfrac{2\pi m}{\Delta}\right)\overline{\widehat{w_\alpha}\left(\xi+\tfrac{2\pi (m+\ell)}{\Delta}\right)}\,
e^{ik\Omega\Delta\,\ell}
=\Delta\Omega\,\delta_{\ell,0}\quad(\forall\,\ell\in\mathbb Z),$$
in particular at critical density $\Delta\Omega=2\pi$ with total fold to constant $1$ reduces to Parseval condition.
\end{theorem}

\begin{proof}
(A) Shift-invariant system's Parseval condition given by Calderón/Walnut representation; (B) Wexler--Raz identity gives frequency-domain pointwise orthogonality and Parseval necessary and sufficient; via $u=\log t$ and log-frequency variable transformation transplants to Mellin model.
\end{proof}

\begin{theorem}[Balian--Low Impossibility: Mellin/Weyl Version]
At critical density $D=1$ with single window well-localized in both $u,\omega$ directions, the system generated by this window and critical lattice cannot be a Riesz basis; to obtain a basis, must relax at least one-side localization or adopt oversampling.
\end{theorem}

\begin{proof}
Via §A3 isometry, reduce problem to standard Gabor lattice BLT (Riesz/ONB version), conclusion follows immediately.
\end{proof}

\section{de Branges--Kreĭn Interface and ``Phase Equidistant'' Sampling}

\begin{definition}[Doubling Measure]
Write $\mu(I)=\int_{I}\rho(E)\,dE$. For any bounded open interval $I\subset\mathbb R$, denote $2I$ as the interval \textbf{with same center and doubled length}. If there exists constant $C_d\ge1$ such that
$$\mu(2I)\le C_d\,\mu(I)\quad\text{for all }I,$$
then call $\mu$ doubling. This, combined with reproducing-kernel diagonal estimates, allows matching local sampling spacing with $\rho$, thus supporting Proposition 5.1's stability conclusion.
\end{definition}

\begin{proposition}[Stable Frame with Phase Equidistance $\Delta\delta=\pi$]
If spectral density/phase measure is ``doubling'', choose multi-windows with non-overlapping frequency bands such that Calderón sum is constant $1$, and let sampling points satisfy
$$
\delta(x_{k+1})-\delta(x_k)=\pi ,
$$
then obtain stable sampling frame; in strict equidistance case, Parseval tight frame. Proof relies on reproducing-kernel diagonal formula and scale consistency of relative density.
\end{proposition}

\begin{proof}[Proof Sketch]
In de Branges space, kernel diagonal consistency with measure and canonical system's spectral correspondence give intrinsic relation between local sampling length and $\rho$; when matching kernel trace density with phase counting, use $\rho_{\mathrm{rel}}=\xi'=-\tfrac1\pi\delta'$ to give sampling density scale; stability estimate's inner product and kernel diagonal always proceed under positive weight $d\mu=\rho\,dE$.
\end{proof}

\section{Parallel and Inheritance with ``Fractal Mirror (FMU)''}

FMU has shown: multiplicatively self-similar signals exhibit ``envelope $\times$ equidistant frequency shift array'' in Mellin domain, with Bessel bound and unconditional convergence under weighted $\ell^2$. UMMIC uses §A3's isometry to merge FMU's frequency--scale geometry with this paper's phase coordinate $v_\phi$ and density measure $d\mu$, making Landau/WR/BLT criteria and Nyquist--Poisson--EM three-fold decomposition close on the same coordinate, directly translatable to window/kernel design and error accounting.

\section{Proof Tools and Minimal Sufficient Premises (Index Style)}

\begin{itemize}
\item \textbf{Mellin isometry/scaling and log variable:} Isometry on $\tfrac12+i\mathbb R$ and ``scaling $\leftrightarrow$ frequency shift'' law.
\item \textbf{Herglotz representation and spectral density:} $\rho=\tfrac1\pi\Im m$ (a.e.) and consistency of $d\mu=\rho\,dE$.
\item \textbf{Poisson and Euler--Maclaurin:} For non-asymptotic accounting of sum--integral difference, aliasing and endpoint correction.
\item \textbf{Birman--Kreĭn + Wigner--Smith:} $\det S=e^{-2\pi i\xi}$, $Q=-iS^*S'$, $\xi'=-\tfrac1{2\pi}\operatorname{tr}Q$.
\item \textbf{Landau necessary density:} Sampling/interpolation threshold for Paley--Wiener spaces.
\item \textbf{Wexler--Raz and BLT:} Tight/dual necessary and sufficient and critical density obstruction.
\item \textbf{de Branges structure:} Dictionary of kernel, measure, and canonical system.
\end{itemize}

\section{Verifiable Predictions and Engineering Interface (Minimal Experimental Template)}

\textbf{P1 | Flux Closure:} In selected working band $\Omega$, compute $\int_{\Omega}\partial_\omega\log|\Lambda(\tfrac12+i\omega)|\,d\omega$, use Poisson+finite-order EM to give error three-fold account, verify constant-level closure.

\textbf{P2 | Phase Scale Sampling Threshold:} Evaluate $\underline D_\phi,\overline D_\phi$ in $v_\phi$ coordinate and observe threshold transition (Landau) from undersampling to reconstructible near critical.

\textbf{P3 | WR-Parseval Design:} Per WR condition solve windows jointly such that $\sum_\alpha |\widehat{w_\alpha}|^2$ (with folding) is constant $1$, if aliasing use ``total fold'' formula.

\textbf{P4 | Delay--Density Consistency:} Numerically construct $S(E)$, compute $Q=-iS^*S'$ and phase of $\det S$, verify $\xi'(E)=-\tfrac1{2\pi}\operatorname{tr}Q(E)=-\tfrac1\pi\delta'(E)$ consistency with $\rho-\rho_0$.

\section{Conclusion}

This paper closes ``mother map--Mellin--de Branges--scattering--frame/sampling--error theory'' under the unified parallel framework of positive weight measure $d\mu=\rho\,dE$ and phase coordinate $v_\phi=\delta/\pi$ into a non-asymptotic and engineering-realizable theoretical framework:
(I) $\nabla\!\cdot(\partial_\sigma u,\partial_\omega u)$ is zero in source-free domain, distributional source terms with zeros/poles yield flux counting identity, all boundary/endpoint costs packaged by \textbf{finite-order EM}; (II) $-\tfrac1\pi\delta'=\xi'=\operatorname{tr}(\rho-\rho_0)$ compresses scattering phase, spectral shift and state density to same scale; (III) \textbf{Nyquist--Poisson--EM} three-fold decomposition gives non-asymptotic error closure; (IV) \textbf{Landau/WR/BLT} on $v_\phi$ coordinate provide complete boundary for sampling--reconstruction--stability; (V) align term-by-term with de Branges structure, Weyl--Titchmarsh dictionary and FMU's frequency--scale geometry, thus landing directly on window/kernel design, spectral readout and delay measurement.

\section*{Appendix A: Common Criteria and Formulas (For Invocation)}

\textbf{A.1 Poisson Summation (Simple Type)}
$\displaystyle \sum_{n\in\mathbb Z} f(n\Delta)=\frac{1}{\Delta}\sum_{m\in\mathbb Z}\widehat f\!\left(\frac{2\pi m}{\Delta}\right)$. Under bandwidth limitation and $\Delta\le\pi/B$, aliasing shuts off.

\textbf{A.2 Euler--Maclaurin (Finite-Order Version)}
$\displaystyle \sum_{n=a}^{b} f(n)=\int_{a}^{b} f(x)\,dx+\frac{f(a)+f(b)}{2}+\sum_{k=1}^{p}\frac{B_{2k}}{(2k)!}\big(f^{(2k-1)}(b)-f^{(2k-1)}(a)\big)+R_p$, with explicit upper bound for $R_p$.

\textbf{A.3 Wigner--Smith Delay Matrix (Unified Notation)}
$Q(E)=-i\,S(E)^{*}\,\partial_E S(E)$, $\operatorname{tr}Q(E)=2\delta'(E)$ (single-channel), and
$$\xi'(E)=-\frac{1}{2\pi}\operatorname{tr}Q(E).$$

\textbf{A.4 Birman--Kreĭn Formula (Supplemented with Logarithmic Derivative)}
$\det S(E)=e^{-2\pi i \xi(E)}$, thus
$$\partial_E\log\det S(E)=-2\pi i\,\xi'(E),\qquad \xi'(E)=\operatorname{tr}\big(\rho-\rho_0\big)(E).$$

\textbf{A.5 de Branges Space and Canonical System}
Correspondence of kernel and measure, subspace total order and canonical system's Hamiltonian dictionary.

\begin{thebibliography}{99}

\bibitem{Landau}
H. J. Landau.
\newblock Necessary density conditions for sampling and interpolation of certain entire functions.
\newblock {\em Acta Math.} 117 (1967) 37--52.

\bibitem{WexlerRaz}
J. Wexler, S. Raz.
\newblock Discrete Gabor expansions.
\newblock {\em Signal Processing} 21 (1990) 207--220.

\bibitem{Janssen}
A. J. E. M. Janssen.
\newblock Duality and biorthogonality for Weyl--Heisenberg frames.
\newblock {\em J. Fourier Anal. Appl.} 1 (1995) 403--436.

\bibitem{Daubechies}
I. Daubechies, H. J. Landau, Z. Landau.
\newblock Gabor Time-Frequency Lattices and the Wexler--Raz Identity.
\newblock {\em JFAA} 1 (1995) 437--478.

\bibitem{Grochenig}
K. Gröchenig.
\newblock Foundations of Time--Frequency Analysis, Birkhäuser, 2001.

\bibitem{WignerSmith}
E. P. Wigner.
\newblock Lower Limit for the Energy Derivative of the Scattering Phase Shift.
\newblock {\em Phys. Rev.} 98 (1955) 145;
F. T. Smith, Lifetime Matrix in Collision Theory, {\em Phys. Rev.} 118 (1960) 349--356.

\bibitem{BirmanKrein}
M. Sh. Birman, M. G. Kreĭn.
\newblock On spectral shift function (and subsequent extensions by Pushnitski, Strohmaier--Waters, et al.);
Surveys and textbooks see Yafaev.

\bibitem{deBranges}
L. de Branges.
\newblock Hilbert Spaces of Entire Functions, Prentice-Hall, 1968.

\bibitem{Remling}
C. Remling.
\newblock Spectral Theory of Canonical Systems, 2017.

\bibitem{DLMF}
NIST DLMF: Poisson summation, Euler--Maclaurin, Mellin methods entries (latest version).

\end{thebibliography}

\end{document}

