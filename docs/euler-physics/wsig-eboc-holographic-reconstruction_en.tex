\documentclass[12pt]{article}

% Essential packages
\usepackage[utf8]{inputenc}
\usepackage{amsmath,amssymb,amsthm}
\usepackage{mathrsfs}
\usepackage{geometry}
\usepackage{hyperref}

% Geometry settings
\geometry{a4paper, margin=1in}

% Hyperref settings
\hypersetup{
    colorlinks=true,
    linkcolor=blue,
    citecolor=blue,
    urlcolor=blue
}

% Theorem environments
\theoremstyle{plain}
\newtheorem{theorem}{Theorem}[section]
\newtheorem{lemma}[theorem]{Lemma}
\newtheorem{proposition}[theorem]{Proposition}
\newtheorem{corollary}[theorem]{Corollary}

\theoremstyle{definition}
\newtheorem{definition}[theorem]{Definition}
\newtheorem{example}[theorem]{Example}
\newtheorem{remark}[theorem]{Remark}

% Title information
\title{Formal Construction of WSIG--EBOC Holographic Reconstruction Theory}
\author{Haobo Ma$^1$ \and Wenlin Zhang$^2$\\
\small $^1$Independent Researcher\\
\small $^2$National University of Singapore}

\date{\today\\
\small Version: 1.2}

\begin{document}

\maketitle

\begin{abstract}
We establish a holographic reconstruction theory with Windowed Scattering--Information Geometry (WSIG) as the observational layer and Eternal Block Observation--Computation (EBOC) as the ontological layer. The core proposition is: under an overlapping window family satisfying frame stability and topological consistency, the ``trinity'' of local readouts consisting of phase derivative--relative state density--Wigner--Smith group delay can be stably glued into a global scattering field within a non-asymptotic Nyquist--Poisson--Euler--Maclaurin three-term closed error budget, thereby providing a global encoding of the EBOC static block. We present unified notation, conventions, and gluing consistency conditions, prove three main theorems on existence--stability--uniqueness (under gauge choice), and propose implementable inversion--projection algorithms along with semantic correspondence to reversible cellular automata (RCA).
\end{abstract}

\noindent\textbf{Keywords:} Holographic reconstruction; WSIG; EBOC; Wigner-Smith delay; Frame theory; Čech cohomology; Birman-Kreĭn formula; Reversible cellular automata

\noindent\textbf{MSC 2020:} 81U05; 47A40; 94A15; 42C15; 55N05

\tableofcontents

\section{Notation, Conventions, and Axioms}

\textbf{Energy domain and multi-port scattering.} Let the energy domain be a measurable set $\Omega\subset\mathbb{R}$, with channel number $N\in\mathbb{N}$. At almost every Lebesgue point of the absolutely continuous spectrum, the Wigner--Smith delay matrix and total phase for scattering matrix $S(E)\in U(N)$ are defined as

$$
\mathsf Q(E):=-\,i\,S(E)^\dagger \partial_E S(E),\qquad \Phi(E):=\operatorname{Arg}\det S(E).
$$

Let the half-phase be $\varphi(E):=\tfrac12\Phi(E)$. Then

$$
\partial_E \Phi(E)=\operatorname{tr}\mathsf Q(E),\qquad \partial_E\varphi(E)=\tfrac12\,\operatorname{tr}\mathsf Q(E).
$$

These three quantities satisfy the gauge identity with the relative state density $\rho_{\rm rel}(E)$:

$$
\boxed{\ \frac{\varphi'(E)}{\pi}=\rho_{\rm rel}(E)=\frac{1}{2\pi}\operatorname{tr}\mathsf Q(E)\ },
$$

where $\rho_{\rm rel}$ equals the derivative of the Kreĭn spectral shift function (this paper adopts scattering phase sign conventions and channel orientation that make the above equation hold). These equalities respectively stem from $\partial_E\log\det S=\operatorname{tr}(S^{-1}S')$ and the Birman--Kreĭn criterion.

\textbf{Windows and frames.} Take window family $\{w_j\}_{j\in J}\subset L^1\cap L^2(\Omega)$ with corresponding analysis operators $W_j f:=f\ast w_j$. Suppose there exist constants $0<A\le B<\infty$ such that

$$
A|f|_{L^2(\Omega)}^2\ \le\ \sum_{j\in J}|W_j f|_{L^2(\Omega)}^2\ \le\ B|f|_{L^2(\Omega)}^2,\quad \forall f\in L^2(\Omega),
$$

then $\{W_j\}$ is called a frame; denote the frame operator $\mathsf S:=\sum_j W_j^\ast W_j$, which is bounded, invertible, and $|\mathsf S^{-1}|\le A^{-1}$.

\textbf{Three-term error closure (NPE).} Let sampling step $h>0$, effective bandwidth $W>0$, and suppose the window has $m\ge2$ orders of integrable derivatives. The total error from discretization--convolution--truncation decomposes as

$$
\eta_{\rm NPE}:=\eta_{\rm alias}+\eta_{\rm Bern}+\eta_{\rm tail},
$$

where the aliasing term is controlled by Poisson summation, the Bernoulli correction layer comes from finite-order Euler--Maclaurin, and the tail term characterizes truncation leakage; corresponding bounds are given in §5.

\textbf{Gauge and topology.} We adopt the Birman--Kreĭn gauge: fix an absolutely continuous branch of $\Phi=\operatorname{Arg}\det S$ and let $\varphi=\tfrac12\Phi$. Assume the window covering is overlapping and has no topological gaps (see §2), to exclude phase multivaluedness from Čech 1-cocycles.

\section{Local Observation Model}

\subsection{Trinity Local Readout}

Define the window-wise local readout

$$
\mathcal R_j(E):=\big(\varphi'_j(E),\ \operatorname{tr}\mathsf Q_j(E),\ \rho_{{\rm rel},j}(E)\big),
$$

satisfying the window-wise trinity constraint $\varphi'_j=\tfrac12\operatorname{tr}\mathsf Q_j=\pi\rho_{{\rm rel},j}$. The observation model assumes the existence of a global field

$$
\mathcal R(E):=\big(\varphi'(E),\ \operatorname{tr}\mathsf Q(E),\ \rho_{\rm rel}(E)\big)
$$

such that

$$
\mathcal R_j\ =\ W_j\mathcal R\ +\ \varepsilon_j,\qquad |\varepsilon_j|_{L^2}\ \le\ C_{\rm NPE}\ \text{bounded by §5}.
$$

\subsection{Unitarity Constraint and Generating Equation}

From the definition $\mathsf Q=-iS^\dagger S'$ we derive the equivalent matrix ordinary differential equation

$$
S'(E)=i\,S(E)\,\mathsf Q(E),\qquad S(E_0)\in U(N).
$$

When $\mathsf Q^\dagger=\mathsf Q\in L^1_{\rm loc}(\Omega)$, the solution is given by the path-ordered exponential, and $S^\dagger S\equiv I$. Therefore, given $\mathsf Q$ (requiring only that its trace satisfies the gauge identity), one can always construct $S$ satisfying unitarity.

\section{Gluing Consistency and Čech Energy}

Let the overlap region be $U_{jk}:=\{E\in\Omega:\ w_j(E)w_k(E)\neq0\}$. Define the phase derivative difference and Čech energy as

$$
\Delta_{jk}(E):=\varphi'_j(E)-\varphi'_k(E),\qquad
\mathcal E_{\check C}:=\sum_{j\sim k}\int_{U_{jk}}|\Delta_{jk}(E)|^2\,{\rm d}E.
$$

\textbf{Consistency requirement:} There exists a constant $\kappa_0>0$ such that $\mathcal E_{\check C}\le \kappa_0\,|\eta_{\rm NPE}|_{L^2(\Omega)}^2$. In particular, for any closed 1-cycle $C$ on the nerve of the covering, we have

$$
\Big|\int_C\Delta_{jk}(E)\,{\rm d}E\Big|\ \le\ c_{\rm top}\,|\eta_{\rm NPE}|_{L^1},
$$

excluding branch cut inconsistencies induced by covering gaps.

\section{Variational Model for Global Reconstruction}

Define the constraint sets

$$
\mathcal C_{\rm tri}:=\Big\{\mathcal R:\ \varphi'=\tfrac12\operatorname{tr}\mathsf Q=\pi\rho_{\rm rel}\Big\},\quad
\mathcal C_{\rm unit}:=\Big\{S:\ S^\dagger S=I,\ -iS^\dagger S'=\mathsf Q\Big\}.
$$

\textbf{Coupling constraint.} The $\mathsf Q(E)$ appearing in both places is the same function, uniquely determined by $S$: $\mathsf Q(E):=-i\,S(E)^\dagger \partial_E S(E)$. Therefore $\operatorname{tr}\mathsf Q$ in $\mathcal C_{\rm tri}$ equals $-i\,\operatorname{tr}\big(S^\dagger S'\big)$, synchronized with $\mathcal C_{\rm unit}$.

In weighted space $|\cdot|_{\mathsf W}$, minimize

$$
\min_{\mathcal R,S}\ \mathscr J(\mathcal R,S):=\sum_{j\in J}|W_j\mathcal R-\mathcal R_j|_{\mathsf W}^2
\quad\text{s.t.}\quad \mathcal R\in\mathcal C_{\rm tri},\ S\in\mathcal C_{\rm unit}.
$$

To suppress overlap inconsistencies, introduce Čech regularization

$$
\mathscr J_\lambda:=\mathscr J+\lambda\,\mathcal E_{\check C},\qquad \lambda>0.
$$

\section{Main Theorems and Proofs}

\subsection{Theorem A (Existence)}

\begin{theorem}[Existence]
\label{thm:existence}
If $\{W_j\}$ is a frame with overlapping covering (existence of lower bound $A>0$), and $\mathcal E_{\check C}\le \kappa_0|\eta_{\rm NPE}|_{L^2}^2$, then there exists a minimizing pair $(\mathcal R_\star,S_\star)$ such that $\mathscr J_\lambda$ attains its minimum, with $\mathcal R_\star\in\mathcal C_{\rm tri}$ and $S_\star\in\mathcal C_{\rm unit}$.
\end{theorem}

\begin{proof}
First ignoring constraints, the normal equation

$$
\mathsf S\,\mathcal R=\sum_j W_j^\ast \mathcal R_j,\qquad \mathsf S=\sum_j W_j^\ast W_j,
$$

has $\mathsf S$ invertible with $|\mathsf S^{-1}|\le A^{-1}$ by the frame lower bound, so the unconstrained minimizing solution
$$\mathcal R^{(0)}=\mathsf S^{-1}\sum_j W_j^\ast \mathcal R_j$$
exists uniquely. Choose any Hermitian $\mathsf Q^{(0)}$ such that $\operatorname{tr}\mathsf Q^{(0)}=2\varphi^{(0)\prime}$ (e.g., $\mathsf Q^{(0)}=\tfrac{2\varphi^{(0)\prime}}{N}I$), and construct $S^{(0)}\in\mathcal C_{\rm unit}$ by §1.2. Apply alternating projection and minimization to the closed set $\mathcal C_{\rm tri}\times\mathcal C_{\rm unit}$ (§6); using non-expansiveness of projections and lower semicontinuity of $\mathscr J_\lambda$, combined with the Čech term absorbing error sources, we obtain existence of a minimizing pair. Frame theory ensures coercivity and boundedness, completing existence via the direct method.
\end{proof}

\subsection{Theorem B (Stability)}

\begin{theorem}[Stability]
\label{thm:stability}
Under the conditions of Theorem~\ref{thm:existence}, let $\widehat{\mathcal R}$ be the $\mathcal R$ component of the minimizing solution. Then

$$
|\widehat{\mathcal R}-\mathcal R|_{L^2(\Omega)}\ \le\ \kappa(A,B)\Big(\varepsilon_{\rm meas}
+|\eta_{\rm NPE}|_{L^2(\Omega)}\Big),\qquad \kappa(A,B)\le\sqrt{B/A},
$$

where $\varepsilon_{\rm meas}:=\max_j|W_j\mathcal R-\mathcal R_j|_{L^2}$.
\end{theorem}

\begin{proof}
By frame inequalities and triangle inequality,

$$
A|\widehat{\mathcal R}-\mathcal R|_{L^2}^2\ \le\ \sum_j|W_j(\widehat{\mathcal R}-\mathcal R)|_{L^2}^2
\ \le\ 2\sum_j|W_j\widehat{\mathcal R}-\mathcal R_j|_{L^2}^2+2\sum_j|W_j\mathcal R-\mathcal R_j|_{L^2}^2.
$$

Minimality and the Čech penalty term give boundedness on the right-hand side; normalizing with upper bound $B$ yields the stated estimate.
\end{proof}

\subsection{Theorem C (Uniqueness Under Gauge)}

\begin{theorem}[Uniqueness Under Gauge]
\label{thm:uniqueness}
If the nerve of the covering has no 1-cycles (all closed loops $C$ satisfy $\int_C\Delta_{jk}\,{\rm d}E=0$), and the BK gauge is fixed, then $(\mathcal R_\star,S_\star)$ is unique up to left multiplication by constant unitary equivalence ($U(N)$).
\end{theorem}

\begin{proof}
The condition implies that $\varphi'$ after covering gluing has a global primitive $\varphi$ consistent with the BK gauge; $\operatorname{tr}\mathsf Q$ is rigidly determined by $\operatorname{tr}\mathsf Q=2\varphi'$. The solution set of equation $S'=iS\mathsf Q$ for given $\mathsf Q$ differs only by left multiplication by a constant unitary element, equivalent to constant unitary equivalence classes ($U(N)$).
\end{proof}

\section{Non-Asymptotic NPE Error Budget}

Suppose window $w_j$ has $m$ orders of integrable derivatives, step size $h$, bandwidth $W$. Then there exist constants $C_1,C_2,C_3$ (depending on window decay and regularity) such that

$$
|\eta_{\rm alias}|_{L^2}\ \le\ C_1\,e^{-\frac{2\pi W}{h}},\quad
|\eta_{\rm Bern}|_{L^2}\ \le\ C_2\,h^{m}\,|\partial_E^{m}\mathcal R|_{L^2},\quad
|\eta_{\rm tail}|_{L^2}\ \le\ C_3\,|\mathcal R\cdot \mathbf 1_{\Omega^{\rm c}}|_{L^2}.
$$

The first term is given by the exponential-type aliasing bound from frequency spectrum periodization in Poisson summation; the second term derives the $O(h^{m})$ estimate from the Euler--Maclaurin bounded remainder (Fourier series bound on Bernoulli polynomials); the third term is controlled by the $L^2$ mass of the truncation residual. Thus

$$
|\widehat{\mathcal R}-\mathcal R|_{L^2}\ \le\ \kappa(A,B)\Big(\varepsilon_{\rm meas}
+ C_1 e^{-\frac{2\pi W}{h}}+ C_2 h^{m}|\partial_E^{m}\mathcal R|_{L^2}+C_3|\mathcal R\cdot \mathbf 1_{\Omega^{\rm c}}|_{L^2}\Big).
$$

\section{Inversion--Projection Algorithm}

\subsection{Frame Inversion Initialization}

From the normal equation, solve

$$
\mathcal R^{(0)}=\mathsf S^{-1}\sum_{j}W_j^\ast \mathcal R_j,\qquad |\mathsf S^{-1}|\le A^{-1}.
$$

\subsection{Alternating Projection and Unitarization}

Use alternating projection to solve constrained feasibility minimization:

$$
\mathcal R^{(n+\tfrac12)}:=\Pi_{\mathcal C_{\rm tri}}[\mathcal R^{(n)}],\quad
S^{(n+1)}:=\Pi_{\mathcal C_{\rm unit}}[S^{(n)};\,\mathcal R^{(n+\tfrac12)}],\quad
\mathcal R^{(n+1)}:=\arg\min_{\mathcal R}\sum_j|W_j\mathcal R-\mathcal R_j|_{\mathsf W}^2.
$$

Here $\Pi_{\mathcal C_{\rm tri}}$ can be implemented by least squares fitting of $(\varphi',\tfrac12\operatorname{tr}\mathsf Q,\pi\rho_{\rm rel})$; $\Pi_{\mathcal C_{\rm unit}}$ is defined as: first at each $E$ find

$$
\mathsf Q_\star(E)\ :=\ \arg\min_{\mathsf Q^\dagger=\mathsf Q,\ \operatorname{tr}\mathsf Q=2\varphi'(E)}\ \big|\mathsf Q+i\,S^\dagger S'\big|_F^2,
$$

whose closed-form solution is $\mathsf Q_\star=\operatorname{sym}\big(-iS^\dagger S'\big)-\frac{1}{N}\left(\operatorname{tr}\operatorname{sym}\big(-iS^\dagger S'\big)-2\varphi'\right)I$; then integrate $\widetilde S'=i\,\widetilde S\,\mathsf Q_\star$ and use the unitary factor of polar decomposition as retraction. Classical global convergence of alternating projection applies only to two closed convex sets; given that $\mathcal C_{\rm unit}$ is nonconvex and $\mathcal U(N)$ is a unitary manifold, this paper treats polar decomposition only as retraction and claims \textbf{local} convergence under local angle conditions/transversality; polar decomposition still provides the nearest unitary matrix in Frobenius norm.

\subsection{Čech Energy Regularization}

In overlap regions, add $\lambda\,\mathcal E_{\check C}$ to localize and suppress phase inconsistency sources. $\lambda$ can be adaptively chosen according to $C_1,C_2,C_3$ in §5 and measurement noise $\varepsilon_{\rm meas}$ to balance bias--variance.

\section{Multi-Port Spectral Decomposition and Channel Selection}

Spectral decomposition $S(E)=\sum_{\ell=1}^{N}e^{i\theta_\ell(E)}P_\ell(E)$ gives

$$
\operatorname{tr}\mathsf Q(E)=\sum_{\ell=1}^{N}\theta_\ell'(E),\qquad
\varphi'(E)=\tfrac12\sum_{\ell=1}^{N}\theta_\ell'(E).
$$

For sparse channel selection, one can select a set $I(E)\subset\{1,\dots,N\}$ on the spectral decomposition $S(E)=\sum_{\ell=1}^{N}e^{i\theta_\ell(E)}P_\ell(E)$ and define composite projection $P_{I}(E):=\sum_{\ell\in I(E)}P_\ell(E)$; then impose rank constraint $\operatorname{rank}P_I(E)\le k$ or penalize with Ky--Fan $k$-norm to favor low-rank coupling while maintaining trinity and unitary coupling.

\section{Semantic Correspondence with EBOC and RCA}

EBOC views $(\Omega,S)$ as the global encoding of a static block; window selection only changes the locally accessible cross-section without altering the ontological encoding. On the RCA side, sliding block codes provide local--global stitching: assign consistent transitions and ``recorded entropy'' readouts to each overlapping local state configuration, then Kolmogorov consistency constructs a global reversible flow. The frame stability and Čech energy in this paper correspond to RCA's conflict-free local transitions and codebook redundancy threshold; the BK gauge corresponds to the unification of global phase (gauge) choice.

\section{Sufficient Conditions and Failure Mechanisms}

\textbf{Sufficient conditions.} Frame covering (existence of lower bound $A>0$); BK gauge fixed; nerve of covering has no 1-cycles; three-term errors within §5 bounds; Čech energy bounded.

\textbf{Failure mechanisms.} Covering gaps (some energy region without overlap); topological gaps (nonzero 1-cycles on nerve inducing multivaluedness); frame degeneration ($A\to0$); aliasing dominance ($W/h$ too small); tail leakage (energy mass outside window not negligible).

\section{Explicit Constants for Kaiser--Bessel Window}

For Kaiser--Bessel window (parameter $\beta>0$), one can take

$$
C_1\lesssim C(\beta),\qquad C_2\lesssim C(\beta)\,W^{-m},\qquad C_3\lesssim C(\beta)\,e^{-\beta},
$$

thus

$$
|\widehat{\mathcal R}-\mathcal R|_{L^2}\ \le\ \kappa(A,B)\Big(\varepsilon_{\rm meas}+C(\beta)e^{-\frac{2\pi W}{h}}
+C(\beta)W^{-m}h^{m}|\partial_E^{m}\mathcal R|_{L^2}+C(\beta)e^{-\beta}\Big).
$$

The mainlobe--sidelobe tradeoff of Kaiser--Bessel window and the role of parameter $\beta$ are detailed in the literature.

\section{Verifiable Predictions}

\begin{enumerate}
\item At fixed bandwidth, increasing overlap (improving frame lower bound $A$) should monotonically decrease reconstruction error;
\item Injecting phase perturbations in overlap regions will manifest as nonzero $\mathcal E_{\check C}$ and be localized by the regularization term;
\item When $W/h$ crosses the Nyquist threshold, the error curve exhibits a phase transition point from ``aliasing-dominated → Bernoulli-dominated'', consistent with sampling density criteria.
\end{enumerate}

\section{Conclusion}

Within an axiomatic framework of ``gauge identity'' and NPE three-term closure, this paper provides stable invertible gluing from local trinity readouts to global scattering field, achieving uniqueness under BK gauge; algorithmically implemented via frame inversion and alternating projection--polar decomposition; semantically in strict correspondence with the holographic--stitching structure of EBOC/RCA. This theory establishes verifiable qualitative predictions and non-asymptotic error budgets among window design, sampling density, topological unification, and channel selection.

\bibliographystyle{plain}
\begin{thebibliography}{99}

\bibitem{wigner}
E. P. Wigner.
\newblock Lower limit for the energy derivative of the scattering phase shift.

\bibitem{smith}
F. T. Smith.
\newblock Lifetime matrix in collision theory.
\newblock {\em Physical Review}, 118:349--356, 1960.

\bibitem{patel}
U. R. Patel and E. Michielssen.
\newblock Wigner--Smith Time Delay Matrix.
\newblock {\em arXiv:2003.06985}, 2020.

\bibitem{birman}
M. Sh. Birman and M. G. Kreĭn.
\newblock On the theory of wave operators and scattering matrix.

\bibitem{strohmaier}
A. Strohmaier and A. Waters.
\newblock The Birman--Kreĭn formula.

\bibitem{christensen}
O. Christensen.
\newblock An Introduction to Frames and Riesz Bases.
\newblock Springer, 2nd edition, 2016.

\bibitem{bauschke}
H. H. Bauschke and J. M. Borwein.
\newblock On Projection Algorithms for Convex Feasibility.
\newblock {\em SIAM Review}, 38(3):367--426, 1996.

\bibitem{higham}
N. J. Higham.
\newblock Computing the Polar Decomposition.
\newblock {\em SIAM Journal on Scientific and Statistical Computing}, 11(6):1160--1174, 1990.

\bibitem{harris}
F. J. Harris.
\newblock On the Use of Windows for Harmonic Analysis with the Discrete Fourier Transform.
\newblock {\em Proceedings of the IEEE}, 66(1):51--83, 1978.

\bibitem{landau}
H. J. Landau.
\newblock Necessary density conditions for sampling and interpolation of certain entire functions.
\newblock {\em Acta Mathematica}, 117:37--52, 1967.

\end{thebibliography}

\end{document}

