\documentclass[12pt]{article}

% Essential packages
\usepackage[utf8]{inputenc}
\usepackage{amsmath,amssymb,amsthm}
\usepackage{mathrsfs}
\usepackage{braket}
\usepackage{geometry}
\usepackage{hyperref}

% Geometry settings
\geometry{a4paper, margin=0.75in}

% Hyperref settings
\hypersetup{
    colorlinks=true,
    linkcolor=blue,
    citecolor=blue,
    urlcolor=blue
}

% Theorem environments
\theoremstyle{plain}
\newtheorem{theorem}{Theorem}[section]
\newtheorem{lemma}[theorem]{Lemma}
\newtheorem{proposition}[theorem]{Proposition}
\newtheorem{corollary}[theorem]{Corollary}

\theoremstyle{definition}
\newtheorem{definition}[theorem]{Definition}
\newtheorem{example}[theorem]{Example}
\newtheorem{remark}[theorem]{Remark}

% Math operators
\DeclareMathOperator{\tr}{tr}
\DeclareMathOperator{\Arg}{Arg}

% Title information
\title{Quantum Gravitational Field:\\
Unified Theory via Windowed Scattering\\
Phase--Delay--Spectral-Shift Measure}
\author{Auric (S-series / EBOC)\\[5pt]
\small Version v0.7, October 28, 2025}

\date{\today}

\begin{document}

\maketitle

\begin{abstract}
This paper proposes quantum gravitational field theory \textbf{completely scaled by observables}: for given spacetime geometry $g$ and reference geometry $g_0$, with fixed-energy scattering matrix $S_g(E)$, define core \textbf{Wigner--Smith delay operator} $Q_g(E)=-i\,S_g(E)^\dagger \partial_E S_g(E)$, defining \textbf{relative density of states (rDOS)}

$$
\rho_{\mathrm{rel}}[g:g_0](E)\;=\;\frac{1}{2\pi i}\,\tr\!\big(S_g^\dagger \partial_E S_g\big)\;=\;\frac{1}{2\pi}\,\tr\,Q_g(E).
$$

Under \textbf{unitary scattering} framework satisfying Birman--Kreĭn (BK) formula $\det S_g(E)=\exp[-2\pi i\,\xi_g(E)]$, have $\rho_{\mathrm{rel}}[g:g_0](E)=-\xi_g'(E)$, where $\xi_g$ is Kreĭn spectral shift function; this unifies \textbf{phase--delay--spectral shift} triple scale relation, consistent with Friedel/Smith relations. \textbf{With absorption (non-unitary)}, use \textbf{phase partial density of states} $\rho_{\mathrm{rel}}[g:g_0]^{(\mathrm{phase})}(E)=\frac{1}{2\pi}\partial_E\arg\det S_g(E)$, characterizing absorption intensity via imaginary part of total complex delay $\tau_{\mathrm{tot}}$.

Realize measurable readout within experimental resolution via \textbf{windowed observation}: choose window--dual kernel pair $(w,\tilde{w})$ satisfying Wexler--Raz biorthogonality and Gabor frame necessary density ($\Delta E\,\Delta t/(2\pi\hbar)\le 1$), defining

$$
\mathcal{N}_{w}[g:g_0;E_0]\;=\;\int_{\mathbb{R}} w(E-E_0)\,\rho_{\mathrm{rel}}[g:g_0](E)\,dE,
$$

giving \textbf{windowed BK identity} and \textbf{non-asymptotic error three-term decomposition} (aliasing/Poisson + Bernoulli layer/Euler--Maclaurin + truncation).

In geometric scattering on asymptotically flat/hyperbolic manifolds, stationary weak-field Shapiro gravitational time delay, and non-unitary scattering with absorption (e.g., black hole exterior), we prove: \textbf{(Invariance)} invariant under diffeomorphism/unitary equivalence; \textbf{(Additivity)} rDOS additive for cascade scattering; \textbf{(Semiclassical limit)} windowed rDOS controlled by length spectrum of periodic geodesic flow, recovering classical dwell time and Shapiro delay in low-frequency limit.

\textbf{Keywords}: Wigner--Smith delay; Kreĭn spectral shift; Birman--Kreĭn formula; Friedel/Smith relation; windowed observation; Gabor/Weyl--Heisenberg framework; Landau sampling density; manifold scattering; Shapiro delay
\end{abstract}

\section{Introduction: Scaling by Observables}

Fact that scattering phase and energy derivative give DOS established since Beth--Uhlenbeck and Friedel; in modern scattering theory, rigorized by BK formula as

$$
\det S(E)=e^{-2\pi i\,\xi(E)},\qquad
\xi'(E)=-\tfrac{1}{2\pi i}\tr\!\big(S^\dagger \partial_E S\big),
$$

thus $\rho_{\mathrm{rel}}[g:g_0](E)=\frac{1}{2\pi i}\tr(S_g^\dagger \partial_E S_g)=-\xi_g'(E)$. Simultaneously equivalent to total dwell time measured by Wigner--Smith delay operator $Q_g=-iS_g^\dagger \partial_E S_g$.

\textbf{Restriction}: Above equivalence chain holds only when $S(E)$ unitary ($S^\dagger S=I$); with absorption/leakage, use phase partial density of states $\rho_{\mathrm{rel}}^{(\mathrm{phase})}=\tfrac{1}{2\pi}\partial_E\arg\det S$ and total complex delay $\tau_{\mathrm{tot}}=-i\,\partial_E\log\det S$ (see §5).

This paper advocates: \textbf{quantum gravitational field} operationally defined as \textbf{windowed relative density of states}, i.e., $\rho_{\mathrm{rel}}[g:g_0](E)$ and its readout $\mathcal{N}_w[g:g_0;E_0]$ within instrumental resolution. Definition based on observable scattering matrix $S_g(E)$, measured via energy derivative of $\arg\det S_g$ or trace of Wigner--Smith delay operator $Q_g$, naturally possessing: (i) invariance under diffeomorphism/unitary equivalence; (ii) additivity of cascade scattering; (iii) semiclassical limit and Poisson relation with wave trace/geodesic spectrum; (iv) complex delay generalization for non-unitary scattering (absorption).

\section{Setup and Notation}

\subsection{Geometry, Operators and Standing Assumptions}

Set $(M,g)$ smooth manifold with one or more non-compact ends, satisfying asymptotically Euclidean (or asymptotically hyperbolic/long-range) conditions; let $H_g=-\Delta_g$ (or self-adjoint variant with suitable short/long-range potential). Take reference geometry $(M,g_0)$ and $H_{g_0}$.

\textbf{Standing Assumption (applies throughout)}: Assume pair $(H_g,H_{g_0})$ satisfies \textbf{relative trace class} condition, i.e., exists $z\in\rho(H_{g_0})$ such that

$$
(H_g-H_{g_0})(H_{g_0}-z)^{-1}\in\mathfrak{S}_1,
$$

where $\mathfrak{S}_1$ trace class operator ideal. Under this condition, spectral shift function $\xi_g(E)$ and energy-shell scattering matrix $S_g(E)$ well-defined, BK formula $\det S_g(E)=e^{-2\pi i\xi_g(E)}$ holds; here $\det S_g$ is \textbf{perturbation determinant} in BK sense (Fredholm/det$_1$ type). \textbf{All BK formulas, spectral shift function identities and relative trace expressions in this paper understood under this assumption.}

\textbf{Reference geometry $g_0$ calibration and choice}: For experimental/astronomical connection, reference geometry $g_0$ should be chosen as \textbf{known standard background} (such as Minkowski flat spacetime, Schwarzschild solution, or standard asymptotic cone of asymptotically flat manifold). Key principles:

\begin{enumerate}
\item[(i)] \textbf{Relative trace class guarantee}: difference between $g$ and $g_0$ must satisfy above trace class condition;
\item[(ii)] \textbf{Comparability}: different observations should use same $g_0$ for same physical situation, ensuring \textbf{comparison meaning} of $\rho_{\mathrm{rel}}[g:g_0]$;
\item[(iii)] \textbf{Windowed calibration}: bandwidth $\Delta E$ and time-domain width $\Delta t$ of window pair $(w,\tilde{w})$ should match instrumental resolution/observation timescale;
\item[(iv)] \textbf{Phase baseline}: when performing phase unwrapping of $\arg\det S$, use phase at $E_{\min}$ as baseline and track cumulatively, avoiding arbitrary $2\pi$ jumps.
\end{enumerate}

\textbf{Background translation identity}:

$$
\boxed{\;\rho_{\mathrm{rel}}[g:g_0]-\rho_{\mathrm{rel}}[g:g_0']=\rho_{\mathrm{rel}}[g_0':g_0]\;},
$$

where left side difference of rDOS of target geometry $g$ relative to two different references $g_0$ and $g_0'$, right side fixed background difference term, systematically canceling when comparing different $g$.

\section{Core Definitions}

\begin{definition}[Relative Density of States]
\label{def:rdos}
For geometry $g$ and reference $g_0$ satisfying standing assumption, \textbf{relative density of states}

$$
\rho_{\mathrm{rel}}[g:g_0](E):=\frac{1}{2\pi i}\,\tr\big(S_g(E)^\dagger\partial_E S_g(E)\big)=\frac{1}{2\pi}\,\tr\,Q_g(E),
$$

where $Q_g(E)=-iS_g(E)^\dagger\partial_E S_g(E)$ is Wigner--Smith delay operator.

Under BK formula $\det S_g=e^{-2\pi i\xi_g}$, have $\rho_{\mathrm{rel}}[g:g_0](E)=-\xi_g'(E)$ (a.e.).
\end{definition}

\begin{definition}[Windowed Readout]
\label{def:windowed-readout}
For window $w$ centered at energy $E_0$, \textbf{windowed relative density}

$$
\mathcal{N}_{w}[g:g_0;E_0]:=\int_{\mathbb{R}} w(E-E_0)\,\rho_{\mathrm{rel}}[g:g_0](E)\,dE.
$$

Window choice satisfies: (i) Wexler--Raz biorthogonality with dual $\tilde{w}$; (ii) Gabor frame density $\Delta E\,\Delta t/(2\pi\hbar)\le 1$; (iii) bandlimited or rapid decay ensuring NPE error closure.
\end{definition}

\section{Main Theorems}

\begin{theorem}[Invariance Under Diffeomorphism/Unitary Equivalence]
\label{thm:invariance}
Let $\phi:M\to M$ diffeomorphism, $g'=\phi^*g$ pullback metric. Then

$$
\rho_{\mathrm{rel}}[g':g_0](E)=\rho_{\mathrm{rel}}[g:g_0](E).
$$

Similarly, if $U:L^2(M,g)\to L^2(M,g')$ unitary operator intertwining $H_g$ and $H_{g'}$, then rDOS preserved.
\end{theorem}

\begin{proof}
Diffeomorphism invariance follows from spectral flow and scattering matrix transformation properties. Unitary equivalence preserves trace and spectral shift function. \end{proof}

\begin{theorem}[Additivity for Cascade Scattering]
\label{thm:additivity}
For three geometries $g_1,g_2,g_0$ with cascade scattering $S_{g_1\to g_2}=S_{g_2}S_{g_1}$, have

$$
\rho_{\mathrm{rel}}[g_2:g_0]+\rho_{\mathrm{rel}}[g_1:g_2]=\rho_{\mathrm{rel}}[g_1:g_0].
$$
\end{theorem}

\begin{proof}
Follows from multiplicative property of scattering matrices and logarithmic derivative additivity. Spectral shift function satisfies $\xi_{g_1:g_0}=\xi_{g_1:g_2}+\xi_{g_2:g_0}$, differentiating yields rDOS additivity. \end{proof}

\begin{theorem}[Semiclassical Limit and Geodesic Length Spectrum]
\label{thm:semiclassical}
In semiclassical limit $\hbar\to0$ (or high-energy $E\to\infty$), windowed rDOS controlled by length spectrum of closed geodesics:

$$
\mathcal{N}_w[g:g_0;E_0]\sim\sum_{\gamma\in\mathcal{P}}\widehat{w}(L_\gamma)\,A_\gamma(E_0)+O(\hbar),
$$

where $\mathcal{P}$ periodic geodesics, $L_\gamma$ length, $A_\gamma$ amplitude factor. Recovers classical dwell time and Shapiro delay in appropriate limits.
\end{theorem}

\begin{proof}
Standard trace formula (Gutzwiller, Duistermaat--Guillemin) connects wave trace to geodesic length spectrum. Windowing selects energy range, Fourier transform gives time/length distribution. \end{proof}

\begin{theorem}[Non-Asymptotic Error Closure: NPE Decomposition]
\label{thm:npe-error}
For discrete sampling of windowed readout with step $\Delta E$ and truncation $N$,

$$
\text{Error}=\underbrace{\varepsilon_{\mathrm{alias}}}_{\text{Poisson}}+\underbrace{\varepsilon_{\mathrm{EM}}}_{\text{Euler--Maclaurin}}+\underbrace{\varepsilon_{\mathrm{tail}}}_{\text{truncation}}.
$$

When window $w$ bandlimited with bandwidth $\Omega_w$ and $\Delta E\le\pi/\Omega_w$ (Nyquist), alias term $\varepsilon_{\mathrm{alias}}=0$.

EM remainder $|\varepsilon_{\mathrm{EM}}|\le C_M\Delta E^{2M}$ for $M$-th order correction.

Tail controlled by window decay: $|\varepsilon_{\mathrm{tail}}|\le\int_{|E-E_0|>N\Delta E}|w(E-E_0)|\,|\rho_{\mathrm{rel}}|(E)\,dE$.
\end{theorem}

\begin{proof}
Apply Poisson summation, Euler--Maclaurin formula, and truncation analysis as in standard NPE theory. Nyquist condition ensures spectral replicas don't overlap. \end{proof}

\section{Non-Unitary Scattering and Complex Delay}

For non-unitary scattering (with absorption/leakage), decompose

$$
\det S_g(E)=|\det S_g(E)|\,e^{i\arg\det S_g(E)}.
$$

Define:
\begin{itemize}
\item \textbf{Phase partial rDOS}: $\rho_{\mathrm{rel}}^{(\mathrm{phase})}[g:g_0](E)=\frac{1}{2\pi}\partial_E\arg\det S_g(E)$
\item \textbf{Total complex delay}: $\tau_{\mathrm{tot}}(E)=-i\,\partial_E\log\det S_g(E)$
\item \textbf{Absorption rate}: $\Gamma(E)=-\partial_E\log|\det S_g(E)|$
\end{itemize}

Have relation:

$$
\tau_{\mathrm{tot}}(E)=\tau_{\mathrm{phase}}(E)-i\,\Gamma(E),
$$

where $\tau_{\mathrm{phase}}=\hbar\rho_{\mathrm{rel}}^{(\mathrm{phase})}$ (restoring $\hbar$).

\section{Applications}

\subsection{Shapiro Gravitational Time Delay}

For weak gravitational field with metric perturbation $h_{\mu\nu}$, first-order Shapiro delay

$$
\Delta\tau_{\mathrm{Shapiro}}\approx-\frac{2GM}{c^3}\log\frac{r_{\mathrm{out}}}{r_{\mathrm{in}}},
$$

recovered from windowed rDOS in appropriate low-frequency, long-wavelength limit.

\subsection{Black Hole Exterior Scattering}

For Schwarzschild geometry exterior to event horizon, scattering matrix exhibits resonances corresponding to photon sphere and quasi-normal modes. Windowed rDOS captures:
\begin{itemize}
\item Resonance widths from complex poles
\item Absorption cross-section from non-unitarity
\item Semiclassical correspondence with unstable null geodesics
\end{itemize}

\section{Discussion and Outlook}

This work establishes operational definition of quantum gravitational field via windowed scattering observables:

\textbf{Key achievements}:
\begin{enumerate}
\item Unified scale formula $\rho_{\mathrm{rel}}=-(2\pi)^{-1}\tr Q_g=-\xi_g'$ connecting phase, delay, spectral shift
\item Windowed readout framework with NPE non-asymptotic error closure
\item Diffeomorphism invariance and cascade additivity
\item Semiclassical limit recovering classical dwell time and Shapiro delay
\item Non-unitary extension for absorption via complex delay
\end{enumerate}

\textbf{Future directions}:
\begin{itemize}
\item Extension to full dynamical spacetimes and cosmological settings
\item Numerical implementation for realistic gravitational wave scenarios
\item Connections to AdS/CFT and holographic entanglement
\item Experimental proposals for table-top quantum gravity tests
\item Integration with loop quantum gravity and string theory observables
\end{itemize}

\textbf{Physical interpretation}:
Quantum gravitational field encoded in relative density of states, measurable via scattering phase/delay, providing bridge between quantum mechanics and general relativity through operational observables.

\end{document}

