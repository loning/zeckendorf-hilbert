\documentclass[11pt,a4paper]{article}
\usepackage{amsmath,amssymb,amsthm}
\usepackage{geometry}
\usepackage{hyperref}
\usepackage{mathtools}
\usepackage{enumitem}
\setlist[itemize]{nosep,leftmargin=1.6em}
\setlist[enumerate]{nosep,leftmargin=1.6em}

\geometry{margin=1in}

\newtheorem{theorem}{Theorem}[section]
\newtheorem{proposition}[theorem]{Proposition}
\newtheorem{lemma}[theorem]{Lemma}
\newtheorem{corollary}[theorem]{Corollary}
\newtheorem{definition}[theorem]{Definition}
\theoremstyle{remark}
\newtheorem{remark}[theorem]{Remark}
\newtheorem{example}[theorem]{Example}

\DeclareMathOperator{\Re}{Re}
\DeclareMathOperator{\Im}{Im}
\DeclareMathOperator{\ord}{ord}
\DeclareMathOperator{\Coeff}{Coeff}
\DeclareMathOperator{\Sing}{Sing}

\raggedbottom

\title{Directional Meromorphization and Pole Localization}

\author{Haobo Ma\thanks{Independent Researcher} \and Wenlin Zhang\thanks{National University of Singapore}}

\date{}

\begin{document}

\maketitle

\begin{center}
\textit{Directional Counting, Exponential--Polynomial Asymptotics and Laplace--Stieltjes Pole Structure}
\end{center}

\begin{abstract}
Building upon S1's tube-domain/interchange baseline and S2/S3/S4's mirror---continuation paradigm, we establish a unified analytic structure between directional counting/weighted cumulation and their Laplace--Stieltjes transforms: when directional counting or weighted cumulation has finite exponential--polynomial asymptotics, its directional Laplace--Stieltjes transform is meromorphic in the appropriate half-plane, with all poles generated by main-scale terms of each exponential rate; pole location is determined by exponential rate and order is bounded by the degree of corresponding polynomials. Under natural monotonicity/non-negativity or bounded-variation control, conversely, pole-type singularities of the transform allow reconstruction of exponential--polynomial leading terms of the counting. This result is consistent with S4's ``pole = main scale'' principle and interfaces with S2's binomial closure---transversal geometry and S3's completed function template ($\Gamma/\pi$ normalization).
\end{abstract}

\section{Notation and Prerequisites (Aligned with S1/S2/S3/S4)}

\textbf{Directional slicing and parametrization.} Fix direction $\mathbf{v}\in\mathbb{S}^{n-1}$ and transverse offset $\rho_\perp$ ($\langle\rho_\perp,\mathbf{v}\rangle=0$); write
\begin{equation}
\rho=\rho_\perp+s\,\mathbf{v},\qquad s\in\mathbb{C}\ \text{(subsequently complexified by tube-domain criterion in absolute convergence half-plane)} .
\end{equation}

For discrete spectrum
\begin{equation}
F(\theta,\rho)=\sum_{j} c_j\,e^{\,i\langle\alpha_j,\theta\rangle}\,e^{\,\langle\beta_j,\rho\rangle},
\end{equation}
define directional displacement
\begin{equation}
t_j:=\langle-\beta_j,\mathbf{v}\rangle .
\end{equation}

\textbf{Directional counting and weighted cumulation.}
\begin{equation}
N_{\mathbf{v}}(t):=\#\{j:\ t_j\le t\},\qquad
M_{\mathbf{v}}(t;\theta,\rho_\perp):=\sum_{t_j\le t} c_j\,e^{\,\langle\beta_j,\rho_\perp\rangle}\,e^{\,i\langle\alpha_j,\theta\rangle}.
\end{equation}
Both are viewed as right-continuous, locally bounded-variation functions (or distribution functions of corresponding Stieltjes measures).

\textbf{Boundary normalization and starting point.} If there exist finitely many initial points with $t_j<0$, their contribution can be absorbed into constant (or entire-function) terms without affecting subsequent pole localization. To avoid sign ambiguity and ensure boundary terms vanish, we adopt the normalization:
\begin{equation}
N_{\mathbf{v}}(0^+)=M_{\mathbf{v}}(0^+)=0,\qquad t_j\ge0\ (\forall j) .
\end{equation}
Without such normalization, finitely many terms with $t_j<0$ contribute only constants/entire functions in the directional transform, not altering pole sets and orders of $\mathscr{N}_{\mathbf{v}}$ and $\mathcal{L}_{\mathbf{v}}$.

\textbf{Directional Laplace--Stieltjes transform (in convergence half-plane).}
\begin{equation}
\mathscr{N}_{\mathbf{v}}(s):=\int_{0}^{\infty} e^{-s t}\,dN_{\mathbf{v}}(t)
=\sum_{t_j\ge0}e^{-s t_j}
=\sum_{t_j\ge0} e^{\,s\langle\beta_j,\mathbf{v}\rangle},
\end{equation}
\begin{equation}
\mathcal{L}_{\mathbf{v}}(s;\theta,\rho_\perp):=\int_{0}^{\infty} e^{-s t}\,dM_{\mathbf{v}}(t;\theta,\rho_\perp)
=\sum_{t_j\ge0} c_j\,e^{\,\langle\beta_j,\rho_\perp+s\mathbf{v}\rangle}\,e^{\,i\langle\alpha_j,\theta\rangle}.
\end{equation}
Hence in the absolute convergence half-plane (and under the above normalization),
\begin{equation}
\mathcal{L}_{\mathbf{v}}(s;\theta,\rho_\perp)=F\big(\theta,\rho_\perp+s\,\mathbf{v}\big).
\end{equation}

\textbf{Working guidelines.} All interchange/term-by-term operations follow S1's tube-domain and dominated convergence criterion; discrete--continuous bridging uses only S4's \textbf{finite-order} Euler--Maclaurin (EM) decomposition, whose Bernoulli layers and remainder are holomorphic/entire in parameter $s$, not altering pole sets.

\textbf{Local zero structure interface.} For dominant two-term cases, use S2's binomial closure---transversality template to determine simplicity and local structure of univariate function zeros on directional slices.

\textbf{Completed function interface.} When necessary, use S3's $\Gamma/\pi$ normalization factor on the multiplicative side to balance growth and construct symmetric completed functions.

\section{Convergence Threshold and Basic Half-Plane}

\begin{lemma}[Threshold for Directional Absolute Convergence]\label{lem:threshold}
Let
\begin{equation}
\gamma_{\mathrm{abs}}:=\limsup_{t\to\infty}\frac{1}{t}\log N_{\mathbf{v}}(t)\in[-\infty,\infty] .
\end{equation}
Then:
\begin{enumerate}
\item[(i)] When $\Re s>\gamma_{\mathrm{abs}}$, $\mathscr{N}_{\mathbf{v}}(s)$ converges absolutely;
\item[(ii)] \textbf{If there are infinitely many jump points} (i.e., $N_{\mathbf{v}}(t)\to\infty$), and $\Re s<\gamma_{\mathrm{abs}}$, then $\mathscr{N}_{\mathbf{v}}(s)$ diverges. \textbf{If there are only finitely many jump points}, then $\mathscr{N}_{\mathbf{v}}(s)$ is a finite sum hence entire in the $s$-plane, and assertion (ii) does not apply.
\end{enumerate}
Under Proposition~\ref{prop:weight}'s \textbf{local polynomial boundedness} assumption (weights controlled within unit-length intervals), $\mathcal{L}_{\mathbf{v}}(s)$ and $\mathscr{N}_{\mathbf{v}}(s)$ have the same directional absolute convergence half-plane and threshold; otherwise weights may alter the absolute convergence threshold.
\end{lemma}

\begin{proof}[Proof (sketch)]
For any $\varepsilon>0$, there exists $T$ such that $N_{\mathbf{v}}(t)\le C\,e^{(\gamma_{\mathrm{abs}}+\varepsilon)t}$ ($t\ge T$). Integration by parts gives
\begin{equation}
\int_0^\infty e^{-s t}\,dN_{\mathbf{v}}(t)
=s\int_0^\infty e^{-s t}N_{\mathbf{v}}(t)\,dt
\; +\;\big[e^{-s t}N_{\mathbf{v}}(t)\big]_{0^+}^{\infty},
\end{equation}
When $\Re s>\gamma_{\mathrm{abs}}+\varepsilon$, the right-hand side is bounded and boundary terms vanish, yielding (i). In the reverse direction, take a subsequence $t_k\to\infty$ such that $N_{\mathbf{v}}(t_k)\ge e^{(\gamma_{\mathrm{abs}}-\varepsilon)t_k}$; comparison with geometric series yields (ii). When and only when there are infinitely many jump points ($N_{\mathbf{v}}(t)\to\infty$), the convergence abscissa satisfies
\begin{equation}
\sigma_a\;=\;\limsup_{t\to\infty}\frac{1}{t}\log N_{\mathbf{v}}(t),
\end{equation}
see Widder, \textit{The Laplace Transform}, or Doetsch, \textit{Laplace Transformation}. If there are only finitely many jumps, we set by convention $\sigma_a:=-\infty$ (in this case $\mathscr{N}_{\mathbf{v}}$ is entire). Moreover, under Proposition~\ref{prop:weight}'s \textbf{local polynomial boundedness} assumption, $\mathcal{L}_{\mathbf{v}}$ and $\mathscr{N}_{\mathbf{v}}$ have the same directional absolute convergence half-plane; if there are only finitely many jumps, both are entire in the $s$-plane, with convergence abscissa taken as $-\infty$ by convention.
\end{proof}

\section{Exponential--Polynomial Leading Term $\Rightarrow$ Meromorphic Continuation and Pole Upper Bound}

\begin{definition}[Directional Exponential--Polynomial Asymptotics]\label{def:exppoly}
We say $M_{\mathbf{v}}(t)$ has \textbf{finite exponential--polynomial asymptotics} as $t\to\infty$ if there exist finite index set $\mathcal{I}$, real numbers $\gamma_\ell$ (repetitions allowed), polynomials $Q_\ell$, and $\eta>0$ such that
\begin{equation}
M_{\mathbf{v}}(t)=\sum_{\ell\in\mathcal{I}} Q_\ell(t)\,e^{\gamma_\ell t}
\;+\; O\left(e^{(\gamma_\ast-\eta)t}\right),\qquad \gamma_\ast:=\max_{\ell}\gamma_\ell ,
\end{equation}
and $M_{\mathbf{v}}$ has locally bounded variation and is right-continuous. When this definition is applied to non-weighted counting $N_{\mathbf{v}}$, $Q_\ell$ are real polynomials with non-negative coefficients.

To handle repeated rates, write the unique rate set $\Gamma:=\{\gamma:\ \exists\,\ell\in\mathcal{I},\ \gamma_\ell=\gamma\}$, and for each $\gamma\in\Gamma$ define the aggregated polynomial
\begin{equation}
Q_\gamma(t):=\sum_{\ell:\,\gamma_\ell=\gamma} Q_\ell(t)
=\sum_{r=0}^{d_\gamma} a_{\gamma,r}\,t^r,\qquad d_\gamma:=\deg Q_\gamma.
\end{equation}
\end{definition}

\begin{theorem}[Abelian Directional Meromorphization and Pole Upper Bound]\label{thm:abelian}
If $M_{\mathbf{v}}$ satisfies Definition~\ref{def:exppoly}, then $\mathcal{L}_{\mathbf{v}}(s)$ is \textbf{meromorphic} in the half-plane $\Re s>\gamma_\ast-\eta$, with all poles located at $\{s=\gamma:\ \gamma\in\Gamma\}$, and
\begin{equation}
\ord_{\,s=\gamma}\mathcal{L}_{\mathbf{v}}\ \le\ d_\gamma+1\qquad(\gamma\ne0),
\end{equation}
When $\gamma=0$, the overall $s$ factor cancels one order, further giving
\begin{equation}
\ord_{\,s=0}\mathcal{L}_{\mathbf{v}}\ \le\ d_0.
\end{equation}

More specifically, if
\begin{equation}
Q_\gamma(t)=\sum_{r=0}^{d_\gamma} a_{\gamma,r}\,t^r,
\end{equation}
then in a neighborhood of $s=\gamma$ there is a principal part expansion
\begin{equation}
\mathcal{L}_{\mathbf{v}}(s)
=\sum_{\gamma\in\Gamma}\left[\,s\sum_{r=0}^{d_\gamma} a_{\gamma,r}\,\frac{r!}{(s-\gamma)^{r+1}}\right]\;+\;H(s),
\end{equation}
where $H$ is holomorphic in $\Re s>\gamma_\ast-\eta$.
\end{theorem}

\begin{proof}[Proof (supplement and reference)]
Write $dM_{\mathbf{v}}$ for the Stieltjes measure of $M_{\mathbf{v}}$. For $\Re s>\gamma_\ast$ we have
\begin{equation}
\mathcal{L}_{\mathbf{v}}(s)=\int_0^\infty e^{-s t}\,dM_{\mathbf{v}}(t)
=s\int_0^\infty e^{-s t}M_{\mathbf{v}}(t)\,dt .
\end{equation}
Substituting the asymptotic formula and integrating term-by-term: the second term is holomorphic in $\Re s>\gamma_\ast-\eta$; the first term gives for each $(\ell,r)$
\begin{equation}
s\,a_{\ell,r}\int_0^\infty t^r\,e^{-(s-\gamma_\ell)t}\,dt
=s\,a_{\ell,r}\,\frac{r!}{(s-\gamma_\ell)^{r+1}},
\end{equation}
The conclusion follows by uniqueness of analytic continuation. This can be viewed as a direct instance of Abelian/transfer theorems: $\int_0^\infty t^re^{-(s-\gamma)t}\,dt=r!/(s-\gamma)^{r+1}$ gives the pole structure (see Flajolet--Sedgewick, \textit{Analytic Combinatorics}; Widder; Doetsch).
\end{proof}

\begin{remark}
\textbf{Counting version.} For $\mathscr{N}_{\mathbf{v}}$ the conclusion is the same; at rate $\gamma$ the highest-order principal part coefficient is $\gamma\,a_{\gamma,d_\gamma}\,d_\gamma!$ ($\gamma\ne0$). If $\gamma\ge 0$ this coefficient is non-negative; when $\gamma=0$ it is $0$; when $\gamma<0$ it is non-positive.

\textbf{Consistency with S4.} Bernoulli layers and remainder are holomorphic in $s$; only main-scale terms can produce poles, hence ``pole = main scale''.
\end{remark}

\section{From Counting to Weighted: Pole Location Does Not Shift Right (Order Increases at Most by $\kappa$)}

\begin{proposition}[Location Does Not Shift Right; Order Upper Bound]\label{prop:weight}
Let $M_{\mathbf{v}}$ and $N_{\mathbf{v}}$ have the same jump points, and suppose there exist constants $C,\kappa\ge0$ such that
\begin{equation}
\sup_{t<\tau\le t+1}\big|M_{\mathbf{v}}(\tau)-M_{\mathbf{v}}(t)\big|
\le C(1+t)^\kappa\,
\sup_{t<\tau\le t+1}\big(N_{\mathbf{v}}(\tau)-N_{\mathbf{v}}(t)\big).
\end{equation}
If $N_{\mathbf{v}}$ satisfies Definition~\ref{def:exppoly}, then $M_{\mathbf{v}}$ satisfies the upper bound
\begin{equation}
M_{\mathbf{v}}(t)\ \ll\ \sum_{\ell}\,(1+t)^{\deg Q_\ell(N_{\mathbf{v}})+\kappa}\,e^{\gamma_\ell t},
\end{equation}
hence its \textbf{maximum exponential rate is not greater than} the counting side (pole location does not shift right), and the \textbf{absolute convergence half-plane does not shift right}. The above upper bound alone \textbf{cannot} determine pole order.

If we \textbf{further assume} that $M_{\mathbf{v}}$ \textbf{also satisfies Definition~\ref{def:exppoly}} (e.g., weights eventually non-negative, polynomially bounded within unit intervals, and \textbf{no systematic cancellation}, so it has the same rate set as the counting side with gap $\eta>0$), \textbf{then} by Theorem~\ref{thm:abelian} we obtain
\begin{equation}
\ord_{\,s=\gamma}\mathcal{L}_{\mathbf{v}}\ \le\ d_\gamma(N_{\mathbf{v}})+\kappa+1\qquad(\gamma\ne 0),
\end{equation}
and when $\gamma=0$,
\begin{equation}
\ord_{\,s=0}\mathcal{L}_{\mathbf{v}}\ \le\ d_0(N_{\mathbf{v}})+\kappa .
\end{equation}

Special case: When $\kappa=0$ (weights uniformly bounded within unit intervals),
\begin{equation}
\ord_{\,s=\gamma_\ell}\mathcal{L}_{\mathbf{v}}\ \le\ \deg Q_\ell(N_{\mathbf{v}})+1,\qquad\text{and pole location does not shift right}.
\end{equation}
If moreover the counting side \textbf{attains the upper bound} at $s=\gamma_\ell$ (e.g., non-negative/no systematic cancellation, so $\ord_{\,s=\gamma_\ell}\mathscr{N}_{\mathbf{v}}=\deg Q_\ell(N_{\mathbf{v}})+1$), then further (order does not increase)
\begin{equation}
\ord_{\,s=\gamma_\ell}\mathcal{L}_{\mathbf{v}}\ \le\ \ord_{\,s=\gamma_\ell}\mathscr{N}_{\mathbf{v}}\qquad\text{(order does not increase)},
\end{equation}
and location and order agree with the counting side (only principal part coefficients change).
\end{proposition}

\begin{proof}[Proof (sketch)]
Summation by parts and local bounded-amplitude control.
\end{proof}

\begin{remark}
In the mother mapping, weights $c_j\,e^{\langle\beta_j,\rho_\perp\rangle}\,e^{i\langle\alpha_j,\theta\rangle}$ are polynomially bounded within unit-length $\langle-\beta_j,\mathbf{v}\rangle$ intervals; pole \textbf{location} in pole geometry is determined by exponential rate and independent of the phase layer; for \textbf{pole order}, it agrees with the counting side only when $\kappa=0$, otherwise increases by at most $\kappa$.
\end{remark}

\section{Directional Tauberian: Pole $\Rightarrow$ Exponential--Polynomial Leading Term}

\begin{theorem}[Tauberian Inversion for Single Pole, Ikehara--Delange Type]\label{thm:tauberian}
Let $\mathscr{N}_{\mathbf{v}}(s)$ be holomorphic in $\Re s>\gamma_0$ and admit continuation to an open neighborhood containing $\Re s\ge\gamma_0$, with a pole of order $m\ge1$ at $s=\gamma_0$, and suppose
\begin{equation}
G(s):=\mathscr{N}_{\mathbf{v}}(s)-\sum_{r=1}^{m}\frac{A_{-r}}{(s-\gamma_0)^r}
\end{equation}
is bounded in the above neighborhood (e.g., has continuous and bounded boundary values on $\Re s=\gamma_0$, and $G(\gamma_0+iy)=o(1)$ as $|y|\to\infty$). If $N_{\mathbf{v}}$ is non-decreasing (or eventually non-decreasing), then:

If $\gamma_0\ne 0$, then
\begin{equation}
N_{\mathbf{v}}(t)=\frac{A_{-m}}{\gamma_0\,(m-1)!}\,e^{\gamma_0 t}\,t^{m-1}
+o\big(e^{\gamma_0 t}t^{m-1}\big)\qquad (t\to\infty).
\end{equation}
If $\gamma_0=0$, then
\begin{equation}
N_{\mathbf{v}}(t)=\frac{A_{-m}}{m!}\,t^{m}\,+o\big(t^{m}\big)\qquad (t\to\infty).
\end{equation}
\end{theorem}

\begin{proof}[Key points (supplement and reference)]
Take $\sigma>\gamma_0$; use Laplace--Stieltjes inversion
\begin{equation}
\frac{N_{\mathbf{v}}(t+)+N_{\mathbf{v}}(t-)}{2}
=\frac{1}{2\pi i}\int_{\Re s=\sigma}\frac{\mathscr{N}_{\mathbf{v}}(s)}{s}\,e^{s t}\,ds,
\end{equation}
Shift the contour left to $\Re s=\gamma_0+\varepsilon$ and take a small circle around $s=\gamma_0$; the principal part comes from the residue at that pole; the remaining boundary is a lower-order term under monotonicity and vertical-line growth control of $G$. This is a typical conclusion in one-sided Tauberian theory: for non-decreasing (or eventually non-decreasing) $N_{\mathbf{v}}$, a finite-order pole at $s=\gamma_0$ implies $N_{\mathbf{v}}(t)\sim c\,e^{\gamma_0 t}t^{m-1}$ (see Korevaar, \textit{Tauberian Theory}; Feller, Vol. II).
\end{proof}

\begin{remark}
If monotonicity is lost, under bounded total variation and local averaging conditions one obtains an upper bound $\ll e^{\gamma_0 t}t^{m-1}$; equality-type conclusions require stronger Tauberian tools.
\end{remark}

\section{Upper and Lower Bounds on Pole Order and Leading Coefficient}

If
\begin{equation}
M_{\mathbf{v}}(t)\sim \sum_{\ell} Q_\ell(t)\,e^{\gamma_\ell t},\qquad
Q_\ell(t)=\sum_{r=0}^{d_\ell} a_{\ell,r} t^r,
\end{equation}
then by Theorem~\ref{thm:abelian}
\begin{equation}
\ord_{\,s=\gamma_\ell}\mathcal{L}_{\mathbf{v}}\le d_\ell+1,
\end{equation}
and the highest-order principal part coefficient is
\begin{equation}
\Coeff_{(s-\gamma_\ell)^{-(d_\ell+1)}}\mathcal{L}_{\mathbf{v}}(s)
=\gamma_\ell\,a_{\ell,d_\ell}\,d_\ell! .
\end{equation}

Conversely, in Theorem~\ref{thm:tauberian}'s single-pole case, if $m=d+1$ and $A_{-m}\neq0$, then
\begin{equation}
a_{d}=\frac{A_{-m}}{\gamma_0\,d!}
\end{equation}
If $\gamma_0=0$, then $m=d$, and
\begin{equation}
a_{d}=\frac{A_{-d}}{d!}.
\end{equation}

When $\gamma_\ell=0$, since the principal part has an overall $s$ factor, the pole order automatically decreases by one; the general upper bound can be tightened to
\begin{equation}
\ord_{\,s=\gamma_\ell}\mathcal{L}_{\mathbf{v}}\ \le\ d_\ell\qquad(\gamma_\ell=0).
\end{equation}
This gives the leading coefficient of the highest-degree term.

\section{Interface with S4's EM Paradigm (``Pole = Main Scale'')}

For directional discrete sums (such as $\sum_k f(a+k\Delta;s)$ or more general spectral sums), use S4's \textbf{finite-order} EM decomposition into ``main-scale integral + several Bernoulli layers + remainder''. Bernoulli layers and remainder are holomorphic/entire in $s$; only main-scale terms can produce poles. Combined with Section 2's exponential--polynomial leading terms, this completes \textbf{meromorphic continuation and pole localization} in the desired strip.

\section{Complementary Geometry of Directional Zeros--Poles (Aligned with S2)}

On the directional slice $\rho=\rho_\perp+s\mathbf{v}$, if a segment is dominated by two terms, then S2's \textbf{binomial closure} gives a local equation for zeros (phase antipodality and amplitude balance); in general position, zeros are \textbf{simple zeros}. Complementarily, \textbf{poles} arise from exponential growth of directional cumulation (counting or weighted), not from local two-term cancellation. Thus zeros and poles of the cross-section function are geometrically complementary.

\section{Completed Function and Growth Balance (Aligned with S3)}

To control vertical-line growth in a wider strip or realize mirror symmetry about $\Re s=\frac{a}{2}$, one can take S3's $a$-symmetric $\Gamma/\pi$ factor $r(s)$ and define the completed function
\begin{equation}
\Xi_{\mathbf{v}}(s):=r(s)\,\mathcal{L}_{\mathbf{v}}(s).
\end{equation}
In Theorem~\ref{thm:abelian}'s half-plane, $\Xi_{\mathbf{v}}$ remains meromorphic; by Stirling estimates, exponential growth along vertical lines is balanced. If $r$ also cancels main-scale poles, then $\Xi_{\mathbf{v}}$ is holomorphic (pole-free) in that half-plane.

\section{Counterexamples and Boundary Families (Failure Reasons Annotated)}

\begin{itemize}
\item \textbf{R5.1 (Sub-exponential cumulation)}: If $N_{\mathbf{v}}(t)$ has only $e^{\sqrt{t}}$-level growth, then $\mathscr{N}_{\mathbf{v}}$ converges only in $\Re s>0$ and generally \textbf{has no poles}; the exponential--polynomial premise does not hold.

\item \textbf{R5.2 (Violently oscillating weights)}: If weighted jumps grow super-polynomially within unit intervals, Proposition~\ref{prop:weight} fails, possibly causing pole order to increase or natural boundaries to appear.

\item \textbf{R5.3 (Infinite Bernoulli layers)}: Misusing EM as an \textbf{infinite series} destroys uniform summability and forges poles; must adhere to \textbf{finite order} and verify remainder holomorphy.

\item \textbf{R5.4 (Directional degeneracy)}: If $\langle\beta_j-\beta_k,\mathbf{v}\rangle\equiv0$ (all terms move at the same speed along that direction), then the directional slice degenerates; zero/pole structure must be expanded in transverse parameters ($\theta$ or $\rho_\perp$).
\end{itemize}

\section{Unified Verifiable Checklist (Minimal Sufficient Conditions)}

\begin{enumerate}
\item \textbf{Convergence half-plane}: Compute $\gamma_{\mathrm{abs}}=\limsup_{t\to\infty}t^{-1}\log N_{\mathbf{v}}(t)$; start in $\Re s>\gamma_{\mathrm{abs}}$.

\item \textbf{Exponential--polynomial leading term}: Provide finite exponential--polynomial asymptotics of $M_{\mathbf{v}}$ or $N_{\mathbf{v}}$ (exponential rates $\gamma_\ell$, degrees $d_\ell$, gap $\eta>0$).

\item \textbf{Meromorphization}: By Theorem~\ref{thm:abelian}, obtain pole locations $s=\gamma_\ell$ and order upper bound $d_\ell+1$ for $\mathcal{L}_{\mathbf{v}}/\mathscr{N}_{\mathbf{v}}$.

\item \textbf{Weight robustness (upper-bound direction)}: Verify Proposition~\ref{prop:weight}'s local polynomial boundedness condition to ensure pole location \textbf{does not shift right} (exponential rate not greater than counting side); pole \textbf{order does not increase} when $\kappa=0$. For ``location/order agreement'', need additional ``no systematic cancellation'' (e.g., non-negative weights) assumption.

\item \textbf{Tauberian inversion (equality-type)}: Under monotonicity/non-negativity premises and satisfying Theorem~\ref{thm:tauberian}'s Ikehara--Delange boundary conditions, deduce exponential--polynomial leading term from poles; without such boundary conditions, only upper bounds are obtained, not equality-type leading terms.

\item \textbf{EM splicing}: Any discrete--continuous interchange is performed using S4's finite-order EM; Bernoulli layers and remainder are holomorphic in $s$.

\item \textbf{Geometric consistency}: In dominant two-term intervals, consistency with S2's binomial closure---transversality template (simplicity of zeros and degenerate branches).

\item \textbf{Growth balance (optional)}: When mirror/growth control is needed, select S3's symmetric $\Gamma/\pi$ factor to construct completed function.
\end{enumerate}

\appendix

\section{Basic Laplace--Stieltjes Formulas (Variation and Integration by Parts)}

Let $G$ be right-continuous, locally bounded-variation function, $dG$ its Stieltjes measure. If $\Re s>\sigma_0$ and $|G(t)|\le C\,e^{(\sigma_0+\varepsilon)t}$ ($t\ge T$), then
\begin{equation}
\int_0^\infty e^{-s t}\,dG(t)
=s\int_0^\infty e^{-s t}G(t)\,dt\; +\;\big[e^{-s t}G(t)\big]_{0^+}^{\infty},
\end{equation}
with boundary terms vanishing under $\Re s>\sigma_0+\varepsilon$. If we further assume $G$ is non-decreasing, then there is an inversion formula
\begin{equation}
\frac{G(t+)+G(t-)}{2}
=\frac{1}{2\pi i}\int_{\Re s=\sigma}\frac{1}{s}\left(\int_0^\infty e^{-s u}\,dG(u)\right)e^{s t}\,ds
\quad (\sigma>\sigma_0),
\end{equation}
and combined with the residue theorem and vertical-line growth estimates, this gives Theorem~\ref{thm:tauberian}'s leading term and error control.

\section*{Concluding Remarks}

Directional counting/weighted cumulation compresses the mother mapping's spectrum--scale data into univariate verifiable objects; when they exhibit finite \textbf{exponential--polynomial} growth, the directional Laplace--Stieltjes transform is \textbf{meromorphic} in the appropriate half-plane, with \textbf{poles completely determined by main scale}: location equals exponential rate, order is at most the corresponding polynomial degree plus one. This directional analytic structure is rigorously consistent with S4's EM paradigm and complements S2's zero-set geometry and S3's completed function template, providing a verifiable, spliceable pole---growth baseline for subsequent S6's information calibration and S7's $L$-function interface.

\begin{thebibliography}{9}

\bibitem{widder}
D.~V. Widder, \emph{The Laplace Transform}, Princeton University Press.

\bibitem{doetsch}
G.~Doetsch, \emph{Introduction to the Theory and Application of the Laplace Transformation}.

\bibitem{korevaar}
J.~Korevaar, \emph{Tauberian Theory: A Century of Developments}, Springer.

\bibitem{feller}
W.~Feller, \emph{An Introduction to Probability Theory and Its Applications, Vol. II}, Wiley.

\bibitem{flajolet}
P.~Flajolet and R.~Sedgewick, \emph{Analytic Combinatorics}, Cambridge University Press.

\end{thebibliography}

\end{document}


