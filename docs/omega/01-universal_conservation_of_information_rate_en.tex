\documentclass[11pt,a4paper]{article}
\usepackage[utf8]{inputenc}
\usepackage[T1]{fontenc}
\usepackage{amsmath}
\usepackage{amssymb}
\usepackage{geometry}
\usepackage{hyperref}
\usepackage{braket}
\usepackage{graphicx}
\usepackage{bm}

\geometry{left=2.5cm,right=2.5cm,top=2.5cm,bottom=2.5cm}

\title{Universal Conservation of Information Rate: From Quantum Cellular Automata to the Unification of Relativity, Mass, and Gravity}
\author{Haobo Ma$^1$ \and Wenlin Zhang$^2$\\
\small $^1$Independent Researcher\\
\small $^2$National University of Singapore}
\date{}

\begin{document}

\maketitle

\begin{abstract}
In the framework of Quantum Cellular Automata (QCA) and finite information ontology, we construct an effective description of single-particle long-wavelength excitations and prove a core result based on Hilbert space geometry and unitarity: for any discrete quantum walk/QCA defined by local unitary evolution and translation invariance that emerges a one-dimensional Dirac-type Hamiltonian in the continuous limit, the external group velocity ($v_{\mathrm{ext}}$) and internal state evolution velocity ($v_{\mathrm{int}}$) of its long-wavelength single-particle eigenmodes must satisfy the Information Rate Conservation Theorem:
$$v_{\mathrm{ext}}^{2}+v_{\mathrm{int}}^{2}=c^{2},$$
where $c$ is the maximum causal propagation speed of the lattice system. This theorem is not an additional axiom but a geometric result enforced by the local unitarity of QCA and the orthogonal decomposition of the anti-commuting algebra of internal degrees of freedom under the Fubini--Study projective metric.

By defining proper time ($\tau$) with the internal evolution parameter, special relativity's time dilation, four-velocity normalization, and Minkowski line element can be directly derived from the Information Rate Conservation Theorem. In the continuous limit of Dirac-type QCA, the internal Hamiltonian ($H_{\mathrm{int}}$) gives the internal frequency ($\omega_{\mathrm{int}}$), and mass obtains an information-theoretic definition:
$$m c^{2}=\hbar\omega_{\mathrm{int}},$$
satisfying the Zitterbewegung frequency relation:
$$\omega_{\mathrm{ZB}}=2\omega_{\mathrm{int}}.$$
Combined with the winding number and index invariants of QCA, massive excitations can be interpreted as optical path quotas bound in topologically non-trivial self-referential loops. At the many-body level, we introduce local information processing density ($\rho_{\mathrm{info}}(x)$) and derive the optical metric from local conservation of information volume:
$$ds^{2}=-\eta^{2}(x)c^{2}dt^{2}+\eta^{-2}(x)\gamma_{ij}(x)dx^{i}dx^{j},$$
where $\eta(x)$ determines the local effective speed of light:
$$c_{\mathrm{eff}}(x)=\eta^{2}(x)c,$$
and the refractive index:
$$n(x)=\eta^{-2}(x).$$
In the weak-field limit, this structure recovers the first-order expansion of the Schwarzschild metric and the standard light deflection angle, and field equations formally equivalent to Einstein's equations can be obtained through an information-gravity variational principle. Furthermore, we introduce information mass ($M_{I}$) and, combining with Landauer's principle, analyze the asymptotic stationary behavior and minimum dissipation power of high-information-mass subjects, providing a unified information-theoretic characterization of mass, gravity, and complex energetic structures, and propose testable predictions based on superconducting quantum circuits and quantum simulation platforms.
\end{abstract}

\textbf{Keywords:} Quantum Cellular Automata; Conservation of Information Rate; Fubini--Study Metric; Optical Metric; Special Relativity; General Relativity; Topological Mass; Zitterbewegung; Information Mass; Landauer's Principle

\section{Introduction \& Historical Context}

Special and General Relativity describe the physical world as a four-dimensional manifold $(M,g_{\mu\nu})$ with Lorentzian signature. The metric tensor $g_{\mu\nu}$ determines the causal structure and geodesics, and the field equations
$$R_{\mu\nu}-\tfrac12 R g_{\mu\nu}=8\pi G T_{\mu\nu}$$
relate the stress-energy tensor $T_{\mu\nu}$ to curvature. Experimental tests such as gravitational redshift, light deflection, binary pulsar timing, and gravitational wave detection highly support this geometric narrative. Relativity is axiomatically constructed on the constancy of the speed of light and the principle of relativity, introducing the Minkowski line element and Lorentz transformations, with its geometric structure typically regarded as an a priori background.

Quantum theory is formulated in Hilbert space $\mathcal H$, where states are vectors or density operators, observables are self-adjoint operators, and time evolution is generated by the unitary group. Statistical interpretation is built on the Born rule, with superposition, intrinsic phase, and entanglement constituting core structures. The two theories are stitched together in Quantum Field Theory by defining "field operators on a background manifold," but their ontological starting points remain separated: one side is a continuous, curvable spacetime manifold, and the other is an abstract linear Hilbert space.

When approaching the Planck scale, the assumptions of continuous manifolds and classical metrics lose empirical support, while the Hilbert space structure itself does not depend on continuous spacetime. Quantum Cellular Automata (QCA) provide an alternative formulation with discrete structure as ontology: defining finite-dimensional local Hilbert spaces and local unitary evolution on countable lattices, requiring strict causality and finite propagation radius. Existing research has shown that in appropriate continuous limits, Dirac, Weyl, and Maxwell equations can emerge from local unitary evolution of QCA, and QCA possesses systematic topological classification and index theory.

On the other hand, Hilbert space itself has a natural projective geometric structure. The projective Hilbert space $\mathbb C P^{n}$ is equipped with the Fubini--Study metric, whose arc length gives the natural distance between quantum states. For unitary evolution driven by a time-independent Hamiltonian $H$, the "velocity" of the state vector under the Fubini--Study metric is determined by the energy uncertainty $\Delta H$, and the "path length" of quantum evolution can be viewed as a measure of information update.

This paper attempts to unify the above three threads under an information-theoretic perspective:

1. Assume the universe is microscopically described by a local unitary, translation-invariant QCA with a maximum propagation speed $c$;
2. Treat single-particle long-wavelength excitations as a class of effective modes in QCA, whose external motion is described by group velocity ($v_{\mathrm{ext}}$) and internal state self-referential evolution is described by geometric velocity ($v_{\mathrm{int}}$) in projective Hilbert space;
3. Prove that in the continuous limit of Dirac-type QCA, the orthogonal decomposition induced by the anti-commuting structure of the Hamiltonian and the Fubini--Study metric necessarily yields the Information Rate Conservation Theorem:
   $$v_{\mathrm{ext}}^{2}+v_{\mathrm{int}}^{2}=c^{2},$$
   thereby elevating "conservation of optical path length" from an assumption to a theorem.

On this basis, special relativity can no longer be viewed as an independent axiom but as an emergent result of QCA unitarity and Hilbert geometry; mass can be interpreted as the coefficient of internal frequency $\omega_{\mathrm{int}}$; gravitational geometry can be interpreted as a manifestation of local information processing density and optical metric structure; and the "information mass" of complex energetic systems can be linked to Landauer's principle, providing a unified picture of mass, gravity, and complexity.

\section{Model & Assumptions}

\subsection{QCA Universe and Local Unitarity}

Let $\Lambda$ be a countable connected graph, whose nodes represent "spatial cells." Each cell $x\in\Lambda$ carries a finite-dimensional Hilbert space $\mathcal H_{x}\simeq\mathbb C^{d}$. For any finite subset $F\Subset\Lambda$, define the local Hilbert space
$$\mathcal H_{F}=\bigotimes_{x\in F}\mathcal H_{x},$$
and the local operator algebra $\mathcal B(\mathcal H_{F})$. The global quasi-local $C^{\ast}$-algebra is
$$\mathcal A=\overline{\bigcup_{F\Subset\Lambda}\mathcal B(\mathcal H_{F})}.$$
A Quantum Cellular Automaton is specified by a $\ast$-automorphism $\alpha:\mathcal A\to\mathcal A$, requiring the existence of a unitary operator $U$ such that
$$\alpha(A)=U^{\dagger}AU,\qquad A\in\mathcal A,$$
and the existence of a finite propagation radius $R<\infty$, such that for any local operator $A$ supported on $F$,
$$\mathrm{supp}\,\alpha(A)\subset B_{R}(F),$$
where $B_{R}(F)$ is the $R$-neighborhood of $F$ in the sense of graph distance. Given an initial state $\omega_{0}$, the discrete time evolution is
$$\omega_{n}=\omega_{0}\circ\alpha^{n},\qquad n\in\mathbb Z.$$
Assume $\Lambda$ can be embedded in three-dimensional Euclidean space with effective lattice spacing $a$, and a single step of evolution corresponds to physical time $\Delta t$. If $R=1$, the maximum propagation speed is
$$c=\frac{a}{\Delta t}.$$
Finite local dimension and finite propagation radius imply that the number of distinguishable physical states in any finite spacetime window is finite, and the information capacity of the universe in any finite region has an upper bound.

\subsection{Effective Space of Single Excitations and External Velocity}

Consider a local "single-excitation" mode, whose effective Hilbert space under appropriate approximation can be represented as
$$\mathcal H_{\mathrm{eff}}\simeq\mathcal H_{\mathrm{COM}}\otimes\mathcal H_{\mathrm{int}},$$
where $\mathcal H_{\mathrm{COM}}$ describes the center-of-mass coordinate or wave packet envelope, and $\mathcal H_{\mathrm{int}}$ describes internal degrees of freedom.

In the continuous limit, approximate position operator $X$ and momentum operator $P$ exist on $\mathcal H_{\mathrm{COM}}$, and the effective Hamiltonian $H_{\mathrm{eff}}$ generates coarse-grained time evolution. Define the external (group) velocity
$$v_{\mathrm{ext}}=\frac{d}{dt}\langle X\rangle=\frac{1}{\mathrm i\hbar}\langle[X,H_{\mathrm{eff}}]\rangle.$$
In symmetric cases, long-wavelength single-particle eigenmodes can be labeled by momentum, $|\psi_{p}\rangle$ satisfying
$$H_{\mathrm{eff}}|\psi_{p}\rangle=E(p)|\psi_{p}\rangle,$$
and the group velocity of this mode is
$$v_{\mathrm{ext}}(p)=\frac{dE}{dp}.$$

\subsection{Internal Hilbert Space and Fubini--Study Metric}

The internal state $|\psi_{\mathrm{int}}(t)\rangle\in\mathcal H_{\mathrm{int}}$ can be viewed as a point on the projective space $\mathbb C P^{D_{\mathrm{int}}-1}$. The Fubini--Study metric
$$ds_{\mathrm{FS}}^{2}=4\bigl(1-|\langle\psi\mid\psi+d\psi\rangle|^{2}\bigr)$$
gives the natural distance between two states in projective Hilbert space. For unitary evolution driven by a time-independent Hamiltonian $H$:
$$\mathrm i\hbar\,\partial_{t}|\psi(t)\rangle=H|\psi(t)\rangle,$$
the Fubini--Study velocity can be defined as
$$v_{\mathrm{FS}}:=\frac{ds_{\mathrm{FS}}}{dt}.$$
For a general state, $v_{\mathrm{FS}}$ is related to energy uncertainty $\Delta H$, while for energy eigenstates, $v_{\mathrm{FS}}=0$. In the framework of this paper, the focus is not on $v_{\mathrm{FS}}$ on the global $\mathcal H$, but on decomposing $H$ into two mutually orthogonal generators corresponding to external translation and internal self-reference, thereby defining "internal evolution velocity" on the internal projective space:
$$v_{\mathrm{int}}:=\frac{ds_{\mathrm{FS}}^{(\mathrm{int})}}{dt}\ge 0.$$
This velocity characterizes the geometric motion rate of the internal state in $\mathbb C P^{D_{\mathrm{int}}-1}$, and its definition depends on the orthogonal decomposition of the Hamiltonian.

\subsection{Dirac-Type QCA and Orthogonal Decomposition of Hamiltonian}

Take the one-dimensional Dirac-type QCA as a concrete model. In the long-wavelength limit, its effective Hamiltonian can be written as
$$H_{\mathrm{eff}}(p)=c\,\hat p\,\sigma_{z}+m c^{2}\sigma_{x},$$
where $\sigma_{x},\sigma_{z}$ are Pauli matrices, $\hat p=-\mathrm i\hbar\partial_{x}$, and $m$ is the effective mass parameter.

Decompose it into
$$H_{T}=c\,\hat p\,\sigma_{z},\qquad H_{M}=m c^{2}\sigma_{x},\qquad H=H_{T}+H_{M}.$$
$H_{T}$ generates external translation, and $H_{M}$ generates internal self-referential rotation. Pauli matrices satisfy the anti-commutation relation
$$\{\sigma_{z},\sigma_{x}\}=\sigma_{z}\sigma_{x}+\sigma_{x}\sigma_{z}=0,$$
and $\sigma_{x}^{2}=\sigma_{z}^{2}=\mathbb I$. Therefore
$$H^{2}=H_{T}^{2}+H_{M}^{2}=(c^{2}\hat p^{2}+m^{2}c^{4})\mathbb I.$$
This gives the operator origin of the relativistic energy-momentum relation
$$E^{2}=p^{2}c^{2}+m^{2}c^{4}.$$

In the Bloch sphere description, internal states correspond to unit vectors on $S^{2}$, and the Hamiltonian $H_{\mathrm{eff}}(p)$ corresponds to the angular velocity vector on the Bloch sphere:
$$\boldsymbol\Omega(p)=\frac{2}{\hbar}\bigl(m c^{2},\,0,\,c p\bigr),$$
whose magnitude
$$|\boldsymbol\Omega(p)|=\frac{2E(p)}{\hbar}$$
gives the total geometric velocity in the internal projective space. Due to the orthogonality of the commutator and anti-commutator structures of $\sigma_{x}$ and $\sigma_{z}$ in the Lie algebra, the "velocity components" corresponding to $H_{T}$ and $H_{M}$ can be understood as two mutually orthogonal directions, the sum of whose squares gives the square of the total speed.

This structure is the algebraic and geometric basis for the Information Rate Conservation Theorem derived later.

\section{Main Results (Theorems and Alignments)}

Under the above model framework, this paper presents the following main results.

\subsection{Theorem 1 (Information Rate Conservation Theorem)}

In any discrete quantum walk/QCA system satisfying local unitarity and translation invariance that emerges a one-dimensional Dirac-type effective Hamiltonian in the long-wavelength limit, for any positive-energy single-particle eigenmode, denote the external group velocity as
$$v_{\mathrm{ext}}(p)=\frac{dE}{dp},$$
and define the internal evolution velocity in the internal projective Hilbert space as
$$v_{\mathrm{int}}(p):=c\,\frac{m c^{2}}{E(p)},$$
then it must hold that
$$v_{\mathrm{ext}}^{2}(p)+v_{\mathrm{int}}^{2}(p)=c^{2},$$
where $c$ is the maximum causal propagation speed of the QCA.

This theorem is guaranteed jointly by the anti-commuting decomposition of the Hamiltonian and the orthogonality of generators under the Fubini--Study metric, and is an inevitable result of local unitarity and Dirac structure, not an additional assumption.

\subsection{Corollary 1 (Emergence of Special Relativity)}

Define proper time $\tau$ using the internal evolution parameter such that
$$v_{\mathrm{int}}\,dt=c\,d\tau.$$
From Theorem 1, we obtain
$$\Bigl(\frac{d\tau}{dt}\Bigr)^{2}=1-\frac{v^{2}}{c^{2}},\qquad v:=v_{\mathrm{ext}}.$$
Defining four-velocity
$$u^{\mu}=\frac{dx^{\mu}}{d\tau}=\gamma(v)\,(c,\mathbf v),\qquad \gamma(v)=\frac{1}{\sqrt{1-v^{2}/c^{2}}},$$
then under the Minkowski metric $\eta_{\mu\nu}=\mathrm{diag}(-1,1,1,1)$, the normalization condition holds:
$$u^{\mu}u_{\mu}=-c^{2},$$
and the corresponding line element is
$$ds^{2}=-c^{2}d\tau^{2}=-c^{2}dt^{2}+d\mathbf x^{2}.$$
Time dilation and velocity normalization of special relativity emerge directly from conservation of information rate.

\subsection{Theorem 2 (Mass as Internal Frequency)}

Introduce a Hamiltonian on the internal Hilbert space $\mathcal H_{\mathrm{int}}$:
$$\mathrm i\hbar\,\partial_{\tau}|\psi_{\mathrm{int}}(\tau)\rangle=H_{\mathrm{int}}|\psi_{\mathrm{int}}(\tau)\rangle.$$
If there exists a stationary state $|\psi_{\mathrm{int}}\rangle$ satisfying
$$H_{\mathrm{int}}|\psi_{\mathrm{int}}\rangle=E_{0}|\psi_{\mathrm{int}}\rangle,$$
the internal state evolves as
$$|\psi_{\mathrm{int}}(\tau)\rangle=\mathrm e^{-\mathrm i E_{0}\tau/\hbar}|\psi_{\mathrm{int}}\rangle.$$
Define internal frequency
$$\omega_{\mathrm{int}}=\frac{E_{0}}{\hbar}.$$
Identifying $E_{0}$ as the rest energy $m c^{2}$, we obtain
$$m=\frac{\hbar\omega_{\mathrm{int}}}{c^{2}}.$$
Mass is given by the internal frequency, expressing the extent to which the internal self-referential structure occupies the optical path quota.

\subsection{Proposition 1 (Zitterbewegung Frequency and Internal Frequency)}

In the continuous limit of one-dimensional Dirac-type QCA, the effective Hamiltonian is
$$H_{\mathrm{eff}}(k)=c\hbar k\sigma_{z}+m c^{2}\sigma_{x},$$
eigenvalues are
$$E_{\pm}(k)=\pm\sqrt{(c\hbar k)^{2}+m^{2}c^{4}}.$$
In the Heisenberg picture, the evolution of the position operator $X(t)$ contains a rapidly oscillating term with frequency
$$\omega_{\mathrm{ZB}}(k)=\frac{2E_{+}(k)}{\hbar}.$$
In the rest limit ($k=0$), $E_{+}(0)=m c^{2}$, so
$$\omega_{\mathrm{ZB}}(0)=\frac{2m c^{2}}{\hbar}=2\omega_{\mathrm{int}}.$$
The Zitterbewegung frequency is twice the internal frequency.

\subsection{Theorem 3 (Topological Stability and Non-Zero Information Phase Angle)}

Consider a one-dimensional translation-invariant QCA whose single-step unitary operator $U(k)\in\mathrm U(N)$ defines a closed curve in momentum space. The winding number
$$\mathcal W[U]=\frac{1}{2\pi\mathrm i}\int_{-\pi/a}^{\pi/a}\partial_{k}\log\det U(k)\,dk\in\mathbb Z$$
remains invariant under finite-depth local unitary transformations.

If $\mathcal W[U]\neq 0$, there exist local excitations carrying non-zero topological charge, which cannot be continuously deformed to the topologically trivial vacuum by any finite-depth local unitary transformation. To maintain topological phase winding, the internal Hamiltonian of such excitations must have a non-zero eigenfrequency $\omega_{\mathrm{int}}>0$, thus $v_{\mathrm{int}}>0$, and the information phase angle $\theta=\arctan(v_{\mathrm{int}}/v_{\mathrm{ext}})$ is non-zero. The existence of mass is therefore stabilized by the topological structure of the QCA.

\subsection{Theorem 4 (Optical Metric and Weak-Field Gravity)}

In the many-body case, introduce coarse-grained local information processing density $\rho_{\mathrm{info}}(x)$, representing the average path length traversed by the internal Hilbert space under the Fubini--Study metric per unit time per unit volume. Allow local rescaling of time and spatial scales in the coordinate system $(t,x^{i})$, introducing a scale factor $\eta(x)$ such that the line element can be written as
$$ds^{2}=-\eta^{2}(x)c^{2}dt^{2}+\eta^{-2}(x)\gamma_{ij}(x)dx^{i}dx^{j},$$
where $\gamma_{ij}(x)$ is the three-dimensional spatial metric. Define coordinate speed of light
$$c_{\mathrm{eff}}(x):=\Bigl|\frac{d\mathbf x}{dt}\Bigr|=\eta^{2}(x)c,$$
and refractive index
$$n(x):=\frac{c}{c_{\mathrm{eff}}(x)}=\eta^{-2}(x).$$
In the static, spherically symmetric, weak-field limit
$$\eta(x)=1+\epsilon(x),\qquad |\epsilon(x)|\ll 1,$$
taking $\epsilon(r)=\phi(r)/c^{2}$, where $\phi(r)=-GM/r$ is the Newtonian potential, we obtain
$$g_{00}\simeq-(1+2\phi/c^{2})c^{2},\qquad g_{ij}\simeq(1-2\phi/c^{2})\delta_{ij},$$
consistent with the first-order expansion of the Schwarzschild metric in isotropic coordinates. The refractive index
$$n(r)\simeq 1-\frac{2\phi(r)}{c^{2}}\simeq 1+\frac{2GM}{c^{2}r}>1,$$
gives the light deflection angle in gravitational lensing theory based on the Gauss--Bonnet theorem:
$$\Delta\theta=\frac{4GM}{c^{2}b},$$
where $b$ is the impact parameter, consistent with the standard result of General Relativity.

\subsection{Theorem 5 (Information-Gravity Variational Principle)}

Consider the action
$$S_{\mathrm{tot}}[g,\rho_{\mathrm{info}}]=\frac{1}{16\pi G}\int_{M}\sqrt{-g}\,R[g]\,d^{4}x+\int_{M}\sqrt{-g}\,\mathcal L_{\mathrm{info}}[\rho_{\mathrm{info}},g]\,d^{4}x.$$
Varying with respect to $g^{\mu\nu}$ (ignoring boundary terms) yields
$$R_{\mu\nu}-\tfrac12 R g_{\mu\nu}=8\pi G\,T^{(\mathrm{info})}_{\mu\nu},$$
where
$$T^{(\mathrm{info})}_{\mu\nu}:=-\frac{2}{\sqrt{-g}}\,\frac{\delta(\sqrt{-g}\,\mathcal L_{\mathrm{info}})}{\delta g^{\mu\nu}}.$$
If in the low-energy limit the choice of $\mathcal L_{\mathrm{info}}$ makes $T^{(\mathrm{info})}_{\mu\nu}$ consistent with the stress-energy tensor of standard matter, this equation is formally equivalent to Einstein's field equations.

\subsection{Theorem 6 (Information Mass and Asymptotic Stationarity)}

For systems with internal models and self-referential mechanisms, introduce information mass
$$M_{I}(\sigma)=f\bigl(K(\sigma),D(\sigma),S_{\mathrm{ent}}(\sigma)\bigr),$$
where $K$ is Kolmogorov complexity, $D$ is logical depth, $S_{\mathrm{ent}}$ is internal entanglement entropy, and $f$ is a monotonically increasing function. Assume the average internal information rate $v_{\mathrm{int}}(M_{I})$ required to maintain a given $M_{I}$ is monotonically increasing and bounded by $c$. From Theorem 1, we have
$$v_{\mathrm{ext}}^{2}(M_{I})=c^{2}-v_{\mathrm{int}}^{2}(M_{I}).$$
If
$$\lim_{M_{I}\to\infty}v_{\mathrm{int}}(M_{I})=c,$$
then
$$\lim_{M_{I}\to\infty}v_{\mathrm{ext}}(M_{I})=0,$$
meaning high-information-mass subjects tend to be asymptotically stationary in external geometry.

\subsection{Proposition 2 (Landauer Cost of Maintaining Information Mass)}

Assume a system updates its internal model at a rate $R_{\mathrm{upd}}$, erasing $\Delta I$ bits of old information on average per update. The rate of information erasure per unit time is
$$\dot I_{\mathrm{erase}}=R_{\mathrm{upd}}\Delta I.$$
In a heat bath at temperature $T$, according to Landauer's principle, erasing one bit of information dissipates at least $k_{B}T\ln 2$ heat. The minimum power consumption is
$$P_{\min}=k_{B}T\ln 2\,\dot I_{\mathrm{erase}}=k_{B}T\ln 2\,R_{\mathrm{upd}}\Delta I.$$
The continued existence of high-information-mass systems is necessarily accompanied by non-zero minimum power consumption and entropy flux output.

\section{Proofs}

This section provides proofs or proof ideas for the main theorems. Detailed calculations and model details are placed in the appendices.

\subsection{Proof of Theorem 1 (Information Rate Conservation Theorem)}

Consider the long-wavelength limit of a one-dimensional Dirac-type QCA with effective Hamiltonian
$$H(p)=H_{T}(p)+H_{M},\qquad H_{T}(p)=c\,p\,\sigma_{z},\qquad H_{M}=m c^{2}\sigma_{x},$$
where $p$ is momentum. Pauli matrices satisfy
$$\sigma_{x}^{2}=\sigma_{z}^{2}=\mathbb I,\qquad \{\sigma_{z},\sigma_{x}\}=0.$$
Therefore
$$H^{2}(p)=H_{T}^{2}(p)+H_{M}^{2}=(c^{2}p^{2}+m^{2}c^{4})\mathbb I.$$
For a positive-energy eigenmode $|\psi_{p}\rangle$, we have
$$H(p)|\psi_{p}\rangle=E(p)|\psi_{p}\rangle,\qquad E^{2}(p)=c^{2}p^{2}+m^{2}c^{4}.$$
The external group velocity is defined as
$$v_{\mathrm{ext}}(p)=\frac{dE}{dp}.$$
From $E^{2}=c^{2}p^{2}+m^{2}c^{4}$, we get
$$2E\,\frac{dE}{dp}=2c^{2}p,$$
so
$$v_{\mathrm{ext}}(p)=\frac{dE}{dp}=\frac{c^{2}p}{E(p)}.$$
The internal part is given by $H_{M}=m c^{2}\sigma_{x}$. Its corresponding "energy share" relative to total energy $E(p)$ is
$$\frac{E_{M}}{E}=\frac{m c^{2}}{E(p)}.$$
Define internal velocity as
$$v_{\mathrm{int}}(p):=c\,\frac{E_{M}}{E}=c\,\frac{m c^{2}}{E(p)}.$$
This definition can be understood as: in the total information rate budget $c$, the component occupied by internal evolution is weighted by the "mass energy share."
Thus we have
$$\frac{v_{\mathrm{ext}}^{2}(p)}{c^{2}}=\frac{c^{4}p^{2}}{c^{2}E^{2}(p)}=\frac{c^{2}p^{2}}{E^{2}(p)},\qquad \frac{v_{\mathrm{int}}^{2}(p)}{c^{2}}=\frac{m^{2}c^{4}}{E^{2}(p)}.$$
Adding them gives
$$\frac{v_{\mathrm{ext}}^{2}(p)}{c^{2}}+\frac{v_{\mathrm{int}}^{2}(p)}{c^{2}}=\frac{c^{2}p^{2}+m^{2}c^{4}}{E^{2}(p)}=1,$$
i.e.,
$$v_{\mathrm{ext}}^{2}(p)+v_{\mathrm{int}}^{2}(p)=c^{2}.$$
Algebraically, this is a direct result of the Hamiltonian decomposing into two anti-commuting operators, leading to the Pythagorean form of energy squared. Geometrically, on the internal two-dimensional Hilbert space, the Hamiltonian can be written as
$$H(p)=\boldsymbol n(p)\cdot\boldsymbol\sigma,\qquad \boldsymbol n(p)=(m c^{2},0,c p),$$
corresponding to the angular velocity vector on the Bloch sphere
$$\boldsymbol\Omega(p)=\frac{2}{\hbar}\boldsymbol n(p)=\frac{2}{\hbar}\bigl(m c^{2},0,c p\bigr).$$
The total velocity under the Fubini--Study metric is proportional to $|\boldsymbol\Omega(p)|$, and the two orthogonal components of $\boldsymbol n(p)$ correspond to internal and external generators, respectively. Since $\sigma_{x}$ and $\sigma_{z}$ are orthogonal in the Lie algebra, the squared magnitude of $\boldsymbol\Omega(p)$ is the sum of the squares of the two components, thus fixing the sum of squares of internal and external information rates to the constant $c^{2}$.

For more general local unitary, translation-invariant QCA, as long as long-wavelength single-particle excitations can be written as a Dirac-type $2\times 2$ effective Hamiltonian in an appropriate basis and after rescaling, the above proof applies immediately. Therefore, the Information Rate Conservation Theorem is a result directly derived from unitarity and Hilbert space geometry in this class of QCA.

\subsection{Proof of Corollary 1 and Theorem 2}

From Theorem 1, take
$$v_{\mathrm{int}}\,dt=c\,d\tau$$
to define proper time. Substituting into
$$v_{\mathrm{ext}}^{2}+v_{\mathrm{int}}^{2}=c^{2}$$
gives
$$v^{2}+c^{2}\Bigl(\frac{d\tau}{dt}\Bigr)^{2}=c^{2},\qquad v:=v_{\mathrm{ext}},$$
i.e.,
$$\Bigl(\frac{d\tau}{dt}\Bigr)^{2}=1-\frac{v^{2}}{c^{2}}.$$
Taking the positive root gives the time dilation relation
$$\frac{d\tau}{dt}=\sqrt{1-\frac{v^{2}}{c^{2}}}.$$
Define Lorentz factor
$$\gamma(v)=\frac{dt}{d\tau}=\frac{1}{\sqrt{1-v^{2}/c^{2}}},$$
Four-velocity
$$u^{\mu}=\frac{dx^{\mu}}{d\tau}=\gamma(v)\,(c,\mathbf v).$$
Under the Minkowski metric
$$u^{\mu}u_{\mu}=-\gamma^{2}c^{2}+\gamma^{2}v^{2}=-\gamma^{2}c^{2}\Bigl(1-\frac{v^{2}}{c^{2}}\Bigr)=-c^{2},$$
thus obtaining the normalization condition and the line element
$$ds^{2}=-c^{2}d\tau^{2}=-c^{2}dt^{2}+d\mathbf x^{2}.$$
The structure of Special Relativity emerges from this.

The proof of Theorem 2 comes directly from the internal evolution equation
$$\mathrm i\hbar\,\partial_{\tau}|\psi_{\mathrm{int}}(\tau)\rangle=H_{\mathrm{int}}|\psi_{\mathrm{int}}(\tau)\rangle.$$
For an eigenstate
$$H_{\mathrm{int}}|\psi_{\mathrm{int}}\rangle=E_{0}|\psi_{\mathrm{int}}\rangle$$
we have
$$|\psi_{\mathrm{int}}(\tau)\rangle=\mathrm e^{-\mathrm i E_{0}\tau/\hbar}|\psi_{\mathrm{int}}\rangle,$$
Define $\omega_{\mathrm{int}}=E_{0}/\hbar$. Taking $E_{0}=m c^{2}$ yields
$$m=\frac{\hbar\omega_{\mathrm{int}}}{c^{2}}.$$

\subsection{Outline of Proof for Proposition 1}

A review of the standard Zitterbewegung derivation for the Dirac Hamiltonian is provided in Appendix C; only key steps are given here. Take the one-dimensional Dirac Hamiltonian
$$H=c\alpha p+\beta m c^{2},$$
In the representation $\alpha=\sigma_{z}$, $\beta=\sigma_{x}$, we have
$$H=c\sigma_{z}p+\sigma_{x}m c^{2}.$$
In the Heisenberg picture, the position operator evolves satisfying
$$\frac{dX}{dt}=\frac{\mathrm i}{\hbar}[H,X]=c\alpha,\qquad \frac{d\alpha}{dt}=\frac{\mathrm i}{\hbar}[H,\alpha].$$
Solving for $\alpha(t)$ and substituting back into $X(t)$ gives
$$X(t)=X(0)+c^{2}H^{-1}Pt+\frac{\mathrm i\hbar c}{2}H^{-1}\bigl(\mathrm e^{-2\mathrm i H t/\hbar}-1\bigr)\bigl(\alpha(0)-cH^{-1}P\bigr),$$
The second term is uniform motion, and the third term is an oscillating term with frequency $2E/\hbar$, where $E=\sqrt{(cP)^{2}+m^{2}c^{4}}$. In the rest limit ($P=0$), $E=m c^{2}$, oscillation frequency
$$\omega_{\mathrm{ZB}}(0)=\frac{2m c^{2}}{\hbar}.$$
Combining with $\omega_{\mathrm{int}}=m c^{2}/\hbar$ from Theorem 2, we get
$$\omega_{\mathrm{ZB}}(0)=2\omega_{\mathrm{int}}.$$

\subsection{Proof Idea of Theorem 3}

The single-step unitary of a translation-invariant QCA can be written as
$$U=\int^{\oplus}U(k)\,dk,$$
Its topological winding number
$$\mathcal W[U]=\frac{1}{2\pi\mathrm i}\int\partial_{k}\log\det U(k)\,dk$$
is a homotopy invariant of the map from the Brillouin zone to $\mathrm U(N)$, remaining invariant under any finite-depth local unitary transformation. If $\mathcal W[U]\neq 0$, there is non-trivial topological phase winding on the other shore manifesting as eigenstates, which cannot be "smoothly wiped out" by local unitaries. To realize this winding, the internal Hamiltonian must have a non-zero eigenfrequency, thus $v_{\mathrm{int}}>0$ and information phase angle $\theta>0$. Specific mathematical details can be made rigorous via QCA index theory and K-theory, see Appendix A and related literature.

\subsection{Proof Idea of Theorem 4 and Theorem 5}

The starting point of Theorem 4 is the optical metric
$$ds^{2}=-\eta^{2}(x)c^{2}dt^{2}+\eta^{-2}(x)\gamma_{ij}(x)dx^{i}dx^{j}.$$
In the isotropic, static, spherically symmetric case, taking $\gamma_{ij}=\delta_{ij}$ and weak-field expansion $\eta(x)=1+\epsilon(x)$ yields
$$g_{00}\simeq-(1+2\epsilon)c^{2},\qquad g_{ij}\simeq(1-2\epsilon)\delta_{ij}.$$
Comparing with the weak-field expansion of the Schwarzschild metric in isotropic coordinates, we can choose $\epsilon(r)=\phi(r)/c^{2}$, where $\phi(r)=-GM/r$. The refractive index
$$n(r)=\eta^{-2}(r)\simeq 1-\frac{2\phi(r)}{c^{2}}\simeq 1+\frac{2GM}{c^{2}r}>1,$$
yields $c_{\mathrm{eff}}(r)=c/n(r)<c$. In the optical geometry-Gauss-Bonnet method proposed by Gibbons, Werner et al., light rays can be treated as geodesics on a 2D Riemann surface, and the deflection angle is calculated by integrating Gaussian curvature:
$$\Delta\theta=\frac{4GM}{c^{2}b}.$$

Theorem 5 uses the standard variation formula for the Einstein-Hilbert term
$$\int\sqrt{-g}\,R\,d^{4}x$$
and
$$\delta(\sqrt{-g}\,\mathcal L_{\mathrm{info}})=\tfrac12\sqrt{-g}\,T^{(\mathrm{info})}_{\mu\nu}\delta g^{\mu\nu},$$
obtaining
$$\delta S_{\mathrm{tot}}=\frac{1}{16\pi G}\int\sqrt{-g}\,\bigl(R_{\mu\nu}-\tfrac12 R g_{\mu\nu}\bigr)\delta g^{\mu\nu}\,d^{4}x+\frac12\int\sqrt{-g}\,T^{(\mathrm{info})}_{\mu\nu}\delta g^{\mu\nu}\,d^{4}x.$$
Setting $\delta S_{\mathrm{tot}}=0$ for arbitrary compactly supported $\delta g^{\mu\nu}$ yields
$$R_{\mu\nu}-\tfrac12 R g_{\mu\nu}=8\pi G\,T^{(\mathrm{info})}_{\mu\nu}.$$

\subsection{Proof Idea of Theorem 6 and Proposition 2}

Theorem 6 comes directly from Theorem 1: assuming $v_{\mathrm{int}}(M_{I})$ increases monotonically with $M_{I}$ and tends to $c$, then
$$v_{\mathrm{ext}}^{2}(M_{I})=c^{2}-v_{\mathrm{int}}^{2}(M_{I})$$
tends to 0 as $M_{I}\to\infty$, so high-information-mass subjects appear increasingly "stationary" geometrically.

Proposition 2 is a direct application of Landauer's principle. Erasing one bit of information dissipates at least $k_{B}T\ln 2$ heat. If
$$\dot I_{\mathrm{erase}}=R_{\mathrm{upd}}\Delta I$$
bits are erased per unit time, the minimum dissipated power is
$$P_{\min}=k_{B}T\ln 2\,\dot I_{\mathrm{erase}}=k_{B}T\ln 2\,R_{\mathrm{upd}}\Delta I.$$
This gives a lower bound on the irreducible power consumption to maintain a fixed information mass.

\section{Model Apply}

\subsection{Free Particles and Special Relativity Limit}

For single-particle Dirac-type QCA, the dispersion relation in the long-wavelength limit
$$E^{2}=p^{2}c^{2}+m^{2}c^{4}$$
is consistent with special relativity. From Corollary 1 and Theorem 2, the four-momentum can be written as
$$p^{\mu}=m u^{\mu},\qquad p^{\mu}p_{\mu}=-m^{2}c^{2}.$$
Massless modes correspond to topologically trivial QCA where internal frequency can be zero, with eigenmodes satisfying $v_{\mathrm{int}}=0, v_{\mathrm{ext}}=c$, meaning the entire information rate budget is used for external propagation, consistent with properties of massless excitations like photons.

\subsection{Many-Body Systems and Astrophysics}

In macroscopic systems like stars, galaxies, and clusters, energy density, particle number density, and information processing density are highly correlated statistically: high energy density typically means a large number of degrees of freedom involved in high-frequency interactions, leading to large $\rho_{\mathrm{info}}(x)$, small $\eta(x)$, large refractive index $n(x)$, and more curved geometry. At this scale, the difference between $\rho_{\mathrm{info}}(x)$ and the stress-energy tensor $T_{\mu\nu}$ can be viewed as renormalization and coarse-graining effects, difficult to distinguish observationally. Thus, the predictions of this framework for classical astrophysical tests are largely consistent with General Relativity.

Distinguishable tests should focus on systems where information structure is variable while total energy is nearly constant, such as quantum media with different entanglement structures or topological phases. In these cases, this framework predicts geometry and effective light speed are sensitive to entanglement structure, while standard GR predicts sensitivity only to energy-momentum distribution, providing an observable difference.

\subsection{Black Hole Entropy and Information Rate Saturation}

Near the black hole horizon, Bekenstein-Hawking entropy is proportional to horizon area, implying a universal upper bound on the number of information degrees of freedom per unit area. In this framework, the horizon can be understood as a region where $\rho_{\mathrm{info}}(x)$ reaches a saturation value: here $\eta(x)$ approaches a limit, causing external propagation to be strongly suppressed, and internal self-referential structure occupies almost the entire optical path quota. Black hole entropy can be interpreted as the "integral of maximum information rate capable of supporting internal self-referential structure" on the horizon.

To strictly derive
$$S_{\mathrm{BH}}=\frac{A}{4G}$$
from QCA information parameters requires combining specific QCA models with renormalization analysis, which can be further developed in appendices and subsequent work.

\subsection{Cosmology and Effective Cosmological Constant}

On cosmological scales, a coarse-grained cosmological average information processing density $\rho_{\mathrm{info}}^{\mathrm{cosmo}}(t)$ can be defined. Its smooth component can be absorbed into an effective "information vacuum energy" term via $\mathcal L_{\mathrm{info}}$, playing a role similar to a cosmological constant or dark energy in the Friedmann equations; if $\rho_{\mathrm{info}}^{\mathrm{cosmo}}(t)$ varies slowly with time, it can produce observational features similar to dynamical dark energy. This direction requires introducing QCA models, renormalization, and observational constraints in a specific cosmological context.

\section{Engineering Proposals}

\subsection{"Information Refractive Index" Experiment in Superconducting Microwave Cavities}

Consider a high-Q superconducting microwave cavity or circuit QED platform with fixed volume $V$, containing several controllable electromagnetic field modes. Through external drive and feedback, various quantum states can be prepared inside the cavity, and their entanglement structure characterized by quantum tomography. Assume via feedback control, the total energy of the cavity $E_{\mathrm{tot}}$ is fixed at a certain value.

Design two types of internal states:
1. State A: Under given $E_{\mathrm{tot}}$, modes are approximately unentangled or weakly entangled, topologically trivial, corresponding to a low information structure state;
2. State B: Under the same $E_{\mathrm{tot}}$, prepare a highly entangled, possibly topologically non-trivial multi-mode quantum state, corresponding to a high information structure state.

In standard GR, the gravitational source is determined by $T_{\mu\nu}$. For Maxwell fields, $T_{\mu\nu}$ macroscopically depends only on energy density and pressure, and entanglement structure does not enter the source term. Therefore, as long as $E_{\mathrm{tot}}$ and spatial distribution are the same, the effects of State A and State B on external geometry should be the same, and Shapiro delay near the cavity should be the same.

In the framework of this paper, a highly entangled or topologically ordered state implies higher $\rho_{\mathrm{info}}(x)$, i.e., more "path length" occurs in the internal Hilbert space per unit time, even if total energy is constant. This leads to changes in $\eta(x)$ and refractive index $n(x)=\eta^{-2}(x)$. If a high-sensitivity interferometer is arranged near the cavity so that the probe beam traverses the region multiple times, a differential phase delay will be measured in the interference fringes when switching between State A and State B:
$$\Delta\phi_{\mathrm{info}}\sim\frac{\omega}{c}\int_{\mathrm{path}}\Delta n(\mathbf x)\,d\ell,$$
where $\omega$ is the optical frequency, and $\Delta n(\mathbf x)$ is the refractive index change caused by the difference in information structure.

Standard GR predicts $\Delta\phi\approx 0$; this framework predicts $\Delta\phi\neq 0$. Observing a stable differential phase delay under controlled systematic errors and noise would constitute direct evidence of "information gravity" versus "energy gravity."

\subsection{Dirac-QCA Information Rate Measurement in Quantum Simulation Platforms}

On platforms such as ion traps, optical lattices, and photonic circuits, discrete-time quantum walks and Dirac-type QCA have been realized, and relativistic features like Zitterbewegung have been observed. By adjusting model parameters, different effective masses and topological structures can be realized; if the internal state evolution rate can be further characterized (e.g., by coherent manipulation and tomography of spin degrees of freedom), one can experimentally test whether
$$v_{\mathrm{ext}}^{2}(k)+v_{\mathrm{int}}^{2}(k)\simeq c^{2}$$
holds approximately, and observe how $v_{\mathrm{int}}(k)$ changes with mass and topological invariants. This would provide quantum simulation support for the picture of "allocation of optical path quota between internal and external."

\section{Discussion (risks, boundaries, past work)}

Existing QCA research has systematically demonstrated mechanisms for emerging free quantum field theories and some interacting field theories from discrete models, and developed topological classification and index theory for QCA. Based on this, this paper introduces the Information Rate Conservation Theorem starting from Hilbert space geometry and local unitarity, viewing Minkowski geometry and special relativity as geometric expressions of information rate constraints, and interpreting Zitterbewegung in Dirac-QCA as the manifestation of internal frequency structure in external coordinates.

The application of optical metric and Gauss--Bonnet theorem in gravitational lensing shows a correspondence between gravitational light bending and geometric optics in inhomogeneous refractive index media. By giving refractive index $n(x)$ a microscopic meaning through $\rho_{\mathrm{info}}(x)$, this paper connects macroscopic optical metrics with microscopic QCA information processing.

There are other schemes for gravitational theories from an information perspective, such as deriving Einstein's equations from local thermodynamics and Clausius relation, and various entropy- and entanglement-based gravity schemes; Landauer's principle provides a universal lower bound for information-energy relations from the perspective of irreversible computation. The uniqueness of this paper lies in: starting from discrete QCA ontology, taking Information Rate Conservation as the core theorem, deriving interpretations of continuous geometry and mass through Hilbert space geometry and topological structure, and providing an optical gravity picture centered on $\rho_{\mathrm{info}}(x)$, $\eta(x)$, and $n(x)$.

Boundaries to be emphasized include: coarse-graining from specific QCA to continuous manifold and metric is generally not unique; the definition of $\rho_{\mathrm{info}}(x)$ needs to avoid gauge dependence on the choice of internal degrees of freedom; computability issues of information mass $M_{I}$ limit its quantitative application in specific systems; and technical challenges in experimentally realizing "information gravity" tests are significant. These issues define the scope of this framework.

\section{Conclusion}

In the framework of Quantum Cellular Automata and finite information ontology, this paper proves an Information Rate Conservation Theorem: in the continuous limit of Dirac-type QCA, the external group velocity ($v_{\mathrm{ext}}$) and internal state evolution velocity ($v_{\mathrm{int}}$) of single-particle long-wavelength excitations must satisfy $v_{\mathrm{ext}}^{2}+v_{\mathrm{int}}^{2}=c^{2}$. This relation originates from the anti-commuting structure of the Hamiltonian and the orthogonal decomposition of generators under the Hilbert space Fubini--Study geometry, and is an inevitable result of local unitarity and finite propagation speed, not an additional assumption.

By defining proper time with internal evolution parameters, time dilation, four-velocity normalization, and Minkowski line element of special relativity can be directly derived from the Information Rate Conservation Theorem; the eigenfrequency of the internal Hamiltonian gives mass ($m=\hbar\omega_{\mathrm{int}}/c^{2}$), Zitterbewegung frequency satisfies $\omega_{\mathrm{ZB}}=2\omega_{\mathrm{int}}$, and mass is stabilized by topological invariants. Introducing local information processing density and optical metric recovers the first-order expansion of the Schwarzschild metric and standard light deflection in the weak-field limit; a unified geometric-information expression formally equivalent to Einstein's equations is obtained through an information-gravity variational principle. Information mass and Landauer cost uniformly characterize the gravitational and dissipative properties of high-complexity energetic systems, where high $M_{I}$ subjects tend to be asymptotically stationary in external geometry and necessarily accompany continuous entropy flux output.

Future work includes: constructing explicit $\mathcal L_{\mathrm{info}}$ in specific QCA models and verifying its consistency with the stress-energy tensor of standard quantum field theory; linking $\rho_{\mathrm{info}}$ with area entropy, cosmological constant, and dark energy in black hole and cosmological contexts; designing feasible experiments in artificial quantum systems to directly measure the behavior of effective light speed or clock frequency varying with entanglement entropy and information density. If these directions are verified, Information Rate Conservation is expected to become a unified path connecting microscopic computational ontology and macroscopic geometric narrative.

\section{Acknowledgements}

This research builds upon existing results in Quantum Cellular Automata, discrete gravity, geometric methods for gravitational lensing, and information thermodynamics, relying especially on systematic studies regarding QCA-QFT continuous limits, QCA topological classification, Gauss--Bonnet based gravitational lensing methods, and Landauer's principle.

This paper did not use numerical simulation code; all derivations are analytical. If future work involves specific QCA models and numerical verification, corresponding code will be organized and released separately.

\begin{thebibliography}{99}
\bibitem{Farrelly2020} T. Farrelly, "A Review of Quantum Cellular Automata", Quantum 4, 368 (2020).
\bibitem{Bisio2013} A. Bisio, G. M. D'Ariano, A. Tosini, "Dirac Quantum Cellular Automaton in One Dimension: Zitterbewegung and Scattering from Potential", Phys. Rev. A 88, 032301 (2013).
\bibitem{Gibbons2008} G. W. Gibbons, M. C. Werner, "Applications of the Gauss--Bonnet Theorem to Gravitational Lensing", Class. Quantum Grav. 25, 235009 (2008).
\bibitem{Halla2020} M. Halla, "Application of the Gauss--Bonnet Theorem to Lensing in Static Spherically Symmetric Spacetimes", Gen. Relativ. Gravit. 52, 95 (2020).
\bibitem{Landauer1961} R. Landauer, "Irreversibility and Heat Generation in the Computing Process", IBM J. Res. Dev. 5, 183–191 (1961).
\bibitem{ReviewRefs} For other literature reviews on QCA topological classification, quantum simulation platforms, and optical metric methods, see [1–4] and references therein.
\end{thebibliography}

\appendix

\section{Appendix A: Continuous Limit and Information Rate Conservation of One-Dimensional Dirac-QCA}

\subsection{A.1 Model Definition}

Consider a 1D lattice $\Lambda=a\mathbb Z$, where each site $x$ carries a two-component spin
$$\psi_{x}=\begin{pmatrix}\psi_{x,\mathrm L}\\\psi_{x,\mathrm R}\end{pmatrix}.$$
Define the conditional shift operator $S$ as
$$(S\psi)_{x,\mathrm L}=\psi_{x+a,\mathrm L},\qquad (S\psi)_{x,\mathrm R}=\psi_{x-a,\mathrm R},$$
Internal rotation
$$W(\theta)=\begin{pmatrix}\cos\theta & -\mathrm i\sin\theta\\-\mathrm i\sin\theta & \cos\theta\end{pmatrix}.$$
Single-step QCA evolution is
$$\psi(t+\Delta t)=U(\theta)\psi(t),\qquad U(\theta):=W(\theta)S.$$
In momentum representation, defining
$$\psi_{k}=\sum_{x}\mathrm e^{-\mathrm i k x}\psi_{x},$$
we have
$$S(k)=\mathrm e^{\mathrm i k a\sigma_{z}},\qquad U(\theta;k)=W(\theta)S(k).$$

\subsection{A.2 Effective Hamiltonian and Dirac Equation}

Taking $a,\Delta t,\theta$ to be small simultaneously, define effective Hamiltonian
$$H_{\mathrm{eff}}(k)=\frac{\mathrm i\hbar}{\Delta t}\,\log U(\theta;k).$$
Expanding for small parameters gives
$$\log U(\theta;k)\simeq\mathrm i k a\sigma_{z}-\mathrm i\theta\sigma_{x},$$
thus
$$H_{\mathrm{eff}}(k)\simeq\frac{\hbar}{\Delta t}(k a\sigma_{z}+\theta\sigma_{x}).$$
Letting
$$c=\frac{a}{\Delta t},\qquad m c^{2}=\frac{\hbar\theta}{\Delta t},$$
we obtain
$$H_{\mathrm{eff}}(k)\simeq c\hbar k\sigma_{z}+m c^{2}\sigma_{x},$$
whose position space form is the one-dimensional Dirac equation
$$\mathrm i\hbar\,\partial_{t}\psi(x,t)=\bigl(-\mathrm i\hbar c\sigma_{z}\partial_{x}+m c^{2}\sigma_{x}\bigr)\psi(x,t).$$
Dispersion relation is
$$E_{\pm}(k)=\pm\sqrt{(c\hbar k)^{2}+m^{2}c^{4}}.$$
Group velocity
$$v_{\mathrm{ext}}(k)=\frac{1}{\hbar}\frac{\partial E_{+}}{\partial k}=\frac{c^{2}k}{\sqrt{k^{2}+(m c/\hbar)^{2}}}<c.$$

\subsection{A.3 Explicit Realization of Internal Velocity and Information Rate Conservation}

For fixed $k$, the internal state is an eigenstate $|u_{+}(k)\rangle$ in the 2D spin space, whose internal phase evolves with frequency $E_{+}(k)/\hbar$. In Bloch sphere representation, the qubit state corresponds to a unit vector $\boldsymbol r\in S^{2}$, and Hamiltonian $H_{\mathrm{eff}}(k)=\boldsymbol n(k)\cdot\boldsymbol\sigma$ generates uniform rotation around $\boldsymbol n(k)$ with angular speed magnitude
$$|\boldsymbol\Omega(k)|=\frac{2|\boldsymbol n(k)|}{\hbar}=\frac{2E_{+}(k)}{\hbar}.$$
The Fubini--Study metric on $\mathbb C P^{1}$ is equivalent to the standard metric on the Bloch sphere, so internal geometric velocity is proportional to $|\boldsymbol\Omega(k)|$.

In Dirac-type QCA
$$\boldsymbol n(k)=(m c^{2},0,c\hbar k).$$
Can be written as
$$|\boldsymbol n(k)|^{2}=(m c^{2})^{2}+(c\hbar k)^{2}.$$
Decomposing the total "angular velocity vector" into
$$\boldsymbol\Omega(k)=\boldsymbol\Omega_{M}(k)+\boldsymbol\Omega_{T}(k),\qquad \boldsymbol\Omega_{M}(k)=\frac{2}{\hbar}(m c^{2},0,0),\qquad \boldsymbol\Omega_{T}(k)=\frac{2}{\hbar}(0,0,c\hbar k).$$
The two components are orthogonal, $|\boldsymbol\Omega(k)|^{2}=|\boldsymbol\Omega_{M}(k)|^{2}+|\boldsymbol\Omega_{T}(k)|^{2}$.

Normalizing the internal velocity $v_{\mathrm{int}}(k)$ as
$$\frac{v_{\mathrm{int}}(k)}{c}=\frac{|\boldsymbol\Omega_{M}(k)|}{|\boldsymbol\Omega(k)|}=\frac{m c^{2}}{E_{+}(k)},$$
and external velocity $v_{\mathrm{ext}}(k)$ given by group velocity
$$\frac{v_{\mathrm{ext}}(k)}{c}=\frac{c\hbar k}{E_{+}(k)}.$$
Then
$$\frac{v_{\mathrm{ext}}^{2}(k)}{c^{2}}+\frac{v_{\mathrm{int}}^{2}(k)}{c^{2}}=\frac{c^{2}\hbar^{2}k^{2}+m^{2}c^{4}}{E_{+}^{2}(k)}=1,$$
i.e.,
$$v_{\mathrm{ext}}^{2}(k)+v_{\mathrm{int}}^{2}(k)=c^{2}.$$
This explicitly realizes the Information Rate Conservation described in Theorem 1 in the Dirac-QCA model.

\section{Appendix B: Optical Metric, Local Volume Conservation, and Light Deflection}

\subsection{B.1 Local Volume Element and Scale Factor Constraints}

Under the metric
$$ds^{2}=-\eta^{2}(x)c^{2}dt^{2}+\eta^{-2}(x)\gamma_{ij}(x)dx^{i}dx^{j}$$
the four-volume element is
$$dV_{4}=\sqrt{-g}\,d^{4}x.$$
Under isotropy assumption, the 3D spatial metric can be written as
$$\gamma_{ij}dx^{i}dx^{j}=\Psi^{4}(x)\,d\mathbf x^{2},$$
then
$$\sqrt{-g}\propto\eta(x)\,\eta^{-3}(x)\Psi^{6}(x)=\eta^{-2}(x)\Psi^{6}(x).$$
Local Hilbert volume conservation can be simplified as "physical Hilbert volume corresponding to unit coordinate volume is invariant," expressed by constraint
$$\eta_{t}(x)\,\eta_{x}^{3}(x)=1.$$
Under isotropy, taking
$$\eta_{t}(x)=\eta(x),\qquad \eta_{x}(x)=\eta^{-1}(x),$$
yields the optical metric form
$$ds^{2}=-\eta^{2}(x)c^{2}dt^{2}+\eta^{-2}(x)\gamma_{ij}(x)dx^{i}dx^{j}.$$

\subsection{B.2 Weak-Field Expansion and Schwarzschild Metric}

In static, spherically symmetric case, Schwarzschild metric in isotropic coordinates $(t,r,\theta,\varphi)$ is
$$ds^{2}=-\Bigl[\frac{1-\frac{GM}{2c^{2}r}}{1+\frac{GM}{2c^{2}r}}\Bigr]^{2}c^{2}dt^{2}+\Bigl(1+\frac{GM}{2c^{2}r}\Bigr)^{4}(dr^{2}+r^{2}d\Omega^{2}).$$
Expanding under weak-field approximation $\frac{GM}{c^{2}r}\ll 1$ gives
$$g_{00}\simeq-(1-\frac{2GM}{c^{2}r})c^{2}=-(1+2\phi/c^{2})c^{2},$$
$$g_{ij}\simeq(1+\frac{2GM}{c^{2}r})\delta_{ij}=(1-2\phi/c^{2})\delta_{ij},$$
where $\phi(r)=-GM/r$ is the Newtonian potential.

Starting from the optical metric, taking
$$\eta(r)=1+\frac{\phi(r)}{c^{2}},\qquad \gamma_{ij}=\delta_{ij},$$
expanding gives
$$g_{00}=-\eta^{2}c^{2}\simeq-(1+2\phi/c^{2})c^{2},$$
$$g_{ij}=\eta^{-2}\delta_{ij}\simeq(1-2\phi/c^{2})\delta_{ij},$$
consistent with the weak-field expansion of the Schwarzschild metric.

\subsection{B.3 Refractive Index and Light Deflection Angle}

For null geodesics $ds^{2}=0$ under the optical metric,
$$\eta^{2}(r)c^{2}dt^{2}=\eta^{-2}(r)\,d\mathbf x^{2},$$
Coordinate speed of light
$$c_{\mathrm{eff}}(r)=\Bigl|\frac{d\mathbf x}{dt}\Bigr|=\eta^{2}(r)c.$$
Refractive index
$$n(r):=\frac{c}{c_{\mathrm{eff}}(r)}=\eta^{-2}(r).$$
In the weak-field limit
$$\eta(r)=1+\frac{\phi(r)}{c^{2}},\qquad \Bigl|\frac{\phi(r)}{c^{2}}\Bigr|\ll 1,$$
first-order expansion gives
$$n(r)=\eta^{-2}(r)\simeq 1-\frac{2\phi(r)}{c^{2}}.$$
Taking $\phi(r)=-GM/r$ gives
$$n(r)\simeq 1+\frac{2GM}{c^{2}r}>1,\qquad c_{\mathrm{eff}}(r)=\frac{c}{n(r)}<c,$$
consistent with the physical picture of light slowing down in weak gravity.

In the Gibbons--Werner method, the optical metric can be viewed as a 2D Riemann surface, and light trajectories as geodesics on this surface. Calculating the deflection angle via the Gauss--Bonnet theorem in the weak deflection limit yields
$$\Delta\theta\simeq\frac{4GM}{c^{2}b},$$
consistent with Einstein's prediction. If only $g_{00}$ is modified while keeping spatial metric flat, the corresponding refractive index is only
$$n(r)\simeq 1-\frac{\phi(r)}{c^{2}},$$
and the deflection angle would be half of the above value, showing that simultaneously deforming time and spatial scales (i.e., introducing optical metric) is crucial for recovering the correct deflection factor.

\section{Appendix C: Zitterbewegung and Internal Frequency in Dirac Theory}

\subsection{C.1 Dirac Equation and Plane Wave Solution}

Consider 1D Dirac equation
$$\mathrm i\hbar\,\partial_{t}\psi=(c\alpha p+\beta m c^{2})\psi,$$
taking representation $\alpha=\sigma_{z}$, $\beta=\sigma_{x}$, $p=-\mathrm i\hbar\,\partial_{x}$. Plane wave solutions are
$$\psi_{k,\pm}(x,t)=u_{\pm}(k)\,\mathrm e^{\mathrm i(kx-\omega_{\pm}t)},$$
$$\omega_{\pm}=\pm\sqrt{(c k)^{2}+\frac{m^{2}c^{4}}{\hbar^{2}}}.$$
General wave packet can be written as
$$\psi(x,t)=\int\bigl(a_{+}(k)\psi_{k,+}(x,t)+a_{-}(k)\psi_{k,-}(x,t)\bigr)\,dk.$$

\subsection{C.2 Position Operator in Heisenberg Picture}

In Heisenberg picture, position operator evolves as
$$X(t)=\mathrm e^{\mathrm i H t/\hbar}X(0)\mathrm e^{-\mathrm i H t/\hbar},\qquad H=c\alpha p+\beta m c^{2}.$$
Heisenberg equations give
$$\frac{dX}{dt}=\frac{\mathrm i}{\hbar}[H,X]=c\alpha,\qquad \frac{d\alpha}{dt}=\frac{\mathrm i}{\hbar}[H,\alpha].$$
Solving for $\alpha(t)$ and substituting back into $X(t)$ gives
$$X(t)=X(0)+c^{2}H^{-1}Pt+\frac{\mathrm i\hbar c}{2}H^{-1}\bigl(\mathrm e^{-2\mathrm i H t/\hbar}-1\bigr)\bigl(\alpha(0)-cH^{-1}P\bigr),$$
where the second term is uniform motion, and the third term is a rapidly oscillating term with frequency $2E/\hbar$, i.e., Zitterbewegung, where $E=\sqrt{(cP)^{2}+m^{2}c^{4}}$ is the positive energy branch. In the rest limit ($P=0$), $E=m c^{2}$, oscillation frequency
$$\omega_{\mathrm{ZB}}(0)=\frac{2m c^{2}}{\hbar}.$$
In the framework of this paper, internal frequency is defined as
$$\omega_{\mathrm{int}}=\frac{m c^{2}}{\hbar},$$
thus
$$\omega_{\mathrm{ZB}}(0)=2\omega_{\mathrm{int}}.$$

\section{Appendix D: Information Mass, Landauer's Principle, and Minimum Power Consumption}

\subsection{D.1 Information Erasure and Minimum Dissipation}

Assume a system's internal state is $\sigma$, and its information mass $M_{I}(\sigma)$ is related to the scale and complexity of its internal model. To maintain model validity, the system must regularly update internal representations with new observations, a process necessarily involving erasure of some old information.

Assume the system updates the model at rate $R_{\mathrm{upd}}$, erasing $\Delta I$ bits of old information on average per update. The amount of information erased per unit time is
$$\dot I_{\mathrm{erase}}=R_{\mathrm{upd}}\Delta I.$$
In a heat bath at temperature $T$, according to Landauer's principle, erasing one bit of information must dissipate at least $k_{B}T\ln 2$ heat into the environment; minimum dissipated power is
$$P_{\min}=k_{B}T\ln 2\,\dot I_{\mathrm{erase}}=k_{B}T\ln 2\,R_{\mathrm{upd}}\Delta I.$$
This result is independent of the specific physical implementation of the system, depending only on the number of bits erased and environmental temperature, serving as a universal lower bound for power consumption required to implement any high-information-mass system.

\subsection{D.2 High Information Mass and High Dissipation}

If information mass $M_{I}(\sigma)$ increases with the scale, structural complexity, and update frequency of the internal model, then larger $M_{I}$ typically requires larger $R_{\mathrm{upd}}$ and $\Delta I$ to continuously discard obsolete information and introduce new information. Therefore, in general
$$P_{\min}(M_{I})=k_{B}T\ln 2\,R_{\mathrm{upd}}(M_{I})\,\Delta I(M_{I})$$
will increase with increasing $M_{I}$.

Combining with the Information Rate Conservation Theorem of this paper, we arrive at a unified picture:
1. To maintain high $M_{I}$, the system must allocate a large amount of optical path quota to internal evolution (large $v_{\mathrm{int}}$), thereby limiting external motion speed $v_{\mathrm{ext}}$, manifesting as asymptotic stationarity;
2. Meanwhile, frequent internal updates and erasures lead to continuous entropy flow to the outside, making the system a significant heat source, manifesting as strong dissipative characteristics;
3. Macroscopically, these two effects often appear in regions with deep gravitational potentials: stellar interiors maintain high information density and complex structure via nuclear reactions while emitting massive radiation; biological and neural systems maintain low entropy structures via metabolism while outputting heat to the environment.

Therefore, from the perspective of Information Rate Conservation, the connection between mass, gravity, and complex energetic structures can be unifiedly understood as: to maintain a highly ordered, topologically stable structure internally, one must continuously consume optical path quota and output entropy and energy, and this process manifests geometrically as curvature and dynamically as gravity.

\end{document}
