\documentclass[12pt]{article}
\usepackage[utf8]{inputenc}
\usepackage[T1]{fontenc}
\usepackage{amsmath,amssymb,amsthm}
\usepackage{mathrsfs}
\usepackage{geometry}
\usepackage{hyperref}
\usepackage{braket}
\usepackage{graphicx}

\geometry{a4paper, margin=1in}
\hypersetup{colorlinks=true,linkcolor=blue,citecolor=blue,urlcolor=blue}

\theoremstyle{plain}
\newtheorem{theorem}{Theorem}[section]
\newtheorem{lemma}[theorem]{Lemma}
\newtheorem{proposition}[theorem]{Proposition}
\newtheorem{corollary}[theorem]{Corollary}

\theoremstyle{definition}
\newtheorem{definition}[theorem]{Definition}
\newtheorem{example}[theorem]{Example}
\newtheorem{remark}[theorem]{Remark}

\title{Mass as Topological Impedance: Self-Referential Scattering and Chiral Symmetry Breaking in Dirac-QCA}

\author{Haobo Ma$^1$ \and Wenlin Zhang$^2$\\
\small $^1$Independent Researcher\\
\small $^2$National University of Singapore}

\date{\today}

\begin{document}

\maketitle

\begin{abstract}
In relativistic quantum field theory, the masses of elementary particles are usually introduced via Yukawa couplings to the Higgs field and appear as free parameters, with no microscopic information-theoretic or topological origin. Within a quantum cellular automaton (QCA) ontology, however, the universe is modelled as a discrete, strictly causal quantum update rule on a lattice, with an intrinsic speed limit $c$ set by the light-cone of local unitaries. This raises a structural question: why do certain excitations propagate subluminally and behave as massive particles, rather than as massless signals that saturate the update speed?

In this work we treat a one-dimensional Dirac-type QCA, formulated as a discrete-time quantum walk with internal (coin) degrees of freedom. Building on earlier results that show the Dirac equation emerging as the continuum limit of such models, we prove that the effective Dirac mass is nothing but the amplitude of a locally self-referential scattering process that mixes left- and right-moving components. At the level of the Floquet operator in momentum space, this mixing induces a non-trivial winding of the Bloch vector over the Brillouin zone, placing the model in a chiral-symmetric Floquet topological phase. The non-zero mass then acquires the interpretation of a topological impedance: a topologically protected obstruction that prevents the decoupling of chiral components into purely ballistic, lightlike propagation.

We show that this mechanism produces Zitterbewegung as an unavoidable consequence of the local back-scattering. In the Dirac continuum limit the position operator splits into a uniform drift term with group velocity $v_{\mathrm{ext}}$ and an oscillatory term with frequency $\omega_{\mathrm{ZB}} = 2 E / \hbar$, where $E$ is the quasienergy. We interpret this oscillation as "internal motion" and demonstrate a precise decomposition
$$
v_{\mathrm{ext}}^{2} + v_{\mathrm{int}}^{2} = c^{2},
$$
where $v_{\mathrm{int}}$ is defined as the standard deviation of the velocity operator. Thus every excitation saturates the microscopic information speed bound $c$, but a topologically enforced share of this budget is trapped in internal oscillations whenever the mass gap is non-zero. This provides a purely kinematical, topological and information-theoretic interpretation of mass in Dirac-QCA models, compatible with and complementary to conventional Higgs-type mass generation.
\end{abstract}

\noindent\textbf{Keywords:} Quantum cellular automaton; discrete-time quantum walk; Dirac equation; topological mass; winding number; chiral symmetry; Zitterbewegung; Floquet topological phase; information speed conservation

---

\section{Introduction \& Historical Context}

\subsection{Mass and its open conceptual status}

In relativistic quantum field theory (QFT), the free Dirac equation
$$
\left( \mathrm{i} \hbar \gamma^{\mu} \partial_{\mu} - m c \right) \psi = 0
$$
introduces the rest mass $m$ as a parameter in the Lagrangian, while in the Standard Model masses arise from Yukawa couplings to the Higgs field. Although this mechanism is phenomenologically successful, it leaves two conceptual issues:
\begin{enumerate}
    \item The numerical values of fermion masses are free parameters, not fixed by deeper principles.
    \item The mass term $m \bar{\psi} \psi$ breaks chiral symmetry explicitly, but this breaking is not directly tied to any topological invariant or discrete information-processing structure.
\end{enumerate}
Outside the Standard Model, a number of "topological mass" mechanisms are known, such as Chern–Simons terms in $(2+1)$-dimensional gauge theories, or domain-wall and overlap fermions in lattice QCD, where a mass gap is associated with index theorems and spectral flow. However, these mechanisms still presuppose a continuum quantum field, and they are usually formulated in terms of effective actions rather than underlying discrete information dynamics.

\subsection{QCA and quantum walks as discrete Dirac dynamics}

Quantum cellular automata (QCAs) provide an alternative ontology in which dynamics is defined by a causal, translation-invariant unitary on a lattice of finite-dimensional quantum systems. The idea that quantum field theory might emerge from such automata dates back to Feynman's suggestion that physics could be simulated by quantum computers, and has since been developed in detail by Bisio, D'Ariano, Tosini and others, who construct QCAs whose long-wavelength limit reproduces Weyl, Dirac and Maxwell equations.

In one spatial dimension, a particularly simple realization of Dirac-type dynamics is provided by discrete-time quantum walks (DTQWs), in which a two-component "coin" degree of freedom controls conditional shifts to left and right. Strauch showed that for suitable scaling of the coin parameters and lattice spacing the continuum limit of a DTQW reproduces the $(1+1)$-dimensional Dirac equation, and subsequent work generalized this connection to higher dimensions and more general walk protocols.

In parallel, a systematic theory of QCAs has been developed, including locality and index theorems classifying one-dimensional automata up to homotopy. Within this setting, DTQWs can be viewed as special cases of one-dimensional QCAs with internal degrees of freedom (coins) and finite propagation radius.

\subsection{Topological phases and quantum walks}

Topological band theory, first developed for static Hamiltonians, has been extended to periodically driven (Floquet) systems, where the time-evolution operator over one period plays the role of a Floquet unitary. The topological classification of such unitaries, especially in the presence of chiral symmetry, naturally applies to DTQWs. Kitagawa and collaborators showed that split-step and multi-step quantum walks realize a rich set of one-dimensional topological phases with integer winding numbers, accompanied by robust edge states at boundaries between domains of different invariants.

Asbóth and Obuse later formulated a bulk-boundary correspondence and a systematic definition of chiral symmetry for Floquet quantum walks, identifying a $\mathbb{Z} \times \mathbb{Z}$ valued topological invariant controlling zero and $\pi$ quasienergy edge states. Experiments using photonic and cold-atom platforms have directly observed these topological bound states and measured winding numbers and dynamical topological order parameters.

Importantly for our purposes, even the simplest one-coin quantum walk, when formulated with an appropriate coin rotation, exhibits a hidden chiral symmetry and a non-trivial winding number for generic coin angles, with the gap closing and winding changing only at special values of the coin parameter.

\subsection{Zitterbewegung in Dirac dynamics and QCAs}

Zitterbewegung (ZB), the rapid oscillatory motion predicted by the Dirac equation, has long been understood as arising from interference between positive and negative energy components of a wave packet. In the Heisenberg picture the position operator splits into a uniform drift term plus an oscillatory term with frequency $2 E / \hbar$, and the velocity operator has eigenvalues $\pm c$, so that the velocity fluctuates between these extreme values even when average motion is subluminal.

In QCA and quantum walk models that approximate Dirac dynamics, the same phenomenon has been analysed both analytically and numerically: the Dirac QCA constructed by Bisio et al. exhibits ZB and Klein tunnelling, with oscillation frequency and amplitude matching those of the continuum Dirac theory in the appropriate limit.

\subsection{Information speed conservation and the role of mass}

In a previous work we proposed an information-theoretic conservation law for local excitations in QCA-type models, namely
$$
v_{\mathrm{ext}}^{2} + v_{\mathrm{int}}^{2} = c^{2},
$$
where $v_{\mathrm{ext}}$ measures the effective group velocity of an excitation in physical space and $v_{\mathrm{int}}$ quantifies the internal evolution rate in Hilbert space, in terms of the variance of the velocity operator or the Fubini–Study metric. This relation expresses the idea that the "information speed budget" is fixed at $c$, with massless excitations allocating all of it to translational motion, while massive ones divert part of it into internal oscillations.

The present paper implements this idea concretely in a Dirac-QCA setting and ties it to topological invariants.

\subsection{Goals and contributions}

The central objective is to reinterpret the mass parameter of the Dirac equation, as realized in a Dirac-QCA model, as a topological impedance: a quantized obstruction to decoupling left- and right-moving degrees of freedom into independent ballistic channels. Concretely, we:

1. Specify a one-dimensional Dirac-type QCA as a discrete-time quantum walk with a single coin angle and show, under standard locality and homogeneity assumptions, that its long-wavelength limit reproduces the $(1+1)$-dimensional Dirac equation with mass $m$ determined by the coin angle.

2. Show that the timestep unitary in momentum space has chiral symmetry and defines a non-trivial winding of the Bloch vector on the Bloch sphere for any non-zero coin angle, in agreement with earlier classifications of Floquet topological phases in quantum walks. This winding number changes only when the gap at quasienergies $0$ or $\pi$ closes, corresponding to $m = 0$.

3. Give a scattering-theoretic interpretation of the coin angle as a local self-referential scattering amplitude that repeatedly converts left-moving components into right-moving ones and vice versa. This produces subluminal group velocities and Zitterbewegung, and we interpret the associated energy gap as a topological impedance blocking purely lightlike propagation.

4. Derive in the Dirac continuum limit an exact decomposition
   $$
   v_{\mathrm{ext}}^{2} + v_{\mathrm{int}}^{2} = c^{2}
   $$
   for arbitrary states, by defining $v_{\mathrm{ext}}$ as the expectation value of the velocity operator and $v_{\mathrm{int}}$ as the square root of its variance, thereby giving a precise information-theoretic meaning to internal motion and linking it to mass.

5. Discuss how the topological impedance picture interacts with conventional mass generation mechanisms, and propose experimental schemes for observing the trade-off between group velocity, internal oscillations and topological invariants in photonic and cold-atom quantum walk platforms.

---

\section{Model \& Assumptions}

\subsection{Lattice, Hilbert space and locality}

We consider a one-dimensional spatial lattice with spacing $a > 0$, whose sites are labelled by integers $n \in \mathbb{Z}$, corresponding to positions $x = n a$. Time is discrete with step $\Delta t > 0$, so that $t = m \Delta t$ for $m \in \mathbb{Z}$. The microscopic causal speed is fixed by
$$
c = \frac{a}{\Delta t},
$$
which will be identified with the emergent speed of light in the continuum limit.

At each lattice site the local Hilbert space is a two-dimensional coin space $\mathbb{C}^{2}$, interpreted as a chiral or left/right degree of freedom. The global Hilbert space is
$$
\mathcal{H} = \ell^{2}(\mathbb{Z}) \otimes \mathbb{C}^{2}.
$$

We write states in the position-coin basis as
$$
\Psi = \sum_{n \in \mathbb{Z}} \left( \psi_{L}(n) \lvert n \rangle \otimes \lvert L \rangle + \psi_{R}(n) \lvert n \rangle \otimes \lvert R \rangle \right),
$$
or in column vector form
$$
\Psi(n, t) = \begin{pmatrix} \psi_{L}(n, t) \\ \psi_{R}(n, t) \end{pmatrix}.
$$

We assume a single-step, translation-invariant QCA evolution operator
$$
U : \mathcal{H} \to \mathcal{H},
$$
which is unitary, homogeneous (commutes with spatial translations), causal with finite propagation radius (each site couples only to a bounded neighbourhood in a single timestep) and parity symmetric.

\subsection{Coin operator and conditional shift}

The timestep unitary is taken to factor into a local coin rotation followed by a conditional shift:
$$
U = S \, C(\theta),
$$
where $\theta \in [0,\pi]$ is a real parameter.

The coin operator is a site-local unitary of the form
$$
C(\theta) = \sum_{n \in \mathbb{Z}} \lvert n \rangle \langle n \rvert \otimes R(\theta),
$$
where
$$
R(\theta) = \mathrm{e}^{-\mathrm{i} \theta \sigma_{x}}
= \begin{pmatrix}
\cos \theta & -\mathrm{i} \sin \theta \\
-\mathrm{i} \sin \theta & \cos \theta
\end{pmatrix}
$$
and $\sigma_{x}$ is the usual Pauli matrix.

The conditional shift operator moves left and right components in opposite directions:
$$
S = \sum_{n \in \mathbb{Z}} \left( \lvert n - 1 \rangle \langle n \rvert \otimes \lvert L \rangle \langle L \rvert
+ \lvert n + 1 \rangle \langle n \rvert \otimes \lvert R \rangle \langle R \rvert \right).
$$
Equivalently,
$$
S = \sum_{n \in \mathbb{Z}} \lvert n \rangle \langle n \rvert \otimes \left( \lvert L \rangle \langle L \rvert \, T_{+} + \lvert R \rangle \langle R \rvert \, T_{-} \right),
$$
where $T_{\pm}$ are lattice translation operators ($T_{\pm} \lvert n \rangle = \lvert n \pm 1 \rangle$).

The evolution equation over one timestep is
$$
\Psi(t + \Delta t) = U \Psi(t).
$$
In position space this yields coupled update equations:
$$
\begin{aligned}
\psi_{L}(n, t + \Delta t) &= \cos \theta \, \psi_{L}(n + 1, t) - \mathrm{i} \sin \theta \, \psi_{R}(n + 1, t), \\
\psi_{R}(n, t + \Delta t) &= - \mathrm{i} \sin \theta \, \psi_{L}(n - 1, t) + \cos \theta \, \psi_{R}(n - 1, t).
\end{aligned}
$$
We interpret the parameter $\theta$ as controlling the strength of a local "self-scattering" between left- and right-moving components; $\theta = 0$ corresponds to purely ballistic motion at speed $c$, while $\theta \neq 0$ introduces back-scattering.

\subsection{Momentum-space representation and effective Hamiltonian}

By translation invariance, we can pass to momentum space via
$$
\Psi(n, t) = \int_{-\pi}^{\pi} \frac{\mathrm{d} k}{2 \pi} \, \mathrm{e}^{\mathrm{i} k n} \, \widetilde{\Psi}(k, t),
$$
with $k \in [-\pi,\pi]$ the dimensionless lattice momentum; the physical momentum is $p = \hbar k / a$.

The shift operator is diagonal in $k$,
$$
\widetilde{S}(k) = \mathrm{e}^{-\mathrm{i} k \sigma_{z}},
$$
and the timestep unitary in momentum space is
$$
\widetilde{U}(k) = \widetilde{S}(k) R(\theta)
= \begin{pmatrix}
\mathrm{e}^{-\mathrm{i} k} \cos \theta & - \mathrm{i} \mathrm{e}^{-\mathrm{i} k} \sin \theta \\
\mathrm{i} \mathrm{e}^{\mathrm{i} k} \sin \theta & \mathrm{e}^{\mathrm{i} k} \cos \theta
\end{pmatrix}.
$$

We define the Floquet quasienergies $E(k)$ and effective Hamiltonian $H_{\mathrm{eff}}(k)$ via
$$
\widetilde{U}(k) = \exp\left( - \frac{\mathrm{i}}{\hbar} H_{\mathrm{eff}}(k) \, \Delta t \right).
$$
The eigenvalues of $\widetilde{U}(k)$ are $\mathrm{e}^{-\mathrm{i} E(k) \Delta t}$ and $\mathrm{e}^{+\mathrm{i} E(k) \Delta t}$, with
$$
\cos \bigl( E(k) \Delta t \bigr) = \cos \theta \, \cos k,
$$
a well-known dispersion relation for coined quantum walks.

We can write
$$
H_{\mathrm{eff}}(k) = E(k) \, \hat{\boldsymbol{n}}(k) \cdot \boldsymbol{\sigma},
$$
with $\hat{\boldsymbol{n}}(k)$ a unit vector on the Bloch sphere.

---

\section{Main Results (Theorems and Alignments)}

In this section we summarize the main theorems; detailed proofs are given in the subsequent section and in the appendices.

\begin{theorem}[Dirac continuum limit]
Let the lattice spacing and timestep obey $a = c \, \Delta t$, and scale the coin angle as
$$
\theta = \frac{m c^{2}}{\hbar} \, \Delta t + \mathcal{O}(\Delta t^{3}),
$$
with fixed parameters $c > 0$ and $m \ge 0$. Consider initial states whose amplitudes vary slowly on the lattice scale, in the sense that
$$
\psi_{L/R}(n \pm 1, 0) - \psi_{L/R}(n, 0) = \mathcal{O}(a)
$$
uniformly in $n$. Then in the limit $\Delta t \to 0$ the discrete evolution converges, up to errors of order $\mathcal{O}(\Delta t^{2})$, to the continuum Dirac equation in $(1+1)$ dimensions,
$$
\mathrm{i} \hbar \, \partial_{t} \Psi(x,t)
= \left( - \mathrm{i} \hbar c \, \sigma_{z} \partial_{x} + m c^{2} \sigma_{x} \right) \Psi(x,t),
$$
with $x = n a$. The convergence holds in operator norm on any finite time interval and for wave packets supported in a compact momentum region.
\end{theorem}

\begin{theorem}[Chiral symmetry and winding number]
Define a chiral symmetry operator
$$
\Gamma = \sigma_{x}.
$$
Then for all momenta $k$ the Floquet unitary satisfies
$$
\Gamma \, \widetilde{U}(k) \, \Gamma = \widetilde{U}^{\dagger}(k),
$$
so the QCA is in the chiral-symmetric Floquet class AIII. The corresponding one-dimensional topological invariant (winding number) is
$$
\mathcal{W} = \frac{1}{2 \pi \mathrm{i}} \int_{-\pi}^{\pi} \mathrm{d} k \,
\mathrm{Tr} \left( \Gamma \, \widetilde{U}^{-1}(k) \, \partial_{k} \widetilde{U}(k) \right).
$$
For the present model,
$$
\mathcal{W} =
\begin{cases}
0, & \theta = 0 \text{ or } \theta = \pi, \\
1, & 0 < \theta < \pi.
\end{cases}
$$
Thus any non-zero mass parameter $m$, as defined in Theorem 1, corresponds to a topologically non-trivial Floquet phase that cannot be adiabatically deformed to the massless case without closing the quasienergy gap at $E = 0$ or $E = \pi / \Delta t$. This is consistent with earlier analyses of topological phases in one-coin and split-step quantum walks.
\end{theorem}

\begin{theorem}[Mass as self-referential scattering]
In position space, the single-step evolution at site $n$ can be written as the action of a local scattering matrix on an incoming two-component field comprised of left- and right-moving amplitudes. Explicitly, in momentum space, which is equivalent to a two-channel unitary scattering with reflection and transmission amplitudes
$$
r(\theta) = - \mathrm{i} \sin \theta,
\quad
t(\theta) = \cos \theta.
$$
For a wave packet sharply peaked around momentum $k_{0}$, the group velocity satisfies
$$
v_{\mathrm{ext}}(k_{0},\theta) = \frac{\partial E(k)}{\partial p} \Bigg\rvert_{k=k_{0}}
= \frac{1}{\hbar} \frac{\partial E(k)}{\partial k} \, a,
$$
with dispersion $\cos(E \Delta t) = \cos \theta \cos k$. In the Dirac continuum limit described in Theorem 1, this yields
$$
v_{\mathrm{ext}}(p) \approx \frac{c^{2} p}{\sqrt{p^{2} c^{2} + m^{2} c^{4}}}
$$
and an energy gap
$$
E(p) \approx \sqrt{p^{2} c^{2} + m^{2} c^{4}}.
$$
Thus the same parameter $\theta$ simultaneously controls (i) the size of the mass gap in the Dirac limit and (ii) the strength of local back-scattering between left- and right-moving components. Repeated application of this local scattering produces an effective inertia: the stronger the mixing (larger $\theta$), the smaller the asymptotic group velocity at given momentum. We interpret this as mass arising from a self-referential scattering process that ties the excitation to its own history.
\end{theorem}

\begin{theorem}[Zitterbewegung and information speed decomposition]
Let $H = c p \sigma_{z} + m c^{2} \sigma_{x}$ denote the Dirac Hamiltonian in the continuum limit, and define the velocity operator
$$
\hat{v} = \frac{\mathrm{i}}{\hbar} [H, X] = c \sigma_{z}.
$$
Then for any normalized state $\Psi$,
$$
\langle \hat{v}^{2} \rangle = c^{2},
$$
and we can define
$$
v_{\mathrm{ext}} = \langle \hat{v} \rangle,
\quad
v_{\mathrm{int}} = \sqrt{ \langle \hat{v}^{2} \rangle - \langle \hat{v} \rangle^{2} }.
$$
For all states,
$$
v_{\mathrm{ext}}^{2} + v_{\mathrm{int}}^{2} = c^{2}.
$$
Moreover, in the Heisenberg picture the position operator decomposes as
$$
X(t) = X(0) + v_{\mathrm{ext}} t + \Xi(t),
$$
where $\Xi(t)$ is an oscillatory term with frequency $\omega_{\mathrm{ZB}} = 2 E / \hbar$ and amplitude proportional to $\hbar c / (2 E)$. This is the usual Zitterbewegung term, which we thereby interpret as the manifestation of the internal information speed $v_{\mathrm{int}}$. Massive excitations necessarily have $0 < \lvert v_{\mathrm{ext}} \rvert < c$ and corresponding non-zero $v_{\mathrm{int}}$; massless ones satisfy $\lvert v_{\mathrm{ext}} \rvert = c$ and $v_{\mathrm{int}} = 0$.

In the discrete QCA, a similar decomposition holds at the level of the effective Hamiltonian and velocity operator in the long-wavelength regime, and Zitterbewegung appears as a fast oscillation of the expectation value of the position operator superimposed on the group-velocity drift.
\end{theorem}

---

\section{Proofs}

In this section we sketch the main arguments; detailed step-by-step derivations are deferred to the appendices.

\subsection{Proof of Theorem 1 (Dirac continuum limit)}

We set $a = c \Delta t$ and $\theta = (m c^{2} / \hbar) \Delta t + \mathcal{O}(\Delta t^{3})$ and work to first order in $\Delta t$. Rewriting the position-space update equations as
$$
\begin{aligned}
\psi_{L}(n, t + \Delta t)
&= \cos \theta \, \psi_{L}(n + 1, t) - \mathrm{i} \sin \theta \, \psi_{R}(n + 1, t), \\
\psi_{R}(n, t + \Delta t)
&= - \mathrm{i} \sin \theta \, \psi_{L}(n - 1, t) + \cos \theta \, \psi_{R}(n - 1, t),
\end{aligned}
$$
we Taylor-expand all terms in $\Delta t$. Using
$$
\cos \theta = 1 + \mathcal{O}(\Delta t^{2}),
\quad
\sin \theta = \frac{m c^{2}}{\hbar} \Delta t + \mathcal{O}(\Delta t^{3}),
$$
and
$$
\psi_{L/R}(n \pm 1, t) = \psi_{L/R}(x \pm a, t)
= \psi_{L/R}(x, t) \pm a \partial_{x} \psi_{L/R}(x,t) + \mathcal{O}(a^{2}),
$$
with $x = n a$, we obtain to first order
$$
\begin{aligned}
\psi_{L}(x, t) + \Delta t \, \partial_{t} \psi_{L}(x,t)
&= \psi_{L}(x,t) + c \Delta t \, \partial_{x} \psi_{L}(x,t)
- \mathrm{i} \frac{m c^{2}}{\hbar} \Delta t \, \psi_{R}(x,t) + \mathcal{O}(\Delta t^{2}), \\
\psi_{R}(x, t) + \Delta t \, \partial_{t} \psi_{R}(x,t)
&= \psi_{R}(x,t) - c \Delta t \, \partial_{x} \psi_{R}(x,t)
- \mathrm{i} \frac{m c^{2}}{\hbar} \Delta t \, \psi_{L}(x,t) + \mathcal{O}(\Delta t^{2}).
\end{aligned}
$$
Subtracting $\psi_{L/R}(x,t)$ from both sides and dividing by $\Delta t$ yields
$$
\begin{aligned}
\partial_{t} \psi_{L} - c \partial_{x} \psi_{L} &= - \mathrm{i} \frac{m c^{2}}{\hbar} \psi_{R} + \mathcal{O}(\Delta t), \\
\partial_{t} \psi_{R} + c \partial_{x} \psi_{R} &= - \mathrm{i} \frac{m c^{2}}{\hbar} \psi_{L} + \mathcal{O}(\Delta t).
\end{aligned}
$$
Multiplying by $\mathrm{i} \hbar$ and collecting terms in spinor form $\Psi = (\psi_{L}, \psi_{R})^{\mathsf{T}}$, we find
$$
\mathrm{i} \hbar \, \partial_{t} \Psi
= \left( - \mathrm{i} \hbar c \, \sigma_{z} \partial_{x} + m c^{2} \sigma_{x} \right) \Psi
+ \mathcal{O}(\Delta t),
$$
which is the desired Dirac equation up to $\mathcal{O}(\Delta t)$ corrections. A more careful analysis using Fourier methods and norm estimates shows that the error remains $\mathcal{O}(\Delta t)$ in operator norm on any fixed time interval for wave packets with bounded momentum support, in line with rigorous continuum-limit results for quantum walks.

\subsection{Proof of Theorem 2 (chiral symmetry and winding number)}

We first verify chiral symmetry. Using $\Gamma = \sigma_{x}$ and the explicit form of $\widetilde{U}(k)$,
$$
\widetilde{U}(k)
= \begin{pmatrix}
\mathrm{e}^{-\mathrm{i} k} \cos \theta & - \mathrm{i} \mathrm{e}^{-\mathrm{i} k} \sin \theta \\
\mathrm{i} \mathrm{e}^{\mathrm{i} k} \sin \theta & \mathrm{e}^{\mathrm{i} k} \cos \theta
\end{pmatrix},
$$
we compute
$$
\Gamma \widetilde{U}(k) \Gamma
= \sigma_{x} \widetilde{U}(k) \sigma_{x}
= \begin{pmatrix}
\mathrm{e}^{\mathrm{i} k} \cos \theta & \mathrm{i} \mathrm{e}^{-\mathrm{i} k} \sin \theta \\
\mathrm{i} \mathrm{e}^{\mathrm{i} k} \sin \theta & \mathrm{e}^{-\mathrm{i} k} \cos \theta
\end{pmatrix}
= \widetilde{U}^{\dagger}(k),
$$
where the last equality follows from unitarity and complex conjugation. Thus the chiral symmetry condition holds.

To compute the winding number, we write $\widetilde{U}(k) = \mathrm{e}^{-\mathrm{i} E(k) \hat{\boldsymbol{n}}(k) \cdot \boldsymbol{\sigma}}$ with
$$
\cos(E \Delta t) = \cos \theta \cos k,
$$
and a Bloch vector $\hat{\boldsymbol{n}}(k)$ lying on the Bloch sphere. A convenient parametrization, used in Lam's analysis of the Hadamard quantum walk, is
$$
\hat{\boldsymbol{n}}(k) = \frac{1}{\sin(E \Delta t)}
\begin{pmatrix}
\sin \theta \sin k \\
\sin \theta \cos k \\
\cos \theta \sin k
\end{pmatrix},
$$
valid away from gap closing points where $\sin(E \Delta t) \neq 0$.

The chiral symmetry implies that the relevant winding is that of the projection of $\hat{\boldsymbol{n}}(k)$ onto the plane orthogonal to $\Gamma$, which here is the $(yz)$-plane. As $k$ runs from $-\pi$ to $\pi$, the vector $(\sin \theta \cos k, \cos \theta \sin k)$ winds once around the origin for $0 < \theta < \pi$, and is trivial for $\theta = 0$ or $\theta = \pi$. The resulting winding number can be written as
$$
\mathcal{W} = \frac{1}{2 \pi} \int_{-\pi}^{\pi} \mathrm{d} k \, \frac{\partial}{\partial k} \arg\left( \sin \theta \cos k + \mathrm{i} \cos \theta \sin k \right)
=
\begin{cases}
1, & 0 < \theta < \pi, \\
0, & \theta = 0 \text{ or } \pi.
\end{cases}
$$
This matches the known result that the one-coin quantum walk with this coin rotation lies in a non-trivial Floquet topological phase for generic coin angles, with gap closings at $\theta = 0$ and $\theta = \pi$.

Combining with Theorem 1, which relates $\theta$ to the Dirac mass $m$, we conclude that any non-zero Dirac mass corresponds to a topologically non-trivial QCA phase.

\subsection{Proof of Theorem 3 (self-referential scattering and inertia)}

The interpretation of $\widetilde{U}(k)$ as a two-channel scattering matrix is immediate from its structure: the diagonal entries represent transmission amplitudes $t(\theta) = \cos \theta \, \mathrm{e}^{\pm \mathrm{i} k}$ for left- and right-moving components, while the off-diagonal entries represent reflection amplitudes $r(\theta) = - \mathrm{i} \sin \theta \, \mathrm{e}^{\mp \mathrm{i} k}$ converting left-moving components into right-moving ones and vice versa. Thus at each timestep the walker undergoes a local self-scattering process coupling the two chiral channels.

The dispersion relation
$$
\cos(E \Delta t) = \cos \theta \cos k
$$
follows from the characteristic polynomial of $\widetilde{U}(k)$, and differentiating with respect to $k$ gives
$$
- \sin(E \Delta t) \, \Delta t \, \frac{\partial E}{\partial k}
= - \cos \theta \sin k.
$$
Thus the group velocity is
$$
v_{\mathrm{ext}}(k,\theta)
= \frac{1}{\hbar} \frac{\partial E}{\partial k} a
= c \, \frac{\cos \theta \sin k}{\sin(E \Delta t)}.
$$
In the long-wavelength regime $\lvert k \rvert \ll 1$ and small $\theta$, one can expand
$$
\cos(E \Delta t) \approx 1 - \frac{(E \Delta t)^{2}}{2},
\quad
\cos k \approx 1 - \frac{k^{2}}{2},
\quad
\cos \theta \approx 1 - \frac{\theta^{2}}{2},
$$
leading to
$$
E^{2} \approx c^{2} \left( \frac{\hbar k}{a} \right)^{2} + \left( \frac{\hbar \theta}{\Delta t} \right)^{2}
= c^{2} p^{2} + m^{2} c^{4},
$$
with $m c^{2} = \hbar \theta / \Delta t$, in agreement with Theorem 1 and with previous analyses of Dirac-type quantum walks.

The group velocity then reduces to the familiar Dirac expression
$$
v_{\mathrm{ext}}(p) = \frac{\partial E}{\partial p}
= \frac{c^{2} p}{\sqrt{p^{2} c^{2} + m^{2} c^{4}}},
$$
which approaches $c$ as $m \to 0$ or $\lvert p \rvert \to \infty$, and vanishes at $p = 0$ for non-zero $m$. The same parameter $\theta$ that controls the reflection amplitude $r(\theta) = - \mathrm{i} \sin \theta$ also controls the mass gap and the suppression of group velocity. It is therefore natural to interpret mass as arising from repeated self-referential scattering between chiral channels, dynamically retarding the net propagation of the excitation.

\subsection{Proof of Theorem 4 (Zitterbewegung and information speed)}

For the continuum Dirac Hamiltonian
$$
H = c p \sigma_{z} + m c^{2} \sigma_{x},
$$
with $X$ the position operator and $p = - \mathrm{i} \hbar \partial_{x}$, the velocity operator is
$$
\hat{v} = \frac{\mathrm{i}}{\hbar} [H, X] = c \sigma_{z}.
$$
Since $\sigma_{z}^{2} = \mathbb{I}$, we immediately have
$$
\hat{v}^{2} = c^{2} \mathbb{I},
$$
and for any normalized state $\Psi$,
$$
\langle \hat{v}^{2} \rangle = c^{2}.
$$
Define
$$
v_{\mathrm{ext}} = \langle \Psi, \hat{v} \Psi \rangle,
\quad
v_{\mathrm{int}} = \sqrt{ \langle \hat{v}^{2} \rangle - \langle \hat{v} \rangle^{2} }.
$$
Then
$$
v_{\mathrm{ext}}^{2} + v_{\mathrm{int}}^{2}
= \langle \hat{v} \rangle^{2} + \left( \langle \hat{v}^{2} \rangle - \langle \hat{v} \rangle^{2} \right)
= \langle \hat{v}^{2} \rangle
= c^{2},
$$
establishing the claimed identity. This is an algebraic consequence of the dichotomic spectrum of the velocity operator.

To connect with Zitterbewegung, we consider the Heisenberg equation of motion
$$
\frac{\mathrm{d} X(t)}{\mathrm{d} t} = \frac{\mathrm{i}}{\hbar} [H, X(t)] = c \sigma_{z}(t),
$$
and
$$
\frac{\mathrm{d} \sigma_{z}(t)}{\mathrm{d} t}
= \frac{\mathrm{i}}{\hbar} [H, \sigma_{z}(t)]
= \frac{2 m c^{2}}{\hbar} \sigma_{y}(t).
$$
Solving these coupled equations yields
$$
X(t) = X(0) + v_{\mathrm{ext}} t + \Xi(t),
$$
where $v_{\mathrm{ext}}$ is the group velocity associated with the expectation value in a given wave packet, and
$$
\Xi(t) = \frac{\hbar c}{2 \mathrm{i} H} \left( \mathrm{e}^{2 \mathrm{i} H t / \hbar} - 1 \right) \left( \hat{v}(0) - v_{\mathrm{ext}} \right)
$$
is an oscillatory term with frequency $2 E / \hbar$ in each energy sector. For states that are superpositions of positive and negative energy eigenstates, $\Xi(t)$ is non-zero and produces a rapid trembling motion of the expectation value of position, whose amplitude scales as $\hbar c / (2 E)$.

In the discrete QCA setting, Bisio et al. explicitly computed Zitterbewegung for the Dirac automaton and showed that the oscillation frequency and amplitude converge to the Dirac values in the small-mass, small-momentum limit. The decomposition of the velocity into a deterministic part $v_{\mathrm{ext}}$ and a fluctuating part with variance $v_{\mathrm{int}}^{2}$ thus provides an information-theoretic interpretation: $v_{\mathrm{ext}}$ measures the net "external" information flow across the lattice, while $v_{\mathrm{int}}$ quantifies the residual "internal" motion locked in the chiral degree of freedom.

---

\section{Model Apply}

In this section we discuss several settings where the "mass as topological impedance" picture yields concrete predictions or reinterpretations.

\subsection{Domain walls and topologically protected bound states}

Consider a spatially inhomogeneous QCA in which the coin angle $\theta$ depends on position $n$, interpolating between two asymptotic values $\theta_{-}$ and $\theta_{+}$ as $n \to -\infty$ and $n \to +\infty$. The corresponding asymptotic winding numbers $\mathcal{W}_{\pm}$ are given by Theorem 2. At an interface where $\mathcal{W}_{+} \neq \mathcal{W}_{-}$, Floquet bulk-boundary correspondence predicts the existence of topologically protected bound states at quasienergies $0$ or $\pi / \Delta t$, localized near the domain wall.

In the Dirac continuum limit, such an interface can be modelled by a position-dependent mass term $m(x)$ with $m(-\infty) < 0$ and $m(+\infty) > 0$. It is well known that this supports a localized zero-mode solution of the Dirac equation, a prototype of the Jackiw–Rebbi bound state. In the present interpretation, the bound state is a configuration where the local topological impedance changes sign, producing a stable "trapped" excitation that cannot delocalize without a topological phase transition.

Thus the QCA viewpoint connects topological edge states in quantum walks directly to spatial variations in the effective mass parameter, and interprets these states as self-referential scattering defects pinned by changes in the winding number.

\subsection{Massless channels at topological phase boundaries}

At parameter values where the bulk gap closes, such as $\theta = 0$ or $\theta = \pi$, the winding number changes and the system can host gapless channels that propagate at the maximal speed $c$. In the continuum Dirac picture, this corresponds to $m = 0$, where left- and right-moving components decouple and Zitterbewegung disappears.

From the perspective of information speed conservation, such channels are pure "external" information flow modes with $v_{\mathrm{ext}} = \pm c$ and $v_{\mathrm{int}} = 0$. In an inhomogeneous QCA, interfaces where $\theta$ crosses such critical values can thus act as "massless waveguides" embedded in a massive background, offering a natural setting for analogues of relativistic domain-wall fermions in quantum walk experiments.

\subsection{Effective field theory with topological impedance}

Given the Dirac continuum limit with mass $m$, one can write an effective low-energy field theory in which the mass term is not simply a local parameter, but is tied to a topological invariant of the underlying QCA. In particular:
\begin{enumerate}
    \item The sign and magnitude of $m$ determine the winding number and thus the presence or absence of protected edge states at inhomogeneities.
    \item Spatially varying $m(x)$ can encode a network of topological interfaces and defects, whose low-energy excitations correspond to localized modes guided by variations in the topological impedance.
    \item Time-dependent coin parameters can realize quench dynamics across topological phase boundaries, allowing dynamical topological order parameters to be measured through time-resolved quantum walk experiments.
\end{enumerate}
This viewpoint integrates QCA-based discretizations of Dirac fields with Floquet topological band theory, and provides a unified language in which mass, topology and information speed appear on equal footing.

---

\section{Engineering Proposals}

The mass-as-topological-impedance picture suggests a number of concrete experimental tests in platforms that implement DTQWs or QCAs.

\subsection{Photonic quantum walks with tunable coin angle}

Integrated photonic setups have successfully realized discrete-time quantum walks with tunable coin operators, observing topological bound states and measuring winding numbers. In such systems:
\begin{enumerate}
    \item The coin angle $\theta$ can be mapped directly to an effective mass parameter via $m c^{2} = \hbar \theta / \Delta t$.
    \item By preparing wave packets with narrow momentum spread and varying $\theta$, one can measure the dependence of the group velocity $v_{\mathrm{ext}}$ on the inferred mass, checking the Dirac-like relation
       $$
       v_{\mathrm{ext}}(p,\theta) \approx \frac{c^{2} p}{\sqrt{p^{2} c^{2} + m^{2} c^{4}}}.
       $$
    \item By encoding the coin degree of freedom in polarization, and resolving both spatial and polarization dynamics, one can reconstruct the effective velocity operator statistics and directly test the information speed decomposition $v_{\mathrm{ext}}^{2} + v_{\mathrm{int}}^{2} = c^{2}$ for different initial states.
\end{enumerate}

\subsection{Cold-atom quantum walks and Zitterbewegung}

Quantum walks implemented with cold atoms in optical lattices offer long coherence times and flexible control over coin parameters. Previous work has proposed and realized topological quantum walks with cold atoms. In such experiments:
\begin{enumerate}
    \item One can prepare wave packets that are coherent superpositions of positive and negative quasienergy bands and measure the time-dependent expectation value $\langle X(t) \rangle$.
    \item The observed oscillatory component can be fitted to extract the Zitterbewegung frequency $\omega_{\mathrm{ZB}}$ and amplitude as functions of $\theta$ and initial momentum.
    \item Comparing these with the Dirac predictions $\omega_{\mathrm{ZB}} = 2 E / \hbar$ and $\mathrm{amp} \sim \hbar c / (2 E)$ provides a direct test of the self-referential scattering picture.
\end{enumerate}

\subsection{QCA-based digital quantum simulations}

On digital quantum processors, one can implement finite-size QCAs by composing layers of local two-qubit unitaries in a brickwork circuit. The Dirac-QCA considered here corresponds to a depth-two circuit consisting of a coin layer and a conditional shift, which can be decomposed into nearest-neighbour gates.
Within this framework:
\begin{enumerate}
    \item Varying the coin rotation angle realizes different effective masses.
    \item Using tomography or randomized measurement protocols, one can reconstruct the Floquet unitary in small systems and numerically compute its winding number.
    \item Studying the response to local defects or inhomogeneous coin angles can reveal topologically protected modes and test the robustness of the topological impedance against noise and decoherence.
\end{enumerate}

---

\section{Discussion (Risks, Boundaries, Past Work)}

\subsection{Relation to standard Higgs mass generation}

The mechanism analysed here is purely kinematical and topological. It attributes mass to the presence of a chiral-mixing term in the QCA update rule, whose coefficient is constrained by a topological invariant. It does not involve spontaneous symmetry breaking or a scalar order parameter, and in particular does not address the origin of the Higgs vacuum expectation value or the numerical values of Standard-Model Yukawa couplings.

However, this does not imply incompatibility with the Higgs mechanism. One can view QCA-based mass as a microscopic origin for the kinematic mass term in an effective Dirac equation, while the Higgs mechanism provides a low-energy field-theoretic description of how such a mass term might emerge dynamically from a scalar sector. Exploring this relationship in higher-dimensional Dirac-QCAs that incorporate gauge symmetries and non-Abelian internal degrees of freedom is an important direction for future work.

\subsection{Dimensionality and chirality}

The present analysis is strictly one-dimensional. In higher dimensions, chiral structure and topological invariants become richer: for instance, in $(2+1)$ dimensions one encounters Chern numbers, and in $(3+1)$ dimensions Weyl nodes and associated monopole charges in momentum space. Dirac-type QCAs have been constructed in higher dimensions, and their dispersion relations and continuum limits have been analysed, but a systematic topological classification in terms of QCA indices and Floquet invariants is still being developed.

Extending the mass-as-topological-impedance picture to $(3+1)$-dimensional Dirac QCAs, where chirality, parity and anomaly cancellation play central roles, is non-trivial and may require more elaborate internal coin spaces and update rules.

\subsection{Index theory of QCAs and stability}

One-dimensional QCAs admit a rigorous index theory characterizing their net information flow per timestep, in terms of an integer that remains invariant under homotopies preserving locality and translation invariance. This index is closely related to the net "current" of information and constrains the possible bulk dynamics. The winding number analysed here is a Floquet-type invariant of the momentum-space unitary, defined in the presence of chiral symmetry.

Understanding the precise relationship between these two invariants is important: they are both stable under local perturbations, but capture different aspects of the QCA. In the present model, the global index vanishes (no net drift), while the winding number is non-zero for $\theta \neq 0$. This suggests a picture in which mass corresponds to a non-trivial internal "circulation" of information with zero net transport, consistent with the topological impedance interpretation.

\subsection{Noise, decoherence and non-unitary effects}

Real experiments inevitably involve losses, decoherence and control imperfections. Recent work has extended topological classifications to non-unitary quantum walks and studied the robustness of topological phases under such perturbations. The essential message is that topological features can persist as long as certain spectral and symmetry conditions are approximately maintained.

In the present context, noise can degrade the exact information speed identity $v_{\mathrm{ext}}^{2} + v_{\mathrm{int}}^{2} = c^{2}$, which relies on unitarity and a dichotomic velocity spectrum. Nonetheless, approximate versions of this relation may still hold on relevant timescales, and deviations themselves can serve as probes of decoherence.

---

\section{Conclusion}

We have analysed a simple yet rich Dirac-type quantum cellular automaton in one spatial dimension and shown that its effective mass parameter admits a natural interpretation as a topological impedance arising from self-referential scattering between chiral channels. The key elements of this picture are:
\begin{enumerate}
    \item The QCA evolution operator, defined as a composition of a coin rotation and conditional shift, reproduces the $(1+1)$-dimensional Dirac equation in the continuum limit, with mass proportional to the coin angle.
    \item The momentum-space Floquet unitary possesses chiral symmetry and carries a non-trivial winding number for any non-zero coin angle, placing the automaton in a topological Floquet phase. The massless point coincides with a topological phase transition where the quasienergy gap closes.
    \item The same coin parameter that sets the mass gap also determines a local reflection amplitude between left- and right-moving modes. The resulting repeated self-referential scattering suppresses the group velocity and provides a microscopic mechanism for inertia.
    \item In the Dirac limit the velocity operator has eigenvalues $\pm c$, implying that any excitation satisfies $v_{\mathrm{ext}}^{2} + v_{\mathrm{int}}^{2} = c^{2}$, where $v_{\mathrm{ext}}$ is the average group velocity and $v_{\mathrm{int}}$ quantifies internal fluctuations associated with Zitterbewegung. Massive excitations thus divert part of the fixed information speed budget into internal motion.
    \item Interfaces where the effective mass changes sign support topologically protected bound states, naturally interpreted as defects in the pattern of topological impedance. Experimental platforms based on photonic and cold-atom quantum walks can test these predictions and directly probe the interplay between mass, topology and information flow.
\end{enumerate}
Taken together, these results suggest that within a QCA ontology, mass need not be an arbitrary parameter inserted into the continuum field theory. Instead, it can emerge as a topologically protected property of discrete information dynamics, constraining how excitations allocate their finite information speed budget between external propagation and internal self-referential motion.

---

\section{Acknowledgements, Code Availability}

The author acknowledges the existing body of work on quantum walks, quantum cellular automata and topological phases that underpins this study, in particular the contributions of Strauch, Childs, Kitagawa, Asbóth, Bisio, D'Ariano, Tosini, Farrelly and many others.

No numerical simulations beyond standard analytical calculations were required for the derivations presented here. Simple quantum-walk simulators sufficient to reproduce the dispersion relations, Zitterbewegung and domain-wall bound states discussed in this paper can be implemented straightforwardly in standard scientific computing environments; no dedicated code repository is provided.

---

\appendix

\section{Appendix A: Detailed Derivation of the Dirac Continuum Limit}

\subsection{A.1 Scaling and smoothness assumptions}

We adopt the scaling
$$
a = c \, \Delta t,
\quad
\theta = \frac{m c^{2}}{\hbar} \Delta t,
$$
and assume that the lattice wave function $\Psi(n,t)$ at $t = 0$ is obtained by sampling a smooth continuum spinor $\Psi(x,0)$ at positions $x = n a$. Explicitly,
$$
\Psi(n,0) = \Psi(x = n a,0),
$$
with $\Psi$ twice continuously differentiable and decaying sufficiently fast at infinity.

The goal is to construct a continuum spinor $\Psi(x,t)$ satisfying the Dirac equation such that
$$
\Psi(n,t) = \Psi(x = n a,t) + \mathcal{O}(\Delta t^{2})
$$
for $t$ in a bounded interval. The error estimate can be made precise in Sobolev norms, following methods used in rigorous continuum-limit analyses of quantum walks.

\subsection{A.2 Expansion of the discrete update}

We write the discrete update equations as
$$
\Psi(n,t + \Delta t) = \mathcal{U} \Psi(\cdot,t)(n),
$$
where $\mathcal{U}$ acts on lattice spinors according to
$$
\mathcal{U} \Psi(n)
= \begin{pmatrix}
\cos \theta \, \psi_{L}(n + 1) - \mathrm{i} \sin \theta \, \psi_{R}(n + 1) \\
- \mathrm{i} \sin \theta \, \psi_{L}(n - 1) + \cos \theta \, \psi_{R}(n - 1)
\end{pmatrix}.
$$
We now interpret $\Psi(n,t)$ as sampling of a continuum field $\Psi(x,t)$, and Taylor-expand around $x = n a$. Using
$$
\Psi(x \pm a,t) = \Psi(x,t) \pm a \partial_{x} \Psi(x,t) + \frac{a^{2}}{2} \partial_{x}^{2} \Psi(x,t) + \mathcal{O}(a^{3}),
$$
and
$$
\cos \theta = 1 - \frac{\theta^{2}}{2} + \mathcal{O}(\theta^{4})
= 1 - \mathcal{O}(\Delta t^{2}),
$$
$$
\sin \theta = \theta + \mathcal{O}(\theta^{3})
= \frac{m c^{2}}{\hbar} \Delta t + \mathcal{O}(\Delta t^{3}),
$$
we obtain
$$
\begin{aligned}
\psi_{L}(x, t + \Delta t)
&= \psi_{L}(x,t) + a \partial_{x} \psi_{L}(x,t)
- \mathrm{i} \frac{m c^{2}}{\hbar} \Delta t \, \psi_{R}(x,t)
+ \mathcal{O}(\Delta t^{2}), \\
\psi_{R}(x, t + \Delta t)
&= \psi_{R}(x,t) - a \partial_{x} \psi_{R}(x,t)
- \mathrm{i} \frac{m c^{2}}{\hbar} \Delta t \, \psi_{L}(x,t)
+ \mathcal{O}(\Delta t^{2}).
\end{aligned}
$$
Here we used $a = c \Delta t$ and neglected terms of order $\Delta t^{2}$ or higher.

Rewriting this as a first-order time discretization of a continuum equation,
$$
\Psi(x,t + \Delta t) = \Psi(x,t) + \Delta t \, \partial_{t} \Psi(x,t) + \mathcal{O}(\Delta t^{2}),
$$
we identify
$$
\partial_{t} \Psi
= - c \sigma_{z} \partial_{x} \Psi - \mathrm{i} \frac{m c^{2}}{\hbar} \sigma_{x} \Psi
+ \mathcal{O}(\Delta t),
$$
and hence
$$
\mathrm{i} \hbar \, \partial_{t} \Psi
= \left( - \mathrm{i} \hbar c \sigma_{z} \partial_{x} + m c^{2} \sigma_{x} \right) \Psi
+ \mathcal{O}(\Delta t).
$$
Standard stability and consistency arguments for one-step schemes then show that the discrete evolution converges to the Dirac evolution with error $\mathcal{O}(\Delta t)$ over finite times.

\section{Appendix B: Zitterbewegung in Dirac and QCA Dynamics}

Here we give the standard derivation of Zitterbewegung for the Dirac Hamiltonian and outline how it appears in the QCA.

\subsection{B.1 Zitterbewegung in the Dirac theory}

Consider the free Dirac Hamiltonian in $(1+1)$ dimensions,
$$
H = c p \sigma_{z} + m c^{2} \sigma_{x},
$$
acting on spinors $\Psi(x)$. In the Heisenberg picture,
$$
\frac{\mathrm{d} X(t)}{\mathrm{d} t} = \frac{\mathrm{i}}{\hbar} [H, X(t)],
\quad
\frac{\mathrm{d} \sigma_{z}(t)}{\mathrm{d} t} = \frac{\mathrm{i}}{\hbar} [H, \sigma_{z}(t)].
$$
Using $[p, X] = - \mathrm{i} \hbar$, one finds
$$
\frac{\mathrm{d} X(t)}{\mathrm{d} t} = c \sigma_{z}(t)
= \hat{v}(t),
$$
and
$$
\frac{\mathrm{d} \sigma_{z}(t)}{\mathrm{d} t}
= \frac{2 m c^{2}}{\hbar} \sigma_{y}(t),
\quad
\frac{\mathrm{d} \sigma_{y}(t)}{\mathrm{d} t}
= - \frac{2}{\hbar} \left( c p \sigma_{x}(t) + m c^{2} \sigma_{z}(t) \right).
$$
Solving these coupled equations yields
$$
\hat{v}(t) = c^{2} p H^{-1} + \mathrm{e}^{2 \mathrm{i} H t / \hbar} \left( \hat{v}(0) - c^{2} p H^{-1} \right),
$$
and integrating over time,
$$
X(t) = X(0) + c^{2} p H^{-1} t + \frac{\mathrm{i} \hbar c}{2 H} \left( \mathrm{e}^{2 \mathrm{i} H t / \hbar} - 1 \right) \left( \hat{v}(0) - c^{2} p H^{-1} \right).
$$
The first term represents uniform motion with group velocity $v_{\mathrm{ext}} = c^{2} p / E$, while the second is an oscillatory term with frequency $\omega_{\mathrm{ZB}} = 2 E / \hbar$ and amplitude of order $\hbar c / (2 E)$. For wave packets consisting purely of positive energy eigenstates, $\hat{v}(0)$ coincides with $c^{2} p H^{-1}$, and the oscillatory term vanishes; Zitterbewegung arises only when both positive and negative energy components are present.

\subsection{B.2 Zitterbewegung in the Dirac-QCA}

For the QCA, the Heisenberg equations must be formulated with respect to the effective Hamiltonian $H_{\mathrm{eff}}(k)$ or, more directly, via the discrete-time Heisenberg evolution
$$
X_{m+1} = U^{\dagger} X_{m} U,
$$
with $m$ labelling timesteps. Working in momentum space, one finds that for wave packets peaked around small momenta and small masses, the discrete evolution of $\langle X_{m} \rangle$ reproduces the Dirac behavior to good accuracy, including an oscillatory contribution with frequency close to $2 E / \hbar$. Detailed calculations and numerical simulations for the Dirac automaton have been carried out in previous work, confirming that Zitterbewegung is an intrinsic feature of the QCA dynamics.

This supports the interpretation that Zitterbewegung, both in the continuum and in the QCA, is the manifestation of the internal component $v_{\mathrm{int}}$ of the fixed information speed budget, while the group velocity $v_{\mathrm{ext}}$ accounts for the net transport.

\section{Appendix C: Computation of the Winding Number}

We briefly detail the computation of the winding number for the one-coin quantum walk considered here.

\subsection{C.1 Chiral decomposition}

With chiral symmetry $\Gamma = \sigma_{x}$ and Floquet unitary $\widetilde{U}(k)$, we can switch to a basis in which $\Gamma$ is diagonal:
$$
\Gamma = \begin{pmatrix} 1 & 0 \\ 0 & -1 \end{pmatrix},
$$
and $\widetilde{U}(k)$ can be written in block form
$$
\widetilde{U}(k) = \begin{pmatrix}
A(k) & B(k) \\
C(k) & D(k)
\end{pmatrix}.
$$
Chiral symmetry implies
$$
\Gamma \widetilde{U}(k) \Gamma = \widetilde{U}^{\dagger}(k),
$$
which leads to constraints $A = D^{\dagger}$, $B = -B^{\dagger}$, $C = -C^{\dagger}$. In particular, the off-diagonal block $B(k)$ encodes the non-trivial topology; its phase as a function of $k$ winds around the origin in the complex plane.

For our two-component model, one can work directly with the Bloch vector $\hat{\boldsymbol{n}}(k)$ and its projection onto the plane orthogonal to $\Gamma$, as described in the main text.

\subsection{C.2 Explicit phase winding}

The crucial object is
$$
z(k) = \sin \theta \cos k + \mathrm{i} \cos \theta \sin k
= \sqrt{ \sin^{2} \theta \cos^{2} k + \cos^{2} \theta \sin^{2} k } \,
\mathrm{e}^{\mathrm{i} \varphi(k)},
$$
whose argument
$$
\varphi(k) = \arg \left( \sin \theta \cos k + \mathrm{i} \cos \theta \sin k \right)
$$
is a continuous function of $k$ for $0 < \theta < \pi$, with
$$
\varphi(-\pi) = - \arctan\left( \cot \theta \right),
\quad
\varphi(\pi) = \varphi(-\pi) + 2 \pi.
$$
Thus as $k$ runs from $-\pi$ to $\pi$, the phase $\varphi(k)$ increases by $2 \pi$, and the winding number
$$
\mathcal{W} = \frac{1}{2 \pi} \int_{-\pi}^{\pi} \frac{\partial \varphi(k)}{\partial k} \, \mathrm{d} k
$$
is equal to $1$.

At $\theta = 0$ or $\theta = \pi$, the quantity $z(k)$ collapses to the real axis, the gap closes at quasienergy $0$ or $\pi / \Delta t$, and the winding becomes ill-defined; in those cases the system is topologically trivial with $\mathcal{W} = 0$.

This computation aligns with general classifications of one-dimensional chiral-symmetric quantum walks and with explicit evaluations of winding numbers in related models.

\begin{thebibliography}{99}

\bibitem{Feynman1982} R. P. Feynman, "Simulating physics with computers," International Journal of Theoretical Physics 21, 467–488 (1982).

\bibitem{Bisio2015} A. Bisio, G. M. D'Ariano, and A. Tosini, "Quantum field as a quantum cellular automaton," Annals of Physics 354, 244–264 (2015).

\bibitem{Bisio2013} A. Bisio, G. M. D'Ariano, and A. Tosini, "Dirac quantum cellular automaton in one dimension," Physical Review A 88, 032301 (2013).

\bibitem{Strauch2006a} F. W. Strauch, "Relativistic quantum walks," Physical Review A 73, 054302 (2006).

\bibitem{Strauch2006b} F. W. Strauch, "Connecting the discrete- and continuous-time quantum walks," Physical Review A 74, 030301 (2006).

\bibitem{Childs2010} A. M. Childs, "On the relationship between continuous- and discrete-time quantum walk," Communications in Mathematical Physics 294, 581–603 (2010).

\bibitem{Manighalam2020} M. Manighalam, "General methods and properties for evaluation of continuum limits of quantum walks," Quantum Information Processing 19, 288 (2020).

\bibitem{Asboth2016} J. K. Asbóth, L. Oroszlány, and A. Pályi, \textit{A Short Course on Topological Insulators: Band-Structure Topology and Edge States in One and Two Dimensions}, Springer, 2016.

\bibitem{Kitagawa2010} T. Kitagawa, M. S. Rudner, E. Berg, and E. Demler, "Exploring topological phases with quantum walks," Physical Review A 82, 033429 (2010).

\bibitem{Kitagawa2012a} T. Kitagawa, "Topological phenomena in quantum walks: Elementary introduction to the physics of topological phases," Quantum Information Processing 11, 1107–1148 (2012).

\bibitem{Asboth2013} J. K. Asbóth and H. Obuse, "Bulk–boundary correspondence for chiral symmetric quantum walks," Physical Review B 88, 121406 (2013).

\bibitem{Lam2015} H. T. Lam, L. C. Kwek, and J. K. Asbóth, "Quantum walks and hidden topological phases," (2015), arXiv:1508.07528.

\bibitem{Farrelly2020} T. Farrelly, "A review of quantum cellular automata," Quantum 4, 368 (2020).

\bibitem{Bisio2013_preprint} A. Bisio, G. M. D'Ariano, and A. Tosini, "The Dirac quantum cellular automaton in one dimension: Zitterbewegung and scattering from potential," preprint (2013).

\bibitem{Kitagawa2012b} T. Kitagawa et al., "Observation of topologically protected bound states in photonic quantum walks," Nature Communications 3, 882 (2012).

\bibitem{Mugel2016} S. Mugel et al., "Topological bound states of a quantum walk with cold atoms," Physical Review A 94, 023631 (2016).

\bibitem{Meng2020} Y. Meng et al., "Topological quantum walks in cavity-based quantum networks," European Physical Journal Plus 135, 355 (2020).

\bibitem{Sun2020} C. Sun et al., "Measuring a dynamical topological order parameter in quantum walks," Light: Science \& Applications 9, 249 (2020).

\bibitem{Jia2021} Y. Jia, C. Lv, and Z. Wang, "High winding number of topological phase in periodic quantum walks," Physics Letters A 404, 127408 (2021).

\bibitem{Mittal2021} V. Mittal et al., "Persistence of topological phases in non-Hermitian quantum walk," Scientific Reports 11, 9095 (2021).

\bibitem{Wang2022} Q. Wang, X. Luo, and J. Gong, "Topological invariants of nonunitary quantum walks with gain and loss," Physica E 141, 115222 (2022).

\bibitem{Guilar2014} J. M. Guilar, B. Tarasinski, and C. W. J. Beenakker, "Chiral symmetry and bulk-boundary correspondence in periodically driven one-dimensional systems," Physical Review B 90, 125143 (2014).

\bibitem{Ma2025} H. Ma, "Universal conservation of information celerity," preprint (2025).

\end{thebibliography}

\end{document}

