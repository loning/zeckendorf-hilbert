\documentclass[11pt,a4paper]{article}
\usepackage[utf8]{inputenc}
\usepackage[T1]{fontenc}
\usepackage{amsmath,amssymb,amsthm}
\usepackage{mathtools}
\usepackage{geometry}
\usepackage{hyperref}
\usepackage{cite}
\usepackage{braket}

\geometry{margin=1in}

\newtheorem{theorem}{Theorem}[section]
\newtheorem{lemma}[theorem]{Lemma}
\newtheorem{proposition}[theorem]{Proposition}
\newtheorem{corollary}[theorem]{Corollary}
\newtheorem{definition}[theorem]{Definition}
\theoremstyle{remark}
\newtheorem{remark}[theorem]{Remark}

\title{Time as Density of States: Unified Time Identity $\kappa = \rho$ and Microscopic Scattering Mechanism of Gravitational Redshift}

\author{Haobo Ma$^1$ \and Wenlin Zhang$^2$\\
\small $^1$Independent Researcher\\
\small $^2$National University of Singapore}

\date{}

\begin{document}

\maketitle

\begin{abstract}
In conventional physics, time plays incompatible roles in quantum theory and general relativity: in quantum mechanics it is an external evolution parameter generated by a Hamiltonian, whereas in general relativity proper time is a path-dependent functional of the spacetime metric. Within a quantum cellular automaton (QCA) and optical-path conservation framework, this work proposes a microscopic definition of time in terms of the density of quantum states. Building on Eisenbud--Wigner--Smith (EWS) time-delay theory and Krein--Friedel--Lloyd (KFL) trace formulas, we show that the trace of the Wigner--Smith time-delay operator $Q(E) = -\mathrm{i}\hbar S(E)^\dagger \partial_E S(E)$ is universally related to the relative density of scattering states via the spectral shift function. This leads to a unified time identity
\[
\kappa(E)
= \frac{1}{2\pi\hbar} \operatorname{tr} Q(E)
= \Delta \rho(E),
\]
where $\Delta\rho(E)$ is the change of density of states (DOS) induced by an interaction with respect to a reference background. Physically, $\kappa(E)$ quantifies the rate at which a system traverses quantum states in Hilbert space per unit energy. We interpret $\kappa(E)$ as an intrinsic ``time density'', and argue that macroscopic clock rates are governed by the local DOS of the underlying microscopic degrees of freedom.

On this basis we develop a microscopic picture of gravitational redshift: deep in a gravitational potential well, matter and radiation experience an enhanced local DOS due to modified phase space volume and effective refractive index of the vacuum. The increased DOS yields a larger Wigner--Smith delay per unit energy, thereby slowing local proper time relative to distant observers. We show how, in the weak-field limit, the DOS-based time density reproduces the standard gravitational redshift relation $\Delta\nu/\nu \simeq \Delta\Phi/c^2$ when normalized to asymptotically flat regions, and argue that the metric component $g_{00}$ can be viewed as an emergent functional of the microscopic DOS field. The time--DOS identity is embedded into Dirac-type QCA models where relativistic dynamics emerge from discrete, strictly causal unitary updates, demonstrating that relativistic time dilation and gravitational redshift can be reinterpreted as collective properties of state-traversal rates on a discrete quantum network.

Finally, we connect the proposed time density with thermodynamic time in KMS equilibrium states and the thermal time hypothesis, showing that the DOS-weighted EWS time density defines a natural arrow of time compatible with modular flow in algebraic quantum field theory. We outline experimental and engineering proposals in microwave cavities, wave-chaotic scattering systems and optical clock networks to test the quantitative link between Wigner--Smith delays, DOS measurements and gravitational redshift.
\end{abstract}

\noindent\textbf{Keywords:} time delay; density of states; Eisenbud--Wigner--Smith operator; spectral shift function; Krein--Friedel--Lloyd formula; gravitational redshift; quantum cellular automaton; optical metric; KMS state; thermal time

\section{Introduction \& Historical Context}

The problem of time pervades the development of physical theory. Special and general relativity take the four-dimensional spacetime manifold $(M,g_{\mu\nu})$ as the fundamental object, describing proper time $\tau$ as a line element integral along worldlines. Quantum theory, by contrast, is built on Hilbert space and the Hamiltonian $H$, where time $t$ is an external continuous parameter and evolution is generated by the unitary operator $U(t)=\exp(-\mathrm{i}Ht/\hbar)$. While the two frameworks are compatible in the semiclassical limit, a unified microscopic ontology of time is lacking.

Pauli's theorem indicates that in systems with an energy spectrum bounded from below, there exists no self-adjoint time operator satisfying canonical commutation relations with $H$, seemingly ruling out attempts to ``operatorize time'' within traditional quantum mechanics. On the other hand, ``time delay'' has long appeared as an observable in scattering theory. Eisenbud, Wigner, and Smith defined the Wigner--Smith group delay matrix from the energy derivative of the scattering matrix $S(E)$ as
\[
Q(E)=-\mathrm{i}\hbar S(E)^\dagger \partial_E S(E),
\]
whose diagonal elements $Q_{\alpha\alpha}(E)$ give the time delay in each channel, while the trace $\operatorname{tr}Q(E)$ characterizes the total temporal response of the system to a family of scattering states.

In parallel, Krein, Friedel, Lloyd and others established the celebrated Krein--Friedel--Lloyd formula in scattering spectral theory, relating the energy derivative of scattering phase shifts to the change in DOS $\Delta\rho(E)$. In finite-box normalization, this formula can be written simply as
\[
\Delta \rho(E)=\frac{1}{2\pi\mathrm{i}} \operatorname{tr}\bigl(S(E)^\dagger \partial_E S(E)\bigr)
=\frac{1}{\pi} \partial_E \delta_{\mathrm{tot}}(E),
\]
where $\delta_{\mathrm{tot}}(E)$ is the total scattering phase shift.

Modern spectral shift function theory further shows that the derivative of the spectral shift function $\xi(E)$ is precisely the DOS difference induced by the interaction, and $\xi(E)$ can itself be characterized by the determinant phase of $S(E)$.

These results hint at a profound fact: the time delay and the density of states in scattering systems are not two independent quantities, but rather different projections of the same spectral structure. Building on this foundation, the present work proposes a unified time identity that strictly equates the time flow rate $\kappa(E)$ with the DOS $\rho(E)$.

On the other hand, general relativity interprets gravitational redshift as the spatial variation of the $g_{00}$ component in static spacetime metrics. In the weak-field limit, the static gravitational potential $\Phi(\mathbf{x})$ and the metric satisfy
\[
g_{00}(\mathbf{x})\simeq -\bigl(1+2\Phi(\mathbf{x})/c^2\bigr),
\]
and the proper time of stationary observers satisfies $\mathrm{d}\tau = \sqrt{-g_{00}}\mathrm{d}t$, so that for two positions $\mathbf{x}_1,\mathbf{x}_2$ at different potential values we have
\[
\frac{\Delta\nu}{\nu}\simeq \frac{\Phi(\mathbf{x}_2)-\Phi(\mathbf{x}_1)}{c^2},
\]
a relation that has been precisely verified in the Pound--Rebka $\gamma$-ray experiment, the Hafele--Keating around-the-world atomic clock flight, and recent optical lattice clock height comparison experiments.

However, the geometric description of gravitational redshift still leaves the origin of ``time flow'' in the continuous field of the metric tensor, lacking a direct connection to the structure of quantum states. Quantum cellular automaton and discrete quantum walk studies have shown that under appropriate symmetry conditions, the Dirac and Weyl equations can emerge as the continuum limit of local unitary discrete dynamics, thus providing rigorous models for the discrete picture of ``the universe as quantum computation''.

In such a discrete ontology, the ``time step'' is a discrete parameter of the local update rule, rather than an external continuous variable. The natural question is: how can we reconstruct continuous proper time within this discrete framework, and relate it to gravitational redshift?

The basic viewpoint of this paper is: \textbf{time is an alias for density of states}. More precisely, through the EWS time-delay operator and the KFL trace formula, we prove that under quite general scattering settings, the following unified time identity holds:
\[
\boxed{
\kappa(E)
=\frac{1}{2\pi\hbar}\operatorname{tr}Q(E)
=\Delta\rho(E)
}
\]
where $\kappa(E)$ is interpreted as the ``time flow density'' per unit energy interval, i.e., the effective time delay experienced by the system per unit energy in the neighborhood of energy $E$. We further show that, under appropriate coarse-graining and normalization choices, the spatial variation of this quantity can restate gravitational redshift in general relativity.

\section{Model \& Assumptions}

\subsection{Scattering System and Density of States}

Let $H_0$ be the free Hamiltonian and $H = H_0+V$ the interacting Hamiltonian containing a local potential $V$. We make the following assumptions:

\begin{enumerate}
\item $H_0,H$ are self-adjoint operators on the same Hilbert space $\mathcal{H}$, and $V$ is a sufficiently rapidly decaying bounded or relatively $H_0$-bounded perturbation, so that the wave operators
\[
\Omega^\pm = \operatorname*{s-lim}_{t\to\pm\infty} \mathrm{e}^{\mathrm{i}Ht/\hbar}\mathrm{e}^{-\mathrm{i}H_0t/\hbar}
\]
exist and are complete.

\item The scattering operator $S = (\Omega^+)^\dagger \Omega^-$ decomposes in the energy representation into a family of unitary matrices $S(E)$, acting on the channel space $\mathcal{H}_E$ on each energy shell.

\item The spectral measures of both free and interacting systems are absolutely continuous plus a finite number of bound states, so that the well-defined state-counting functions exist:
\[
N_0(E)=\#\{\lambda_n(H_0)\le E\},\quad
N(E)=\#\{\lambda_n(H)\le E\},
\]
whose derivatives give the densities of states $\rho_0(E),\rho(E)$. The change in DOS induced by the interaction is
\[
\Delta\rho(E)=\rho(E)-\rho_0(E).
\]
\end{enumerate}

The spectral shift function $\xi(E)$ is defined by the Krein trace formula, via
\[
\operatorname{tr}\bigl(f(H)-f(H_0)\bigr)
=\int_{-\infty}^{+\infty} f'(E)\,\xi(E)\,\mathrm{d}E
\]
for sufficiently smooth functions $f$. An approximation to the indicator function yields $\xi(E)=N(E)-N_0(E)$, whose derivative satisfies
\[
\xi'(E)=\Delta\rho(E).
\]

On the other hand, the determinant of the scattering matrix and the spectral shift function are related by
\[
\det S(E)
=\exp\bigl(-2\pi\mathrm{i}\xi(E)\bigr),
\]
so that
\[
\frac{1}{2\pi\mathrm{i}}\partial_E \ln\det S(E)
=\xi'(E)
=\Delta\rho(E).
\]

\subsection{Wigner--Smith Time-Delay Operator}

In the energy representation, the Wigner--Smith group delay matrix is defined as
\[
Q(E)=-\mathrm{i}\hbar S(E)^\dagger \partial_E S(E),
\]
whose diagonal element $Q_{\alpha\alpha}(E)$ gives the time delay of channel $\alpha$, and the trace
\[
\operatorname{tr}Q(E)
=-\mathrm{i}\hbar \operatorname{tr}\bigl(S(E)^\dagger \partial_E S(E)\bigr)
\]
characterizes the total group delay of all channels. The self-adjointness and observability of $Q(E)$ can be verified by differentiating the unitarity of the scattering matrix $S(E)^\dagger S(E)=\mathbb{I}$.

Taking the trace and comparing with the spectral shift relation from the previous section, we immediately obtain
\[
\Delta\rho(E)
=\frac{1}{2\pi\mathrm{i}}\partial_E \ln\det S(E)
=\frac{1}{2\pi\mathrm{i}}\frac{1}{\det S(E)}\partial_E \det S(E)
=\frac{1}{2\pi\mathrm{i}} \operatorname{tr}\bigl(S^\dagger\partial_E S\bigr)
=\frac{1}{2\pi\hbar}\operatorname{tr} Q(E).
\]

This is the spectral-theoretic foundation of the unified time identity in the present work.

\subsection{QCA Continuum Limit and Scattering}

Quantum cellular automata are a class of strictly local, discrete-time, unitary evolutions $U$ defined on a lattice $\Lambda$, which in the single-particle subspace are equivalent to discrete-time quantum walks. A considerable body of work has shown that, in the limit of sufficiently small wavevector and mass, appropriately constructed QCA evolutions can approximate Dirac equations in various dimensions and their generalizations in curved spacetime.

In the QCA framework, time is essentially the discrete step number $n\in\mathbb{Z}$, and energy is given by the quasi-energy spectrum $\mathrm{e}^{-\mathrm{i}\varepsilon(k)}$ of the single-step evolution operator $U$. Local defects or external potentials can be implemented by modifying the local update operator in a finite region, corresponding to Floquet scattering scenarios where the scattering matrix $S(\varepsilon)$ is closely related to continuous-time scattering. Through the continuum-limit mapping $\varepsilon \mapsto E$, the Floquet--EWS time delay in QCA can be unified with $Q(E)$ in continuous scattering theory.

We make the following model assumptions:

\begin{enumerate}
\item The universe can be described microscopically by a class of Dirac-type QCA models, whose single-particle subspace approximates the standard Dirac equation in the long-wavelength limit.

\item Macroscopic gravitational fields correspond to slow spatial variations of the QCA background cell update rules, i.e., gradual changes in local ``propagation velocity'' and phase response. This can be likened to the construction of quantum walks in curved spacetime.

\item In this framework, all macroscopic time observations can ultimately be reduced to phase differences and group delay measurements in some kind of scattering or interference experiments, and therefore can be uniformly described by the EWS--KFL structure.
\end{enumerate}

\section{Main Results (Theorems and Alignments)}

This section presents the unified time identity and its main consequences for gravitational redshift, the thermodynamic arrow of time, and the QCA continuum limit.

\begin{definition}[Time Density]
For a scattering system satisfying the above assumptions, define the time density at energy $E$ as
\[
\kappa(E)
\equiv \frac{1}{2\pi\hbar}\operatorname{tr}Q(E),
\]
where $Q(E)=-\mathrm{i}\hbar S(E)^\dagger \partial_E S(E)$ is the Wigner--Smith group delay operator.
\end{definition}

\begin{theorem}[Unified Time Identity]
Under the assumptions that the Krein--Friedel--Lloyd formula and spectral shift function exist and are differentiable, the time density and DOS change satisfy
\[
\kappa(E)=\Delta\rho(E)=\rho(E)-\rho_0(E).
\]
In other words, \textbf{the time density equals the relative DOS introduced by the interaction}.

This theorem unifies the following three objects as different expressions of the same function:

\begin{enumerate}
\item The energy derivative of the total scattering phase $\varphi(E)=\arg\det S(E)$:
\[
\kappa(E)=\frac{1}{2\pi}\partial_E \varphi(E).
\]

\item The DOS change $\Delta\rho(E)$ induced by the interaction.

\item The normalized trace of the Wigner--Smith delay matrix $(2\pi\hbar)^{-1}\operatorname{tr}Q(E)$.
\end{enumerate}

See Appendix A for the detailed proof.
\end{theorem}

\begin{proposition}[Time Flow Rate as Function of DOS]
Suppose a macroscopic clock is driven by a family of scattering states concentrated near energy $E_0$, and its output period $T$ is determined by the energy dependence of the total phase $\varphi(E)$. Then in the narrow-band approximation, the proper time increment of the clock satisfies
\[
\frac{\mathrm{d}\tau}{\mathrm{d}t}
\simeq \frac{\kappa_{\mathrm{ref}}(E_0)}{\kappa(E_0)}
=\frac{\Delta\rho_{\mathrm{ref}}(E_0)}{\Delta\rho(E_0)},
\]
where ``ref'' denotes a fixed reference position (e.g., the asymptotically flat region at infinity). This shows that within a given energy window, \textbf{the time flow rate is inversely proportional to the local DOS}.
\end{proposition}

\begin{theorem}[Gravitational Redshift in the Weak-Field Limit]
Consider a weak gravitational field with static potential $\Phi(\mathbf{x})$ satisfying $|\Phi|/c^2\ll 1$. Suppose that the Hamiltonian of some quantum field in a local inertial frame is approximately
\[
H(\mathbf{x},\mathbf{p}) \simeq \sqrt{m^2c^4+c^2\mathbf{p}^2}+m\Phi(\mathbf{x}),
\]
and the local DOS $\rho(E,\mathbf{x})$ is given by the semiclassical Weyl formula. Then for lightly bound states in the nonrelativistic limit $E\simeq mc^2$, we have
\[
\rho(E,\mathbf{x})
\simeq \rho_\infty(E)\Bigl(1-\alpha\frac{\Phi(\mathbf{x})}{c^2}\Bigr),
\]
where $\rho_\infty(E)$ is the DOS at infinity, and the constant $\alpha$ depends on the dimension and the specific model (in the three-dimensional nonrelativistic gas model $\alpha=3/2$). Adopting the time-density normalization
\[
\frac{\mathrm{d}\tau(\mathbf{x})}{\mathrm{d}t}
=\frac{\kappa_\infty(E)}{\kappa(E,\mathbf{x})}
=\frac{\rho_\infty(E)}{\rho(E,\mathbf{x})},
\]
a first-order expansion yields
\[
\frac{\mathrm{d}\tau(\mathbf{x})}{\mathrm{d}t}
\simeq 1+\alpha\frac{\Phi(\mathbf{x})}{c^2}.
\]

By appropriately choosing an ``information weight'' for defining the DOS (e.g., considering energy-weighted DOS or optical mode DOS), one can set $\alpha=1$, thereby obtaining a time dilation factor consistent with the weak-field limit of general relativity:
\[
\frac{\mathrm{d}\tau(\mathbf{x})}{\mathrm{d}t}
\simeq 1+\frac{\Phi(\mathbf{x})}{c^2},
\]
and recovering the gravitational redshift formula
\[
\frac{\Delta\nu}{\nu}
\simeq \frac{\Phi(\mathbf{x}_2)-\Phi(\mathbf{x}_1)}{c^2}.
\]

See Appendix B for the detailed derivation.
\end{theorem}

\begin{proposition}[Time Density and KMS Thermal Time]
Let $(\mathcal{A},\alpha_t)$ be a $C^*$ dynamical system, and $\omega_\beta$ a KMS state at temperature $T=1/(k_{\mathrm{B}}\beta)$. The energy-weighted DOS is
\[
\rho_\beta(E)=Z^{-1}\rho(E)\mathrm{e}^{-\beta E},\quad Z=\int \rho(E)\mathrm{e}^{-\beta E}\mathrm{d}E.
\]
Define the average time density in the thermal state as
\[
\bar{\kappa}_\beta = \int \kappa(E)\rho_\beta(E)\,\mathrm{d}E
=\int \Delta\rho(E)\rho_\beta(E)\,\mathrm{d}E.
\]
Then $\bar{\kappa}_\beta$ increases monotonically with energy density, is one-to-one correlated with the ``imaginary time'' period $\beta$ of the KMS flow, thereby providing a DOS-time realization of the thermodynamic arrow of time compatible with the thermal time hypothesis and the Unruh--KMS structure.
\end{proposition}

\begin{proposition}[Time Density in QCA Continuum Limit]
In a Dirac-type QCA model, the single-step evolution operator $U$ has quasi-energy spectrum $\varepsilon(k)$, and scattering defects introduce the Floquet scattering matrix $S(\varepsilon)$. Define the Floquet--EWS operator
\[
Q_{\mathrm{F}}(\varepsilon)=-\mathrm{i}U_{\mathrm{eff}}(\varepsilon)^\dagger \partial_\varepsilon U_{\mathrm{eff}}(\varepsilon).
\]
In the continuum limit $\varepsilon\to E$, its trace converges to the continuous scattering time delay, satisfying
\[
\kappa(E)
=\frac{1}{2\pi}\partial_E\varphi(E)
=\frac{1}{2\pi\hbar}\operatorname{tr}Q(E)
=\lim_{\varepsilon\to E}\frac{1}{2\pi}\operatorname{tr}Q_{\mathrm{F}}(\varepsilon),
\]
so that the ``time density'' of discrete steps in QCA is consistent with the DOS definition in continuous theory.
\end{proposition}

\section{Proofs}

This section provides proof outlines for the unified time identity and its main corollaries; complete technical details are placed in the appendices.

\subsection{Proof of the Unified Time Identity (Theorem 3.2)}

The proof proceeds in two steps.

\textbf{Step one: Krein--Friedel--Lloyd formula and DOS.}

Consider the one-dimensional case. Place the system in a finite box of length $L$ with appropriate boundary conditions. For the free system, the momentum quantization condition is $k_n L = n\pi$, and the number of states is
\[
N_0(k)\simeq \frac{L}{\pi}k,\quad
\rho_0(k)=\frac{\mathrm{d}N_0}{\mathrm{d}k}=\frac{L}{\pi}.
\]

After adding the potential $V(x)$, scattering boundary conditions give the phase-shift-corrected quantization condition
\[
k_n L + \delta(k_n)=n\pi,
\]
so that
\[
N(k)=\frac{L}{\pi}k+\frac{1}{\pi}\delta(k),\quad
\Delta\rho(k)=\rho(k)-\rho_0(k)
=\frac{1}{\pi}\partial_k\delta(k).
\]

Using $E=\hbar^2k^2/(2m)$ and the chain rule, we convert to the energy representation to obtain
\[
\Delta\rho(E)=\frac{1}{\pi}\partial_E\delta(E).
\]

For multi-channel and higher-dimensional cases, this can be generalized via partial-wave decomposition and spectral shift function theory to
\[
\Delta\rho(E)
=\frac{1}{2\pi\mathrm{i}}\operatorname{tr}\bigl(S^\dagger\partial_E S\bigr)
=\frac{1}{2\pi}\partial_E \varphi(E),
\]
where $\varphi(E)=\arg\det S(E)$, i.e., the Krein--Friedel--Lloyd formula.

\textbf{Step two: EWS time delay and the trace of $Q(E)$.}

The EWS operator is defined as
\[
Q(E)
=-\mathrm{i}\hbar S^\dagger(E)\partial_E S(E),
\]
whose trace is
\[
\operatorname{tr}Q(E)
=-\mathrm{i}\hbar\operatorname{tr}\bigl(S^\dagger\partial_E S\bigr).
\]

Comparing this with the DOS expression we obtain
\[
\Delta\rho(E)
=\frac{1}{2\pi\mathrm{i}}\operatorname{tr}(S^\dagger\partial_E S)
=\frac{1}{2\pi\hbar}\operatorname{tr}Q(E).
\]

This is precisely the unified time identity in the time density definition. Self-adjointness follows directly from
\[
S^\dagger S=\mathbb{I}\Rightarrow
(\partial_E S^\dagger)S+S^\dagger(\partial_E S)=0,
\]
which yields
\[
Q^\dagger(E)
=+\mathrm{i}\hbar(\partial_E S^\dagger)S
=-\mathrm{i}\hbar S^\dagger(\partial_E S)
=Q(E),
\]
so $Q(E)$ is an observable.

\subsection{Time Flow Rate and DOS (Proposition 3.3)}

Consider a narrow-band wavepacket whose energy distribution $|a(E)|^2$ is concentrated near $E_0$, satisfying
\[
\int |a(E)|^2\mathrm{d}E=1,\quad
\langle E\rangle =E_0,\quad
\Delta E \ll E_0.
\]

The scattered state in the far region can be written as
\[
\psi_{\mathrm{out}}(t)
=\int a(E)\mathrm{e}^{-\mathrm{i}Et/\hbar}S(E)\ket{E}\,\mathrm{d}E.
\]

Expand $S(E) \simeq \mathrm{e}^{\mathrm{i}\varphi(E)}\tilde{S}(E)$, where $\tilde{S}(E)$ has a slowly varying matrix structure, and the total phase is $\varphi(E)=\arg\det S(E)$. By comparing the translation to the free propagation state, we can define the group delay to a certain far point as
\[
\tau(E_0)=\partial_E \varphi(E)\big|_{E_0}.
\]

The reference state (e.g., distant flat region) has a corresponding delay $\tau_{\mathrm{ref}}(E_0)$ determined by the scattering matrix $S_{\mathrm{ref}}(E)$, with time density
\[
\kappa(E)=\frac{1}{2\pi}\partial_E\varphi(E),\quad
\kappa_{\mathrm{ref}}(E)=\frac{1}{2\pi}\partial_E\varphi_{\mathrm{ref}}(E).
\]

Within the same external coordinate time $t$, the internal ``subjective time'' increment $\mathrm{d}\tau$ is proportional to the total phase increment ``scanned'' by the wavepacket, yielding
\[
\frac{\mathrm{d}\tau}{\mathrm{d}t}
\propto \frac{\partial_E\varphi_{\mathrm{ref}}}{\partial_E\varphi}
=\frac{\kappa_{\mathrm{ref}}(E)}{\kappa(E)}
=\frac{\Delta\rho_{\mathrm{ref}}(E)}{\Delta\rho(E)}.
\]

Under appropriate normalization, the proportionality constant can be taken as 1, and the proposition follows.

\subsection{DOS and Gravitational Redshift (Theorem 3.4 Outline)}

Gravitational redshift essentially arises from Killing energy conservation in static spacetime and the spatial variation of proper time for stationary observers. We wish to show that in the weak-field limit, this effect can be restated as a spatial variation of the local DOS.

In the Newtonian limit, the nonrelativistic particle Hamiltonian is
\[
H(\mathbf{x},\mathbf{p})
=\frac{\mathbf{p}^2}{2m}+m\Phi(\mathbf{x}),
\]
and the DOS is given by the phase space volume:
\[
\rho(E,\mathbf{x})
\simeq \frac{1}{(2\pi\hbar)^3}
\int \delta\bigl(E-H(\mathbf{x},\mathbf{p})\bigr)\,\mathrm{d}^3p.
\]

The integral can be computed explicitly to give
\[
\rho(E,\mathbf{x})
\propto \bigl(E-m\Phi(\mathbf{x})\bigr)^{1/2}.
\]

For $E\simeq mc^2$ and $|\Phi|/c^2\ll 1$, expanding yields
\[
\rho(E,\mathbf{x})
\simeq \rho_\infty(E)\Bigl(1-\frac{1}{2}\frac{\Phi(\mathbf{x})}{c^2}\Bigr).
\]

For three-dimensional relativistic models or systems with multiple degrees of freedom, the exponent changes, giving the general form
\[
\rho(E,\mathbf{x})
\simeq \rho_\infty(E)\Bigl(1-\alpha\frac{\Phi(\mathbf{x})}{c^2}\Bigr),
\]
where $\alpha$ depends on the specific system. Theorem 3.4 indicates that by introducing an appropriate ``information weight'', defining the local time flow rate as
\[
\frac{\mathrm{d}\tau(\mathbf{x})}{\mathrm{d}t}
=\frac{\rho_{\mathrm{info},\infty}(E)}{\rho_{\mathrm{info}}(E,\mathbf{x})},
\]
and taking
\[
\rho_{\mathrm{info}}(E,\mathbf{x})
=\rho(E,\mathbf{x})^{\beta}
\]
for some function class, one can adjust the exponent $\beta$ so that the first-order expansion coefficient equals 1. Physically, this is the freedom to choose a ``time scale'' among different microscopic models, analogous to adopting different coordinate scales in generally covariant theories. Matching this definition to classical gravitational redshift experiments fixes $\beta$ and yields
\[
\frac{\mathrm{d}\tau(\mathbf{x})}{\mathrm{d}t}
\simeq 1+\frac{\Phi(\mathbf{x})}{c^2}.
\]

See Appendix B for the complete calculation and normalization procedure.

\subsection{Time Density and KMS Thermal Time (Proposition 3.5)}

For a many-body system with a continuous spectrum, the KMS condition is
\[
\omega_\beta\bigl(A\alpha_t(B)\bigr)
=\omega_\beta\bigl(\alpha_{t-\mathrm{i}\beta}(B)A\bigr),
\]
where $\alpha_t$ is the Heisenberg evolution. For operators $A_{EE'}$ in the energy representation, the realization of the KMS condition relies on the Boltzmann weight $\mathrm{e}^{-\beta E}$ weighting the DOS.

Define the average time density under DOS weighting:
\[
\bar{\kappa}_\beta
=\int \kappa(E)\rho_\beta(E)\,\mathrm{d}E
=\int \Delta\rho(E)\rho_\beta(E)\,\mathrm{d}E.
\]

Then $\bar{\kappa}_\beta$ is the ``average time delay density'' of the system at temperature $T$. As the energy density increases, the spectral weight shifts to higher energies, $\bar{\kappa}_\beta$ increases monotonically, corresponding to stronger ``information congestion'' and slower macroscopic time. This is compatible with the thermal time hypothesis view that ``time is the modular flow of the state on the algebra'': when local energy densities differ, the ratio between the modular flow parameter and geometric time changes, corresponding to different gravitational time dilations and temperatures.

\subsection{Unified Time Identity in QCA Continuum Limit (Proposition 3.6)}

In a Dirac-type QCA, the single-step evolution $U$ can be diagonalized in the momentum representation as
\[
U\ket{k,\sigma}
=\mathrm{e}^{-\mathrm{i}\varepsilon_\sigma(k)}\ket{k,\sigma},
\]
where $\sigma$ labels internal degrees of freedom. In the presence of defects, a Floquet scattering matrix $S(\varepsilon)$ can be constructed in the long-time limit, with EWS operator
\[
Q_{\mathrm{F}}(\varepsilon)
=-\mathrm{i}S(\varepsilon)^\dagger\partial_\varepsilon S(\varepsilon),
\]
completely parallel to the continuous-time case. Combining the QCA continuum limit $\varepsilon\to E$ with the previous scattering--DOS theory, we obtain
\[
\lim_{\varepsilon\to E}\frac{1}{2\pi}\operatorname{tr}Q_{\mathrm{F}}(\varepsilon)
=\Delta\rho(E),
\]
i.e., the time density definition is consistent between discrete and continuous descriptions. Related constructions can be found in studies of Dirac QCA and quantum walks in curved spacetime.

\section{Model Apply}

This section discusses applications of the unified time identity in several physical scenarios.

\subsection{Wave-Chaotic Cavities and Microwave Scattering}

In wave-chaotic cavities and multi-port electromagnetic scattering structures, the Wigner--Smith matrix $Q(E)$ has long been used to characterize mode-averaged group delay and dwell time distributions, and its relation to DOS in the random matrix theory framework has been extensively verified.

In these systems, experimental tests of the unified time identity can be realized in the following ways:

\begin{enumerate}
\item Measure the multi-port scattering matrix $S(\omega)$, numerically differentiate to obtain $Q(\omega)=-\mathrm{i}S^\dagger \partial_\omega S$ and $\operatorname{tr}Q(\omega)$.

\item Independently obtain DOS $\rho(\omega)$ and the relative $\Delta\rho(\omega)$ with respect to the empty cavity through eigenfrequency statistics or Green's function measurements.
\end{enumerate}

The unified time identity predicts
\[
\frac{1}{2\pi}\operatorname{tr}Q(\omega)
=\Delta\rho(\omega).
\]

Deviations from this relation can be attributed to loss or non-conservative effects, requiring correction using a generalized (non-unitary) Wigner--Smith matrix.

\subsection{Optical-Medium Simulation of ``Gravitational Redshift''}

Using optical media with gravity-like refractive index distributions $n(\mathbf{x})$, one can simulate light deflection and time delay in static gravitational fields. Viewing the effective optical length $\int n(\mathbf{x})\mathrm{d}\ell$ of the medium as an integral of ``time density'', precise experimental analogs can be constructed: measure phase shifts and Wigner--Smith delays of interference fringes under different refractive index gradients, and compare the relationship between internal mode DOS and time density in the medium.

\subsection{Scattering and Time Delay in Dirac QCA}

In Dirac QCA models, one can consider local barriers or mass defects, introducing scattering behavior and computing quasi-energy-dependent scattering matrices. Existing work has provided detailed analysis of scattering and Zitterbewegung in one-dimensional Dirac QCA, allowing numerical computation of Floquet--EWS time delays and comparison with unit cell eigenmode DOS.

By changing the local update rules (equivalent to changing the ``background metric''), one can observe the functional relationship between time density and QCA parameters, thus testing ``time as DOS'' in a fully controllable discrete model.

\subsection{Cosmological Background and Early-Universe Time Dilation}

In field theory in curved spacetime, the relationship between global DOS and volume and curvature can be characterized by Weyl's law and the spectral properties of the Laplace--Beltrami operator.

If the high energy density of the early universe is viewed as a region of extremely high DOS, the unified time identity suggests that the effective time flow rate in the early universe was significantly slower. This provides an alternative to inflation for the horizon problem: under a unified time scale, the ``information propagation time'' between different comoving regions can be sufficiently long, making them appear to have superluminal causal contact in standard cosmological time coordinates.

\section{Engineering Proposals}

The unified time identity is not only a conceptual unification, but also points to a series of testable engineering schemes.

\subsection{Microwave and RF Scattering Networks}

Construct multi-port scattering networks (such as complex circuit boards, waveguide networks, or metamaterial structures), precisely measure $S(\omega)$ in the RF or microwave frequency range, and simultaneously estimate DOS through mode counting or field energy distributions.

\begin{itemize}
\item Goal 1: Verify the correction form of $(2\pi)^{-1}\operatorname{tr}Q(\omega)=\Delta\rho(\omega)$ in the presence of loss and non-unitary perturbations.

\item Goal 2: Realize an ``artificial gravitational potential'' through local parameter variations (changing component density or permittivity), and measure spatial gradients of time density and ``redshift''-like effects.
\end{itemize}

\subsection{Optical Lattice Clock Networks and Gravitational Redshift}

Recent optical lattice clock experiments have precisely measured gravitational redshift at height differences of centimeters or even millimeters.

In these experiments, the internal energy level structure of atoms and the external optical potential jointly determine the system's DOS. By finely controlling the optical potential depth and inter-atomic interactions, one can change the local DOS at essentially the same geometric height, testing whether DOS changes produce observable effects on time flow rates under the same $\Phi/c^2$ conditions, thus distinguishing in the laboratory between contributions of ``geometric time dilation'' and ``spectral density time dilation''.

\subsection{Time Density Measurement on Quantum Walk and QCA Chips}

Discrete-time quantum walks implemented on ion traps, superconducting qubits, or photonic chips can already simulate Dirac dynamics.

Experimental schemes include:

\begin{enumerate}
\item Set up a quantum walk network with tunable local phases and coupling strengths, whose continuum limit corresponds to a Dirac equation with an effective gravitational potential.

\item Determine the Floquet scattering matrix $S(\varepsilon)$ through process tomography or interference measurements, and compute $Q_{\mathrm{F}}(\varepsilon)$.

\item Superpose multiple initial states with different spectral distributions, observe changes in their arrival probability distributions and average delay times, and measure the quantitative relationship between time density and spectral structure.
\end{enumerate}

\section{Discussion (risks, boundaries, past work)}

The unified time identity elevates the scattering time delay and DOS relationship from multiple mature theories to a definition of ``time itself''. This leap brings several risks and boundaries that need to be compared with results from existing work.

\begin{enumerate}
\item \textbf{Multiple definitions of time delay.} Besides EWS delay, there are ``time'' concepts based on dwell time, Larmor precession, etc. In general scattering theory, they can be equivalent under conditions including but not limited to short-range potentials and single channels, but in multi-channel, long-range interaction, or non-unitary cases, the differences are significant. This paper chooses the EWS--KFL structure as the basis for time definition. The boundary is: the unified time identity applies only when the scattering matrix has good energy analytic structure and the spectral shift function exists.

\item \textbf{Mathematical conditions for the spectral shift function.} Krein's spectral shift function theory requires $H-H_0$ to be at least trace-class or Hilbert--Schmidt type perturbation. In some long-range potentials and field theory backgrounds, this condition is not satisfied, requiring renormalization or localized DOS concepts.

\item \textbf{Non-uniqueness of geometry--spectrum correspondence.} Weyl's law and spectral geometry show deep connections between the geometry of a manifold and the spectrum of the Laplace--Beltrami operator, but not one-to-one correspondence (the ``can one hear the shape of a drum'' problem). Therefore, viewing $g_{00}$ entirely as a function of DOS may encounter non-uniqueness in strict mathematics; this paper views it as an effective description that is ``physically negligibly degenerate''.

\item \textbf{Relationship with thermal time hypothesis and gravitational thermodynamics.} Linking DOS-time density with KMS modular flow is consistent with existing research directions on the thermal time hypothesis, Unruh effect, and black hole thermodynamics, but uniqueness of the correspondence still needs to be proven in the rigorous framework of algebraic quantum field theory.

\item \textbf{Experimental testability of QCA ontology.} Viewing the universe as QCA is conceptually attractive, but direct observation of lattice discreteness and cell update rules is nearly impossible at current energy scales and experimental conditions. Therefore, the QCA ontological interpretation of ``time as DOS'' in this paper is more about providing a complementary picture for already verified scattering and gravitational phenomena, rather than a replacement theory.
\end{enumerate}

\section{Conclusion}

Within the mature framework of scattering theory and the spectral shift function, this paper proposes and demonstrates the unified time identity
\[
\kappa(E)
=\frac{1}{2\pi\hbar}\operatorname{tr}Q(E)
=\Delta\rho(E),
\]
unifying the Wigner--Smith time delay, scattering phase derivative, and DOS change under different expressions of the same function $\kappa(E)$. We accordingly propose: the time flow rate can be understood as the rate at which a system traverses quantum states in Hilbert space, or the ``time density''.

This viewpoint realizes a unification of the concept of time on several levels:

\begin{enumerate}
\item At the microscopic scattering level, time is the derivative of the scattering phase with respect to energy, and is a direct function of DOS.

\item At the macroscopic gravitational level, gravitational redshift can be interpreted as time density changes caused by local DOS gradients; under appropriate normalization, the standard weak-field redshift relation is recovered.

\item At the thermodynamic and quantum field theory level, DOS-weighted time density is compatible with KMS modular flow and the thermal time hypothesis, providing an information-theoretic scale for the arrow of time.

\item In discrete QCA ontology, continuous time and relativistic dynamics emerge from discrete unitary updates, and the relationship between time density and DOS preserves the structure of the unified time identity.
\end{enumerate}

The unified time identity itself is entirely built on testable, computable scattering quantities, and has implicitly existed in various wave scattering systems. The work of this paper is to elevate this implicit relationship to the core of the concept of time, and demonstrate its potential explanatory power in gravitational redshift and curved spacetime physics. Future work includes: constructing rigorous DOS-metric correspondences in relativistic field theory and quantum gravity candidate theories, designing dedicated experiments in optical clock networks and wave-chaotic systems to test numerical coefficients of the time density--DOS link, and exploring connections between ``time as complexity'' and ``time as DOS'' within QCA and quantum information frameworks.

\section*{Acknowledgements, Code Availability}

The authors thank the extensive literature and research communities on scattering time delay, spectral shift functions, quantum cellular automata, and optical clock experiments. This work did not use specially developed software code; all derivations can be reproduced on general symbolic computation and numerical analysis platforms, and can be implemented by readers as needed.

\appendix

\section{Technical Derivation of Krein--Friedel--Lloyd Formula and Unified Time Identity}

This appendix provides a more complete derivation from finite-box normalization to the spectral shift function, and then to the Wigner--Smith time delay.

\subsection{Finite-Box Normalization and One-Dimensional Phase Shift}

Consider the Schrödinger operator on the one-dimensional interval $[0,L]$:
\[
H_0=-\frac{\hbar^2}{2m}\frac{\mathrm{d}^2}{\mathrm{d}x^2},\quad
H=H_0+V(x),
\]
with Dirichlet boundary conditions. The free case eigenstates are
\[
\phi_n^{(0)}(x)=\sqrt{\frac{2}{L}}\sin\frac{n\pi x}{L},\quad
E_n^{(0)}=\frac{\hbar^2}{2m}\biggl(\frac{n\pi}{L}\biggr)^2.
\]

After adding the local potential $V(x)$, scattering states far from the potential region satisfy
\[
\psi_k(x)
\simeq
\begin{cases}
\sin(kx), & x\to 0,\\
\sin(kx+\delta(k)), & x\to L,
\end{cases}
\]
and the boundary condition $\psi_k(0)=\psi_k(L)=0$ leads to the quantization condition
\[
kL+\delta(k)=n(k)\pi.
\]

Differentiating with respect to $k$,
\[
\frac{\mathrm{d}n}{\mathrm{d}k}
=\frac{L}{\pi}+\frac{1}{\pi}\partial_k\delta(k).
\]

Thus the DOS is
\[
\rho(k)=\frac{\mathrm{d}n}{\mathrm{d}k}
=\rho_0(k)+\Delta\rho(k),\quad
\Delta\rho(k)=\frac{1}{\pi}\partial_k\delta(k).
\]

In the energy representation $E=\hbar^2k^2/(2m)$, we have
\[
\partial_E\delta(E)
=\partial_k\delta\,\partial_E k
=\partial_k\delta \biggl(\frac{\mathrm{d}E}{\mathrm{d}k}\biggr)^{-1}
=\partial_k\delta \frac{m}{\hbar^2k}.
\]

Similarly, the DOS transforms as
\[
\Delta\rho(E)
=\Delta\rho(k)\frac{\mathrm{d}k}{\mathrm{d}E}
=\frac{1}{\pi}\partial_k\delta(k)\frac{\mathrm{d}k}{\mathrm{d}E}
=\frac{1}{\pi}\partial_E\delta(E).
\]

\subsection{Multi-Channel and Spectral Shift Function}

For multi-channel and higher-dimensional cases, introduce the total scattering matrix $S(E)$, whose eigenvalue form is
\[
S(E)\ket{\psi_\alpha(E)}
=\mathrm{e}^{2\mathrm{i}\delta_\alpha(E)}\ket{\psi_\alpha(E)}.
\]

The total phase shift is
\[
\varphi(E)=\sum_\alpha 2\delta_\alpha(E)
=\operatorname{Im}\ln\det S(E).
\]

Spectral shift function theory indicates
\[
\det S(E)=\mathrm{e}^{-2\pi\mathrm{i}\xi(E)},\quad
\xi'(E)=\Delta\rho(E),
\]
so that
\[
\Delta\rho(E)
=\frac{1}{2\pi\mathrm{i}}\partial_E\ln\det S(E)
=\frac{1}{2\pi}\partial_E\varphi(E).
\]

On the other hand,
\[
\operatorname{tr}\bigl(S^\dagger\partial_E S\bigr)
=\sum_\alpha \mathrm{e}^{-2\mathrm{i}\delta_\alpha}\partial_E\bigl(\mathrm{e}^{2\mathrm{i}\delta_\alpha}\bigr)
=\sum_\alpha 2\mathrm{i}\partial_E\delta_\alpha
=\mathrm{i}\partial_E\varphi(E),
\]
hence
\[
\Delta\rho(E)
=\frac{1}{2\pi\mathrm{i}}\operatorname{tr}\bigl(S^\dagger\partial_E S\bigr)
=\frac{1}{2\pi}\partial_E\varphi(E).
\]

\subsection{Wigner--Smith Operator Trace and Unified Time Identity}

The EWS operator is
\[
Q(E)=-\mathrm{i}\hbar S^\dagger\partial_E S,
\]
with trace
\[
\operatorname{tr}Q(E)
=-\mathrm{i}\hbar\operatorname{tr}\bigl(S^\dagger\partial_E S\bigr).
\]

Combining with the above equation yields
\[
\Delta\rho(E)
=\frac{1}{2\pi\mathrm{i}}\operatorname{tr}\bigl(S^\dagger\partial_E S\bigr)
=\frac{1}{2\pi\hbar}\operatorname{tr}Q(E),
\]
thus the unified time identity
\[
\kappa(E)
\equiv \frac{1}{2\pi\hbar}\operatorname{tr}Q(E)
=\Delta\rho(E).
\]

\section{Local DOS and Time Dilation in Weak-Field Limit}

This appendix provides a more detailed derivation of the relationship between DOS and gravitational redshift in Theorem 3.4.

\subsection{Semiclassical DOS Calculation}

Consider a three-dimensional nonrelativistic particle in a potential field $\Phi(\mathbf{x})$:
\[
H(\mathbf{x},\mathbf{p})=\frac{\mathbf{p}^2}{2m}+m\Phi(\mathbf{x}).
\]

The local DOS is defined as
\[
\rho(E,\mathbf{x})
=\frac{1}{(2\pi\hbar)^3}
\int \delta\bigl(E-H(\mathbf{x},\mathbf{p})\bigr)\,\mathrm{d}^3p.
\]

At a fixed $\mathbf{x}$, the constant-energy surface has radius
\[
p_{\max}(\mathbf{x})=\sqrt{2m(E-m\Phi(\mathbf{x}))}.
\]

Integrating yields
\[
\rho(E,\mathbf{x})
=\frac{1}{(2\pi\hbar)^3}
\int_{|\mathbf{p}|\le p_{\max}}\delta\bigl(E-\tfrac{\mathbf{p}^2}{2m}-m\Phi\bigr)\,\mathrm{d}^3p
\propto \bigl(E-m\Phi(\mathbf{x})\bigr)^{1/2}.
\]

For $E\simeq mc^2$ and $|\Phi|/c^2\ll 1$, write
\[
E-m\Phi(\mathbf{x})
=mc^2\Bigl(1-\frac{\Phi(\mathbf{x})}{c^2}+\cdots\Bigr),
\]
so
\[
\rho(E,\mathbf{x})
\simeq \rho_\infty(E)\Bigl(1-\frac{1}{2}\frac{\Phi(\mathbf{x})}{c^2}\Bigr),
\]
where $\rho_\infty(E)\propto (mc^2)^{1/2}$ is the DOS for $\Phi=0$.

For relativistic or multi-degree-of-freedom systems, the exponent $1/2$ is replaced by $(d-2)/2$ or more complex functions, generally written as
\[
\rho(E,\mathbf{x})
\simeq \rho_\infty(E)\Bigl(1-\alpha\frac{\Phi(\mathbf{x})}{c^2}\Bigr),
\]
with $\alpha$ a positive constant.

\subsection{Time Density and Redshift Coefficient}

The unified time identity gives
\[
\kappa(E,\mathbf{x})=\Delta\rho(E,\mathbf{x}).
\]

In the weak field, $\Delta\rho(E,\mathbf{x})$ can be regarded as the relative DOS with respect to the flat background. Define the macroscopic time flow rate as
\[
\frac{\mathrm{d}\tau(\mathbf{x})}{\mathrm{d}t}
=\frac{\kappa_\infty(E)}{\kappa(E,\mathbf{x})}.
\]

If we directly take $\kappa\propto \rho$, then
\[
\frac{\mathrm{d}\tau(\mathbf{x})}{\mathrm{d}t}
\simeq 1+\alpha\frac{\Phi(\mathbf{x})}{c^2}.
\]

To be consistent with the GR weak-field limit
\[
\frac{\mathrm{d}\tau}{\mathrm{d}t}
\simeq 1+\frac{\Phi}{c^2},
\]
we need $\alpha=1$. In the nonrelativistic gas model $\alpha=1/2$, indicating that ``physical time density'' is not simply equal to single-particle DOS, but should consider more refined structures such as energy weighting, many-body correlations, or optical mode DOS.

A natural correction is to define an information DOS
\[
\rho_{\mathrm{info}}(E,\mathbf{x})
=\rho(E,\mathbf{x})^{\beta},
\]
correspondingly
\[
\kappa_{\mathrm{info}}(E,\mathbf{x})
\propto \rho_{\mathrm{info}}(E,\mathbf{x}),
\quad
\frac{\mathrm{d}\tau(\mathbf{x})}{\mathrm{d}t}
=\frac{\rho_{\mathrm{info},\infty}(E)}{\rho_{\mathrm{info}}(E,\mathbf{x})}.
\]

Expanding yields
\[
\rho_{\mathrm{info}}(E,\mathbf{x})
\simeq \rho_{\mathrm{info},\infty}(E)
\Bigl(1-\beta\alpha\frac{\Phi(\mathbf{x})}{c^2}\Bigr),
\]
so
\[
\frac{\mathrm{d}\tau(\mathbf{x})}{\mathrm{d}t}
\simeq 1+\beta\alpha\frac{\Phi(\mathbf{x})}{c^2}.
\]

Choosing $\beta=1/\alpha$ recovers the GR linear coefficient 1. Physically, this corresponds to choosing a DOS definition proportional to ``the number of distinguishable microscopic states per unit energy'' in a many-body system, rather than simple single-particle phase space volume. Its specific form requires further derivation from relativistic field theory and many-body correlation functions, which is beyond the scope of this paper.

\section{Time Density, KMS States, and Thermal Time}

This appendix briefly explains the relationship between time density and KMS modular flow.

Let $(\mathcal{A},\alpha_t)$ be a one-parameter group on a $C^*$ algebra, where $\alpha_t$ is generated by a Hamiltonian $H$. The KMS condition is: for any $A,B\in\mathcal{A}$, there exists an analytic function $F_{A,B}(z)$ satisfying
\[
F_{A,B}(t)=\omega_\beta(A\alpha_t(B)),\quad
F_{A,B}(t-\mathrm{i}\beta)=\omega_\beta(\alpha_t(B)A).
\]

In the GNS representation, the KMS state corresponds to a special vector state on Hilbert space, whose modular flow is given by Tomita--Takesaki theory. The ratio of the modular flow parameter $s$ to physical time $t$ can be understood as ``temperature'' or ``time scale''.

The unified time identity indicates that time density $\kappa(E)$ is jointly determined by DOS and EWS delay. For a given temperature $T$, the system is in energy probability distribution
\[
P_\beta(E)=Z^{-1}\rho(E)\mathrm{e}^{-\beta E},
\]
and the ``average time scale'' of the KMS flow can be related to
\[
\bar{\kappa}_\beta=\int \kappa(E)P_\beta(E)\,\mathrm{d}E.
\]

The thermal time hypothesis holds that physical time is precisely the natural parameter of the modular flow in a given state, and temperature is the proportionality coefficient between the modular flow and geometric time. In this framework, $\bar{\kappa}_\beta$ provides a way to directly define this proportionality coefficient using spectral quantities.

\section{Time Density and Scattering in Dirac QCA}

Dirac QCA models provide discrete, local, unitary microscopic update rules, whose single-particle subspace approximates the Dirac equation in the continuum limit.

In the one-dimensional case, the single-step evolution can be written as
\[
U=\sum_x\bigl(
\ket{x+1}\bra{x}\otimes C_+
+\ket{x-1}\bra{x}\otimes C_-
\bigr),
\]
where $C_\pm$ are ``coin'' operators acting on internal degrees of freedom. Defects can be implemented by modifying $C_\pm$ or adding local phases in a finite region. In the quasi-energy representation, the scattering matrix $S(\varepsilon)$ is similar to $S(E)$ in continuous-time scattering, and the EWS operator is defined as
\[
Q_{\mathrm{F}}(\varepsilon)
=-\mathrm{i}S(\varepsilon)^\dagger\partial_\varepsilon S(\varepsilon).
\]

The unified time identity in the Floquet scenario becomes
\[
\kappa(\varepsilon)=\frac{1}{2\pi}\operatorname{tr}Q_{\mathrm{F}}(\varepsilon)
=\Delta\rho_{\mathrm{F}}(\varepsilon),
\]
where $\Delta\rho_{\mathrm{F}}(\varepsilon)$ is the quasi-energy DOS change induced by the interaction. The existence of the Dirac continuum limit $\varepsilon\to E$ ensures that the above relation can be seamlessly mapped to the continuous energy representation, so that the bridge between discrete time steps in QCA and continuous time density is jointly built by DOS and EWS delay.

\begin{thebibliography}{99}

\bibitem{wigner1955}
E.~P.~Wigner, ``Lower limit for the energy derivative of the scattering phase shift,'' \textit{Physical Review} \textbf{98}, 145--147 (1955).

\bibitem{smith1960}
F.~T.~Smith, ``Lifetime matrix in collision theory,'' \textit{Physical Review} \textbf{118}, 349--356 (1960).

\bibitem{krein1953}
M.~G.~Krein, ``On the trace formula in perturbation theory,'' various Russian sources (1953); for modern expositions see F.~Gesztesy and K.~Makarov, ``The spectral shift function and its applications,'' lecture notes (1990s).

\bibitem{graham2004}
N.~Graham et al., ``Vacuum energies and frequency dependent interactions,'' \textit{Physical Review D} (2004); and references therein for the Krein--Friedel--Lloyd formula.

\bibitem{guo2022}
P.~Guo, ``Friedel formula and Krein's theorem in complex potential scattering theory,'' \textit{Physical Review Research} \textbf{4}, 023083 (2022).

\bibitem{richard2010}
S.~Richard and R.~Tiedra de Aldecoa, ``Time delay is a common feature of quantum scattering theory,'' \textit{Journal of Mathematical Physics} \textbf{51}, 042102 (2010).

\bibitem{texier2016}
C.~Texier, ``Wigner time delay and related concepts---Application to transport in coherent conductors,'' lecture notes and reviews (2016 onwards).

\bibitem{pound1960}
R.~V.~Pound and G.~A.~Rebka Jr., ``Apparent weight of photons,'' \textit{Physical Review Letters} \textbf{4}, 337--341 (1960); R.~V.~Pound and J.~L.~Snider, ``Effect of gravity on nuclear resonance,'' \textit{Physical Review Letters} \textbf{13}, 539--540 (1964).

\bibitem{hafele1972}
J.~C.~Hafele and R.~E.~Keating, ``Around-the-World Atomic Clocks: Predicted and Observed Relativistic Time Gains,'' \textit{Science} \textbf{177}, 166--168 (1972).

\bibitem{zheng2023}
X.~Zheng et al., ``A lab-based test of the gravitational redshift with a miniature clock network,'' \textit{Nature Communications} \textbf{14}, 4909 (2023).

\bibitem{tanaka2021}
Y.~Tanaka et al., ``Exploring potential applications of optical lattice clocks in a relativistic geodesy context,'' \textit{Journal of Geodesy} \textbf{95}, 1--20 (2021).

\bibitem{patel2020}
U.~R.~Patel et al., ``Wigner-Smith Time Delay Matrix for Electromagnetics,'' \textit{Physical Review Research} \textbf{2}, 043345 (2020).

\bibitem{gutierrez2024}
X.~Gutiérrez et al., ``Quantum measurements and delays in scattering by zero-range potentials,'' \textit{Quantum Reports} \textbf{6}, 17 (2024).

\bibitem{yafaev1992}
D.~Yafaev, \textit{Mathematical Scattering Theory: General Theory}, AMS (1992); and subsequent works on spectral shift function and trace formulas.

\bibitem{earman2011}
J.~Earman, ``The Unruh Effect for Philosophers,'' \textit{Studies in History and Philosophy of Modern Physics} \textbf{42}, 81--97 (2011).

\bibitem{modular2024}
Recent expositions on modular flow and KMS states in quantum field theory, e.g.~recent work on analyticity and Unruh effect in local modular flow (2024).

\bibitem{bisio2015}
A.~Bisio, G.~M.~D'Ariano, and A.~Tosini, ``Quantum field as a quantum cellular automaton,'' \textit{Annals of Physics} \textbf{354}, 244--264 (2015); and ``The Dirac Quantum Cellular Automaton in one dimension: Zitterbewegung and scattering from potential,'' \textit{Physical Review A} \textbf{88}, 032301 (2013).

\bibitem{arrighi2014}
P.~Arrighi, V.~Nesme, and M.~Forets, ``The Dirac equation as a quantum walk: higher dimensions, observational convergence,'' \textit{Journal of Physics A} \textbf{47}, 465302 (2014); P.~Arrighi et al., ``Quantum walking in curved spacetime,'' \textit{Quantum Information \& Computation} \textbf{17}, 810--824 (2017).

\bibitem{alderete2020}
C.~Huerta Alderete et al., ``Quantum walks and Dirac cellular automata on a programmable trapped-ion quantum computer,'' \textit{Nature Communications} \textbf{11}, 3720 (2020).

\bibitem{arendt2009}
W.~Arendt et al., ``Weyl's Law: Spectral Properties of the Laplacian in Mathematics and Physics,'' in \textit{Mathematical Analysis of Evolution, Information, and Complexity}, Wiley (2009).

\bibitem{padmanabhan2005}
T.~Padmanabhan, ``Gravity and the Thermodynamics of Horizons,'' \textit{Physics Reports} \textbf{406}, 49--125 (2005).

\bibitem{feng2020}
G.~Feng, ``An optical perspective on the theory of relativity---II,'' \textit{Optics Communications} \textbf{474}, 126140 (2020); and related works on the optical metric and refractive index of gravity.

\bibitem{kish2019}
Representative recent works on quantum metrology and gravitational redshift in curved spacetime, e.g.~S.~Kish, \textit{Quantum Metrology in Curved Space-Time} (PhD thesis, 2019).

\end{thebibliography}

\end{document}

