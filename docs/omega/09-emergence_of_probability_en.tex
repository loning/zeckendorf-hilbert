\documentclass[11pt]{article}
\usepackage[utf8]{inputenc}
\usepackage[T1]{fontenc}
\usepackage{amsmath,amssymb,amsthm}
\usepackage{mathtools}
\usepackage{geometry}
\geometry{margin=1in}
\usepackage{hyperref}
\usepackage{cite}
\usepackage{braket}

\newtheorem{theorem}{Theorem}
\newtheorem{lemma}[theorem]{Lemma}
\newtheorem{proposition}[theorem]{Proposition}
\newtheorem{corollary}[theorem]{Corollary}
\theoremstyle{definition}
\newtheorem{definition}[theorem]{Definition}
\newtheorem{assumption}[theorem]{Assumption}
\theoremstyle{remark}
\newtheorem{remark}[theorem]{Remark}

\title{The Emergence of Probability: Deriving the Born Rule from Global Unitarity and Observer-Relative States in Quantum Cellular Automata}

\author{Haobo Ma$^1$ \and Wenlin Zhang$^2$\\
\small $^1$Independent Researcher\\
\small $^2$National University of Singapore}

\date{}

\begin{document}

\maketitle

\begin{abstract}
Quantum cellular automata (QCA) provide a discrete, local, unitary ontological picture of the universe: the global state evolves via deterministic unitary operators between discrete time steps, containing no intrinsic stochastic process. On the other hand, laboratory quantum mechanics centers on the Born rule, treating measurement outcomes as inherently probabilistic events. The tension between these two perspectives constitutes the core of the quantum measurement problem.

In this paper, within the QCA framework, we introduce ``local observers with finite information capacity'' and their environmental entanglement structure, providing a purely combinatorial mechanism for probability emergence. Specifically, in a QCA universe with strictly finite propagation velocity and global unitarity: the measurement process is modeled as local unitary coupling among the measured system, observer memory register, and environment; decoherence (einselection) selects stable pointer states; due to information horizon and memory capacity limitations, the observer can only access macroscopic branches compatible with their own record and cannot distinguish all microscopic QCA basis states underlying these branches. Applying environment-assisted invariance (envariance) symmetry arguments to these microscopic basis states, combined with an ``equal ontological weight'' hypothesis, we interpret the modulus of complex amplitudes as the square root of microscopic path degeneracy. Thus we prove: the subjective probability that an observer assigns to each measurement outcome equals the proportion of microscopic configurations compatible with that outcome among all accessible configurations, namely the standard Born weight $p_k = \lvert \psi_k \rvert^2$.

In the sense of rational consistency and axiomatization, we obtain the following result: in local Hilbert spaces of dimension at least three, if we require (i) global dynamics given by local QCA unitary evolution, (ii) measurement probabilities satisfying non-contextuality and environment insensitivity, (iii) no superluminal signal propagation, then Gleason-type theorems combined with QCA microscopic counting uniquely yield the Born rule as the only viable probability assignment. Wavefunction collapse is reinterpreted here as: Bayesian conditionalization and self-location update of the local observer on the global unitary state, rather than a nonlinear interruption of fundamental dynamics. Therefore, quantum probability is not an essential component of nature, but a statistical necessity for finite observers performing self-location in a discrete holographic entanglement network.
\end{abstract}

\noindent\textbf{Keywords:} Quantum cellular automaton; Born rule; quantum measurement problem; environment-assisted invariance; decoherence; relative state interpretation; self-locating uncertainty; information horizon

\section{Introduction \& Historical Context}

\subsection{The Quantum Measurement Problem and Status of the Born Rule}

Standard quantum theory consists of two seemingly incompatible evolution laws: the time evolution of isolated systems is governed by the linear, unitary Schr\"odinger equation, while the measurement process is described by a nonlinear, stochastic ``collapse'' rule. Experimentally obtained frequencies conform to Born's formula $p_k = \lvert \langle k \vert \psi \rangle \rvert^2$, yet this formula is directly introduced as an independent axiom in most textbooks, without derivation from more primitive structures.

Decoherence theory demonstrates that coupling between system and environment leads to rapid suppression of coherence terms in the effective description, thereby selecting stable pointer states and explaining the emergence of classical trajectories. Decoherence has been systematically developed mathematically and experimentally, occupying a central position in quantum-to-classical transition research. However, most authors acknowledge: decoherence itself does not generate specific numerical probabilities, but rather characterizes how interference terms are ``annihilated'' by the environment, assuming the Born rule holds.

Thus, a natural question arises: can the Born rule be derived from more fundamental structures, without assuming probability axioms a priori?

\subsection{Existing Born Rule Derivation Schemes}

A broad literature has formed around Born rule derivation. Gleason's theorem proves in Hilbert spaces of dimension at least three: if probabilities assigned to projection measures satisfy non-contextuality and countable additivity, then these probabilities must be realized by some density operator through the trace formula, essentially equivalent to the Born rule. Deutsch, Wallace, and others, under the Everett interpretation, use classical decision theory to argue that utility maximization for rational agents across many-world branches requires Born weights.

Zurek's environment-assisted invariance (envariance) approach takes entanglement symmetry between system and environment as the starting point: in perfect entangled states, certain unitary transformations applied to the system can be ``compensated'' by applying another unitary transformation to the environment, leaving the overall state invariant. Using this symmetry and path subdivision for equal-amplitude superposition states, one can formally derive $p_k \propto \lvert \psi_k \rvert^2$. Other works attempt to derive the Born rule from non-contextual probability assumptions, no-signaling principles, etc.

Meanwhile, criticisms of these derivations point out: all known schemes explicitly or implicitly introduce assumptions approximately equivalent to the Born rule, such as continuity, additivity, or independence from measurement context. Therefore, a strictly ``no additional axiom'' derivation remains an open problem.

\subsection{QCA and Cellular Automaton Interpretation}

In parallel, discussions about whether quantum theory can be reduced to discrete, local, deterministic cellular automaton dynamics continue to develop. 't Hooft's cellular automaton interpretation treats quantum states as statistical envelopes over a more fundamental classical automaton, seeking an essentially deterministic microscopic theory. On the other hand, quantum cellular automaton (QCA) models constructed by D'Ariano, Bisio, Perinotti, and others demonstrate that on discrete lattices satisfying locality, homogeneity, and isotropy, Weyl, Dirac, and Maxwell equations can emerge in the continuum limit, making QCA a concrete mathematical realization of the ``universe as quantum computation'' vision.

In the QCA universe picture, the global state $\lvert \Psi_t \rangle$ evolves according to a local unitary operator $U$ across discrete time steps: $\lvert \Psi_{t+1} \rangle = U \lvert \Psi_t \rangle$. Dynamics are strictly deterministic, with no intrinsic randomness. The origin of probabilistic features manifested in measurements is the most pressing foundational question in this picture.

\subsection{Goals and Basic Strategy of This Paper}

This paper's goal is to provide an emergent derivation of the Born rule under the following premises:

1. The universe is described by a local, translation-invariant, unitary QCA;

2. Observers are local subsystems on the QCA with finite information capacity, whose ``records'' are realized as pointer states selected by environmental decoherence;

3. Probability is understood as the observer's self-locating uncertainty on the global unitary state, not an ontological stochastic process.

Within this framework, measurement is modeled as a local entanglement process among measured system $S$, observer memory register $O$, and environment $E$. Subsequent environment-induced decoherence makes interference between different measurement outcome branches invisible to $O$'s accessible operators. We introduce an ``equal ontological weight hypothesis'': in the QCA's ontological basis, each orthogonal microscopic configuration (or path) has equal ontological weight. Exploiting QCA's discreteness, we write complex amplitudes as square roots of degeneracies of equal-weight microscopic state superpositions, interpreting probability as the normalized ratio of ``the number of microscopic configurations compatible with a given macroscopic record''.

To make this construction universal and unique, this paper proves under support from Gleason's theorem, no-signaling constraints, and envariance symmetry: in a QCA universe, as long as local measurement probabilities satisfy non-contextuality and environment insensitivity, the Born rule is the only probability assignment compatible with global unitarity, locality, and observer finiteness.

\section{Model \& Assumptions}

\subsection{QCA Universe and Ontological Basis}

Let the universe be modeled as a quantum cellular automaton defined on a discrete lattice set $\Lambda \subset \mathbb{Z}^d$. Each site $x \in \Lambda$ carries a finite-dimensional Hilbert space $\mathcal{H}_x \simeq \mathbb{C}^{d_{\mathrm{cell}}}$, and the global Hilbert space is the tensor product $\mathcal{H} = \bigotimes_{x \in \Lambda} \mathcal{H}_x$. QCA dynamics are given by a family of discrete-time unitary operators $U$ satisfying:

1. Locality: $U$ can be decomposed into finite-depth circuits of local unitary gates acting on finite neighborhoods;

2. Homogeneity and translation invariance: local rules are identical between sites;

3. Finite propagation velocity: there exists a Lieb--Robinson velocity $v_{\mathrm{LR}}$ such that any local perturbation's support is confined within an effective light cone of radius approximately $v_{\mathrm{LR}} t$ at time $t$.

Selecting an eigenbasis $\{ \lvert \gamma_x \rangle \}$ for each cell, the global ontological basis consists of tensor products $\lvert \gamma \rangle = \bigotimes_{x} \lvert \gamma_x \rangle$. In this paper, ``microscopic configuration'' and ``ontological basis vector'' are synonymous.

\begin{assumption}[Equal Ontological Weight (A1)]
In QCA ontology, each orthogonal ontological basis state $\lvert \gamma \rangle$ has equal basic weight; any probabilistic statement arises from counting or weighting these basis states under certain conditions.
\end{assumption}

This assumption corresponds to the ``ontological sequence equivalence'' idea in 't Hooft's cellular automaton schemes, while harmonizing with translation invariance of the ontological basis and uniformity of local rules in QCA.

\subsection{System--Observer--Environment Partition}

In the global QCA universe, select a finite region $\Lambda_S$ as the measured system $S$, another finite region $\Lambda_O$ as the observer and measurement apparatus $O$, and the remaining sites as environment $E$. The corresponding Hilbert space decomposes as
$$
\mathcal{H} \simeq \mathcal{H}_S \otimes \mathcal{H}_O \otimes \mathcal{H}_E.
$$

The observer's Hilbert space $\mathcal{H}_O$ is divided into ``record subspace'' and ``remaining degrees of freedom'':
$$
\mathcal{H}_O \simeq \mathcal{H}_M \otimes \mathcal{H}_{O,\mathrm{rest}},
$$
where an orthogonal basis $\{ \lvert M_j \rangle \}$ of $\mathcal{H}_M$ realizes classical measurement outcome memory. Decoherence theory and environment-induced superselection (einselection) show that for macroscopic apparatuses, there exists a set of ``pointer states'' that remain robust under environmental monitoring, playing the role of classical records in actual measurements.

\begin{assumption}[Finite Information Capacity (A2)]
The observer record subspace $\mathcal{H}_M$ has finite dimension; their knowledge and memory of the universe must be encoded on a finite number of ``distinguishable records'' $\lvert M_j \rangle$. We take $\log_2 \dim \mathcal{H}_M$ as the upper bound on classical information the observer can store.
\end{assumption}

\subsection{Measurement Interaction and Decoherence}

Consider the measured system's initial state
$$
\lvert \psi_S \rangle = \sum_{k} \alpha_k \lvert s_k \rangle_S,
$$
where $\{ \lvert s_k \rangle \}$ is the eigenbasis of the observable to be measured (or corresponding pointer state basis). Observer and environment initially are in some reference state
$$
\lvert \Psi_0 \rangle = \lvert \psi_S \rangle \otimes \lvert M_{\mathrm{ready}} \rangle_O \otimes \lvert E_0 \rangle_E.
$$

The measurement process is realized by a segment of local unitary evolution on the QCA, abstractly representable as a unitary operator $U_{\mathrm{meas}}$ supported on $\Lambda_S \cup \Lambda_O \cup \Lambda_{E,\mathrm{near}}$ within finite time. Its ideal form satisfies
$$
U_{\mathrm{meas}} \bigl( \lvert s_k \rangle_S \otimes \lvert M_{\mathrm{ready}} \rangle_O \otimes \lvert E_0 \rangle_E \bigr)
= \lvert s_k \rangle_S \otimes \lvert M_k \rangle_O \otimes \lvert \tilde{E}_k \rangle_E.
$$

Linear extension to superposition states yields pre- and post-measurement states
$$
\lvert \Psi_{\mathrm{pre}} \rangle = \sum_k \alpha_k \lvert s_k \rangle_S \otimes \lvert M_{\mathrm{ready}} \rangle_O \otimes \lvert E_0 \rangle_E,
$$
$$
\lvert \Psi_{\mathrm{post}} \rangle = U_{\mathrm{meas}} \lvert \Psi_{\mathrm{pre}} \rangle
= \sum_k \alpha_k \lvert s_k \rangle_S \otimes \lvert M_k \rangle_O \otimes \lvert \tilde{E}_k \rangle_E.
$$

Subsequently, under QCA global evolution, the environment continues to couple with the $S$--$O$ composite, producing further decoherence, causing environmental states of different $k$ branches to become nearly orthogonal:
$$
\langle \tilde{E}_k(t) \vert \tilde{E}_\ell(t) \rangle \approx \delta_{k\ell},
$$
while maintaining robustness of pointer states $\lvert M_k \rangle$. Thus, for observer-local observables, the global pure state of system--apparatus--environment is statistically indistinguishable from a classical mixed state.

\subsection{Probability Axioms and Symmetry Assumptions}

To structurally derive specific probability weights, we need several general requirements on ``how observers distribute subjective probabilities among branches''.

\begin{assumption}[Non-contextuality (A3)]
Let $\{ P_k \}$ be a mutually exclusive complete set of projection operators in the subspace $\mathcal{H}_S \otimes \mathcal{H}_M$ with $\sum_k P_k = I$. The probability $p_k$ that an observer assigns to outcome $k$ depends only on the projection of the current state onto $P_k$, not on how this measurement is embedded in a larger Hilbert space. This assumption is identical to the non-contextuality requirement for probability measures in Gleason's theorem.
\end{assumption}

\begin{assumption}[Environment Insensitivity (A4)]
For any measurement scheme, as long as the system--observer reduced state $\rho_{S O}$ is unchanged, the observer's probability distribution over outcomes does not change due to changes in the environment state. This assumption is consistent with the envariance derivation and the Everett framework principle that ``pure environment transformations should not affect local probabilities''.
\end{assumption}

\begin{assumption}[Finite Rationality and Continuity (A5)]
For a given measurement basis $\{ \lvert s_k \rangle \}$, if state $\lvert \psi \rangle$ is replaced by a small-norm perturbation $\lvert \psi' \rangle$, reasonable probability assignments $p_k(\psi)$ and $p_k(\psi')$ should not undergo discontinuous jumps. Formally, $p_k$ is a continuous function of the state.
\end{assumption}

\begin{assumption}[No Superluminal Signaling Constraint (A6)]
The combination of QCA dynamics and measurement rules must not allow classical information transmission faster than light using entanglement and local operations. This principle has been used in extensive literature to constrain nonlinear quantum evolution and unconventional probability rules.
\end{assumption}

Through these structural assumptions, this paper will derive the Born rule within the QCA universe and discuss its uniqueness.

\section{Main Results (Theorems and Statements)}

For clarity of presentation, this section first states the main theorems and propositions, with proof details in later sections and appendices.

\begin{theorem}[Relative State Structure of QCA Measurement]
In a QCA universe satisfying A1--A2, consider any finite-dimensional measured system $S$ and observer memory register $M$, along with a local unitary operator $U_{\mathrm{meas}}$ implementing ideal measurement. Then for any initial state
$$
\lvert \psi_S \rangle = \sum_k \alpha_k \lvert s_k \rangle_S, \quad
\lvert \Psi_0 \rangle = \lvert \psi_S \rangle \otimes \lvert M_{\mathrm{ready}} \rangle_O \otimes \lvert E_0 \rangle_E,
$$
the global state produced by measurement and subsequent decoherence can be written in Schmidt decomposition form
$$
\lvert \Psi_{\mathrm{post}} \rangle
= \sum_k \alpha_k \lvert s_k \rangle_S \otimes \lvert M_k \rangle_O \otimes \lvert E_k \rangle_E,
$$
where $\{ \lvert E_k \rangle \}$ are nearly orthogonal in the environment Hilbert space. The observer's subjective ``which branch they are in'' can be modeled as self-locating uncertainty over this set of labeled terms.
\end{theorem}

\begin{theorem}[Equal-Amplitude Branches and Equal Probability in Discrete QCA]
Under Theorem 1's setup, if $\lvert \alpha_k \rvert^2 = N_k / N$ are rational numbers with $N = \sum_k N_k$ a positive integer, then there exists a fine-grained decomposition of the environment Hilbert space such that
$$
\lvert \Psi_{\mathrm{post}} \rangle
= \frac{1}{\sqrt{N}}
\sum_{k} \sum_{j=1}^{N_k}
\lvert s_k \rangle_S \otimes \lvert M_k \rangle_O \otimes \lvert \varepsilon_{k,j} \rangle_E,
$$
where all $\lvert \varepsilon_{k,j} \rangle$ are mutually orthogonal, and each term has the same complex amplitude modulus $1/\sqrt{N}$. If we further assume:
\begin{enumerate}
\item Each microscopic configuration $\lvert s_k, M_k, \varepsilon_{k,j} \rangle$ has equal ontological weight (A1);
\item Any permutation of the environment subspace is physically envariant, with corresponding branches equivalent from the observer's perspective;
\end{enumerate}
then the observer's subjective probability of self-locating in a branch carrying record $M_k$ is
$$
p_k = \frac{N_k}{N} = \lvert \alpha_k \rvert^2.
$$
\end{theorem}

\begin{theorem}[Continuum Limit and General Born Rule]
Under assumptions A1--A5, generalizing Theorem 2's rational ratios $N_k/N$ to general complex amplitudes: for any normalized state
$$
\lvert \psi_S \rangle = \sum_k \alpha_k \lvert s_k \rangle,
$$
there exists a sequence of rational numbers $\{ N_k^{(n)}/N^{(n)} \}$ such that $N_k^{(n)}/N^{(n)} \to \lvert \alpha_k \rvert^2$. For each $n$, construct corresponding equal-amplitude QCA microscopic states according to Theorem 2, and set probability assignment $p_k^{(n)} = N_k^{(n)}/N^{(n)}$. Continuity assumption A5 guarantees the limit $p_k = \lim_{n\to\infty} p_k^{(n)}$ exists and satisfies
$$
p_k = \lvert \alpha_k \rvert^2.
$$
Therefore, the Born rule holds for all pure states.
\end{theorem}

\begin{theorem}[Gleason-Type Uniqueness and Non-contextuality]
On local Hilbert spaces of dimension at least three, assuming probabilities $p(P)$ assigned to each projection operator $P$ satisfy:
\begin{enumerate}
\item Non-negativity and normalization;
\item If $\{ P_i \}$ are mutually orthogonal with $\sum_i P_i = I$, then $\sum_i p(P_i) = 1$;
\item Non-contextuality (A3);
\end{enumerate}
then there exists a unique density operator $\rho$ such that $p(P) = \mathrm{tr}(\rho P)$. This is Gleason's theorem statement. In a QCA universe, combined with A4--A6, we can take $\rho$ as the system--observer reduced state, uniquely selecting Born-type probabilities $p_k = \lvert \alpha_k \rvert^2$, and ruling out all unconventional probability rules incompatible with the no-signaling principle.
\end{theorem}

\begin{theorem}[Effective Collapse and Information Horizon]
In the above framework, the global QCA state always evolves by unitary operators; there is no physical nonlinear collapse. However, for any local observer and their accessible operator algebra, replacing the global state
$$
\lvert \Psi_{\mathrm{post}} \rangle
= \sum_k \alpha_k \lvert s_k \rangle \otimes \lvert M_k \rangle \otimes \lvert E_k \rangle
$$
with the conditionalized state
$$
\lvert \Psi_{\mathrm{eff}}^{(k)} \rangle
= \lvert s_k \rangle \otimes \lvert M_k \rangle \otimes \lvert E_k \rangle
$$
and using $p_k = \lvert \alpha_k \rvert^2$ as weight, is completely equivalent for any future experimental observable statistical consequence. This ``effective collapse'' can be interpreted as Bayesian updating performed by the observer within their information horizon, not a change in global dynamics.
\end{theorem}

\section{Proofs}

This section provides proof outlines for Theorems 1--5, with more detailed constructions and technical details in the appendices.

\subsection{Theorem 1: Relative State Structure in QCA}

\textbf{Proof outline:}

1. \textbf{Realization of measurement unitarity.} Under the QCA framework, any finite-dimensional system--apparatus composite can realize the required unitary $U_{\mathrm{meas}}$ by splicing local gate arrays within a finite spacetime block, consistent with universality results for implementing arbitrary finite-dimensional unitaries in finite quantum circuits.

2. \textbf{Schmidt decomposition and decoherence.} For a bipartite system $\mathcal{H}_{SO} \otimes \mathcal{H}_E$, any pure state has a Schmidt decomposition. Appropriately choosing pointer basis $\{ \lvert M_k \rangle \}$ after measurement, we can write $\lvert \Psi_{\mathrm{post}} \rangle$ as
$$
\lvert \Psi_{\mathrm{post}} \rangle = \sum_k \alpha_k \lvert s_k \rangle \otimes \lvert M_k \rangle \otimes \lvert E_k \rangle,
$$
where $\lvert E_k \rangle$ become nearly orthogonal through environment--system interaction in short time. Decoherence theory and model calculations show that at macroscopic apparatus scales, pointer state interference terms rapidly decay on observable timescales, with negligible contribution to subsequent dynamics.

3. \textbf{Relative states and self-locating uncertainty.} Everett's relative state interpretation holds that after measurement, the global state is a set of labeled branches, each label corresponding to an observer record. Although the observer is ``intrinsic'' to one branch, they can view their uncertainty as self-locating uncertainty of ``which branch am I in'' during the time window between branch formation and record reading. This structure requires no additional assumptions, arising directly from QCA unitary evolution and pointer states selected by decoherence.

Thus Theorem 1 is proved. $\square$

\subsection{Theorem 2: Equal-Amplitude Branches and Microscopic Counting}

Theorem 2 is the combinatorial core of the Born rule.

\textbf{Step 1: Rational amplitude squared and environment subdivision.}

Assume $\lvert \alpha_k \rvert^2 = N_k / N$ where $N_k$ and $N$ are integers with $N = \sum_k N_k$. In Hilbert space, we can introduce an auxiliary environment subspace, splitting each term $\alpha_k \lvert s_k, M_k, E_k \rangle$ into $N_k$ equal-amplitude orthogonal states:
$$
\alpha_k \lvert s_k, M_k, E_k \rangle
= \sqrt{\frac{N_k}{N}} \mathrm{e}^{\mathrm{i}\theta_k} \lvert s_k, M_k, E_k \rangle
= \frac{1}{\sqrt{N}}
\sum_{j=1}^{N_k} \mathrm{e}^{\mathrm{i}\theta_k} \lvert s_k, M_k, \varepsilon_{k,j} \rangle,
$$
where $\{ \lvert \varepsilon_{k,j} \rangle \}_{j=1}^{N_k}$ are orthogonal in the environment Hilbert space, and for all $k$ and $j$ form part of an orthogonal basis. Such decomposition can always be realized by extending environment dimension, corresponding in the QCA picture to encoding microscopic labels on additional cells or internal degrees of freedom.

Summing over all $k$:
$$
\lvert \Psi_{\mathrm{post}} \rangle
= \frac{1}{\sqrt{N}}
\sum_{k} \sum_{j=1}^{N_k} \mathrm{e}^{\mathrm{i}\theta_k} \lvert s_k, M_k, \varepsilon_{k,j} \rangle.
$$

Overall phases $\mathrm{e}^{\mathrm{i}\theta_k}$ are irrelevant for probability and can be ignored.

\textbf{Step 2: Envariance and equal probability.}

Consider a given $k$. Within the environment Hilbert space, any permutation of $\{ \lvert \varepsilon_{k,j} \rangle \}_{j=1}^{N_k}$ can be realized by an environment unitary $V_k$. Define the joint unitary transformation
$$
V = \bigotimes_{k} V_k
$$
acting on the environment while keeping system--observer identity, then the global state is formally invariant:
$$
V \lvert \Psi_{\mathrm{post}} \rangle = \lvert \Psi_{\mathrm{post}} \rangle.
$$

This is precisely Zurek's environment-assisted invariance (envariance): when system--observer and environment are perfectly entangled, certain transformations on the environment can ``compensate'' transformations on the system, leaving the overall state invariant, forcing the observer to assign equal probabilities to related branches.

In this case, for any fixed $k$, the only difference among various $j$ labels lies in the environment state $\lvert \varepsilon_{k,j} \rangle$, which is inaccessible to the observer. If the observer assigns different subjective probabilities to these microscopic differences, this would cause probability changes under operations that only alter the environment without changing system--observer observables, directly contradicting A4 (environment insensitivity). Therefore, the only choice satisfying envariance and A4 is: assign equal probability across all $j$ under the same $k$.

Let $q_{k,j}$ be the probability of observer self-location in microscopic branch $(k,j)$. Then for any fixed $k$:
$$
q_{k,1} = q_{k,2} = \cdots = q_{k,N_k} \equiv q_k.
$$

The overall normalization condition is
$$
\sum_{k} \sum_{j=1}^{N_k} q_{k,j} = 1
\quad \Rightarrow \quad
\sum_k N_k q_k = 1.
$$

By symmetry and A1's ``equal ontological weight'', we can directly take $q_{k,j} = 1/N$. Combining these:
$$
q_k = \frac{1}{N},\quad p_k \equiv \sum_{j=1}^{N_k} q_{k,j} = \frac{N_k}{N}.
$$

Thus
$$
p_k = \lvert \alpha_k \rvert^2.
$$
Theorem 2 is proved. $\square$

\subsection{Theorem 3: Continuous Extension to General Amplitudes}

For any pure state $\lvert \psi_S \rangle = \sum_k \alpha_k \lvert s_k \rangle$, let $r_k = \lvert \alpha_k \rvert^2$ satisfying $\sum_k r_k = 1$. Since rationals are dense in reals, we can construct a rational approximation sequence
$$
r_k^{(n)} = \frac{N_k^{(n)}}{N^{(n)}} \to r_k,
\quad
\sum_k r_k^{(n)} = 1
$$
(for example, take $r_k^{(n)}$ as $r_k$ rounded to $n$ decimal places and renormalized). For each $n$, construct equal-amplitude microscopic states and probability assignment $p_k^{(n)} = r_k^{(n)}$ according to Theorem 2. Continuity assumption A5 requires: when $\lVert \psi^{(n)} - \psi \rVert \to 0$, we have $p_k^{(n)} \to p_k$. Since $r_k^{(n)} \to r_k$:
$$
p_k = \lim_{n\to\infty} p_k^{(n)} = \lim_{n\to\infty} r_k^{(n)} = r_k = \lvert \alpha_k \rvert^2.
$$

Therefore, the Born rule holds in general. $\square$

\subsection{Theorem 4: Gleason-Type Uniqueness and No-Signaling}

Theorems 2 and 3 provide constructive derivations of the Born rule for specific measurement scenarios. To demonstrate uniqueness, we invoke Gleason-type results and no-signaling constraints.

Gleason's theorem shows: in Hilbert spaces of dimension at least three, if probability measure $\mu(P)$ assigned to projection operator $P$ satisfies:
\begin{enumerate}
\item $\mu(P) \ge 0$ and $\mu(I) = 1$;
\item For any countable family of mutually orthogonal projections $\{ P_i \}$: $\mu(\sum_i P_i) = \sum_i \mu(P_i)$;
\end{enumerate}
then there must exist a unique density operator $\rho$ such that $\mu(P) = \mathrm{tr}(\rho P)$.

In the QCA universe, measurement probabilities satisfying non-contextuality and additivity are uniquely given by the Born rule via Gleason's theorem. Combined with no-signaling constraints, any alternative probability rule violates either non-contextuality or causality. $\square$

\subsection{Theorem 5: Effective Collapse}

The argument for Theorem 5 is essentially a rigorous formulation of the ``effective collapse'' idea from decoherence theory in the QCA context. Due to decoherence-induced orthogonality of environment states in different branches, local observables cannot distinguish between the full superposition and an effective mixture. This justifies treating collapse as Bayesian updating within the observer's information horizon. $\square$

\section{Model Applications}

\subsection{Spin Measurement: Stern--Gerlach Experiment}

Consider a spin-$1/2$ particle modeled as a local cell group $S$ on the QCA, with internal degrees of freedom realizing spin. Initial state:
$$
\lvert \psi_S \rangle = \alpha \lvert \uparrow_z \rangle + \beta \lvert \downarrow_z \rangle.
$$

Measurement apparatus $O$ contains a switchable magnetic field and screen, with pointer states $\lvert M_\uparrow \rangle, \lvert M_\downarrow \rangle$ corresponding to ``deflected up'' and ``deflected down'' macroscopic outcomes. Post-measurement:
$$
\lvert \Psi_{\mathrm{post}} \rangle
= \alpha \lvert \uparrow_z \rangle \otimes \lvert M_\uparrow \rangle \otimes \lvert E_\uparrow \rangle
+ \beta \lvert \downarrow_z \rangle \otimes \lvert M_\downarrow \rangle \otimes \lvert E_\downarrow \rangle.
$$

In QCA ontological basis fine-graining, we write $\alpha, \beta$ as square roots of degeneracies, yielding by Theorems 2--3:
$$
p(\uparrow_z) = \lvert \alpha \rvert^2,\quad p(\downarrow_z) = \lvert \beta \rvert^2,
$$
consistent with standard quantum mechanics.

\subsection{Interference Experiments: Double-Slit and Which-Way Information}

In double-slit interference, the QCA evolution creates interference patterns on the screen when no which-way measurement is performed. Introducing path detectors encodes which-way information into observer records, environment decoherence suppresses coherence terms, destroying interference. This demonstrates that probability always arises from microscopic path degeneracy; whether interference appears depends only on whether which-way information is encoded on the QCA by observer and environment.

\subsection{Repeated Measurements and Frequency Law}

For $n$ independent measurements of identically prepared states, the global state contains $\binom{n}{m}$ microscopic branches for each frequency $m/n$, with degeneracy:
$$
N_m \propto \binom{n}{m} \lvert \alpha \rvert^{2m} \lvert \beta \rvert^{2(n-m)}.
$$

The observer's self-location probability in ``seeing $m$ outcomes of type 0'' is proportional to $N_m$, following a binomial distribution. Combined with the law of large numbers, in most branches the observed frequency $m/n$ approaches $\lvert \alpha \rvert^2$, naturally unifying frequency and combinatorial interpretations in the QCA universe.

\section{Engineering Proposals}

\subsection{Simulating QCA Measurements on Controllable Quantum Platforms}

Current superconducting qubit, ion trap, and photonic platforms can already implement one- and two-dimensional quantum walks and simplified QCA experimental simulations. One can design schemes encoding part of the qubits as measured system $S$, others as measurement register $O$ and environment $E$, implementing measurement coupling $U_{\mathrm{meas}}$ through programmable unitary circuits and controlled decoherence channels, verifying that local statistics depend only on $\lvert \psi \rvert^2$ independent of specific environment implementation, indirectly verifying A4 and envariance principles.

\subsection{Testing Constraints on Unconventional Probability Rules}

Numerous experimental schemes aim to detect small deviations from the Born rule through high-precision interference and quantum optical statistics. Combined with analyses by Gisin and others on nonlinear quantum mechanics and superluminal signaling, one can design QCA-based tests: construct spatially separated entangled pairs, perform different measurement basis choices on one side, high-precision statistical measurements on the other, and use QCA locality analysis to determine whether any visible deviation from Born distributions would cause effective propagation speeds exceeding Lieb--Robinson velocity, imposing new constraints on unconventional probability rules.

\subsection{Finite Observer Models and Quantum Information Compression}

Finite information capacity assumption A2 has natural realization in quantum information theory: observer ``memory'' can be viewed as finite-dimensional registers, with any environmental knowledge requiring storage through quantum compression or classical encoding. Combining this probability emergence framework with reversible compression and entanglement-assisted encoding techniques may introduce explicit ``observer'' modules in practical quantum communication or computation protocols, analyzing how experimental results are reinterpreted under different information horizons.

\section{Discussion (Risks, Boundaries, Past Work)}

\subsection{Comparison with Existing Born Rule Derivations}

This derivation route has significant formal similarity to Zurek's envariance scheme: both use environment subdivision and symmetry arguments for equal-amplitude superposition states to obtain the Born rule for rational amplitudes, then extend to general cases through continuity. Unlike the Deutsch--Wallace decision-theoretic route, we do not explicitly invoke utility functions and rationality axioms, instead taking ``self-locating uncertainty of finite-information observers'' and ``equal ontological weight'' as core intuitions.

Compared to pure Hilbert space abstract derivations, the QCA framework's main added value is:
\begin{enumerate}
\item Providing an explicit ontological basis and microscopic configuration counting method, giving ``microscopic degeneracy of equal-amplitude states'' a concrete discrete realization;
\item Linking the no-signaling principle to Lieb--Robinson velocity, explaining at the microscopic level why unconventional probability rules are generally incompatible with local QCA dynamics;
\item Providing a vehicle for physical interpretation of ``finite-information observers'': memory registers, pointer states, and environment can all be concretely realized on the QCA.
\end{enumerate}

\subsection{Potential Risks and Strength of Assumptions}

Despite structural closure of the above derivation, several boundaries and risks merit emphasis:

1. \textbf{Status of equal ontological weight assumption.} A1 requires all ontological basis states to have equal basic weight. This assumption is often viewed as natural in QCA and cellular automaton interpretations, but can still be questioned in broader foundational discussions. Its physical basis can be defended from QCA local rule translation invariance and absence of preferred basis.

2. \textbf{Non-contextuality and environment insensitivity.} Gleason-type derivations show non-contextuality is key to connecting Hilbert space structure and Born rule, but this requirement is viewed as a strong assumption in some interpretations. This paper argues for A4's reasonableness through envariance and QCA locality, but whether weaker physical principles suffice remains open.

3. \textbf{Dimension and POVM generalization.} Gleason's theorem requires additional assumptions or POVM generalization for two-dimensional Hilbert spaces. In QCA universes, effective system--observer Hilbert spaces are typically high-dimensional, so this technical limitation has limited impact on macroscopic measurements, but requires care when analyzing single-qubit systems.

4. \textbf{Usage of no-signaling.} Gisin and subsequent work show many nonlinear modifications lead to superluminal signaling, but carefully constructed unconventional schemes attempt to circumvent this conclusion. In QCA universes, due to rigid structure of discrete lattices and local updates, feasibility of these circumvention schemes awaits further analysis.

\subsection{Decoherence Schemes and Remaining Parts of the Measurement Problem}

Extensive literature points out that decoherence, while explaining emergence of classical observable structure, does not alone solve the measurement problem. This paper's contribution is: on the foundation of pointer states and branch structure provided by decoherence, introducing QCA microscopic counting and finite observer models, thereby giving probability a combinatorial and information-theoretic origin within the Everett--relative state framework. But from a philosophical perspective, concepts like ``equal ontological weight'' and ``self-locating uncertainty'' themselves have interpretive color; whether they are superior to traditional interpretations requires further comparison.

\section{Conclusion}

In the framework of a discrete, local, unitary quantum cellular automaton universe, this paper provides a combinatorial--information-theoretic derivation of the Born rule, interpreting quantum probability as the self-locating uncertainty of finite-information observers on the global unitary state, rather than a fundamental stochastic process. The core points can be summarized:

1. QCA universe provides ontological foundation for globally deterministic dynamics: each ontological basis state evolves by local cellular rules, with no intrinsic randomness;

2. Measurement process is modeled as local unitary coupling among system--observer--environment; decoherence selects stable branches in pointer state space, giving the global state a definite relative state structure;

3. Under ``equal ontological weight'', envariance, and environment insensitivity assumptions, equal-amplitude superposition branches are equiprobable from finite observer perspective; microscopic configuration degeneracy naturally corresponds to amplitude modulus squared;

4. Through rational approximation and continuity, equal-amplitude cases generalize to general amplitudes, yielding Born rule $p_k = \lvert \alpha_k \rvert^2$;

5. Combined with Gleason's theorem and no-signaling constraints, we argue that in QCA universes, the Born rule is the only probability assignment compatible with Hilbert space structure, locality, and causality;

6. Wavefunction collapse is reinterpreted as Bayesian updating and self-location by the observer within their information horizon, not a genuine mutation of the global state.

In this sense, nature does not ``roll dice'' at the microscopic level; the appearance of probability is a statistical necessity from the perspective of finite observers in a discrete holographic entanglement network.

\section*{Acknowledgements}

This research is based on comprehensive synthesis and extension of existing work on quantum cellular automata, decoherence theory, and quantum probability foundations. We deeply appreciate related literature and previous contributions.

All derivations in this paper are analytical, not relying on numerical simulations. Small-scale QCA measurement models on specific quantum platforms can be constructed using general quantum circuit compilers, with details left for future work.

\begin{thebibliography}{99}

\bibitem{zurek2003} W. H. Zurek, ``Environment-assisted invariance, entanglement, and probabilities in quantum physics,'' \textit{Phys. Rev. Lett.} \textbf{90}, 120404 (2003).

\bibitem{zurek2005} W. H. Zurek, ``Probabilities from entanglement, Born's rule ($p_k = \lvert \psi_k \rvert^2$) from envariance,'' \textit{Phys. Rev. A} \textbf{71}, 052105 (2005).

\bibitem{schlosshauer2007} M. Schlosshauer, \textit{Decoherence and the Quantum-to-Classical Transition}, Springer (2007).

\bibitem{joos2003} E. Joos et al., \textit{Decoherence and the Appearance of a Classical World in Quantum Theory}, Springer (2003).

\bibitem{gleason1957} A. M. Gleason, ``Measures on the closed subspaces of a Hilbert space,'' \textit{J. Math. Mech.} \textbf{6}, 885--893 (1957).

\bibitem{logiurato2012} F. Logiurato and A. Smerzi, ``Born Rule and Noncontextual Probability,'' \textit{J. Mod. Phys.} \textbf{3}, 1930--1939 (2012).

\bibitem{deutsch1999} D. Deutsch, ``Quantum theory of probability and decisions,'' \textit{Proc. R. Soc. Lond. A} \textbf{455}, 3129--3137 (1999).

\bibitem{sebens2018} C. T. Sebens and S. M. Carroll, ``Self-locating Uncertainty and the Origin of Probability in Everettian Quantum Mechanics,'' \textit{Br. J. Philos. Sci.} \textbf{69}, 25--74 (2018).

\bibitem{vaidman2019} L. Vaidman, ``Derivations of the Born Rule,'' \textit{arXiv:1907.03623} (2019).

\bibitem{thooft2016} G. 't Hooft, \textit{The Cellular Automaton Interpretation of Quantum Mechanics}, Springer (2016).

\bibitem{dariano2017} G. M. D'Ariano and P. Perinotti, ``Quantum cellular automata and free quantum field theory,'' \textit{Front. Phys.} \textbf{12}, 120301 (2017).

\bibitem{raynal2017} P. Raynal et al., ``Simple derivation of the Weyl and Dirac quantum cellular automata,'' \textit{Phys. Rev. A} \textbf{95}, 062344 (2017).

\bibitem{gisin1990} N. Gisin, ``Weinberg's non-linear quantum mechanics and superluminal communications,'' \textit{Phys. Lett. A} \textbf{143}, 1--2 (1990).

\bibitem{simon2001} C. Simon, V. Buzek, and N. Gisin, ``The no-signaling condition and quantum dynamics,'' \textit{Phys. Rev. Lett.} \textbf{87}, 170405 (2001).

\bibitem{schlosshauer2004} M. Schlosshauer, ``Decoherence, the measurement problem, and interpretations of quantum mechanics,'' \textit{Rev. Mod. Phys.} \textbf{76}, 1267--1305 (2004).

\bibitem{bacciagaluppi2011} G. Bacciagaluppi, ``The Role of Decoherence in Quantum Mechanics,'' \textit{Stanford Encyclopedia of Philosophy} (2003/2011).

\bibitem{saunders2010} S. Saunders et al. (eds.), \textit{Many Worlds? Everett, Quantum Theory, and Reality}, Oxford University Press (2010).

\bibitem{earman2022} J. Earman, ``The Status of the Born Rule and the Role of Gleason's Theorem,'' \textit{PhilSci Archive} (2022).

\end{thebibliography}

\end{document}

