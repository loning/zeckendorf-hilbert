\documentclass[11pt,a4paper]{article}
\usepackage[utf8]{inputenc}
\usepackage{amsmath,amssymb,amsthm}
\usepackage{mathrsfs}
\usepackage{geometry}
\geometry{margin=1in}
\usepackage{hyperref}
\usepackage{braket}

\newtheorem{theorem}{Theorem}[section]
\newtheorem{proposition}[theorem]{Proposition}
\newtheorem{corollary}[theorem]{Corollary}
\newtheorem{definition}[theorem]{Definition}
\newtheorem{axiom}[theorem]{Axiom}

\title{Universal Conservation of Information Rate:\\
Unification from Quantum Cellular Automata to Relativity, Mass and Gravity}

\author{Anonymous Author}
\date{\today}

\begin{document}
\maketitle

\begin{abstract}
Within quantum cellular automaton (QCA) and finite information ontology framework, we introduce unified axiom: for any local excitation, its external group velocity $v_{\mathrm{ext}}$ and internal state evolution velocity $v_{\mathrm{int}}$ satisfy information rate conservation
$$
v_{\mathrm{ext}}^{2}+v_{\mathrm{int}}^{2}=c^{2}.
$$
Defining proper time $\tau$ by internal evolution parameter, can directly derive special relativity's time dilation, four-velocity normalization and Minkowski line element from this axiom. In continuum limit of linear Dirac-type QCA, internal Hamiltonian $H_{\mathrm{int}}$ gives internal frequency $\omega_{\mathrm{int}}$, mass obtains information-theoretic definition
$$
m c^{2}=\hbar\omega_{\mathrm{int}},
$$
satisfying Zitterbewegung frequency relation
$$
\omega_{\mathrm{ZB}}=2\omega_{\mathrm{int}}.
$$
Combining QCA's winding number and index invariants, massive excitations can be interpreted as optical path quota bound in topologically nontrivial self-referential loops. At many-body level, introducing local information processing density $\rho_{\mathrm{info}}(x)$, from local information volume conservation derives optical metric
$$
ds^{2}=-\eta^{2}(x)c^{2}dt^{2}+\eta^{-2}(x)\gamma_{ij}(x)dx^{i}dx^{j},
$$
where $\eta(x)$ determines local effective light speed
$$
c_{\mathrm{eff}}(x)=\eta^{2}(x)c,
$$
and refractive index
$$
n(x)=\eta^{-2}(x).
$$
In weak-field limit, this structure recovers first-order expansion of Schwarzschild metric and standard light deflection angle, and through information--gravity variational principle obtains field equation formally equivalent to Einstein equation. Further introducing information mass $M_{I}$, combining Landauer principle to analyze asymptotic rest behavior and minimal dissipation power of high information mass subjects, giving unified information-theoretic characterization of mass, gravity and complex energetic structures, proposing testable predictions based on superconducting quantum circuits and quantum simulation platforms.
\end{abstract}

\textbf{Keywords:} Quantum cellular automaton; Information rate conservation; Optical metric; Special relativity; General relativity; Topological mass; Zitterbewegung; Information mass; Landauer principle

\section{Introduction \& Historical Context}

Special and general relativity characterize universe as four-dimensional manifold $(M,g_{\mu\nu})$ with Lorentz signature. Metric tensor $g_{\mu\nu}$ determines spacetime causal structure and geodesics, field equation
$$
R_{\mu\nu}-\tfrac12 R g_{\mu\nu}=8\pi G T_{\mu\nu}
$$
connects stress--energy tensor $T_{\mu\nu}$ with curvature. Experimental tests such as gravitational redshift, light deflection and gravitational wave detection highly support this geometric narrative.

Quantum theory is formulated in Hilbert space $\mathcal H$, states as vectors or density operators, observables as self-adjoint operators, time evolution generated by unitary groups or semigroups. Statistical interpretation built on Born rule, with superposition, phase and entanglement forming core structure. Two theories splice in quantum field theory through ``field operators defined on background manifold'', but ontological starting points remain separate: one side is deformable geometric stage, other side is linear space of probability amplitudes.

When approaching Planck scale, continuity assumption of manifold and classical metric loses empirical support, while Hilbert space structure itself does not depend on continuous spacetime. This motivates ontology based on discrete, finite information structure as natural candidate. Quantum cellular automata (QCA) define local, causal and unitary evolution on countable lattice points, proven to emerge Dirac, Weyl and Maxwell field equations in appropriate limits, with complete structural and classification theory, serving as precise operator realization of ``universe as quantum computation'' framework [see references [1,2]].

On other hand, gravitational lensing theory widely uses ``optical metric'' geometric method, viewing light rays as geodesics in optical geometry, calculating deflection angles through Gauss--Bonnet theorem, establishing equivalence between ``gravity bending light'' and ``geometric optics in non-uniform refractive index media'' [see references [3,4]].

Information thermodynamics connects logical irreversibility with heat dissipation through Landauer principle: erasing one bit of information in thermal bath at temperature $T$ dissipates at least heat $k_{B}T\ln 2$, showing information update is resource consumption process constrained by physical laws [see reference [5]].

This paper proposes: in unified perspective of QCA and finite information ontology, can use ``information rate conservation'' as sole microscopic axiom. Each local excitation in universe possesses finite ``information rate budget'' at each microscopic time step, this budget corresponding to maximal propagation velocity $c$, allocated in Pythagorean manner between ``displacement'' (external motion) and ``internal evolution'' (local Hilbert space self-referential update). This paper will prove this axiom sufficient to derive special relativistic geometry, mass--frequency relation, optical structure of weak-field gravity, and basic constraints on information mass and dissipation, thereby providing unified information-theoretic characterization of mass, gravity and complex energetic structures.

\section{Model \& Assumptions}

\subsection{QCA Universe and Finite Information}

Let $\Lambda$ be countable connected graph, nodes representing ``spatial cells''. Each node $x\in\Lambda$ carries finite-dimensional Hilbert space $\mathcal H_{x}\simeq\mathbb C^{d}$. For any finite subset $F\Subset\Lambda$ define local Hilbert space
$$
\mathcal H_{F}=\bigotimes_{x\in F}\mathcal H_{x},
$$
local operator algebra is $\mathcal B(\mathcal H_{F})$. Global quasilocal $C^{\ast}$ algebra is
$$
\mathcal A=\overline{\bigcup_{F\Subset\Lambda}\mathcal B(\mathcal H_{F})}.
$$

Quantum cellular automaton specified by $\ast$-automorphism $\alpha:\mathcal A\to\mathcal A$, requiring existence of unitary operator $U$ such that
$$
\alpha(A)=U^{\dagger}AU,\qquad A\in\mathcal A,
$$
with finite propagation radius $R<\infty$ such that for any local operator $A$ supported on $F$ have
$$
\mathrm{supp}\,\alpha(A)\subset B_{R}(F),
$$
where $B_{R}(F)$ is radius $R$ neighborhood of $F$ in graph distance sense. Given initial state $\omega_{0}$, discrete time evolution is
$$
\omega_{n}=\omega_{0}\circ\alpha^{n},\qquad n\in\mathbb Z.
$$

Assume $\Lambda$ can embed into three-dimensional Euclidean space with effective lattice spacing $a$, single-step evolution corresponding to physical time $\Delta t$. If $R=1$, maximal propagation rate is
$$
c=\frac{a}{\Delta t}.
$$

Finite local dimension and finite propagation radius mean that in any finite spacetime window, number of distinguishable physical states is finite, universe has upper bound on information capacity in any finite region.

\subsection{Single-Excitation Effective Space and External/Internal Velocities}

Consider local ``single-excitation'' mode, in appropriate approximation its effective Hilbert space can be represented as
$$
\mathcal H_{\mathrm{eff}}\simeq\mathcal H_{\mathrm{COM}}\otimes\mathcal H_{\mathrm{int}},
$$
where $\mathcal H_{\mathrm{COM}}$ describes center coordinate or wave packet envelope, $\mathcal H_{\mathrm{int}}$ describes internal degrees of freedom.

In continuum limit, exists approximate position operator $X$ and momentum operator $P$ on $\mathcal H_{\mathrm{COM}}$, effective Hamiltonian $H_{\mathrm{eff}}$ generates coarse-grained time evolution. Define external (group) velocity
$$
v_{\mathrm{ext}}=\frac{d}{dt}\langle X\rangle
=\frac{1}{\mathrm i\hbar}\langle[X,H_{\mathrm{eff}}]\rangle.
$$

Internal state $\ket{\psi_{\mathrm{int}}(t)}\in\mathcal H_{\mathrm{int}}$ can be viewed as point on projective space $\mathbb C P^{D_{\mathrm{int}}-1}$, equipped with Fubini--Study metric
$$
ds_{\mathrm{FS}}^{2}=4\bigl(1-|\langle\psi\mid\psi+d\psi\rangle|^{2}\bigr).
$$
Define internal velocity
$$
v_{\mathrm{int}}:=\frac{ds_{\mathrm{FS}}}{dt}\ge 0.
$$

\subsection{Information Rate Vector and Universal Conservation Axiom}

In two-dimensional ``information rate space''
$$
\mathbb R^{2}_{\mathrm{info}}=\mathrm{span}\{e_{\mathrm{ext}},e_{\mathrm{int}}\}
$$
define information rate vector
$$
\mathbf u=v_{\mathrm{ext}}e_{\mathrm{ext}}+v_{\mathrm{int}}e_{\mathrm{int}},
\qquad
|\mathbf u|^{2}=v_{\mathrm{ext}}^{2}+v_{\mathrm{int}}^{2}.
$$

\begin{axiom}[Information rate conservation]
Exists constant $c>0$ such that for any local excitation and any time $t$ have
$$
v_{\mathrm{ext}}^{2}(t)+v_{\mathrm{int}}^{2}(t)=c^{2}.
$$
\end{axiom}

Define information phase angle
$$
\theta(t):=\arctan\frac{v_{\mathrm{int}}(t)}{v_{\mathrm{ext}}(t)}\in[0,\pi/2].
$$
$\theta=0$ corresponds to lightlike mode, $\theta=\pi/2$ corresponds to completely local internal mode, $0<\theta<\pi/2$ corresponds to massive mode.

\subsection{Proper Time and Internal Evolution}

For any excitation worldline define proper time $\tau$ satisfying
$$
v_{\mathrm{int}}\,dt=c\,d\tau,
$$
i.e.
$$
v_{\mathrm{int}}=c\,\frac{d\tau}{dt}.
$$
Substituting into information rate conservation
$$
v_{\mathrm{ext}}^{2}+v_{\mathrm{int}}^{2}=c^{2},
$$
letting $v:=v_{\mathrm{ext}}=|d\mathbf x/dt|$, obtain
$$
\Bigl(\frac{d\tau}{dt}\Bigr)^{2}=1-\frac{v^{2}}{c^{2}},
$$
taking positive root
$$
\frac{d\tau}{dt}=\sqrt{1-\frac{v^{2}}{c^{2}}},
$$
i.e., special relativity time dilation relation. Proper time can be understood as natural parameter of ``internal information path'', its reduction relative to coordinate time is direct result of external motion occupying optical path quota.

\subsection{Local Information Processing Density and Optical Metric}

In many-body situation, introduce coarse-grained local information processing density $\rho_{\mathrm{info}}(x)$, representing average path length per unit time per unit volume walked by local Hilbert space under Fubini--Study metric, or equivalently, effective expectation value of internal Hamiltonian density. High $\rho_{\mathrm{info}}(x)$ regions can be viewed as regions where ``internal computation is highly active''.

Allow local rescaling of time and space scales in coordinate system $(t,x^{i})$, introducing scale factors $\eta_{t}(x)$, $\eta_{x}(x)$ such that
$$
dt_{\mathrm{eff}}=\eta_{t}(x)\,dt,\qquad
d\ell_{\mathrm{eff}}=\eta_{x}(x)\,d\ell.
$$

Unitarity of underlying QCA means global Hilbert volume preserved under evolution. Localizing this requirement and simplifying to ``physical Hilbert volume corresponding to unit coordinate volume does not change with time'', can model with constraint
$$
\eta_{t}(x)\,\eta_{x}^{3}(x)=1.
$$

In isotropic approximation, taking
$$
\eta_{t}(x)=\eta(x),\qquad
\eta_{x}(x)=\eta^{-1}(x),
$$
four-dimensional line element can be written as
$$
ds^{2}=-\eta^{2}(x)c^{2}dt^{2}+\eta^{-2}(x)\gamma_{ij}(x)dx^{i}dx^{j},
$$
where $\gamma_{ij}(x)$ is three-dimensional spatial metric.

In isotropic case with $\gamma_{ij}=\delta_{ij}$, for null geodesic $ds^{2}=0$ have
$$
0=-\eta^{2}(x)c^{2}dt^{2}+\eta^{-2}(x)\,d\mathbf x^{2},
$$
thus coordinate light speed
$$
\Bigl|\frac{d\mathbf x}{dt}\Bigr|
=\frac{\eta_{t}(x)}{\eta_{x}(x)}\,c
=\eta^{2}(x)c
=:c_{\mathrm{eff}}(x),
$$
equivalent refractive index
$$
n(x):=\frac{c}{c_{\mathrm{eff}}(x)}=\eta^{-2}(x).
$$

High $\rho_{\mathrm{info}}(x)$ means more frequent local internal evolution; by information rate conservation, external propagation is suppressed; therefore smaller $\eta(x)$, smaller local effective light speed $c_{\mathrm{eff}}(x)=\eta^{2}(x)c$, corresponding to larger refractive index $n(x)=\eta^{-2}(x)$. This is compatible with weak-field gravity result
$$
n(r)\simeq 1-\frac{2\phi(r)}{c^{2}}
$$
($\phi<0$ is Newton potential), giving $n(r)>1$ for $\phi(r)=-GM/r$, thus $c_{\mathrm{eff}}(r)<c$.

\section{Main Results (Theorems and Alignments)}

On basis of above model and assumptions, summarize main mathematical results and physical correspondences of this paper.

\subsection{Theorem 1 (Emergence of Special Relativity)}

Under information rate conservation axiom
$$
v_{\mathrm{ext}}^{2}+v_{\mathrm{int}}^{2}=c^{2}
$$
and proper time definition
$$
v_{\mathrm{int}}\,dt=c\,d\tau
$$
have:

\begin{enumerate}
\item Proper time satisfies
$$
\Bigl(\frac{d\tau}{dt}\Bigr)^{2}=1-\frac{v^{2}}{c^{2}},
\qquad
v:=v_{\mathrm{ext}}=\Bigl|\frac{d\mathbf x}{dt}\Bigr|.
$$

\item Four-velocity of worldline $x^{\mu}(\tau)=(ct(\tau),\mathbf x(\tau))$
$$
u^{\mu}=\frac{dx^{\mu}}{d\tau}
=\gamma(v)\,(c,\mathbf v),
\qquad
\gamma(v)=\frac{1}{\sqrt{1-v^{2}/c^{2}}},
$$
satisfies normalization under Minkowski metric $\eta_{\mu\nu}=\mathrm{diag}(-1,1,1,1)$
$$
u^{\mu}u_{\mu}=-c^{2}.
$$

\item Line element
$$
ds^{2}:=-c^{2}d\tau^{2}=-c^{2}dt^{2}+d\mathbf x^{2}
$$
consistent with Minkowski spacetime metric.
\end{enumerate}

Thus special relativistic dynamics appears in this framework as direct corollary of information rate conservation and proper time definition.

\subsection{Theorem 2 (Mass as Internal Frequency)}

On internal Hilbert space $\mathcal H_{\mathrm{int}}$ introduce Hamiltonian
$$
\mathrm i\hbar\,\partial_{\tau}\ket{\psi_{\mathrm{int}}(\tau)}
=H_{\mathrm{int}}\ket{\psi_{\mathrm{int}}(\tau)}.
$$
If exists eigenstate $\ket{\psi_{\mathrm{int}}}$ satisfying
$$
H_{\mathrm{int}}\ket{\psi_{\mathrm{int}}}=E_{0}\ket{\psi_{\mathrm{int}}},
$$
then internal state evolves as
$$
\ket{\psi_{\mathrm{int}}(\tau)}
=\mathrm e^{-\mathrm i E_{0}\tau/\hbar}\ket{\psi_{\mathrm{int}}},
$$
define internal frequency
$$
\omega_{\mathrm{int}}:=\frac{E_{0}}{\hbar}.
$$
Identifying rest energy $E_{0}=m c^{2}$, obtain mass--frequency relation
$$
m=\frac{\hbar\omega_{\mathrm{int}}}{c^{2}}.
$$

\subsection{Proposition 1 (Zitterbewegung Frequency and Internal Frequency)}

In continuum limit of one-dimensional Dirac-type QCA, effective Hamiltonian
$$
H_{\mathrm{eff}}(k)\simeq c\hbar k\,\sigma_{z}
+m c^{2}\sigma_{x},
$$
eigenvalues
$$
E_{\pm}(k)=\pm\sqrt{(c\hbar k)^{2}+m^{2}c^{4}}.
$$
In Heisenberg picture, evolution of position operator $X(t)$ contains rapid oscillation term with frequency
$$
\omega_{\mathrm{ZB}}=\frac{2E}{\hbar}
$$
(Zitterbewegung). In rest limit $k=0$, $E=m c^{2}$, thus
$$
\omega_{\mathrm{ZB}}(0)=\frac{2m c^{2}}{\hbar}=2\omega_{\mathrm{int}}.
$$

\subsection{Theorem 3 (Topological Stability and Nonzero Information Phase Angle)}

Consider one-dimensional translation-invariant QCA, single-step unitary operator $U(k)\in\mathrm U(N)$ defines closed curve in momentum space. Winding number
$$
\mathcal W[U]
=\frac{1}{2\pi\mathrm i}\int_{-\pi/a}^{\pi/a}
\partial_{k}\log\det U(k)\,dk\in\mathbb Z
$$
preserved under finite-depth local unitary transformations.

\begin{enumerate}
\item If $\mathcal W[U]=0$, exists continuous deformation reducing QCA to form containing only massless propagation modes, with internal frequency $\omega_{\mathrm{int}}$ possibly zero, corresponding to $v_{\mathrm{int}}=0$, $\theta=0$.

\item If $\mathcal W[U]\neq 0$, exist local excitations carrying nonzero topological charge, any finite-depth local unitary cannot continuously deform them to topologically trivial vacuum. To maintain topological phase winding, internal Hamiltonian of such excitations must have nonzero eigenfrequency $\omega_{\mathrm{int}}>0$, thus $v_{\mathrm{int}}>0$, $\theta>0$.
\end{enumerate}

Therefore nonzero information phase angle $\theta$ and existence of mass are topologically stabilized by QCA structure, massive excitations can be interpreted as macroscopic manifestation of optical path quota bound by topologically nontrivial self-referential loops.

\subsection{Theorem 4 (Optical Metric and Weak-Field Gravity)}

In isotropic assumption, taking
$$
\eta_{t}(x)=\eta(x),\qquad
\eta_{x}(x)=\eta^{-1}(x),
$$
construct optical metric
$$
ds^{2}=-\eta^{2}(x)c^{2}dt^{2}
+\eta^{-2}(x)\gamma_{ij}(x)dx^{i}dx^{j}.
$$
In weak-field limit
$$
\eta(x)=1+\epsilon(x),\qquad
|\epsilon(x)|\ll 1,
$$
first-order expansion gives
$$
g_{00}\simeq-(1+2\epsilon)c^{2},
\qquad
g_{ij}\simeq(1-2\epsilon)\gamma_{ij}.
$$
In static spherically symmetric case with isotropic coordinates, identifying
$$
\epsilon(r)=\frac{\phi(r)}{c^{2}},
$$
where $\phi(r)$ is Newton potential, then
$$
g_{00}\simeq-\bigl(1+2\phi/c^{2}\bigr)c^{2},
\qquad
g_{ij}\simeq\bigl(1-2\phi/c^{2}\bigr)\delta_{ij},
$$
consistent with first-order expansion of Schwarzschild metric in isotropic coordinates. Refractive index
$$
n(r)=\eta^{-2}(r)\simeq 1-\frac{2\phi(r)}{c^{2}},
$$
if $\phi(r)=-GM/r$, then
$$
n(r)\simeq 1+\frac{2GM}{c^{2}r}>1,
\qquad
c_{\mathrm{eff}}(r)=\frac{c}{n(r)}<c.
$$
Solving for null geodesics under this metric gives light deflection angle
$$
\Delta\theta=\frac{4GM}{c^{2}b},
$$
where $b$ is impact parameter, consistent with standard general relativity result.

\subsection{Theorem 5 (Information--Gravity Variational Principle)}

Consider action
$$
S_{\mathrm{tot}}[g,\rho_{\mathrm{info}}]
=\frac{1}{16\pi G}\int_{M}\sqrt{-g}\,R[g]\,d^{4}x
+\int_{M}\sqrt{-g}\,\mathcal L_{\mathrm{info}}[\rho_{\mathrm{info}},g]\,d^{4}x.
$$
Varying with respect to $g^{\mu\nu}$ (ignoring boundary terms) gives
$$
R_{\mu\nu}-\tfrac12 R g_{\mu\nu}
=8\pi G\,T^{(\mathrm{info})}_{\mu\nu},
$$
where
$$
T^{(\mathrm{info})}_{\mu\nu}
:=-\frac{2}{\sqrt{-g}}\,
\frac{\delta(\sqrt{-g}\,\mathcal L_{\mathrm{info}})}{\delta g^{\mu\nu}}.
$$
If choose $\mathcal L_{\mathrm{info}}$ such that $T^{(\mathrm{info})}_{\mu\nu}$ consistent with standard matter stress--energy tensor in low-energy limit, then this equation formally equivalent to Einstein field equation.

\subsection{Theorem 6 (Information Mass and Asymptotic Rest)}

For systems with internal models and self-referential mechanisms, introduce information mass
$$
M_{I}(\sigma)
=f\bigl(K(\sigma),D(\sigma),S_{\mathrm{ent}}(\sigma)\bigr),
$$
where $K$ is Kolmogorov complexity, $D$ is logical depth, $S_{\mathrm{ent}}$ is internal entanglement entropy, $f$ is monotonically increasing function. Assume average internal information rate $v_{\mathrm{int}}(M_{I})$ needed to maintain given $M_{I}$ is monotonically increasing and bounded by $c$. From information rate conservation get
$$
v_{\mathrm{ext}}^{2}(M_{I})
=c^{2}-v_{\mathrm{int}}^{2}(M_{I}).
$$
If
$$
\lim_{M_{I}\to\infty}v_{\mathrm{int}}(M_{I})=c,
$$
then
$$
\lim_{M_{I}\to\infty}v_{\mathrm{ext}}(M_{I})=0,
$$
i.e., high information mass subjects tend to asymptotic rest in external geometry.

\subsection{Proposition 2 (Landauer Cost of Maintaining Information Mass)}

Let system update internal model at rate $R_{\mathrm{upd}}$, each update erasing average $\Delta I$ bits of old information, then information erasure per unit time
$$
\dot I_{\mathrm{erase}}
=R_{\mathrm{upd}}\Delta I.
$$
In thermal bath at temperature $T$, according to Landauer principle, erasing one bit dissipates at least heat $k_{B}T\ln 2$, minimal power consumption is
$$
P_{\min}
=k_{B}T\ln 2\,\dot I_{\mathrm{erase}}
=k_{B}T\ln 2\,R_{\mathrm{upd}}\Delta I.
$$

\section{Proofs}

This section gives derivation structure of above main results, with detailed calculations in appendices.

\subsection{Proof of Theorem 1}

From proper time definition
$$
v_{\mathrm{int}}\,dt=c\,d\tau
\quad\Rightarrow\quad
v_{\mathrm{int}}=c\,\frac{d\tau}{dt},
$$
substituting into
$$
v^{2}+v_{\mathrm{int}}^{2}=c^{2},
\qquad
v:=v_{\mathrm{ext}},
$$
obtain
$$
v^{2}+c^{2}\Bigl(\frac{d\tau}{dt}\Bigr)^{2}=c^{2},
$$
i.e.
$$
\Bigl(\frac{d\tau}{dt}\Bigr)^{2}
=1-\frac{v^{2}}{c^{2}}.
$$
Taking positive root gives time dilation relation
$$
\frac{d\tau}{dt}
=\sqrt{1-\frac{v^{2}}{c^{2}}}.
$$
Define Lorentz factor
$$
\gamma(v)=\frac{dt}{d\tau}
=\frac{1}{\sqrt{1-v^{2}/c^{2}}},
$$
four-velocity
$$
u^{\mu}=\frac{dx^{\mu}}{d\tau}
=\gamma(v)\,(c,\mathbf v).
$$
Under Minkowski metric $\eta_{\mu\nu}=\mathrm{diag}(-1,1,1,1)$,
$$
u^{\mu}u_{\mu}
=-\gamma^{2}c^{2}+\gamma^{2}v^{2}
=-\gamma^{2}c^{2}\Bigl(1-\frac{v^{2}}{c^{2}}\Bigr)
=-c^{2}.
$$
Define line element $ds^{2}=-c^{2}d\tau^{2}$, substituting
$$
d\tau^{2}
=dt^{2}-\frac{d\mathbf x^{2}}{c^{2}}
$$
obtain
$$
ds^{2}=-c^{2}dt^{2}+d\mathbf x^{2},
$$
i.e., Minkowski metric.

\subsection{Proof Outline of Theorem 2 and Proposition 1}

Internal evolution equation
$$
\mathrm i\hbar\,\partial_{\tau}\ket{\psi_{\mathrm{int}}(\tau)}
=H_{\mathrm{int}}\ket{\psi_{\mathrm{int}}(\tau)}
$$
has eigenstate solution satisfying
$$
\ket{\psi_{\mathrm{int}}(\tau)}
=\mathrm e^{-\mathrm i E_{0}\tau/\hbar}\ket{\psi_{\mathrm{int}}},
\qquad
H_{\mathrm{int}}\ket{\psi_{\mathrm{int}}}=E_{0}\ket{\psi_{\mathrm{int}}}.
$$
Define
$$
\omega_{\mathrm{int}}=\frac{E_{0}}{\hbar},
$$
if identifying rest energy $E_{0}=m c^{2}$, then obtain
$$
m=\frac{\hbar\omega_{\mathrm{int}}}{c^{2}},
$$
i.e., Theorem 2.

In continuum limit of Dirac-type QCA, effective Hamiltonian
$$
H_{\mathrm{eff}}(k)\simeq c\hbar k\sigma_{z}
+m c^{2}\sigma_{x},
$$
eigenvalues
$$
E_{\pm}(k)=\pm\sqrt{(c\hbar k)^{2}+m^{2}c^{4}}
$$
completely consistent with Dirac Hamiltonian. In Heisenberg picture, position operator $X(t)$ satisfies
$$
\frac{dX}{dt}
=\frac{\mathrm i}{\hbar}[H,X]
=c\,\alpha,
\qquad
\frac{d\alpha}{dt}
=\frac{\mathrm i}{\hbar}[H,\alpha],
$$
where $\alpha$ generally denoted as Dirac matrix. Integrating gives
$$
X(t)=X(0)+c^{2}H^{-1}Pt
+\frac{\mathrm i\hbar c}{2}H^{-1}
\bigl(\mathrm e^{-2\mathrm i H t/\hbar}-1\bigr)
\bigl(\alpha(0)-cH^{-1}P\bigr),
$$
second term is uniform motion, third term is rapid oscillation with frequency $2E/\hbar$, i.e., Zitterbewegung, where
$$
E=+\sqrt{(cP)^{2}+m^{2}c^{4}}.
$$
In rest limit $P=0$, $E=m c^{2}$, oscillation frequency
$$
\omega_{\mathrm{ZB}}(0)
=\frac{2m c^{2}}{\hbar}
=2\omega_{\mathrm{int}},
$$
obtaining Proposition 1.

\subsection{Proof Idea of Theorem 3}

For translation-invariant QCA, can write single-step unitary as
$$
U=\int^{\oplus}U(k)\,dk,
$$
winding number
$$
\mathcal W[U]
=\frac{1}{2\pi\mathrm i}\int\partial_{k}\log\det U(k)\,dk
$$
is homotopy invariant of mapping from Brillouin zone to $\mathrm U(N)$, preserved under finite-depth local unitary transformations. In topologically trivial sector $\mathcal W[U]=0$, exists deformation reducing $U(k)$ to form containing only massless modes, with internal frequency possibly zero. In topologically nontrivial sector $\mathcal W[U]\neq 0$, exist non-eliminable Berry phase windings; to realize such windings, internal Hamiltonian must have nonzero frequency eigenvalue, thus $v_{\mathrm{int}}>0$. This argument can be rigorous through QCA index theory and $K$ theory, see references [1,2] for details.

\subsection{Proof Ideas of Theorem 4 and Theorem 5}

Theorem 4 starts from optical metric
$$
ds^{2}=-\eta^{2}(x)c^{2}dt^{2}
+\eta^{-2}(x)\gamma_{ij}(x)dx^{i}dx^{j}.
$$
In weak field taking
$$
\eta(x)=1+\epsilon(x),\qquad
|\epsilon(x)|\ll 1,
$$
expansion gives
$$
g_{00}\simeq-(1+2\epsilon)c^{2},
\qquad
g_{ij}\simeq(1-2\epsilon)\gamma_{ij}.
$$
In static spherically symmetric case, selecting isotropic coordinates and identifying
$$
\epsilon(r)=\frac{\phi(r)}{c^{2}},
\qquad
\phi(r)=-\frac{GM}{r},
$$
result consistent with Schwarzschild metric weak-field expansion. Refractive index defined as
$$
n(r)=\eta^{-2}(r)\simeq 1-\frac{2\phi(r)}{c^{2}},
$$
thus
$$
c_{\mathrm{eff}}(r)=\frac{c}{n(r)}.
$$
Substituting this refractive index distribution into gravitational lensing method based on Gauss--Bonnet theorem, can obtain light deflection angle $\Delta\theta=4GM/(c^{2}b)$, see references [3,4].

Theorem 5 uses Einstein--Hilbert term
$$
\int\sqrt{-g}\,R\,d^{4}x
$$
standard variation formula and
$$
\delta(\sqrt{-g}\,\mathcal L_{\mathrm{info}})
=\tfrac12\sqrt{-g}\,T^{(\mathrm{info})}_{\mu\nu}\delta g^{\mu\nu},
$$
obtaining
$$
\delta S_{\mathrm{tot}}
=\frac{1}{16\pi G}\int\sqrt{-g}
\bigl(R_{\mu\nu}-\tfrac12 R g_{\mu\nu}\bigr)
\delta g^{\mu\nu}\,d^{4}x
+\frac12\int\sqrt{-g}\,T^{(\mathrm{info})}_{\mu\nu}
\delta g^{\mu\nu}\,d^{4}x.
$$
Requiring $\delta S_{\mathrm{tot}}=0$ for any compactly supported $\delta g^{\mu\nu}$, immediately obtain
$$
R_{\mu\nu}-\tfrac12 R g_{\mu\nu}
=8\pi G\,T^{(\mathrm{info})}_{\mu\nu}.
$$

\subsection{Proof Ideas of Theorem 6 and Proposition 2}

Assume $v_{\mathrm{int}}(M_{I})$ monotonically increasing and
$$
\lim_{M_{I}\to\infty}v_{\mathrm{int}}(M_{I})=c.
$$
From information rate conservation have
$$
v_{\mathrm{ext}}^{2}(M_{I})
=c^{2}-v_{\mathrm{int}}^{2}(M_{I}),
$$
when $M_{I}\to\infty$, $v_{\mathrm{ext}}(M_{I})\to 0$, obtaining asymptotic rest.

Proposition 2 comes from Landauer principle: erasing one bit dissipates minimum heat $k_{B}T\ln 2$. If erasing per unit time
$$
\dot I_{\mathrm{erase}}
=R_{\mathrm{upd}}\Delta I
$$
bits, then minimal power
$$
P_{\min}
=k_{B}T\ln 2\,\dot I_{\mathrm{erase}}
=k_{B}T\ln 2\,R_{\mathrm{upd}}\Delta I,
$$
independent of system's specific physical implementation, depending only on erased information amount and environmental temperature.

\section{Model Apply}

\subsection{Free Particles and Special Relativistic Limit}

In single-particle QCA model, assuming continuum limit has effective Hamiltonian with relativistic dispersion relation
$$
E^{2}=p^{2}c^{2}+m^{2}c^{4},
$$
then four-momentum
$$
p^{\mu}=m u^{\mu}
$$
automatically satisfies
$$
p^{\mu}p_{\mu}=-m^{2}c^{2}.
$$
Since Theorem 1 already normalizes four-velocity $u^{\mu}$, energy--momentum relation and Lorentz transformation properties naturally follow. For massless modes (lightlike excitations in topologically trivial QCA), internal frequency can be zero, $v_{\mathrm{int}}=0$, $v_{\mathrm{ext}}=c$, internal evolution frozen, consistent with photon as massless excitation.

\subsection{Many-Body Systems and Astrophysics}

In macroscopic systems such as stars, galaxies and galaxy clusters, energy density, particle number density and information processing density are highly correlated statistically: high energy density usually means large numbers of degrees of freedom participating in high-frequency interactions, thus larger $\rho_{\mathrm{info}}(x)$, smaller $\eta(x)$, larger refractive index $n(x)$, more curved geometry. At this scale, difference between $\rho_{\mathrm{info}}(x)$ and stress--energy tensor $T_{\mu\nu}$ can be viewed as renormalization and coarse-graining effects, observationally difficult to distinguish. Therefore this framework's predictions at classical astrophysics tests basically consistent with general relativity.

Distinguishable tests should focus on systems where information structure can vary while total energy remains nearly unchanged, such as quantum media with different entanglement structures or topological phases.

\subsection{Black Hole Entropy and Information Rate Saturation}

Near black hole horizon, Bekenstein--Hawking entropy proportional to horizon area, suggesting universal upper bound constraint on number of information degrees of freedom per unit area. In this framework, horizon can be understood as region where $\rho_{\mathrm{info}}(x)$ reaches saturation value: here $\eta(x)$ approaches certain limit, causing severe suppression of external propagation, internal degrees of freedom occupy almost all optical path quota. Black hole entropy can be interpreted as ``maximal information rate integral supportable'' on horizon. To rigorously derive
$$
S_{\mathrm{BH}}=\frac{A}{4G}
$$
from QCA information parameters requires combining specific QCA models with renormalization analysis.

\subsection{Cosmology and Effective Cosmological Constant}

At cosmological scales, can define coarse-grained cosmological average information processing density $\rho_{\mathrm{info}}^{\mathrm{cosmo}}(t)$. Its smooth component can be absorbed into effective ``information vacuum energy'' term through $\mathcal L_{\mathrm{info}}$, playing role similar to cosmological constant or dark energy in Friedmann equations; if $\rho_{\mathrm{info}}^{\mathrm{cosmo}}(t)$ varies slowly with time, can produce observational features similar to dynamical dark energy. This direction requires introducing QCA models and renormalization schemes in specific cosmological backgrounds.

\section{Engineering Proposals}

\subsection{``Artificial High Refractive Index'' Experiment in Superconducting Microwave Cavities}

Consider high quality-factor superconducting microwave cavity or circuit QED platform with fixed volume $V$, containing several controllable electromagnetic field modes. Through external driving and feedback, can prepare various quantum states in cavity and characterize entanglement structure through quantum tomography. Assume through feedback control, fix cavity total energy $E_{\mathrm{tot}}$ at some value.

Design two types of internal states:
\begin{itemize}
\item State A: At given $E_{\mathrm{tot}}$, modes approximately non-entangled or weakly entangled, topologically trivial, corresponding to ``low information structure state'';
\item State B: At same $E_{\mathrm{tot}}$, prepare highly entangled, possibly topologically ordered multi-mode quantum state, corresponding to ``high information structure state''.
\end{itemize}

In standard general relativity, gravity source determined by stress--energy tensor $T_{\mu\nu}$, and Maxwell field's $T_{\mu\nu}$ in this setting macroscopically depends only on energy density and pressure, entanglement structure does not enter source term. Therefore as long as $E_{\mathrm{tot}}$ and macroscopic spatial distribution same, states A and B should have same effect on external geometry, Shapiro delay near cavity also same.

In this paper's framework, high entanglement or topologically ordered state means higher $\rho_{\mathrm{info}}(x)$, i.e., more ``path length'' occurs in internal Hilbert space per unit time, even though total energy unchanged. This causes changes in $\eta(x)$ and refractive index $n(x)=\eta^{-2}(x)$. If arrange high-sensitivity interferometer near cavity making probe light beam traverse this region multiple times, then when switching between states A and B, will measure differential phase delay in interference fringes
$$
\Delta\phi_{\mathrm{info}}
\sim\frac{\omega}{c}\int_{\mathrm{path}}\Delta n(\mathbf x)\,d\ell,
$$
where $\omega$ is light frequency, $\Delta n(\mathbf x)$ is refractive index change caused by information structure difference.

In standard general relativity, since macroscopic $T_{\mu\nu}$ unchanged, predicts $\Delta\phi\approx 0$; in this paper's framework, predicts $\Delta\phi\neq 0$. If stable differential phase delay observed under control of systematic errors and noise, will constitute direct evidence of ``entanglement gravity'' relative to ``energy gravity''.

\subsection{Dirac-QCA Experiments on Quantum Simulation Platforms}

Multiple platforms (such as ion traps, optical lattices, photonic circuits) have realized discrete-time quantum walks and Dirac-type QCA, observing relativistic features such as Zitterbewegung. By adjusting model parameters can realize different effective masses and topological structures; if further able to characterize internal state evolution rate (e.g., through coherent manipulation and tomography of spin degrees of freedom), then can experimentally test approximate relation
$$
v_{\mathrm{ext}}^{2}(k)+v_{\mathrm{int}}^{2}(k)\simeq c^{2}
$$
whether holds, and observe how $v_{\mathrm{int}}(k)$ varies with mass and topological invariants. This will provide quantum simulation support for ``optical path quota allocation between internal and external'' picture.

\section{Discussion (risks, boundaries, past work)}

Existing QCA research has systematically shown how to emerge free quantum field theory and partial interacting field theory from discrete models, giving rigorous description of QCA index theory and topological classification [see references [1,2]]. This paper introduces information rate conservation axiom on this basis, viewing Minkowski geometry and special relativity as geometric expression of information rate constraints, interpreting Zitterbewegung in Dirac-QCA as manifestation of internal frequency structure in external coordinates.

Optical metric and application of Gauss--Bonnet theorem in gravitational lensing theory have shown equivalence between ``gravity bending light'' and ``changing refractive index'' [see references [3,4]]. This paper gives refractive index $n(x)$ microscopic meaning through $\rho_{\mathrm{info}}(x)$, connecting macroscopic optical metric with microscopic QCA information processing.

Information perspective gravity theories also have other schemes, such as work deriving Einstein equations from local thermodynamics and Clausius relation, and various ``entropic gravity'' schemes; Landauer principle provides universal lower bound for information--energy relation from irreversible computation perspective [see reference [5]]. This paper's feature is: starting from discrete QCA ontology, using information rate conservation as microscopic strong constraint, then deriving interpretation of continuous geometry and mass through Hilbert space geometry and topological structure, giving optical gravity picture with $\rho_{\mathrm{info}}(x)$, $\eta(x)$, $n(x)$ as core.

Risks to emphasize include: coarse-graining from QCA to continuous manifold and metric generally not unique, definition of $\rho_{\mathrm{info}}(x)$ needs to avoid gauge dependence on internal degree of freedom choice; computability problem of information mass $M_{I}$ limits its quantitative application in specific systems; technical challenges of experimentally realizing ``entanglement gravity'' tests are significant. These problems define applicable boundaries of this paper's framework.

\section{Conclusion}

Within quantum cellular automaton and finite information ontology framework, introducing information rate (optical path) conservation as single axiom can give unified description of relations among special relativity, mass, gravity and information mass. Main conclusions include:

\begin{enumerate}
\item Information rate conservation and proper time definition sufficient to derive Minkowski geometry and special relativity's time dilation and four-velocity normalization;

\item Mass can be interpreted as proportionality coefficient $m=\hbar\omega_{\mathrm{int}}/c^{2}$ of internal self-referential frequency $\omega_{\mathrm{int}}$, manifested in Dirac-QCA through Zitterbewegung frequency $\omega_{\mathrm{ZB}}=2\omega_{\mathrm{int}}$;

\item Combination of local information processing density $\rho_{\mathrm{info}}(x)$ and optical metric
$$
ds^{2}=-\eta^{2}(x)c^{2}dt^{2}+\eta^{-2}(x)\gamma_{ij}(x)dx^{i}dx^{j}
$$
recovers first-order expansion of Schwarzschild metric and correct light deflection factor in weak-field limit;

\item Information--gravity variational principle gives geometry--information unified expression formally equivalent to Einstein field equation, where $T^{(\mathrm{info})}_{\mu\nu}$ generalizes source term action of information processing on geometry;

\item Information mass $M_{I}$ and its Landauer cost provide unified characterization of gravitational and dissipation properties of high-complexity energetic systems, high $M_{I}$ subjects tend to asymptotic rest in external geometry and necessarily accompanied by continuous entropy output.
\end{enumerate}

Future work includes: constructing explicit $\mathcal L_{\mathrm{info}}$ in specific QCA models and verifying its consistency with standard quantum field theory stress--energy tensor; in black hole and cosmological backgrounds, connecting $\rho_{\mathrm{info}}$ with area entropy, cosmological constant, dark energy and other problems; designing feasible experiments in artificial quantum systems to directly measure effective light speed or clock frequency behavior varying with entanglement entropy and information density. If progress achieved in these directions, information rate conservation could become unified pathway connecting microscopic computational ontology and macroscopic geometric narrative.

\section*{Acknowledgements, Code Availability}

This research builds on existing achievements in quantum cellular automata, discrete gravity, gravitational lensing geometric methods and information thermodynamics, especially relying on systematic research on QCA--quantum field theory continuum limits, QCA topological classification, Gauss--Bonnet based gravitational lensing methods and Landauer principle [see references [1--5]].

This paper does not use numerical simulation code, all derivations are analytical. If future work in specific QCA models and numerical verification progresses, corresponding code will be organized and made public separately.

\begin{thebibliography}{99}
\bibitem{farrelly} T. Farrelly, ``A Review of Quantum Cellular Automata'', Quantum 4, 368 (2020).
\bibitem{bisio} A. Bisio, G. M. D'Ariano, A. Tosini, ``Dirac Quantum Cellular Automaton in One Dimension: Zitterbewegung and Scattering from Potential'', Phys. Rev. A 88, 032301 (2013).
\bibitem{gibbons} G. W. Gibbons, M. C. Werner, ``Applications of the Gauss--Bonnet Theorem to Gravitational Lensing'', Class. Quantum Grav. 25, 235009 (2008).
\bibitem{halla} M. Halla, ``Application of the Gauss--Bonnet Theorem to Lensing in Static Spherically Symmetric Spacetimes'', Gen. Relativ. Gravit. 52, 95 (2020).
\bibitem{landauer} R. Landauer, ``Irreversibility and Heat Generation in the Computing Process'', IBM J. Res. Dev. 5, 183--191 (1961).
\bibitem{more} For more literature reviews on QCA topological classification, quantum simulation platforms and generalized optical metric methods, see references [1--4] and their cited references.
\end{thebibliography}

\end{document}

