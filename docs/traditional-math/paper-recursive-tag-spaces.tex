\documentclass[12pt]{article}

% Essential packages
\usepackage[utf8]{inputenc}
\usepackage{amsmath,amssymb,amsthm}
\usepackage{mathrsfs}
\usepackage{geometry}
\usepackage{hyperref}
\usepackage{enumerate}
\usepackage{mathtools}

% Geometry settings
\geometry{a4paper, margin=1in}

% Hyperref settings
\hypersetup{
    colorlinks=true,
    linkcolor=blue,
    citecolor=blue,
    urlcolor=blue
}

% Theorem environments
\theoremstyle{plain}
\newtheorem{theorem}{Theorem}[section]
\newtheorem{lemma}[theorem]{Lemma}
\newtheorem{proposition}[theorem]{Proposition}
\newtheorem{corollary}[theorem]{Corollary}

\theoremstyle{definition}
\newtheorem{definition}[theorem]{Definition}
\newtheorem{example}[theorem]{Example}
\newtheorem{remark}[theorem]{Remark}

% Title information
\title{Window Ratio Indices for Mathematical Constants: \\ A Recursive Hilbert Space Approach}
\author{Anonymous}
\date{\today}

\begin{document}

\maketitle

\begin{abstract}
We introduce window ratio indices $\eta^{(R)}(k; m)$ as analytical tools for studying convergence patterns of mathematical constants within recursive Hilbert space constructions. Our framework defines spaces $\mathcal{H}_n^{(R)} = \mathcal{H}_{n-1}^{(R)} \oplus \mathbb{C} e_n$ with coefficient sequences $f_n = \sum_{k=0}^n a_k e_k$ where $a_k$ follow specific recursive or series patterns. We demonstrate that constants $\pi$, $e$, and $\varphi$ correspond to distinct coefficient modes: Leibniz series, factorial decay, and Fibonacci recursion. The window ratio index $\eta^{(R)}(k; m)$ provides finite computational windows for analyzing infinite sequences through ratio comparisons of pattern functions over different intervals. Alexandroff compactification of the index space $\mathcal{I}^{(R)} = \mathbb{N} \cup \{\infty\}$ gives asymptotic continuity properties for the indices.
\end{abstract}

\noindent\textbf{Keywords:} Hilbert spaces, mathematical constants, window indices, coefficient sequences, convergence analysis

\noindent\textbf{MSC 2020:} Primary 46C05, 40A25; Secondary 11B39

\section{Introduction}

Consider three classical representations of mathematical constants:
\begin{align}
\pi &= 4 \sum_{k=1}^{\infty} \frac{(-1)^{k-1}}{2k-1} \quad \text{(Leibniz series)} \\
e &= \sum_{k=0}^{\infty} \frac{1}{k!} \quad \text{(factorial series)} \\
\varphi &= \lim_{n \to \infty} \frac{F_n}{F_{n-1}} \quad \text{(Fibonacci ratio)}
\end{align}

Each constant emerges from a distinct coefficient pattern: alternating rational terms, factorial decay, and recursive relations respectively. This paper develops analytical tools for studying such convergence patterns within a unified Hilbert space framework.

\subsection{Main Contributions}

\begin{enumerate}
\item Definition of window ratio indices $\eta^{(R)}(k; m)$ for finite interval analysis of infinite sequences
\item Recursive Hilbert space construction $\mathcal{H}_n^{(R)} = \mathcal{H}_{n-1}^{(R)} \oplus \mathbb{C} e_n$ with coefficient sequence analysis
\item Systematic treatment of coefficient modes for $\pi$, $e$, and $\varphi$ within the unified framework
\item Alexandroff compactification of index spaces and asymptotic properties
\end{enumerate}

\section{Recursive Hilbert Space Construction}

\subsection{Basic Setup}

Let $\mathbb{C}$ denote the complex field. A \textbf{Hilbert space} $\mathcal{H}$ is a complete inner product space under the norm $\|x\| = \sqrt{\langle x, x \rangle}$.

\begin{definition}[One-Dimensional Extension]
Given a Hilbert space $\mathcal{H}$ and a unit vector $e$ orthogonal to $\mathcal{H}$, define the \textbf{one-dimensional extension}:
$$\mathcal{H} \oplus \mathbb{C} e = \{h + \alpha e : h \in \mathcal{H}, \alpha \in \mathbb{C}\}$$
\end{definition}

\subsection{Recursive Construction}

\begin{definition}[Recursive Hilbert Space Sequence]
Define a sequence of Hilbert spaces by:
\begin{align}
\mathcal{H}_0^{(R)} &= \mathbb{C} e_0 \quad \text{(one-dimensional initial space)} \\
\mathcal{H}_n^{(R)} &= \mathcal{H}_{n-1}^{(R)} \oplus \mathbb{C} e_n \quad \text{for } n \geq 1
\end{align}
where $\{e_k\}_{k=0}^\infty$ are orthonormal basis vectors.
\end{definition}

The nesting property gives:
$$\mathcal{H}_0^{(R)} \subset \mathcal{H}_1^{(R)} \subset \mathcal{H}_2^{(R)} \subset \cdots$$

The \textbf{complete space} is:
$$\mathcal{H}^{(R)} = \overline{\bigcup_{n=0}^\infty \mathcal{H}_n^{(R)}}$$

\subsection{Coefficient Sequences}

\begin{definition}[Coefficient Sequence]
A \textbf{coefficient sequence} in $\mathcal{H}_n^{(R)}$ is:
$$f_n = \sum_{k=0}^n a_k e_k$$
where $a_k \in \mathbb{C}$ are coefficients and $\{e_k\}_{k=0}^n$ is the orthonormal basis for $\mathcal{H}_n^{(R)}$.
\end{definition}

\section{Window Ratio Index Theory}

\subsection{Pattern Functions}

\begin{definition}[Pattern Function]
For a coefficient sequence $\{a_k\}_{k=0}^n$, define pattern functions:

\begin{itemize}
\item \textbf{Ratio type}: $F_{\text{ratio}}(\{a_k\}) = \lim_{n \to \infty} \frac{a_n}{a_{n-1}}$ (when the limit exists)
\item \textbf{Sum type}: $F_{\text{sum}}(\{a_k\}) = \lim_{n \to \infty} \sum_{k=0}^n a_k$ (when convergent)
\item \textbf{Weighted sum}: $F_{\text{weighted}}(\{a_k\}) = \lim_{n \to \infty} \sum_{k=0}^n c_k a_k$ where $c_k$ are fixed weights
\end{itemize}
\end{definition}

\subsection{Window Ratio Index}

\begin{definition}[Window Ratio Index]
For intervals $[0,m]$ and $[m+1, m+k]$, define the \textbf{window ratio index}:
$$\eta^{(R)}(k; m) = \frac{F_{\text{finite}}(\{a_{m+j}\}_{j=1}^k)}{F_{\text{finite}}(\{a_j\}_{j=0}^m)}$$

where $F_{\text{finite}}$ applies the pattern function to finite intervals:
\begin{itemize}
\item \textbf{Ratio type}: $F_{\text{finite}}(\{a_p, \ldots, a_q\}) = \frac{a_q}{a_p}$ 
\item \textbf{Sum type}: $F_{\text{finite}}(\{a_p, \ldots, a_q\}) = \sum_{j=p}^q a_j$
\item \textbf{Weighted sum}: $F_{\text{finite}}(\{a_p, \ldots, a_q\}) = \sum_{j=p}^q c_j a_j$
\end{itemize}
\end{definition}

\begin{remark}[Boundary Cases]
For $m = 0$ cases, we require appropriate handling:
\begin{itemize}
\item \textbf{$\varphi$ mode}: Defined for $a_m \neq 0$; use $|a_m|$ when $a_m$ might be zero
\item \textbf{$\pi$ mode}: Defined for $m \geq 1$ to avoid empty sums
\item \textbf{$e$ mode}: Well-defined for $m \geq 0$ since $a_0 = 1$
\end{itemize}
\end{remark>

\section{Compactification of Recursive Index Spaces}

\subsection{Alexandroff Compactification Framework}

Recursive tag sequences in infinite extension can be embedded in the \textbf{one-point compactification} topological structure $\mathcal{I}^{(R)} = \mathbb{N} \cup \{\infty\}$, where $\infty$ serves as the ideal point.

\begin{definition}[Asymptotic Properties of Relativistic Indices]
The mode-specific asymptotic properties of relativistic indices are:

\begin{itemize}
\item \textbf{$\varphi$ mode}: $\lim_{k \to \infty} \eta^{(\varphi)}(k; m) = \lim \frac{a_{m+k}}{a_m} \approx \varphi^k \to \infty$ (divergent growth)
\item \textbf{$e$ mode}: $\lim_{k \to \infty} \eta^{(e)}(k; m) = \frac{e - s_m}{s_m}$, where $s_m = \sum_{j=0}^m \frac{1}{j!}$ (remaining tail ratio)
\item \textbf{$\pi$ mode}: $\lim_{k \to \infty} \eta^{(\pi)}(k; m) = \frac{\pi/4 - t_m}{t_m}$, where $t_m = \sum_{j=1}^m \frac{(-1)^{j-1}}{2j-1}$ (convergence residue)
\end{itemize}
\end{definition}

\begin{theorem}[Asymptotic Continuity in Compactified Topology]
The function $\eta$ is asymptotically continuous in the compactified topology, with $\eta(\infty; m)$ defined as the mode-specific $\lim_{k \to \infty} \eta(k; m)$. If the limit does not exist, the extension to $\infty$ is not performed.
\end{theorem}

This maintains the atomic copying property of infinite dimensional initialization, ensuring that tag references for each new orthogonal basis increase logically in endless recursion.

\section{Mathematical Constants as Tag Patterns}

Based on the preceding definitions, we now implement specific mathematical constant tag patterns.

\subsection{$\varphi$ Tag Pattern}

\textbf{Coefficient recursion}: $a_0 = 0, a_1 = 1, a_2 = 1$, followed by $a_k = a_{k-1} + a_{k-2}$ (Fibonacci recursion)

\textbf{Pattern function}: $F_\varphi = F_{\text{ratio}}(\{a_k\}) = \lim \frac{a_n}{a_{n-1}} = \varphi$

\textbf{Relativistic index}: $\eta^{(\varphi)}(k; m) = \frac{a_{m+k}}{a_m}$ (for $m \geq 1$ or $a_m \neq 0$), with the $m=0$ case handled through Fibonacci property limit values.

\subsection{$\pi$ Tag Pattern}

\textbf{Coefficient definition}: $a_0 = 0$, $a_k = \frac{(-1)^{k-1}}{2k-1}$ for $k \geq 1$ (Leibniz series)

\textbf{Pattern function}: $F_\pi = F_{\text{weighted}}(\{a_k\}) = \lim 4\sum_{k=1}^n a_k = \pi$

\textbf{Relativistic index}: $\eta^{(\pi)}(k; m) = \frac{\sum_{j=m+1}^{m+k} \frac{(-1)^{j-1}}{2j-1}}{\sum_{j=1}^{m} \frac{(-1)^{j-1}}{2j-1}}$ (for $m \geq 1$)

\subsection{$e$ Tag Pattern}

\textbf{Coefficient definition}: $a_k = \frac{1}{k!}$ for $k \geq 0$ (factorial decay)

\textbf{Pattern function}: $F_e = F_{\text{sum}}(\{a_k\}) = \lim \sum_{k=0}^n a_k = e$

\textbf{Relativistic index}: $\eta^{(e)}(k; m) = \frac{\sum_{j=m+1}^{m+k} \frac{1}{j!}}{\sum_{j=0}^{m} \frac{1}{j!}}$

\section{Fundamental Theorems}

\subsection{Coordinate System Isomorphisms}

\begin{theorem}[Recursive Operator Coordinate System Isomorphism]
Different recursion operators $R$ are isomorphic via basis transformations to a \textbf{unified infinite recursive space} $\mathcal{H}^{(\infty)}$, embodying the same self-contained recursive principle but with different tag patterns in their respective coordinate systems.
\end{theorem}

\textbf{Mathematical statement}: There exist explicit isomorphic mappings that map all $\mathcal{H}^{(R)}$ to a unified infinite recursive space $\mathcal{H}^{(\infty)}$, where the coordinate systems are determined by basis transformations and tag pattern selections induced by $R$.

\subsection{Tag Pattern Implementation}

\begin{theorem}[Tag Pattern Recursive Implementation]
Different tag patterns are implemented through the same recursion operator $R$, differing only in the choice of tag coefficients $a_k$:

\begin{itemize}
\item \textbf{$\varphi$ pattern}: Through Fibonacci coefficients $a_k = a_{k-1} + a_{k-2}$
\item \textbf{$\pi$ pattern}: Through Leibniz coefficients $a_k = \frac{(-1)^{k-1}}{2k-1}$
\item \textbf{$e$ pattern}: Through factorial coefficients $a_k = \frac{1}{k!}$
\end{itemize}
\end{theorem}

\begin{proof}
All patterns use the same $R$ and $\oplus_{\text{embed}} \mathbb{C} e_n$, differing only in the recursive or series definitions of tag coefficients. \qed
\end{proof}

\subsection{Unified Construction}

\begin{theorem}[Recursive Construction Unification]
All recursive constructions satisfying self-contained recursive principles are unified through isomorphic mappings to a single self-similar space $\mathcal{H}^{(\infty)}$, differing only in tag coefficient choices and embedding methods.
\end{theorem}

\section{Computational Aspects}

\subsection{Numerical Implementation}

The relativistic index framework provides practical computational tools:

\begin{enumerate}
\item \textbf{Finite window computation}: Calculate $\eta^{(R)}(k; m)$ for specific values of $k$ and $m$
\item \textbf{Convergence analysis}: Monitor how $F_{\text{finite}}(\{a_k\})$ approaches the target constant
\item \textbf{Error estimation}: Use the difference between finite and infinite pattern functions
\end{enumerate}

\subsection{Algorithmic Applications}

The framework suggests several algorithmic applications:

\begin{itemize}
\item \textbf{Constant approximation}: Efficient computation of mathematical constants through optimized tag sequences
\item \textbf{Convergence acceleration}: Use knowledge of tag patterns to accelerate series convergence
\item \textbf{Pattern recognition}: Identify new mathematical constants through tag pattern analysis
\end{itemize}

\section{Discussion and Future Directions}

\subsection{Theoretical Significance}

Our framework provides a unified mathematical language for understanding how different types of mathematical constants emerge from recursive structures. The key insight is that constants are not isolated numerical values but represent convergence patterns of specific recursive processes.

\subsection{Limitations and Extensions}

Several directions for future research emerge:

\begin{enumerate}
\item \textbf{Extension to other constants}: Investigation of tag patterns for constants like $\gamma$ (Euler-Mascheroni constant) or algebraic numbers
\item \textbf{Higher-order recursions}: Extension to recursion operators depending on more than two previous terms
\item \textbf{Non-commutative generalizations}: Application to matrix-valued or operator-valued recursive structures
\end{enumerate}

\subsection{Computational Complexity}

The relativistic index approach offers computational advantages:
\begin{itemize}
\item \textbf{Finite computation}: All calculations involve finite sums, avoiding infinite series truncation issues
\item \textbf{Parallelization}: Different relativistic indices can be computed independently
\item \textbf{Adaptive precision}: The choice of window parameters $k$ and $m$ allows precision control
\end{itemize}

\section{Conclusion}

We have developed a comprehensive framework for analyzing mathematical constants through recursive tag spaces and relativistic indices. Our approach reveals that fundamental constants like $\pi$, $e$, and $\varphi$ can be understood as emergent patterns from specific recursive structures within appropriately constructed Hilbert spaces.

The key innovation is the relativistic index $\eta^{(R)}(k; m)$ that enables rigorous finite analysis of infinite dimensional structures. This provides both theoretical insight into the nature of mathematical constants and practical computational tools for their analysis.

The framework maintains strict mathematical rigor while offering new perspectives on classical problems in analysis and computation. Future work will explore extensions to broader classes of constants and recursive structures.

\begin{thebibliography}{99}

\bibitem{conway1990}
J. H. Conway and R. K. Guy, \textit{The Book of Numbers}, Springer-Verlag, 1996.

\bibitem{graham1994}
R. L. Graham, D. E. Knuth, and O. Patashnik, \textit{Concrete Mathematics}, 2nd edition, Addison-Wesley, 1994.

\bibitem{hardy1979}
G. H. Hardy and E. M. Wright, \textit{An Introduction to the Theory of Numbers}, 5th edition, Oxford University Press, 1979.

\bibitem{khinchin1997}
A. Ya. Khinchin, \textit{Continued Fractions}, Dover Publications, 1997.

\bibitem{reed1980}
M. Reed and B. Simon, \textit{Methods of Modern Mathematical Physics, Vol. 1: Functional Analysis}, Academic Press, 1980.

\bibitem{rudin1991}
W. Rudin, \textit{Functional Analysis}, 2nd edition, McGraw-Hill, 1991.

\end{thebibliography}

\end{document}